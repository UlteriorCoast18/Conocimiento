\documentclass[12pt]{report}

% --- Idioma y codificación ---
\usepackage[english]{babel}
\usepackage[utf8]{inputenc}

% --- Matemáticas ---
\usepackage{amsmath, amssymb, amsthm}

% --- Gráficos y figuras ---
\usepackage{graphics, graphicx, subfigure}
\usepackage{tikz, pgffor, ifthen}

% --- Tablas y estructuras ---
\usepackage{array, multicol, longtable, booktabs}

% --- Listas y enumeraciones ---
\usepackage{enumerate, enumitem}

% --- Márgenes y geometría ---
\usepackage[a4paper, margin=1.5cm]{geometry}

% --- Diseño y marco ---
\usepackage[framemethod=TikZ]{mdframed}

% --- Texto y contenido de prueba ---
\usepackage{lipsum}

\usepackage{subfiles}

% --- Hipervínculos ---
\usepackage{hyperref}
\usepackage{multirow}
\usepackage{bbm}

\hypersetup{
    colorlinks=true,
    linkcolor=black,
    filecolor=magenta,
    urlcolor=cyan
}

% --- Código fuente (listings) ---
\usepackage{listings}
\usepackage{xcolor}

\definecolor{listing-background}{HTML}{F7F7F7}
\definecolor{listing-rule}{HTML}{B3B2B3}
\definecolor{listing-numbers}{HTML}{B3B2B3}
\definecolor{listing-text-color}{HTML}{000000}
\definecolor{listing-keyword}{HTML}{435489}
\definecolor{listing-keyword-2}{HTML}{1284CA}
\definecolor{listing-keyword-3}{HTML}{9137CB}
\definecolor{listing-identifier}{HTML}{435489}
\definecolor{listing-string}{HTML}{00999A}
\definecolor{listing-comment}{HTML}{8E8E8E}

\lstdefinestyle{myStyle}{
    language=C++,
    numbers=left,
    xleftmargin=2.7em,
    framexleftmargin=2.5em,
    backgroundcolor=\color{gray!15},
    basicstyle=\color{listing-text-color}\linespread{1.0}\ttfamily,
    breaklines=true,
    frameshape={RYR}{Y}{Y}{RYR},
    rulecolor=\color{black},
    tabsize=2,
    numberstyle=\color{listing-numbers}\linespread{1.0}\small\ttfamily,
    aboveskip=1.0em,
    belowskip=0.1em,
    abovecaptionskip=0em,
    belowcaptionskip=1.0em,
    keywordstyle={\color{listing-keyword}\bfseries},
    keywordstyle={[2]\color{listing-keyword-2}\bfseries},
    keywordstyle={[3]\color{listing-keyword-3}\bfseries\itshape},
    sensitive=true,
    identifierstyle=\color{listing-identifier},
    commentstyle=\color{listing-comment},
    stringstyle=\color{listing-string},
    showstringspaces=false,
    label=lst:bar,
    captionpos=b
}
\lstset{style=myStyle}

% --- Marca de agua ---
\usepackage{eso-pic}
\AddToHook{shipout/foreground}{
    \begin{tikzpicture}[remember picture,overlay]
        \node at (current page.center){
            \includegraphics[width=\paperwidth,height=\paperheight,keepaspectratio]{watermark-1.png}
        };
    \end{tikzpicture}
}

% --- Redefiniciones de encabezados de capítulo y sección ---
\makeatletter
% Capítulo (estilo original conservado)
\def\@makechapterhead#1{%
  {\parindent \z@ \raggedright
    \reset@font
    \hrule
    \vspace*{10\p@}%
    \par
    \center \LARGE \scshape \@chapapp{} \huge \thechapter
    \vspace*{10\p@}%
    \par\nobreak
    \vspace*{10\p@}%
    \par
    \vspace*{1\p@}%
    \hrule
    \vspace*{30\p@}  % Espaciado reducido
    \centering\Huge \scshape #1\par\nobreak  % Centrado y scshape
    \vskip 30\p@  % Espaciado reducido
  }}


% Sección
\renewcommand{\section}{\@startsection{section}{1}{\z@}%
  {-2.5ex \@plus -0.5ex \@minus -0.1ex}%  % Espaciado superior reducido
  {1ex \@plus 0.1ex}%                     % Espaciado inferior reducido
  {\normalfont\Large\sectionstyle}}
\newcommand{\sectionstyle}[1]{%
  \par\noindent\hrule
  \vspace{0.2ex}%   % Espaciado entre líneas reducido
  {\scshape{#1}\par}%  % Centrado perfecto y scshape
  \vspace{0.4ex}%   % Espaciado entre líneas reducido
  \hrule
}

% Subsección
\renewcommand{\subsection}{\@startsection{subsection}{2}{\z@}%
  {-2ex \@plus -0.4ex \@minus -0.1ex}%  % Espaciado superior reducido
  {0.8ex \@plus 0.1ex}%                 % Espaciado inferior reducido
  {\normalfont\large\subsectionstyle}}
\newcommand{\subsectionstyle}[1]{%
  \par\noindent\hrule
  \vspace{-0.4ex}%  % Espaciado entre líneas reducido
  {\scshape #1\par}%  % Centrado perfecto y scshape
  \vspace{0.4ex}%  % Espaciado entre líneas reducido
  \hrule
}

% Subsubsección
\renewcommand{\subsubsection}{\@startsection{subsubsection}{3}{\z@}%
  {-1.5ex \@plus -0.3ex \@minus -0.1ex}%  % Espaciado superior reducido
  {0.5ex \@plus 0.1ex}%                   % Espaciado inferior reducido
  {\normalfont\normalsize\subsubsectionstyle}}
\newcommand{\subsubsectionstyle}[1]{%
  \par\noindent\hrule
  \vspace{0.4ex}%   % Espaciado entre líneas reducido
  {\scshape #1\par}%  % Centrado perfecto y scshape
  \vspace{0.4ex}%   % Espaciado entre líneas reducido
  \hrule
}
\makeatother

% --- Entornos personalizados ---
% (aquí puedes definir tus theorems, definiciones, etc.)


% --- Entornos personalizados ---
\newtheoremstyle{largebreak}{}{ }{\normalfont}{}{\bfseries}{}{\newline}{}
\theoremstyle{largebreak}

\newmdtheoremenv[hidealllines=true,roundcorner=5pt,backgroundcolor=gray!60!red!30]{exa}{Example}[section]
\newmdtheoremenv[hidealllines=true,roundcorner=5pt,backgroundcolor=gray!50!blue!30]{obs}{Observation}[section]
\newmdtheoremenv[hidealllines=true,roundcorner=5pt,backgroundcolor=green!50!blue!30]{preg}{Question}[section]
\newmdtheoremenv[hidealllines=true,roundcorner=5pt,backgroundcolor=yellow!40]{idea}{Idea}[section]
\newmdtheoremenv[rightline=false,leftline=false]{theor}{Theorm}[section]
\newmdtheoremenv[rightline=false,leftline=false]{propo}{Proposition}[section]
\newmdtheoremenv[rightline=false,leftline=false]{cor}{Corollary}[section]
\newmdtheoremenv[rightline=false,leftline=false]{lema}{Lemma}[section]
\newmdtheoremenv[roundcorner=5pt,backgroundcolor=gray!30,hidealllines=true]{mydef}{Definition}[section]
\newmdtheoremenv[roundcorner=5pt]{excer}{Excercise}[section]

% --- Comandos auxiliares ---
\def\proof{\paragraph{Proof:\\}}
\def\endproof{\hfill$\blacksquare$}
\def\sol{\paragraph{Solution:\\}}
\def\endsol{\hfill$\square$}

\newcommand\logit[1]{\ensuremath{\textup{logit}\left(#1\right)}}
\newcommand\abs[1]{\ensuremath{\left|#1\right|}}
\newcommand\divides{\ensuremath{\bigm|}}
\newcommand\cf[3]{\ensuremath{#1:#2\rightarrow#3}}
\newcommand\contradiction{\ensuremath{\#_c}}
\newcommand\natint[1]{\ensuremath{\left[\big|#1\big|\right]}}
\newcommand\bbm[1]{\ensuremath{\mathbbm{#1}}}

\newcounter{figcount}
\setcounter{figcount}{1}

\renewcommand{\lstlistingname}{Code}
\renewcommand{\lstlistlistingname}{{\lstlistingname} List}

\setlist{  
    itemsep=0pt,      % Space between items
    parsep=0pt,       % Space between paragraphs in items
    topsep=5pt,       % Space before/after list
    partopsep=0pt     % Extra space when list starts new paragraph
}

% Or set separately for different list types
\setlist[itemize]{  
    itemsep=2pt,
    parsep=2pt,
    topsep=5pt
}

\setlist[enumerate]{  
    itemsep=2pt,
    parsep=2pt,
    topsep=5pt
}

% --- Comienzo del documento ---
\begin{document}
    \setlength{\parskip}{5pt}
    \setlength{\parindent}{12pt}
    \title{Basics Programs
    
    in C++}
    \author{Cristo Daniel Alvarado}
    \maketitle

    \tableofcontents

    \lstlistoflistings

    \chapter{Basic Codes}

    \section{About C++}

    Citing the book \textit{A tour of \lstinline|C++|} by Bjarne Stroustrup:

    \begin{center}
      \textit{\lstinline|C++| is a compiled language. For a program to run, its source text has to be processed by a compiler, producing object files, which are combined by a linker yielding an executable program.}
    \end{center}

    Let's analyze this sentence. One of the questions that arises from this is the following: what is a \textbf{compiled language}?

    \begin{mydef}[\textbf{Compiled Language}]
      A \textbf{compiled language} is one where \textit{the program, once compiled, is expressed in the instructions of the target machine}.
      
      For example, an addition "\lstinline|+|" operation in your source code could be translated directly to the "\lstinline|ADD|" instruction in machine code.
    \end{mydef}

    There are other kind of languages, called \textbf{interpreted languages}:

    \begin{mydef}[\textbf{Interpreted Language}]
      An \textbf{interpreted language} is one where \textit{the instructions are not directly executed by the target machine, but instead read and executed by some other program} (which normally is written in the language of the native machine).

      For example, the same "\lstinline|+|" operation would be recognised by the interpreter at run time, which would then call its own "\lstinline|add(a,b)|" function with the appropriate arguments, which would then execute the machine code "\lstinline|ADD|" instruction.
    \end{mydef}

    Now, what is an \textbf{object file}?

    \begin{mydef}[\textbf{Object File}]
      An \textbf{object file} is \textit{the real output from the compilation phase}. 
      
      It's mostly machine code, but has info that allows a linker to see what symbols are in it as well as symbols it requires in order to work.
    \end{mydef}

    And, what is a linker?

    \begin{mydef}[\textbf{Linker}]
      A \textbf{linker} or \textbf{link editor} is a \textit{computer program that combines intermediate software build files such as object and library files into a single executable file such as a program or library}.
    \end{mydef}

    So, for example, a \lstinline|C++| program consist of many source file codes (sometimes simply called \textbf{source files}).

    The process is as follows:

    \begin{figure}[h]
      \begin{minipage}{\textwidth}
        \centering
        \includegraphics[scale=1]{images/cpp_compile_process/cpp_compile_process.pdf} \\
        \caption{\lstinline|C++| compile process.}
        \label{figure:cpp_compile_process}
      \end{minipage}
    \end{figure}

    \begin{obs}[\textbf{Portability of Source Code}]
      An executable program is created for a specific hardware/system combination; it is not portable, say, from a Mac to a Windows PC.
      
      When we talk about portability of \lstinline|C++| programs, we usually mean \textbf{portability of source code}; that is, the source code can be successfully compiled and run on a variety of systems.
    \end{obs}

    \begin{mydef}[\textbf{Entities in the ISO C++}]
      The ISO C++ standard defines two kinds of entities:
      \begin{itemize}
        \item \textbf{Core language features}, such as built-in types (e.g., \lstinline|char| and \lstinline|int|) and loops (e.g., \lstinline|for-statements| and \lstinline|while-statements|)
        \item \textbf{Standard-library components}, such as containers (e.g., \lstinline|vector| and \lstinline|map|) and I/O operations (e.g., \lstinline|<<| and \lstinline|getline()|)
      \end{itemize}
    \end{mydef}

    \newpage

    \subfile{OOP/oop.tex}

\end{document}