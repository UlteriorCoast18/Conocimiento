\documentclass[../fundamentals_modeling.tex]{subfiles}

\begin{document}

    \chapter{Introduction}

    \section{Topics}

    Some of the topics covered in this module include:

    \begin{itemize}
        \item Exposure to the language of modeling.
        \item Exploration of different types of models used in business and how to apply them in practice.
        \item Process of modeling.
        \item Characteristics of the models.
        \item Value and limitations of quantitative models. What sort of things they can and cannot do. Understand the limitations of the models.
        \item Provide a set of foundational for other courses in specialization.
    \end{itemize}

    \begin{center}
        \textit{Which model should I use?}
    \end{center}

    Process: map the characteristics of the business into the characteristics of the model.

    \section{Definition, Uses of a Model, Common Functions}

    In the business context, the models we are talking about are not physical models (different from the idea an arquitect has). We are talking about a \textbf{formal description of a business process}.

    This description is going to involve a set of mathematical equations and/or random variables.

    \begin{obs}
        A quantitative model it is almost always a simplification of a more complex structure (in particular, of a business process). We do not want to over simplify, but we also do not want to overcomplicate.

        Turns out it's going to be really difficult to make an exact and acurate representation of what we really want to model.
    \end{obs}

    Also, there is a set of \textbf{assumptions} that underly the model. We have to check weather these assumptions are reasonable or not in our business process.

    \begin{obs}
        A model is usually implemented in Excel or a spreadsheet tool like Google Sheets (or using programming langauges as $R$).
    \end{obs}

    In a more mathematical way, we can define a model as follows:

    \begin{mydef}[\textbf{Model}]
        A model is a triplet $M=(X,\Theta, F)$, where:
        \begin{itemize}
            \item $X$ is a set of measurable input variables (data).
            \item $\Theta$ is a set of parameters and,
            \item $\cf{F}{X\times\Theta}{Y}$ is a function, that maps input and parameters to measurable output in $Y$.
        \end{itemize}
    \end{mydef}

    Let's give some examples:

    \begin{exa}
        Some examples of models are the following:
        \begin{itemize}
            \item Determine the price of a diamond given its weight.
            \item The spread of an epidemic over time.
            \item Relationship between demand for, and price of a product.
            \item The uptake of a new product in market.
        \end{itemize}
        In particular, if I have a product and I want to maximize the gains we acquire when selling it, how can we achieve this goal?
    \end{exa}

    All of these examples are from different areas, but can be addressed by using a quantitative model.

    \subsection{Diamonds and Weight}

    Let's consider the weight of a diamond and the price it is going to have. We have the following equation that models the price of a diamond given its weight:
    \begin{equation*}
        p(w)=-260+3721w
    \end{equation*}
    This is a linear model. The price is given in dolars and the weight is given in carats. Sometimes we use a visual representation like this:

    \begin{figure}[h]
        \begin{minipage}{\textwidth}
            \centering
            \includegraphics[scale=1]{images/fig_1/fig_1.pdf} \\
            \caption{Model of the Price of a Diamond}
            \label{figure:model_price_diamond}
        \end{minipage}
    \end{figure}

\end{document}