\documentclass[../fundamentals_modeling.tex]{subfiles}

\begin{document}

    \chapter{Introduction}

    \section{Topics}

    Some of the topics covered in this module include:

    \begin{itemize}
        \item Exposure to the language of modeling.
        \item Exploration of different types of models used in business and how to apply them in practice.
        \item Process of modeling.
        \item Characteristics of the models.
        \item Value and limitations of quantitative models. What sort of things they can and cannot do. Understand the limitations of the models.
        \item Provide a set of foundational for other courses in specialization.
    \end{itemize}

    \begin{center}
        \textit{Which model should I use?}
    \end{center}

    Process: map the characteristics of the business into the characteristics of the model.

    \section{Definition, Uses of a Model, Common Functions}

    In the business context, the models we are talking about are not physical models (different from the idea an arquitect has). We are talking about a \textbf{formal description of a business process}.

    This description is going to involve a set of mathematical equations and/or random variables.

    \begin{obs}
        A quantitative model it is almost always a simplification of a more complex structure (in particular, of a business process). We do not want to over simplify, but we also do not want to overcomplicate.

        Turns out it's going to be really difficult to make an exact and acurate representation of what we really want to model.
    \end{obs}

    Also, there is a set of \textbf{assumptions} that underly the model. We have to check weather these assumptions are reasonable or not in our business process.

    \begin{obs}
        A model is usually implemented in Excel or a spreadsheet tool like Google Sheets (or using programming langauges as $R$).
    \end{obs}

    In a more mathematical way, we can define a model as follows:

    \begin{mydef}[\textbf{Model}]
        A model is a triplet $M=(X,\Theta, F)$, where:
        \begin{itemize}
            \item $X$ is a set of measurable input variables (data).
            \item $\Theta$ is a set of parameters and,
            \item $\cf{F}{X\times\Theta}{Y}$ is a function, that maps input and parameters to measurable output in $Y$.
        \end{itemize}
    \end{mydef}

    Let's give some examples:

    \begin{exa}
        Some examples of models are the following:
        \begin{itemize}
            \item Determine the price of a diamond given its weight.
            \item The spread of an epidemic over time.
            \item Relationship between demand for, and price of a product.
            \item The uptake of a new product in market.
        \end{itemize}
        In particular, if I have a product and I want to maximize the gains we acquire when selling it, how can we achieve this goal?
    \end{exa}

    All of these examples are from different areas, but can be addressed by using a quantitative model.

    \subsection{Diamonds and Weight}

    Let's consider the weight of a diamond and the price it is going to have. We have the following equation that models the price of a diamond given its weight:
    \begin{equation*}
        p(w)=-260+3721w
    \end{equation*}
    This is a linear model. The price is given in dolars and the weight is given in carats. Sometimes we use a visual representation like this:

    \begin{figure}[h]
        \begin{minipage}{\textwidth}
            \centering
            \includegraphics[scale=1]{images/fig_1/fig_1.pdf} \\
            \caption{Model of the Price of a Diamond}
            \label{figure:model_price_diamond}
        \end{minipage}
    \end{figure}

    Basically, this model allows us to forcast the price of a certain object given certain properties of it or \textit{data about it}. This is a \textbf{linear model}.

    \begin{obs}
        A model doesn't necesarly has to work in all the circustamces. For example, the model given to obtain the price of a diamond doesn't necesarly has to be applicable to a different company.
    \end{obs}

    \subsection{Spreading of an Epidemic}

    Depending on the disease and other factors (such as time in which the epidemc ocurred), one model arises.

    One of the basic models to start with is the \textbf{exponential model}. On the bottom axis we have weeks (since the start of the epidemic), and on the vertical axis we have the number of cases reported.

    \begin{figure}[h]
        \begin{minipage}{\textwidth}
            \centering
            \includegraphics[scale=1]{images/fig_2/fig_2.pdf} \\
            \caption{Exponential Model for the spread of an epidemic.}
            \label{Texto}
        \end{minipage}
    \end{figure}

    The equation to describe this model is:
    \begin{equation*}
        y = 6.69e^{0.18x}
    \end{equation*}
    
    \begin{obs}
        This model is suitable for the start of the pandemic, since there is not an unlimited number of people, we cannot expect that the growth of the graph is unstoppable.
    \end{obs}

    \begin{obs}[\textbf{Exponential Graphs}]
        In the context of business they are called \textbf{hockey sticks}.
    \end{obs}

    \begin{center}
        \textit{This model is not suitable for a long term, but its suitable for an approximation over the first weeks of the pandemic}.
    \end{center}

    \subsection{Price and Demand}

    Let's suppose a situation of price and demand (very common in the industry). Demand models basically tell when we know the price of a certain product, the quantity of it available in the market.

    \begin{obs}[\textbf{Possitive Association}]
        The graphs in the first two examples are said to have \textbf{positive association}.
    \end{obs}

    For products, when there is a huge amount of product, the price decreases and, when the price is too hight, the amout available is too low. The model:

    \begin{figure}[h]
        \begin{minipage}{\textwidth}
            \centering
            \includegraphics[scale=1]{images/fig_3/fig_3.pdf} \\
            \caption{Price and Demand Model}
            \label{figure:price_demand_model}
        \end{minipage}
    \end{figure}

    This model is given by the equation:
    \begin{equation*}
        Q=60,000 P^{ -2.5}
    \end{equation*}

    In this model we use a power function (which is basically the exponential function in essence, but many people who doesn't have mathematical background calls is \textit{power function}).

    \begin{idea}
        One use of this model is to find what the optimal price for a product should be.
    \end{idea}

    \subsection{The Uptake of a Product}

    \begin{mydef}[\textbf{Uptake}]
        The \textbf{uptake of a product}, often referred to as \textbf{product adoption} or \textbf{market uptake}, is \textit{the process by which individuals or organizations learn about, start using, and ultimately integrate a new product or service into their routine life or business operations}.
    \end{mydef}
    
    The following describes the uptake of a product given the amount of years it has been on the market.

    \begin{figure}[h]
        \begin{minipage}{\textwidth}
            \centering
            \includegraphics[scale=1]{images/fig_4/fig_4.pdf} \\
            \caption{Model of the Uptake of a Product}
            \label{figure:uptake_logistic_model}
        \end{minipage}
    \end{figure}

    This particular function is called a logistic function, and is given by:
    \begin{equation*}
        P=\frac{e^{ 2(Y-2.5)}}{1+e^{ 2(Y-2.5)}}
    \end{equation*}
    Here, $Y$ is the number of years and, $P$ is the proportion of target population with product.

    \begin{obs}
        This model has the potential to model a process where at the first stages we see a slow start (here, it's when target people start adopting the product), and then it has a fast grow that proceeds to slow as time goes by.
    \end{obs}

    \section{Use of Models in Practice}

    There are four main usages of a model in the business enviroment:
    \begin{itemize}
        \item \textbf{Prediciton}.
        \item \textbf{Forecasting}.
        \item \textbf{Optimization}.
        \item \textbf{Ranking and targeting}.
    \end{itemize}

    \subsection{Prediction}

    Once we got a quantitative model is \textbf{prediction}. Basically is taking a model, puting an input and create a prediction.

    \begin{obs}
        The use of most quantitative models is for predictive analysis.
    \end{obs}

    \subsection{Forecasting}

    When talking about forecasting we are talking about time series.

    \begin{mydef}[\textbf{Forecasting}]
        \textbf{Forecasting} is the \textit{process of predicting future developments by analyzing past and present data and trends}.
    \end{mydef}

    \begin{exa}
        For example, with the models presented earlier, we can ask the following questions:
        \begin{itemize}
            \item How many people are expected to be infected in 6 weeks?
            \item Using a scheduling model, who is likely to turn up for their outpatient appointment?
        \end{itemize}
    \end{exa}

    This activity has often a lot to do in businesses and to do with resource planning.

    \subsection{Optimization}

    How to maximize or minimize something?

    \subsection{Ranking and targeting}

    When selling a product, we may be interested in which ones we would like to purchase. Since we cannot have a look at all the diamonds in the world, 

\end{document}