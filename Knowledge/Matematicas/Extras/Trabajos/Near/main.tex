\documentclass[twoside,12pt,a4 paper,openright]{book}
%\setlength{\textwidth}{15cm} \setlength{\textheight}{21cm} %esta instrucci\'on sirve para indicar que tanto quieres ocupar de tu hoja
%\setlength{\oddsidemargin}{1.5cm} \topmargin-0.5cm %esta es para indicar el margen, si quieres puedes omitir esta y la instruccion anterior
\usepackage[latin1]{inputenc}
\usepackage[spanish]{babel}
\usepackage{latexsym}
\usepackage{amssymb}
\usepackage{amsthm}
\usepackage[all]{xy}
\usepackage{makeidx}
\usepackage{multicol}
\usepackage{amsfonts}
\usepackage{amsmath}
%\usepackage{graphicx}
\usepackage{color}
\usepackage{graphics}
\usepackage{amssymb,amsthm,amsmath,amsfonts}
\usepackage{graphicx}
\usepackage{cite}
\usepackage{color}
\usepackage{enumitem}
\usepackage{mathrsfs}



\newtheorem{claim}{Proposition}[chapter]
\newtheorem{teo}[claim]{Teorema}
\newtheorem{cor}[claim]{Corolario}
\newtheorem{lem}[claim]{Lema}
\newtheorem{pro}[claim]{Proposici\'on}
\newtheorem{ejer}[claim]{Ejercicio}
\newtheorem{ejem}[claim]{Ejemplo}
\newtheorem{defi}[claim]{Definici\'on}
\newtheorem{fig}[claim]{Figura}
\newtheorem{tab}[claim]{Tabla}
\newtheorem{tar}[claim]{Tarea}
\newtheorem{ob}[claim]{Observaci\'on}
\newtheorem{nota}[claim]{Notaci\'on}
\title{Teor\'ia de la medida}
\author{      Instituto Politecnico Nacional \\
\\
							Profesor: Jos\'e Oscar Gonz\'alez Cervantes}
                      

\begin{document}




\maketitle

\tableofcontents

\vspace{2cm}
Evaluaci\'on parcial (al finalizar cada ca\'itulo): 
\\
\\

50 \% entregra de ejercicios + 50 \% examen escrito.  
\\
\\


Bibliograf\'ia:

\begin{enumerate}
\item Introductory Real Analysis. A. N. Kolmogorov, S.V. Fomin, Ed. Dover publications.

\item Measure Theory. Paul Halmos. Springer Verlag.


\item Introduction to  Real Analysis,  Bartle, Ed. John Wiley.

\item Real and Complex Analysis. W. Rudin, Ed. Mc Graw-Hill.  
\end{enumerate}



\chapter{Preliminares}  




\begin{ejer} \ {}

\newpage

\begin{enumerate}


\item Son las leyes de distributividad de la teor\'ia de conjuntos: 
%2
{\small
\begin{enumerate}
\item[\textcolor{red}{$\bullet$}] {} $ (A\cup B)\cap C   =(A\cap C)\cap (B\cap C) , \quad (A\cap B)\cap C   =(A\cup C)\cap (B\cup C) $. 
\item[\textcolor{blue}{$\bullet$}]  {} $ (A\cup B)\cap C   =(A\cap C)\cup (B\cap C) , \quad (A\cap B)\cup C   =(A\cup C)\cap (B\cup C) .$ 
\item[\textcolor{yellow}{$\bullet$}] {} $(A\cap B)\cap C   =(A\cap C)\cup (B\cap C) , \quad (A\cap B)\cup C   =(A\cup C)\cap (B\cap C) .$ 
\item[\textcolor{green}{$\bullet$}] {} $(A\cup B)\cap C   =(A\cap C)\cup (B\cup C) , \quad (A\cup B)\cup C   =(A\cup C)\cap (B\cup C) .$
\end{enumerate}
}


\newpage

\item Son los principios de dualidad (leyes de Morgan):
%4
{\small
\begin{enumerate}
\item[\textcolor{red}{$\bullet$}] {} $\left(\bigcup_{\alpha \in I}A_{\alpha}\right)^{c} =\bigcap_{\alpha \in I}A_{\alpha}^{c}, \quad \left(\bigcap_{\alpha \in I} A_{\alpha}\right)^{c} = \bigcap_{\alpha \in I} A_{\alpha}^{c} $. 
\item[\textcolor{blue}{$\bullet$}] {} $\left(\bigcup_{\alpha \in I}A_{\alpha}\right)^{c} =\bigcap_{\alpha \in I}A_{\alpha}^{c}, \quad \left(\bigcap_{\alpha \in I} A_{\alpha}\right)^{c} = \bigcup_{\alpha \in I} A_{\alpha}  $. 
\item[\textcolor{yellow}{$\bullet$}] {} $\left(\bigcup_{\alpha \in I}A_{\alpha}\right)^{c} =\bigcap_{\alpha \in I}A_{\alpha}^{c}, \quad \left(\bigcup_{\alpha \in I} A_{\alpha}\right)^{c} = \bigcup_{\alpha \in I} A_{\alpha}^{c} $. 
\item[\textcolor{green}{$\bullet$}] {} $\left(\bigcup_{\alpha \in I}A_{\alpha}\right)^{c} =\bigcap_{\alpha \in I}A_{\alpha}^{c}, \quad \left(\bigcap_{\alpha \in I} A_{\alpha}\right)^{c} = \bigcup_{\alpha \in I} A_{\alpha}^{c} $. 
\end{enumerate}
}


\newpage


\item Es la diferencia sim\'etrica entre $A$ y $B$:
%3
{\small
\begin{enumerate}
\item[\textcolor{red}{$\bullet$}] {}
 $A\Delta B: = (A\setminus  B )\cap (B\setminus A)   = (A\cup B) \setminus (A\cap B)$.
\item[\textcolor{blue}{$\bullet$}] {}
 $A\Delta B: = (A\setminus  B )\cup (B\setminus A)   = (A\cup B) \setminus (A\cup B)$.
 \item[\textcolor{yellow}{$\bullet$}] {}
 $A\Delta B: = (A\setminus  B )\cup (B\setminus A)   = (A\cup B) \setminus (A\cap B)$.
\item[\textcolor{green}{$\bullet$}] {}
 $A\Delta B: = (A\setminus  B )\cup (B\setminus A)   = (A\cup B) \cup (A\cap B)$.
\end{enumerate}
}



\newpage


\item Sean $a,b\in \mathbb{R}$ tales que $a< b$. Hallar 
$$\displaystyle \bigcup_{n\in \mathbb N} \left[a+\frac{1}{n}, b-\frac{1}{n}\right] , \quad   \displaystyle \bigcap_{n\in \mathbb N} \left(a-\frac{1}{n}, b+\frac{1}{n}\right) .$$
%1
{\small
\begin{enumerate}
\item[\textcolor{red}{$\bullet$}] {}  $(a,b)$ y $[a,b]$, respectivamente.
\item[\textcolor{blue}{$\bullet$}] {}    $[a,b]$ y $(a,b)$, respectivamente.
\item[\textcolor{yellow}{$\bullet$}] {} $\emptyset$ y $\mathbb R$, respectivamente.
\item[\textcolor{green}{$\bullet$}] {} $\mathbb R$ y $\emptyset$, respectivamente.
\end{enumerate}
}


\newpage


\item     Sean N,M conjuntos no vac\'ios y sea $f:N \rightarrow M$, dados $A,B \subset M$ se cumple:
%4
{\small
\begin{enumerate}
\item[\textcolor{red}{$\bullet$}] {}
     $f^{-1}(A \cap B) = f^{-1}(A) \cup f^{-1}(B)$, \ \ \ $f^{-1}(A \cap B) = f^{-1}(A) \cap f^{-1}(B)$.
\item[\textcolor{blue}{$\bullet$}] {}
     $f^{-1}(A \cup B) = f^{-1}(A) \cup f^{-1}(B)$, \ \ \ $f^{-1}(A \cap B) = f^{-1}(A) \cup f^{-1}(B)$.
\item[\textcolor{yellow}{$\bullet$}] {}
     $f^{-1}(A \cup B) = f^{-1}(A) \cap f^{-1}(B)$, \ \ \ $f^{-1}(A \cap B) = f^{-1}(A) \cap f^{-1}(B)$.
\item[\textcolor{green}{$\bullet$}] {}
     $f^{-1}(A \cup B) = f^{-1}(A) \cup f^{-1}(B)$, \ \ \ $f^{-1}(A \cap B) = f^{-1}(A) \cap f^{-1}(B)$.
\end{enumerate}
}
     
     

\newpage



 

\item Dado un conjunto no vac\'io $A$. Una relaci\'on $R$ en $A$ es un subconjunto de $A\times A$.   Diremos que $R$, ($\sim$) es que equivalencia si es    
%2
{\small
\begin{enumerate}
 \item[\textcolor{red}{$\bullet$}] {} reflexiva,   anti-sim\'etrica y transitiva.
  \item[\textcolor{blue}{$\bullet$}] {}  reflexiva,   sim\'etrica y transitiva.
\item[\textcolor{yellow}{$\bullet$}] {} reflexiva y   sim\'etrica.
\item[\textcolor{green}{$\bullet$}] {} sim\'etrica y transitiva.
  \end{enumerate}
}



\newpage




\item Un  conjunto no vac\'io $A$ y  una relaci\'on $R$ en $A$. Entonces $A$ es descompuesto 
por conjuntos $ \{ y\in A \ \mid \ x R y\} = [y]$   si y solo si 
%3
{\small
\begin{enumerate}
 \item[\textcolor{red}{$\bullet$}] {}  $R$ es una relaci\'on.
\item[\textcolor{blue}{$\bullet$}] {}  $R$ es un orden parcial.
\item[\textcolor{yellow}{$\bullet$}] {} $R$ es una relaci\'on de equivalencia.
\item[\textcolor{green}{$\bullet$}] {}  $R$ es una relaci\'on transitiva.
\end{enumerate}
}



\newpage


\item Considere $\quad f(x,y) = x-y$, para cada $(x,y)\in \mathbb N^2$. Defina:  
        $(a,b)R(c,d) \Leftrightarrow f(a,b) = f(c,d)$ para cada $(a,b)\in \mathbb N^2$. La relaci\'on es:
 %1
        {\small
           \begin{enumerate}         
           \item[\textcolor{red}{$\bullet$}] {} Reflexiva, sim\'etrica y transitiva.
        \item[\textcolor{blue}{$\bullet$}] {} Reflexiva, pero no sim\'etrica y ni transitiva.
        \item[\textcolor{yellow}{$\bullet$}] {} Sim\'etrica pero no reflexiva, ni transitiva.
        \item[\textcolor{green}{$\bullet$}] {} Transitiva pero no reflexiva, ni sim\'etrica
    \end{enumerate}
}



\newpage



\item Diremos que  $A$ es numerable o contable si es finito o infinito numerable.
  Ejemplos de conjuntos infinito numerables
 %2
        {\small
    \begin{enumerate}
            \item[\textcolor{red}{$\bullet$}] {} $\mathbb{Z}$,  $\mathbb{Q}$, $\mathbb R$.
             \item[\textcolor{blue}{$\bullet$}] {} $\mathbb{Z}$,  $\mathbb{Q}$, $\mathbb N$.
      \item[\textcolor{yellow}{$\bullet$}] {}     $\mathbb{Z}$,  $\mathbb{Q}$, $\mathbb C$.
         \item[\textcolor{green}{$\bullet$}] {}    $\mathbb{Z}$,  $\mathbb{Q}$, $[0,1]$.
                 \end{enumerate}
}



\newpage




\item Dado un conjunto $A$. Los siguientes hechos son equivalentes: 
  % 3
          {\small
    \begin{enumerate}
\item[\textcolor{red}{$\bullet$}] {} \begin{enumerate}
 \item    $A$ es numerable. 
\item Existe  $f: \mathbb N \to A$ inyectiva.
\item Existe $g: A \to \mathbb N$ inyectiva.
\end{enumerate} 
\item[\textcolor{blue}{$\bullet$}] {} \begin{enumerate}
 \item $A$ es numerable. 
\item Existe  $f: \mathbb N \to A$ inyectiva.
\item Existe $g: A \to \mathbb N$ suprayectiva.
\end{enumerate} 
 \item[\textcolor{yellow}{$\bullet$}] {} \begin{enumerate}
 \item $A$ es numerable. 
\item Existe  $f: \mathbb N \to A$ suprayectiva.
\item Existe $g: A \to \mathbb N$ inyectiva.
\end{enumerate} 
\item[\textcolor{green}{$\bullet$}] {} \begin{enumerate}
 \item $A$ es numerable. 
\item Existe  $f: \mathbb N \to A$ suprayectiva.
\item Existe $g: A \to \mathbb N$ suprayectiva.
\end{enumerate} 
\end{enumerate} 
}

\newpage


\item Si $A_n$ es infinito numerable para cada $n\in \mathbb N $ entonces 
 % 4
          {\small
    \begin{enumerate}
\item[\textcolor{red}{$\bullet$}] {}
$\bigcup_{n\in \mathbb N} A_n$ es finito. 
\item[\textcolor{blue}{$\bullet$}] {}
$\bigcup_{n\in \mathbb N} A_n$ es infinito no numerable. 
\item[\textcolor{yellow}{$\bullet$}] {}
$\bigcup_{n\in \mathbb N} A_n$ es vac\'io. 
\item[\textcolor{green}{$\bullet$}] {}
$\bigcup_{n\in \mathbb N} A_n$ es infinito numerable. 
\end{enumerate}
}

\newpage


\item  Teorema de Cantor:
Dado un conjunto no vac\'io  $A$ entonces no existe $g: A \to \mathcal P(A)$  que sea 
 % 2
          {\small
    \begin{enumerate}
\item[\textcolor{red}{$\bullet$}] {}  inyectiva.
\item[\textcolor{blue}{$\bullet$}] {} suprayectiva.
\item[\textcolor{yellow}{$\bullet$}] {} lineal
\item[\textcolor{green}{$\bullet$}] {} no lineal
\end{enumerate}
}
 
\newpage


\item Por $\aleph_0$ (Aleph cero) representaremos la cantidad de n\'umeros naturales. Se verifica que $\mathcal P(\mathbb N)$, $[0,1]$ y $\mathbb R$ son todos equivalentes. Como $[0,1] $ es infinito no numerable se dice que tiene la cardinalidad 
 % 1
          {\small
    \begin{enumerate}
\item[\textcolor{red}{$\bullet$}] {}  del cont\'inuo o la cardinalidad   $\aleph_1$ (Aleph uno).
\item[\textcolor{blue}{$\bullet$}] {}  de $\mathbb N$.
\item[\textcolor{yellow}{$\bullet$}] {}  $\aleph_0$ (Aleph cero).
\item[\textcolor{green}{$\bullet$}] {}   $\aleph_2$ (Aleph dos).
\end{enumerate}
}


\newpage


\item  Teorema de Cantor-Bernstein: Sea $A$ y $B$ dos conjuntos tales que existen $A_1\subset A$ y $B_1\subset B$ donde $A_1$ es equivalente a $B$ y $B_1$ es equivalente a $A$. 
 % 3
          {\small
    \begin{enumerate}
\item[\textcolor{red}{$\bullet$}] {} Entonces $A_1$ y $B_1$ son equivalentes. 
\item[\textcolor{blue}{$\bullet$}] {} Entonces $A_1$ y $B$ son equivalentes. 
\item[\textcolor{yellow}{$\bullet$}] {}  Entonces $A$ y $B$ son equivalentes. 
\item[\textcolor{green}{$\bullet$}] {}  Entonces $A$ y $B_1$ son equivalentes. 
\end{enumerate}
}
 

\newpage


\item Dado un conjunto no vaci\'o $A$. Una relaci\'on $R$ en $A$ se dice orden parcial si 
 % 4
          {\small
    \begin{enumerate}
\item  \textcolor{red}{$\bullet$} {} es  reflexiva, sim\'etrica, transitiva. 
\item  \textcolor{blue}{$\bullet$} {} es  antisim\'etrica, transitiva y no es reflexiva. 
\item  \textcolor{yellow}{$\bullet$} {} es  reflexiva, sim\'etrica y no es  transitiva. 
\item  \textcolor{green}{$\bullet$} {} es  reflexividad, antisimetr\'ia, transitividad. 
\end{enumerate}
}

\newpage



\item  $( \mathbb{R}, \leq )$, $(\mathcal{P(A)}, \subset )$,  $(\mathbb{N}\setminus\{0\}, \ | )$(divisi\'on) 
 son ejemplos de conjuntos 
 % 1
          {\small
    \begin{enumerate}
\item[\textcolor{red}{$\bullet$}] {} parcialmente ordenados. 
\item[\textcolor{blue}{$\bullet$}] {} ordenados. 
\item[\textcolor{yellow}{$\bullet$}] {} bien ordenados. 
\item[\textcolor{green}{$\bullet$}] {} ninguno de los de arriba. 
\end{enumerate}
}


\newpage

\item  Sea  $(A,R) $ un conjunto parcialmente ordenado. Diremos que  $B\subset A$ es una cadena en $A$ 
si
 % 2
          {\small
    \begin{enumerate}
\item[\textcolor{red}{$\bullet$}] {} $xRx$ para cada $x\in B$
\item[\textcolor{blue}{$\bullet$}] {} $x R y$ \'o $yRx$ para cualesquiera $x,y\in B$. 
\item[\textcolor{yellow}{$\bullet$}] {} $x R y$ \'o $yRx$ para cualesquiera $x,y\in A$. 
\item[\textcolor{green}{$\bullet$}] {} $x R y$ \'o $yRx$ para cualesquiera $x,y\in A\setminus B$. 
\end{enumerate}
}

\newpage


 \item $(A,R)$ se dice ordenado, simple, linealmente o totalmente ordenado, si 
 % 4
          {\small
    \begin{enumerate}
\item[\textcolor{red}{$\bullet$}] {} es un conjunto parcialmente ordenado $A$. 
\item[\textcolor{blue}{$\bullet$}] {} es no es un conjunto parcialmente ordenado $A$.
\item[\textcolor{yellow}{$\bullet$}] {} es un conjunto parcialmente ordenado $A$ y no existen cadenas en $A$. 
\item[\textcolor{green}{$\bullet$}] {} es un conjunto parcialmente ordenado y $A$ es un cadena en $A$. 
 \end{enumerate}
}


 



 


\newpage


\item Un conjunto bien ordenado es un conjunto 
 % 3
          {\small
    \begin{enumerate}
\item[\textcolor{red}{$\bullet$}] {} es un conjunto parcialmente ordenado $A$. 
\item[\textcolor{blue}{$\bullet$}] {} es no es un conjunto parcialmente ordenado $A$.
\item[\textcolor{yellow}{$\bullet$}] {} ordenado  $(A,R) $ en el que todo subconjunto no vac\'io tiene un elemento m\'inimo (o primer elemento).
\item[\textcolor{green}{$\bullet$}] {} es un conjunto parcialmente ordenado y $A$ es un cadena en $A$. 
 \end{enumerate}
}


\newpage



\item Conjuntos finitos y $\mathbb N$ son ejemplos de conjuntos  
 % 1
          {\small
    \begin{enumerate}
\item[\textcolor{red}{$\bullet$}]{} bien ordenados. 
\item[\textcolor{blue}{$\bullet$}] {} que no son  parcialmente ordenados.
\item[\textcolor{yellow}{$\bullet$}] {}  infnitos no numerables.
\item[\textcolor{green}{$\bullet$}] {} que nos son  ordenados.
  \end{enumerate}
}





\end{enumerate}
\end{ejer}


\newpage
 
 Sobre algunos resultados importantes de la teor\'ia de conjuntos en la matem\'atica discreta.  Los siguientes hechos son equivalentes: 
\begin{enumerate}
\item  Principio del buen orden: Todo conjunto puede ser dotado  de un buen orden. 1904. Ernst Zermelo. Universidad de Berlin.    

\item Axioma maximal de Hausdorff. En un conjunto parcialmente ordenado siempre existe al menos una cadena maximal. 1909. Felix Hausdorff. Alemania.

\item Lema de Zorn: Todo conjunto parcialmente ordenado en el que toda cadena tiene una cota superior contiene al menos una elemento maximal. 1935. Max Zorn. Alemania.

\item Axioma de elecci\'on: Dado cualquier conjunto $A$ existe una funci\'on de elecci\'on  $f$ tal que $f(B)\in B$ para cada $B\subset A$ no vac\'io. Erns Zermelo.  

\item Axioma de inducci\'on matem\'atica. Ejercicio: Redactar.  
\end{enumerate}
 




\chapter{Teor\'ia de la medida} 








\section{Sistemas de conjuntos}
 Semianillos, anillos, \'algebras y $\sigma$-\'algebras de conjuntos.


\begin{defi}
Una familia $R$, o un sistema, de conjuntos no vac\'ia es llamada anillo si 
$A\Delta B,\ A\cap B \in \mathcal R$ para cada $A,B\in \mathcal R$.  
\end{defi}

\noindent
Como $A\cup B= (A\Delta B) \Delta (A\cap B) $ y $A\setminus B= A\Delta (A\cap  B)$ para cada $A,B \in \mathcal R$ entonces en un anillo tambi\'en se cumple que  $A\cup B, A\setminus B \in \mathcal R$ para cada $A,B\in \mathcal R$.

\noindent
Note que $\emptyset =A\setminus A$ para $A\in \mathcal R$ tenemos que $\emptyset \in \mathcal R$.

\noindent
M\'as a\'un, si $ A_1,\dots, A_n\in \mathcal R$ entonces $\displaystyle \bigcap_{k=1}^n A_k, \bigcup_{k=1}^n  A_k \in \mathcal R$ para cada $A_1,\dots, A_n \in \mathcal R$. 


\begin{defi} Un elemento $E\in A$ de una familia $R$  de conjuntos no vac\'ia es llamado  unidad   si 
$A \cap E  = A$ para cada $A\in \mathcal R$.

\noindent
 Una familia $R$  de conjuntos no vac\'ia es llamada \'algebra si  es un anillo con unidad.  
\end{defi}

\noindent
  Dado un conjunto $A $ no vac\'io claramente  tenemos que el conjunto potencia de $A$, denotado por  $\mathcal P(A)$, es un  \'algebra donde  $A$ es la identidad.


\begin{ejer} \ {} 
\begin{enumerate}
\item Dado $A$ un conjunto no vac\'io muestre  que  $\{\emptyset, A\}$ es un \'algebra con unidad $A$
\item Dado $A$ un conjunto  no vac\'io muestre que 
$   \{F\subset A \ \mid \ F \textrm{ es finito }\}$  es un anillo, pero si $A$ es finito entonces es un  \'algebra con unidad $A$.   
\item Muestre que $\{ B \subset \mathbb R^n  \ \mid \ B \textrm{ es acotado}\}$
 es una anillo pero no un \'algebra.
 
 \item Muestre que la intersecci\'on de cualquier familia de anillos es un anillo.

 \end{enumerate}



\end{ejer}

\begin{teo}
Dada una familia de conjuntos $\Phi$ no vac\'ia existe un anillo $\mathcal R$ tal que  $\Phi \subset \mathcal R$ y si $\mathcal R' $ es otro anillo que cumple 
  $\Phi \subset \mathcal R'$ entonces  $ \mathcal R   \subset \mathcal R'$
\end{teo}
\begin{proof}
Sea $\displaystyle A=  \bigcup_{B\in \Phi} B$. As\'i que  $\mathcal P(A)$ es un anillo  y $\Phi \subset \mathcal P(A)$.  Defina 
$$\mathcal R = \bigcap_{ \begin{array}{c} \Phi \subset R  \subset \mathcal P(A)  \\ 
R \  \textrm{ es anillo }  \end{array} } R. 
$$
 Luego $\mathcal R$ es un anillo y si $\mathcal R' $ es otro anillo que cumple 
  $\Phi \subset \mathcal R'$  entonces $\mathcal R \subset \mathcal P(A)  \cap \mathcal R' \subset \mathcal R'$.
\end{proof}

Notaci\'on: Dada una familia de conjuntos $\Phi$ no vac\'ia. El anillo $\mathcal R$ mostrado en el teorema anterior se denotar\'a por $\mathcal R(\Phi)$.


\begin{defi}
Una familia $\mathcal S$ de conjuntos no vac\'ia es llamada semianillo de conjuntos si y s\'olo si 
\begin{enumerate}
\item  $\emptyset \in \mathcal S$.
\item  $A\cap B \in \mathcal A$ para cada $A,B\in \mathcal S$.
\item  Si $A, A_1 \in \mathcal S$ con $A_1 \subset A $ entonces existen $A_2, \dots A_n \in \mathcal S$ ajenos a pares  tales 	que $A= A_1\cup A_2 \cup \cdots \cup A_n$. Adem\'as, la familia $\{A_1,\dots, A_n\}$ es llamada expansi\'on finita de $A$.  
\end{enumerate}

\end{defi}





\begin{ejer}Muestre lo siguiente:
\begin{enumerate}
\item Todo anillo es un semianillo. 
\item   La familia de conjuntos formada por  intervalos de la forma $(a,b)$, $ [a,b]$,  $ [a,b)$,  $(a,b]$ con $a\leq b$ es un  semianillo pero no  es un anillo. 
\item La familia de rect\'angulos $I\times J$ en $\mathbb R^2$, donde $I$ y $J$ son intervalos  dados por el ejercicio anterior,  es un  semianillo pero no  es un anillo. 

\end{enumerate}
\end{ejer}

\begin{lem}
Dado un semianillo $\mathcal S$ y $A,A_1,\dots, A_n\in \mathcal S$ tales que $A_1,\dots, A_n $ son subconjuntos ajenos a pares de $A$. Entonces $A$ tiene una expansi\'on finita de la forma 
$A=A_1\cup \cdots \cup A_s $ con $s\geq n$.
\end{lem}
\begin{proof} Inducci\'on matem\'atica. 

\noindent
Si $n=1$ y como  $\mathcal S $ es un semianillo  existen $A_2, \dots, A_s\in \mathcal S$ tales que $A_1\cup A_2\cup \cdots \cup A_s = A$.

\noindent
Hipotesis de inducci\'on. Suponga que el hecho es cierto para  $A_1,\dots, A_n \in\mathcal S$ son subconjuntos ajenos a pares de $A$.

\noindent
Sean $A,A_1,\dots, A_n, A_{n+1}\in \mathcal S$  subconjuntos ajenos a pares de $A$. Como   $A_1,\dots, A_n $ y $A$ cumplen la hip\'otesis de inducci\'on   existen $B_{n+1},\dots ,B_s \in S$ ajenos a pares tales que  
$$A=A_1\cup \cdots \cup A_n \cup B_{n+1} \cup \cdots \cup B_s.$$ 
\noindent
Adem\'as,  $A_{n+1} = (B_{n+1}\cap A_{n+1}) \cup \cdots \cup (A_s \cap A_{n+1})$. 
\noindent
Por otra parte,  para cada   $i=n+1, \dots, s$ tenemos que    $B_{i} \cap A_{n+1} \subset B_i $ son elementos de $S$. As\'i que  existen $C^i_{1},\dots, C^i_{p_i} \in S$ tales que 
una expansi\'on finita de  $B_i $ partiendo de $B_{i} \cap A_{n+1}$ es
 $B_i=  (B_{i} \cap A_{n+1}) \cup C^i_{1}\cup \cdots \cup  C^i_{p_i} $,
  para $i=n+1,\dots, s$.  Entonces 
\begin{align*} 
A= & A_1\cup \cdots \cup A_n \cup B_{n+1} \cup \cdots \cup B_s\\ 
  = & A_1\cup \cdots \cup A_n  \cup \left[ \bigcup_{i=n+1}^s  (B_{i} \cap A_{n+1}) \cup C^i_{1}\cup \cdots \cup  C^i_{p_i} \right]  \\
   = & A_1\cup \cdots \cup A_n  \cup \left[ \bigcup_{i=n+1}^s  (B_{i} \cap A_{n+1}) \right] \cup  \left[ \bigcup_{i=n+1}^s  C^i_{1}\cup \cdots \cup  C^i_{p_i} \right]  \\
      = & A_1\cup \cdots \cup A_n  \cup  A_{n+1}  \cup  \left[ \bigcup_{i=n+1}^s  C^i_{1}\cup \cdots \cup  C^i_{p_i} \right]. 
  \end{align*} 

La cual es una expansi\'on finita de  $A$. 

\end{proof}


 


\begin{pro}
Si  $S$ es un semianillo entonces  $ \mathcal R (S)$
 esta formado por todas las expansiones finitas en elementos de $S$; es decir, $A\in \mathcal R(S) $ si y s\'olo si existen 
 $A_1,\dots, A_n\in S$ ajenos a pares tales que $\displaystyle A=\bigcup_{k=1}^n  A_k$.
\end{pro}
\begin{proof}
Veremos que la familia de expansiones finitas en elementos de $S$ forman un anillo. Sean $\displaystyle \bigcup_{i = 1}^{n} A_i, \bigcup_{j= 1}^{m} B_j$ expansiones finitas en elementos de $S$; es decir,     $\{A_i  \in S \ \mid  \  i = 1,\cdots , n\}$ y $\{B_j  \in S \ \mid \ j=1,\cdots,m\}$ son familias de conjuntos ajenos a pares. 

Denote $C_{ij} = A_i \cap B_j \in S$ y use  el lema anterior en $A_i, C_{i1}, \cdots C_{im}$. Luego  $A_i$ tiene una expansi\'on finita de la forma

$$A_i = \left(\bigcup_{j= 1}^{m} C_{ij} \right)\cup \left(\bigcup_{k= 1}^{r_i}D_{ik}\right) \quad (D_{ik} \in S)$$

Similarmente, 
el lema anterior en $B_j, C_{1j}, \cdots C_{nj}$ nos da la expansi\'on finita:  
$$B_j = \left(\bigcup_{i= 1}^{n} C_{ij} \right)\cup \left(\bigcup_{k= 1}^{s_j}E_{kj}\right) \quad (E_{kj} \in S)$$
Vemos que 
\begin{align*}
    \left(\bigcup_{i = 1}^{n}A_i\right) \cap \left(\bigcup_{j=1}^{m}B_j\right) = & \bigcup_{j=1}^{m} \bigcup_{i=1}^{n}C_{ij}   ,\\
     \left(\bigcup_{i = 1}^{n}A_i\right) \Delta  \left(\bigcup_{j=1}^{m}B_j\right) = & \left(\bigcup_{i = 1}^{n}\bigcup_{k= 1}^{r_i}D_{ik}\right)\cup \left(\bigcup_{j = 1}^{m}\bigcup_{k= 1}^{s_j}E_{kj}\right).
\end{align*}
Las cuales son de nuevo expansiones finitas en elementos de $S$. 

Por \'ultimo, note  que cualquier    anillo que contenga a $S$ debe  tener  cualquier expansi\'on finita  entre sus elementos. Por lo tanto $\mathcal{R(S)}$ est\'a formada por expansiones finitas en elementos de $S$

\end{proof}



\begin{ejem} \
\begin{enumerate}
\item 
Dados $a<b$.  La  familia  formada por uniones de   intervalos ajenos a pares contenidos en  $ [a,b]$ es   un \'algebra.
\item Dado $n\in \mathbb N$. Por simplicidad diremos que un rect\'angulo  en $\mathbb R^n$ es un conjunto de la forma $I_1 \times \cdots \times I_n$, donde $I_1,\dots, I_n \subset \mathbb R$ son intervalos.  Se verifica que la  familia  formada por uniones de  rect\'angulos en $\mathbb R^n$  ajenos a pares contenidos en  $ [0,1 ]^n$ es un \'algebra.    
\end{enumerate}
\end{ejem}



\begin{defi} \ {} \begin{enumerate}
\item 
Un anillo $R$ es llamado $\sigma$-anillo, si para cada familia $\{A_i \}_{i\in \mathbb N}$ de elementos  de $R$ se cumple que $\bigcup_{i\in\mathbb N} A_i\in R$.

\item Un anillo $R$ es llamado $\delta$-anillo, si para cada familia $\{A_i \}_{i\in \mathbb N}$ de elementos  de $R$ se cumple que $\bigcap_{i\in\mathbb N} A_i\in R$.



\item Un $\sigma$-anillo  con identidad es llamado $\sigma$-\'algebra.    


\item Un $\delta$-anillo  con identidad es llamado $\delta$-\'algebra.  

\end{enumerate}

\end{defi}





\begin{teo}
Toda $\sigma$-\'algebra es $\delta$-\'algebra y rec\'iprocamente.
\end{teo}
\begin{proof}
Es consecuencia de las identidades 
\begin{align*}
E \setminus ( \bigcap_{i\in \mathbb N} E\setminus A_i ) = \bigcup_{i\in \mathbb N} A_i, \quad   
E \setminus ( \bigcup_{i\in \mathbb N} E\setminus A_i ) = \bigcap_{i\in \mathbb N} A_i,
\end{align*}
donde $E$ es la unidad.
\end{proof}




\begin{ejer}Muestre que las siguientes familias son $\sigma$-\'algebras. 
\begin{enumerate} 
\item Sea $E=\{1, 2, 3, 4,5\}$. Considere las familias $A = \{\emptyset, E,
\{1,2,3\}, \{4,5\}\}$, $ B = \{\emptyset , E, \{1\}, \{2\}, \{1,2\}, \{3,4,5\}, \{2,3,4,5\}, \{1,3,4,5\}     \}$.

\item   La familia
$ F = \{ A \subset \mathbb R^n  \ \mid \  A \ \textrm{es contable} \  \textrm{ o} \ A^c \ \textrm{es contable}\}$.




\end{enumerate}

\end{ejer}

\begin{defi}
Dado un espacio topol\'ogico $(X, \tau)$, el \'algebra de  Borel, o $B$-\'algebra,  asociada a $\tau$,   es la $\sigma$-\'algebra generada por $\tau$, es decir, es la m\'inima $\sigma$-\'algebra que contiene a $\tau$.  Es   la intersecci\'on de todas las $\sigma$-\'algebras que contienen a $\tau$.

 Note que si $A$ es un conjunto no vac\'io entonces $\mathcal P(A)$ es una $B$-\'algebra, la cual esta asociada a la topolog\'ia discreta en $A$. M\'as a\'un, si $\phi$ es una familia de conjuntos entonces 
 $\displaystyle \mathcal P( \bigcup_{A\in \phi} A)$ es una $B$-\'algebra que contiene a $\phi$.
 
 
Sea $\mathcal B$ una $B$-\'algebra y $\phi$ una familia de conjuntos tal que $\phi\subset \mathcal B$. Entonces $\bigcup_{A\in \phi} A \subset E$, donde $E$ es la identidad de $\mathcal B$. Si $\bigcup_{A\in \phi} A = E$ entonces de dice que $\mathcal B$ es irreducible respecto de $\phi$.
\end{defi}


\begin{teo}
Sea $\phi$    una familia  de conjuntos no vac\'ia. Existe una \'unica 
$B$-\'algebra $\mathcal B(\phi)$ que contiene a $\phi $, es irreducible respecto de $\phi$ y que est\'a contenida en cualquier $B$-\'algebra que contiene a  $\phi$. 
\end{teo}
\begin{proof}
Ejercicio.
\end{proof}

$\mathcal B(\phi)$ es llamada la Borel cerradura de $\phi$.

\begin{ejer}
Escriba un ejemplo de cada uno de los siguientes conceptos:  $\sigma$-anillo,   $\delta$-anillo,   $\sigma$-\'algebra y $\delta$-\'algebra.  
\end{ejer}


\section{Medida en el plano}


\noindent
2. Medidas sobre semianillos
3. Medidas exteriores.
4. Conjuntos medibles.
5. La medida exterior generada por una medida.
6. La medida de Lebesgue.
7. La medida de Borel.
\\

\noindent
Comenzaremos por establecer el concepto de medida de Lebesgue en $\mathbb R^2$ debido a su accesibilidad y a que su extensi\'on hacia   $\mathbb R^n$, con $n\geq 3$, se da de manera  natural.


\begin{defi}
Por  $A$ denotaremos a la familia  de intervalos abiertos, cerrado, o semiabiertos contenidos en $\mathbb R$. Considere el semianillo    $$\varphi :=\{ I\times J \mid I,J\in A\}.$$ 

\noindent
Para  cada  rect\'agulo $P$ dado por  $[a,b]\times (c,d)$,
$(a,b)\times [c,d]$, $(a,b)\times (c,d)$, $[a,b]\times [c,d]$ o como $(a,b]\times [c,d)$,
 con $a\leq b$,  defina su medida como $m(P):= (b-a)(d-c)$.
\end{defi}



\begin{pro}
Propiedades de $m$.
\begin{enumerate}
\item  $m(P)\geq 0$ para cada $P\in \varphi$. 
\item  Sean $P, P_1, \dots, P_n \in \varphi$ tales que $\displaystyle P=\bigcup_{i=1}^n P_i$  y  $P_i \cap P_j = \emptyset$, si $i\neq j$. Entonces $m(P)= \sum_{i=1}^n m(P_i)$. 
\end{enumerate}
\end{pro}                                                                                                                                                                                                                                                                                                                                           
\begin{proof}
Ejercicio.
\end{proof}




\begin{defi}La uni\'on de una cantidad finita de rectangulos ajenos a pares es llamada   conjunto elemental.  
\end{defi}


\begin{teo}
La uni\'on, intersecci\'on, diferencia y la diferencia sim\'etrica de conjuntos elementales con tambi\'en conjuntos elementales 
\end{teo}
\begin{proof}
Sea $A = \bigcup P_k$, $B = \bigcup Q_l$, notemos que $A\cap B = \bigcup P_k \cap Q_l$, donde los conjuntos $P_k\cap Q_l$ son rectangulos y ajenos a pares; es decir, $A\cap B$ es elemental. Note que si $P$ y $Q$ son rectangulos entonces $P\setminus Q$ es un conjunto elemental. M\'as a\'un, un rectangulo $P$ menos un conjunto elemental se verifica que es de nuevo un conjunto elemental. De esta manera si $P$ es un rectangulo que contiene a $A$ y a $B$ tenemos que

$$A\cup B = P\setminus [(P\setminus A)\cap(P\setminus B)]$$

el cual por razonamientos anterior es elemental.\\

Adem\'as $$A \setminus B = A\cap (P\setminus B)$$
           $$ A \vartriangle B = (A\cap B)\setminus(A\cap B)$$

    muestran que tambi\'en son conjuntos elementales.
\end{proof}


Lo anterior muestra que la familia de todos los conjuntos elementales en el plano forman un anillo. 




\begin{defi}
Dado $A\subset \mathbb R^2$ un conjunto elemental con $A= \bigcup_{i=1}^n Q_i
$,  donde $\{Q_i  \}_{i=1}^n$ es una familia de rect\'angulos ajenos a  pares. Entonces definimos a medida de $A$ como $\tilde{m}(A)= \sum_{i=0}^n m(Q_i)$. 

\end{defi}

 Si   $A= \bigcup_{j=1}^r P_j
$,  donde $\{P_j  \}_{j=1}^m$ es otra familia de rect\'angulos ajenos a  pares. Entonces 
$$   \bigcup_{i=1}^n Q_i =  \bigcup_{i=1}^n ( \bigcup_{j=1}^r 
  Q_i \cap  P_j ) = \bigcup_{j=1}^r ( \bigcup_{i=1}^n 
   P_j  \cap  Q_i )  =   \bigcup_{j=1}^r P_j  $$
   De las propiedades de la medida de rect\'angulo vemos que
   $$   \sum_{i=1}^n m(Q_i) = \sum_{i=1}^n \sum_{j=1}^r 
 m( Q_i \cap  P_j )= \sum _{j=1}^r m(P_j).  $$
     Entonces la medida de un conjunto elemental dado en la definici\'on anterior esta bien establecida ya que es independiente de la familia de rect\'angulos ajenos a pares. Adem\'as,   tenemos que 
     $\tilde{m}(A) \geq 0$ para cada conjunto elemental $A$ en el plano. 
     
Adem\'as note que si  $A$ y $B$ son conjuntos elementales ajenos entonces   
     $\tilde{m}(A \cup B) = \tilde{m}(A  ) + \tilde{m}(  B)$; es decir, $\tilde{m} $ es aditiva en la familia de conjuntos elementales.  
 


Note que cualquier rect\'angulo $A$ es un conjunto elemental y $m(A) =\tilde{ m
}(A) $.     
     
      \begin{pro}\label{pro1} Sea $A$ un conjunto elemental y sea $(A_n)_{n\in I}
       $  una familia numerable de conjuntos elementales tales  que $A\subset \bigcup_{n\in I} A_n $. Entonces $$ \tilde{m}(A) \leq \sum_{n\in I} \tilde{m}(A_n) .$$
           \end{pro}
\begin{proof}
Dado  $\epsilon>0$ sea $C$ un conjunto elemental cerrado tal que $C\subset A$ y 
$$\tilde{m}(A) \leq \tilde{m} (C ) + \frac{\epsilon}{2}  . $$ 
Para cada $n\in I$ sea $B_n$ un elemental abierto tal que $A_n \subset B_n$ y 
$$ \tilde{m} (B_n) \leq \tilde{m} (A_n) + \frac{ \epsilon}{ 2^{n+1}}.  $$
As\'i que $\displaystyle C\subset \bigcup_{n\in I} B_n   $.  
Del Teorema de  Heine -Borel sabemos que  $C$ es un conjunto  compacto luego 
    $$ C \subset B_{n_1} \cup\cdots \cup B_{n_m} $$
    y  se verifica de manera directa que 
    $$  \tilde{m} (C ) \leq  \tilde{m}( B_{n_1}) +\cdots +  \tilde{m} (B_{n_m})   .$$
Entonces 
\begin{align*}
 \tilde{m}(A) \leq &  \tilde{m} (C ) + \frac{\epsilon}{2} \\
 \leq &  \tilde{m}( B_{n_1}) +\cdots +  \tilde{m} (B_{n_m})  + \frac{\epsilon}{2} \\
  \leq &  \sum_{n\in I}   \tilde{m}( B_{n})  +  \frac{\epsilon}{2} \\
 \leq &\sum_{n\in I} \left( \tilde{m} (A_n) + \frac{ \epsilon}{ 2^{n+1}} \right)   
 +  \frac{\epsilon}{2} \\
 \leq &\sum_{n\in I}   \tilde{m} (A_n)    
 +  \epsilon . \\
\end{align*}
Como $\epsilon>0$ fue arbitrario se concluye 
$$\tilde{m}(A) \leq  \sum_{n\in I}   \tilde{m} (A_n)    
 . $$ 
\end{proof} 



El siguiente razonamiento puede presentarse considerando  cualquier rect\'angulo como unidad pero sin p\'erdida de generalidad supondremos $E=[0,1]\times [0,1]$.


\begin{defi}
Dado $A\subset E$. La media exterior de $A$ se define como 
$$ \mu^* (A ) = \inf\{   \sum_{i\in I} m(Q_i)  \ \mid \  A\subset \bigcup_{i\in I} Q_i , \textrm{ una cubierta numerable de rect\'angulos de } A   \}.$$

Por otra parte, la medida interior de $A$ se define como 
$$ \mu_* (A )= 1-  \mu^* (E\setminus A ).$$

\end{defi}


\begin{teo}
Para cada $A\subset E$ se cumple que
$$ \mu_* (A )\leq   \mu^* ( A ).$$
\end{teo}
\begin{proof}
Supongamos  que existe $A\subset E$ tal que 
$$ \mu_* (A )>  \mu^* ( A ).$$
Entonces 
$$ 1> \mu^* (E\setminus A ) +  \mu^* ( A ).$$
Por  lo tanto, existe  $(Q_i)_{i\in I}$ cubierta numerable  de rect\'angulos de $E\setminus A$  y    $(P_j)_{i\in J}$ cubierta numerable  de rect\'angulos de $ A$  tales que 
$$ 1>  \sum_{i\in I} m(Q_i)) +  \sum_{j\in J} m(P_j).$$
Pero $E =A \cup (E\setminus A ) \subset   ( \bigcup_{i\in I} Q_i ) \sup ( \bigcup_{j\in J} P_j) $ y por la Proposici\'on \ref{pro1} sabemos que 
$$1=m(E) \leq \sum_{i\in I} m(Q_i)  + \sum_{j\in J} m(P_j) ,$$
lo cual es una contradicci\'on. 


Por lo tanto,  
   
$$ \mu_* (A )\leq   \mu^* ( A ), \quad \forall A\subset  E.$$


\end{proof}


\begin{defi}
Diremos que $A\subset E$ es medible, si  $ \mu_* (A )= \mu^* ( A )$ 

\end{defi}




\begin{defi}
Si  $A\subset E$ es medible entonces   $\mu (A):= \mu_* (A )= \mu^* ( A )$ 
 es llamada medida (medida de Lebesgue) de $A$.
\end{defi}




\begin{teo}
\label{teo2}
Dados $A\subset E$ y una familia numerable $(A_i)_{i\in I} $ de subconjuntos de $E$ tales que $A\subset \bigcup_{i\in I} A_i $. Entonces 
$$  \mu^* (A) \leq \sum_{i\in I} \mu^* (  A_i). $$
\end{teo}
\begin{proof}
Dado $\epsilon>0$ para cada $i\in I$ existe una familia numerable de rect\'angulos $(Q_{j,i})_{j\in I_i}$ tales que 
$$ A_i \subset \bigcup _{j\in I_i} Q_{j,i} \quad \textrm{ y} \quad 
\sum_{j\in I_i}m(Q_{j,i}) < \mu^*(A_i) + \frac{\epsilon}{2^i}.
$$
Entonces 
$$A \subset \bigcup_{i\in I} \bigcup_{j\in I_i } Q_{j,i} $$
Por definici\'onde medida exterior sabemos que 
$$\mu^*(A) \leq  \sum_{i\in I} \sum_{j\in I_i }m( Q_{j,i}) < \sum_{i\in I} (  \mu^*(A_i) + \frac{\epsilon}{2^i}) \leq  \sum_{i\in I}   \mu^*(A_i)  + \epsilon   $$
Como $\epsilon >0$ fue arbitrario concluimos que 
$$  \mu^* (A) \leq \sum_{i\in I} \mu^* (  A_i). $$
\end{proof}




\begin{teo}Sea $A\subset E$. Si $A$ es  elemental entonces  es medible 
 y $\mu (A) = \tilde{m}(A)$. 
\end{teo}
\begin{proof}
Sea $(P_i)_{i=1}^n$ una familia de rect\'angulos ajenos a pares tales que $A= \bigcup_{i=1}^n P_i$. Luego $\tilde{m}(A) = \sum_{i=1}^n m(P_i)$. De la definici\'on de medida exterior tendremos que $\mu^*(A)\leq  \tilde{m}(A)$.

Adem\'as, para cualquier   cubierta  numerable de rectangulos $(Q_i)_{i\in }$  de $A $ se cumple que    $\tilde{m}(A) \leq \sum_{i\in I} m(Q_i)$ por lo tanto $\tilde{m}(A   )  \leq    \mu^*(A)$. 
En consecuencia, $\tilde{m}(A   )  =  \mu^*(A)$.  



Por otra parte, $E\setminus A$ es otro elemental y por lo anterior vemos que 
 $ \mu^*(E\setminus A) = \tilde{m}(E\setminus A )  =1-  \tilde{m}(A)$.  Por la definici\'on de medida interior tenemos que  
 $$ \mu_*(  A)= 1 - \mu^* (E\setminus A) = 1- \tilde{m}(E\setminus A ) =1-(1-  \tilde{m}(  A)) = \tilde{m}(A)=   \mu^*(  A). $$
 Por lo tanto $A$ es medible.
 
\end{proof}




\begin{pro}
Para cada $A,B\subset E$ se cumple que 
$$|\mu^*(A)- \mu^*(B)  | \leq \mu^*(A\Delta B).$$
\end{pro}
\begin{proof}
Como $A\subset B\cup (A\Delta B)$ y  $B\subset A\cup (A\Delta B)$. Del Teorema \ref{teo2} tendremos que 
$$\mu^*(A)\leq \mu^* (B)+ \mu^* (A\Delta B), \quad \mu^*(B)\leq \mu^*( A)+ \mu^* (A\Delta B),$$
es decir 
$$- \mu^* (A\Delta B)\leq  \mu^*(B)-  \mu^*( A)\leq  \mu^* (A\Delta B).$$

\end{proof}



\begin{teo}\label{teomedible}
Sea $A\subset E$. Luego  $A$ es medible  si y s\'olo si para cada $\epsilon >0$ existe un conjunto elemental $B\subset E$ tal que 
$$\mu^*(A\Delta B)<\epsilon.$$ 
\end{teo} 
\begin{proof}\ {}

\noindent
Suficiencia. De la proposici\'on anterior y como 
$$(E\setminus A )\Delta (E\setminus B) = A\Delta B$$    tenemos que 
$$  |\mu^*(A)- \mu^*(B)  | , \  |\mu^*(E\setminus  A)- \mu^*(E\setminus B)  | < \epsilon  ;$$
es decir 
\begin{align*}  \mu^*(B)  - \epsilon <  \mu^*(A)  < &  \mu^*(B)  + \epsilon ,\\
  \mu^*(E\setminus B)  - \epsilon < \mu^*(E\setminus  A)   <  &  \mu^*(E\setminus B)  + \epsilon .
  \end{align*}

Como $B$ y $E\setminus B$ son elementales recuerde que 
$$\mu^*(B)  + \mu^*(E\setminus B)= \tilde{m}(B) +\tilde{m}(E\setminus B) =1  . $$
Entonces  sumando las los t\'erminos  de las anteriores   desigualdades tendremos  
\begin{align*}  1 -2 \epsilon \leq   \mu^*(A)  +  \mu^*(E\setminus  A)   \leq &  1 + 2 \epsilon,
  \end{align*}
es decir  $ | \mu^*(A)  +  \mu^*(E\setminus  A) -1 | < 2\epsilon$. Entonces $ \mu^*(A) = 1-  \mu^*(E\setminus  A) = \mu_*(A) $.


\noindent
Necesidad. Dado $\epsilon >0$ existen dos familias numerables de rect\'agulos  $(B_i)_{i\in I}$ y $(C_j)_{j\in J}$ en $E$ tales que 
$A \subset \bigcup B_{i\in I}$ y $E\setminus A \subset \bigcup C_{j\in J} $ con 
 $$\sum_{i\in I} m(B_i) < \mu^* ( A) + \epsilon , \quad \sum_{j\in J} m(C_j) < \mu^* ( E\setminus A) + \epsilon  . $$
 Como $\displaystyle \sum_{i\in I} m(B_i)$ converge existe $N\in \mathbb N $ tal que 
 $$\sum_{n> N} m(B_n) <  \epsilon .$$
Sea $\displaystyle B= \bigcup_{n=1}^N  B_n $ y note que 
$$ A\setminus B \subset  P := \bigcup_{n> N}  B_n     , \quad   B\setminus A  \subset  Q:=  \bigcup_{j\in J}  C_j\cap B    .$$
Adem\'as, 
$A\Delta B \subset P\cup Q$ y 
\begin{align*}    \mu^* (P)  \leq \sum_{n> N} m(B_n) < &  \epsilon, \\
 \sum_{i\in I} m(B_i) + \sum_{j\in J} m(C_j)  < &  \mu^* ( A) +  \mu^* ( E\setminus A) + 2\epsilon = 1 + 2\epsilon, \\
  \sum_{i\in I} m(B_i) + \sum_{j\in J} \tilde{ m} (C_j- B) \geq & 1, 
\end{align*}
Lo \'ultimo se debe a que $\displaystyle (\bigcup_{i\in I} B_i )\cup (\bigcup_{j\in J} ( C_j\setminus B) ) =E$. 
 Por  lo tanto,  
$$
   \sum_{j\in J} m(C_j) -  \sum_{j\in J} \tilde{ m} (C_j- B)   <  2\epsilon,$$
 Debido a que $$  \sum_{j\in J} m(C_j) =  \sum_{j\in J} m(C_j\cap B) +   \sum_{j\in J} \tilde{ m} (C_j- B) , $$
 tenemos 
$$
\mu^*(Q)\leq     \sum_{j\in J} m(C_j\cap  B)   <  2\epsilon. $$
Por \'ultimo note que 
$$\mu^* (A\Delta B) \leq  \mu^* ( P\cup Q)\leq  \mu^* ( P) + \mu^* (   Q) < 3\epsilon.$$  


\end{proof}






\begin{teo} 
La uni\'on e intersecci\'on de una cantidad finita de conjuntos medibles es medible. 
\end{teo}
\begin{proof}
Dados $A_1$,$A_2$ medibles y $\epsilon > 0$, del teorema anterior existen $B_1$, $B_2$ conjuntos elementales tales que 

$$\mu^*(A_1 \Delta B_1) < \frac{\epsilon}{2},\quad \quad \mu^*(A_2 \Delta B_2) < \frac{\epsilon}{2}$$

Luego $(A_1\cup A_2)\Delta (B_1 \cup B_2) \subset (A_1\Delta B_1) \cup (A_2 \Delta B_2)$ implica 

$$\mu^*((A_1\cup A_2)\Delta (B_1 \cup B_2)) \leq \mu^*(A_1\Delta B_1) + \mu^*(A_2 \Delta B_2) < \epsilon$$

Como $B_1\cup B_2$ es un conjunto elemental de nuevo  el teorema anterior nos muestra que $A_1 \cup A_2$ es medible. 

Not\'e que $A_1$ es medible si y s\'olo si $\mu^*(A_1) + \mu^*(E\setminus A_1) = 1$, es decir, si y s\'olo si $E\setminus A_1$ es medible. 
Lo mismo para $A_2$. As\'i que $$A_1 \cap A_2 = E\setminus [(E\setminus A_1)\cup(E\setminus A_2) ]$$
Por lo tanto,  $A_1 \cap A_2$ es medible
\end{proof}


\begin{cor} 
La diferencia y la diferencia sim\'etrica de dos conjuntos medibles  es medible. 
\end{cor}
\begin{proof}
Ejercicio. 
\end{proof}









\begin{teo}
    Si $A_1$ y $A_2$ conjuntos medibles y ajenos  entonces  $\mu(A_1\cup A_2) = \mu(A_1) + \mu(A_2)$
\end{teo}
\begin{proof}
    Sea $A = A_1 \cup A_2$ entonces, por el Teorema \ref{teo2} sabemos que  
 \begin{align}\label{equation111} 
\mu(A) \leq \mu(A_1) + \mu(A_2).
\end{align}

    Como $A_1$ y $A_2$ son conjuntos medibles sabemos que dado $\epsilon> 0$ existen  conjuntos elementales $B_1$ y  $B_2$ tales que 
    $\mu^*(A_1\Delta B_1)<\epsilon$ y $\mu^*(A_2\Delta B_2)<\epsilon$. 

    Sea $B = B_1 \cup B_2$, por otro lado como $A_1$ y $A_2$ son ajenos tenemos que $B_1 \cap B_2 \subset (A_1\Delta B_1)\cup (A_2 \Delta B_2)$ ( en efecto, dado $x \in B_1\cap B_2$, los posibles casos:  a)  $x\in A_1$  y $x\notin  A_2$,  b) $x\notin A_1$  y $x\in  A_2$  y c) 
    $x\notin A_1$  y $x\notin  A_2$ nos ayudan a ver que 
       $x\in (B_1\setminus A_1)\cup (B_2 \setminus A_2)$). Por lo tanto, del Teorema \ref{teo2}
    vemos que  
    \begin{align}\label{equation222} 
\tilde{m}(B_1\cap B_2) = \mu^*(B_1\cap B_2)  \leq \mu^*(A_1\Delta B_1) + \mu^*(A_2\Delta B_2) < 2\epsilon .
\end{align}
Del Teorema \ref{teomedible} tenemos que 
    \begin{align}\label{equation333} 
|\tilde{m}(B_1) - \mu^*(A_1)|\leq \mu^*(A_1 \Delta B_1) < \epsilon, \quad 
|\tilde{m}(B_2) - \mu^*(A_2)| < \epsilon . 
\end{align}

    Como la medida es aditiva en conjuntos elementales sabemos que 
    \begin{align}\label{equation444}
        \tilde{m}(B) =& \tilde{m}(B_1) + \tilde{m}(B_2) - \tilde{m}(B_1\cap B_2)  >  \tilde{m}(B_1) + \tilde{m}(B_2) - 2\epsilon  \nonumber \\
        \geq& \mu^*(A_1) + \mu^*(A_2) - 4\epsilon ,    \end{align}
        en donde fueron usadas las ecuaciones \eqref{equation222} y \eqref{equation333}.
        
        
Se verifica  que $A\Delta B \subset (A_1\Delta B_1) \cup (A_2\Delta B_2) $. Luego,  
     \begin{align}\label{equation555}\mu^*(A\Delta B)\leq \mu^*(A_1\Delta B_1) + \mu^*(A_2\Delta B_2)< 2\epsilon.
     \end{align} 
        Adem\'as, como $B \subset (A\Delta B) \cup A $ tenemos que 
    \begin{align*}
        \mu^*(A) >& \tilde{m}(B) - \mu^*(A\Delta B)\\
                 >& \tilde{m}(B) - 2\epsilon \\
                 \geq& \mu^*(A_1) + \mu^*(A_2) - 6\epsilon \\  
                 \therefore \quad \mu^*(A) \geq&\mu^*(A_1) + \mu^*(A_2),
    \end{align*}
en donde hemos usado \eqref{equation444} y \eqref{equation555}. 
 De la ecuaci\'on anterior y de \eqref{equation111}  se conluye la demostraci\'on.
\end{proof}



\begin{teo}
    La uni\'on e intersecci\'on de una cantidad infinita numerable de conjuntos medibles  son conjuntos medibles
\end{teo}

\begin{proof}
    Considere $(A_n)_{n\in \mathbb{N}}$ una familia de conjuntos medibles y defina
    \begin{align*}
        A'_1 =  A_1,\quad 
        A'_2 =  A_2 \setminus A_1,\quad 
        \dots \quad 
         A'_n =  A_n \setminus\left(\bigcup_{i=1 }^{n - 1}A_i\right), \
         \dots
    \end{align*}
    Como las operaciones usadas arriba son cerradas en la familai de los conjuntos medibles. Tenemos que  $(A'_n)_{n\in \mathbb{N}}$ es una familia de conjuntos medibles ajenos a pares que cumple:
    $$ A = \bigcup_{n \in \mathbb{N}}A_n = \bigcup_{n \in \mathbb{N}}A'_n.$$
    Del teorema anterior y de la definici\'on de medida exterior sabemos que 
    $$\sum_{n = 1}^{N}\mu(A'_n) = \mu\left(\bigcup_{n=1}^{N}A'_n\right)\leq\mu^{*}(A)$$
    para cada $N\in \mathbb{N}$. As\'i que la serie de t\'erminos no negativos $\displaystyle \sum_{n = 1}^{\infty}\mu(A'_n) $ al ser acotada sabemos que  converge. 
    
    Por lo tanto,  dado $\epsilon > 0$ existe $l \in \mathbb{N}$ tal que 
    $$\sum_{n=l + 1}^{\infty} \mu(A'_n) < \epsilon.$$

    Como $\displaystyle  \bigcup_{n= 1}^{l}A'_n$ es medible existe un conjunto elemental $B$ tal que 
    $$\displaystyle  \mu^*\left[\left(\bigcup_{n= 1}^{l}A'_n\right)\Delta B\right]< \epsilon .$$     Adem\'as,
$\displaystyle  A \Delta B\subset \left[\left(\bigcup_{n= 1}^{l}A'_n\right)\Delta B\right] \cup \left(\bigcup_{n = l + 1}^{\infty}A'_n\right)$. 

    Por lo tanto

    \begin{align*}
        \mu^{*}(A\Delta B) \leq&  \mu^*\left[\left(\bigcup_{n= 1}^{l}A'_n\right)\Delta B\right] + \mu^*\left(\bigcup_{n = l + 1}^{\infty}A'_n\right)\\
        <& \epsilon + \sum_{n = l + 1}^{\infty}\mu^*(A'_n)\\
        <& 2 \epsilon
    \end{align*}
    Por tanto $A$ es medible\\

    Por \'ultimo note que

    $$\bigcap_{n=1}^{\infty}A_n =  E\setminus \bigcup_{n = 1}^{\infty}(E\setminus A_n)$$

    As\'i que $\bigcap_{n=1}^{\infty}A_n$ es tambi\'en un conjunto medible
\end{proof}









\begin{teo}
    Si $(A_n)_{n \in \mathbb{N}}$ es una familia de conjuntos medibles que son ajenos a pares  entonces 
    $$\mu\left(\bigcup_{n=1}^{\infty}A_n\right) = \sum_{n=1}^{\infty}\mu(A_n)$$
\end{teo}
\begin{proof}
    Sea $A = \bigcup_{n=1}^{\infty}A_n$. Del Teorema \ref{teo2}
 sabemos que 
$\displaystyle  \mu(A) \leq \sum_{n=1}^{\infty}\mu(A_n)$.  

Por otra parte, como  $\bigcup_{n= 1}^{N}A_n \subset A$ tenemos que 
    $$\sum_{n=1}^{N}\mu(A_n) = \mu \left(\sum_{n=1}^{N}A_n\right) < \mu(A) \quad \forall N\in \mathbb{N}.$$
Entonces $ \displaystyle  \sum_{n=1}^{\infty}\mu(A_n)\leq \mu(A)$.
\end{proof}

\begin{teo} \label{teo235}
   Sea $(A_n)_{n\in \mathbb{N}}$ es una sucesi\'on de conjuntos medibles decreciente, es decir, $A_1\supset A_2\supset A_3\supset\cdots$. Entonces

    $$\lim_{n\to \infty}\mu(A_n)  = \mu\left(\bigcap_{n=1}^{\infty} A_n\right)$$ 
\end{teo}
\begin{proof}
    Suponga que $\bigcap_{n=1}^{\infty}A_n = \emptyset$, luego $A_1 = (A_1\setminus A_2)\cup(A_2\setminus A_3)\cup \cdots$ y en general para cada $n\in \mathbb{N}\quad A_n = \bigcup_{k= n}^{\infty}A_k\setminus A_{k + 1}$. Del teorema anterior tenemos que 
    $$\mu(A_1) = \sum_{k=1}^{\infty}\mu(A_k\setminus A_{k + 1}),$$
es decir, la serie $\sum_{k=1}^{\infty}\mu(A_k\setminus A_{k + 1})$ converge y entonces 
    $$\lim_{n\to \infty}\mu(A_n) = \lim_{n\to \infty}\sum_{k=n}^{\infty}\mu(A_k\setminus A_{k + 1}) = 0 = \mu(\emptyset) = \mu \left(\bigcap_{n=1}^{\infty} A_n\right)$$

    Si $\bigcap_{n=1}^{\infty}A_n \neq \emptyset$ defina $A'_k = A_k \setminus\left(\bigcap_{n=1}^{\infty}A_n\right)$ y note que $\bigcap_{n=1}^{\infty} A'_k = \emptyset$. Por lo anterior 

    $$0 = \lim_{k\to \infty} \mu(A'_k) = \lim_{k\to \infty}\mu\left[A_k\setminus\left(\bigcap_{n=1}^{\infty}\right)\right] = \lim_{k\to \infty}\mu(A_k) - \mu\left(\bigcap_{n=1}^{\infty}A_n\right)$$
\end{proof}



\begin{cor}
    Sea $(A_n)_{n\in \mathbb{N}}$ una sucesi\'on de conjuntos medible creciente, entonces

    $$\lim_{n\to \infty}\mu(A_n) = \mu \left(\bigcup_{n=1}^{\infty}A_n\right)$$
\end{cor}
\begin{proof}
    Consiere $A'_n = E\setminus A_n$ y aplique el teorema anterior a $(A'_n)_{n\in \mathbb{N}}$
\end{proof}





\begin{ob}
La medida    $\mu$ es aditiva, pues si $A_1$ y $A_2$ son conjuntos medibles ajenos entonces $\mu(A_1\cup A_2) = \mu(A_1) + \mu(A_2)$. Adem\'as,   tambi\'en se llama $\sigma$-aditiva, pues si $(A_n)_{n\in \mathbb{N}}$ es una familia  de conjuntos medibles ajenos a pares se cumple $\mu\left(\bigcup_{n=1}^{\infty}A_n\right) = \sum_{n=1}^{\infty}\mu(A_n)$.

Por otra parte,  vemos que los conjuntos abiertos son uniones numerables de rect\'angulos y por lo tanto son medibles. Los conjuntos cerrados tambi\'en son medibles  al ser   complementos  de  conjuntos abiertos  en $E$. 

\end{ob}


\begin{nota}
Defina las siguientes familias de conjuntos:     \begin{align*}
        \mathcal{F}_{m}:&\text{ Familia de rectangulos contenidos en E},\\
        \mathcal{F}_{\tilde{m}}:&\text{ Familia de conjuntos elementales en E},\\
        \mathcal{F}_{\mu}:&\text{ Familia de conjuntos medibles en E}.
    \end{align*}
         
Vemos  que $\mathcal{F}_{m}\subset\mathcal{F}_{\tilde{m}}\subset\mathcal{F}_{\mu}$. Adem\'as, 
 $\mathcal{F}_{m}$ es un semianillo y   $ \mathcal{F}_{\tilde{m}}$ es un anillo. Por\'ultimo, la familia  
 $ \mathcal{F}_{\mu}$ es una $\sigma$-\'algebra.



\end{nota}







\begin{ob}
En los c\'alculos anteriores se consideran s\'olo  subconjuntos de E. Pero podemos extender los resultados a todo el plano en dos formas 
    \begin{enumerate}
        \item Denote $E_{mn} := [m,m+1]\times [n,n+1]$ para $m,n\in \mathbb{Z}$ de esta manera diremos que $A\subset \mathbb{R}^{2}$ es medible si y solo si   para cada $m,n\in \mathbb{Z}$ el conjunto $A_{mn} = A\cap E_{mn}$ es medible. Adem\'as, si la serie $\displaystyle \sum \mu(A_{mn})$ converge, diremos que su medida es $\displaystyle \mu(A) = \sum \sum (A_{mn})$. Por otra parte,   si $\displaystyle \sum (A_{mn})$ diverge diremos que el conjunto medible $A$ tiene medida infinita.

        \item Considere $E_n := [-n,n]\times[-n,n]$, para cada $n\in \mathbb{N}$. Diremos que  $A\subset\mathbb{R}^2$ es medible si y s\'olo si $A_n = A\cap E_n$ es medible para cada $n\in \mathbb{N}$. Si la sucesi\'on $(\mu(A_n))_{n\in\mathbb N} $ converge la medida de $A$ es $\mu(A) = \lim_{n\to \infty} \mu(A_n)$ en  caso contrario  diremos que $A$ tiene medida infinita.

    \end{enumerate}
La medida $\mu$ extendida de cualquiera de las dos formas anteriores  continua siendo aditiva  y $\sigma$-aditiva. M\'as a\'un,   se trata de la misma medida $\mu$ extendida.
    \end{ob}



\begin{ejer}  \ {} \begin{enumerate}
 
\item Construya la medida de Lebesgue en $\mathbb{R}$ empleando intervalos en lugar de rect\'angulos.



\item Muestre que si $A$ es medible en $\mathbb{R}^2$ entonces para todo $a\in \mathbb{R}^2$, $A + a = \{x + a | x\in A\}$ es tambi\'en medible y 
$\mu (A)= \mu (A+a)$.  

Sugerencia: Note que  si $P\subset \mathbb R^2$ es un rect\'angulo entonces  $P+a$ es otro rect\'angulo y $m(P)= m(P+a)$.




\item Sea $A\subset E$ tal que $\mu^*(A)<+\infty$. Muestre que si existe $B\subset A$ conjunto medible con $\mu(B) = \mu^*(A)$, entonces $A$ es medible ?`Qu\'e sucede si $\mu^*(A) = +\infty ?$

Sugerencia: Note que $E\setminus A \subset E\setminus B  $ y $\mu(B) = \mu^*(A)$  implican 
$$1 \leq \mu^{*}(E\setminus A )  + \mu^{*}(  A )  \leq \mu^{*}(E\setminus B )  + \mu ( B ) = \mu^* (E\setminus B )  + \mu^* ( B )  = 1.$$

\item Muestre que si $A\subset E=[0,1]\times [0,1]$ es un conjunto cuya frontera es un conjunto medible de medida cero es medible.

Sugerencia: Note que $\overline{A}$ y  $A^{\circ}$  son conjuntos medibles. Adem\'as, $A^{\circ} \subset A \subset \overline{A}$, $ \overline{A} = \partial A \cup A^{\circ}$  implican   
$\mu^{*} (A^{\circ}) =\mu^{*} (   A ) = \mu^{*} (  \overline{A})$. Al final use el ejercicio anterior.

\item Sea $A\subset \mathbb{R}^2$ y $\alpha\in \mathbb{R}$. Muestre que $\mu^*(\alpha A) =  \alpha^2\mu^*(A)$ donde $\alpha A = \{\alpha x | x\in A\}$ y que si $A$ es medible entonces $\alpha A$ es tambi\'en medible.


\item Muestre que todo conjunto finito en el rect\'angulo $E=[0,1]\times [0,1]$   es medible y tiene medida cero. 


Sugerencia: Si $A=\{(x,y) \}$ entonces para cada $\epsilon >0 $ tenemos $A\subset (x-\frac{\sqrt{\epsilon}}{2}, x+\frac{\sqrt{\epsilon}}{2}  ) \times  (y-\frac{\sqrt{\epsilon}}{2}, y+\frac{\sqrt{\epsilon}}{2}  )$. Luego $\mu^* (A) <\epsilon $  y  $1-\epsilon < \mu^*(E\setminus A)  <1$

\item Pruebe que $\mathbb{Q}\subset\mathbb{R}$ es medible y tiene medida cero. 
  

  Sugerencia:   $\mathbb Q$ es un conjunto infinito numerable. Entonces denotemos cada n\'umero racional por  $q_n$, donde $n\in \mathbb N$. Luego 
  $ \mathbb Q=\cup_{n\in \mathbb N}\{q_n\} $ y para cada $\epsilon >0$ se cumple  
  $$ \mathbb Q \subset \bigcup_{n\in \mathbb N} (q_n- \frac{\epsilon}{2^n }, q_n+ \frac{\epsilon}{2^n }). $$  


\item Considere $F_0 = [0,1]$, $F_1 = [0,1/3]\cup [2/3,1]$, $F_2 = [0,1/9]\cup[2/9,3/9]\cup[6/9,7/9]\cup[8/9,1]$. As\'i sucesivamente, el siguiente conjunto se forma de dividir cada intervalo en tres y solo tomamos en cuenta las dos partes extremas. En general $F_0\supset F_1\supset F_2\supset \cdots$. El conjunto de cantor se define como $\bigcap_{n=0}^{\infty}F_n$, muestre que el conjunto de cantor es medible y tiene medida cero.

Sugerencia: Teorema \ref{teo235}.


\item Muestre que la cardinalidad de la familia de todos los conjuntos medibles contenidos en $[0,1]$ es mayor o igual  que la del continuo.

Sugerencia: Para cada $x\in [0,1]$ defina $f(x) =[0,x]$ el cual es un conjunto medible. As\'i que la funci\'on $f$ es inyectiva. Luego la cardinalidad de familia de todos los conjuntos medibles contenidos en $[0,1]$ es mayor o igual  que la del continuo.


\item Muestre que todo conjunto medible de medida positiva en el intervalo $[0,1]$ contiene un par de puntos cuya distancia es un n\'umero racional

\begin{proof}

\noindent
Contradicci\'on.  Suponemos que $A\subset [0,1]$ tiene medida positiva  y que ning\'un par de elementos de  $A$ tienen distancia un n\'umero racional. 

Veamos dos casos: 

\begin{enumerate}
\item  Suponga que $\sup A<1$. Por la propiedad Arquimediana existe $N\in \mathbb N$ tal que  $\dfrac{1}{N}< 1- \sup A  $ luego  $ \dfrac{1}{n}< 1- \sup A   $ para  cada $n\geq N$. Entonces,  $\sup A +  \dfrac{1}{n} < 1 $ para cada $n\geq N$. Note que cada elemento de $ A +  \dfrac{1}{n}  $  con   $n\geq N$ es de la forma $a+\dfrac{1}{n}$ con $a\in A$ luego   $0\leq a+ \dfrac{1}{n} \leq \sup A +  \dfrac{1}{n} < 1 $; es decir, 
  $a+ \dfrac{1}{n} \in [0,1]$. Entonces  $ A +  \dfrac{1}{n} \subset [0,1] $  para cada    $n\geq N$.
  
  
 \noindent
 Note que si $x \in  ( A +  \dfrac{1}{n} ) \cap A$,  con $n\geq N$, entonces existen $a,b\in A$ tales que 
  $   x = a + \dfrac{1}{n} = b $. Luego $|a-b | =\dfrac{1}{n}$. Lo cual no puede ocurrir. As\'i que $(  A +  \dfrac{1}{n} ) \cap A = \emptyset$ para cada $n\geq N$. De forma similar se verifica que  si $n,m\in \mathbb N$ son diferentes entonces $(  A +  \dfrac{1}{n} ) \cap ( A +  \dfrac{1}{n} ) = \emptyset$.
  
  \noindent
  Por lo anterior $\bigcup _{n\geq N}(  A +  \dfrac{1}{n} )\subset [0,1]$ y $$\infty =   \sum_{n\geq N} \mu (A ) =   \sum_{n\geq N} \mu (A+\dfrac{1}{n}) 
  = \mu( \ \bigcup _{n\geq N}(  A +  \dfrac{1}{n} ) \  ) 
  \leq \mu([0,1])=1 . $$
 Lo cual es una contradicci\'on.
 
 
 \item En caso de que $\sup A=1$. Por la propiedad Arquimediana existe  $N\in \mathbb N$ tal que $\dfrac{1}{N}< \mu(A) $.  Note que el conjunto $B= A\setminus [1-\dfrac{1}{N}, 1]$ tiene medida positiva      y que ning\'un par de elementos de  $B$ tienen distancia un n\'umero racional. Ya que 
 $\mu (B) \geq  \mu ( A ) -\dfrac{1}{N} > 0$ y $B\subset A$. Adem\'as, $\sup B   \leq 1-\dfrac{1}{N}< 1$. Por el caso anterior llegaremos a una contradicci\'on.
 \end{enumerate}
  
  \end{proof}
  
\item Sea $C = \{z\in \mathbb{C} : |z| = 1\}$. Sea $\alpha\in \mathbb I$, defina $z_1,z_2\in \mathbb{C}$, $z_1\sim z_2$ si y solo si $z_2$ se obtiene v\'ia una rotaci\'on un \'angulo $n\pi \alpha$  ($n\in \mathbb{Z}$) de $z_1$. Verifique que $\sim$ es una relaci\'on de equivalencia y muestre que $\Phi_0$ formado eligiendo un punto de cada clase, no es medible.

\begin{proof}
Contradicci\'on.   Suponga que $\Phi_0$  es medible. Para $n,m\in\mathbb Z$ diferentes si $z \in  e^{i n\pi \alpha } \Phi_0 \cap  e^{i m\pi \alpha } \Phi_0$
 entonces existen $z_1, z_2 \in \Phi_0$ tales que $$z=e^{i n\pi \alpha } z_1 = e^{i m\pi \alpha } z_2$$
$$z_1 = e^{i(- n+m)\pi \alpha } z_2,$$
entonces  $z_1\sim z_2$ por lo que $z_1$ y $z_2$ est\'an en la misma clase. Debido a la creaci\'on de $\Phi_0$ tendremos que $z_1=z_2$.  Entonces 
 $1 = e^{i(- n+m)\pi \alpha }$, lo cual implica $(- n+m)\pi \alpha = 2k\pi$ para alg\'un $k\in \mathbb Z$. As\'i que 
 $  \alpha =\dfrac{ 2k}{ - n+m}\in \mathbb Q$. Pero $\alpha\in \mathbb I$, as\'i que  
 $$ e^{i n\pi \alpha } \Phi_0 \cap  e^{i m\pi \alpha } \Phi_0 =\emptyset.$$

Dado $z\in C$ entonces existe $w\in \Phi_0$ tal que $w \in [z]$ luego $z\sim w$ y existe $n\in \mathbb N $ tal que $z = e^{i n\pi \alpha} w $. Entonces $z \in  e^{i n\pi \alpha}  \Phi_0 $. As\'i que $C  =   \bigcup_{n\in\mathbb Z}  e^{i n\pi \alpha}  \Phi_0 $.
 
Si  $ \Phi_0 $ es medible  y $  \mu (  \Phi_0) > 0 $ entonces 

$$\mu (C) = \sum_{n\in \mathbb Z} \mu ( e^{i n\pi \alpha}  \Phi_0) =  \sum_{n\in \mathbb Z} \mu (  \Phi_0)  =\infty .$$


Si  $ \Phi_0 $ es medible  y $  \mu (  \Phi_0) = 0 $  entonces 

$$\mu (C) = \sum_{n\in \mathbb Z} \mu ( e^{i n\pi \alpha}  \Phi_0) =  \sum_{n\in \mathbb Z} \mu (  \Phi_0)  =0.$$

Lo cual es falso. 

\end{proof}


\end{enumerate}
\end{ejer}


 

\begin{ob} Un conjunto $B\subset \mathbb R^2$, considere  $\mu^*$ la medida exterior,  se dice de Caratheodory medible si $\mu^*(C) = \mu^*(C\cap B) + \mu^*(C\setminus B)$ para todo $C\subset \mathbb R^2$.
    Adem\'as, si $B \subset E$    entonces $B$ es Caratheodory medible si y s\'olo si es Lebesgue medible en $E$. Por ejemplo, si $B$ es Caratheodory medible entonces $C=E$ tenemos 
    $$\mu^*(E) = \mu^*(E\cap B) + \mu^*(E\setminus B),$$
    $$1 = \mu^*(B) + \mu^*(E\setminus B).$$
      
\end{ob}





\section{Teor\'ia de la medida general}

\begin{defi}
    Dado un semianillo de conjuntos $\varphi_ m $, una funci\'on $m:\varphi_\mu \to \mathbb{R}$ es llamada medida, si 
    \begin{enumerate}[label=\alph*)]
        \item   $m(A)\geq 0,\quad\forall A\in \varphi_\mu$.

        \item $m$ es aditiva.
    \end{enumerate}
\end{defi}

\begin{ob} Note que $m(\emptyset) = 0$. Ya que 
  $\emptyset = \emptyset\cup \emptyset$ y $\mu$ es aditiva tendremos que  $m(\emptyset) = m(\emptyset) + m(\emptyset)$. Por lo tanto $m(\emptyset) = 0$.
\end{ob}

\begin{teo}
    Sea $m$ una medida en el semianillo $\varphi_m$. Si $A, A_1,\cdots, A_n \in \varphi_m$ tales que  $A_1,\cdots,A_n$ son ajenos a pares y son subconjuntos de $A$ entonces 

    $$\sum_{k=1}^{n}m(A_k) \leq m(A) $$
\end{teo}
\begin{proof}
    Sabemos que existe una expansi\'on finita de $A$ de la forma $A = \bigcup_{k=1}^{s}A_k$ con $s\geq n$. Luego

    $$\sum_{k=1}^{n}m(A_k) \leq \sum_{k=1}^{s}m(A_k) = m(A).$$
\end{proof}

\begin{teo}
    Sea $m$ una medida sobre un semianillo $\varphi_m$, sean $A, A_1,\cdots, A_n \in \varphi_m$, tales que $A \subset \bigcup_{k=1}^{n}A_k$. Entonces 
    $$m(A) \leq \sum_{k=1}^{n}m(A_k) . $$
\end{teo}
\begin{proof}
    Existe una familia finita $\{B_1, \cdots, B_n\}$ de elementos de $\varphi_m$, ajenos a pares tales que $ \displaystyle A=\bigcup_{s\in M_0}B_s$ y $\displaystyle A_k = \bigcup_{s\in M_k}B_s$ para $k = 1,\cdots,n$.   Para esto se emplea  la propiedad de expansiones finitas  del semianillo.\\

    Por \'ultimo, 

    $$m(A) = \sum_{s\in M_0}m(B_s) \leq \sum_{k=1}^{n}\sum_{s\in M_k}m(B_s)\leq \sum_{k=1}^{n}m(A_k).$$
\end{proof}
\begin{cor}
    Si $A,A'\in \varphi_m$ y $A\subset A'$, entonces $m(A)\leq m(A')$.
\end{cor}
\begin{defi}
    La medida $\mu$ es llamada extensi\'on de la medida $m$ si $\varphi_m \subset \varphi_\mu$ y $m(A)= \mu(A)$ para toda $A\in \varphi_m$.
\end{defi}

\begin{teo}
    Toda medida $m$ en un semianillo $\varphi_m$ tiene una \'unica extensi\'on $\tilde{m}$ definida en $\mathcal{R}(\varphi_m)$, el  anillo generado por $\varphi_m$.
\end{teo}
\begin{proof}
    Recuerde que $\mathcal{R}(\varphi_m )$ esta formado por uniones finitas de elementos de $\varphi_m$, ajenos a pares. De esta manera 

    $$\tilde{m}(A) = \tilde{m}\left(\bigcup_{k=1}^{n}B_k\right) = \sum_{k=1}^{n}\tilde{m}(B_k) = \sum_{k=1}^{n}m(B_k).$$

    Note que est\'a bien definida.  Por \'ultimo si $\mu $ es una extensi\'on de $m$ en $\mathcal{R}(\varphi_m)$ entonces 

    $$\mu (A) = \sum_{k = 1}^n \mu (B_k) = \sum_{k = 1}^n m(B_k) = \tilde{m}(A).$$

    Es decir, $\mu  = m$.
\end{proof}

\begin{defi}
    Una medida $m$ en un semianillo $\varphi_m$ se dice $\sigma-$aditiva, si para cualquier subfamilia de $\varphi_m$ infinita numerables de ajenos a pares $\{A_k\}_{k\in \mathbb{N}}$ tales que $\bigcup_{k=1}^{\infty} A_k \in \varphi_m$ se cumple
    $$m\left(\bigcup_{k=1}^{\infty} A_k\right) = \sum_{k=1}^\infty m(A_k).$$
\end{defi}
\begin{teo}
    Si $m$ es una medida $\sigma$-aditvia en $\varphi_m$, entonces su extensi\'on $\tilde{m}$ a $\mathcal{R}(\varphi_m)$ tambi\'en es $\sigma$-aditiva
\end{teo}
\begin{proof}
      Sea $\{B_k\}_{k\in \mathbb{N}}$ una subfamilia de $\mathcal{R}(\varphi_m)$ de conjuntos ajenos a pares tales     $A = \bigcup_{k=1}^{\infty}B_k\in \mathcal{R}(\varphi_m)$. Como $A,B_k\in \mathcal{R}(\varphi_m)$ para cada $k\in \mathbb{N}$ existen $A_j,B_{n,k}$ en $\varphi_m$ tales que $A = \bigcup A_j$ y $B_{k} = \bigcup B_{n,k}$ (uniones finitas de conjuntos ajenos a pares). Note que los conjuntos de la forma $A_j \cap B_{n,k}$ son ajenos a pares. M\'as a\'un 

      $$A_j = \bigcup_k \bigcup_n A_j \cap B_{n,k}\quad\quad B_{n,k} = \bigcup_j A_j \cap B_{n,k} $$

      As\'i que 

      \begin{align*}
           \tilde{m}(A) =& \sum_j m(A_j) = \sum_j \sum_k \sum_n m(A_j \cap B_{n,k}) \\
           = &\sum_k \sum_n \sum_j m(A_j \cap B_{n,k}) \\
           = & \sum_k\sum_n m(B_{n,k}) = \sum_k \tilde{m}(B_k).
      \end{align*}
    
\end{proof}

\begin{cor}
    Sea $m$ una medida $\sigma$-aditiva en el semianillo $\varphi_m$. Considere $A, A_1,\cdots, A_k,\cdots,$ elementos de $\varphi_m$ tales que $A_1,\cdots, A_k,\cdots,$ son subconjuntos de $A$ y son ajenos a pares. Entonces
     $$ \sum_{k=1}^{\infty} m(A_k) \leq m(A).$$
\end{cor}

\begin{proof}Se sigue de 
    $\displaystyle \sum_{k=1}^{N}m(A_k) \leq m(A)$ para cada $ N\in \mathbb{N}$ y luego  hacemos $N\to \infty$.
\end{proof}
\begin{cor}
    Sea $m$ una medida $\sigma-$aditiva en el semianillo $\varphi_m$. Sean $A, A_1,\cdots, A_k,\cdots,$ elementos de $\varphi_m$ tales que $A\subset \bigcup_{k=1}^{\infty} A_k$. Entonces 
    $$m(A) \leq \sum_{k=1}^{\infty} m(A_k)$$
\end{cor}
\begin{cor}
Sea  $\tilde{m}$ a la extensi\'on de $m$ en $\mathcal{R}(\varphi_m)$. Los dos corolario anteriores son v\'alidos en $\mathcal{R}(\varphi_m)$. 
\end{cor}
Sugerencia para el \'ultimo corolario: Sea  $B_n = (A\cap A_n)\setminus\bigcup_{k=1}^{n-1} A_k$. De esta manera $A = \bigcup_{n=1}^{\infty} B_n$, $A_n \supset B_n$, la familia $(B_n)_{n\in \mathbb{N}}$ esta contenida en $\mathcal{R}(\varphi_m)$ y es de conjuntos ajenos a pares. Luego
    $$\mu(A) = \sum_{n=1}^{\infty}\mu(B_n) \leq \sum_{n=1}^{\infty}A_n .$$ 
 \begin{defi}
    Sea $m$ una medida $\sigma-$aditiva en un semianillo $\varphi_m$ con unidad $E$. Dado $A\subset E$ defina
    \begin{enumerate}
        \item Medida exterior de $A$ denotada por $\mu^*(A)$ definida como
         $$\mu^*(A) = \inf \{ \sum_{k\in I\subset\mathbb N} m(B_k) \ \mid \   A\subset \bigcup_{k\in I }     B_k, \ \  B_k\in \varphi_m,  \  \ k\in I  \}.$$ 
         \item Medida interior de $A$ dada por $\mu_*(A) = m(E) - \mu^*(E\setminus A)$.

         \item Diremos que $A\subset E$ es Lebesgue medible si y solo si $\mu^*(A) + \mu_*(A)=m(E)$. La medida de Lebesgue de $A$ es $\mu(A) := \mu^*(A) = \mu_*(A)$.
    \end{enumerate}
\end{defi}

 

\begin{ob}
    
     
Las siguientes propiedades tienen demostraciones  profundamente   an\'alogas a sus hom\'ologas dadas en la secci\'on anterior. Por tal motivo solo se omiten las demostraciones.
     \begin{enumerate}
     
\item Para cada $A\subset E$ se verifica que  $\mu_*(A)\leq \mu^*(A)$. Adem\'as, $A\subset E$ es Lebesgue medible si y s\'olo si $\mu^*(A) + \mu^*(E\setminus A) = m(E)$, o equivalentemente,  $E\setminus A$  es  Lebesgue medible. 
        \item Dados $A\subset E$ y $(A_n)$ una familia a lo sumo numerable de subconjuntos de $E$ tales que $A\subset \bigcup_n A_n$, entonces
         $$\mu^*(A) \leq \sum_n \mu^*(A_n).$$

         \item Todo elemento $A\in \mathcal{R}(\varphi_m)$ es medible y $\mu(A) = \tilde{m}(A)$, extensi\'on de $m$ a $\mathcal{R}(\varphi_m)$.

         \item $A\subset E$ es medible si y solo si para cada $\epsilon>0$ existe $B\in \mathcal{R}(\varphi_m)$ tal que $\mu(A\Delta B) < \epsilon$.

         \item La familia $\varphi_\mu$ de todos los subconjuntos medibles de $E$  es un \'algebra con unidad $E$.

         \item $\varphi_\mu$ es un \'algebra de Borel.


    \end{enumerate}
\end{ob}


En algunas ocasiones  un conjunto $A\subset E$ se dice medible   si para todo $ \epsilon > 0$ existe $B\in \mathcal{R}(\varphi_m)$ tal que $\mu^*(A\Delta B)<\epsilon$.

\begin{ejer} 
   {}\
   \begin{enumerate}
       \item Sea $m$ una medida $\sigma-$aditiva en un semianillo $\varphi_m$  con unidad $E$. Sea $\mu$ la extensi\'on de Lebesgue de $m$ y sea $\tilde{\mu}$ cualquier medida $\sigma-$aditiva que es extensi\'on de $m$. Muestre que  $\mu_*(A)\leq \tilde{\mu}(A)\leq \mu^*(A) $ para cada $A\subset E$   en que este definida $\tilde{\mu}$. En particular, vemos que si  $A$ es medible y $\tilde{\mu}(A)$ est\'a definida entonces   $\mu(A) = \tilde{\mu}(A)$. 
\begin{proof}
Para cualquier $(P_i)_{i\in I \subset \mathbb N} $  
familia numerable de elementos de  $\varphi_m$ tal que 
$A \subset \cup_{i\in I} P_i  $ entonces $$\tilde {\mu} (A) \leq \tilde {\mu}( \cup_{i\in I} P_i )\leq \leq \sum_{i\in I} \tilde {\mu}(  P_i ) ;$$
De la definici\'on de $\mu^*$ tenemos que $\tilde {\mu} (A)\leq \mu^*(A)$  para cada $A\subset E$   en que este definida $\tilde{\mu}$.

\noindent
Adem\'as,  $ \tilde {\mu} (E\setminus A)\leq \mu^*(E\setminus A) $, luego  
$$ \mu_*(A) =  m  (E)-  \mu^*(E\setminus A) \leq m  (E) - \tilde {\mu} (E\setminus A) = \tilde{m}  (E) - \left( \tilde {\mu} (E) -  \tilde {\mu}  (A)\right) = 
\tilde {\mu}  (A). $$
Entonces 
$\mu_*(A)\leq \tilde{\mu}(A)\leq \mu^*(A) $ para cada $A\subset E$   en que este definida $\tilde{\mu}$.


\end{proof}



       \item Sea $\tilde{m}$ la extensi\'on de $m$ a $\mathcal{R}(\varphi_m)$. Muestre que la medida exterior de $A\subset E$ est\'a dada como 
       $$\mu^*(A) = \displaystyle \inf \{  \sum_{k\in I \subset \mathbb N} \tilde{m}(B_k) \ \mid \   A\subset\bigcup_{k\in I  } B_k , \ \mid \ B_k\in  \mathcal{R}(\varphi_m), \ \forall k\in I \}.$$
  \begin{proof}Sea 
  \begin{align*}
T_A= &       \{  \sum_{i\in I \subset \mathbb N}  {m}(P_i) \ \mid \   A\subset\bigcup_{i\in I  } P_i , \ \mid \ P_i\in  \varphi_m , \ \forall i\in I \}  , \\
S_A =  &      \{  \sum_{k\in I \subset \mathbb N} \tilde{m}(B_k) \ \mid \   A\subset\bigcup_{k\in I  } B_k , \ \mid \ B_k\in  \mathcal{R}(\varphi_m), \ \forall k\in I \}.
  \end{align*} 
  Como    
   $ P_i\in  \varphi_m \subset  \mathcal{R}(\varphi_m) $,  para cada  $ i\in I$ entonces 
   $\displaystyle \sum_{i\in I \subset \mathbb N}  {m}(P_i)\in S_A$. Por lo tanto, $T_A\subset S_A$ 
   
   Rec\'iprocamente, si $ B_k\in  \mathcal{R}(\varphi_m)$,  para cada $ k\in I$.
   entonces existen $P_1^k, \dots, P_{j_k}^k \in \varphi_m$ ajenos a pares tales que  
    $\displaystyle B_k = \bigcup_{\ell =1}^{j_k}   P_{\ell}^k$. Entonces 
      $$\tilde{m}(B_k)  =  \sum_{\ell =1}^{j_k}  m( P_{\ell}^k)$$
       y 
     $$  \sum_{k\in I \subset \mathbb N} \tilde{m}(B_k) =  \sum_{k\in I \subset \mathbb N} \sum_{\ell =1}^{j_k}  m( P_{\ell}^k)  =  \sum_{n\in L \subset \mathbb N}   m( C_n) , $$
   donde $C_n\in \varphi_m$ para cada $n\in L$. Entonces 
   $$\sum_{k\in I \subset \mathbb N} \tilde{m}(B_k) \in T_A.$$ 
    As\'i que $S_A\subset T_A$.
  \end{proof}
  
  
   \end{enumerate}
\end{ejer}


\begin{ob}
Una medida   en una $\sigma$-\'algebra   se dice completa completa  si todo  subconjunto no vac\'io de cada conjunto  medible de medida cero es medible (y tiene  medida cero). 
\\
\noindent
La medida de Lebesgue    es completa  completa. Ya que   si todo subconjunto no vac\'io de un conjunto medible de medida cero es medible y de medida cero.  

\end{ob}


\chapter{Integraci\'on}

\section{Funciones medibles y funciones simples}

\begin{defi}
    Dada $f:X \to Y$ y dadas $\varphi,\varphi'$ familias de subconjuntos de $X$ y de $Y$, respectivamente. Diremos que $f$ es $(\varphi,\varphi')-$medible si $A\in \varphi'$ implica $f^{-1}(A)\in \varphi$
\end{defi}
\begin{ob}
    Si $(X,T_x)$ y $(Y, T_y)$ son espacios topol\'ogicos, entonces $f:X\to Y$ es continua si y s\'olo si es $(T_x,T_y)-$medible. En particular, si $X=Y=\mathbb{R}$ con $T_x= T_y$ siendo el \'algebra  de  Borel $\mathcal{B}^1$ inducida por la topolog\'ia usual en $\mathbb R$   entonces $f$ se dice Borel-medible o B-medible. Recuerde  que un \'algebra de Borel es la $\sigma-$\'algebra generada por una topolog\'ia.  
\end{ob}

\begin{defi}
    Sea  $\mu$  una medida  en un \'algebra de Borel $\varphi_\mu$ de subconjuntos de $X$, con $X$ la unidad de $\varphi_\mu$. Considere $f:X\to\mathbb{R}$. Entonces $f$ se dice $\mu-$medible en $X$ si $A\in \mathcal{B}^1$ implica $f^{-1}(A)\in \varphi_\mu$, o equivalentemente $f^{-1}(\mathcal{B}^1)\subset \varphi_\mu$. 
\end{defi}
En nuestro caso supondremos $X=\mathbb R^n$ con $n\in \mathbb N$.
 
\begin{teo}
    La funci\'on $f$ es $\mu$-medible si y s\'olo si $\{x\in X \ | \  f(x)<c\}$ es $\mu-$medible para cada real $c$.
\end{teo}
\begin{proof}
    Si $f$ es $\mu$-medible como $\{x\in X \ | \ f(x)<c\} = f^{-1}((-\infty,c))$ y $(-\infty,c)\in \mathcal{B}^1$ entonces $\{x\in X \ | \  f(x)<c\}$ es $\mu$-medible. Note que $\mathcal{B}^1$ es la m\'inima $\sigma$-\'algebra que contiene a los intervalos cerrados en $\mathbb{R}$ y usando las siguientes identidades
    $$(-\infty,c) = \bigcup_{n\in \mathbb{N}} \left[-n, c - \frac{1}{n}\right]$$

    $$[a,b] = (-\infty,a)^c\cap \left(\bigcap_{n\in \mathbb{N}}\left(-\infty, b + \frac{1}{n}\right)\right)$$

    Notamos que $\mathcal{B}^1$ es tambi\'en la m\'inima $\sigma$-\'algebra que contiene a $(-\infty,c)$ para cada $c\in \mathbb{R}$. Por \'ultimo empleando las siguientes identidades:

    $$f^{-1}\left(\bigcup_{\alpha}A_\alpha\right) = \bigcup_\alpha f^{-1}(A_\alpha), \quad f^{-1}\left(\bigcap_{\alpha}A_\alpha\right) = \bigcap_\alpha f^{-1}(A_\alpha), \quad  f^{-1}(A)^c = f^{-1}(A^c)$$

    Se verifica que $f^{-1}(\mathcal{B}^1)\subset\varphi_\mu$
\end{proof}
\begin{teo}\label{convergmed}
    Sea $(f_n)_{n\in \mathbb{N}}$ una sucesi\'on de funciones medibles en $X$ y sea $f$ una funci\'on en $X$ tal que $\lim f_n(x) = f(x)$ para toda $x\in X$. Entonces $f$ es medible. 
\end{teo} 
\begin{proof}
    Se deduce de la identidad 
    $$\{x \in X \Big{|} f(x)<c\} = \bigcup_k\bigcup_n \bigcap_{m\geq n}\left\{ x \in X \  \Big{|} f_m(x)< c - \frac{1}{k}\right\}.$$
    Verifiquemos dicha identidad. Si  $x\in X$ es tal que $f(x)< c$ entonces  existen $k,N\in \mathbb{N}$ tales que
 $$f(x)< c -\frac{2}{k}, \quad f_n(x) - f(x)< \frac{1}{k}, \quad \forall n \geq N.$$ Luego $f_n(x)< c - \frac{1}{k}$.  

\noindent
Rec\'iprocamente, si 
    $$x\in \bigcup_k\bigcup_n \bigcap_{m\geq n}\left\{ x\quad \Big{|} f_m(x)< c - \frac{1}{k}\right\}.$$
Entonces existen  $k,N\in \mathbb{N}$ tal que $f_n(x)< c- 1/k$ para toda $n\geq N$ y haciendo $n\to \infty $ obtenemos $f(x)\leq  c - \dfrac{1}{k}<c$.
\end{proof}
\begin{teo}
    La composici\'on de una funci\'on $\mu$-medible $\psi:X\to \mathbb R$ con una funci\'on $\mathcal{B}$-medible $\varphi :\mathbb R \to \mathbb R$ es $\mu$-medible.
\end{teo}
\begin{proof}
    Sea $f = \varphi \circ\psi$, luego note que   $f^{-1}(A) = \psi^{-1}(\varphi^{-1}(A))$  para todo $A\subset \mathbb{R}$. 
\end{proof}
\begin{cor}
    La composici\'on de una funci\'on medible con una continua es una funci\'on medible.
\end{cor}
\begin{proof}
 Se debe a que una funci\'on continua es una funci\'on $\mathcal{B}$-medible.
\end{proof}
\begin{defi}
    Una funci\'on simple es una funci\'on medible con   una cantidad   numerable de valores diferentes.
\end{defi}





\begin{teo}
    Si $f$ toma los valores $(y_n)_{n\in I \subset \mathbb{N}}$, entonces $f$ es simple si y solo si $A_n = \{x \in X \  | \  f(x) = y_n\}$ es medible para cada $n\in I$. 
\end{teo}

\begin{proof}
    Necesidad. 
     Como $\{y_n\}\in \mathcal{B}^1$ entonces $A_n = f^{-1}(\{y_n\})\in \varphi_\mu$ para cada $n\in I$. 
     
     \noindent 
     Suficiencia. Para cada $c\in \mathbb{R}$ tenemos

    $$f^{-1}((-\infty,c)) = \bigcup_{y_n<c}\{x|f(x)=y_n\}$$
  es un conjunto medible. As\'i que $f$ es una funci\'on medible.
\end{proof}
\begin{teo}
    Una funci\'on $f$ es $\mu$-medible si y solo si existe una sucesi\'on de funciones simples que converge uniformemente a $f$
\end{teo}
\begin{proof}

    Suficiencia. La convergencia uniforme implica convergencia puntual y el Teorema \ref{convergmed} nos muestra que $f$ es medible.
    
    
    \noindent 
  Necesidad.  Sea $f$ una funci\'on $\mu$-medible, para cada $n\in\mathbb{N}$ defina

    $$f_n(x) = \frac{m}{n} , \quad \text{si} \quad \frac{m}{n}\leq f(x)< \frac{m + 1}{n}, \ \ \text{con} \ m\in \mathbb Z.$$

    Note que $\displaystyle A_ {m}= f^{-1}\left(\left[\frac{m}{n}, \frac{m + 1}{n}\right)\right)$    es  un conjunto medible. As\'i que $(f_n)_{n\in\mathbb N}$ es una sucesi\'on de funciones simples. 
    
     
    
    Por \'ultimo, vemos que $0 \leq f(x)- f_n(x)< \dfrac{1}{n} $, para cada $n\in \mathbb{N}$, de donde se deduce la convergencia uniforme.
    
\end{proof}


\begin{ob}
Dado $A\subset X$ define la funci\'on caracter\'istica de $A$ como  $\mathcal X_A$:
$$\mathcal X _A(x ) :=\left\{ \begin{array}{ll} 1, & \  \text{si} \ x\in A , \\  0, & \  \text{si} \ x\in X\setminus A.  \end{array} \right.$$
Adem\'as, se verifica de manera directamente   $f:X\to \mathbb R$ es una funci\'on simple si y s\'olo si existen $ \lambda_n \in \mathbb R  $   
  y $A_n\subset X$ un conjunto medible para cada $n\in I\subset \mathbb N$ tales que  
  $$ f(x) = \sum_{n\in I}  \lambda_n \mathcal X _{A_n}.$$
\end{ob}




\begin{teo}
    Si $f,g$ son funciones medibles y $c\in\mathbb R$ entonces $f + g$ y $cf$ son funciones medibles.
\end{teo}
\begin{proof} Mostremos  que sumas y producto por una constante de funciones simples son funciones simples.
\noindent 
   Si $f,g$ son funciones simples.  Sean 
    $A_n = \{x\in X \ | f(x) = y_n\}$ y $B_n =\{ x\in X \  | g(x) = z_n\}$ entonces 
     $$\{x \in X \ | \ f + g = y_n+z_m\} = \bigcup_{y_n + z_m = y_{n'} + z_{m'}} (A_{n'}\cap B_{m'})$$
     el cual resulta ser conjunto medible. 

\noindent 
    Si $c\neq 0$  entonces  $\{x \in X \ | cf(x) = cy_n\} = A_n$. 

\noindent
Si $(f_n)$ y $(g_n)$ son sucesiones de funciones simples que convergen uniformemente a $f$ y $g$ respectivamente. Entonces $(f_n + g_n)$ y $(cf_n)$ son sucesiones de funciones simples que convergen uniformemente a $f + g$ y $cf$ respectivamente y del teorema anterior conluimos que $f + g$ y $cf$ son funciones medibles.
\end{proof}






\begin{ob}
Si $f$ y $g$ son funciones simples entonces existen  
$ \lambda_n \delta_m \in \mathbb R $ y  ${A_n} B_m \subset X$  
medibles  para $ n\in I\subset \mathbb N$ y $ m\in J\subset \mathbb N$ tales que 
$$ f(x) = \sum_{n\in I\subset \mathbb N}  \lambda_n \mathcal X _{A_n}, \ \ \   
g(x) = \sum_{m\in J\subset \mathbb N}  \delta_m \mathcal X _{B_m}.$$  
Entonces 
\begin{align*}
(f+g)(x)=& \sum_{(n,m)\in I\times J} (\lambda_n + \delta_m)  \mathcal X _{A_n\cap B_m} , \\
(fg)(x)= & \sum_{(n,m)\in I\times J} \lambda_n  \delta_m  \mathcal X _{A_n\cap B_m}.
\end{align*}
\end{ob}


\begin{cor}
    Si $f,g$ son funciones medibles entonces $fg$ es tambi\'en medible
\end{cor}
\begin{proof} Si $h $ es una funci\'on medible y $c\in\mathbb R$ entonces la identidades
$$\{x\in X \ \mid   (h(x))^2 <c \} =  \left\{ \begin{array}{ll} \emptyset , &  \textrm{si} \ c \leq 0 , \\
   \{x\in X \ \mid   -\sqrt{c} < h(x)  < \sqrt{c} \}, & \textrm{si} \ c > 0  .     \end{array}    \right. $$
nos muestra que   $\{x\in X \ \mid   (h(x))^2 <c \}$ un conjunto  medible. Por lo tanto, $h^2$ es otra funci\'on medible.

\noindent
Por \'ultimo, note que
    $$fg = \frac{1}{4}[(f+g)^2 - (f- g)^2]$$
\end{proof}

\begin{teo}
    Si $f$ no tiene ceros y es medible, entonces $\dfrac{1}{f}$ es medible.
\end{teo}

\begin{proof}Dado $c\in \mathbb{R}$
    vea los siguientes casos: 

    \noindent
Si $c > 0$ entonces 

 $\quad \left\{x\in X \ \Big{|} \frac{1}{f(x)}<c\right\} = \{x \in X\   |f(x)\leq 0\}\cup\left\{x \in X\   | f(x)> \frac{1}{c}\right\}$.
  
    \noindent
Si $c = 0$ entonces 

 $\quad \left\{x \in X\  \Big{|} \frac{1}{f(x)}<c\right\} = \{x \in X\  |f(x)<0\}$. 

    \noindent
Si $c < 0$ entonces 
 
  $\quad \left\{x \in X\  \Big{|} \frac{1}{f(x)}<c\right\} = \{x \in X\  |\frac{1}{c}<f(x)<0\}$.
   
Como todos los conjuntos anteriores son medibles concluimos que  $\dfrac{1}{f}$ es una funci\'on medible.
\end{proof}


\begin{cor}
    Si $f,g$ son funciones medibles y $g$ no tiene ceros entonces $\dfrac{f}{g}$ es una funci\'on medible.
\end{cor}




\begin{defi}
    Sean $f,g$ funciones medibles, diremos que $f$ y $g$ son equivalentes respecto a la medida $\mu$,  si 
    $\mu(\{x\in X \ | \  f(x)\neq g(x)\}) = 0$. Tambi\'en se dice que $f$ y $g$ son iguales salvo un conjunto de medida cero o que son iguales c.t.p. (casi en todas partes) y  de denota como  $f\sim g$.
\end{defi}



\begin{ob}
La relaci\'on $\sim$ es una relaci\'on de equivalencia en el espacio de funciones medibles. Las propiedades de reflexividad y
 simetr\'ia se verifican de manera directa. Por tal motivo s\'olo platicaremos las propiedad transitiva.   
Si $f,g,h$ son funciones medibles   tales que $f\sim g$ y $g\sim h$. Entonces 
     $$\mu(\{x\in X \ | \  f(x)\neq g(x)\}) = \mu(\{x\in X \ | \  g(x)\neq h(x)\}) = 0.$$
     Note que  
   $$\{x\in X  \ | \  f(x) = g(x)\} \cap \{x\in X  \ | \  g(x) = h(x)\} \subset \{x\in X  \ | \  f(x) = h(x)\}  $$
y al considerar los complementos tendremos 
   $$\{x\in X  \ | \  f(x) \neq h(x)\} \subset  \{x\in X  \ | \  f(x) \neq g(x)\} \cup  \{x\in X  \ | \  g(x) \neq  h(x)\} .$$
As\'i que 
$$\mu ( \  \{x\in X  \ | \  f(x) \neq h(x)\} \ ) \leq $$
$$\mu ( \ \{x\in X  \ | \  f(x) \neq g(x)\} \ ) + \mu ( \   \{x\in X  \ | \  g(x) \neq  h(x)\} \ ) =0 .$$
 
     
\end{ob}

\begin{teo}
    Sean $f$ y $g$ dos funciones continuas en un intervalo $I\subset\mathbb{R}$  que son equivalentes de acuerdo a la medida de Lebesgue construida en $\mathbb{R}$. Entonces $f\equiv g$ 
\end{teo}
\begin{proof}
    (Contradicci\'on) Si existe $x_0\in I$ tal que $f(x_0)\neq g(x_0)$, de la continuidad de $f$ y $g$ existe un $\delta >0$ tal que $f(x)\neq g(x)$ para toda $x \in I\cap(x_0-\delta, x_0+\delta)$, pero el conjunto $I\cap(x_0-\delta, x_0+\delta)$ tiene medida positiva.  Por lo tanto, $\mu(\{x\in\mathbb R \ \mid \ f(x)\neq g(x)\}) >0$. Lo cual es falso. 
\end{proof}




\begin{teo}
    Una funci\'on equivalente a una funci\'on medible es tambi\'en medible. 
    \end{teo}
    \begin{proof}
    Si $f$ es medible y $f\sim g$ entonces
    $$\{x \in  X \ | \ g(x)<c\} = (\{x \in  X \ | \  f(x)< c\}\cap M^c)\cup (\{x \in  X \ | \  g(x)<c \}\cap M)$$
    Donde $M = \{x \in  X \ | \  f(x)\neq g(x)\}$ es medible y de medida cero
\end{proof}
\begin{defi}
    Considere $(f_n)$ una sucesi\'on de funciones en $X$ y sea $f$ una funci\'on definida en $X$. Diremos que $(f_n)$ converge c.t.p. a $f$ si 
 $\lim_{n\to\infty}f_n(x) = f(x)$ 
    para casi toda $x\in X$, es decir 
    $$\mu\left(\{x\in X | \lim_{n\to \infty}f_n(x)\neq f(x) \quad o\quad \nexists\lim_{n\to \infty}f_n(x)\}\right) = 0$$
\end{defi}






\begin{teo}
    Si $(f_n)$ es una sucesi\'on de funciones medibles en $X$ que  converge c.t.p. a $f$, definida en $X$, entonces $f$ es medible .
\end{teo}

\begin{proof}

Defina    $ A := \{x\in X \  \mid  \ \lim_{n\to \infty}f_n(x) = f(x)\} $. Entonces  $\mu(X\setminus A)= 0$. Luego $X\setminus A$ y $A$ son  medibles. Luego $(f_n)_{n\in \mathbb N} $ converge puntualmente a $f$ en $A$ y por el Teorema \ref{convergmed}   
obtenemos que  $f$ es medible en $A$. 
  Como $\mu $ es completa tendremos que
    $f$ es medible en $X\setminus A$, pues todo subconjunto de este es medible y de medida cero. 
    \\
    Por tanto $f$ es medible en $(X\setminus A)\cup A = X$.
\end{proof}
    





\begin{teo}(Egorov). Sea $(f_n)$ una sucesi\'on de funciones medibles que converge c.t.p. a una funci\'on $f$, en un conjunto medible $E$. Entonces para cualquier $\delta>0$ existe un conjunto medible $E_\delta \subset E$ tal que 
\begin{enumerate}
    \item $\mu(E_\delta)>\mu(E) - \delta$
    \item $(f_n)$ converge uniformemente a $f$ en $E_\delta$
\end{enumerate}
\end{teo}
\begin{proof}
    Note que $f$ es medible  y defina
    \begin{align*}
    E_n^m := &  \bigcap_{i\geq n} \left\{x\in E \ \Big{|} \  |f_i(x) - f(x)| < \frac{1}{m}\right\}\\
    =  & \left\{x\in E \ \Big{|} \ |f_i(x) - f(x)| < \frac{1}{m}, \forall i \geq n\right\}. 
    \end{align*}

    Note que $E_1^m \subset E_2^m \subset \cdots  E_{n}^m \subset E_{n+1}^m \subset \cdots$ y  defina  $\displaystyle E^m := \bigcup_{n=1}^{\infty} E_n^m$, entonces
\begin{align*}
(E^m\setminus E_1^m)\supset (E^m\setminus E_2^m)\supset \cdots  \supset (E^m\setminus E_n^m)\supset (E^m\setminus E_{n+1}^m) \supset \cdots   
 \end{align*} 
 y $ \displaystyle 
 \bigcap_{n=1}^{\infty}(E^m\setminus E_n^m)=\emptyset$.

Por un resultado anterior tendremos que 
  $$\displaystyle \lim_{n\to \infty} \mu(E^m\setminus E_n^m) = \lim_{n\to \infty} \mu ( \  \bigcap_{n=1}^{\infty}(E^m\setminus E_n^m)  \  ) = 0.$$ 

Por lo tanto,  dado $\delta> 0$ existe   $n_0(m)\in \mathbb N$ tal que 
$$\displaystyle\mu(E^m\setminus E_{n_0(m)}^m)<\frac{\delta}{2^m}$$ y defina 
$\displaystyle E_\delta = \bigcap_{m=1}^{\infty} E_{n_0(m)}$. Adem\'as, si  $x\in E_\delta$ entonces 
    $$|f_i(x) - f(x)|<\frac{1}{m}\quad\forall i \geq n_0(m)$$
    es decir, $(f_n)$ converge uniformemente a $f$ en $E_\delta$. 
    \\
    Adem\'as,   $\mu(E\setminus E^m)= 0$. Ya que  si $x\in E\setminus E^m$ entonces para cada $n\in \mathbb{N}$ existe $i\geq n$ tal que $\displaystyle|f_i(x)-f(x)|>\frac{1}{m}$. Por lo que, si  $x\in E\setminus E^m$ tendremos que  $(f_n(x))$ no converge  a $f(x)$.
    
    Por otra parte, 
   \begin{align*}
   \displaystyle \mu\left(E\setminus E_{n_0(m)}^m\right) = & \mu\left(   \  ( E \setminus E^m) \cup  (E^m \setminus E_{n_0(m)}^m) \ \right)  \\ 
      \leq  &  \mu\left(    E \setminus E^m  \right)   +   \mu\left(  E^m \setminus E_{n_0(m)}^m   \right)   \\
      \leq   & \mu(E^m\setminus E_{n_0(m)}^m)<\frac{\delta}{2^m} 
      \end{align*} y 
                \begin{align*}
                                 \mu(E)- \mu(E_\delta) = &                  \mu(E\setminus E_\delta) \\ 
             = &  \mu\left(E\setminus\bigcap_{m=1}^{\infty}E_{n_0(m)}^m\right) = \mu\left(\bigcup_{m=1}^{\infty} (E\setminus E_{n_0(m)}^m)\right)\\
                 \leq & \sum_{m=1}^{\infty}\mu(E\setminus E_{n_0(m)}^m) <  \sum_{m=1}^{\infty}  \frac{\delta}{2^m} =\delta .  
                 \end{align*} 
Entonces      $\mu(E_\delta)>\mu(E) - \delta$.


\end{proof}



\begin{defi}Una sucesi\'on de funciones medibles $(f_n)_{n\in \mathbb N}$  converge en medida a una funci\'on   $f$ en $X$ si  para cada $\epsilon> 0$ se cumple 
$$ \lim_{n\to \infty } \mu( \{  x\in X \ \mid \  |f_n(x)- f(x)| \geq \epsilon \} )=0.$$

\end{defi}





\begin{teo}
 Si una sucesi\'on de funciones medibles $(f_n)_{n\in \mathbb N}$  converge c.t.p.  a $f$ en $X$ entonces  $(f_n)_{n\in \mathbb N}$  converge en medida  a $f$ en $X$.
 \end{teo}
\begin{proof}
Del Teorema \ref{convergmed} sabemos que $f $ es una funci\'on medible. Defina  
\begin{align*}
E_{k,\epsilon} := &  \{ x\in X \ \mid \  |f_k(x) - f(x) | \geq \epsilon \} , \quad  
F_{n, \epsilon} :=   \bigcup_{k\geq n} E_{k,\epsilon} ,\\
G_\epsilon :=  & \bigcap_{n\in\mathbb N} F_{n,\epsilon}  
\end{align*}
Note que todos estos son conjuntos medibles y que  $F_{n+1, \epsilon} \subset F_{n, \epsilon} $    para cada $n\in \mathbb N$ y por un resultado sobre sucesiones decrecientes de conjuntos medibles tendremos que 
$$\lim_{n\to \infty }\mu( F_{n, \epsilon} ) = \mu (G_{ \epsilon}) . $$
Por \'ultimo, sea 
$$A = \{x\in X | \lim_{n\to \infty}f_n(x)\neq f(x) \quad o\quad \nexists\lim_{n\to \infty}f_n(x)\}  $$
Vemos que si $x\notin A  $ entonces para $\epsilon >0$ existe    $n\in \mathbb N$ tal que para cada  $k\geq n$   tendremos 
$ |  f_k(x) - f(x)| < \epsilon $; es decir, $x\notin G_{\epsilon} $. Por lo tanto $ G_{\epsilon}\subset A$ y $\mu(  G_{\epsilon}) =0$. Por lo tanto  
$$\lim_{n\to \infty }\mu( F_{n, \epsilon} ) =0. $$
 \end{proof}



\begin{ob}
El razonamiento rec\'iproco del teorema anterior no es v\'alido. Por ejemplo, para cada $n\in \mathbb N$  e $i\in \{1,\dots, n\}$ considere las funciones
$$f_ n ^i (x)=\left\{ \begin{array}{ll} 1 , & \textrm{si} \ \ x\in (\frac{i-1}{n} , \frac{i}{n} ] \\ 0 , & \textrm{en otro caso,} \end{array} \right. \quad\forall x\in (0,1]$$
Note que  si $f(x)=0$ y $0<\epsilon <1$ entonces 
 $$ \lim_{n\to \infty } \mu( \{  x\in (0,1]  \ \mid \  |f_n^i(x)- f(x)| \geq \epsilon \} )=  \lim_{n\to \infty }  \mu(  \ (\frac{i-1}{n} , \frac{i}{n} ]  \  )= \lim_{n\to \infty } \frac{1}{n} =0.$$
 Por lo tanto $(f_ n ^i)$ converge en medida a la funci\'on cero. Pero no converge a ninguna funci\'on. Ya que para cualquier  $x\in (0,1]$ y cualquier valor $n\in \mathbb N$ existe $i\in \{ 1,\dots, n  \}$ tal que $x\in  (\frac{i-1}{n} , \frac{i}{n} ] $, es decir, $f_n^i(x)=1$ pero para $j\neq i$  $f_n^j(x)=0$.  
 
 Sin embargo, si $(f_n)_{n\in \mathbb N}$ converge en medida a una funci\'on $f$ se puede plantear un algoritmo para 
 elegir sub\'indices  $(n_k)_{k\in \mathbb N}$ y obtener  una subsucesi\'on $(f_{n_k})_{k\in \mathbb N}$  que converge c.t.p. a $f$. 
 
 Por \'ultimo, el Teorema de Luzin  muestra que una funci\'on $f:[a,b]\to \mathbb R$ es medible si y s\'olo 
 para cada $\epsilon >0$ si existe $\varphi_{\epsilon} \in C([a,b], \mathbb R) $ tal que $$\mu ( \  \{x\in [a,b] \ \mid \ f(x)\neq \varphi(x)\} \ )<\epsilon.$$
 
\end{ob}




\begin{teo}
    Sean $u,v:X\to\mathbb R$ funciones medibles  y sea $\phi:\mathbb R^2\to \mathbb R$ una funci\'on continua. Entonces la funci\'on  $h(x) := \phi(u(x),v(x))$, con $x\in X$,  es medible.
\end{teo}
\begin{proof}
    Como $\phi$ es continua, solo falta mostrar que la funci\'on $f(x) = (u(x),v(x))$ es medible y para ello note que si $R = I_1\times I_2$ es un rect\'angulo en $\mathbb{R}^2$ con $I_1,I_2\subset\mathbb{R}$ intervalos, entonces $f^{-1}(R) = u^{-1}(I_1)\cap v^{-1}(I_2)$ el cual es medible. Asi que si consideramos cualquier uni\'on infinita numerable de rect\'angulos $\displaystyle A = \bigcup_{k\in \mathbb{N}}R_k$ tendremos $\displaystyle f^{-1}(A) = f^{-1}\left(\bigcup_{k\in \mathbb{N}}R_k\right) = \bigcup_{k\in \mathbb{N}}(f^{-1}(R_k))$ que es medible.
\end{proof}

Recuerde que $\mathbb C$ y $\mathbb R^2$ son espacios normados reales isom\'etricos e isomorfos.
 
\begin{cor} Considere $f:X \to\mathbb C$.
    \begin{enumerate}
        \item Si $\Re f, \ \Im f :X\to \mathbb R$ son medibles entonces $f$ es medible en $X$.

        \item Si $f $ es una funci\'on  medible   entonces $\Re f$, $\Im f$ y $|f|$ son funciones medibles. 

        \item Si $f, g: X \to \mathbb C$ son funciones  medibles  entonces $f\underline{+}g$, $fg$ son  medibles en $X$.
    \end{enumerate}
\end{cor}


 

\begin{ejer}\ {} 
\begin{enumerate}
\item  Considere $f:X \to \mathbb R$.  Si    $| f |$ es medible muestre que no siempre  $f$ es medible. 

Sugerencia: Sea $C\subset [0,1]$ un subconjunto no medible. Defina 
$$f(x):=\left\{ \begin{array}{cl} 1  , & \ \textrm{si} \  x\in C ,\\   
 -1  , & \ \textrm{si} \  x\in [0,1]\setminus C. \end{array}   \right. $$  
Note que  $f^{-1}(\{1\}) =C$ pero $|f(x)|=1$ para cada $x\in [0,1]$.

\item Si $f,g :X\to \mathbb R$ muestre que las funciones $\max\{f,g\}$ y $\min\{f,g\}$ son funciones medibles. 


Sugerencia:   
Note que  
\begin{align*}
 \max\{f,g\}(x)   = & \frac{1}{2} \left\{ f (x) + g(x)  + |  f (x) -g(x) | \right\}  ,\\
  \min\{f,g\}(x)   = & \frac{1}{2} \left\{ f (x) + g(x)  - |  f (x) -g(x) | \right\}  ,\quad \forall x\in X.
  \end{align*}
  
\item Sea $(f_n)_{n\in\mathbb N} $   una sucesi\'on de funciones medibles de $X$ a $\mathbb R$.  Muestre que 
las funciones $\sup_{n\in \mathbb N}\{f_n\}$ e $\inf_{n\in \mathbb N}\{f_n\}$ son medibles.

Sugerencia:   
 \begin{align*}
\{ x\in X \ \mid \  \inf_{n\in \mathbb N}\{f_n\} (x) <c \} = & \bigcup_{n\in \mathbb N}  \{ x\in X \ \mid \   f_n (x) <c \}\\
   \sup_{n\in \mathbb N}\{f_n\}  = & -  \inf_{n\in \mathbb N}\{-f_n\} . \end{align*}



\item Sea $(f_n)_{n\in\mathbb N} $   una sucesi\'on de funciones medibles de $X$ a $\mathbb R$.  Muestre que 
las funciones $\limsup_{n\in \mathbb N}\{f_n\}$ e $\liminf_{n\in \mathbb N}\{f_n\}$ son medibles.

Sugerencia:
 \begin{align*}
 \limsup_{n\in \mathbb N}\{f_n\} = & \inf_{n\in \mathbb N} \{ \sup_{ k\geq n}\{f_k\} \} , \\
   \liminf_{n\in \mathbb N}\{f_n\} = & \sup_{n\in \mathbb N}\{  \inf_{ k\geq n}\{f_k\} \}. 
   \end{align*}
    

\item Muestre que la sucesi\'on de funciones medibles  $f_n(x)= x^n$, para cada $x\in [0,1]$, converge c.t.p. a la funci\'on constante $0$.

Sugerencia: Note que $\{ x\in [0,1] \ \mid  \  \lim_{n\to \infty} x^n \neq 0 \}=\{1\}$.


\item Para cada $n\in \mathbb N$ defina
$$
f_n(x):= \left\{  \begin{array}{ll} 0, & \   \textrm{si} \ \ x\in [\frac{1}{n }, \frac{1}{n-1}] , \\
  1 , & \   \textrm{en otro caso,}        \end{array}  \right. \textrm{ para cada $x\in [0,1]$.}
$$ 
Muestre que $(f_n)$ converge en medida a la funci\'on constante $f(x)=1$ para cada $x\in [0,1]$.

Sugerencia: Dado $0<\epsilon\leq 1$  luego
\begin{align*}
\mu( \ \{   x\in [0,1] \ \mid \  |f_n (x)- f(x)| \geq \epsilon \} \   ) = & \mu( \  \{ x\in [\frac{1}{n}, \frac{1}{n-1}]  \ \mid \  1  \geq \epsilon \} \ )  \\
                                                                                       = & \frac{1}{(n-1)n}      .
                                                                                \end{align*}
                                                                                 
\end{enumerate}
\end{ejer}









\subsection{Integraci\'on de Lebesgue}






\begin{defi}
    Sea $f$ una funci\'on simple con $(y_n)_{n\in \mathbb{N}}$ sus diferentes valores definida en $X$. Sea $A\subset X$ un conjunto medible, por la integral de Lebesgue de $f$ sobre $A$ (con respecto a $\mu$) entenderemos 
    $$\int_A f(x)d\mu = \sum_{n\in \mathbb{N}}y_n \mu(A_n),$$
donde $A_n = \{x\in A | f(x) = y_n\}$ y la serie es absolutamente convergente. 
\end{defi}


\begin{lem}\label{lemasuma}
    Sea $f$ una funci\'on simple definida en la uni\'on $A = \bigcup B_k$ de conjuntos medibles y ajenos a pares tales que $f$ toma un solo valor $c_k$ en $B_k$. Entonces $f$ es integrable en $A$ si y solo si, la serie
    $$\sum_{k}c_k\mu(B_k)$$
    es absolutamente convergente y en este caso 

    $$\int_A f(x)d\mu = \sum_k c_k \mu(B_k)$$
\end{lem}
\begin{proof}
    Si $A_n = \{x\in A \ | \ f(x) = y_n\}$ entonces 
    $$\displaystyle\sum_n y_n\mu(A_n) = \sum_n y_n\left(\bigcup_{c_k = y_n}B_k \right) = \sum_n y_n \sum_{c_k = y_n} \mu(B_k) = \sum_k c_k\mu(B_k)$$
    Como $\mu$ es no negativa:

    $$\sum_n |y_n| \mu(A_n) = \sum_n |y_n|\sum_{c_k = y_n}\mu(B_k) = \sum_k |c_k| \mu(B_k)$$
\end{proof}     
Para acortar la redacci\'on todos los conjuntos se considerar\'an medibles.
\begin{teo}
    Si $f,g$ son funciones simples e integrables en $A$ y $k\in \mathbb{R}$. Entonces $f + g$ y $kf$ son integrables en $A$.  M\'as a\'un, 
   \begin{align}
   \int_A( f(x) + g(x))d\mu = &  \int_A f(x)d\mu + \int_A g(x)d\mu  \label{equation1}, \\ 
     \int_A kf(x)d\mu  = &  k\int_A f(x)d\mu  . \label{equation2}
\end{align}
\end{teo}
 
\begin{proof}
    Sean $F_n = \{x\in A \ | \  f(x) = y_n\}$ y  $G_m = \{x \in A \ | \ g(x) = z_m\}$. Denote 
     $$B_{nm} = F_n\cap G_m = \{x\in A\ | \ (f + g)(x) = y_n + z_m\}$$
    y vemos que 
     $\displaystyle F_n =\bigcup_{m}B_{nm} $, $\displaystyle G_m =\bigcup_{n}B_{nm} $ siendo estos conjuntos ajenos a pares. Luego
    \begin{align*}
  \int_A f(x)d\mu + \int_A g(x) d\mu  = & \sum_n y_n\mu(F_n) + \sum_m z_m \mu(G_m) \\
        &= \sum_n y_n \sum_m \mu(B_{nm}) + \sum_m z_m \sum_n \mu(B_{nm})\\
        & = \sum_n\sum_m(y_n + z_m)\mu(B_{nm})\\
        & = \int_A ( f(x)  +  g(x) ) d\mu.
    \end{align*}
     Muestre \eqref{equation2}.
\end{proof}

\begin{teo}
    Sea $f$ una funci\'on simple  y sea  $A\subset X$  en un conjunto medible. Si   $|f(x)|\leq M$, para toda $x\in A$, entonces $f$ es  integrable y 
    $$\left| \int_A f(x) d\mu\right|\leq M\mu(A).$$
\end{teo}
\begin{proof}
    Sea $A_n = \{x\in A \ | \ f(x)= y_n\}$. Luego 
          $$\left|\sum_n y_n\mu(A_n)\right|\leq \sum_n |y_n|\mu(A_n)\leq\sum_n M\mu(A_n) = M \mu(A).$$
\end{proof}

\begin{defi}
    Una funci\'on $f$ se dice sumable o integrable en $A$ (con respecto a la medida $\mu$) si existe una sucesi\'on de funciones $(f_n)$ simples e integrables en $A$ que converge uniformemente a $f$ en $A$. La integral de Lebesgue de $f$ en $A$ se define  como 
    $$\int_A f(x)d\mu = \lim_{n\to \infty}\int_A f_n(x)d\mu.$$
\end{defi}


\begin{ob} {}\
    \begin{enumerate}
        \item La desigualdad
        \begin{align}\label{ecuaCauchy}
            \left|\int_A f_m(x)d\mu - \int_A f_n(x)d\mu\right| &= \left|\int_A(f_m(x) - f_n(x))d\mu\right|\nonumber \\
            &\leq \sup\left\{\left|f_m(x)-f_n(x)\right|  \  \mid \  x\in A\right\}\mu(A)
        \end{align}
   muestra      que existe $\displaystyle\int_A f(x)d\mu$.
        \item M\'as a\'un, $\int_A f(x)d\mu$ es independiente de la sucesi\'on de funciones simples e integrables en $A$ que convergen uniformemente a $f$. Pues si $(f_n)$ y $(g_n)$ son tales sucesiones. Entonces    la sucesion $f_1, g_1, f_2,g_2,\cdots$ converge uniformemente a $f$ en $A$. Por \eqref{ecuaCauchy} 
        existe   el l\'imite de 
        $$\displaystyle\int_A f_1(x)d\mu, \int_Ag_1(x)d\mu, \int_Af_2(x)d\mu,\cdots$$ y por lo tanto 
        $$\lim_{n\to\infty}\int_A f_n(x) = \lim_{n\to \infty}\int_A g(x)d\mu.$$


        \item Si $f$ es simple e integrable en $A$ entonces la definici\'on anterior coincide con la integral de la funci\'on simple $f$.
    \end{enumerate}
\end{ob}


\begin{teo}\label{teoremacomparacion}
    Sean $f,\varphi$ funciones definidas en $A\subset X$ medible. Si   $\varphi$ es no negativa e integrable en $A$ y  $|f(x)|\leq \varphi(x)$ c.t.p. en $A$ entonces $f$ es integrable en $A$ y 
    $$\left|\int_A f(x)du\right|< \int_A \varphi(x) du.$$
\end{teo}
\begin{proof}
    Sean $f,\varphi$ funciones simples y denote  $E=\{x\in A \ | \ |f(x)| >\varphi(x)\}$. Sean 
    $$A_n = \{x\in A | f(x) = y_n\} = \{x\in A | \varphi(x) = \rho_n\}.$$
     Luego
    \begin{align*}
        \sum |y_n|\mu(A_n) &= \sum |y_n|[\mu(A_n\cap E) + \mu(A_n \cap E^c)]\\
                           &= \sum |y_n| \mu(A_n\cap E^c) \leq \sum \rho_n \mu(A_n\cap E^c)\\
                           &\leq  \sum \rho_n \mu(A_n) .
    \end{align*}
    Por lo tanto $f$ es integrable y 
    $$\left|\int_A f(x)du\right|\leq \sum |y_n|\mu(A_n) \leq  \sum \rho_n \mu(A_n) .$$


Por otra parte, si $f,\varphi$ son funciones medibles, considere las sucesiones de fuciones simples $(f_n)$ y $(\varphi_n)$ creadas como 
  \begin{align*}
  \displaystyle f_n(x) =   \frac{m}{n} ,  &  \   \textrm{ si }  \displaystyle \frac{m}{n}\leq f(x) < \frac{m+ 1}{n},\\
  \displaystyle \varphi_n(x) =   \frac{m+1}{n} , &  \   \textrm{ si }  \frac{m}{n}< \varphi (x) \leq  \frac{m+ 1}{n},
    \end{align*}
       Entonces $|f_n(x) | \leq \varphi_n(x)$ para cada $n\in \mathbb{N}$ y por el hecho anterior se tiene que 
       $| \displaystyle\int_A f_n(x) | du \leq \int_A \varphi_n(x) du$ para cada $n\in \mathbb{N}$. El resultado es obtenido haciendo $n$ tender  a $\infty$.
\end{proof}




\begin{cor}
    Si $f$ es medible y acotada en $A$ entonces $f$ es integrable en $A$
\end{cor}
\begin{proof}
    Si $|f(x)| < M$ para toda $x\in A$ basta con definir $\varphi(x) = M$ para toda $x\in A$ y aplicar el resultado anterior.
\end{proof}






\begin{teo}
    Sea $A = \bigcup_{n\in J\subset \mathbb N} A_n$ uni\'on a lo sumo numerable de conjuntos medibles. Si $f$ es integrable en $A$, entonces $f$ es integrable en cada $A_n$ y 
    $$\int_A f(x) du = \sum_{n\in J } \int_{A_n}f(x) du ,$$
    donde la serie es absolutamente convergente
\end{teo}
\begin{proof}
Verificamos la identidad para funciones simples: Sea  $\varphi$ una funci\'on simple en  $A$ y sean $(\lambda_m)_{m\in I\subset \mathbb N} $ los diferentes valores y sea $B_m=\{ x\in A \  \mid \ f(x)=\lambda_ m\}$ para cada $ m \in I $. Los conjuntos $(A_n\cap B_m)_{(m,n)\in I\times J }$ son ajenos a pares y en 
$A_n\cap B_m $ la funcion $\varphi$  toma el valor $\lambda_m$ y del Lema \ref{lemasuma}
 tenemos que  
 \begin{align*}
 \int_{A} \varphi (x) d\mu  = & \sum_{(m,n)\in I\times J } \lambda_m \mu(A_n\cap B_m)  \\
  = &  \sum_{ n \in J }  (   \sum_{ m \in I } \lambda_m \mu(A_n\cap B_m) ) = \sum_{n\in J} \int_{A_n}\varphi(x) du. 
 \end{align*}
 
 Si $(\varphi_m)_{m\in\mathbb N}$ es una sucesi\'on de funciones simples que convergen unifomemente a $f$ tendremos 
  \begin{align*}
  \int_{A} f (x) d\mu   = & \lim_{m\to \infty }
  \int_{A} \varphi_m (x) d\mu     = \lim_{m\to \infty } \sum_{n\in J} \int_{A_n}\varphi_m(x) du \\
  =&  \sum_{n\in J}   \lim_{m\to \infty } \int_{A_n}\varphi_m(x) du  = \sum_{n\in J} \int_{A_n} f (x) d\mu  . 
 \end{align*}
 
  \end{proof}





\begin{cor}
    Si $f$ es integrable en $A$ y $A'\subset A$ es medible, entonces $f$ es integrable en $A'$
\end{cor}
\begin{proof}
Verificamos la identidad para funciones simples: Sea  $\varphi$ una funci\'on simple en  $A$ y sean $(\lambda_m)_{m\in I\subset \mathbb N} $ los diferentes valores y sea $B_m=\{ x\in A \  \mid \ f(x)=\lambda_ m\}$ para cada $ m \in I $. Los conjuntos $(A'\cap B_m)_{m\in I }$ son ajenos a pares y en 
$A'\cap B_m $ la funcion $\varphi$  toma el valor $\lambda_m$ y del Lema \ref{lemasuma}
 tenemos que  
 \begin{align*}
  \sum_{ m \in I  }| \lambda_m |\mu(A'\cap B_m)  \leq     \sum_{ m \in I }| \lambda_m |\mu( B_m)  . 
 \end{align*}
 As\'i que existe $\int_{A'} \varphi(x)d\mu $. 
 \\
 
Para las dem\'as funciones se deduce de la propiedad anterior.

  \end{proof}





\begin{teo}
    (Desigualdad de Chebyshev) Si $f$ es integrable y no negativa en $A$ y $c>0$ entonces\\
    $$\mu(\{   x\in A \ \mid \  f(x) \geq c\})\leq \frac{1}{c}\int_A f(x) du.$$
\end{teo}
\begin{proof}
El conjunto $B =  \{x\in A \ \mid \  f(x) \geq c\})$ es medible  y si $g(x)=c$ para cada $x\in B$ tenemos que
$g\leq f$ en $B$. Por el Teorema \ref{teoremacomparacion} concluimos 
\begin{align*}
 c\mu(\{   x\in A \ \mid \  f(x) \geq c\})  =  & c\mu(B)=  \int_B g(x)d\mu  \leq \int_B f (x)d\mu.  
 \end{align*}
 Por \'ultimo, como $f $ es no negativa y $B\subset A$ se muestra 
 $$\int_B f (x)d\mu\leq \int_A f (x)d\mu $$
 partiendo de funciones simples.
 
\end{proof}




\begin{cor}
Sea $f$ una funci\'on integrable y no negativa en $A$.   Si $\displaystyle\int_Af(x)du = 0$ entonces $f(x) = 0$ c.t.p. en $A$
\end{cor}
\begin{proof}
    Ejercicio. 
    Sugerencia: Use la Desigualdad de Chebyshev.
     \end{proof}




\begin{teo}
    Si $f$ es integrable en $A$, entonces para cada $\epsilon> 0$ existe $\delta>0$ tal que 
    $$\left|\int_Ef(x) du\right|<\epsilon$$
    para todo $E\subset A$ conjunto medible y con medida menor que $\delta$
\end{teo}
\begin{proof}
 Para cada $n\in \mathbb N$ define 
 $$A_n =\{  x\in A \ \mid \ n-1 \leq  |f(x)|< n \}.$$ 
Note que $A=\bigcup_{n\in\mathbb N}A_n$ luego 
$$\int_A |f(x)| d\mu = \sum_{n\in\mathbb N}  \int_{A_n} |f(x)| d\mu <\infty.$$
Existe $N\in \mathbb N$ tal que $$\sum_{n= N+1 }^{\infty }  \int_{A_n} |f(x)| d\mu<\frac{\epsilon}{2}. $$
Sea $B_N =\bigcup_{n=1}^N A_n$ y $C_N =\bigcup_{n=N+1}^{\infty} A_n $. 


Por \'ultimo, consideremos $0<\delta<\frac{\epsilon}{2N}$. As\'i que para cada $E \subset A$ tal que $\mu(E) <\delta$ tendremos 
    \begin{align*} 
    \left|\int_E f(x) du\right|\leq  &  \int_E   \left|  f(x)  \right| du  =  \int_{E\cap B_N}   \left|  f(x)  \right| du + \int_{E\cap C_N }  \left|  f(x)  \right| du \\
    \leq  &   N \mu (E) + \int_{C_N}   \left|  f(x)  \right| du \\
    \leq &   N \frac{\epsilon}{2N} +    \sum_{n= N+1 }^{\infty }  \int_{A_n} |f(x)| d\mu< \epsilon.\\
        \end{align*}
 \end{proof}
 
 
 
\begin{teo}
    (Lebesgue). Si $(f_n)_{n\in \mathbb N}$ es una sucesi\'on de funciones medibles que  converge a $f$ en $A$,  $|f_n(x)| \leq \varphi(x)$ para todo $n\in \mathbb N$ y $\varphi$ es integrable en $A$  entonces $f$ es integrable en $A$ y 
    $$\lim \int_A f_n(x) du = \int_Af(x)du.$$
\end{teo}
\begin{proof}
    Haciendo $n\to\infty$ de las hip\'otesis se tiene $|f(x)|\leq \varphi(x)$. Del Teorema \ref{teoremacomparacion} sabemos que  $f$ es integrable en $A$.
    
    

   Para cada $k\in \mathbb N$ defina  $A_k := \{x\in A | k-1\leq \varphi(x)< k\}$ y $\displaystyle B_m := \bigcup_{k\geq m} A_k$. Los cuales son conjuntos medibles.  
 Luego    $$\int_A \varphi du = \sum_k \int_{A_k} \varphi du < \infty$$
Dado $\epsilon>0$ existe $m^*\in \mathbb N$ tal que 
    $$\int_{B_{m^*}}\varphi du <\frac{\epsilon}{5} $$
y vemos que   $\varphi(x)<m^*$ para cada $x\in A\setminus B_{m^*}$. 
    
    Del  teorema de Egorov existen conjuntos medibles  $C,D$ tales que $A\setminus B_{m^*} = C\cup D$ con $\displaystyle\mu(D)<\frac{\epsilon}{ 5m^*}$ y $\{f_n\}$ converge uniformemente a $f$ en $C$.    Sea $N\in\mathbb N $ tal que $\displaystyle |f_n(x) - f(x)| < \frac{\epsilon}{5\mu(C)}$ para toda $n\geq N$, en $C$. Luego
    $$\int_A(f_n - f)du = \int_{B_{m^*}}f_ndu - \int_{B_{m^*}}fdu + \int_D f_n du- \int_D f du+ \int_C(f_n - f) du$$
    Asi que 
    $$\left|\int_A(f_n - f)du\right|\leq \frac{\epsilon}{5} + \frac{\epsilon}{5} + m^*\mu(D)+m^* \mu(D) + \frac{\epsilon}{5\mu(C)}\mu(C) = \epsilon$$
\end{proof}
\begin{teo}
    (Beppo-Levi). Sea $(f_n)_{n\in \mathbb N}$ es una sucesi\'on de funciones medibles tal que  $f_1\leq f_2\leq \cdots \leq f_n\leq \cdots$ en $A$. Si  $f_n$ es integrable en $A$ y existe $k\in \mathbb R$ tal que 
    $$\int_A f_n du \leq k \quad \forall n\in \mathbb N$$
entonces existe $\displaystyle f(x) = \lim_{n\to \infty} f_n(x)$ c.t.p. en $A$ que  es integrable en $A$ y 
    $$\lim_{n\to\infty} \int_A f_n = \int_A f.$$
\end{teo}
\begin{proof}
    Suponga $f_1\geq 0$ en $A$ o en caso contrario trabaje con $g_n = f_n - f_1$ para cada $n\in \mathbb N$.
    
    Defina $\displaystyle\Omega = \left\{x\in A | \lim_{n\to\infty} f_n(x) = \infty\right\}$ y 
    $\displaystyle\Omega_n^{(r)} = \left\{x\in A |  f_n(x) > r\right\}$ para cada $n,r\in\mathbb N $. \\
    Se deduce que $\displaystyle\Omega = \bigcap_{r\in\mathbb N} \ \bigcup_{n\in\mathbb N} \Omega_n^{(r)}$ y de la desigualdad de Ch\'ebishev se obtiene  que 
    $$\mu(\Omega_n^{(r)} )\leq \frac{k}{r} ,\quad \forall n\in \mathbb N.$$

    Note que $\Omega_1^{(r)}\subset\Omega_2^{(r)}\subset\cdots$, por lo que 
    $$\mu\left(\bigcup_{n\in\mathbb N} \Omega_n^{(r)}\right)\leq \frac{k}{r}$$
    Adem\'as,
    $$\Omega \subset \bigcup_{n\in\mathbb N} \Omega_n^{(r)},\quad \forall r\in \mathbb N.$$
Luego $\mu(\Omega)<k/r$ para cada $r\in \mathbb N $. Por lo tanto, $\mu(\Omega)=0$.
    Asi que existe\\
    $\displaystyle f = \lim_{n\to\infty} f_n$ c.t.p. en A.

    Defina $\varphi(x) = r$ si $r - 1 \leq f(x) < r$ para $r = 1, 2 , 3, \cdots$. Luego  $f(x) < \varphi(x)$ para cada  $x\in A$. Usando el Teorema \ref{teoremacomparacion}  basta mostrar que $\varphi$ es integrable. 

    Sea $A_r = \{x\in A | \varphi(x) = r\}$ y sea $\displaystyle B_s = \bigcup_{r = 1}^s A_r$ con $s\in \mathbb N$. Note que 	$ f$ y $f_n$ son acotadas en $B_s$. Luego $\varphi(x) < f(x) + 1$ en $A$, implica
    \begin{align*}
        \int_{B_s}\varphi du \leq & \int_{B_s}fdu + \mu(A) = \lim_{n\to \infty}\int_{B_s}f_n du+ \mu(A)\\
        \leq &k + \mu(A)
    \end{align*}
    Como
    $$\int_{B_s}\varphi du = \sum_{r=1}^s r\mu(A_r)\leq k + \mu(A)$$
    para cada $s$, se tiene que la serie es convergente, luego $\varphi$ es integrable
    en  $B_s$ para cada $s\in \mathbb N$. As\'i que  
    $$\sum_{r=1}^s\int_{A_r}\varphi du  \leq k + \mu(A).$$
Entonces 
$$\int_{A}\varphi du = \sum_{r=1}^\infty  \int_{A_r}\varphi du  \leq k + \mu(A).$$
\end{proof}
\begin{cor}
    Si $\varphi_n \geq 0$ en $A$ si 
    $$\sum_{n=1}^\infty \int_A \varphi_n du <\infty $$
    Entonces $(\displaystyle \sum_{n=1}^s \varphi_n)_{s\in\mathbb N}$ converge c.t.p. en $A$ y 

    $$\int_A\left(\sum_{n = 1}^{\infty}\varphi_ndu\right) = \sum_{n=1}^\infty \int_A \varphi_n du$$
\end{cor}
\begin{proof}
Ejercicio.

 Sugerencia: Considere $f_n = \sum_{k = 1}^{n}\varphi_k$ para cada $n\in \mathbb N$ y emplee el teorema anterior.
\end{proof}



\begin{teo}
    (Fatou). Sea $(f_n)_{n\in\mathbb N} $ una sucesi\'on de funciones medibles no negativas que converge c.t.p. a $f$ en $A$ y 
    $$\int_A f_n du \leq k$$
    Entonces $f$ es integrable en $A$ y 
    $$\int_A  f du \leq k$$
\end{teo}
\begin{proof}
    Para $n\in \mathbb N$ defina $\varphi_n(x) := \inf \{f_k(x) | k\geq n\}$ para cada  $x\in A$. Como 
    $$\{x\in A | \varphi_n(x)< c\} = \bigcup_{k\geq n }\{x\in A| f_k(x)< c\}$$
    Se deduce que $\varphi_n$ es medible para cada $n\in \mathbb N$. Adem\'as, $0\leq \varphi_n\leq f_n$
    en $A$. Por lo tanto 

    $$\int_A \varphi_n du \leq \int_A f_n du \leq k$$
    Note que $\varphi_1\leq \varphi_2\leq \cdots\varphi_n \leq \cdots$ en $A$ y 
    $\displaystyle\lim_{n\to \infty}\varphi_n(x) = f(x)$. Del teorema anterior tenemos  se tiene que    $f$ es integrable en $A$ y 
   $$ \int_A f=\lim_{n\to\infty} \int_A \varphi_n \leq k.$$
    

\end{proof}


\chapter{Espacio de funciones $L_p$ } 



 
 
\section{Sobre espacios reales lineales} 

Sea $\mathbb K= \mathbb R,\mathbb C$
Dado $V$  un espacio $\mathbb K$-lineal es un conjunto equipado con dos operaciones suma $+ V\times V \to V$ y la multiplicaci\'on por escalar real $*: \mathbb K \times V\to V$ tales que

1.- $(f+g)+h= f+ (g+h)$ para cada $f,g,h\in V$

2.- $f+g= g+f$ para cada $f,g\ in V$

3.- Existe $0\in V$ tal que $0+f=f$ para cada $f\in V$

4.- Para cada $f\in V$ existe $-f\in V$ tal que $f+(-f) =0$.

5.- $r*(f+g)= r*f+r*g$ para cada $r\in \mathbb K$ y cada $f,g\in V$.

6.- $(r+s)* f = r*f+s*f$ para cada $r,s\in \mathbb K$ y cada $f\in V$.

7.- $r*(s* f) =(rs)* f  $ para cada $r,s\in \mathbb K$ y cada $f\in V$.

8.- $1*f= f$ para cada $f\in V$.

Para simplificar denotamos $r*f$ como $rf$.



Recuerde que  $\{ f_1,\dots, f_n \} \subset V$  es un conjunto l.i. (linealmente independientes) si y s\'olo si 
$$ a_1 f_1+ \cdots a_n f_n = 0
 , \ \   \textrm{con} \  \  a_1, \dots, a_n \in \mathbb K ,$$  implica $a_1= \cdots= a_n =0$. En caso contrario, $\{ f_1,\dots, f_n \}$  se dice l.d. (linealmente  dependientes)



Sabemos que $W\subset V$ es un subespacio de $V$ si $W $ dotado con las operaciones heredadas de $V$ es un espacio $\mathbb R$-lineal. Claramente se verifica que la intersecci\'on de cualquier familia  de subespacios vectoriales de  $V$  es tambi\'en un subespacio vectorial de $V$ 





Una familia $F$ de elementos de $V$ es  l.i. si cualquier subconjunto finito de $F$ es l.i.


Se puede mostrar que toda familia de elementos de $V$ contiene una subfamilia maximal, no necesariamente \'unica, de elementos l.i. y todas estas subfamilias tienen la misma potencia. En general $V$    contiene  subfamilias maximales  de elementos l.i. llamadas  bases algebraicas de $V$ y la potencia de cada una de estas es llamada dimensi\'on algebraica de  $V$. 

 

Recordemos lo siguiente: 

\begin{enumerate}

\item  Dado cualquier subconjunto $U$ de $V$ sabemos que existe el menor subespacio de $V$ que contiene a $U$ y es denotado por $[U]$. Se verifica que 
$$ [U] =\{  a_1 f_1+ \cdots+ a_n f_n\ \mid \  a_1,\dots ,a_n \in \mathbb K, \  f_1,\dots, f_n\in U\}$$
y es llamado subespacio vectorial generado por $U$. Toda subfamilia maximal l.i. de $U$ es una base algebraica de $U$.  

\item Si $W_1,W_2$ son subespacios de $V$. Diremos que $W$ es suma  $W_1$ y $ W_2$ ($W= W_1+ W_2$)  sii todo elemento $f$ de $W$ se representa como $f= f_1 + f_2$ con $f_k\in W_k $  para $k=1,2$.  Diremos que $W$ es suma direta de $W_1$ y $W_2$ ($W=W_1\oplus W_2$), si $W= W_1+ W_2$ y la representaci\'on de cada elemento de $W$ en t\'erminos de la suma de lementos de $W_1$ y $W_2$  es \'unica. Se verifica que  $W=W_1\oplus W_2$ sii $W=W_1+W_2$ y $\emptyset = W_1\cap W_2$.  

\item Un espacio $\mathbb K$-lineal $V$ con una  topolog\'ia $\tau$ se dice topol\'ogicamente compatible con la estructura de espacio $\mathbb K$-lineal de $V$ si las operaciones suma y multiplicaci\'on   por escalar de $V$  son funciones continuas con respecto  a $\tau$.





\item Dado un espacio $\mathbb K $-lineal $V$. Una funci\'on $N:V\to \mathbb R$ es llamada norma si 
\begin{align*}
N(f)\geq & 0,\quad  \forall f\in V,\\
N(f) = & 0 \quad \Leftrightarrow \quad f=0, \\
N(\lambda f) = &|\lambda |N(f), \quad \forall \lambda \in \mathbb K, \quad \forall f\in V,\\
N(f+g)\leq & N(f) + N(g), \quad\forall f,g\in V.
\end{align*}
Un espacio    $\mathbb K$-lineal $V$ equipado con una norma $N$ se llamar\'a espacio normado  y denotado como $(V,N)$.

 

\item En un conjunto no vac\'io  $V$ una funci\'on $\rho:V\to \mathbb R$ es llamada m\'etrica  si 
\begin{align*}
\rho(f,g)\geq & 0,\quad  \forall f,g\in V,\\
\rho(f,g) = & 0 \quad \Leftrightarrow \quad f=g, \\
 \rho(f,g)\leq &\rho(f,h) + \rho(h,g) , \quad \forall f,g,h\in V.
\end{align*}
El conjunto  $V$ equipado con una m\'etrica  $\rho$ se llamar\'a espacio m\'etrico y denotado como   $(V,\rho)$.

Si $(V,N)$ es  espacio normado  entonces se verifica de manera directa que  
$\rho(f,g) =N(f-g)$ para cada $f,g\in V$ es una m\'etrica en $V$. 














 
\item Un espacio   m\'etrico   $(V,\rho)$ se dice completo si toda sucesi\'on $(f_n)_{n\in \mathbb N}$ que cumple la condici\'on de Cauchy:  
$$\forall \epsilon<0, \ \  \exists N\in\mathbb N \ \ \pitchfork \rho(f_n, f_m) <\epsilon, \quad \forall n,m\geq N$$
es una sucesi\\'on convergente.






 
\item Un espacio $\mathbb K$-lineal normado es  llamado  $\mathbb K$ espacio  se Banach  si 
 el espacio  m\'etrico $(V, \rho) $ es completo, donde  
 $\rho(f,g) =N(f-g)$ para cada $f,g\in V$.
 
 
 \item 	  Dado un espacio $\mathbb C$-lineal $V$. Una funci\'on $\langle \cdot,  \cdot\rangle: V\times V \to \mathbb C$ es llamado producto escalar o interno si 
	\begin{enumerate}
	\item $\langle f,g\rangle =\overline{\langle g,f\rangle }$,
	\item $\langle f,g+h \rangle = {\langle f,g \rangle } + \langle f,h \rangle$,
	\item $\langle \lambda  f,g\rangle =\lambda {\langle f,g\rangle }$,
	\item $\langle  f, f \rangle \geq 0$. Por otra parte,  $\langle  f, f \rangle = 0$ sii  $f=0$ ,
	\end{enumerate}
	para cada $f,g,h\in V$ y para cada $\lambda\in \mathbb C$.
 La pareja $(V, \langle \cdot,  \cdot\rangle)$ suele llamarse espacio euclideano complejo y se verifica que  la funci\'on 
$$\|f\|= \sqrt{\langle  f,f \rangle}, \quad \forall f\in V,$$
es una norma.  

En el caso de  un espacio $\mathbb R$-lineal $V$. Una funci\'on $\langle \cdot,  \cdot\rangle: V\times V \to \mathbb R$ es llamado producto escalar o interno  si cumple las propiedades anteriores salvo la primera que se convierte en 
$$\langle    f,g\rangle = \langle g,f\rangle , \forall f,g\in V  $$
y  la pareja $(V, \langle \cdot,  \cdot\rangle)$ se llama espacio euclideano real.
 Tambi\'en $$\|f\|= \sqrt{\langle  f,f \rangle}, \quad \forall f\in V,$$
es una norma.  



(Desigualdad de Cauchy-Buniakowsky)  Dados $,x y $ en un espacio euclideano $(V, \langle \cdot,  \cdot\rangle)$ se cumple que 
$$  | \langle f, g \rangle \leq \|f \| \|g \|  $$
La igualdad se cumple si y s\'olo si   existe $\lambda \in \mathbb C$ tal que $f=\lambda g$ o $f=0$.
 

\item  Un espacio euclideano (complejo o real) es una espacio $\mathbb K$-lineal con producto interno:  $(V, \langle \cdot,  \cdot\rangle)$. Ejemplos:

\begin{enumerate}
\item El espacio $\mathbb C-$lineal  $\mathbb C^n$ dotado con  $$\langle (z_1, \dots, z_n ), (w_1, \dots, w_n) \rangle= \sum_{k=1}^n  z_k \overline{w_k}, \quad \forall  (z_1, \dots, z_n ), (w_1, \dots, w_n) \in \mathbb C^n$$
es un espacio euclideano. 
\item    El espacio $\mathbb C-$lineal  formado por sucesiones de n\'umeros complejos $(z_n)_{n\geq 0}$ tales que  
 $\sum_{k=0}^{\infty } \|z_k\|^2<  \infty $   dotado con  $$\langle (  z_n ), (  w_n) \rangle= \sum_{k=0}^\infty  z_k\overline{  w_k} $$
es un espacio euclideano. 
\item  Dados $a< b$,  el espacio $\mathbb C-$lineal $ C([a,b], \mathbb C)$ dotado con  
$$\langle f,g \rangle= \int_a^b f(t) \overline{ g(t)} dt  , \quad \forall  f,g \in C([a,b], \mathbb C).$$
es un espacio euclideano. 
\end{enumerate}


 En un espacio euclideano  $(V, \langle \cdot,  \cdot\rangle)$ la norma    $ \|f\| = \sqrt{  \langle f, f \rangle  } $, para cada  $f\in V $ cumple la identidad del paralelogramo: 
$$   \| f+g\|^2 + \| f-g\|^2 = 2\| f \|^2 + 2 \| g\|^2 ,\quad \forall f,g \in V .   $$



  
Un espacio normado no euclideano no necesariamente satisface la propiedad del paralelogramo. Por ejemplo en
 $(C([0, 2\pi], \mathbb C), \|\cdot\|_{\infty}) $   la norma no se cumple la identidad.  Basta con elegir $f(t)= \cos^2(t)$ y 
 $g(t)= \sin^2(t)$. 
 

Sea   $(V, \langle \cdot,  \cdot\rangle)$ espacio euclideano. Una familia $F=\{ x_{\alpha}\}_{\alpha\in I}$ de elementos de $V$ se dir\'a   ortogonal  si   $  \alpha \neq  \beta$ entonces 
$\langle x_{\alpha},x_{\beta}\rangle=0$. Adem\'as,   $F $ se dir\'a   ortogonormal  
$\langle x_{\alpha},x_{\beta}\rangle=\delta_{\alpha,\beta}$, donde $\delta_{\alpha,\beta}$ es la delta de Kronecker, i.e., 
$$\delta_{\alpha,\beta}  = \left\{ \begin{array}{ll} 1, & \alpha =\beta, \\  0, & \alpha \neq \beta.    \end{array}\right. $$
Una familia de elementos  $G=\{ y_{\alpha}\}_{\alpha\in I}$ de elementos de $V$ se dir\'a   biortogonal  a la familia $F$ si   $  \alpha \neq  \beta$ implica  
$\langle x_{\alpha},y_{\beta}\rangle=0$.  Adm\'as, si $\langle x_{\alpha},y_{\beta}\rangle=\delta_{\alpha,  \beta}$
 entonces  $G $ se dir\'a   biortonormal   a   $F$. 




\item Un espacio $\mathbb K$-lineal con producto interno que es de Banach con la norma inducida por el producto interno es llamado $\mathbb K$ espacio de Hilbert.

 


 



\end{enumerate}

\newcommand{\abs}[1]{\ensuremath{\left|#1\right|}}

%Notas: cambié unas cosas de orden para que no haya tanto problema

\section{Sobre los espacios $L_p$ y $\ell_p$}
    \begin{enumerate}
    \item Considere una funci\'on medible $f:[0,1]\to \mathbb C$ y $p>1$ diremos que $f\in L_p(0,1)$, si
    \begin{equation*}
        \int_{0}^1 |f(t)|^p dt < \infty
    \end{equation*}
    Si $f,g\in L_p(0,1)$ se verifica directamente que 
    \begin{equation*}
        |f(t) + g(t)|^p \leq 2^p (|f(t) |^p + | g(t)|^p) , \quad \forall  t\in [0,1]
    \end{equation*}

    (\textcolor{red}{Mostrar})

    \begin{proof}
        Sean $f,g\in L_p(0,1)$, se tiene que
        \begin{equation*}
            \begin{split}
                \abs{f(t)+g(t)}&\leq(\abs{f(t)}+\abs{g(t)})\\
                &\leq2\max\left\{\abs{f(t)},\abs{g(t)}\right\}\\
            \end{split}
        \end{equation*}
        como la funci\'on $x\mapsto x^p$ es creciente en $[0,\infty)$, se sigue que:
        \begin{equation*}
            \begin{split}
                \abs{f(t)+g(t)}^p&\leq 2^p\max\left\{\abs{f(t)}^p,\abs{g(t)}^p\right\}\\
                &\leq 2^p\max\left\{\abs{f(t)}^p,\abs{g(t)}^p\right\}+2^p\min\left\{\abs{f(t)}^p,\abs{g(t)}^p\right\}\\
                &=2^p\left(\abs{f(t)}^p+\abs{g(t)}^p\right),\quad\forall t\in[0,1] \\
            \end{split}
        \end{equation*}
    \end{proof}
    Luego $f+g\in L_p(0,1)$.

    \begin{lem}
        Sean $a,b\in\mathbb{R}_{\geq0}$ y $\alpha,\beta>0$ tales que $\alpha+\beta=1$, entonces:
        \begin{equation*}
            ab\leq \alpha a^{1/\alpha}+\beta b^{1/\beta}
        \end{equation*}
    \end{lem}

    \begin{proof}
        Considere la funci\'on $x\mapsto \ln(x)$. Esta funci\'on es c\'oncava, por lo cual:
        \begin{equation*}
            \ln(\alpha a^{\frac{1}{\alpha}}+\beta b^{1/\beta})\geq\ln(a)+\ln(b)=\ln(ab)
        \end{equation*}
        y, como esta funci\'on es creciente se sigue que:
        \begin{equation*}
            ab\leq\alpha a^{\frac{1}{\alpha}}+\beta b^{1/\beta}
        \end{equation*}
    \end{proof}

    Si  $f \in L_p(0,1)$ y $g\in L_q(0,1)$ con $p,q>0$ y $\frac{1}{p}+ \frac{1}{q} =1$  ($p$ y $q$ se llaman n\'umeros conjugados) se muestra la desigualdad de  H\"older: 
    \begin{equation*}
        \int_0^1 |f(t) g(t)| dt \leq   \left(  \int_0^1 |f(t)|^p dt\right) ^{\frac{1}{p}}\left(  \int_0^1 |g(t)|^q dt\right) ^{\frac{1}{q}}
    \end{equation*}

    (\textcolor{red}{Mostrar})

    \begin{proof}
        Si $f=0$ c.t.p. o $g=0$ c.t.p. la desigualdad se tiene de forma inmediata. Supongamos que no sucede tal cosa, entonces se tiene que:
        \begin{equation*}
            \left(\int_0^1|f(t)|^p dt\right) ^{\frac{1}{p}}>0\quad\textup{y}\quad\left(\int_0^1 |g(t)|^q dt\right) ^{\frac{1}{q}}>0
        \end{equation*}
        Si $t\in[0,1]$ se tiene usando el Lema anterior tomando:
        \begin{equation*}
            a(t)=\frac{\abs{f(t)}}{\left(\int_0^1|f(t)|^p dt\right) ^{\frac{1}{p}}}\quad\textup{y}\quad b(t)=\frac{\abs{g(t)}}{\left(\int_0^1 |g(t)|^q dt\right) ^{\frac{1}{q}}}
        \end{equation*}
        y, $\alpha=\frac{1}{p}$ y $\beta=\frac{1}{q}$, que:
        \begin{equation*}
            a(t)b(t)\leq\alpha a(t)^{\frac{1}{\alpha}}+\beta b(t)^{\frac{1}{\beta}},\quad\forall t\in[0,1]
        \end{equation*}
        es decir:
        \begin{equation*}
            \frac{\abs{f(t)g(t)}}{\left(\int_0^1|f(t)|^p dt\right) ^{\frac{1}{p}}\left(\int_0^1 |g(t)|^q dt\right) ^{\frac{1}{q}}}\leq\frac{1}{p}\cdot\frac{\abs{f(t)}^{\frac{1}{p}}}{\int_0^1|f(t)|^p dt}+\frac{1}{1}\cdot\frac{\abs{g(t)}^{\frac{1}{q}}}{\int_0^1|g(t)|^q dt}
        \end{equation*}
        para todo $t\in[0,1]$. Integrando ambos lados de 0 a 1 se sigue que:
        \begin{equation*}
            \begin{split}
                \frac{\int_{0}^{1}\abs{f(t)g(t)dt}}{\left(\int_0^1|f(t)|^p dt\right) ^{\frac{1}{p}}\left(\int_0^1 |g(t)|^q dt\right) ^{\frac{1}{q}}}&\leq\frac{1}{p}\cdot\frac{\int_0^1|f(t)|^p dt}{\int_0^1|f(t)|^p dt}+\frac{1}{q}\cdot\frac{\int_0^1|g(t)|^q dt}{\int_0^1|g(t)|^q dt}\\
                \Rightarrow \frac{\int_{0}^{1}\abs{f(t)g(t)dt}}{\left(\int_0^1|f(t)|^p dt\right) ^{\frac{1}{p}}\left(\int_0^1 |g(t)|^q dt\right) ^{\frac{1}{q}}}&\leq\frac{1}{p}+\frac{1}{q}=1\\
                \Rightarrow\int_{0}^{1}\abs{f(t)g(t)dt}&\leq\left(\int_0^1|f(t)|^p dt\right) ^{\frac{1}{p}}\left(\int_0^1 |g(t)|^q dt\right) ^{\frac{1}{q}}\\
            \end{split}
        \end{equation*}
    \end{proof}

    Si $p=q=2$ la desigualdad anterior es llamada desigualdad de Schwarz en $L_2(0,1)$.

    Se tiene la desigualdad de Minkowski para $f,g\in L_p(0,1)$: 
    \begin{equation*}
        \left(\int_0^1 |f(t) + g(t)|^p dt \right)^{\frac{1}{p}}\leq   \left(  \int_0^1 |f(t)|^p dt\right) ^{\frac{1}{p}} + \left(  \int_0^1 |g(t)|^p dt\right) ^{\frac{1}{p}}
    \end{equation*}

    (\textcolor{red}{Mostrar})

    \begin{proof}
        El resultado se tiene de forma inmediata si $\int_{0}^{1}\abs{f(t)+g(t)}^pdt=0$. Suponga lo contrario, entonces $\left(\int_{0}^{1}\abs{f(t)+g(t)}^pdt \right)^{\frac{1}{p}}>0$. 
        
        Se tiene entonces que:
        \begin{equation*}
            \begin{split}
                \int_{0}^{1}\abs{f(t)+g(t)}^pdt&\leq\int_{0}^{1}\abs{f(t)+g(t)}^{p-1}(\abs{f(t)}+\abs{g(t)})dt\\
                &\leq\int_{0}^{1}\abs{f(t)+g(t)}^{p-1}\abs{f(t)}dt+\int_{0}^{1}\abs{f(t)+g(t)}^{p-1}\abs{g(t)}dt\\
            \end{split}
        \end{equation*}

        Como $p>1$ entonces existe $q>1$ tal que $\frac{1}{p}+\frac{1}{q}=1$, en particular $pq-q=p$. Se tiene as\'i que $\abs{f+g}^{p-1}\in L_{q}(0,1)$ pues:
        \begin{equation*}
            \begin{split}
                (\abs{f+g}^{p-1})^q&=\abs{f+g}^{pq-q}\\
                &=\abs{f+g}^p\\
                \Rightarrow \int_{0}^{1}(\abs{f(t)+g(t)}^{p-1})^q&=\int_{0}^{1}\abs{f(t)+g(t)}^p
            \end{split}
        \end{equation*}
        Luego, por la desigualdad de H\"older:
        \begin{equation*}
            \begin{split}
                \int_{0}^{1}\abs{f(t)+g(t)}^{p-1}\abs{f(t)}dt&\leq\left(\int_{0}^{1}(\abs{f(t)+g(t)}^{p-1})^qdt\right)^{\frac{1}{q}}\left(\int_{0}^{1}\abs{f(t)}^p dt\right)^{\frac{1}{p}}\\
                &=\left(\int_{0}^{1}\abs{f(t)+g(t)}^pdt \right)^{\frac{1}{q}}\left(\int_{0}^{1}\abs{f(t)}^p dt\right)^{\frac{1}{p}}\\
            \end{split}
        \end{equation*}
        y, de forma an\'aloga:
        \begin{equation*}
            \begin{split}
                \int_{0}^{1}\abs{f(t)+g(t)}^{p-1}\abs{g(t)}dt&\leq\left(\int_{0}^{1}\abs{f(t)+g(t)}^pdt \right)^{\frac{1}{q}}\left(\int_{0}^{1}\abs{g(t)}^p dt\right)^{\frac{1}{p}}\\
            \end{split}
        \end{equation*}
        Por tanto:
        \begin{equation*}
            \begin{split}
                \int_{0}^{1}\abs{f(t)+g(t)}^pdt&\\
                \leq\left(\int_{0}^{1}\abs{f(t)+g(t)}^pdt \right)^{\frac{1}{q}}&\left(\left(\int_{0}^{1}\abs{f(t)}^p dt\right)^{\frac{1}{p}}+\left(\int_{0}^{1}\abs{g(t)}^p dt\right)^{\frac{1}{p}}\right)\\
                \Rightarrow \left(\int_{0}^{1}\abs{f(t)+g(t)}^pdt\right)^{1-\frac{1}{q}}&=\left(\int_{0}^{1}\abs{f(t)}^p dt\right)^{\frac{1}{p}}+\left(\int_{0}^{1}\abs{g(t)}^p dt\right)^{\frac{1}{p}}\\
                \Rightarrow \left(\int_{0}^{1}\abs{f(t)+g(t)}^pdt\right)^{\frac{1}{p}}&=\left(\int_{0}^{1}\abs{f(t)}^p dt\right)^{\frac{1}{p}}+\left(\int_{0}^{1}\abs{g(t)}^p dt\right)^{\frac{1}{p}}\\
            \end{split}
        \end{equation*}
    \end{proof}

    Con lo anterior vemos que $ L_p(0,1)$ es un espacio lineal complejo y que definiendo la relaci\'on de equivalencia $f \sim g$ sii $f=g$ c.t.p. para cada $f,g \in   L_p(0,1)$ tendremos que el mapeo 
    \begin{equation*}
        [f]\mapsto \left(  \int_0^1 |f(t)|^p dt\right) ^{\frac{1}{p}}=N_p(f)
    \end{equation*}
    est\'a bien definido en $  L_p(0,1) \diagup \sim $ y m\'as a\'un que es una norma. Luego
    \begin{equation*}
        \rho([f],[g])=  \left(  \int_0^1 |f(t)- g(t)|^p dt\right) ^{\frac{1}{p}}
    \end{equation*}
    es una m\'etrica en  $ L_p(0,1) \diagup  \sim$

    \begin{lem}
        Sea $p>1$ y $\varepsilon>0$. Si $f\in L_p(0,1)$ es tal que $(\int_{0}^{1}\abs{f(t)}^pdt)^{\frac{1}{p}}<\varepsilon^{\frac{2}{p}}$, entonces el conjunto:
        \begin{equation*}
            A=\left\{x\in[0,1]\Big|\abs{f(x)}>\varepsilon^{\frac{1}{p}} \right\}
        \end{equation*}
        tiene medida menor o igual a $\varepsilon$.
    \end{lem}

    \begin{proof}
        Veamos que:
        \begin{equation*}
            \varepsilon^2>\int_{0}^{1}\abs{f(t)}^pdt\geq\int_{A}\abs{f(t)}^pdt\geq\varepsilon\cdot \mu(A)
        \end{equation*}
        por ende, $\mu(A)<\varepsilon$.
    \end{proof}

    \begin{lem}
        Sea $p>1$. Si $\left(f_n \right)_{ n=1}^\infty$ es una sucesi\'on de Cauchy en $L_p(0,1)$, entonces existen una funci\'on creciente $\alpha$ de $\mathbb{N}$ en $\mathbb{N}$ y una funci\'on $f$ de $[0,1]$ en $\mathbb{C}$ tal que $\left(f_{\alpha(n)} \right)_{ n=1}^\infty$ converge c.t.p. a $f$.
    \end{lem}

    \begin{proof}
        Como la sucesi\'on $\left(f_{n} \right)_{ n=1}^\infty$ es de Cauchy en $L_p(0,1)$, entonces existe una funci\'on creciente $\alpha$ de $\mathbb{N}$ en $\mathbb{N}$ tal que:
        \begin{equation*}
            N_p(f_{\alpha(n+1)}-f_{\alpha(n)})<\left(\frac{1}{2^n}\right)^{\frac{2}{p}},\quad\forall n\in\mathbb{N}
        \end{equation*}
        Sea:
        \begin{equation*}
            A_n=\left\{t\in[0,1]\Big|\abs{f_{\alpha(n+1)}(t)-f_{\alpha(n)}(t)}\geq\left(\frac{1}{2^n}\right)^{\frac{1}{p}} \right\},\quad\forall n\in\mathbb{N}
        \end{equation*}
        por el Lema anterior se tiene que $\mu(A_n)<\frac{1}{2^n}$ para todo $n\in\mathbb{N}$. Entonces:
        \begin{equation*}
            \abs{f_{\alpha(n+1)}(t)-f_{\alpha(n)}(t)}<\left(\frac{1}{2^n}\right)^{\frac{1}{p}},\quad\forall t\in[0,1]\setminus A_n,\forall n\in\mathbb{N}
        \end{equation*}
        Tomemos $M_n=\bigcup_{k=n}^\infty A_k$, entonces $\mu(M_n)\leq\frac{1}{2^{n-1}}$ para todo $n\in\mathbb{N}$, luego:
        \begin{equation*}
            \abs{f_{\alpha(n+1)}(t)-f_{\alpha(n)}(t)}<\left(\frac{1}{2^n}\right)^{\frac{1}{p}},\quad\forall t\in[0,1]\setminus M_k,\forall n\geq k
        \end{equation*}
        para todo $k\in\mathbb{N}$. Como:
        \begin{equation*}
            \sum_{ n=1}^\infty\left(\frac{1}{2^n}\right)^{\frac{1}{p}}<\infty
        \end{equation*}
        se sigue por el Criterio $M$ de Weierestrass que la serie:
        \begin{equation*}
            \sum_{ n=1}^\infty\abs{f_{\alpha(n+1)}-f_{\alpha(n)}}
        \end{equation*}
        converge puntualmente (pues converge uniformemente) en $[0,1]\setminus M_k$. Como $\mathbb{C}$ es de Banach entonces la serie:
        \begin{equation*}
            \sum_{ n=1}^\infty f_{\alpha(n+1)}-f_{\alpha(n)}
        \end{equation*}
        converge puntualmente en $[0,1]\setminus M_k$. Luego, como:
        \begin{equation*}
            f_{\alpha(n)}=f_{\alpha(1)}+\sum_{i=1}^{n-1}[f_{\alpha( i+1)}-f_{\alpha(i)}],\quad\forall n\in\mathbb{N}
        \end{equation*}
        converge puntualmente en $[0,1]\setminus M_k$ a una funci\'on $g_k$ de $[0,1]\setminus M_k$ en $\mathbb{C}$. Como:
        \begin{equation*}
            M_{ k+1}\subseteq M_k,\quad\forall k\in\mathbb{N}
        \end{equation*}
        y $\mu(M_k)<\frac{1}{2^{ k-1}}$ para todo $k\in\mathbb{N}$, entonces existe una funci\'on $f$ de $[0,1]$ en $\mathbb{C}$ tal que:
        \begin{equation*}
            f\Big|_{[0,1]\setminus M_k}=g_k,\quad\forall k\in\mathbb{N}
        \end{equation*}
        as\i que $\left(f_{\alpha(n)} \right)_{ n=1}^\infty$ converge a $f$ c.t.p.
    \end{proof}

    Se tiene que el espacio $L_p(0,1)$ es completo.

    (\textcolor{red}{Mostrar})

    \begin{proof}
        Sea $\left(f_{n} \right)_{ n=1}^\infty$ una sucesi\'on de Cauchy en $L_p(0,1)$. Por el Lema anterior existe una funci\'on creciente $\alpha$ de $\mathbb{N}$ en $\mathbb{N}$. y  una funci\'on $f$ de $[0,1]$ en $\mathbb{C}$ tal que $\left(f_{\alpha(n)} \right)_{ n=1}^\infty$ converge a $f$ c.t.p. Sea $\varepsilon>0$, entonces existe $N\in\mathbb{N}$ tal que:
        \begin{equation*}
            m,n\geq N\Rightarrow \int_{0}^{1}\abs{f_n-f_m}^p<\varepsilon^{p}
        \end{equation*}
        como $\alpha(m)\geq m$ para todo $m\in\mathbb{N}$, entonces:
        \begin{equation*}
            m,n\geq N\Rightarrow \int_{0}^{1}\abs{f_n-f_{\alpha(m)}}^p<\varepsilon^{p}
        \end{equation*}
        adem\'as, se tiene para $n\geq N$ fijo:
        \begin{equation*}
            \lim_{ m\rightarrow\infty}\abs{f_n-f_{\alpha(m)}}^p=\abs{f_n-f}^p\textup{ c.t.p. en }[0,1]
        \end{equation*}
        por el Lema de Fatou se sigue que:
        \begin{equation*}
            \int_{0}^{1}\abs{f_n-f}^p\leq\liminf_{ m\rightarrow\infty}\int_{0}^{1}\abs{f_n-f_{\alpha(m)}}^p\leq\varepsilon^p
        \end{equation*}
        Por tanto:
        \begin{equation*}
            \left(\int_{0}^{1}\abs{f_n-f}^p\right)^{\frac{1}{p}}\leq\varepsilon
        \end{equation*}
        as\'i que $n\geq N$ implica que $N_p(f_n-f)\leq\varepsilon$. Por tanto, la sucesi\'on de Cauchy converge a $f$ en $L_p(0,1)$ y se tiene que $f_1-f\in L_p(0,1)$, como $f_1\in L_p(0,1)$ se sigue que $f_1\in L_p(0,1)$. Por tanto, $L_p(0,1)$ es de Banach.
    \end{proof}

    \item Por $\ell_p$ entendemos el espacio  de sucesiones de n\'umeros complejos $(z_n)_{n\in \mathbb N}$ tal que 
    \begin{equation*}
        \sum_{n=1}^{ \infty} |z_n|^{p} < \infty.
    \end{equation*}

    Si $(z_n)_{n\in \mathbb N} \in \ell_p$ y $(w_n)_{n\in \mathbb N} \in \ell_q$ con $p,q>0$ y $\frac{1}{p}+ \frac{1}{q} =1$  se muestra la desigualdad de  H\"older: 
    \begin{equation*}
        \sum_{n=1}^{ \infty} |z_n w_n|  \leq   \left( 
        \sum_{n=1}^{ \infty} |z_n|^{p}
        \right) ^{\frac{1}{p}}\left(  \sum_{n=1}^{ \infty} |w_n|^{q}  \right) ^{\frac{1}{q}}
    \end{equation*}
    Si $p=q=2$ la desigualdad anterior es llamada desigualdad de Schwarz en $\ell_2$.

    (\textcolor{red}{Mostrar})

    \begin{proof}
        Se tienen tres casos:
        \begin{enumerate}
            \item Alguna de las dos sucesiones $(z_n)_{n\in \mathbb N}$ y $(w_n)_{n\in \mathbb N}$ es cero, en cuyo caso se tiene la desigualdad de forma inmediata.
            \item Las sucesiones $(z_n)_{n\in \mathbb N}$ y $(w_n)_{n\in \mathbb N}$ son ambas no cero tales que:
            \begin{equation*}
                \left( \sum_{n=1}^{ \infty} |z_n|^{p}\right) ^{\frac{1}{p}}=\left(\sum_{n=1}^{\infty} |w_n|^{q}\right)^{\frac{1}{q}}=1
            \end{equation*}
            Por el Lema (4.1) para todo $n\in\mathbb{N}$ tomando $a(n)=\abs{z_n}$, $b(n)=\abs{w_n}$, $\alpha=\frac{1}{p}$ y $\beta=\frac{1}{q}$ se tiene que:
            \begin{equation*}
                \begin{split}
                    \abs{z_nw_n}&\leq\frac{1}{p}\abs{z_n}^{p}+\frac{1}{q}\abs{w_n}^{q},\quad\forall n\in\mathbb{N}\\
                    \Rightarrow\sum_{ n=1}^\infty\abs{z_nw_n}&\leq\frac{1}{p}\sum_{ n=1}^\infty\abs{z_n}^p+\frac{1}{q}\sum_{ n=1}^\infty\abs{w_n}^q\\
                    &=\frac{1}{p}+\frac{1}{q}\\
                    &=1\\
                    &=\left( \sum_{n=1}^{ \infty} |z_n|^{p}\right) ^{\frac{1}{p}}\left(\sum_{n=1}^{\infty} |w_n|^{q}\right)^{\frac{1}{q}}\\
                \end{split}
            \end{equation*}
            \item Las sucesiones $(z_n)_{n\in\mathbb{N}},(w_n)_{n\in\mathbb{N}}$ son ambas no cero. Tomemos las sucesiones:
            \begin{equation*}
                (x_n)_{n\in\mathbb{N}}=\left(\frac{z_n}{\left(\sum_{ k=1}^\infty\abs{z_k}^p\right)^{\frac{1}{p}}} \right)_{ n\in\mathbb{N}}\quad\textup{y}\quad(y_n)_{n\in\mathbb{N}}=\left(\frac{w_n}{\left(\sum_{ k=1}^\infty\abs{w_k}^q\right)^{\frac{1}{q}}} \right)_{ n\in\mathbb{N}}
            \end{equation*}
            Entonces se tiene de forma inmediata que:
            \begin{equation*}
                \sum_{ n=1}^\infty\abs{x_n}^p=\sum_{ n=1}^\infty\abs{y_n}^p=1
            \end{equation*}
            por tanto $(x_n)_{n\in\mathbb{N}}\in\ell_p$ y $(y_n)_{n\in\mathbb{N}}\in\ell_q$, de la parte anterior se tiene que:
            \begin{equation*}
                \begin{split}
                    \sum_{ n=1}^\infty\abs{x_ny_n}&\leq1\\
                    \Rightarrow\sum_{ n=1}^\infty\frac{\abs{z_nw_n}}{\left(\sum_{ k=1}^\infty\abs{z_k}^p\right)^{\frac{1}{p}}\left(\sum_{ k=1}^\infty\abs{w_k}^q\right)^{\frac{1}{q}}}&\leq1\\
                    \Rightarrow \sum_{ n=1}^\infty\abs{z_nw_n}&\leq\left(\sum_{ k=1}^\infty\abs{z_k}^p\right)^{\frac{1}{p}}\left(\sum_{ k=1}^\infty\abs{w_k}^q\right)^{\frac{1}{q}}\\
                \end{split}
            \end{equation*}
            lo cual prueba el resultado.
        \end{enumerate}
    \end{proof}

    La desigualdad de Minkowski para $p>1$:
    \begin{equation*}
        \left(\sum_{n=1}^{ \infty} |z_n+w_n|^{p} \right)^{\frac{1}{p}}\leq   \left( \sum_{n=1}^{ \infty} |z_n|^{p}\right) ^{\frac{1}{p}} + \left(\sum_{n=1}^{ \infty} |w_n|^{p}\right) ^{\frac{1}{p}}
    \end{equation*}

    (\textcolor{red}{Mostrar})

    \begin{proof}
        Se tiene que:
        \begin{equation*}
            \begin{split}
                \abs{z_n+w_n}&\leq\left(\abs{z_n}+\abs{w_n}\right)^p\\
                &\leq\left(2\max\left\{\abs{z_n},\abs{w_n} \right\} \right)^p\\
                &\leq2^p\max\left\{\abs{z_n}^p,\abs{w_n}^p \right\}+2^p\min\left\{\abs{z_n}^p,\abs{w_n}^p \right\}\\
                &=2^p\left(\abs{z_n}^p+\abs{w_n}^p \right),\quad\forall n\in\mathbb{N} \\
            \end{split}
        \end{equation*}
        por tanto:
        \begin{equation*}
            \sum_{ n=1}^\infty\abs{z_n+w_n}^p\leq 2^p\left(\sum_{ n=1}^\infty\abs{z_n}^p+\sum_{ n=1}^\infty\abs{w_n}^p \right)<\infty
        \end{equation*}
        as\'i que $(z_n+w_n)_{ n\in\mathbb{N}}\in\ell_p$. La desigualdad se tiene de forma inmediata si la sucesi\'on $(z_n+w_n)_{ n\in\mathbb{N}}$ es cero, así que supongamos que no lo es. Se tiene que:
        \begin{equation*}
            \begin{split}
                \abs{z_n+w_n}^p&\leq\abs{z_n+w_n}^{p-1}\left(\abs{z_n}+\abs{w_n}\right)\\
                &=\abs{z_n+w_n}^{p-1}\abs{z_n}+\abs{z_n+w_n}^{p-1}\abs{w_n},\quad\forall n\in\mathbb{N}\\
                \Rightarrow\sum_{ n=1}^\infty\abs{z_n+w_n}^p&\leq \sum_{ n=1}^\infty\abs{z_n+w_n}^{p-1}\abs{z_n}+\sum_{ n=1}^\infty\abs{z_n+w_n}^{p-1}\abs{w_n}
            \end{split}
        \end{equation*}
        Como $(\abs{z_n+w_n}^{ p-1})_{ n=1}^\infty\in\ell_{\frac{p}{p-1}}$ y:
        \begin{equation*}
            \frac{1}{p}+\frac{1}{\frac{p}{p-1}}=1
        \end{equation*}
        se sigue por H\"older que:
        \begin{equation*}
            \begin{split}
                \sum_{ n=1}^\infty\abs{z_n+w_n}^{p-1}\abs{z_n}&\leq\left(\sum_{ n=1}^\infty\left(\abs{z_n+w_n}^{p-1}\right)^{\frac{p}{ p-1}} \right)^{\frac{p-1}{p}}\left(\sum_{ n=1}^\infty\abs{z_n}^p \right)^{\frac{1}{p}}\\
                &=\left(\sum_{ n=1}^\infty\abs{z_n+w_n}^{p} \right)^{1-\frac{1}{p}}\left(\sum_{ n=1}^\infty\abs{z_n}^p \right)^{\frac{1}{p}}\\
            \end{split}
        \end{equation*}
        de forma an\'aloga:
        \begin{equation*}
            \sum_{ n=1}^\infty\abs{z_n+w_n}^{p-1}\abs{w_n}\leq\left(\sum_{ n=1}^\infty\abs{z_n+w_n}^{p} \right)^{1-\frac{1}{p}}\left(\sum_{ n=1}^\infty\abs{w_n}^p \right)^{\frac{1}{p}}
        \end{equation*}
        Por tanto:
        \begin{equation*}
            \sum_{ n=1}^\infty\abs{z_n+w_n}^p\leq\left(\sum_{ n=1}^\infty\abs{z_n+w_n}^{p} \right)^{1-\frac{1}{p}}\left(\left(\sum_{ n=1}^\infty\abs{z_n}^p \right)^{\frac{1}{p}}+\left(\sum_{ n=1}^\infty\abs{w_n}^p \right)^{\frac{1}{p}} \right)
        \end{equation*}
        lo cual implica por ser $(z_n+w_n)_{ n\in\mathbb{N}}$ no cero que:
        \begin{equation*}
            \left(\sum_{ n=1}^\infty\abs{z_n+w_n}^{p} \right)^{\frac{1}{p}}\leq\left(\sum_{ n=1}^\infty\abs{z_n}^p \right)^{\frac{1}{p}}+\left(\sum_{ n=1}^\infty\abs{w_n}^p \right)^{\frac{1}{p}}
        \end{equation*}
    \end{proof}

    Con lo anterior vemos que $ \ell_p$ es un espacio lineal complejo y que 
    \begin{equation*}
        z=(z_n)_{n\in \mathbb N}\mapsto 
    \left( 
    \sum_{n=1}^{ \infty} |z_n|^{p}
    \right) ^{\frac{1}{p}}=N_p(z)$$ es una norma en $ \ell_p$. Luego 
    $$ \rho((z_n),(w_n))=  \left(    
    \sum_{n=1}^{ \infty} |z_n-w_n|^{p}
    \right) ^{\frac{1}{p}}
    \end{equation*}
    
    es una m\'etrica en $\ell_p$. Se tiene que $\ell_p$ es completo
    (\textcolor{red}{Mostrar})

    \begin{proof}
        Sea $\left(\zeta_n \right)_{ n\in\mathbb{N}}=\left(\left(z_k^n \right)_{ k\in\mathbb{N}} \right)_{ n\in\mathbb{N}}$ una sucesi\'on de Cauchy en $\ell_p$, es decir que para todo $\varepsilon>0$ existe $N\in\mathbb{N}$ tal que:
        \begin{equation*}
            n,m\geq N\Rightarrow N_p(\zeta_n-\zeta_m)=\left(\sum_{ k=1}^\infty\abs{z_k^n-z_k^m}^p\right)^{\frac{1}{p}} <\varepsilon
        \end{equation*}
        Se tiene que la sucesi\'on en $\mathbb{C}$, $\left(z_k^n \right)_{ n\in\mathbb{N}}$ es de Cauchy, para todo $k\in\mathbb{N}$, pues por lo anterior:
        \begin{equation*}
            n,m\geq N\Rightarrow\abs{z_k^n-z_k^m}<\varepsilon,\quad\forall k\in\mathbb{N}
        \end{equation*}
        Por ser $\mathbb{C}$ completo, para todo $k\in\mathbb{N}$ existe $w_k\in\mathbb{C}$ tal que:
        \begin{equation*}
            \lim_{n\rightarrow\infty}z_k^n=w_k
        \end{equation*}
        Afirmamos que $\left( y_k\right)_{ k\in\mathbb{N}}$ est\'a en $\ell_p$ y que $\left(\zeta_n \right)_{ n\in\mathbb{N}}$ converge a esta en $\ell_p$. Sea $\varepsilon>0$, entonces existe $N\in\mathbb{N}$ tal que:
        \begin{equation*}
            n,m\geq N\Rightarrow\sum_{ k=1}^\infty\abs{z_k^n-z_k^m}^p<\varepsilon^p
        \end{equation*}
        En particular para $M\in\mathbb{N}$ fijo:
        \begin{equation*}
            n,m\geq N\Rightarrow\sum_{ k=1}^M\abs{z_k^n-z_k^m}^p<\varepsilon^p
        \end{equation*}
        fijando $n$ y tomando l\'imite cuando $m\rightarrow\infty$ se obtiene que:
        \begin{equation*}
            n\geq N\Rightarrow\sum_{ k=1}^M\abs{z_k^n-w_k}^p\leq\varepsilon^p
        \end{equation*}
        y, como el $M\in\mathbb{N}$ fue arbitrario, se sigue que al ser la serie no decreciente:
        \begin{equation*}
            n\geq N\Rightarrow\sum_{ k=1}^\infty\abs{z_k^n-w_k}^p\leq\varepsilon^p
        \end{equation*}
        as\'i que:
        \begin{equation*}
            n\geq N\Rightarrow N_p(\zeta_n-\left(w_k\right)_{ k\in\mathbb{N}})\leq\varepsilon
        \end{equation*}
        con lo que la sucesi\'on $\left(\zeta_n\right)_{ n\in\mathbb{N}}$ converge a $\left(w_k\right)_{ k\in\mathbb{N}}$ en $\ell_p$, adem\'as $\left(w_k\right)_{ k\in\mathbb{N}}\in\ell_p$ ya que en particular:
        \begin{equation*}
            N_p(\zeta_n-\left(w_k\right)_{ k\in\mathbb{N}})<\varepsilon
        \end{equation*}
        se tiene que $\zeta_n\in\ell_p$ y por lo anterior $\zeta_n-\left(w_k\right)_{ k\in\mathbb{N}}\in\ell_p$, por lo cual $\left(w_k\right)_{ k\in\mathbb{N}}\in\ell_p$.

        Se sigue que $\ell_p$ es de Banach.
    \end{proof}

\end{enumerate}
 








  








 




\begin{ejem}  {} \ 
\begin{enumerate}
\item Considere el espacio de Bergman  en el disco unitario: $\textrm{Hol}(\mathbb D) \cap L_2(\mathbb D, \mathbb C)$.
  
Note que si $n,m\in \mathbb N\cup \{0\}$ con $n< m$  tendremos que 
\begin{align*}
 \langle z^n, z^m \rangle= & \int_{ \mathbb B(0,1)} w^n \overline{ (w^m)} dw=  \int_{\partial \mathbb B(0,1)}
  \|w\|^{2n} (\overline{w})^{m-n} dw \\
= &   \int_{0}^{2\pi} \int_{0}^1  
  \|r e^{i\theta   }\|^{2n} (\overline{r e^{i\theta   }})^{m-n} rdr d\theta   =  \int_{0}^{2\pi}    \int_{0}^1  
   r^{ m+ n +1 }   e^{- i\theta(m-n)   }   dr d\theta  \\
= & \frac{1  }{ m+ n +2  }  \int_{0}^{2\pi}    
     e^{- i\theta(m-n)   }    d\theta  =0 \\
\end{align*} 
As\'i que la familia $(z^n)_{n\geq 0}$ es ortogonal y que la familia $ ( \dfrac{  \sqrt{  n +1}   }{  \sqrt{\pi}  }    z^n )_{n\geq 0}  $ es una familia ortonormal.  Ya que  
\begin{align*}
\|z^n \| = \sqrt{  \langle z^n, z^n \rangle} =  \sqrt{ \frac{ 1 }{2n +2  }  2\pi} =  \frac{   \sqrt{\pi} }{  \sqrt{  n +1}  }  ,
\end{align*} 
 para cada $n\geq 0$.

\item     Si $(x_n)_{n\in  I \subset \mathbb N}$ es una base del espacio euclideano $(V, \langle \cdot,  \cdot\rangle)$
y la familia $(y_n)_{n\in  I}$ es biortonormal a la base $(x_n)_{n\in  I}$ entonces podemos calcular de manera directa los coeficientes en que cada elemento de $V$ se escribe en t\'erminos de la base como se muestra:
\begin{align*}
 \langle x, y_k \rangle=  \langle \sum_{n\in I} c_n x_n , y_k \rangle =  c_k \langle   x_k , y_k \rangle = c_k. 
\end{align*} 
para cada $k\in I$. Por lo tanto,  $$x= \sum_{n\in I} c_n x_n  =   \sum_{n\in I}  \langle x, y_n \rangle x_n   .$$

\item Empleando biortogonalidad,  una  parte del concepto comentado en 2, en el espacio de Bergman comentado en 1. Tendremos que todo elemento del espacio  de Bergman se escribe como 
$$f(z) = \sum_{n=0}^{\infty } a_n z^n ,$$
Entonces  $\langle f, z^k \rangle = a_k \langle z^k , z^k \rangle = a_k  \dfrac{ \pi }{ k +1  }   $ para cada $k \geq 0$. As\'i que $a_k = \dfrac{ k +1  }{ \pi }   \langle f, z^k \rangle  $ para cada $k$.  Continuando con este an\'alisis tendremos
$$f(z) = \sum_{n=0}^{\infty } \dfrac{ n +1  }{ \pi }   \langle f, z^n \rangle   z^n = \sum_{n=0}^{\infty }   \langle f,  
(\dfrac{ \sqrt{n +1}  }{\sqrt {\pi }   } z^n) \rangle   (\dfrac{ \sqrt{n +1}  }{\sqrt {\pi }   } z^n ),$$
que era de esperarse ya que $(\dfrac{ \sqrt{n +1}  }{\sqrt {\pi }   } z^n )_{n\geq 0}$ es una familia ortonormal, o equivalentemente esta familia es biortonormal de s\'i misma.
\end{enumerate}
\end{ejem}

\end{document}