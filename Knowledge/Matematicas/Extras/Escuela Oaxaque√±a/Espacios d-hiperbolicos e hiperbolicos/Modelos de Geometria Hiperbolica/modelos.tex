\documentclass[12pt]{report}
\usepackage[spanish]{babel}
\usepackage[utf8]{inputenc}
\usepackage{amsmath}
\usepackage{amssymb}
\usepackage{amsthm}
\usepackage{graphics}
\usepackage{subfigure}
\usepackage{lipsum}
\usepackage{array}
\usepackage{multicol}
\usepackage{enumerate}
\usepackage[framemethod=TikZ]{mdframed}
\usepackage[a4paper, margin = 1.5cm]{geometry}
\usepackage{tikz}
\usepackage{pgffor}
\usepackage{ifthen}
\usepackage{enumitem}
\usepackage{hyperref}
\usepackage{listings}
\usepackage{bbm}

%Gestión de Hipervínculos

\hypersetup{
    colorlinks=true,
    linkcolor=black,
    filecolor=magenta,      
    urlcolor=cyan
}

%Gestión de Código de Programación

\definecolor{listing-background}{HTML}{F7F7F7}
\definecolor{listing-rule}{HTML}{B3B2B3}
\definecolor{listing-numbers}{HTML}{B3B2B3}
\definecolor{listing-text-color}{HTML}{000000}
\definecolor{listing-keyword}{HTML}{435489}
\definecolor{listing-keyword-2}{HTML}{1284CA} % additional keywords
\definecolor{listing-keyword-3}{HTML}{9137CB} % additional keywords
\definecolor{listing-identifier}{HTML}{435489}
\definecolor{listing-string}{HTML}{00999A}
\definecolor{listing-comment}{HTML}{8E8E8E}

\lstdefinestyle{myStyle}{
    language         = C++,
    alsolanguage     = scala,
    numbers          = left,
    xleftmargin      = 2.7em,
    framexleftmargin = 2.5em,
    backgroundcolor  = \color{gray!15},
    basicstyle       = \color{listing-text-color}\linespread{1.0}\ttfamily,
    breaklines       = true,
    frameshape       = {RYR}{Y}{Y}{RYR},
    rulecolor        = \color{black},
    tabsize          = 2,
    numberstyle      = \color{listing-numbers}\linespread{1.0}\small\ttfamily,
    aboveskip        = 1.0em,
    belowskip        = 0.1em,
    abovecaptionskip = 0em,
    belowcaptionskip = 1.0em,
    keywordstyle     = {\color{listing-keyword}\bfseries},
    keywordstyle     = {[2]\color{listing-keyword-2}\bfseries},
    keywordstyle     = {[3]\color{listing-keyword-3}\bfseries\itshape},
    sensitive        = true,
    identifierstyle  = \color{listing-identifier},
    commentstyle     = \color{listing-comment},
    stringstyle      = \color{listing-string},
    showstringspaces = false,
    label            = lst:bar,
    captionpos       = b,
}

\lstset{style = myStyle}

%Estilo del capítulo y sección

\makeatletter
\def\thickhrulefill{\leavevmode \leaders \hrule height 1ex \hfill \kern \z@}
\def\@makechapterhead#1{%
  {\parindent \z@ \raggedright
    \reset@font
    \hrule
    \vspace*{10\p@}%
    \par
    \center \LARGE \scshape \@chapapp{} \huge \thechapter
    \vspace*{10\p@}%
    \par\nobreak
    \vspace*{10\p@}%
    \par
    \vspace*{1\p@}%
    \hrule
    %\vskip 40\p@
    \vspace*{60\p@}
    \Huge #1\par\nobreak
    \vskip 50\p@
  }}

\def\section#1{%
  \par\bigskip\bigskip
  \hrule\par\nobreak\noindent
  \refstepcounter{section}%
  \addcontentsline{toc}{chapter}{#1}%
  \reset@font
  { \large \scshape
    \strut\S \thesection \quad
    #1}% 
    \hrule   
  \par
  \medskip
}

%Gestión marca de agua

\usetikzlibrary{shapes.multipart}

\newcounter{it}
\newcommand*\watermarktext[1]{\begin{tabular}{c}
    \setcounter{it}{1}%
    \whiledo{\theit<100}{%
    \foreach \col in {0,...,15}{#1\ \ } \\ \\ \\
    \stepcounter{it}%
    }
    \end{tabular}
    }

\AddToHook{shipout/foreground}{
    \begin{tikzpicture}[remember picture,overlay, every text node part/.style={align=center}]
        \node[rectangle,black,rotate=30,scale=2,opacity=0.04] at (current page.center) {\watermarktext{Cristo Daniel Alvarado ESFM\quad}};
  \end{tikzpicture}
}

%En esta parte se hacen redefiniciones de algunos comandos para que resulte agradable el verlos%

\def\proof{\paragraph{Demostración:\\}}
\def\endproof{\hfill$\blacksquare$}

\def\sol{\paragraph{Solución:\\}}
\def\endsol{\hfill$\square$}

%En esta parte se definen los comandos a usar dentro del documento para enlistar%

\newtheoremstyle{largebreak}
  {}% use the default space above
  {}% use the default space below
  {\normalfont}% body font
  {}% indent (0pt)
  {\bfseries}% header font
  {}% punctuation
  {\newline}% break after header
  {}% header spec

\theoremstyle{largebreak}

\newmdtheoremenv[
    leftmargin=0em,
    rightmargin=0em,
    innertopmargin=0pt,
    innerbottommargin=5pt,
    hidealllines = true,
    roundcorner = 5pt,
    backgroundcolor = gray!60!red!30
]{exa}{Ejemplo}[section]

\newmdtheoremenv[
    leftmargin=0em,
    rightmargin=0em,
    innertopmargin=0pt,
    innerbottommargin=5pt,
    hidealllines = true,
    roundcorner = 5pt,
    backgroundcolor = gray!50!blue!30
]{obs}{Observación}[section]

\newmdtheoremenv[
    leftmargin=0em,
    rightmargin=0em,
    innertopmargin=0pt,
    innerbottommargin=5pt,
    rightline = false,
    leftline = false
]{theor}{Teorema}[section]

\newmdtheoremenv[
    leftmargin=0em,
    rightmargin=0em,
    innertopmargin=0pt,
    innerbottommargin=5pt,
    rightline = false,
    leftline = false
]{propo}{Proposición}[section]

\newmdtheoremenv[
    leftmargin=0em,
    rightmargin=0em,
    innertopmargin=0pt,
    innerbottommargin=5pt,
    rightline = false,
    leftline = false
]{cor}{Corolario}[section]

\newmdtheoremenv[
    leftmargin=0em,
    rightmargin=0em,
    innertopmargin=0pt,
    innerbottommargin=5pt,
    rightline = false,
    leftline = false
]{lema}{Lema}[section]

\newmdtheoremenv[
    leftmargin=0em,
    rightmargin=0em,
    innertopmargin=0pt,
    innerbottommargin=5pt,
    roundcorner=5pt,
    backgroundcolor = gray!30,
    hidealllines = true
]{mydef}{Definición}[section]

\newmdtheoremenv[
    leftmargin=0em,
    rightmargin=0em,
    innertopmargin=0pt,
    innerbottommargin=5pt,
    roundcorner=5pt
]{excer}{Ejercicio}[section]

%En esta parte se colocan comandos que definen la forma en la que se van a escribir ciertas funciones%

\newcommand\abs[1]{\ensuremath{\left|#1\right|}}
\newcommand\divides{\ensuremath{\bigm|}}
\newcommand\cf[3]{\ensuremath{#1:#2\rightarrow#3}}
\newcommand\contradiction{\ensuremath{\#_c}}
\newcommand\natint[1]{\ensuremath{\left[\big|#1\big|\right]}}
\newcounter{figcount}
\setcounter{figcount}{1}
\newcommand{\bbm}[1]{\ensuremath{\mathbbm{#1}}}
\newcommand{\pint}[2]{\langle#1\big|#2 \rangle}
\newcommand{\norm}[1]{\|#1\|}
\newcommand{\Isom}[1]{\ensuremath{\textup{Isom}\left(#1\right)}}

\begin{document}
    \setlength{\parskip}{5pt} % Añade 5 puntos de espacio entre párrafos
    \setlength{\parindent}{12pt} % Pone la sangría como me gusta
    \title{Sobre Espacios $\delta$-hiperbólicos y aplicaciones del Teorema de Svarc-Milnor}
    \author{Cristo Daniel Alvarado}
    \maketitle

    \tableofcontents %Con este comando se genera el índice general del libro%

    \newpage

    \chapter{Modelos de geometría hiperbólica}

    \section{Construcción del plano hiperbólico}

    En esta sección se construirá un modelo del plano hiperbólico a partir de una variedad Riemanniana.

    \begin{mydef}[\textbf{Plano superior}]
        Escribimos:
        \begin{equation*}
            H=\left\{(x,y)\in\bbm{R}^2\Big|y>0 \right\}\subseteq\bbm{R}^2
        \end{equation*}
        para el \textbf{plano superior}.
    \end{mydef}

    \begin{obs}
        Dependiendo del contexto, veremos a $H$ como subconjunto de $\bbm{C}$, haciendo las identificaciones:
        \begin{equation*}
            H\rightarrow\left\{z\in\bbm{C}\Big|\Im z>0 \right\}
        \end{equation*}
        con la aplicación biyectiva $(x,y)\mapsto x+iy$.
    \end{obs}

    \begin{mydef}[\textbf{Haz tangente}]
        Sea $M$ una variedad $C^k$-diferenciable. El \textbf{fibrado tangente} o \textbf{haz tangente} es la unión disjunta de los espacios tangentes a cada punto de la variedad, dado por:
        \begin{equation*}
            TM=\bigsqcup_{ p\in M}T_pM=\bigcup_{ p\in M}\left\{p\right\}\times T_pM
        \end{equation*}
        donde $T_pM$ denota el espacio tangente a $M$ en el punto $p\in M$.
    \end{mydef}

    Como el conjunto $H$ es abierto y subconjunto de $\bbm{R}^2$, entonces este hereda la estructura de variedad suave de $\bbm{R}^2$. Además, como el haz tangente a $p\in\bbm{R}^2$ es trivial, se sigue también que el haz tangente a $H$ es trivial y por ende, podemos identificar de forma natural al espacio $T_zH$ como el espacio tangente de $x\in H$.

    Además, como $T_zH\cong\bbm{R}^2$, haremos la identificación de estos dos espacios como el mismo.

    \begin{mydef}[\textbf{Métrica Riemanniana}]
        Una \textbf{métrica Riemanniana} en una variedad $C^k$-diferenciable $M$ es una aplicación bilineal simétrica $\cf{g_p}{T_pM\times T_pM}{\bbm{R}}$ en cada uno de los espacios tangentes $T_pM$ de $M$.
    \end{mydef}

    \begin{obs}
        De la definición anterior se sigue que para cada $p\in M$ se satisface:
        \begin{enumerate}[label = \textit{(\arabic*)}]
            \item $g_p(u,v)=g_p(v,u)$ para todo $u,v\in T_pM$.
            \item $g_p(u,u)\geq0$ para todo $u\in T_pM$.
            \item $g_p(u,u)=0$ si y sólo si $u=0$.
        \end{enumerate}
    \end{obs}

    \begin{mydef}[\textbf{Plano Hiperbólico}]
        El \textbf{plano hiperbólico} $\bbm{H}^2$ es la variedad Riemanniana $(H,g_H)$, donde:
        \begin{itemize}
            \item $H\subseteq\bbm{R}^2$ hereda la estructura suave de $\bbm{R}^2$.
            \item Consideramos la métrica Riemanniana $\cf{g_{H,p}}{T_pH\times T_pH=\bbm{R}^2\times \bbm{R}^2}{\bbm{R}}$ dada por:
            \begin{equation*}
                g_{H,(x,y)}(u,v)=\frac{1}{y^2}\langle u,v\rangle,\quad\forall u,v\in\bbm{R}^2
            \end{equation*}
            para todo $(x,y)\in H$, donde $\langle\cdot\big|\cdot \rangle$ denota el producto interno usual de $\bbm{R}^2$. Más aún, escribiremos $\langle\cdot\big|\cdot \rangle_{ H,z}$ en vez de $g_{H,z}$ y a la norma inducida se le denotará por $\norm{\cdot}_{H,z}$.
        \end{itemize}
    \end{mydef}

    Nuestro interés ahora será hablar de las isometrías de $\bbm{H}^2$, para lo cual tendremos que construír una métrica en este espacio.
    
    \begin{mydef}[\textbf{Longitud hiperbólica de una curva}]
        Sea $\cf{\gamma}{[a,b]}{H}$ una curva suave. Se define la \textbf{longitud hiperbólica de $\gamma$} por:
        \begin{equation*}
            L_{\bbm{H}^2}(\gamma)=\int_{a}^{b}\norm{\dot{\gamma}(t)}_{ H,\gamma(t)}\:dt=\int_{a}^{b}\frac{\sqrt{\dot{\gamma_1}^2(t)+\dot{\gamma_2}^2(t)}}{\gamma_2(t)}\:dt
        \end{equation*}
        siendo $\gamma=(\gamma_1,\gamma_2)$.
    \end{mydef}

    \begin{propo}
        La función $\cf{d_H}{H\times H}{\bbm{R}_\geq0}$ dada por:
        \begin{equation*}
            (z,z')\mapsto\inf\left\{L_{\bbm{H}^2}(\gamma)\Big|\gamma\textup{ es una curva suave en $H$ que une a $z$ con $z'$} \right\}
        \end{equation*}
        es una métrica en $H$.
    \end{propo}

    \begin{proof}
        La simetría es inmediata, la desigualdad del triángulo se sigue de la definición.
    \end{proof}

    \begin{propo}
        Sea $\cf{\gamma}{[a,b]}{H}$ una curva suave. Entonces:
        \begin{equation*}
            L_{\bbm{H}^2}(\gamma)=L_{(H,d_H)}(\gamma)
        \end{equation*}
        donde $L_{(H,d_H)}$ es llamada la \textbf{longitud métrica} y está dada por:
        \begin{equation*}
            L_{(H,d_H)}=\sup\left\{\sum_{ j=0}^{k-1}d_H(\gamma(t_j),\gamma(t_{j+1}))\Big|k\in\bbm{N}, t_0,t_1,...,t_k\in[a,b], t_0<t_1<\cdots <t_k \right\}
        \end{equation*}
    \end{propo}

    Conociendo la métrica de este espacio, nos interesa conocer ahora las geodésicas del mismo. Para ello, primero veremos quiénes son las isometrías de este espacio.

    \begin{mydef}[\textbf{Grupo de isometrías Riemanniano}]
        Una \textbf{isometría Riemanniana de $\bbm{H}^2$} es un difeomorfismo suave $\cf{f}{H}{H}$ que satisface:
        \begin{equation*}
            \forall z\in H, \forall v,v'\in T_zH,\quad\pint{(Df)_z(v)}{(Df)_z(v')}_{H,f(z)}=\pint{v}{v'}_{H,z}
        \end{equation*}
    \end{mydef}

    \begin{propo}[\textbf{Isometrías Riemannianias son isometrías}]
        Toda isometría Riemanniana de $\bbm{H}^2$ es una isometría métrica de $(H,d_H)$. En particular, existe un monomorfismo de grupos:
        \begin{equation*}
            \Isom{\bbm{H}^2}\rightarrow \Isom{H,d_H}
        \end{equation*}
    \end{propo}

    \begin{proof}
        %TODO
    \end{proof}

    \section{Grupos Fuchsianos}

    \newcommand{\SL}[1]{\ensuremath{\textup{SL}\left(#1\right)}}
    \newcommand{\PSL}[1]{\ensuremath{\textup{PSL}\left(#1\right)}}

    \begin{mydef}
        $\SL{n,\bbm{A}}$ denota al espacio de todas las matrices $2\times 2$ con entradas en $\bbm{A}\subseteq\bbm{C}$ tales que:
        \begin{equation*}
            \det(A)=1,\quad\forall A\in\bbm{A}
        \end{equation*}
    \end{mydef}

    \begin{mydef}[\textbf{Transformaciones de Möbius}]
        Para la matriz $2\times 2$:
        \begin{equation*}
            \left(
                \begin{array}{cc}
                    a & b \\
                    c & d \\
                \end{array}
             \right)\in\SL{2,\bbm{R}}
        \end{equation*}
        definimos la \textbf{transformación de Möbius asociada} $\cf{f_A}{H}{H}$, dada por:
        \begin{equation*}
            z\mapsto\frac{a\cdot z+b}{c\cdot z+d}
        \end{equation*}
    \end{mydef}

    \begin{obs}
        Toda transformación de Möbius está bien definida, ya que como $H$ es el plano superior, entonces la parte real de $z$ nunca será un número con parte imaginaria cero, así que $c\cdot z+d\neq 0$ para todo $z\in H$.
    \end{obs}

    \begin{exa}
        La función $z\mapsto z$ es una transformación de Möbius. Al igual que la función $z\mapsto\frac{1}{z}$. En particular, todas las funciones lineales de $H$ en $H$ son transformaciones de Möbius.
    \end{exa}

    \begin{propo}
        Se tiene lo siguiente:
        \begin{enumerate}[label = \textit{(\arabic*)}]
            \item $f_A$ está bien definido y es un difeomorfismo $C^\infty$ (o suave).
            \item Para todo $A,B\in\SL{2,\bbm{R}}$ se tiene que $f_{A\cdot B}=f_A\circ f_B$.
            \item $f_A=f_{-A}$ para todo $A\in\SL{2,\bbm{R}}$.
        \end{enumerate}
    \end{propo}

    \begin{proof}
        De \textit{(1)} y \textit{(2)}: Son inmediatas.

        De \textit{(3)}: Si
        \begin{equation*}
            A=\left(
                \begin{array}{cc}
                    a & b \\
                    c & d \\
                \end{array}
             \right)\in\SL{2,\bbm{R}}
        \end{equation*}
        entonces,
        \begin{equation*}
            f_A(z)=\frac{a\cdot z+b}{c\cdot z+d}=\frac{-a\cdot z+-b}{-c\cdot z+-d}=f_{-A}(z)
        \end{equation*}
        para todo $z\in H$.
    \end{proof}

    \begin{exa}[\textbf{Generadores $\SL{2,\bbm{R}}$}]
        Tenemos los siguientes dos tipos de transformaciones de Möbius:
        \begin{itemize}
            \item Sea $b\in\bbm{R}$. Entonces, la transformación de Möbius asociada a la matriz:
            \begin{equation*}
                \left(\begin{array}{cc}
                    1 & b \\
                    0 & 1 \\
                \end{array} \right)\in\SL{2,\bbm{R}}
            \end{equation*}
            es la traslación horizontal $z\mapsto z+b$ en $H$ por un factor $b$ se denotará por $T_b$.
            \item La transformación de Möbius asociada a la matriz:
            \begin{equation*}
                \left(\begin{array}{cc}
                    0 & 1 \\
                    -1 & 0 \\
                \end{array} \right)\in\SL{2,\bbm{R}}
            \end{equation*}
            es la función $z\mapsto\frac{1}{z}$ se denotará por $In$.
        \end{itemize}

        Se tiene que el grupo $\SL{2,\bbm{R}}$ es generado por:
        \begin{equation*}
            \left\{\left(\begin{array}{cc}
                0 & 1 \\
                -1 & 0 \\
            \end{array} \right)\right\}\cup\left\{\left(\begin{array}{cc}
                1 & b \\
                0 & 1 \\
            \end{array} \right)\Big|b\in\bbm{R} \right\}
        \end{equation*}
    \end{exa}

    \begin{proof}
        Notemos que:
        \begin{equation*}
            \left(\begin{array}{cc}
                0 & 1 \\
                -1 & 0 \\
            \end{array} \right)\cdot\left(\begin{array}{cc}
                1 & b \\
                0 & 1 \\
            \end{array} \right)\cdot\left(\begin{array}{cc}
                0 & 1 \\
                -1 & 0 \\
            \end{array} \right)=\left(\begin{array}{cc}
                1 & 0 \\
                -b & 1 \\
            \end{array} \right)
        \end{equation*}
        para todo $b\in\bbm{R}$. Así que todas las matrices de la forma:
        \begin{equation*}
            \left(\begin{array}{cc}
                1 & 0 \\
                a & 1 \\
            \end{array} \right)
        \end{equation*}
        está en el grupo generado por el conjunto anterior. Para terminar, basta notar que toda matriz en $\SL{2,\bbm{R}}$ admite una descomposición $LU$ o $UL$, dependiendo del caso.
        %TODO justificar
    \end{proof}

    \begin{propo}[\textbf{Transformaciones de Möbius son isometrías}]
        Si $A\in\SL{2,\bbm{R}}$, entonces la transformación de Möbius asociada $\cf{f_A}{H}{H}$ es una isometría Riemanniana de $\bbm{H}^2$. En particular, tenemos un monomorfismo de grupos:
        \begin{equation*}
            \PSL{2,\bbm{R}}=\SL{2,\bbm{R}}/\left\{I,-I \right\}\rightarrow\Isom{H,d_H}
        \end{equation*}
        dado por $[A]\mapsto f_A$.
    \end{propo}

    \begin{proof}
        Por el ejemplo anterior basta con ver que $T_b$ y $In$ son isometrías Riemannianas de $\bbm{H}^2$, ya que la composición de isometrías Riemannianias sigue siendo una isometría Riemanniana. Analicemos los dos casos:
        \begin{itemize}
            \item 
        \end{itemize}
    \end{proof}

    \begin{theor}[\textbf{El grupo de isometrías hiperbólicas}]
        El grupo $\Isom{H,d_H}$ es generado por:
        \begin{equation*}
            \left\{f_A\Big|A\in\SL{2,\bbm{R}} \right\}\cup\left\{z\mapsto-\overline{z} \right\}
        \end{equation*}
        En particular, toda isometría de $(H,d_H)$ es una isometría Riemanniana suave y, $\Isom{H,d_H}=\Isom{\bbm{H}^2}$. Además, la función:
        
    \end{theor}

    \newcommand{\Tr}[1]{\ensuremath{\textup{Tr}\left(#1\right)}}

    \begin{mydef}
        Sea $A\in\PSL{2,\bbm{R}}$, con:
        \begin{equation*}
            A=\left(\begin{array}{cc}
                a & b \\
                c & d \\
            \end{array} \right)
        \end{equation*}
        \begin{itemize}
            \item Si $\Tr{A}<2$, entonces $A$ es llamada \textbf{elíptica}.
            \item Si $\Tr{A}=2$, entonces $A$ es llamada \textbf{parabólica}.
            \item Si $\Tr{A}>2$, entonces $A$ es llamada \textbf{hiperbólica}.
        \end{itemize}
    \end{mydef}



    \section{Hiperbólicidad y $\delta$-hiperbolicidad}

    Estudiaremos la propiedad de hiperbolicidad, que más adelante resutará de utilidad para estudiar invariantes cuasi-isométricos.

    \subsection{Espacios Hiperbólicos}

    \newcommand{\im}[1]{\ensuremath{\textup{im}\left(#1\right)}}
    \newcommand{\Diam}[1]{\ensuremath{\textup{diam}\left(#1\right)}}

    \begin{mydef}
        Sea $(X,d)$ un espacio métrico. Para cada $\delta>0$ y para cada $A\subseteq X$ se define el conjunto:
        \begin{equation*}
            B_\delta^{(X,d)}(A)=\left\{x\in X\Big|\exists a\in A\textup{ tal que }d(x,a)\leq\delta \right\}
        \end{equation*}
    \end{mydef}

    \begin{mydef}[\textbf{Triángulos geodésicos $\delta$-delgados}]
        Sea $(X,d)$ un espacio métrico.
        \begin{enumerate}[label = \textit{\arabic*}]
            \item Un \textbf{triángulo geodésico en $X$} es una tripleta $(\gamma_0,\gamma_1,\gamma_2)$ de geodésicas $\cf{\gamma_i}{[0,L_i]}{X}$ en $X$ tales que:
            \begin{equation*}
                \gamma_0(L_0)=\gamma_1(0),\quad \gamma_1(L_1)=\gamma_2(0),\quad \gamma_2(L_2)=\gamma_0(0)
            \end{equation*}
            \item Un triángulo geodésico es \textbf{$\delta$-delgado} si:
            \begin{equation*}
                \begin{split}
                    \im{\gamma_0}&\subseteq B_{\delta}^{(X,d)}(\im{\gamma_1}\cup\im{\gamma_2}),\\
                    \im{\gamma_1}&\subseteq B_{\delta}^{(X,d)}(\im{\gamma_0}\cup\im{\gamma_2}),\\
                    \im{\gamma_2}&\subseteq B_{\delta}^{(X,d)}(\im{\gamma_0}\cup\im{\gamma_1})\\
                \end{split}
            \end{equation*}
        \end{enumerate}
    \end{mydef}

    \begin{exa}
        
    \end{exa}

    \begin{mydef}
        Sea $(X,d)$ un espacio métrico.
        \begin{enumerate}[label = \textit{(\arabic*)}]
            \item Sea $\delta\bbm{R}_{\geq0}$. Decimos que $(X,d)$ es \textbf{$\delta$-hiperbólico} si $X$ es geodésico y todos los triángulos geodésicos de $X$ son $\delta$-delgados.
            \item $(X,d)$ es \textbf{hiperbólico} si existe $\delta\in\bbm{R}_{\geq0}$ tal que $(X,d)$ es $\delta$-hiperbólico.
        \end{enumerate}
    \end{mydef}

    \begin{exa}
        Todo espacio métrico geodésico $X$ de diámetro finito es $\Diam{X}$-hiperbólico. 
    \end{exa}

    \begin{exa}
        La recta real $\bbm{R}$ es $0$-hiperbólico ya que cada triángulo geodésico en $\bbm{R}$ es degenerado, pues estos se ven simplemente como líneas rectas.
    \end{exa}

    \begin{exa}
        El plano euclideano $\bbm{R}^2$ no es hiperbólico.
    \end{exa}

    %TODO: Hacer triángulo par que no sea hiperbólico

    \subsection{Hiperbolicidad de $\bbm{H}^2$}

    Nuestro objetivo en esta subsección será probar el siguiente resultado:

    \begin{propo}
        El plano hiperbólico $\bbm{H}^2$ es un espacio métrico hiperbólico en el sentido de la definición anterior.
    \end{propo}

    Antes de llegar a ello, probaremos algunos resultados adicionales y enunciaremos algunas definciones fundamentales.

    \begin{mydef}[\textbf{Área hiperbólica}]
        Sea $\cf{f}{H}{\bbm{R}_\geq0}$ una función Lebesgue integrable. Se define la \textbf{integral de $f$ sobre $\bbm{H}^2$} como:
        \begin{equation*}
            \begin{split}
                \int_{H}f\:dV_H&=\int_{H}f(x,y)\sqrt{\det(G_{H,(x,y)})}\:dxdy\\
                &=\int_{H}\frac{f(x,y)}{y^2}\:dxdy\\
            \end{split}
        \end{equation*}
        donde:
        \begin{equation*}
            G_{ H,,(x,y)}=\left(\begin{array}{cc}
                g_{ H,(x,y)}(e_1,e_1) & g_{ H,(x,y)}(e_1,e_2) \\
                g_{ H,(x,y)}(e_2,e_1) & g_{ H,(x,y)}(e_2,e_2) \\
            \end{array} \right)=\left(\begin{array}{cc}
                1/y^2 & 0 \\
                0 & 1/y^2 \\
            \end{array} \right)
        \end{equation*}
        siendo $e_1,e_2\in T_{(x,y)}H=\bbm{R}^2$ los vectores coordenados usuales.

        Si $A\subseteq H$ es un conjunto Lebesgue medible, definimos el \textbf{área hiperbólica de $A$} por:
        \begin{equation*}
            \mu_{\bbm{H}^2}(A)=\int_{H}\chi_A\:dV_H
        \end{equation*}
        siendo $\chi_A$ la función característica de $A$.
    \end{mydef}

    \begin{propo}[\textbf{Crecimiento exponencial del área hiperbólica}]
        Para todo $r\in\bbm{R}_{>10}$ tenemos que:
        \begin{equation*}
            \mu_{\bbm{H}^2}(B_r^{(H,d_H)}(i))\geq e^{\frac{r}{10}}(1-e^{-\frac{r}{2}})
        \end{equation*}
    \end{propo}

    \begin{proof}
        Sea $r\in\bbm{R}_{>10}$. Se tiene que el conjunto:
        \begin{equation*}
            Q_r=\left\{x+iy\Big|x\in[0,e^{ r/10}],y\in[1,e^{r/2}] \right\}
        \end{equation*}
        está contenido en $B_r^{(H,d_H)}(i)$. En particular, obtenemos que:
        \begin{equation*}
            \begin{split}
                \mu_{\bbm{H}^2}(B_r^{(H,d_H)}(i))&\geq\mu_{\bbm{H}^2}(Q_r)\\
                &=\int_{0}^{e^{r/10}}\int_{1}^{e^{r/2}}\frac{dxdy}{y^2}\\
                =&e^{\frac{r}{10}}(1-e^{-\frac{r}{2}})\\
            \end{split}
        \end{equation*}
    \end{proof}

    \begin{theor}[\textbf{Triángulos son delgados}]
        Existe una constante $C\in\bbm{R}_{\geq0}$ tal que todo triángulo geodésico en $(H,d_H)$ es $C$-delgado.
    \end{theor}



    

    \begin{minipage}{\textwidth}
    	\begin{center}
    	    %\includegraphics[scale=0.5]{direccion_imagen}\\
	    Figura \thefigcount. Caption.
	    \stepcounter{figcount}
	\end{center}
    \end{minipage}

\end{document}