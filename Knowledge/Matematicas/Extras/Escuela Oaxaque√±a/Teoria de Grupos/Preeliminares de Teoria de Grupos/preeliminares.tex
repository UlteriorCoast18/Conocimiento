\documentclass[12pt]{report}
\usepackage[spanish]{babel}
\usepackage[utf8]{inputenc}
\usepackage{amsmath}
\usepackage{amssymb}
\usepackage{amsthm}
\usepackage{graphics}
\usepackage{subfigure}
\usepackage{lipsum}
\usepackage{array}
\usepackage{multicol}
\usepackage{enumerate}
\usepackage[framemethod=TikZ]{mdframed}
\usepackage[a4paper, margin = 1.5cm]{geometry}
\usepackage{tikz}
\usepackage{pgffor}
\usepackage{ifthen}
\usepackage{enumitem}
\usepackage{hyperref}
\usepackage{bbm}

%Estilo del capítulo y sección

\makeatletter
\def\thickhrulefill{\leavevmode \leaders \hrule height 1ex \hfill \kern \z@}
\def\@makechapterhead#1{%
  {\parindent \z@ \raggedright
    \reset@font
    \hrule
    \vspace*{10\p@}%
    \par
    \center \LARGE \scshape \@chapapp{} \huge \thechapter
    \vspace*{10\p@}%
    \par\nobreak
    \vspace*{10\p@}%
    \par
    \vspace*{1\p@}%
    \hrule
    %\vskip 40\p@
    \vspace*{60\p@}
    \Huge #1\par\nobreak
    \vskip 50\p@
  }}

\def\section#1{%
  \par\bigskip\bigskip
  \hrule\par\nobreak\noindent
  \refstepcounter{section}%
  \addcontentsline{toc}{chapter}{#1}%
  \reset@font
  { \large \scshape
    \strut\S \thesection \quad
    #1}% 
    \hrule   
  \par
  \medskip
}

%Gestión marca de agua

\usetikzlibrary{shapes.multipart}

\newcounter{it}
\newcommand*\watermarktext[1]{\begin{tabular}{c}
    \setcounter{it}{1}%
    \whiledo{\theit<100}{%
    \foreach \col in {0,...,15}{#1\ \ } \\ \\ \\
    \stepcounter{it}%
    }
    \end{tabular}
    }

\AddToHook{shipout/foreground}{
    \begin{tikzpicture}[remember picture,overlay, every text node part/.style={align=center}]
        \node[rectangle,black,rotate=30,scale=2,opacity=0.04] at (current page.center) {\watermarktext{Cristo Daniel Alvarado ESFM\quad}};
  \end{tikzpicture}
}

%En esta parte se hacen redefiniciones de algunos comandos para que resulte agradable el verlos%

\def\proof{\paragraph{Demostración:\\}}
\def\endproof{\hfill$\blacksquare$}

\def\sol{\paragraph{Solución:\\}}
\def\endsol{\hfill$\square$}

%En esta parte se definen los comandos a usar dentro del documento para enlistar%

\newtheoremstyle{largebreak}
  {}% use the default space above
  {}% use the default space below
  {\normalfont}% body font
  {}% indent (0pt)
  {\bfseries}% header font
  {}% punctuation
  {\newline}% break after header
  {}% header spec

\theoremstyle{largebreak}

\newmdtheoremenv[
    leftmargin=0em,
    rightmargin=0em,
    innertopmargin=0pt,
    innerbottommargin=5pt,
    hidealllines = true,
    roundcorner = 5pt,
    backgroundcolor = gray!60!red!30
]{exa}{Ejemplo}[section]

\newmdtheoremenv[
    leftmargin=0em,
    rightmargin=0em,
    innertopmargin=0pt,
    innerbottommargin=5pt,
    hidealllines = true,
    roundcorner = 5pt,
    backgroundcolor = gray!50!blue!30
]{obs}{Observación}[section]

\newmdtheoremenv[
    leftmargin=0em,
    rightmargin=0em,
    innertopmargin=0pt,
    innerbottommargin=5pt,
    rightline = false,
    leftline = false
]{theor}{Teorema}[section]

\newmdtheoremenv[
    leftmargin=0em,
    rightmargin=0em,
    innertopmargin=0pt,
    innerbottommargin=5pt,
    rightline = false,
    leftline = false
]{propo}{Proposición}[section]

\newmdtheoremenv[
    leftmargin=0em,
    rightmargin=0em,
    innertopmargin=0pt,
    innerbottommargin=5pt,
    rightline = false,
    leftline = false
]{cor}{Corolario}[section]

\newmdtheoremenv[
    leftmargin=0em,
    rightmargin=0em,
    innertopmargin=0pt,
    innerbottommargin=5pt,
    rightline = false,
    leftline = false
]{lema}{Lema}[section]

\newmdtheoremenv[
    leftmargin=0em,
    rightmargin=0em,
    innertopmargin=0pt,
    innerbottommargin=5pt,
    roundcorner=5pt,
    backgroundcolor = gray!30,
    hidealllines = true
]{mydef}{Definición}[section]

\newmdtheoremenv[
    leftmargin=0em,
    rightmargin=0em,
    innertopmargin=0pt,
    innerbottommargin=5pt,
    roundcorner=5pt
]{excer}{Ejercicio}[section]

%En esta parte se colocan comandos que definen la forma en la que se van a escribir ciertas funciones%

\newcommand\abs[1]{\ensuremath{\left|#1\right|}}
\newcommand\divides{\ensuremath{\bigm|}}
\newcommand\cf[3]{\ensuremath{#1:#2\rightarrow#3}}
\newcommand\contradiction{\ensuremath{\#_c}}
\newcommand\natint[1]{\ensuremath{\left[\big|#1\big|\right]}}
\newcommand{\bbm}[1]{\ensuremath{\mathbbm{#1}}}
\newcommand{\Aut}[1]{\ensuremath{\textup{Aut}\left(#1\right)}}
\newcommand{\gen}[1]{\ensuremath{\langle#1\rangle}}
\newcommand{\im}[1]{\ensuremath{\textup{Im}\left(#1\right)}}

\begin{document}
    \setlength{\parskip}{5pt} % Añade 5 puntos de espacio entre párrafos
    \setlength{\parindent}{12pt} % Pone la sangría como me gusta
    \title{10° Escuela Oaxaqueña de Matemáticas
    
    Notas}
    \author{Cristo Daniel Alvarado}
    \maketitle

    %\setcounter{chapter}{3} %En esta parte lo que se hace es cambiar la enumeración del capítulo%

    \newpage

    \chapter{Ejercicios y Problemas\\ Teoría de Grupos}

    \section{Preliminares Teoría de Grupos}

    \begin{excer}
        Supongamos que $G$ es un grupo que tiene un subgrupo de índice finito $H$. Demuestra que $G$ tiene un subgrupo normal de índice finito. 
    \end{excer}

    \begin{proof}
        Se tienen dos casos:
        \begin{itemize}
            \item $G$ es finito, en cuyo caso $G$ es un subgrupo normal de $G$ de índice finito.
            \item $G$ es infinito. 
        \end{itemize}
    \end{proof}

    \begin{excer}
        ¿Cuál es el grupo de automorfismos del grupo aditivo $\mathbb{Z}$?
    \end{excer}

    \begin{sol}
        Considere al grupo de automorfismos del grupo aditivo $\mathbb{Z}$, digamos:
        \begin{equation*}
            A=\Aut{\mathbb{Z}}=\left\{\cf{f}{\mathbb{Z}}{\mathbb{Z}}\Big|f\textup{ es isomorfismo} \right\}
        \end{equation*}
        Afirmamos que $\Aut{\mathbb{Z}}\cong\mathbb{Z}/2\mathbb{Z}$ donde $\mathbb{Z}/2\mathbb{Z}$ es el grupo aditivo de los enteros módulo 2. En efecto, afirmamos que:
        \begin{equation*}
            \Aut{\mathbb{Z}}=\left\{\bbm{1}_{\mathbb{Z}},-\bbm{1}_{\mathbb{Z}}\right\}
        \end{equation*}
        donde $\cf{\bbm{1}_{\mathbb{Z}}}{\mathbb{Z}}{\mathbb{Z}}$ es la identidad de $\mathbb{Z}$ y $\cf{-\bbm{1}_{\mathbb{Z}}}{\mathbb{Z}}{\mathbb{Z}}$ es tal que $-\bbm{1}_{\mathbb{Z}}\left(m\right)=-m$ para todo $m\in\mathbb{Z}$. En efecto, es claro que $\left\{\bbm{1}_{\mathbb{Z}},-\bbm{1}_{\mathbb{Z}}\right\}\subseteq\Aut{\mathbb{Z}}$.

        Sea ahora $f\in\Aut{\mathbb{Z}}$, se tiene que:
        \begin{equation*}
            f(m)=f(\underset{m\textup{-veces}}{\underbrace{1+\cdots+1}})=\underset{m\textup{-veces}}{\underbrace{f(1)+\cdots+f(1)}}=mf(1)
        \end{equation*}
        para todo $m\in\mathbb{N}$. De forma análoga se demuestra que:
        \begin{equation*}
            f(-m)=-mf(1),\quad\forall m\in\mathbb{N}
        \end{equation*}
        Así que:
        \begin{equation*}
            f(m)=mf(1),\quad\forall m\in\mathbb{Z}
        \end{equation*}
        por lo que $f$ está únicamente determinada por su valor en $1$. Como $\mathbb{Z}$ tiene únicamente dos generadores (por ser un grupo cíclico infinito), al ser $f$ automorfismo debe suceder que $\mathbb{Z}=\gen{f(1)}$, así que $f(1)=1$ ó $f(1)=-1$, es decir que:
        \begin{equation*}
            \begin{split}
                f(m)&=mf(1)\\
                &=\left\{
                    \begin{array}{rl}
                        m & \textup{ si }f(1) = 1\\
                        -m & \textup{ si }f(1) = -1\\
                    \end{array}
                \right.\\
                &=\left\{
                    \begin{array}{rl}
                        \bbm{1}_\mathbb{Z}(m)& \textup{ si }f(1) = 1\\
                        -\bbm{1}_\mathbb{Z}(m)& \textup{ si }f(1) = 1\\
                    \end{array}
                 \right.\\
            \end{split}
        \end{equation*}
        es decir, que $f=\bbm{1}_\mathbb{Z}$ o $f=-\bbm{1}_{\mathbb{Z}}$. Por tanto, $\Aut{\mathbb{Z}}=\left\{\bbm{1}_\mathbb{Z},-\bbm{1}_\mathbb{Z}\right\}$. Para la otra parte, es inmediato que el grupo $\left\{\bbm{1}_\mathbb{Z},-\bbm{1}_\mathbb{Z}\right\}$ con la composición de funciones es isomorfo al grupo aditivo $\mathbb{Z}/2\mathbb{Z}$.
    \end{sol}

    \begin{excer}
        Supongamos que tenemos una sucesión exacta corta de grupos:
        \begin{equation*}
            1\rightarrow N\rightarrow G\rightarrow K\rightarrow 1
        \end{equation*}
        demuestra que si $N$ y $K$ son grupos finitamente generados, entonces $G$ es finitamente generado.
    \end{excer}

    \begin{proof}
        Al tenerse la sucesión exacta corta de grupos, estamos diciendo que existen homomorfismos $\cf{f_0}{\gen{1}}{N}$, $\cf{f_1}{N}{G}$, $\cf{f_2}{G}{K}$ y $\cf{f_3}{K}{\gen{1}}$ tales que:
        \begin{equation*}
            \im{f_{i-1}}=\ker\left(f_i \right),\quad\forall i=1,2,3
        \end{equation*}
        En particular, notemos que $f_1$ es monomorfismo y que $f_2$ es epimorfismo, ya que:
        \begin{equation*}
            \ker\left(f_1\right)=\im{f_0}=\gen{e_N}
        \end{equation*}
        siendo $e_N$ la identidad del grupo $N$ y, además:
        \begin{equation*}
            \im{f_2}=\ker\left(f_3\right)=K
        \end{equation*}
        por lo que se tiene lo afirmado.

        Supongamos ahora que $N$ y $K$ son finitamnete generados, entonces existen elementos $n_1,...,n_m\in N$ y $k_1,...,k_l\in K$ tales que:
        \begin{equation*}
            N=\gen{n_1,...,n_m}\quad\textup{y}\quad K=\gen{k_1,...,k_l}
        \end{equation*}
        Como $f_3$ es epimorfismo, entonces del Primer Teorema de Isomorfismo se sigue que:
        \begin{equation*}
            K\cong G/\ker(f_3)=G/\im{f_2}=G/N'
        \end{equation*}
        donde $N'=f_2(N)$.
        
    \end{proof}

    \begin{excer}
        Demuestra que en el producto semidirecto $N\rtimes_{\varphi}H$, $H$ es un subgrupo normal si y sólo si $\varphi$ es el homomorfismo trivial.
    \end{excer}

    \begin{proof}
        Recordemos que el producto semidirecto $N\rtimes_\varphi H$ es el grupo $N\times H$ dotado de la operación:
        \begin{equation*}
            (n,h)(n',h')=(n\varphi_h(n'),hh')
        \end{equation*}
        donde $\cf{\varphi}{H}{\Aut{N}}$ es un homomorfismo tal que $h\mapsto\varphi_h$. El elemento neutro de este grupo es $(e_N,e_H)$, donde cada elemento tiene como inverso:
        \begin{equation*}
            (n,h)^{-1}=\left((\varphi_{h^{-1}}(n))^{-1},h^{-1}\right)
        \end{equation*}

        $\Rightarrow)$: Suponga que $H$ es un subgrupo normal de $N\rtimes_\varphi H$, esto es que el grupo $H$ visto como subgrupo de $N\rtimes_\varphi H$:
        \begin{equation*}
            H=\left\{(e_N,h)\Big|h\in H \right\}
        \end{equation*}
        es subgrupo normal de $N\rtimes_\varphi H$. Como es normal, se sigue que:
        \begin{equation*}
            (n_1,h_1)(e_N,h)(n_1,h_1)^{-1}\in H
        \end{equation*}
        para todo $(n_1,h_1)\in H$ y para todo $h\in H$.
    \end{proof}

    \begin{excer}
        Demuestra que el producto libre en $n$ generadores $F_n$ es isomorfo al producto libre de $n$ copias de $\mathbb{Z}$, $\mathbb{Z}*\mathbb{Z}*\cdots*\mathbb{Z}$.
    \end{excer}

    \begin{proof}
        
    \end{proof}

    \begin{excer}
        Demuestra que el producto libre $G*H$ de grupos no triviales $H$ y $G$ tiene centro trivial.
    \end{excer}

    \begin{proof}
        Sean $G$ y $H$ grupos no triviales. Considere $G*H$ su producto libre. El centro de $G*H$ se define por:
        \begin{equation*}
            Z(G*H)=\left\{x\in G*H\Big|xy=yx,\forall y\in G*H \right\}
        \end{equation*}
    \end{proof}

    \begin{excer}
        Demuestra que $\mathbb{Z}_2*\mathbb{Z}_2$ es isomorfo a $\mathbb{Z}\rtimes\mathbb{Z}_2$.
    \end{excer}

    \begin{proof}
        
    \end{proof}

    \begin{excer}
        Denotemos por $F_n$ al grupo libre en $n$ generadores. Demuestre que $F_n$ es isomorfo a $F_m$ si y sólo si $n=m$.
    \end{excer}

    \begin{proof}
        Como $F_n$ es grupo libre en $n$ generadores y $F_m$ lo es en $m$, tomamos $x_1,...,x_n$ y $y_1,...,y_m$ tales que:
        \begin{equation*}
            F_n=
        \end{equation*}

        $\Rightarrow):$ Supongamos que $F_n$ es isomorfo a $F_m$.
    \end{proof}

\end{document}