\documentclass[12pt]{report}
\usepackage[spanish]{babel}
\usepackage[utf8]{inputenc}
\usepackage{amsmath}
\usepackage{amssymb}
\usepackage{amsthm}
\usepackage{graphics}
\usepackage{subfigure}
\usepackage{lipsum}
\usepackage{array}
\usepackage{multicol}
\usepackage{enumerate}
\usepackage[framemethod=TikZ]{mdframed}
\usepackage[a4paper, margin = 1.5cm]{geometry}
\usepackage{tikz}
\usepackage{pgffor}
\usepackage{ifthen}
\usepackage{enumitem}
\usepackage{hyperref}
\usepackage{bbm}

%Estilo del capítulo y sección

\makeatletter
\def\thickhrulefill{\leavevmode \leaders \hrule height 1ex \hfill \kern \z@}
\def\@makechapterhead#1{%
  {\parindent \z@ \raggedright
    \reset@font
    \hrule
    \vspace*{10\p@}%
    \par
    \center \LARGE \scshape \@chapapp{} \huge \thechapter
    \vspace*{10\p@}%
    \par\nobreak
    \vspace*{10\p@}%
    \par
    \vspace*{1\p@}%
    \hrule
    %\vskip 40\p@
    \vspace*{60\p@}
    \Huge #1\par\nobreak
    \vskip 50\p@
  }}

\def\section#1{%
  \par\bigskip\bigskip
  \hrule\par\nobreak\noindent
  \refstepcounter{section}%
  \addcontentsline{toc}{chapter}{#1}%
  \reset@font
  { \large \scshape
    \strut\S \thesection \quad
    #1}% 
    \hrule   
  \par
  \medskip
}

%Gestión marca de agua

\usetikzlibrary{shapes.multipart}

\newcounter{it}
\newcommand*\watermarktext[1]{\begin{tabular}{c}
    \setcounter{it}{1}%
    \whiledo{\theit<100}{%
    \foreach \col in {0,...,15}{#1\ \ } \\ \\ \\
    \stepcounter{it}%
    }
    \end{tabular}
    }

\AddToHook{shipout/foreground}{
    \begin{tikzpicture}[remember picture,overlay, every text node part/.style={align=center}]
        \node[rectangle,black,rotate=30,scale=2,opacity=0.04] at (current page.center) {\watermarktext{Cristo Daniel Alvarado ESFM\quad}};
  \end{tikzpicture}
}

%En esta parte se hacen redefiniciones de algunos comandos para que resulte agradable el verlos%

\def\proof{\paragraph{Demostración:\\}}
\def\endproof{\hfill$\blacksquare$}

\def\sol{\paragraph{Solución:\\}}
\def\endsol{\hfill$\square$}

%En esta parte se definen los comandos a usar dentro del documento para enlistar%

\newtheoremstyle{largebreak}
  {}% use the default space above
  {}% use the default space below
  {\normalfont}% body font
  {}% indent (0pt)
  {\bfseries}% header font
  {}% punctuation
  {\newline}% break after header
  {}% header spec

\theoremstyle{largebreak}

\newmdtheoremenv[
    leftmargin=0em,
    rightmargin=0em,
    innertopmargin=0pt,
    innerbottommargin=5pt,
    hidealllines = true,
    roundcorner = 5pt,
    backgroundcolor = gray!60!red!30
]{exa}{Ejemplo}[section]

\newmdtheoremenv[
    leftmargin=0em,
    rightmargin=0em,
    innertopmargin=0pt,
    innerbottommargin=5pt,
    hidealllines = true,
    roundcorner = 5pt,
    backgroundcolor = gray!50!blue!30
]{obs}{Observación}[section]

\newmdtheoremenv[
    leftmargin=0em,
    rightmargin=0em,
    innertopmargin=0pt,
    innerbottommargin=5pt,
    rightline = false,
    leftline = false
]{theor}{Teorema}[section]

\newmdtheoremenv[
    leftmargin=0em,
    rightmargin=0em,
    innertopmargin=0pt,
    innerbottommargin=5pt,
    rightline = false,
    leftline = false
]{propo}{Proposición}[section]

\newmdtheoremenv[
    leftmargin=0em,
    rightmargin=0em,
    innertopmargin=0pt,
    innerbottommargin=5pt,
    rightline = false,
    leftline = false
]{cor}{Corolario}[section]

\newmdtheoremenv[
    leftmargin=0em,
    rightmargin=0em,
    innertopmargin=0pt,
    innerbottommargin=5pt,
    rightline = false,
    leftline = false
]{lema}{Lema}[section]

\newmdtheoremenv[
    leftmargin=0em,
    rightmargin=0em,
    innertopmargin=0pt,
    innerbottommargin=5pt,
    roundcorner=5pt,
    backgroundcolor = gray!30,
    hidealllines = true
]{mydef}{Definición}[section]

\newmdtheoremenv[
    leftmargin=0em,
    rightmargin=0em,
    innertopmargin=0pt,
    innerbottommargin=5pt,
    roundcorner=5pt
]{excer}{Ejercicio}[section]

%En esta parte se colocan comandos que definen la forma en la que se van a escribir ciertas funciones%

\newcommand\abs[1]{\ensuremath{\left|#1\right|}}
\newcommand\divides{\ensuremath{\bigm|}}
\newcommand\cf[3]{\ensuremath{#1:#2\rightarrow#3}}
\newcommand\contradiction{\ensuremath{\#_c}}
\newcommand\natint[1]{\ensuremath{\left[\big|#1\big|\right]}}
\newcommand{\bbm}[1]{\ensuremath{\mathbbm{#1}}}
\newcommand{\Aut}[1]{\ensuremath{\textup{Aut}\left(#1\right)}}
\newcommand{\gen}[1]{\ensuremath{\langle#1\rangle}}
\newcommand{\im}[1]{\ensuremath{\textup{Im}\left(#1\right)}}

\begin{document}
    \setlength{\parskip}{5pt} % Añade 5 puntos de espacio entre párrafos
    \setlength{\parindent}{12pt} % Pone la sangría como me gusta
    \title{10° Escuela Oaxaqueña de Matemáticas
    
    Notas}
    \author{Cristo Daniel Alvarado}
    \maketitle

    %\setcounter{chapter}{3} %En esta parte lo que se hace es cambiar la enumeración del capítulo%

    \newpage

    \chapter{Básicos de Teoría de Grupos y Acciones de Grupos}

    Estudiaremos en todo el curso algo llamado la \textbf{teoría geométrica de grupos}. Esta teoría está en la intersección de varias áreas, como son la teoría de grupos, la topología algebraica y la geometría diferencial.

    Veremos básicos de teoría de grupos (junto con cosas de acciones de grupos) y cosas de topología algebraica.

    \section{Grupos Libres}

    La motivación de grupos libres es que cuando tenemos dos espacios vectoriales $V$ y $W$, para definir un morfismo $f$ entre ambos espacios basta con definirlo en la base de $V$. Sin embargo, en grupos resulta más complicado hacer esta definición para poder definir el morfismo.

    \begin{mydef}
        Sea $S$ un conjunto y $\hat{S}$ un conjunto disjunto de $S$ y biyectivo a $S$. Una \textbf{palabra} en $S\cup\hat{S}$ es una sucesión finita en $S\cup\hat{S}$. Denotamos por $A(S)$ al conjunto de todas las palabras en $S$.
    \end{mydef}

    \begin{obs}
        Lo último en la definición anterior quiere decir que tomemos una función biyectiva $\cf{\varphi}{S}{\hat{S}}$.
    \end{obs}

    \begin{propo}
        Sea $S$ un conjunto, entonces $A(S)$ es un monoide con la operación de concatenación. Tal que:
        \begin{enumerate}
            \item La palabra vacía $\emptyset=\varepsilon$ es el elemento neutro.
            \item La operación es asociativa.
        \end{enumerate}
    \end{propo}

    \begin{proof}
        
    \end{proof}

    Pero, ¿cómo agregamos inversos?

    \begin{mydef}
        Sea $S$ un conjunto. Definimos la relación $\sim$ en $A(S)$ como la generada por:
        \begin{equation*}
            \begin{split}
                \forall x,y\in A(S)\forall s\in S xs\hat{s}y&\sim xy;\\
                \forall x,y\in A(S)\forall s\in S x\hat{s}sy&\sim xy;\\
            \end{split}
        \end{equation*}
    \end{mydef}

    \begin{propo}
        El espacio cociente $F(S)=A[S]/\sim$ es un grupo con la operación concatenación $[w]\cdot[v]=[wv]$. $F(S)$ es llamado \textbf{grupo libre}.
    \end{propo}

    \begin{proof}
        
    \end{proof}

    \begin{exa}
        Si $S=\left\{a\right\}$, entonces $F(S)=\left\{,...,\hat{a}\hat{a}\hat{a},\hat{a}\hat{a},\hat{a},\emptyset,a,aa,aaa,... \right\}\cong\mathbb{Z}$.
    \end{exa}

    En el ejemplo anterior la concatenación de palabras se denotará simplemente por potencia y, al elemento asociado en $\hat{S}$ se le denotará por $s^{-1}$.

    \begin{exa}
        Si $\abs{S}>1$, entonces $F(S)$ no es abeliano.
    \end{exa}
    
    \begin{propo}[\textbf{Propiedad Universal de Grupos Libres}]
        Sea $F(S)$ el grupo libre generado por $S$. Para todo grupo $H$ y toda función $\cf{f}{S}{H}$ existe un único homomorfismo de grupos $\cf{\hat{f}}{F(S)}{H}$ tal que el diagrama:
        
        es conmutativo, esto es que:
        \begin{equation*}
            \hat{f}\circ\iota=f
        \end{equation*}
    \end{propo}

    \begin{proof}
        
    \end{proof}

    \begin{mydef}
        Sea $n\in\mathbb{N}$ y $S=\left\{x_1,...,x_n \right\}$ con $x_i\neq x_j$ para todo $i,j\in\natint{1,n}$ con $i\neq j$. Enotnces, escribimos por $F_n$ al \textbf{grupo libre generado por $S$} y $F_n$ es llamado \textbf{grupo libre de rango $n$}.
    \end{mydef}

    \section{Generadores y Relaciones}

    \begin{mydef}
        Sea $G$ un grupo y $S\subseteq G$. El subgrupo normal generado por $S$ es el subgrupo normal más pequeño que contiene a $S$, denotamos este conjunto por: $\gen{S}^{\vartriangleleft}=\gen{\gen{S}}$.
    \end{mydef}

    \begin{exa}
        Si $G$ es un grupo abeliano, entonces para todo $S\subseteq G$:
        \begin{equation*}
            \gen{S}^\vartriangleleft=\gen{S}
        \end{equation*}
    \end{exa}

    \begin{mydef}
        Sea $S$ un conjunto y considere el conjunto de palabras de $S\cup S^{-1}$ denotado por $(S\cup S^{-1})^*$. Entonces, para $R\subseteq S\cup S^{-1}$ definimos:
        \begin{equation*}
            \gen{S|R}=F(S)/\gen{R}^\vartriangleleft
        \end{equation*}

        Si $G$ es grupo con $G\cong\gen{S|R}$, entonces el par $(S,R)$ es llamado una \textbf{presentación de $G$}.
    \end{mydef}

    \begin{exa}
        Para todo $n\in\mathbb{N}$, $\gen{x|x^n}\cong\mathbb{Z}/n\mathbb{Z}$.
    \end{exa}

    \begin{exa}
        $\mathbb{Z}\cong\gen{a|\emptyset}$.
    \end{exa}

    \begin{exa}
        Considere $F_n$ y $\mathbb{Z}^n$. Ambos no son isomorfos ya que $F_n$ no necesariamente es abeliano. Sea:
        \begin{equation*}
            R=\left\{x_ix_jx_i^{-1}x_j^{-1}\Big|x_i,x_j\in F_n \right\}\subseteq F_n
        \end{equation*}
        entonces:
        \begin{equation*}
            \mathbb{Z}^n\cong F_n/\gen{R}^\vartriangleleft
        \end{equation*}
        tal que:
        \begin{equation*}
            (0,)
        \end{equation*}
    \end{exa}

    \begin{propo}[\textbf{Propiedad Universal de la presentación de Grupos}]
        
    \end{propo}

    \begin{proof}
        
    \end{proof}
    
    \textbf{El problema de la palabra}: Sea $G=\gen{S|R}$, dar un algoritmo que determine cuando una palabra representa una palabra trivial o no.

    \begin{mydef}
        Sea $G$ un grupo. Decimos que $G$ es \textbf{finitamente presentado} (abreviado por f.p.) si existe un conjunto finito $S$ y un conjunto finito $S\subseteq(S\cup S^{-1})^*$ tal que:
        \begin{equation*}
            \gen{S|R}\cong G
        \end{equation*}
    \end{mydef}

    \begin{exa}
        $\mathbb{Z}$, $\mathbb{Z}^n$ y $\mathbb{Z}/n\mathbb{Z}$ son f.p.
    \end{exa}

    \section{Producto Libre de Grupos}

    \begin{mydef}
        Sea $\left\{G_i\right\}_{ i\in I}$ una familia no vacía de grupos. El \textbf{producto libre de $\left\{G_i\right\}_{ i\in I}$}, denotado por:
        \begin{equation*}
            *_{ i\in I}G_i
        \end{equation*}
        es el conjunto $\Omega$ de todas las palabras reducidas $g_1\cdots g_n$, donde $g_i\in G$ y $g_i\neq e_i$. Además, $g_i$ y $g_{i+1}$ no están en el mismo $G_j$.
    \end{mydef}

    \begin{propo}
        $\Omega$ con la operación de concatenación y reducción es un grupo.
    \end{propo}

    \begin{proof}
        
    \end{proof}

    \begin{excer}
        Si $G_i=\gen{S_i|R_i}$, entonces $*_{i\in I}G_i=\gen{\bigcup_{ i\in I}S_i|\bigcup_{ i\in I}R_i}$.
    \end{excer}

    \begin{proof}
        
    \end{proof}

    \begin{excer}
        Investigar la propiedad universal del producto libre de grupos.
    \end{excer}

    \section{Pushout de Grupos}

    Supongamos que tenemos el siguiente diagrama de grupos:

    ¿será posible construir el grupo $L$ junto con los morfismos $\beta_1$ y $\beta_2$?

    Resulta que esto también satisface una propiedad universal.

    \begin{mydef}
        Sean $A$ un grupo y $\cf{\alpha_i}{A}{G_i}$, $i=1,2$ morfismos de grupos. Un grupo $G$ junto con morfismos $\cf{\beta_i}{G_i}{G}$ satisfaciendo:
        \begin{equation*}
            \beta_1\circ\alpha_1=\beta_2\circ\alpha_2
        \end{equation*}
        es llamado un \textbf{pushout de $G_1$ y $G_2$} sobre $A$ si la siguiente propiedad universal se satisface:
    \end{mydef}

    \section{Acciones de Grupos}

    \begin{mydef}
        Sean $G$ un grupo y $X$ un conjunto. \textbf{Una acción de $G$ en $X$} es una función binaria $G\times X\rightarrow X$, $(g,x)\mapsto gx$ que satisface dos axiomas:
        \begin{enumerate}
            \item $ex=x$.
            \item $\forall g,h\in G$, $g(hx)=(gh)x$, para todo $x\in X$.
        \end{enumerate}
        Esta acción se denota por $G\curvearrowright X$.
    \end{mydef}
    
    \begin{exa}
        $\mathbb{Z}\curvearrowright\mathbb{R}$ dada por $(n,x)\mapsto n+x$.

        Esta acción se puede generalizar a una $\mathbb{Z}^n\curvearrowright\mathbb{R}^n$, tal que $(\vec{n},\vec{x})\mapsto\vec{n}+\vec{x}$.

        Estas dos acciones cumplen los dos axiomsa de la definición anterior.
    \end{exa}

    \begin{exa}
        $\mathbb{Z}/2\mathbb{Z}\curvearrowright\mathbb{S}^2$. Tomando $\mathbb{Z}/2\mathbb{Z}=\gen{a|a^2}$, hacemos $ax=-x$ para todo $x\in\mathbb{S}^2$. Esta acción es llamada \textbf{acción antipodal}.
    \end{exa}

    \begin{exa}
        $GL(n,\mathbb{R})\curvearrowright\mathbb{R}^n$ tal que $(A,\vec{x})\mapsto A\vec{x}$ el producto de matrices usual viendo a $\vec{x}$ como vector columna.
    \end{exa}

    \begin{obs}
        Una acción $G\curvearrowright X$ es lo mismo que un morfismo de grupos $\cf{\varphi}{G}{\Aut{X}}$.
        
        Dependiendo de $X$, podemos pedir diferentes cosas para $\Aut{X}$. En el caso anterior hacemos $g\mapsto\varphi_g$ donde $\cf{\varphi_g}{X}{X}$ es tal que $x\mapsto\varphi_g(x)=gx$.

        De forma viceversa, podemos definir un morfimso de grupos a partir de una acción de grupos.
    \end{obs}

    \begin{mydef}
        Sea $G\curvearrowright X$. Dado $x\in X$ definimos la \textbf{órbita de $x$} como:
        \begin{equation*}
            \mathcal{O}_x=\left\{gx\Big|g\in G \right\}
        \end{equation*}
    \end{mydef}

    \begin{exa}
        En la acción $\mathbb{Z}\curvearrowright\mathbb{R}$
    \end{exa}

    \begin{mydef}
        Dada una acción de grupos $G\curvearrowright X$, definimos el \textbf{espacio cociente $X/G$} como el espacio cociente generado a partir de la relación $\sim$ dada por:
        \begin{equation*}
            x\sim y\textup{ si y sólo si }\exists g\in G\textup{ tal que }y=gx
        \end{equation*}
    \end{mydef}
    
    \begin{exa}
        Si $\mathbb{Z}^2\curvearrowright\mathbb{R}^2$, entonces el espacio $\mathbb{R}^2/\mathbb{Z}^2=\mathbb{T}=\mathbb{S}^1\times\mathbb{S}^1$.
    \end{exa}

    \begin{exa}
        En la acción $\mathbb{Z}/2\mathbb{Z}\curvearrowright\mathbb{S}^2$, el espacio resulta que $\mathbb{S}^2/(\mathbb{Z}/2\mathbb{Z})=\mathbb{R}P^2$.
    \end{exa}

    \chapter{Gráficas y Árboles}

    \section{Básicos}

    \begin{obs}
        En lo que sigue, todas las gráficas serán no dirigidas, simples y sin lazos.
    \end{obs}

    Que sean no dirigidas es que las aristas no tienen dirección, que sean simples es que no haya más de una arista uniendo dos vértices y que no tengan lazos es que un vértice no sea unido hacia sí mismo por una arista.

    \begin{mydef}
        Sea $A$ un conjunto, se define el conjunto de \textbf{$k$-tuplas de $A$}, denotado por $[A]^k$, como el conjunto de todos los subconjuntos de $A$ de cardinalidad $k$. 
    \end{mydef}

    \begin{mydef}
        Una \textbf{gráfica} es un par $G=(V,E)$ de conjuntos disjuntos, donde $V$ es un conjunto no vacío de \textbf{vértices} o \textbf{nodos} $V$ y un conjunto $E$ de \textbf{aristas} tal que $E\subseteq[V]^2$.
    \end{mydef}

    \begin{exa}
        Considere la gráfica $G=(V,E)$ donde $V=\left\{a,b,c,d \right\}$ y $E=\left\{\left\{a,b \right\},\left\{a,c \right\},\left\{b,c \right\},\left\{c,d \right\} \right\}$.
    \end{exa}

    \begin{mydef}
        Sea $(V,E)$ una gráfica.
        \begin{enumerate}
            \item Decimos que dos vértices $v,v'\in V$ son \textbf{vecinos} o \textbf{adyacentes} si están unidos por una arista, es decir si $\left\{v,v'\right\}\in E$.
            \item El número de vecinos de un vértice $v$ es el \textbf{grado del vértice}, denotado por $\deg(v)$.
            \item Si el grado de todos los vértices de una gráfica es el mismo, decimos que la gráfica es \textbf{regular}.
        \end{enumerate}
    \end{mydef}

    \begin{mydef}
        Una gráfica se dice \textbf{completa} si todos los vértices son vecinos unos de otros (salvo él mismo).
    \end{mydef}

    \begin{exa}
        $(\left\{a,b \right\},\left\{\left\{a,b\right\} \right\})$ es completa y regular.

        $(\left\{a,b,c \right\},\left\{\left\{a,b\right\},\left\{a,c\right\},\left\{b,c\right\} \right\})$ es completa y regular.

        $(\left\{a,b,c,d \right\},\left\{\left\{a,b\right\},\left\{b,c\right\},\left\{c,d\right\},\left\{d,a\right\} \right\})$ no es completa y pero sí es regular.
    \end{exa}

    \begin{mydef}
        Sean $X$ y $Y$ gráficas.
        \begin{enumerate}
            \item Una función $\cf{f}{V(X)}{V(Y)}$ es de \textbf{gráficas} si envía aristas en aristas, es decir para todo $\left\{v,w \right\}\in E(X)\Rightarrow\left\{f(v),f(w) \right\}\in E(Y)$.
            \item Decimos que $X$ y $Y$ son \textbf{isomorfas} si existe una función de gráficas que es biyectiva.
        \end{enumerate}
    \end{mydef}

    \begin{mydef}
        Sea $X$ una gráfica. Un \textbf{camino de longitud $n\in\mathbb{N}\cup\left\{\infty\right\}$ en $X$} es una sucesión de vértices $v_0,v_1,...$ tal que $v_i\neq v_j$ si $i\neq j$ y $\left\{v_i,v_{ i+1} \right\}\in E(X)$.

        Si $n<\infty$, decimos que $v_0$ y $v_n$ son unidos por un camino.

        \begin{enumerate}
            \item Decimos que $X$ es \textbf{conexo} si cualesquiera dos vértices están unidos por un camino.
            \item Sea $n\in\mathbb{N}$. Un \textbf{ciclo} de longitud $n$ es un camino $v_0,...,v_{ n-1}$ en $X$ con $\left\{v_{ n-1},v_0 \right\}\in X$.
        \end{enumerate}
    \end{mydef}

    \begin{mydef}
        Decimos que una gráfica $X$ es un \textbf{árbol} si es conexa y no tiene ciclos.
    \end{mydef}

    \begin{propo}
        Una gráfica es un árbol si y sólo si para cualesquiera dos vértices existe un único camino que une a ambos vértices.
    \end{propo}

    \begin{proof}
        Ejercicio.
    \end{proof}

    \begin{mydef}
        Un grupo \textbf{$G$ actúa libremente en un conjunto $X$} si $g\cdot x\neq x$ para todo $g\in G\setminus\left\{e_G\right\}$.
    \end{mydef}

    \begin{theor}
        Un grupo $G$ actúa libremente en un árbol si y sólo si $G$ es grupo libre.
    \end{theor}

    \begin{proof}
        Esto será inmediata después de que veamos la parte de topología algebraica.
    \end{proof}

    Naturalmente surge la siguiente pregunta: ¿qué pasa si relajamos la condición de actuar libremente? ¿qué tipos de grupos pueden aparecer? Resulta que hay un teorema que enuncia lo que sucede en esta sucesión y aparecen productos libres de grupos, pushouts de grupos y grupos $HNN$. Esto es conocido como la \textbf{teoría de Basser-Serre}.

    \begin{mydef}
        Una \textbf{$n$-variedad} es un espacio topológico $(X,\tau)$ que localmente es homeomorfo a $\mathbb{R}^n$.
    \end{mydef}

    \begin{exa}
        Ejemplos de 3-variedades son $\mathbb{T}^3=\mathbb{S}^1\times\mathbb{S}^1\times\mathbb{S}^1$, $\mathbb{R}^3$, cualquier abierto de $\mathbb{R}^3$, una superficie $\Sigma$ producto con $\mathbb{S}^1$ es también una 3-variedad.
    \end{exa}

    Resulta que hay un teorema que si, tomamos una 3-variedad $M^3$, podemos ver a:
    \begin{equation*}
        M^3=M_1\#M_2\#\cdots\#M_r
    \end{equation*}
    (donde se está haciedo aquí suma conexa). Va a resultar que:
    \begin{equation*}
        \pi_1(M^3)=\pi_1(M_1)*\pi_1(M_2)*\cdots*\pi_1(M_r)
    \end{equation*}
    es el producto libre de estos grupos.

    \chapter{Ejercicios y Problemas\\ Teoría de Grupos}

    \section{Preliminares Teoría de Grupos}

    \begin{excer}
        Supongamos que $G$ es un grupo que tiene un subgrupo de índice finito $H$. Demuestra que $G$ tiene un subgrupo normal de índice finito. 
    \end{excer}

    \begin{proof}
        Sea:
        \begin{equation*}
            N=\gen{H}^\vartriangleleft
        \end{equation*}
        tenemos los siguientes subgrupos de $G$:
        \begin{equation*}
            H<N<G
        \end{equation*}
        que satisfacen (por ser el índice multiplicativo):
        \begin{equation*}
            [G:H]=[G:N][N:H]
        \end{equation*}
        como $[G:H]<\infty$, se sigue que $[G:N]<\infty$, con lo que $N$ es el subgrupo normal buscado.
    \end{proof}

    \begin{excer}
        ¿Cuál es el grupo de automorfismos del grupo aditivo $\mathbb{Z}$?
    \end{excer}

    \begin{sol}
        Considere al grupo de automorfismos del grupo aditivo $\mathbb{Z}$, digamos:
        \begin{equation*}
            A=\Aut{\mathbb{Z}}=\left\{\cf{f}{\mathbb{Z}}{\mathbb{Z}}\Big|f\textup{ es isomorfismo} \right\}
        \end{equation*}
        Afirmamos que $\Aut{\mathbb{Z}}\cong\mathbb{Z}/2\mathbb{Z}$ donde $\mathbb{Z}/2\mathbb{Z}$ es el grupo aditivo de los enteros módulo 2. En efecto, afirmamos que:
        \begin{equation*}
            \Aut{\mathbb{Z}}=\left\{\bbm{1}_{\mathbb{Z}},-\bbm{1}_{\mathbb{Z}}\right\}
        \end{equation*}
        donde $\cf{\bbm{1}_{\mathbb{Z}}}{\mathbb{Z}}{\mathbb{Z}}$ es la identidad de $\mathbb{Z}$ y $\cf{-\bbm{1}_{\mathbb{Z}}}{\mathbb{Z}}{\mathbb{Z}}$ es tal que $-\bbm{1}_{\mathbb{Z}}\left(m\right)=-m$ para todo $m\in\mathbb{Z}$. En efecto, es claro que $\left\{\bbm{1}_{\mathbb{Z}},-\bbm{1}_{\mathbb{Z}}\right\}\subseteq\Aut{\mathbb{Z}}$.

        Sea ahora $f\in\Aut{\mathbb{Z}}$, se tiene que:
        \begin{equation*}
            f(m)=f(\underset{m\textup{-veces}}{\underbrace{1+\cdots+1}})=\underset{m\textup{-veces}}{\underbrace{f(1)+\cdots+f(1)}}=mf(1)
        \end{equation*}
        para todo $m\in\mathbb{N}$. De forma análoga se demuestra que:
        \begin{equation*}
            f(-m)=-mf(1),\quad\forall m\in\mathbb{N}
        \end{equation*}
        Así que:
        \begin{equation*}
            f(m)=mf(1),\quad\forall m\in\mathbb{Z}
        \end{equation*}
        por lo que $f$ está únicamente determinada por su valor en $1$. Como $\mathbb{Z}$ tiene únicamente dos generadores (por ser un grupo cíclico infinito), al ser $f$ automorfismo debe suceder que $\mathbb{Z}=\gen{f(1)}$, así que $f(1)=1$ ó $f(1)=-1$, es decir que:
        \begin{equation*}
            \begin{split}
                f(m)&=mf(1)\\
                &=\left\{
                    \begin{array}{rl}
                        m & \textup{ si }f(1) = 1\\
                        -m & \textup{ si }f(1) = -1\\
                    \end{array}
                \right.\\
                &=\left\{
                    \begin{array}{rl}
                        \bbm{1}_\mathbb{Z}(m)& \textup{ si }f(1) = 1\\
                        -\bbm{1}_\mathbb{Z}(m)& \textup{ si }f(1) = 1\\
                    \end{array}
                 \right.\\
            \end{split}
        \end{equation*}
        es decir, que $f=\bbm{1}_\mathbb{Z}$ o $f=-\bbm{1}_{\mathbb{Z}}$. Por tanto, $\Aut{\mathbb{Z}}=\left\{\bbm{1}_\mathbb{Z},-\bbm{1}_\mathbb{Z}\right\}$. Para la otra parte, es inmediato que el grupo $\left\{\bbm{1}_\mathbb{Z},-\bbm{1}_\mathbb{Z}\right\}$ con la composición de funciones es isomorfo al grupo aditivo $\mathbb{Z}/2\mathbb{Z}$.
    \end{sol}

    \begin{excer}
        Supongamos que tenemos una sucesión exacta corta de grupos:
        \begin{equation*}
            1\rightarrow N\rightarrow G\rightarrow K\rightarrow 1
        \end{equation*}
        demuestra que si $N$ y $K$ son grupos finitamente generados, entonces $G$ es finitamente generado.
    \end{excer}

    \begin{proof}
        Al tenerse la sucesión exacta corta de grupos, estamos diciendo que existen homomorfismos $\cf{f_0}{\gen{1}}{N}$, $\cf{f_1}{N}{G}$, $\cf{f_2}{G}{K}$ y $\cf{f_3}{K}{\gen{1}}$ tales que:
        \begin{equation*}
            \im{f_{i-1}}=\ker\left(f_i \right),\quad\forall i=1,2,3
        \end{equation*}
        En particular, notemos que $f_1$ es monomorfismo y que $f_2$ es epimorfismo, ya que:
        \begin{equation*}
            \ker\left(f_1\right)=\im{f_0}=\gen{e_N}
        \end{equation*}
        siendo $e_N$ la identidad del grupo $N$ y, además:
        \begin{equation*}
            \im{f_2}=\ker\left(f_3\right)=K
        \end{equation*}
        por lo que se tiene lo afirmado.

        Supongamos ahora que $N$ y $K$ son finitamnete generados, entonces existen elementos $n_1,...,n_m\in N$ y $k_1,...,k_l\in K$ tales que:
        \begin{equation*}
            N=\gen{n_1,...,n_m}\quad\textup{y}\quad K=\gen{k_1,...,k_l}
        \end{equation*}
        Como $f_3$ es epimorfismo, entonces del Primer Teorema de Isomorfismo se sigue que:
        \begin{equation*}
            K\cong G/\ker(f_3)=G/\im{f_2}=G/N'
        \end{equation*}
        donde $N'=f_2(N)$.
        
        Afirmamos que:
        \begin{equation*}
            G=\gen{f_1(n_1),...,f_1(n_m),f_2^{-1}(k_1),...,f_2^{-1}(k_l)}
        \end{equation*}

    \end{proof}

    \begin{excer}
        Demuestra que en el producto semidirecto $N\rtimes_{\varphi}H$, $H$ es un subgrupo normal si y sólo si $\varphi$ es el homomorfismo trivial.
    \end{excer}

    \begin{proof}
        Recordemos que el producto semidirecto $N\rtimes_\varphi H$ es el grupo $N\times H$ dotado de la operación:
        \begin{equation*}
            (n,h)(n',h')=(n\varphi_h(n'),hh')
        \end{equation*}
        donde $\cf{\varphi}{H}{\Aut{N}}$ es un homomorfismo tal que $h\mapsto\varphi_h$. El elemento neutro de este grupo es $(e_N,e_H)$, donde cada elemento tiene como inverso:
        \begin{equation*}
            (n,h)^{-1}=\left((\varphi_{h^{-1}}(n))^{-1},h^{-1}\right)
        \end{equation*}

        Sean $(n_1,h_1)\in N\rtimes_{\varphi}H$ y $h\in H$, se tiene que:
        \begin{equation*}
            \begin{split}
                (n_1,h_1)(e_N,h)(n_1,h_1)^{-1}&=(n_1,h_1)(e_N,h)\left((\varphi_{h_1^{-1}}(n_1))^{-1},h_1^{-1}\right)\\
                &=(n_1\varphi_{h_1}(e_N),h_1h)\left((\varphi_{h_1^{-1}}(n_1))^{-1},h_1^{-1}\right)\\
                &=(n_1\varphi_{h_1}(e_N),h_1h)\left((\varphi_{h_1^{-1}}(n_1))^{-1},h_1^{-1}\right)\\
                &=(n_1e_N,h_1h)\left((\varphi_{h_1^{-1}}(n_1))^{-1},h_1^{-1}\right)\\
                &=(n_1,h_1h)\left((\varphi_{h_1^{-1}}(n_1))^{-1},h_1^{-1}\right)\\
                &=\left(n_1\varphi_{h_1h}\left((\varphi_{h_1^{-1}}(n_1))^{-1}\right),h_1hh_1^{-1} \right)\\
                &=\left(n_1\varphi_{h_1h}\left((\varphi_{h_1^{-1}}(n_1^{-1}))\right),h_1hh_1^{-1} \right)\\
                &=\left(n_1\varphi_{h_1h h_1^{-1}}\left(n_1^{-1}\right),h_1hh_1^{-1} \right)\\
            \end{split}
        \end{equation*}
        pues, $\varphi_{h_1}(e_N)=e_N$ y por ser $h\mapsto\varphi_h$ homomorfismo.

        $\Rightarrow)$: Suponga que $H$ es un subgrupo normal de $N\rtimes_\varphi H$, esto es que el grupo $H$ visto como subgrupo de $N\rtimes_\varphi H$:
        \begin{equation*}
            H=\left\{(e_N,h)\Big|h\in H \right\}
        \end{equation*}
        es subgrupo normal de $N\rtimes_\varphi H$. Como es normal, se sigue que:
        \begin{equation*}
            (n_1,h_1)(e_N,h)(n_1,h_1)^{-1}\in H
        \end{equation*}
        para todo $(n_1,h_1)\in N\rtimes_{\varphi}H$ y $h\in H$, por lo que:
        \begin{equation*}
            \left(n_1\varphi_{h_1h h_1^{-1}}\left(n_1^{-1}\right),h_1hh_1^{-1} \right)\in H
        \end{equation*}
        nuevamente, para todo $(n_1,h_1)\in N\rtimes_{\varphi}H$ y $h\in H$. En particular:
        \begin{equation*}
            n_1\varphi_{h_1h h_1^{-1}}\left(n_1^{-1}\right)=e_N
        \end{equation*}
        por lo que para todo $n\in N$ y $h\in H$:
        \begin{equation*}
            n^{-1}\varphi_{h}\left(n\right)=e_N\Rightarrow\varphi_h(n)=n
        \end{equation*}
        es decir, que $\varphi_h=\bbm{1}_H$, por lo que $h\mapsto \varphi_h$ es el homomorfismo trivial.
        
        $\Leftarrow):$ Suponga que $\varphi$ es trivial, se sigue que:
        \begin{equation*}
            \begin{split}
                (n_1,h_1)(e_N,h)(n_1,h_1)^{-1}&=\left(n_1\varphi_{h_1h h_1^{-1}}\left(n_1^{-1}\right),h_1hh_1^{-1} \right)\\
                &=(n_1\bbm{1}_H(n_1^{-1}),h_1hh_1^{-1})\\
                &=(n_1n_1^{-1},h_1hh_1^{-1})\\
                &=(e_N,h_1hh_1^{-1})\in H\\
            \end{split}
        \end{equation*}
        para todo $(n_1,h_1)\in N\rtimes_{\varphi}H$ y $h\in H$, por lo que $H$ es normal en $N\rtimes_{\varphi}H$.
    \end{proof}

    \begin{excer}
        Demuestra que el producto libre en $n$ generadores $F_n$ es isomorfo al producto libre de $n$ copias de $\mathbb{Z}$, $\mathbb{Z}*\mathbb{Z}*\cdots*\mathbb{Z}$.
    \end{excer}

    \begin{proof}
        
    \end{proof}

    \begin{excer}
        Demuestra que el producto libre $G*H$ de grupos no triviales $H$ y $G$ tiene centro trivial.
    \end{excer}

    \begin{proof}
        Sean $G$ y $H$ grupos no triviales. Considere $G*H$ su producto libre. El centro de $G*H$ se define por:
        \begin{equation*}
            Z(G*H)=\left\{x\in G*H\Big|xy=yx,\forall y\in G*H \right\}
        \end{equation*}
        Sea $u\in Z(G*H)$, se tiene que:
        \begin{equation*}
            ux=xu,\quad\forall x\in G*H
        \end{equation*}
        como $G$ y $H$ son no triviales, podemos tomar $u=gh$ donde $g\in G\setminus\left\{ e_G\right\}$ y $h\in H\setminus\left\{e_H \right\}$. Se sigue así que:
        \begin{equation*}
            ugh=ghu
        \end{equation*}
        Si $u\neq e_{G*H}$, entonces existirían $x_1,...,x_n\in G\cup H$ (alternándose un elemento con otro estando uno en $G$ y otro en $H$) junto con $m_1,...,m_n\in\mathbb{Z}$ tales que:
        \begin{equation*}
            u=x_1^{ m_1}\cdots x_n^{ m_n}
        \end{equation*}
        así que:
        \begin{equation*}
            x_1^{ m_1}\cdots x_n^{ m_n}gh=ghx_1^{ m_1}\cdots x_n^{ m_n}
        \end{equation*}
        reduciendo ambas palabras resulta que $x_n$ está en $G$ y $H$ a la vez, cosa que no puede suceder ya que ello implicaría que $x_i\in G\cap H$ para todo $i=1,...,n$. Por tanto, $u=e_{G*H}$.
    \end{proof}

    \begin{excer}
        Demuestra que $\mathbb{Z}_2*\mathbb{Z}_2$ es isomorfo a $\mathbb{Z}\rtimes\mathbb{Z}_2$.
    \end{excer}

    \begin{proof}
        
    \end{proof}

    \begin{excer}
        Denotemos por $F_n$ al grupo libre en $n$ generadores. Demuestre que $F_n$ es isomorfo a $F_m$ si y sólo si $n=m$.
    \end{excer}

    \begin{proof}
        Como $F_n$ es grupo libre en $n$ generadores y $F_m$ lo es en $m$, tomamos $x_1,...,x_n$ y $y_1,...,y_m$ tales que:
        \begin{equation*}
            F_n=
        \end{equation*}

        $\Rightarrow):$ Supongamos que $F_n$ es isomorfo a $F_m$.
    \end{proof}

    \begin{excer}
        Demuestra que todo grupo admite una presentación.
    \end{excer}

    \begin{proof}
        Sea $G$ un grupo y tomemos $S=G$. Considere $F(S)$ el grupo libre sobre el conjunto $S$. Sea $\cf{f}{S}{G}$ la función dada por:
        \begin{equation*}
            f(s)=s,\quad\forall s\in S=G
        \end{equation*}
        entonces, por la propiedad universal de grupos libres existe un único homomorfismo $\cf{\hat{f}}{F(S)}{G}$ tal que:
        \begin{equation*}
            \hat{f}\circ\iota=f
        \end{equation*}
        en particular, $\hat{f}$ es epimorfismo, pues:
        \begin{equation*}
            \hat{f}\circ\iota(S)=f(S)=G
        \end{equation*}
        así que, por el primer teorema de isomorfismos existe un único isomorfismo $\cf{g}{G}{F(S)/\ker(\hat{f})}$. Tomando:
        \begin{equation*}
            K=\ker(\hat{f})
        \end{equation*}
        se sigue que:
        \begin{equation*}
            \gen{K}^\vartriangleleft=\ker(\hat{f})
        \end{equation*}
        por lo que $G\cong F(S)/\gen{K}^\vartriangleleft$
    \end{proof}

    \begin{excer}
        Demuestra que el grupo con presentación:
        \begin{equation*}
            \gen{x,y|xyx^{-1}y^{-1}}
        \end{equation*}
        es isomorfo a $\mathbb{Z}^2$.
    \end{excer}

    \section{Acciones de Grupos}

    \begin{excer}
        Sean $G$ un grupo y $X$ un $G$-conjunto, es decir que $G$ actúa en $X$. Para $x\in X$ definimos el \textbf{estabilizador de $x$} como:
        \begin{equation*}
            G_x=\left\{g\in G\Big|gx=x \right\}
        \end{equation*}
        Sean $x,y\in X$ tales que existe $g\in G$ tal que $y=gx$. Demuestre que $G_y=gG_xg^{-1}$.
    \end{excer}

    \begin{proof}
        Veamos la doble contención:
        \begin{itemize}
            \item Sea $g_1\in G_y$, entonces $g_1y=y$, por lo cual $g_1gx=gx$, luego $g^{-1}g_1gx=g^{-1}gx=x$, así que $g^{-1}g_1g\in G_x$, por tanto $g_1\in gG_xg^{-1}$.
            \item Sea $gg_1g^{-1}\in gG_xg^{-1}$, entonces se cumple que $g_1x=x$, así que:
            \begin{equation*}
                \begin{split}
                    gg_1g^{-1}y&=gg_1(g^{-1}y)\\
                    &=gg_1x\\
                    &=gx\\
                    &=y\\
                \end{split}
            \end{equation*}
            por tanto, $gg_1g^{-1}\in G_y$.
        \end{itemize}
        por los dos incisos se sigue la igualdad.
    \end{proof}

    \begin{excer}
        Sean $G$ un grupo y $X$ un $G$-conjunto, es decir, $G$ actúa en $X$. Haga lo siguiente:
        \begin{enumerate}[label = \textit{(\alph*)}]
            \item Demuestra que la siguiente relación en $X$ es una relación de equivalencia: $x\sim y$ si y sólo si existe $g\in G$ tal que $y=gx$.
            \item Demuestra que $G_x$ es un subgrupo de $G$ para todo $x\in X$.
        \end{enumerate}
    \end{excer}

    \begin{proof}
        De \textit{(a)}: Veamos que la relación $\sim$ en $X$ es de equivalencia:
        \begin{itemize}
            \item (\textbf{Reflexividad}) Sea $x\in X$, entonces existe $e_G\in G$ tal que $x=e_Gx$, por lo cual $x\sim x$.
            \item \textbf{(Simetría)} Sean $x,y\in X$ tales que $x\sim y$, entonces existe $g\in G$ tal que $y=gx$. Se cumple además que existe $g^{-1}\in G$ tal que:
            \begin{equation*}
                \begin{split}
                    g^{-1}y&=g^{-1}(gx)\\
                    &=(g^{-1}g)x\\
                    &=e_Gx\\
                    &=x\\
                \end{split}
            \end{equation*}
            por lo cual, $y\sim x$.
            \item \textbf{(Transitividad)}: Sean $x,y,z\in X$ tales que $x\sim y$ y $y\sim z$, entonces existen $g_1,g_2\in G$ tales que:
            \begin{equation*}
                y=g_1x\textup{ y }z=g_2y
            \end{equation*}
            por lo cual:
            \begin{equation*}
                \begin{split}
                    z&=g_2y\\
                    &=g_2(g_1x)\\
                    &=(g_2g_1)x\\
                \end{split}
            \end{equation*}
            así que existe $g_2g_1\in G$ tal que $z=(g_2g_1)x$, por ende $x\sim z$.
        \end{itemize}
        de los tres incisos anteriores se sigue que $\sim$ es relación de equivalencia.

        De \textit{(b)}: Sea $x\in X$ y considere el conjunto $G_x$. Este conjunto es no vacío pues $e_G\in G_x$. Sean $a,b\in G_x$, se tiene que:
        \begin{equation*}
            x=ax\textup{ y }x=bx
        \end{equation*}
        en particular, de la segunda igualdad se tiene que $x=b^{-1}x$, por lo cual $x=ax=a(b^{-1}x)=(ab^{-1})x$. Por tanto, $ab^{-1}\in G_x$. Luego, $G_x$ es subgrupo de $G$.
    \end{proof}

    \begin{mydef}[\textbf{Árboles como espacios métricos}]
        Sea $T$ un árbol. Una \textbf{geodésica entre dos puntos $x_1$ y $x_2$ de $T$} es un camino de longitud mínima que une $x_1$ y $x_2$.
    \end{mydef}

    \begin{mydef}
        Sea $G$ un grupo actuando en una gráfica $Y$. Una \textbf{inversión} consiste de un elemento $g\in G$ y una arista $\left\{u,v \right\}$ de $Y$ tal que $g\left\{u,v\right\}=\left\{u,v \right\}$ y $gu=v$.
    \end{mydef}

    \begin{excer}
        Sean $T$ un árbol y $s$ un automorfismo de $T$. Si $\alpha$ es una geodésica, entonces $s\alpha$ es una geodésica.
    \end{excer}

    \begin{proof}
        Sea $\alpha$ una geodésica entre dos vértices del árbol $T$, digamos $v_1$ y $v_2$. Como $s$ es un automorfismo de $T$, entonces $s$ es una función de gráficas (que va de $T$ en sí misma) que es biyectiva.
        
        Además, al ser $s$ automorfismo se sigue que los vértices $s(v_1)$ y $s(v_2)$ son unidos por el camino $s\alpha$. Veamos que $s\alpha$ es una geodésica. En efecto, en caso de que no fuese un camino de longitud mínima, existiría otro camino, digamos $\beta$ que une a los vértices $s(v_1)$ con $s(v_2)$.

        Por ser $s$ automorfismo, podemos tomar automorfismo inverso $s^{-1}$ y sería tal que $s^{-1}\beta$ es un camino que une a los vértices $v_1$ y $v_2$, el cual debe tener longitud menor que el camino $\alpha$ (ya que los caminos $\alpha$ y $s\alpha$ tienen la misma longitud)\contradiction, pues $\alpha$ es una geodésica. Por tanto, $s\alpha$ es geodésica.
    \end{proof}

    \begin{propo}
        Sea $X$ una gráfica y $G$ un grupo actúando sin inversiones en $X$ y $H<G$. Si $x,y\in X$ son elementos distintos, entonces:
        \begin{equation*}
            x,y\in X^H\textup{ si y sólo si }\left\{x,y \right\}\in X^H
        \end{equation*} 
    \end{propo}

    \begin{proof}
        Recordemos que una acción de un grupo $G$ en una gráfica $X$ es simplemente un homomorfismo entre $G$ y $\Aut{X}$, $g\mapsto\varphi_g$, por lo que el hecho de que un punto $x\in X$ se quede fijo significa que:
        \begin{equation*}
            \varphi_g(x)=x,\quad\forall g\in G
        \end{equation*}

        $\Rightarrow):$ Suponga que $x,y\in X^H$, entonces para todo $h\in H$: $\varphi_h(x)=x$ y $\varphi_h(y)=y$. En particular, como $\varphi_h$ es automorfismo para todo $h\in G$, se tiene que el vértice:
        \begin{equation*}
            \left\{\varphi_h(x),\varphi_h(y) \right\}=\left\{x,y\right\}
        \end{equation*}
        debe estar en la gráfica $X$. En particular, sigue que $h\left\{x,y \right\}=\left\{x,y\right\}$ para todo $h\in H$, por lo que $\left\{x,y \right\}\in X^H$.

        $\Leftarrow):$ Suponga que $\left\{x,y\right\}\in X^H$, entonces:
        \begin{equation*}
            h\left\{x,y\right\}=\left\{x,y\right\},\quad\forall h\in H
        \end{equation*}
        como $G$ actúa sin inversiones, en particular no puede suceder que exista $g\in G$ tal que $g\left\{x,y \right\}=\left\{x,y \right\}$ y $y=gx$, así que en particular para todo $h\in H$ sucede que $hx=x$ y $hy=y$, es decir que $x,y\in X^H$.
    \end{proof}

    \begin{excer}
        Sea $T$ un árbol y $G$ un grupo actuando en $T$ sin inversiones. Sea $H$ un subgrupo de $G$ tal que el conjunto de puntos fijos:
        \begin{equation*}
            T^H=\left\{x\in T\Big|hx=x,\forall h\in H \right\}
        \end{equation*}
        es no vacío. Entonces, $T^H$ es un subárbol de $T$.
    \end{excer}

    \begin{proof}
        Es inmediato que $T^H$ es una subgráfica de $T$. Para ver que es subárbol de $T$ basta con ver que $T^H$ es conexo y que no tiene ciclos. En caso de que tenga solamente un vértice, es inmediato que es un árbol, por lo que supongamos que tiene a lo sumo 2 vértices.
        \begin{itemize}
            \item Sean $x,y\in T^H$ diferentes, entonces $hx=x$ y $hy=y$ para todo $h\in H$. Como $T$ es un árbol, existe un único camino $\alpha$ que une a $x$ con $y$, sean $v_1,...,v_n$. Procederemos por inducción sobre $n$. Para $n=2$ el resultado es inmediato. Suponga que existe $k\in\mathbb{N}$ tal que si $n\leq k$ y los vértices $v_1,v_n\in T^H$, entonces existe un camino que une a los puntos está en $T^H$.
            
            Sean $x,y\in T^H$ vértices tales que el camino $\alpha$ que los une tiene longitud $k+1$. Suponga que existe $i_0\in\natint{2,k}$ tal que $v_{ i_0}\notin T^H$ (ya que en caso contrario por hipótesis de inducción se tendría que existe un camino que une a $v_1$ con $v_{k+1}$). Por la proposición anterior se sigue que los vértices que unen a $v_{i_0}$ con $v_{i_0-1}$ y $v_{i_0+1}$ no están en $T^H$, respectivamente.

            Considere el conjunto:

            \item Veamos que no tiene ciclos. Suponga que $T^H$ tiene un cíclo, entonces $T$ no podría ser un árbol, ya que por ser $T^H$ subgráfica de $T$ se seguiría que $T$ tiene al menos un ciclo\contradiction. Por tanto, $T^H$ no tiene ciclos.
        \end{itemize}
        por ambos incisos se sigue el resultado.
    \end{proof}

    \begin{excer}
        Sea $\gamma$ la línea recta real con su estructura de gráfica canónica, es decir los vértices son los enteros $\mathbb{Z}$, y las aristas son $\left\{n,n+1\right\}$ con $n\in\mathbb{Z}$. Demuestra que el grupo $\Aut{\gamma}$ es isomorfo a $D_\infty$.
    \end{excer}

    \begin{proof}
        Recordemos que:
        \begin{equation*}
            D_\infty=\gen{r,s\Big|s^2=1\textup{ y }srs=r^{-1}}
        \end{equation*}
        Sea $f\in\Aut{\gamma}$, entonces $\cf{f}{\gamma}{\gamma}$ es función entre dos gráficas y es biyectiva.

        Sean $\cf{g,h}{\gamma}{\gamma}$ los automorfismos dados por:
        \begin{equation*}
            g(m)=-m,\quad\forall m\in\mathbb{Z}
        \end{equation*}
        y,
        \begin{equation*}
            h(m)=m+1,\quad\forall m\in\mathbb{Z}
        \end{equation*}
        Se tiene que:
        \begin{equation*}
            \begin{split}
                g\circ h\circ g(m)&=g\circ h(-m)\\
                &=g(-m+1)\\
                &=m-1\\
                &=h^{-1}(m),\quad\forall m\in\mathbb{Z} \\
            \end{split}
        \end{equation*}
        donde $\cf{h^{-1}}{\gamma}{\gamma}$ es tal que $m\mapsto m-1$. $g^2=\bbm{1}_\mathbb{Z}$.

        Afirmamos que $f$ se expresa como composición de $g$ y $h$ sucesivamente. En efecto, todo automorfismo de $\gamma$ está caracterizado por su valor en 0 y lo que haga con sus dos vecinos. Sea:
        \begin{equation*}
            n=f(0)
        \end{equation*}
        Se tienen dos casos:
        \begin{itemize}
            \item $n\geq0$: en cuyo caso se sigue que $f=\underset{n-\textup{veces}}{\underbrace{h\circ\cdots\circ h}}$.
            \item $n<0$, en cuyo caso se sigue que $f=\underset{-n-\textup{veces}}{\underbrace{h^{-1}\circ\cdots\circ h^{-1}}}=g\circ\underset{-n-\textup{veces}}{\underbrace{h\circ\cdots\circ h}}\circ g$.
        \end{itemize}
        en ambos casos, se obtiene el resultado.
    \end{proof}

    \begin{excer}
        Demuestra que todo grupo finito actuando en un árbol tiene un punto fijo.
    \end{excer}

    \begin{proof}
        Sea $T$ un árbol y $G\curvearrowright T$ una acción del grupo $G$ en $T$, siendo $G$ grupo finito. Veamos que $T$ tiene un punto fijo, es decir que existe $x\in T$ tal que:
        \begin{equation*}
            gx=x,\quad\forall g\in G
        \end{equation*}
        En efecto, suponga que no existe $x\in T$ tal que sucede lo anterior, por lo que para todo $x\in T$ existe $g_x\in G$ tal que:
        \begin{equation*}
            g_xx\neq x
        \end{equation*}
        Sea:
        \begin{equation*}
            \left\{g_xx\Big|x\in T \right\}
        \end{equation*}

    \end{proof}

\end{document}