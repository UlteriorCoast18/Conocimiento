\documentclass[12pt]{report}
\usepackage[spanish]{babel}
\usepackage[utf8]{inputenc}
\usepackage{amsmath}
\usepackage{amssymb}
\usepackage{amsthm}
\usepackage{graphics}
\usepackage{subfigure}
\usepackage{lipsum}
\usepackage{array}
\usepackage{multicol}
\usepackage{enumerate}
\usepackage[framemethod=TikZ]{mdframed}
\usepackage[a4paper, margin = 1.5cm]{geometry}

%En esta parte se hacen redefiniciones de algunos comandos para que resulte agradable el verlos%

\renewcommand{\theenumii}{\roman{enumii}}

\def\proof{\paragraph{Demostración:\\}}
\def\endproof{\hfill$\blacksquare$}

\def\sol{\paragraph{Solución:\\}}
\def\endsol{\hfill$\square$}

%En esta parte se definen los comandos a usar dentro del documento para enlistar%

\newtheoremstyle{largebreak}
  {}% use the default space above
  {}% use the default space below
  {\normalfont}% body font
  {}% indent (0pt)
  {\bfseries}% header font
  {}% punctuation
  {\newline}% break after header
  {}% header spec

\theoremstyle{largebreak}

\newmdtheoremenv[
    leftmargin=0em,
    rightmargin=0em,
    innertopmargin=-2pt,
    innerbottommargin=8pt,
    hidealllines = true,
    roundcorner = 5pt,
    backgroundcolor = gray!60!red!30
]{exa}{Ejemplo}[section]

\newmdtheoremenv[
    leftmargin=0em,
    rightmargin=0em,
    innertopmargin=-2pt,
    innerbottommargin=8pt,
    hidealllines = true,
    roundcorner = 5pt,
    backgroundcolor = gray!50!blue!30
]{obs}{Observación}[section]

\newmdtheoremenv[
    leftmargin=0em,
    rightmargin=0em,
    innertopmargin=-2pt,
    innerbottommargin=8pt,
    rightline = false,
    leftline = false
]{theor}{Teorema}[section]

\newmdtheoremenv[
    leftmargin=0em,
    rightmargin=0em,
    innertopmargin=-2pt,
    innerbottommargin=8pt,
    rightline = false,
    leftline = false
]{propo}{Proposición}[section]

\newmdtheoremenv[
    leftmargin=0em,
    rightmargin=0em,
    innertopmargin=-2pt,
    innerbottommargin=8pt,
    rightline = false,
    leftline = false
]{cor}{Corolario}[section]

\newmdtheoremenv[
    leftmargin=0em,
    rightmargin=0em,
    innertopmargin=-2pt,
    innerbottommargin=8pt,
    rightline = false,
    leftline = false
]{lema}{Lema}[section]

\newmdtheoremenv[
    leftmargin=0em,
    rightmargin=0em,
    innertopmargin=-2pt,
    innerbottommargin=8pt,
    roundcorner=5pt,
    backgroundcolor = gray!30,
    hidealllines = true
]{mydef}{Definición}[section]

\newmdtheoremenv[
    leftmargin=0em,
    rightmargin=0em,
    innertopmargin=-2pt,
    innerbottommargin=8pt,
    roundcorner=5pt
]{excer}{Ejercicio}[section]

%En esta parte se colocan comandos que definen la forma en la que se van a escribir ciertas funciones%

\newcommand\abs[1]{\ensuremath{\left|#1\right|}}
\newcommand\divides{\ensuremath{\bigm|}}
\newcommand\cf[3]{\ensuremath{#1:#2\rightarrow#3}}
\newcommand\natint[1]{\ensuremath{\left[\!\left[ #1\right]\!\right]}}
\newcommand{\afa}{\:
    \begin{tikzpicture}
        \draw [line width = 0.17 mm, black] (0,0) -- (-0.115,0.29);
        \draw [line width = 0.17 mm, black] (0,0) -- (0.115,0.29);
        \draw [line width = 0.17 mm, black] (-0.12,0) arc (190:-10:0.12cm);
    \end{tikzpicture}
    \:
}
%Este símvolo es para casi todo salvo una cantidad finita

%recuerda usar \clearpage para hacer un salto de página

\begin{document}
    \setlength{\parskip}{5pt} % Añade 5 puntos de espacio entre párrafos
    \setlength{\parindent}{12pt} % Pone la sangría como me gusta
    \title{Taller: Métodos algebraicos y tropicales para resolver ecuaciones diferenciales algebraicas}
    \author{Cristo Daniel Alvarado}
    \maketitle

    \tableofcontents %Con este comando se genera el índice general del libro%

    %\setcounter{chapter}{3} %En esta parte lo que se hace es cambiar la enumeración del capítulo%
    
    \chapter{Introducción}
    
    Curso impartido por Cristhian Garay, correo: cristhian.garay@cimat.mx.

    \section{Introducción y motivación}

    \newcommand{\norm}[1]{\ensuremath{\|#1\|}}

    \begin{obs}
        Las ecuaciones algebraicas son aquellas que usan solo productos de funciones y sus derivadas. Por ejemplo, la ecuación
        \begin{equation*}
            \ln u+\norm{u}\frac{\partial u}{\partial x}+\frac{\partial u}{\partial t}u^{ 1/2}=0
        \end{equation*}
        no es algebraica.
    \end{obs}

    Tradicionalmente el estudio de las ecuaciones diferenciales y sus soluciones pertenece al análisis y a los métodos geométricos.

    \begin{mydef}
        La \textbf{geometría algebraica} es el área de las matemática que estudia los objetos que localmente se describen como conjuntos de soluciones de sistemsa de ecuaciones polinomiales
        \begin{equation*}
            \begin{split}
                f_1(x_1,...,x_n)&=0\\
                \vdots\\
                f_m(x_1,...,x_n)&=0\\
            \end{split}
        \end{equation*}
    \end{mydef}

    EL álgebra conmutativa estudia detalladamente los modelos locales de la geometría algebraica, llamada también geometría algebraica afín.
    
    En general, gran mayoría de fenómenos se pueden modelar mediante sistemas de ecuaciones polinomiales. Entre estos destacan los lineales, es decir
    \begin{equation*}
        f_i(x_1,...,x_n)=\sum_{ i=1}^n a_{i,j}x_i+b_j,\quad\forall j\in\natint{1,m}
    \end{equation*}

    Notemos que se tiene que para calcular la variedad
    \begin{equation*}
        S=\left\{(x_1,...,x_m)\in\mathbb{C}^m\Big|AX=-B \right\}
    \end{equation*}
    reducimos el sistema a $A'X=-B'$ con las mismas soluciones. Por esta razón es que el álgebra lineal es un área particular de la geometría algebraica.

    \begin{mydef}
        Sea $\mathbb{C}[x_1,...,x_n]$ el anillo de polinomios en $n$ variables con coeficientes en $\mathbb{C}$. El \textbf{ideal generado por $A\subseteq\mathbb{C}[x_1,...,x_n]$} es el conjunto
        \begin{equation*}
            \langle A\rangle=\left\{\sum_{ i}h_if_i\textup{ finitas }\Big|f_i\in A, h_i\in\mathbb{C}[x_1,...,x_n]\right\}
        \end{equation*}
    \end{mydef}

    Para resolver problemas polinomiales se tiene como motivación las bases de Gröbner.

    \section{Modelación algebraica del problema}

    \begin{obs}
        Convendremos inicialmnete que $\mathbb{N}=\left\{0,1,2,...\right\}$
    \end{obs}

    Consideremos inicialmente la ecuación diferencial
    \begin{equation*}
        \frac{d^2u}{dt^2}=-\frac{k}{m}u
    \end{equation*}
    Hacemos
    \begin{equation*}
        x_0=u\quad\textup{y}\quad x_2=\frac{d^2u}{dt^2}
    \end{equation*}
    obtenemos así el polinomio que modela la ecuación diferencial
    \begin{equation*}
        f(x_0,x_2)=x_2+\frac{k}{m}x_0
    \end{equation*}

    Como otro ejemplo, considere la ecuación del oscilador lineal
    \begin{equation*}
        y_1'=-y_2\quad\textup{y}\quad y_2'=y_1
    \end{equation*}
    Hacemos
    \begin{equation*}
        \begin{split}
            x_{ 1,0}=y_1\quad&\textup{y}\quad x_{ 2,0}=y_2\\
            x_{ 1,1}=y_1'\quad&\textup{y}\quad x_{ 2,1}=y_2'\\
        \end{split}
    \end{equation*}
    obtenemos así el sistema
    \begin{equation*}
        \begin{split}
            f(x_{ 1,0},x_{ 2,0},x_{ 1,1},x_{ 2,1})&=x_{ 1,1}+x_{2,0}\\
            f(x_{ 1,0},x_{ 2,0},x_{ 1,1},x_{ 2,1})&=x_{ 2,1}-x_{1,0}\\
        \end{split}
    \end{equation*}
    El área que estudia este tipo de problemas, convirtiendo los problemas en polinomios es llamada \textbf{álgebra diferencial}.

    Analicemos el caso lineal, la secuaciones diferenciales lineales en $n$ variables diferenciales: $u_1,...,u_n$. Hallemos los vectores de las funciones $(u_1,...,u_n)$ con $u_i=u_i(t)$ siendo $t\in\natint{1,n}$, tales que:
    \begin{equation*}
        \sum_{ i=1}^n\sum_{ j=1}^r a_{ i,j}u_i^{(j)}+b=0,\quad a_{ i,j},b\in\mathbb{C}
    \end{equation*}
    el \textbf{orden} de la ecuación anterior es $r=\max_{ j}\left\{a_{ i,j}\neq0 \right\}$

    \begin{exa}
        Consideremos la siguiente ecuación
        \begin{equation*}
            a_{ 1,3}\frac{\partial^3u_1}{\partial t_1\partial t_2^2}+a_{ 2,4}\frac{\partial^4u_2}{\partial t_1^3\partial t_2}=0
        \end{equation*}
        haciendo $x_1=u_1$ y $x_2=u_2$, tenemos el siguiente polinomio:
        \begin{equation*}
            f(x_{1,(0,0)},x_{2,(0,0)},...,)a_{1,3}x_{ 1,(1,2)}+a_{2,4}x_{ 2,(3,1)}
        \end{equation*}
        que modela la ecuación diferencial anterior.
    \end{exa}

    \begin{exa}
        Consideremos la ecuación de Burgers:
        \begin{equation*}
            \frac{\partial u}{\partial t}+u\frac{\partial u}{\partial x}=\nu\frac{\partial^2 u}{\partial u^2}
        \end{equation*}
        se traduce algebraicamente como
        \begin{equation*}
            f(x_{ (1,0)},x_{ (0,0)},x_{ (0,1)},x_{ (0,2)})=x_{ (1,0)}+x_{(0,0)}x_{ (0,1)}-\nu x_{(0,2)}
        \end{equation*}
    \end{exa}

    \begin{mydef}
        Una \textbf{serie de potencias formal en $m\geq 1$ variables}, es una expresión de la forma
        \begin{equation*}
            \varphi=\sum_{ I\in\mathbb{N}^m}a_IT^I,\quad a_I\in\mathbb{C}
        \end{equation*}
        definimos las \textbf{derivadas formales} de la manera usual
        \begin{equation*}
            \frac{d\varphi}{d t_j}=\sum_{ I\in\mathbb{N}^m}i_j a_IT^{ I-e_j},\quad j\in\natint{1,m}
        \end{equation*}
        Dados
        \begin{itemize}
            \item $f=f(x_{ i,J}| i\in\natint{1,n}, J\in\mathbb{N}^m)$ un polinomio diferencial en las variables diferenciales $x_1,...,x_n$.
            \item $\varphi=(\varphi_1,...,\varphi_n)$ es un vector de series formales en $m$ variables.
        \end{itemize}
        La \textbf{evaluación diferencial} de $f$ en $\varphi$ es:
        \begin{equation*}
            evd_f(\varphi)=f(x_{ i,J}\mapsto\frac{\partial^{j_1+...+j_m}\varphi_i}{\partial^{ j_1}t_1\cdots\partial^{ j_m}t_m})
        \end{equation*}
        En este caso, decimos que $\varphi$ es \textbf{solución} de $f$ si $evd_f(\varphi)=0$.
    \end{mydef}

    \begin{mydef}
        Sea $\varphi=(\varphi_1,...,\varphi_n)$ con $\varphi_i=\sum_{ I\in\mathbb{N}^m}a_{ i,I}T^I$. La \textbf{evaluación algebraica} usual de $f=f(x_{ i,J}\Big|i\in\natint{1,n}, J\in\mathbb{N}^m)$ en $\varphi$ es
        \begin{equation*}
            ev_f(\varphi)=f(x_{ i,J}\mapsto a_{ i,J})
        \end{equation*}

    \end{mydef}

    \section{Herramientas algebraicas}

    \begin{mydef}
        El conjunto $\mathbb{C}[[T]]=\mathbb{C}[[t_1,...,t_m]]$ de series formales es un anillo conmutativo con 1. La pareja $(\mathbb{C}[[T]],D)$ es un \textbf{anillo diferencial} (siendo $D=\left\{\frac{\partial}{\partial t_1},...,\frac{\partial}{\partial t_m} \right\}$ operadores diferenciales).

        Un ideal $I\subseteq\mathbb{C}[[T]]$ es \textbf{diferencial} si dado $f\in I$ se tiene que $\frac{\partial f}{\partial t_j}\in I$ para $j\in\natint{1,m}$.
    \end{mydef}

    Sea $\mathbb{C}_{ m,n}$ el anillo de polinomios en infinitas (numerables) variables $\mathbb{C}[x_{ i,J}\Big| i\in\natint{1,n}, J\in\mathbb{N}^m]$. Como en el caos usual, los elementos de este anillo son sumas finitas de \textbf{monomios}:
    \begin{equation*}
        f=a_{M_1}E_{ M_1}+\cdots+
    \end{equation*}

    \chapter{Ejercicios}

\end{document}