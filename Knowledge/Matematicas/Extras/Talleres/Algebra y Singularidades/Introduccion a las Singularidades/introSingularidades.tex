\documentclass[12pt]{report}
\usepackage[spanish]{babel}
\usepackage[utf8]{inputenc}
\usepackage{amsmath}
\usepackage{amssymb}
\usepackage{amsthm}
\usepackage{graphics}
\usepackage{subfigure}
\usepackage{lipsum}
\usepackage{array}
\usepackage{multicol}
\usepackage{enumerate}
\usepackage[framemethod=TikZ]{mdframed}
\usepackage[a4paper, margin = 1.5cm]{geometry}
\usepackage{tikz}
\usepackage{pgffor}
\usepackage{ifthen}
\usepackage{enumitem}

\usetikzlibrary{shapes.multipart}

\newcounter{it}
\newcommand*\watermarktext[1]{\begin{tabular}{c}
    \setcounter{it}{1}%
    \whiledo{\theit<100}{%
    \foreach \col in {0,...,15}{#1\ \ } \\ \\ \\
    \stepcounter{it}%
    }
    \end{tabular}
    }

\AddToHook{shipout/foreground}{
    \begin{tikzpicture}[remember picture,overlay, every text node part/.style={align=center}]
        \node[rectangle,black,rotate=30,scale=2,opacity=0.04] at (current page.center) {\watermarktext{Cristo Daniel Alvarado ESFM\quad}};
  \end{tikzpicture}
}
%En esta parte se hacen redefiniciones de algunos comandos para que resulte agradable el verlos%

\def\proof{\paragraph{Demostración:\\}}
\def\endproof{\hfill$\blacksquare$}

\def\sol{\paragraph{Solución:\\}}
\def\endsol{\hfill$\square$}

%En esta parte se definen los comandos a usar dentro del documento para enlistar%

\newtheoremstyle{largebreak}
  {}% use the default space above
  {}% use the default space below
  {\normalfont}% body font
  {}% indent (0pt)
  {\bfseries}% header font
  {}% punctuation
  {\newline}% break after header
  {}% header spec

\theoremstyle{largebreak}

\newmdtheoremenv[
    leftmargin=0em,
    rightmargin=0em,
    innertopmargin=0pt,
    innerbottommargin=5pt,
    hidealllines = true,
    roundcorner = 5pt,
    backgroundcolor = gray!60!red!30
]{exa}{Ejemplo}[section]

\newmdtheoremenv[
    leftmargin=0em,
    rightmargin=0em,
    innertopmargin=0pt,
    innerbottommargin=5pt,
    hidealllines = true,
    roundcorner = 5pt,
    backgroundcolor = gray!50!blue!30
]{obs}{Observación}[section]

\newmdtheoremenv[
    leftmargin=0em,
    rightmargin=0em,
    innertopmargin=0pt,
    innerbottommargin=5pt,
    rightline = false,
    leftline = false
]{theor}{Teorema}[section]

\newmdtheoremenv[
    leftmargin=0em,
    rightmargin=0em,
    innertopmargin=0pt,
    innerbottommargin=5pt,
    rightline = false,
    leftline = false
]{propo}{Proposición}[section]

\newmdtheoremenv[
    leftmargin=0em,
    rightmargin=0em,
    innertopmargin=0pt,
    innerbottommargin=5pt,
    rightline = false,
    leftline = false
]{cor}{Corolario}[section]

\newmdtheoremenv[
    leftmargin=0em,
    rightmargin=0em,
    innertopmargin=0pt,
    innerbottommargin=5pt,
    rightline = false,
    leftline = false
]{lema}{Lema}[section]

\newmdtheoremenv[
    leftmargin=0em,
    rightmargin=0em,
    innertopmargin=0pt,
    innerbottommargin=5pt,
    roundcorner=5pt,
    backgroundcolor = gray!30,
    hidealllines = true
]{mydef}{Definición}[section]

\newmdtheoremenv[
    leftmargin=0em,
    rightmargin=0em,
    innertopmargin=0pt,
    innerbottommargin=5pt,
    roundcorner=5pt
]{excer}{Ejercicio}[section]

%En esta parte se colocan comandos que definen la forma en la que se van a escribir ciertas funciones%

\newcommand\abs[1]{\ensuremath{\left|#1\right|}}
\newcommand\divides{\ensuremath{\bigm|}}
\newcommand\cf[3]{\ensuremath{#1:#2\rightarrow#3}}
\newcommand\contradiction{\ensuremath{\#_c}}
\newcommand\natint[1]{\ensuremath{\left[\big|#1\big|\right]}}

\begin{document}
    \setlength{\parskip}{5pt} % Añade 5 puntos de espacio entre párrafos
    \setlength{\parindent}{12pt} % Pone la sangría como me gusta
    \title{Introducción a las singularidades}
    \author{Cristo Daniel Alvarado}
    \maketitle

    \tableofcontents %Con este comando se genera el índice general del libro%  

    \chapter{Nociones Básicas}

    \section{Preeliminares Algebraicos}

    \begin{mydef}
        Un anillo $R$ es \textbf{graduado (por $\mathbb{N}$)} si $R$ puede ser escrito como la suma directa (como grupo abeliano):
        \begin{equation*}
            R=\bigoplus_{ n=0}^\infty R_n
        \end{equation*}
        tal que para todos $m,n\in\mathbb{Z}_{\geq0}$ tenemos que $A_nA_m\subseteq A_{ n+m}$. Se sigue en particular que $A_0$ es un subanillo y que cada componente $A_n$ es un $A_0$-módulo.
    \end{mydef}

    
    
    \section{Variedades Algebraicas}

    En síntesis, las singularidades abarcan muchas ramas de las matemáticas, como son la geometría algebraica, el álgebra conmutativa, el análisis compleo, la topología algebraica y cosas sobre teoría de nudos.
    
    Considremos a $K$ un campo (o cuerpo), en ocasiones este puede ser considerado simplemente como un anillo, el cuál siempre será de característica 0.

    En el anillo de polinomios $K[x_1,...,x_n]$ tenemos los monomios
    \begin{equation*}
        x^d=x_1^{d_1}\cdots x_n^{d_n}
    \end{equation*}
    donde $d_1+...+d_n=d$. Así que todo polinomio $f$ se puede ver como:
    \begin{equation*}
        f=\sum_{\textup{finita}}c_dx^d
    \end{equation*}
    donde $c_d\in K\setminus\left\{0 \right\}$. Se define el \textbf{grado de $f$} por:
    \begin{equation*}
        \deg f=\max\left\{d_1+...+d_n\Big|c_d\neq 0 \right\}
    \end{equation*}

    \begin{exa}
        El anillo de polinomios $K[x_1,...,x_n]$ es graduado, a saber los subgrupos abelianos que lo graduano son aquellas polinomios con todas sus componentes de mismo grado. En este caso, 
    \end{exa}

    Consideramos el \textbf{espacio afín} $K^n$ de todas las tuplas $(a_1,...,a_k)$. Podemos también ver el \textbf{espacio proyectivo} $\mathbb{P}^n_k$, con coordenadas homogéneas $[x_0:x_1:...:x_n]$.

    \begin{obs}
        En las coordenadas homogéneas, $[x_0:x_1:...:x_n]$ es tal que $x_i$ no es cero para todo $i$. En particular también se tiene que:
        \begin{equation*}
            [x]=[\lambda x]=[\lambda x_0:\lambda x_1:...:\lambda x_n]
        \end{equation*}
        con $\lambda\in K\setminus\left\{ 0\right\}$
    \end{obs}

    \begin{obs}
        Podemos descomponer a la variedad proyectiva $\mathbb{P}^n_k$ como:
        \begin{equation*}
            \mathbb{P}^n_k=K^n\cup \mathbb{P}^{ n-1}_k
        \end{equation*}
        donde la primera parte es una variedad afín y la segunda es un hiperplano en el infinito (no sé a qué se refiera esto). Repitiendo este proceso podemos verlo como:
        \begin{equation*}
            \mathbb{P}^n_k=K^n\cup K^{ n-1}\cup\cdots\cup K\cup p^t
        \end{equation*}
    \end{obs}

    \begin{obs}
        Podemos también descomponer al espacio proyectivo como:
        \begin{equation*}
            \mathbb{P}^n_k=\bigcup_{ i=0}^n U_i
        \end{equation*}
        donde
        \begin{equation*}
            U_i=\left\{[x]\Big|x_i\neq0 \right\}
        \end{equation*}
        cada uno de estos $U_i$ es isomorfo a $K^n$, con isomorfismo dado por:
        \begin{equation*}
            [x]=[x_0:x_1:...:x_{ i-1}:x_i:x_{ i+1}:...:x_n]\mapsto \left(\frac{x_0}{x_i},\frac{x_1}{x_i},\cdots,\frac{x_{ i-1}}{x_i},\frac{x_{ i+1}}{x_i},...,\frac{x_n}{x_i} \right)
        \end{equation*}
    \end{obs}

    Consideraremos variedades algebraicas:
    \begin{equation*}
        V(f)=\left\{x\in K^n\Big|f(x)=0 \right\}
    \end{equation*}

    \begin{mydef}
        Decimos que un polinomio $F\in K[x_0,x_1,...,x_n]$ es \textbf{homogéneo}, si todos sus monomios tienen el mismo grado.
    \end{mydef}

    \begin{obs}
        La definición anterior es equivalente a que para todo $\lambda\in K$:
        \begin{equation*}
            F(\lambda x)=\lambda^{\deg F}F(x)
        \end{equation*}
        para todo $x=(x_0,x_1,...,x_n)\in K^{ n+1}$.
    \end{obs}

    \begin{mydef}
        Si $F$ es homogéneo, entonces $V(F)$ es una \textbf{hipersuperficie}.
    \end{mydef}

    Podemos hacer un proceso para deshomogeneizar un polinomio homogéneo, de la siguiente manera:

    \begin{equation*}
        F\left(1,\frac{x_1}{x_0},...,\frac{x_n}{x_0} \right)=f(x_1,...,x_n)
    \end{equation*}

    y, podemos homogeneizar un polinomio haciendo:
    \begin{equation*}
        F(x_0,x_1,...,x_n)=x^{\deg f}f\left(\frac{x_1}{x_0},...,\frac{x_n}{x_0} \right)
    \end{equation*}

    \begin{exa}
        Considere el polinomio $f=3+x_1+x_2$, entonces $F$ homogéneo sería:
        \begin{equation*}
            \begin{split}
                F(x_0,x_1,x_2)&=x_0^1f\left(\frac{x_1}{x_0},\frac{x_2}{x_0}\right)\\
                &=3x_0+x_1+x_2\\
            \end{split}
        \end{equation*}
    \end{exa}

    \begin{obs}
        En ocasiones interesa que $K$ sea algebraicamente cerrado. En este caso, se nos permite escribir un polinomio como:
        \begin{equation*}
            f=c\cdot(x-a_1)(x-a_2)\cdots (x-a_d),\quad a_i\in K
        \end{equation*}
        donde $d$ es el grado del polinomio, esto para polinomios en una variable.
    \end{obs}

    \begin{obs}
        En el caso en que $F$ sea un polinomio homogéneo en varias variables, podemos escribirlo como:
        \begin{equation*}
            F=c\cdot( b_1x-a_1y)\cdots( bd_x-a_dy),\quad a_i,b_i\in K
        \end{equation*}
        por lo que resulta importante tener la noción de polinomio homogéneo.
    \end{obs}

    \begin{mydef}
        Dados $f=a_0x^m+a_1x^{ m-1}+\cdots+a_{ m-1}x+a_m$ y $g=b_0x^n+b_1x^{ n-1}+\cdots+b_{ n-1}x+b_n$. Se define el \textbf{resultante de $f$ y $g$}, como:
        \begin{equation*}
            Res(f,g)=\det A_{ m+n}(a_i,b_j)
        \end{equation*}
    \end{mydef}

    Esta matriz se vería de esta manera:
    \begin{equation*}
        \left(
            \begin{array}{cccccccccc}
                a_0 & a_1 & a_2 & \cdots & a_m & 0 & 0 & \cdots & 0 \\
                0 & a_0 & a_1 &  \cdots & a_{ m-1} & a_m & 0 & \cdots & 0 \\
                0 & 0 & a_0 & \cdots & a_{ m-2} & a_{ m-1} & a_m & \cdots & 0 \\
                \vdots & \vdots & \vdots & \vdots  & \vdots & \vdots & \vdots & \cdots & \vdots \\
                0 & 0 & 0 &  \underset{(n-1)\textup{-veces recorrido}}{\underbrace{\cdots}} & a_0 & a_1 & a_2 & \cdots & a_m \\
                b_0 & b_1 & b_2 & \cdots & b_n & 0 & 0 & \cdots & 0 \\
                0 & b_0 & b_1 &  \cdots & b_{ n-1} & b_n & 0 & \cdots & 0 \\
                0 & 0 & b_0 & \cdots & b_{ n-2} & b_{ n-1} & b_n& \cdots & 0 \\
                \vdots & \vdots & \vdots & \vdots  & \vdots & \vdots & \vdots & \cdots & \vdots \\
                0 & 0 & 0 &  \underset{(m-1)\textup{-veces recorrido}}{\underbrace{\cdots}} & b_0 & b_1 & b_2 & \cdots & b_n \\
            \end{array}
            \right)
    \end{equation*}

    \begin{propo}
        $Res(f,g)=0$ si y sólo si $f$ y $g$ tienen una raíz común.
    \end{propo}

    \begin{proof}
        $\Rightarrow):$

        $\Leftarrow):$ Suponga que existe $r\in K$ tal que $f(r)=g(r)$, entonces:
        \begin{equation*}
            f(x)=(x-r)p(x)\quad\textup{y}\quad g(x)=(x-r)q(x)
        \end{equation*}
        donde $\deg p=m-1$ y $\deg q=n-1$. Se cumple además la igualdad:
        \begin{equation*}
            fq-gp=0
        \end{equation*}
        la ecuación anterior, la podemos ver como la matriz cuadrada $B_{m+n}(a_i,b_j)$ de tamaño $m+n$. Si hacemos
        \begin{equation*}
            p(x)=\alpha_0x^{ m-1}+\cdots+\alpha_{ m-1}
        \end{equation*}
        y,
        \begin{equation*}
            q(x)=\beta_0x^{ n-1}+\cdots+\beta_{ n-1}
        \end{equation*}
        Se reduciría todo a un sistema:
        \begin{equation*}
            B_{ n+m}(a_i,b_j)\left[ 
                \begin{array}{c}
                    \alpha_i \\
                    \beta_j \\
                \end{array}
            \right]=\left[ 
                \begin{array}{c}
                    0 \\
                    \vdots \\
                    0 \\
                \end{array}
            \right]
        \end{equation*}
        (completar la demostración).
    \end{proof}

    \begin{excer}
        Hacer lo de la proposición anterior cuando $f_1=f_2=x^2-3x+2$ y $g_1=x-1$ (calcular los sistemas necesarios).
    \end{excer}

    \begin{sol}
        
    \end{sol}

    \begin{exa}
        Considere los polinomios $f=x^3-3x^2+2x+1$ y $g=x^2-x+2$. Entonces $m=3$ y $n=2$, por lo que:
        \begin{equation*}
            A_{5}=\left(
                \begin{array}{ccccc}
                    1 & -3 & 2 & 1 & 0 \\
                    0 & 1 & -3 & 2 & 1 \\
                    1 & -1 & 2 & 0 & 0 \\
                    0 & 1 & -1 & 2 & 0 \\
                    0 & 0 & 1 & -1 & 2 \\
                \end{array}
            \right)
        \end{equation*}
        sería la matriz asociada al resultante de los polinomios $f$ y $g$.
    \end{exa}

    Para la siguiente proposición, $K$ es un campo algebraicamente cerrado.

    \begin{propo}
        Sean $f,g\in K[\underline{x}]$ (anillo de polinomios en varias variables). Entonces:
        \begin{enumerate}
            \item $V(f)=V(g)$ si y sólo si $f$ y $g$ tienen las mismas componentes irreducibles.
            \item $V(f)\neq\emptyset$ si y sólo si $f\in K\setminus\left\{0\right\}$.
        \end{enumerate}
    \end{propo}

    \begin{proof}
        
    \end{proof}

    \begin{mydef}
        Sea $p\in V(f)\subseteq K^n$. Decimos que $p$ es un \textbf{punto singular de $V(f)$}, si
        \begin{equation*}
            f(p)=\frac{\partial f}{\partial x_i}(p)=0
        \end{equation*}
        para todo $i=1,...,n$. El conjunto de puntos singulares de $f$ se denota por $Sing(V(f))$. Si $p\notin Sing(V(f))$, se dice que $p$ es \textbf{no singular} o \textbf{liso}.

        Si $V(f)$ es tal que $Sing(V(f))=\emptyset$, se dice que $V(f)$ es \textbf{no singular}.
    \end{mydef}

    \begin{exa}
        Considere el polinomio $f=ax+by$, $a,b\in K$ no ambas nulas. Entonces, $V(f)$ es no singular.
    \end{exa}

    \begin{exa}
        Considere $f=xy$. Entonces:
        \begin{equation*}
            Sing(V(f))=\left\{(0,0,*,*,...,*)\in K^n \right\}
        \end{equation*}
        En el caso de $K^n=\mathbb{C}^2$, se tiene que:
        \begin{equation*}
            Sing(V(f))=\left\{(0,0) \right\}\subseteq\mathbb{C}^2
        \end{equation*}
        se dice \textbf{singularidad aislada}.

        Si estamos en $\mathbb{C}^3$, entonces
        \begin{equation*}
            Sing(V(f))=\left\{(0,0,*) \right\}\subseteq\mathbb{C}^3
        \end{equation*}
        es \textbf{no aislada}.
    \end{exa}

    \begin{exa}
        En el caso en que $f=f_1\cdot f_2$, se tiene que $V(f_1)\cap V(f_2)\subseteq Sing(V(f))$.
    \end{exa}

    \begin{exa}
        Los siguientes tienen puntos singulares de diferentes tipos:
        \begin{itemize}
            \item $g=y^2-x^3$.
            \item $h=y^2-x^2(x+1)$.
            \item $k=z^2-xy^3$.
        \end{itemize}
    \end{exa}

    \section{Geometría y Topología de Curvas Algebraicas en $\mathbb{P}^2_{\mathbb{C}}$ (o en $\mathbb{C}^2$).}

    En esta parte, tendremos como objetivos dos cosas:

    \begin{enumerate}[label = \textit{(\arabic*)}]
        \item Entender la topología abstracta de $C=V(F)\subseteq\mathbb{P}^2_{\mathbb{C}}$.
        \item Entender la geometría de $C=V(F)\subseteq\mathbb{P}^2_{\mathbb{C}}$.
    \end{enumerate}

    \begin{theor}[\textbf{Teorema de Bezout}]
        Sean $C=V(P)$ y $D=V(Q)$ curvas contenidas en $\mathbb{P}^2_{\mathbb{C}}$ con $\deg P=n$ y $\deg Q=m$. Entonces, $C\cap D$ es un conjunto de $n\cdot m$ puntos (contando multiplicidades).
    \end{theor}

    \begin{theor}[\textbf{Fórmula de género-grado}]
        Sea $C=V(P)\subseteq\mathbb{P}^2_{\mathbb{C}}$ no singular y de grado $n$ irreducible. Entonces, $C$ es topológicamente una superficie (dimensión 2 sobre $\mathbb{R}$) conexa, compacta, orientable y sin borde con $\chi=2-(n-1)(n-2)$ (siendo $\chi$ la característica de Euler de la superficie).
    \end{theor}

    Luego hubo una explicación sobre la característica de Euler para superficices (en particular, algunas triangulaciones de la 2-esfera).

    \begin{theor}[\textbf{Teorema de Clasificación de Superficies}]
        La característica de Euler de toda superficie compacta, orientable, conexa y sin borde es:
        \begin{equation*}
            \chi=2-2g
        \end{equation*}
        donde $g$ es el género de la superficie.
    \end{theor}

    Notemos que:
    \begin{equation*}
        \begin{tabular}{ c | c }
            $n$ & $g=\frac{(n-1)(n-2)}{2}$ \\
            \hline
            1 & 0 \\
            2 & 0 \\
            3 & 1 \\
            4 & 3 \\
            5 & 6 \\
            6 & 10 \\
        \end{tabular}
    \end{equation*}
    por lo que no todos los géneros se pueden obtener a partir de curvas $C\subseteq\mathbb{P}_{\mathbb{C}}^2$.

    Uno puede construir todas las superifices orientables, conexas, compactas y sin borde a partir de la identificación usual que se hacía con la esfera, el toro, el 2-toro, etc...

    Hablaremos del teorema de Bezout pero desde el punto de vista de resultantes con polinomios en varias variables. Recordemos que si $f,g\in\mathbb{C}[x]$, entonces
    \begin{equation*}
        Res(f,g)=\det A_{ m+n}(a_i,b_j)
    \end{equation*}
    siendo
    \begin{equation*}
        \left(
            \begin{array}{cccccccccc}
                a_0 & a_1 & a_2 & \cdots & a_m & 0 & 0 & \cdots & 0 \\
                0 & a_0 & a_1 &  \cdots & a_{ m-1} & a_m & 0 & \cdots & 0 \\
                0 & 0 & a_0 & \cdots & a_{ m-2} & a_{ m-1} & a_m & \cdots & 0 \\
                \vdots & \vdots & \vdots & \vdots  & \vdots & \vdots & \vdots & \cdots & \vdots \\
                0 & 0 & 0 &  \underset{(n-1)\textup{-veces recorrido}}{\underbrace{\cdots}} & a_0 & a_1 & a_2 & \cdots & a_m \\
                b_0 & b_1 & b_2 & \cdots & b_n & 0 & 0 & \cdots & 0 \\
                0 & b_0 & b_1 &  \cdots & b_{ n-1} & b_n & 0 & \cdots & 0 \\
                0 & 0 & b_0 & \cdots & b_{ n-2} & b_{ n-1} & b_n& \cdots & 0 \\
                \vdots & \vdots & \vdots & \vdots  & \vdots & \vdots & \vdots & \cdots & \vdots \\
                0 & 0 & 0 &  \underset{(m-1)\textup{-veces recorrido}}{\underbrace{\cdots}} & b_0 & b_1 & b_2 & \cdots & b_n \\
            \end{array}
            \right)
    \end{equation*}
    con $\deg f=m$ y $\deg g = n$ (siendo $a_i$ los coeficientes de $f$ y $b_j$ los de $g$). Si consideramos ahora polinomios en 3 variables:
    \begin{equation*}
        f(x,y,z)=a_0(x,y)z^m+a_1(x,y)z^{ m-1}+\cdots+a_m(x,y)
    \end{equation*}
    y,
    \begin{equation*}
        g(x,y,z)=b_0(x,y)z^n+b_1(x,y)z^{ n-1}+\cdots+b_n(x,y)
    \end{equation*}
    tomamos a los polinomios $f,g\in\mathbb{C}[x,y][z]=\mathbb{C}[x,y,z]$ homogéneos. En este caso, los grados de $f$ y $g$ son $m$ y $n$, respectivamente, por lo que $a_i(x,y)$ y $b_j(x,y)$ son polinomios homogéneos de grado $i$ y $j$, respectivamente.

    \begin{mydef}
        Sean $F,G\in\mathbb{C}[x,y][z]$ polinomios homogéneos de grados $m$ y $n$, respectivamente. Entonces:
        \begin{equation*}
            Res_{z}(F,G)=\det A_{ m+n}(a_i(x,y),b_j(x,y))
        \end{equation*}
        con el $A$ dado como se hizo anteriormente.
    \end{mydef}

    \begin{obs}
        Se tiene que $Res_{z}(F,G)\in\mathbb{C}[x,z]$ es un polinomio homogéneo de grado $n\cdot m$ (a lo más ya que puede ser cero). Por tanto,
        \begin{equation*}
            Res_z(F,G)=\prod_{ i=n}^{ n\cdot m}(b_ix+a_iy)
        \end{equation*}
        (por ser $\mathbb{C}$ algebraicamente cerrado).
    \end{obs}

    \begin{obs}
        Dados $a,b\in\mathbb{C}$, hacemos:
        \begin{equation*}
            F(a,b,z)=f(z)\quad\textup{y}\quad G(a,b,z)=g(z)
        \end{equation*}
        entonces,
        \begin{equation*}
            Res_z(F,G)(a,b)=Res(f,g)
        \end{equation*}
        
        Recordemos que $Res(f,g)=0$ si existe $c\in\mathbb{C}$ tal que $f(c)=g(c)$.
    \end{obs}

    Por tanto, de las observaciones anteriores, se tiene que para cada $i=1,...,n\cdot m$ existen $c_i$ tales que
    \begin{equation*}
        f(c_i)=g(c_i)=0
    \end{equation*}
    esto es que
    \begin{equation*}
        F(a,b,c_i)=G(a,b,c_i)
    \end{equation*}

    \begin{obs}
        Se tiene que $C\cap D\neq\emptyset$ ¿?.
    \end{obs}

    \begin{propo}
        $Res_z(F,G)=0\in\mathbb{C}[x,y]$ si y sólo si $F$ y $G$ tienen una componente común.
    \end{propo}

    \begin{proof}
        Procederemos por reducción al absurdo. Supongamos que
        \begin{itemize}
            \item $Res_z(F,G)\in\mathbb{C}[x,y]$ y es no constante.
            \item $F$ y $G$ tienen al menos $n\cdot m+1$ puntos comunes.
        \end{itemize}
        Podemos tomar coordeadas de modo que cada punto común a $F$ y $G$ induce un factor lineal $b_ix+a_iy$ de $Res_z(F,G)$ y son no proporcionales dos a dos. Por lo cual al menos hay $n\cdot m+1$ factores lineales\contradiction.
    \end{proof}

    \begin{mydef}
        Sean $C,D$ dos curvas en $\mathbb{P}_\mathbb{C}^2$ tales que:
        \begin{itemize}
            \item $[0:0:1]\notin C\cup D$.
            \item $[0:0:1]$ no pertenece a una recta por dos puntos de $C\cap D$.
            \item $[0:0:1]$ no pertenece a la tangente de $C$ ni a la tangente a $D$ por un punto común $Q\in C\cap D$.
        \end{itemize}
        y, sea $O=[a:b:c]\in\mathbb{P}_{\mathbb{C}}^2$. Definimos la \textbf{multiplicidad de intersección de $O$}
        \begin{equation*}
            I_O(C,D)=\left\{
                \begin{array}{lcr}
                    0 & \textup{ si } & 0\notin C\cap D\\
                    \max\left\{k \right\} & \textup{ t.q. } & (bx-ay)^k\divides Res_z(F,G) \\
                    \infty & \textup{ si } & O\textup{ pertenece a una componente común a $C$ y $D$} \\
                \end{array}
            \right.
        \end{equation*}
    \end{mydef}

    Con la definición anterior se sigue que:
    \begin{equation*}
        \begin{split}
            Res_z(F,G)&=\prod_{ O=[a,b,c]\in C\cap D}(bx-ay)^k\\
        \end{split}
    \end{equation*}
    por lo cual:
    \begin{equation*}
        \sum_{ I\in C\cap D}I_O(F,G)=n\cdot m
    \end{equation*}

    Un puede definir la multiplicidad de un punto en una curva $C$, a partir de ver la mínima derivada parcial donde el punto no se anula, denotada por $mult_O(C)$. Se tiene que:
    \begin{equation*}
        I_O(C,D)\geq mult_O(C)+mult_O(D)
    \end{equation*}

    Ahora hablaremos de la fórumla de grado-género.

    \begin{theor}[\textbf{Teorema de la función implícita}]
        Sea $\underline{0}=(0,0,z)\in C=V(f)\subseteq\mathbb{C}^2$ tal que $f_y(\underline{0})\neq0$, entonces existen entornos abiertos $O_1\in V\subseteq \mathbb{C}$ y $\underline{O}\in U\subseteq C$ y una función analítica $\cf{g}{V}{U}$ tal que
        \begin{equation*}
            f(x,g(x))=0,\quad\forall x\in V
        \end{equation*}
    \end{theor}

    Por tanto, una curva lisa $C$ y $p\in C$ implica que $\frac{\partial f}{\partial x}(p)\neq 0$ o bien $\frac{\partial f}{\partial y}(p)\neq0$, por lo que del teorema de la función implícita se sigue que un entorno de $p$ en $C$ es homeomorfo a un entnro de $\mathbb{C}$ (de $\mathbb{R}^2$).
    
    Por lo que, $C$ es efectivamente una superficie.

    Se sabe de cursos de topología de conexión por arcos implica conexión.

    Considere puntos en la curva $p,q\in C$. Se tiene que existe $\cf{\gamma}{[0,1]}{C}$ continua tal que $\gamma(0)=q$ y $\gamma(1)=p$.

    Tenemos la curva $C=V(f)$. Podemos suponer que
    \begin{equation*}
        f=y^n+a_1(x)y^{ n-1}+\cdots+a_{ n-1}(x)y+a_n(x)
    \end{equation*}
    se tiene que $f(x_0,y)\in\mathbb{C}[y]$ es tal que $\deg f(x_0,y)=n$. Si consideramos la proyección $\cf{\pi_1}{C}{\mathbb{C}}$, entonces $\pi^{-1}(x_0)$ nos dará las raíces distinas de $f(x_0,y)$.

    Se define:
    \begin{equation*}
        Disc(f(x_0,y))=Res(f(x_0,y),f_y(x_0,y))
    \end{equation*}
    entonces
    \begin{equation*}
        \abs{\pi^{-1}(x_0)}=n\iff Disc(f(x_0,y))\neq0
    \end{equation*}

    En conclusión, $C\setminus\bigcup_{ q\in\mathbb{C} }\pi_1^{-1}(q)$ donde $q$ es raíz del discriminante, es un recubrimiento de $n$-hojas.

    Lo demás se deduce de forma más sencilla.

    Por último, la fórmula del género se deduce de otro hecho que no se menciona adecuadamente aquí.

    \section{Algoritmo de Newton-Puiseux}

    Básicmaente esto es una generalización del teorema de la función implícita para entornos de puntos singulares de $C\subseteq\mathbb{P}^2_{\mathbb{C}}$.

    Podemos analizar por ejemplo la cúspide de $f(x,y)=y^2-x^3$. Se tiene que $f_x(x,y)=-3x^2$ y $f_y(x,y)=2y$, por lo que $(0,0)$ es un punto singular de la curva afín $C=V(f)$.

    \begin{theor}
        Sea $C=V(f)\subseteq\mathbb{C}^2$ una curva afín tal que $f(0,0)=0$ tal que $\deg_y(f)=n$. Entonces, existe una serie de potencias $\xi(x)\in\mathbb{C}[[x^{1/n}]]$ tal que $f(x,\xi(x))=0$.

        Además, $\xi$ es convergente en un entorno de $0\in\mathbb{C}$.
    \end{theor}

    \begin{proof}
        
    \end{proof}

    \begin{obs}
        Una consencuencia del teorema es que $\bigcup_{ n\geq1}\mathbb{C}((x^{1/n}))$ es algebraicamente cerrado y es llamado el \textbf{campo de series de Puiseux}.
    \end{obs}

    Dado un $\xi\in\mathbb{C}[[x^{1/n}]]$, tomemos $\epsilon$ la raíz primitiva de la unidad de orden $n$.

    Por ejemplo, dada $\xi=\sum a_ix^{i/n}$, entonces $\sigma_{\epsilon}(\xi)=\sum a_i\epsilon^ix^{ i/n}$.

    \begin{excer}
        Calcular las conjugadas de $x^{ 6/4}+x^{ 7/4}$.
    \end{excer}

    Si tomamos a
    \begin{equation*}
        f(x,y)=\prod_{ \textup{conjugadas de }\xi}(y-\xi_k(x))\in\mathbb{C}[x][y]
    \end{equation*}
    esta es una función local irreducible.

    \begin{exa}
        Considere $f(x,y)=y^4+2x^3y^2+x^6+x^5y+x^12$. Entonces:
        \begin{equation*}
            f(x,y)=\sum_{i,j}a_{ ij}x^iy^j
        \end{equation*}
        \begin{itemize}
            \item Dibujar los puntos $(i,j)\in\mathbb{N}^2$ tales que $a_{ i,j}\neq0$. En este caso, se verían los puntos $(0,4),(3,2),(6,0),(5,1)$ y $(12,0)$.
            \item Tomamos $\Delta=envconv\left[\bigcup_{ i,j}\left((i,j)+(\mathbb{R}_{\geq0})^2\right)\right]$ donde $envconv$ es la envolvente convexa de este conjunto y consideramos los segmentos que unen a cada uno de los puntos que forman a éste conjunto parametrizados por las coordenadas $i$ y $j$, en este caso solo tenemos al segmento $2i+3j=12$.
            \item Para cada segmento $\Gamma$ finito de la frontera de $\Delta$ teenmos un par de números naturales primos rel. tales que $\Gamma$ está contenido en l arecta $ni+mj=M$ y vamos a considerar:
            \begin{equation*}
                f_\Gamma=\sum_{ ni+mj=M}a_{ij}x^iy^j
            \end{equation*}
            en este caso, para el segmento $\Gamma$ dado por $2i+3j=12$, se tiene que $f_\gamma=y^4+2x^3y^2+x^6$.
            \item Con $f_\Gamma$ consideramos una raíz $a$ de $f_\Gamma(1,y)\in\mathbb{C}[y^n]$, en este caso para $\Gamma$ tendríamos a $-\frac{2}{3}$.
            \item Hacemos $x=x_1^n$ y $y=x_1^m(a+y_1)$. Entonces ahora con la raíz $a$ de $(y^2+1)^2$, tomamos $a=i$ y hacemos:
            \begin{equation*}
                \left\{
                    \begin{array}{rcl}
                        x & = & x_1^2\\
                        y & = & x_1^3(i+y_1)\\
                    \end{array}
                \right.
            \end{equation*}
            hacemos:
            \begin{equation*}
                \begin{split}
                    f(x,y)&=f(x_1^n,x_1^m(y+y_1))\\
                    &=x_1^Mf(x_1,y_1)\\
                \end{split}
            \end{equation*}
            en este caso, notemos que como:
            \begin{equation*}
                x^iy^j=x_1^{ ni}x_1^{ mj}(a+y_1)^j=x_1^M(a+y_1)^j
            \end{equation*}
            en nuestro caso, obtendríamos que:
            \begin{equation*}
                \begin{split}
                    f(x,y)&=y^4+2x^3y^2+x^6+x^5y+x{12}\\
                    &=x_1^{12}(i+y_1)^4+2x_1^6x_1^6(i+y_1)^2+x_1^{12}+x_1^{10}x_1^3(i+y_1)+x_1^{24}\\
                    &=x_1^{12}\left[(i+y_1^4)+2(i+y_1)^2+1+x_1(i+y_1)+x_1^{12}\right]\\
                    &=x_1^{12}\left[\left((i+y_1)^2+1\right)^2+... \right]\\
                    &=x_1^{12}\left[(2iy_1+y_1^2)^2+x_1(i+y_1)+x_1^{12}\right]\\
                    &=x_1^{12}f_1(x_1,y_1)\\
                \end{split}
            \end{equation*}
            \item Reiteramos el proceso con la nueva $f_1$, la nueva raíz de $f_\Gamma(1,y)$ con este nuevo polinomio es $a_a$.
            \item Cuando paremos, hacemos las sustituciones para que SOLO QUEDE $x$.
            \item Podemos tomar otro segmento y obtendremos otra raíz conjugada inicial de nuestra variedad inicial. Independientemente de esto, llegaríamos a una aproximación de una raíz.
        \end{itemize}
        En el ejemplo anterior, tomamos $h(f)=\min\left\{j\Big|a_{ 0j}\neq0 \right\}$. En este caso, se tiene que:
        \begin{equation*}
            h(f_1)
        \end{equation*}
        es la multiplicidad de $a_1$ como raíz de $f_\Gamma(1,y)$. Pueden pasar dos cosas:
        \begin{enumerate}
            \item Que para alguna iteración $i$, $h(f_i)=0$, luego $y_i=0$ y esto haría que la serie de potencias sea finita.
            \item $h(f)\geq h(f_1)\geq h(f_2)\geq\cdots\geq h(f_i)=1$.
        \end{enumerate}
    \end{exa}

    Considerando una curva $C$ irreducible, por lo anterior podemos encontrar una serie de potencisa $\xi$ tal que con $C=V(f)$ se tiene que $f(t^n,\xi(t^n))=0$. Si $D=V(g)$, entonces
    \begin{equation*}
        I_0(C,D)=ord_tg(t^n,\sum a_it^i)
    \end{equation*}
    y,
    \begin{equation*}
        I_0(C_1\cup C_2,D)=I_0(C_1,D)+I_0(C_2,D)
    \end{equation*}

    \section{Explosión de $\mathbb{C}^2$ en el origen}

    Considere la superficie
    \begin{equation*}
        Bl_0\mathbb{C}^2=\left\{xv-yu=0 \right\}\subseteq\mathbb{A}^2\times\mathbb{P}_{\mathbb{C}}^1
    \end{equation*}
    queremos ver esto como superficie en le origen. Hacemos dos cartas dadas por:
    \begin{equation*}
        \left\{ 
            \begin{array}{rcl}
                x & = & u_1v_1 \\
                y & = & v_1\\
            \end{array}
        \right.\textup{ y }\left\{ 
            \begin{array}{rcl}
                x & = & u_2 \\
                y & = & u_2v_2\\
            \end{array}
        \right.
    \end{equation*}
    donde habrá un conjunto $E=\mathbb{P}^1$ tal que $E\subseteq Bl_0\mathbb{C}^2$. Lo que nos permite esto es quitar singularidades dobles exóticas como la que muestra la imagen.

    Considere la superficie $y^2=x^2+x^3$. Entonces:
    \begin{equation*}
        x^3=(y+x)(y-x)
    \end{equation*}

    \begin{itemize}
        \item En la carta 1, se tiene que la superficie se convierte en
        \begin{equation*}
            y^2-x^2-x^3=0\Rightarrow v_1^2(1-u_1^2-u_1^3v_1)=0
        \end{equation*}
        haciendo $f_1=1-u_1^2-u_1^3v_1$, cuando $E$ sea la recta $v_1=0$ sea cero se seguirá que $u_1$ tiene dos multiplicidades en dos puntos distintos.
    \end{itemize}

    \begin{excer}
        En la carta 1, hacer la traslación $u_1\mapsto u_1+1$ y comprobar que $E(v_1=0)$ y que $f_1$ es lisa y corta transversalmente a $E$.
    \end{excer}

    \begin{exa}
        Considere la curva $y^2+x^7$.
        \begin{itemize}
            \item En la carta 1, va suceder que:
            \begin{equation*}
                \begin{split}
                    v_1^2-u_1^7v_1^7=v_1^2(1-u_1^7v_1^5)=0
                \end{split}
            \end{equation*}
            entonces, $E(v_1=0)$ y $f_1$ no se cortan.
            \item En la carta 2, va a pasar que
            \begin{equation*}
                \begin{split}
                    (u_2v_2)^2-u_2^7=u_2^2(v_2^2-u_2^5)=0
                \end{split}
            \end{equation*}
            tomando $f_2=y^2-x^5$ se tiene que efectivamente, aquí sí corta $f_2$ a $E(u_2=0)$.
        \end{itemize}
    \end{exa}

    \begin{obs}
        En general, va a suceder que si la multiplicidad de $f$ es $m$, entonces
        \begin{itemize}
            \item En carta 1, tenemos algo de la forma $v_1^mf_1(u_1,v_1)$.
            \item En carta 2, tenemos algo de la forma $u_2^mf_2(u_2,v_2)$.
        \end{itemize}
    \end{obs}

    ¿Qué pasa cuando explotamos singularidades? Ahora tenemos la función $f$ y a $\xi$ que cumple lo anterior.

    Lo que hicimos en las clases pasadas fue tomar una curva en $\mathbb{C}^2$, haciamos una expolosión y teníamos una nueva curva en $X_0=Bl_0\mathbb{C}^2$.

    Tomando un punto $p\in Sing(V(f))$ le asociamos unas sucesiones de multiplicidad. Si suponemos que $f$ es irreducible, tenemos la sucesión $m_1,...,m_v$. Entonces si $(C,0)$ y $(D,0)$ son irreducibles:
    \begin{equation*}
        I_0(C,D)=\sum_{ \textup{puntos infinitamente próximos a $m$ y $n$}}m_1^C\cdot m_1^D
    \end{equation*}

    \begin{exa}
        Consideremos el polinomio $f=(y^2-x^3)(x^2-y^3)$. El punto $0$ tiene multiplicidad $m_1=4$. Cuando se explote, las dos cúspides que quedaban tenían ahora multiplicidades $m_2^1$ y $m_2^2$.

        En la primera cúspide teníamos multiplicidad $x_1^4(y^2-x_1)$. Seguimos expolotando y va bajando la multiplicidad. En este caso:
        \begin{equation*}
            I_0(C,D)=4
        \end{equation*}

        Tomando a $C$ la curva generada por $y^2-x^3=0$ y $D$ a la generada por $x^2-y^3=0$.
    \end{exa}

    Todo lo anterior en el ejemplo lo podemos representar en un grafo. Nos interesa decorar este grafo. Considere el grafo generado por los $E_i$. Generamos el grafo dual y decoración, haciendo
    \begin{equation*}
        E_i\rightarrow (N_i,\nu_i)
    \end{equation*}
    donde $N_i$ es el orden de $E_i$ bajo el pullback de la proyección hacia $\mathbb{C}^2$ de la curva generada por $f$, terminando de quitar las singularidades.

    No entendí lo que siguió después.

    Para $\nu_i$, éste está dado por:
    \begin{equation*}
        \nu_i=ord_{ E_i}\pi^{*}(dx\wedge dy)-1
    \end{equation*}
    los demás se definen de forma inductiva dependiendo de como se vaya separando la singularidad en todos los pasos.

    Lo demás perdón, ya que no fui capaz de hacer tantos diagramas y describir lo que estaba pasando con palabras me resultó imposible. Espero que con el ejemplo quede más claro.

    Si queremos ver lo que sucede con los grafos y todo eso, ver la página 519 de la última referencia anotada hasta abajo.
    
    \begin{exa}[\textbf{Resolución de una curva localmente analítica/irreducible}]
        Considere
        \begin{equation*}
            \xi(x^{ 1/n})\iff\left\{
                \begin{split}
                    x & = t^n\\
                    y & = \sum a_it^i
                \end{split}
             \right.;
             \left\{
                \begin{split}
                    & t^{ 12} \\
                    & t^{ 12}+t^{ 18}+t^{ 24}+t^{ 27}+t^{ 30}+t^{ 35}+ \\
                \end{split}
             \right.
        \end{equation*}
        Con
        \begin{equation*}
            f=\prod_{ \textup{conjugadas de }\xi}(y-\xi_k)
        \end{equation*}
    \end{exa}

    Considere la curva $(C,0)$ localmente irreducible y a la singularidad aislada $\left\{f=0 \right\}$ (en esta fibra) con $f\in\mathbb{C}[[x]][y]$ y podemos considerar las fibras $\left\{f=\varepsilon \right\}$

    \begin{theor}[\textbf{Milnor}]
        Existen $0<\delta<<\varepsilon$ suficientemente pequeños tales que el conjunto
        \begin{equation*}
            F=\left\{f=\varepsilon \right\}\cap B_g(\delta)
        \end{equation*}
        la fibra de una fibración localmente trivial.
    \end{theor}

    $F$ es una superficie conexa, compacta orientable y con borde
    \begin{equation*}
        \partial F=\bigsqcup_{ \textup{\# componentes irreducibles de }(C,0)}S^1 
    \end{equation*}
    y el $\chi(F)=1-\mu$ donde $\mu$ es el número de Milnor y está dado por:
    \begin{equation*}
        \mu=\dim_{\mathbb{C}}\frac{\mathbb{C}[[x]][y]}{\left(\frac{\partial f}{\partial x},\frac{\partial f}{\partial y} \right)}<\infty
    \end{equation*}

    \begin{exa}
        En el polinomio $f=y^p-x^q$, entonces $\frac{\partial f}{\partial x}=-qx^{q-1}$ y $\frac{\partial f}{\partial y}=py^{p-1}$. Por ende,
        \begin{equation*}
            \mu=(p-1)(q-1)
        \end{equation*}
    \end{exa}

    lo que sucede es que básicamente estamos metiendo esferas dentro de esferas dentro de esferas sucesivamente.

    Finalmente se habló de haces fibrados y cosas demás.

    Consideremos $p\in Sing(V(f)\subseteq\mathbb{C}^n)$. Hicimos dos cosas:
    \begin{itemize}
        \item Usar la fibración de Milnor.
        \item Autovalores de las partes semisimples de los homomorfismos de monodromía.
    \end{itemize}
    y,
    \begin{itemize}
        \item Resolución medinate un grafo dual con decoraciones. Finalmente se usó una funcioń racional $\mathcal{C}(s)$ que es una función zeta topológica local con polos $\frac{\nu_i}{N_i}$.
    \end{itemize}

    \begin{mydef}
        Se define la función \textbf{zeta topológica local} por:
        \begin{equation*}
            Z_{top,p}(f,s)=\sum_{ i}\frac{\chi(E_i)}{N_{ is}+\nu_i}+\sum_{ E_i\cap E_j\neq\emptyset}\frac{1}{(N_{ is}+\nu_i)(N_{ js}+\nu_j)}
        \end{equation*}
    \end{mydef}

    \begin{obs}
        Esta función cumple lo siguiente:
        \begin{itemize}
            \item Es independiente de la elección de la resolución.
            \item Especialización mediante la característica de euler de la función motívica (espacios de arcos o de $n$-jets).
        \end{itemize}
    \end{obs}

    \begin{exa}
        Considere la curva formada por $f=y^3-x^5$. Los candidatos a polos son $3s+2$, $4s+5$, $15s+8$ y $5s+3$. Haciendo la suma de la función zeta topológica local resulta que:
        \begin{equation*}
            Z_{ top}(f,s)=\frac{8+7s}{(1+5)(8+15s)}
        \end{equation*}
    \end{exa}

    \newpage

    \section{Sesión de Ejercicios}

    \begin{exa}
        Calcular las conjugadas de $x^{ 3/2}+x^{ 7/4}$.
    \end{exa}

    \begin{sol}
        Calculamos las raíces cuartas de 1, las cuáles son 1, -1,$i$ y $-i$. Tomamos
        \begin{equation*}
            \xi_1=\xi(x^{ 1/4})=(x^{1/4})^6+(x^{1/4})^7=x^{6/4}+x^{7/4}
        \end{equation*}
        las conjugadas serían:
        \begin{equation*}
            \begin{split}
                \xi_2=\xi(ix^{ 1/4})&=i^6(x^{1/4})^6+i^7(x^{1/4})^7\\
                &=-x^{6/4}-ix^{7/4}\\
            \end{split}
        \end{equation*}
        \begin{equation*}
            \begin{split}
                \xi_3=\xi(-ix^{ 1/4})&=(-i)^6(x^{1/4})^6+(-i)^7(x^{1/4})^7\\
                &=-x^{6/4}+ix^{7/4}\\
            \end{split}
        \end{equation*}
        y,
        \begin{equation*}
            \begin{split}
                \xi_4=\xi(-x^{ 1/4})&=(-x^{1/4})^6+(-x^{1/4})^7\\
                &=x^{6/4}-x^{7/4}\\
            \end{split}
        \end{equation*}
    \end{sol}

    Notemos que en el ejemplo anterior hay 4 conjugadas distintos del polinomio dado. Queremos ahora nuestro polinomio construído a partir de estos polinomios.
    \begin{equation*}
        f=(y-\xi_1)\cdot(y-\xi_2)\cdot(y-\xi_3)\cdot(y-\xi_4)
    \end{equation*}
    el cuál es:
    \begin{equation*}
        \begin{split}
            f&=(y-\xi_1)\cdot(y-\xi_2)\cdot(y-\xi_3)\cdot(y-\xi_4)\\
            &=(y-x^{6/4}-x^{7/4})\cdot(y+x^{6/4}+ix^{7/4})\cdot(y-x^{6/4}+x^{7/4})\cdot(y+x^{6/4}-ix^{7/4})\\
            &=\left((y-x^{ 6/4})^2-x^{7/2} \right)\cdot\left((y+x^{ 6/4})^2+x^{7/2}\right)\\
            &=\left(y^2+x^3+2x^{ 6/4}y+x^{ 7/2}\right)\cdot\left(y^2+x^3-2x^{ 6/4}y-x^{ 7/2}\right)\\
            &=(y^2+x^3)^2-(2x^{6/4}y+x^{7/2})^2\\
            &=(y^2+x^3)^2-4x^3y^2-4x^{5}y-x^7\\
        \end{split}
    \end{equation*}
    
    \begin{exa}
        Considere los polinomios $f=x^3-3x^2+2x+1$ y $g=x^2-x+2$. Entonces $m=3$ y $n=2$, por lo que:
        \begin{equation*}
            A_{5}=\left(
                \begin{array}{ccccc}
                    1 & -3 & 2 & 1 & 0 \\
                    0 & 1 & -3 & 2 & 1 \\
                    1 & -1 & 2 & 0 & 0 \\
                    0 & 1 & -1 & 2 & 0 \\
                    0 & 0 & 1 & -1 & 2 \\
                \end{array}
            \right)
        \end{equation*}
        sería la matriz asociada al resultante de los polinomios $f$ y $g$. ¿Cuál es el resultante de ambos polinomios?
    \end{exa}

    \begin{sol}
        Supongamos que ambos polinomios tienen una raíz en común, digamos $r$. Escribimos los polinomios de la siguiente forma:
        \begin{equation*}
            \left\{
                \begin{array}{rcl}
                    x^3-3x^2+2x+1 & = & (x-r)\underset{p}{\underbrace{(a_0x^2+a_1x+a_2)}} \\
                    x^2-x+2 & = & (x-r)\underset{q}{\underbrace{(b_0x+b_1)}} \\
                \end{array}
            \right.
        \end{equation*}
        calculamos ahora $qf-pg$, donde:
        \begin{equation*}
            qf = b_0x^4+(-3b_0+b_1)x^3+(2b_0+2b_1)x^2+(b_0+2b_1)x+b_1
        \end{equation*}
        y,
        \begin{equation*}
            a_0x^4+(a_1-a_0)x^3+(a_2-a_1+2a_0)x^2+(-a_2+2a_1)x+2a_2
        \end{equation*}
        por lo que los términos del polinomio $qf-pg$ clasificados por grados son:
        \begin{center}
            \begin{tabular}{c | c}
                Grado & Término \\
                \hline
                4 & $b_0-a_0$ \\
                3 & $b_1+a_0-3b_0-a_1$ \\
                2 & $2b_0-3b_1-a_2+a_1-2a_0$ \\
                1 & $2b_1+b_0+a_2-2a_1$ \\
                0 & $b_1-2a_2$ \\
            \end{tabular}
        \end{center}
        Y, para determinar el determinante bastará con resolver el sistema $5\times 5$ y ver si hay solución no trivial para el sistema dado por:
        \begin{equation*}
            \left(
                \begin{array}{ccccc}
                    -1 & 0 & 0 & 1 & 0 \\
                    1 & -1 & 0 & -3 & 1 \\
                    -2 & 1 & -1 & 2 & -3 \\
                    0 & -2 & 1 & 1 & 2 \\
                    0 & 0 & -2 & 1 & 0 \\
                \end{array}
            \right)\cdot\left(\begin{array}{c}
                a_0 \\
                a_1 \\
                a_2 \\
                b_0 \\
                b_1 \\
            \end{array} \right)=\left(\begin{array}{c}
                0 \\
                0 \\
                0 \\
                0 \\
                0 \\
            \end{array} \right)
        \end{equation*}
        reduzcamos el sistema:
        \begin{equation*}
            \begin{split}
                \left(
                \begin{array}{ccccc}
                    -1 & 0 & 0 & 1 & 0 \\
                    1 & -1 & 0 & -3 & 1 \\
                    -2 & 1 & -1 & 2 & -3 \\
                    0 & -2 & 1 & 1 & 2 \\
                    0 & 0 & -2 & 1 & 0 \\
                \end{array}
            \right)&\sim \left(
                \begin{array}{ccccc}
                    -1 & 0 & 0 & 1 & 0 \\
                    0 & -1 & 0 & -2 & 1 \\
                    0 & 1 & -1 & 0 & -3 \\
                    0 & -2 & 1 & 1 & 2 \\
                    0 & 0 & -2 & 1 & 0 \\
                \end{array}
            \right)\\
            &\sim \left(
                \begin{array}{ccccc}
                    -1 & 0 & 0 & 1 & 0 \\
                    0 & -1 & 0 & -2 & 1 \\
                    0 & 0 & -1 & -2 & -2 \\
                    0 & 0 & 1 & 5 & 0 \\
                    0 & 0 & -2 & 1 & 0 \\
                \end{array}
            \right)\\
            &\sim \left(
                \begin{array}{ccccc}
                    -1 & 0 & 0 & 1 & 0 \\
                    0 & -1 & 0 & -2 & 1 \\
                    0 & 0 & -1 & -2 & -2 \\
                    0 & 0 & 0 & 3 & -2 \\
                    0 & 0 & 0 & 5 & 4 \\
                \end{array}
            \right)\\
            &\sim \left(
                \begin{array}{ccccc}
                    -1 & 0 & 0 & 1 & 0 \\
                    0 & -1 & 0 & -2 & 1 \\
                    0 & 0 & -1 & -2 & -2 \\
                    0 & 0 & 0 & 1 & -\frac{2}{3} \\
                    0 & 0 & 0 & 5 & 4 \\
                \end{array}
            \right)\\
            &\sim \left(
                \begin{array}{ccccc}
                    -1 & 0 & 0 & 1 & 0 \\
                    0 & -1 & 0 & -2 & 1 \\
                    0 & 0 & -1 & -2 & -2 \\
                    0 & 0 & 0 & 1 & -\frac{2}{3} \\
                    0 & 0 & 0 & 0 & \frac{2}{3} \\
                \end{array}
            \right)\\
            &\sim \left(
                \begin{array}{ccccc}
                    -1 & 0 & 0 & 1 & 0 \\
                    0 & -1 & 0 & -2 & 1 \\
                    0 & 0 & -1 & -2 & -2 \\
                    0 & 0 & 0 & 1 & -\frac{2}{3} \\
                    0 & 0 & 0 & 0 & 1 \\
                \end{array}
            \right)\\
            \end{split}
        \end{equation*}
        por lo que el sistema es trivial (rápidamente se obtiene con operaciones de suma y resta de filas de la matriz). Por lo que ambos polinomios no comparten raíces en común.
    \end{sol}

    \begin{excer}
        En la carta 1, con la curva $y^2-x^2-x^3=0$ hacer la traslación $u_1\mapsto u_1+1$ y comprobar que $E(v_1=0)$ y que $f_1$ es lisa y corta transversalmente a $E$.
    \end{excer}

    \begin{sol}
        Obtuvimos la nueva curva a partir de la explosión:
        \begin{equation*}
            v_1^2(1-u_1^2-u_1^3v_1)=0
        \end{equation*}
        haciendo $f_1(u_1,v_1)=1-u_1^2-u_1^3v_1$, hacemos $v_1=0$, con lo que obtenemos:
        \begin{equation*}
            f(u_1,0)=1-u_1^2=(1-u_1)(1+u_1)
        \end{equation*}
        queremos ir al punto $(1,0)$. Cambiamos $u_1\mapsto u_2+1$ y a $v_1$ por $v_2$. Obtenemos en la curva original:
        \begin{equation*}
            \begin{split}
                v_2^2(1-(u_2+1)^2-(u_2+1)^3v_2)=0&\iff v_2^2(1-u_2^2-2u_2-1-v_2(u_2^3+3u_2^2+3u_2+1))\\
                &\iff v_2^2(-u_2^2-2u_2-v_2(u_2^3+3u_2^2+3u_2+1))\\
            \end{split}
        \end{equation*}
        considerando el nuevo polinomio $f_2(u_2,v_2)=-u_2^2-2u_2-v_2(u_2^3+3u_2^2+3u_2+1)$, con $v_2=0$ obtenemos que:
        \begin{equation*}
            f(u_2,0)=-u_2^2-2u_2=-u_2(2+u_2)
        \end{equation*}
        por lo que en efecto, $E$ corta a la curva en dos puntos transversalmente (en $0$ y $-1$). Además, es inmediato que esta nueva $f_2$ es lisa.

        Se puede hacer lo análogo usando la transformación $u_1\mapsto \bar{u_2}-1$ y $v_1\mapsto\bar{v_2}$ (resultando en lo análogo).
    \end{sol}

    \newpage

    \section*{Referencias}

    \begin{itemize}
        \item F. Kirwan \textit{Complex Algebraic Curves}, LMS 1992.
        \item K. Kendig \textit{A guide to plane algebraic curves}, MAA, 2011.
        \item E. Casas-Alveror, \textit{Algebraic Curves, the Bill-Noether Way}, Springer 2019.
        \item E. Brickeson H. Krorrer, \textit{Plane Algebraic Curves}, Birkhaser, 80's.
    \end{itemize}
    
\end{document}