\documentclass[12pt]{report}
\usepackage[spanish]{babel}
\usepackage[utf8]{inputenc}
\usepackage{amsmath}
\usepackage{amssymb}
\usepackage{amsthm}
\usepackage{graphics}
\usepackage{subfigure}
\usepackage{lipsum}
\usepackage{array}
\usepackage{multicol}
\usepackage{enumerate}
\usepackage[framemethod=TikZ]{mdframed}
\usepackage[a4paper, margin = 1.5cm]{geometry}
\usepackage{tikz}
\usepackage{pgffor}
\usepackage{ifthen}
\usepackage{enumitem}
\usepackage{hyperref}

\usepackage{listings}

%Gestión de Hipervínculos

\hypersetup{
    colorlinks=true,
    linkcolor=black,
    filecolor=magenta,      
    urlcolor=cyan
}

%Gestión de Código de Programación

\definecolor{listing-background}{HTML}{F7F7F7}
\definecolor{listing-rule}{HTML}{B3B2B3}
\definecolor{listing-numbers}{HTML}{B3B2B3}
\definecolor{listing-text-color}{HTML}{000000}
\definecolor{listing-keyword}{HTML}{435489}
\definecolor{listing-keyword-2}{HTML}{1284CA} % additional keywords
\definecolor{listing-keyword-3}{HTML}{9137CB} % additional keywords
\definecolor{listing-identifier}{HTML}{435489}
\definecolor{listing-string}{HTML}{00999A}
\definecolor{listing-comment}{HTML}{8E8E8E}

\lstdefinestyle{myStyle}{
    language         = C++,
    alsolanguage     = scala,
    numbers          = left,
    xleftmargin      = 2.7em,
    framexleftmargin = 2.5em,
    backgroundcolor  = \color{gray!15},
    basicstyle       = \color{listing-text-color}\linespread{1.0}\ttfamily,
    breaklines       = true,
    frameshape       = {RYR}{Y}{Y}{RYR},
    rulecolor        = \color{black},
    tabsize          = 2,
    numberstyle      = \color{listing-numbers}\linespread{1.0}\small\ttfamily,
    aboveskip        = 1.0em,
    belowskip        = 0.1em,
    abovecaptionskip = 0em,
    belowcaptionskip = 1.0em,
    keywordstyle     = {\color{listing-keyword}\bfseries},
    keywordstyle     = {[2]\color{listing-keyword-2}\bfseries},
    keywordstyle     = {[3]\color{listing-keyword-3}\bfseries\itshape},
    sensitive        = true,
    identifierstyle  = \color{listing-identifier},
    commentstyle     = \color{listing-comment},
    stringstyle      = \color{listing-string},
    showstringspaces = false,
    label            = lst:bar,
    captionpos       = b,
}

\lstset{style = myStyle}

%Estilo del capítulo y sección

\makeatletter
\def\thickhrulefill{\leavevmode \leaders \hrule height 1ex \hfill \kern \z@}
\def\@makechapterhead#1{%
  {\parindent \z@ \raggedright
    \reset@font
    \hrule
    \vspace*{10\p@}%
    \par
    \center \LARGE \scshape \@chapapp{} \huge \thechapter
    \vspace*{10\p@}%
    \par\nobreak
    \vspace*{10\p@}%
    \par
    \vspace*{1\p@}%
    \hrule
    %\vskip 40\p@
    \vspace*{60\p@}
    \Huge #1\par\nobreak
    \vskip 50\p@
  }}

\def\section#1{%
  \par\bigskip\bigskip
  \hrule\par\nobreak\noindent
  \refstepcounter{section}%
  \addcontentsline{toc}{chapter}{#1}%
  \reset@font
  { \large \scshape
    \strut\S \thesection \quad
    #1}% 
    \hrule   
  \par
  \medskip
}

\def\subsection#1{%
  \par\bigskip\bigskip
  \hrule\par\nobreak\noindent
  \refstepcounter{subsection}%
  \addcontentsline{toc}{section}{#1}%
  \reset@font
  { \normalsize \scshape
    \strut\S \thesubsection \quad
    #1}% 
    \hrule   
  \par
  \medskip
}

%Gestión marca de agua

\usetikzlibrary{shapes.multipart}

\newcounter{it}
\newcommand*\watermarktext[1]{\begin{tabular}{c}
    \setcounter{it}{1}%
    \whiledo{\theit<100}{%
    \foreach \col in {0,...,15}{#1\ \ } \\ \\ \\
    \stepcounter{it}%
    }
    \end{tabular}
    }

\AddToHook{shipout/foreground}{
    \begin{tikzpicture}[remember picture,overlay, every text node part/.style={align=center}]
        \node[rectangle,black,rotate=30,scale=2,opacity=0.04] at (current page.center) {\watermarktext{Cristo Daniel Alvarado ESFM\quad}};
  \end{tikzpicture}
}

%En esta parte se hacen redefiniciones de algunos comandos para que resulte agradable el verlos%

\def\proof{\paragraph{Demostración:\\}}
\def\endproof{\hfill$\blacksquare$}

\def\sol{\paragraph{Solución:\\}}
\def\endsol{\hfill$\square$}

%En esta parte se definen los comandos a usar dentro del documento para enlistar%

\newtheoremstyle{largebreak}
  {}% use the default space above
  {}% use the default space below
  {\normalfont}% body font
  {}% indent (0pt)
  {\bfseries}% header font
  {}% punctuation
  {\newline}% break after header
  {}% header spec

\theoremstyle{largebreak}

\newmdtheoremenv[
    leftmargin=0em,
    rightmargin=0em,
    innertopmargin=0pt,
    innerbottommargin=5pt,
    hidealllines = true,
    roundcorner = 5pt,
    backgroundcolor = gray!60!red!30
]{exa}{Ejemplo}[section]

\newmdtheoremenv[
    leftmargin=0em,
    rightmargin=0em,
    innertopmargin=0pt,
    innerbottommargin=5pt,
    hidealllines = true,
    roundcorner = 5pt,
    backgroundcolor = gray!50!blue!30
]{obs}{Observación}[section]

\newmdtheoremenv[
    leftmargin=0em,
    rightmargin=0em,
    innertopmargin=0pt,
    innerbottommargin=5pt,
    rightline = false,
    leftline = false
]{theor}{Teorema}[section]

\newmdtheoremenv[
    leftmargin=0em,
    rightmargin=0em,
    innertopmargin=0pt,
    innerbottommargin=5pt,
    rightline = false,
    leftline = false
]{propo}{Proposición}[section]

\newmdtheoremenv[
    leftmargin=0em,
    rightmargin=0em,
    innertopmargin=0pt,
    innerbottommargin=5pt,
    rightline = false,
    leftline = false
]{cor}{Corolario}[section]

\newmdtheoremenv[
    leftmargin=0em,
    rightmargin=0em,
    innertopmargin=0pt,
    innerbottommargin=5pt,
    rightline = false,
    leftline = false
]{lema}{Lema}[section]

\newmdtheoremenv[
    leftmargin=0em,
    rightmargin=0em,
    innertopmargin=0pt,
    innerbottommargin=5pt,
    roundcorner=5pt,
    backgroundcolor = gray!30,
    hidealllines = true
]{mydef}{Definición}[section]

\newmdtheoremenv[
    leftmargin=0em,
    rightmargin=0em,
    innertopmargin=0pt,
    innerbottommargin=5pt,
    roundcorner=5pt
]{excer}{Ejercicio}[section]

%En esta parte se colocan comandos que definen la forma en la que se van a escribir ciertas funciones%

\newcommand\abs[1]{\ensuremath{\left|#1\right|}}
\newcommand\divides{\ensuremath{\bigm|}}
\newcommand\cf[3]{\ensuremath{#1:#2\rightarrow#3}}
\newcommand\contradiction{\ensuremath{\#_c}}
\newcommand\natint[1]{\ensuremath{\left[\big|#1\big|\right]}}
\newcounter{figcount}
\setcounter{figcount}{1}

\begin{document}
    \setlength{\parskip}{5pt} % Añade 5 puntos de espacio entre párrafos
    \setlength{\parindent}{12pt} % Pone la sangría como me gusta
    \title{Título o Nombre de las notas}
    \author{Cristo Daniel Alvarado}
    \maketitle
    
    \tableofcontents %Con este comando se genera el índice general del libro%

    %\setcounter{chapter}{3} %En esta parte lo que se hace es cambiar la enumeración del capítulo%

    \chapter*{Información y Recursos}

    \section{Libros}

    \textit{Escrito por ChatGPT:} Si buscas un enfoque teórico y práctico para aprender Machine Learning, estos libros son altamente recomendados:
    
    \begin{itemize}
        \item \textbf{"Pattern Recognition and Machine Learning"} - Christopher M. Bishop  
        Matemáticamente riguroso, ideal para alguien con formación en matemáticas.
        
        \item \textbf{"The Elements of Statistical Learning"} - Hastie, Tibshirani y Friedman  
        Recomendado para obtener fundamentos teóricos sólidos.
        
        \item \textbf{"Hands-On Machine Learning with Scikit-Learn, Keras, and TensorFlow"} - Aurélien Géron  
        Enfoque práctico con Python, excelente para comenzar con código.
    \end{itemize}
    
    \section{Lenguajes de Programación}
    
    \begin{itemize}
        \item \textbf{Python} (Principal)  
        - Librerías clave: NumPy, Pandas, Scikit-Learn, TensorFlow, PyTorch.
        
        \item \textbf{R}  
        - Útil para estadística avanzada y visualización de datos.
    \end{itemize}
    
    \section{Páginas y Cursos en Línea}
    
    \subsection{Cursos en Coursera}
    
    \begin{itemize}
        \item Curso de Andrew Ng: \href{https://www.coursera.org/learn/machine-learning}{Machine Learning}  
        Matemáticas y conceptos clave explicados de manera clara.
        
        \item \href{https://www.coursera.org/specializations/deep-learning}{Deep Learning Specialization}  
        Curso avanzado enfocado en redes neuronales.
    \end{itemize}
    
    \subsection{Otros Cursos y Recursos}
    
    \begin{itemize}
        \item \href{https://course.fast.ai/}{Fast.ai: Practical Deep Learning for Coders}  
        Curso gratuito con enfoque práctico en código.
        
        \item \href{https://www.kaggle.com/learn}{Kaggle Learn}  
        Mini cursos interactivos y competencias con datos reales.
        
        \item \href{http://cs229.stanford.edu/}{CS229 de Stanford}  
        Notas de clase de Andrew Ng en Stanford.
        
        \item \textbf{StatQuest (YouTube)}  
        Explicaciones claras de teoría estadística aplicada a ML.
    \end{itemize}
    
    \chapter{Introducción}

    El objetivo de estas notas es el de dar un panorama general sobre el Machine Learning.
    
    %TODO: Poner para que sirve el machine learning

    \section{Aprendizaje Estadístico}

    \begin{mydef}[\textbf{Aprendizaje Estadístico}]
        El \textbf{aprendizaje estadístico} consiste en 
    \end{mydef}

    El aprendizaje estadístico juega un rol clave en muchas áreas de la ciencia, finanzas y la industria. Tales ejemplos pueden verse en los siguientes problemas:

    \begin{itemize}
        \item Predecir cuando un paciente hospitalizado debido a un ataque cardiaco tendrá un segundo ataque cardiaco. Esta predicción debe basarse en la deomgrafía del paciente.
        \item Predecir precios de acciones a 6 meses a futuro.
        \item Identificar números escritos a mano en códigos postales.
        \item Identificar factores de riesgo para cáncer de próstata, basado en variables clínicas y demográficas.
    \end{itemize}

    \begin{mydef}[\textbf{Demografía}]
        La \textbf{demografía} es la ciencia que estudia las poblaciones humanas, su dimensión, estructura, evolución y características generales, así como los procesos que determinan su formación, conservación y desaparición.
    \end{mydef}

    El objetivo es tomar información, interpretarla y decir algo acerca de ella, más precisamente, se pretende dar una \textit{predicción}.

    Usando la información se construirá un \textbf{aprendedor} (o \textbf{learner}) que será la pieza clave para predecir la salida de nueva fuente de información. Queremos un \textit{good learner}.

    Hay dos tipos de aprendizaje:
    \begin{itemize}
        \item \textbf{Supervisado}: En este tipo de aprendizaje hay la presencia de una variable de salida que nos permite guiar el proceso de aprendizaje.
        \item \textbf{No Supervisado}: Observamos solo las características y no tenemos medida de que tan buena es la salida del proceso. 
    \end{itemize}

    Hablaremos de algunos ejemplos de aprendizaje supervisado:

    \section{Spam de Email}

    \begin{excer}[\textbf{Spam de Email}]
        Se tienen 4601 mensajes en email hacia una persona y pretendemos determinar si cada uno de ellos era un email basura o \textit{spam}. Diseñar un detector de spam automático que pueda filtrar el spam antes de que las bandejas de entrada de los usuarios se llenen.
    \end{excer}

    \begin{sol}
        Para cada uno de los 4601 mensajes, podemos asignar dos estados de salida:
        \begin{equation}
            \textup{}
        \end{equation}
    \end{sol}

\end{document}