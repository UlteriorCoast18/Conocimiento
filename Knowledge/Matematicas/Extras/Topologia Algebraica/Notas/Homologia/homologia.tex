\documentclass[12pt]{report}
\usepackage[spanish]{babel}
\usepackage[utf8]{inputenc}
\usepackage{amsmath}
\usepackage{amssymb}
\usepackage{amsthm}
\usepackage{graphics}
\usepackage{subfigure}
\usepackage{lipsum}
\usepackage{array}
\usepackage{multicol}
\usepackage{enumerate}
\usepackage[framemethod=TikZ]{mdframed}
\usepackage[a4paper, margin = 1.5cm]{geometry}
\usepackage{tikz}
\usepackage{pgffor}
\usepackage{ifthen}
\usepackage{enumitem}
\usepackage{hyperref}
\usepackage{bbm}

%Gestión de Hipervínculos

\hypersetup{
    colorlinks=true,
    linkcolor=black,
    filecolor=magenta,      
    urlcolor=cyan
}

\usetikzlibrary{shapes.multipart}

\newcounter{it}
\newcommand*\watermarktext[1]{\begin{tabular}{c}
    \setcounter{it}{1}%
    \whiledo{\theit<100}{%
    \foreach \col in {0,\dots,15}{#1\ \ } \\ \\ \\
    \stepcounter{it}%
    }
    \end{tabular}
    }

\AddToHook{shipout/foreground}{
    \begin{tikzpicture}[remember picture,overlay, every text node part/.style={align=center}]
        \node[rectangle,black,rotate=30,scale=2,opacity=0.04] at (current page.center) {\watermarktext{Cristo Daniel Alvarado ESFM\quad}};
  \end{tikzpicture}
}
%En esta parte se hacen redefiniciones de algunos comandos para que resulte agradable el verlos%

\def\proof{\paragraph{Demostración:\\}}
\def\endproof{\hfill$\blacksquare$}

\def\sol{\paragraph{Solución:\\}}
\def\endsol{\hfill$\square$}

%En esta parte se definen los comandos a usar dentro del documento para enlistar%

\newtheoremstyle{largebreak}
  {}% use the default space above
  {}% use the default space below
  {\normalfont}% body font
  {}% indent (0pt)
  {\bfseries}% header font
  {}% punctuation
  {\newline}% break after header
  {}% header spec

\theoremstyle{largebreak}

\newmdtheoremenv[
    leftmargin=0em,
    rightmargin=0em,
    innertopmargin=0pt,
    innerbottommargin=5pt,
    hidealllines = true,
    roundcorner = 5pt,
    backgroundcolor = gray!60!red!30
]{exa}{Ejemplo}[section]

\newmdtheoremenv[
    leftmargin=0em,
    rightmargin=0em,
    innertopmargin=0pt,
    innerbottommargin=5pt,
    hidealllines = true,
    roundcorner = 5pt,
    backgroundcolor = gray!50!blue!30
]{obs}{Observación}[section]

\newmdtheoremenv[
    leftmargin=0em,
    rightmargin=0em,
    innertopmargin=0pt,
    innerbottommargin=5pt,
    rightline = false,
    leftline = false
]{theor}{Teorema}[section]

\newmdtheoremenv[
    leftmargin=0em,
    rightmargin=0em,
    innertopmargin=0pt,
    innerbottommargin=5pt,
    rightline = false,
    leftline = false
]{propo}{Proposición}[section]

\newmdtheoremenv[
    leftmargin=0em,
    rightmargin=0em,
    innertopmargin=0pt,
    innerbottommargin=5pt,
    rightline = false,
    leftline = false
]{cor}{Corolario}[section]

\newmdtheoremenv[
    leftmargin=0em,
    rightmargin=0em,
    innertopmargin=0pt,
    innerbottommargin=5pt,
    rightline = false,
    leftline = false
]{lema}{Lema}[section]

\newmdtheoremenv[
    leftmargin=0em,
    rightmargin=0em,
    innertopmargin=0pt,
    innerbottommargin=5pt,
    roundcorner=5pt,
    backgroundcolor = gray!30,
    hidealllines = true
]{mydef}{Definición}[section]

\newmdtheoremenv[
    leftmargin=0em,
    rightmargin=0em,
    innertopmargin=0pt,
    innerbottommargin=5pt,
    roundcorner=5pt
]{excer}{Ejercicio}[section]

%En esta parte se colocan comandos que definen la forma en la que se van a escribir ciertas funciones%

\newcommand\abs[1]{\ensuremath{\left|#1\right|}}
\newcommand\divides{\ensuremath{\bigm|}}
\newcommand\cf[3]{\ensuremath{#1:#2\rightarrow#3}}
\newcommand\contradiction{\ensuremath{\#_c}}
\newcommand\natint[1]{\ensuremath{\left[\!\left[ #1\right]\!\right]}}
\newcommand\gen[1]{\ensuremath{\langle#1\rangle }}
\newcommand\bbm[1]{\ensuremath{\mathbbm{#1}}}

\newcounter{figcount}
\setcounter{figcount}{1}

\begin{document}
    \setlength{\parskip}{5pt} % Añade 5 puntos de espacio entre párrafos
    \setlength{\parindent}{12pt} % Pone la sangría como me gusta
    \title{Taller Topología Algebraica
    
    Homología}
    \author{Cristo Daniel Alvarado}
    \maketitle

    \tableofcontents %Con este comando se genera el índice general del libro%

    \setcounter{chapter}{3} %En esta parte lo que se hace es cambiar la enumeración del capítulo%

    \newpage

    \chapter{Homología Simplicial y Singular}

    Antes de empezar con la parte de homología singular, veremos un poco de homología singular, que es una versión primitiva de ésta la cual nos permitirá entender los conceptos abstractos de la homología singular de forma más sencilla.

    \section{Homología Simplcial}

    El dominio natural de la definición de la homología simplicial es una clase de espacios llamado \textbf{$\Delta$-complejos}, los cuales son una generalización de una noción más clásica, la de \textbf{complejo simplicial}.

    \subsection{$\Delta$-complejos}

    El toro, el plano proyectivo y la botella de Klein pueden ser obtenidas mediante un procedimiento de identificación de lados opuestos de un cuadrado, manteniendo la orientación deseada.

    Cortar un cuadrado sobre la diagona produce dos triángulos. De forma análoga, podemos cortar un polígono en triángolos más pequeños. Más aún, toda superficie cerrada puede ser construida usando triángulos e identificando sus lados de forma adecuada.

    Así que, tenemos un sólo bloque de construcción para todas las superficies. Usando sólo triángulos podemos construir una clase de espacios 2-dimensionales que no son superficies en el sentido estricto, perimitiendo que más de dos lados se identifiquen juntos a la vez.

    La idea de un $\Delta$-complejo es la de generalizar este tipo de construcciones a cualquier número de dimensiones. El análogo $n$-dimensional de un triángulo es el \textbf{$n$-simplejo}.

    \begin{mydef}
        Sean $n,m\in\mathbb{N}\cup\left\{0 \right\}$ con $m>0$. Se define el \textbf{$n$-simplejo} como el subconjunto convexo más pequeño en $\mathbb{R}^m$ tal que contiene a $n+1$ puntos $v_0,\dots,v_n\in\mathbb{R}^m$ que no yacen en el mismo hiperplano de dimensión menor o igual a $n$.

        Los puntos $v_i$ son llamados \textbf{vértices del simplejo} y éste es denotado por $[v_0,\dots,v_n]$
    \end{mydef}

    \begin{obs}
        En la práctica, esto se vería más o menos así:
        
        \begin{minipage}{\textwidth}
            \begin{center}
                \includegraphics[scale=1.5]{images/fig_1.pdf}\\
                Figura \thefigcount. $n$ simplejos para $n=0,1,2$ y $3$.
                \stepcounter{figcount}
            \end{center}
        \end{minipage}

        De izquierda a derecha se muestran como serían el 0-simplejo, 1-simplejo, 2-simplejo y 3-simplejo metidos en $\mathbb{R}^3$.
    \end{obs}

    \begin{exa}
        El $n$-simplejo que contiene a los vectores canónicos $e_1,\dots,e_n$ y al cero en $\mathbb{R}^m$ es el conjunto:
        \begin{equation*}
            \Delta^n=\left\{(t_0,\dots,t_n)\in\mathbb{R}^{ n+1}\Big|\sum_{ i=0}^n t_i\textup{ y }t_i\geq0\textup{ }\forall i=0,\dots,n \right\}
        \end{equation*}
        es llamado \textbf{$n$-simplejo estándar}. Notemos que $\Delta^n=[0,e_1,\dots,e_n]$ donde $O$ es el origen de $\mathbb{R}^{ n+1}$.
    \end{exa}

    En homología va a resultar importante mantener el orden de los vértices del simplejo, por lo que cuando digamos \textit{$n$-simplejo}, realmente estaremos diciendo \textit{$n$-simplejo con un ordenamiento de vértices}.

    Una consecuencia de ordenar los vértices de un simplejo $[v_0,\dots,v_n]$ es que éstos determinan la orientación de los vértices $[v_i,v_j]$ de acuerdo a los subíndices ordenados de forma creciente.

    Especificar este orden de los vértices determina un homomorfismo lineal del $n$-simplejo estándar en cualquier otro $n$-simplejo $[v_0,\dots,v_n]$ que preserve el orden de los vértices, a saber:
    \begin{equation*}
        (t_0,\dots,t_n)\mapsto \sum_{ i=0}^n t_iv_i
    \end{equation*}
    los coeficientes $t_i$ son llamados $i$-ésimas \textbf{coordenadas baricéntricas} del punto $\sum_{ i=0}^n t_iv_i$ en el simplejo $[v_0,\dots,v_n]$.

    \begin{mydef}
        Sea $[v_0,\dots,v_n]$ un $n$-simplejo y tomemos $j=0,\dots,n$. Entonces el simplejo $[v_0,\dots,\hat{v_j},\dots,v_n]$ es un $(n-1)$-simplejo llamado \textbf{$j$-ésima cara de $[v_0,\dots,v_n]$}.
    \end{mydef}

    \begin{obs}
        Bajo la definición que hicimos anteriormente, todo subsimplejo de un simplejo estará siempre con los vértices ordenados de forma creciente, de acuerdo al orden que establecimos en el simplejo original. 
    \end{obs}

    \begin{mydef}
        El conjunto formado por la unión de todas las caras de un simplejo $\Delta^n$ es la \textbf{frontera de $\Delta^n$} y se denota por $\partial\Delta^n$. El \textbf{simplejo abierto} $\mathring{\Delta}^n$ es el conjunto $\Delta^n\setminus\partial\Delta^n$.
    \end{mydef}

    \begin{mydef}
        Una \textbf{estructura de $\Delta$-complejo en un espacio $X$} (o llamado simplemente \textbf{$\Delta$-complejo}) es una colección $\left\{\cf{\sigma_\alpha}{\Delta^{ n_\alpha}}{X} \right\}_{\alpha\in I}$ (con $n_\alpha\in\mathbb{N}\cup\left\{0 \right\}$ para todo $\alpha\in I$) tal que:
        \begin{enumerate}[label = \textit{(\arabic*)}]
            \item Para todo $\alpha\in I$, la reestricción $\sigma_\alpha\big|_{\mathring{\Delta}^{ n_\alpha}}$ es inyectiva, y todo punto de $X$ es la imagen de exactamente una reestriccion $\sigma_\alpha\big|_{\mathring{\Delta}^{ n_\alpha}}$.
            \item Para todo $\alpha\in I$ y para cada reestricción de $\sigma_\alpha$ a alguna cara de $\Delta^{ n_\alpha}$ existe $\beta\in I$ tal que esta reestricción coincide con $\cf{\sigma_\beta}{\Delta^{ n_\alpha-1}}{X}$ (donde $n_\beta=n_\alpha-1$).
            \item $A\subseteq X$ es abierto si y sólo si $\sigma_\alpha^{-1}(A)$ es abierto en $\Delta^{n_\alpha}$ para todo $\alpha\in I$.
        \end{enumerate}
    \end{mydef}

    \begin{obs}
        En el inciso \textit{(2)} identificamos cada cara de $\Delta^n$ con $\Delta^{ n-1}$ mediante el homomorfismo lineal entre ellos que preserva la orientación.
    \end{obs}

    La condición \textit{(3)} nos quita la condición trivial de que todos los puntos de $X$ sean vértices de algún simplejo.

    Una consecuencia inmediata de todas estas condiciones es que un espacio $X$ puede ser construido a partir de una colección de simplejos disjuntos $\Delta_\alpha^{n_\alpha}$, uno por cada $\cf{\sigma_\alpha}{\Delta^{ n_\alpha}}{X}$, el espacio obtenido a partir de identificar cada cara de $\Delta_\alpha^{ n_\alpha}$ con el correspondiente $\Delta_\beta^{ n_\alpha-1}$, correspondeinte a la reestricción de $\sigma_\beta$ de $\sigma_\alpha$ de la cara en cuestión.

    Para nuestro caso, basta con analizar por ejemplo al Toro:

    \begin{exa}
        %TODO Poner analisis del toro y definir sobre él una estructura de $\Delta$ complejo.
    \end{exa}

    \subsection{Homología Simplicial}

    Nuestro objetivo ahora es definir los grupos de homología de un $\Delta$-complejo $X$. Para ello, será imprescindible contar con todo lo hecho en la teoría sobre grupos libres y grupos abelianos libres.

    \begin{mydef}
        Sea 
    \end{mydef}

    \section{Homología Singular}

    Con lo hecho en la parte de homología simplicial, resultará un poco más sencillo observar qué es lo que está sucediendo en homología singular. En esta sección se dan definiciones y algunas otras propiedades básicas.

    \subsection{Definición de los grupos cúbicos singulares de homología}

    \begin{obs}
        De ahora en adelante $I=[0,1]$ denotará a este intervalo. Además, convenimos en que $I^0$ es un conjunto con un sólo punto, a saber $I^0=\left\{0\right\}$. Además, el conjunto $\mathbb{N}^*$ denotará a $\mathbb{N}\cup\left\{0\right\}.$
    \end{obs}

    \begin{mydef}
        Sea $X$ un espacio topológico. Si existe $n\in\mathbb{N}$ tal que $X\cong I^n$, diremos que $X$ es un \textbf{cubo $n$-dimensional}.
    \end{mydef}

    \begin{mydef}
        Sea $X$ espacio topológico y $n\in\mathbb{N}^*$ \textbf{$n$-cubo singular en $X$} es una función continua $\cf{T}{I^n}{X}$. 
    \end{mydef}

    \begin{exa}
        Si $n=0$, entonces $\cf{T}{\left\{* \right\}}{X}$ es una función que manda un punto en un punto.

        Si $n=1$, entonces $\cf{T}{I}{X}$ es un camino con puntos extremos $T(0)$ y $T(1)$.
    \end{exa}

    \begin{obs}
        Note que a diferencia de la parte de homología simplicial, en esta parte permitimos que nuestro mapeo $T$ no necesariamente sea lineal, ya que en la parte anterior se deduce rápidamente que $\Delta^n\cong I^n$.

        Más aún, esta es llamada homología singular ya que puede que la función $T$ tenga singularidades.
    \end{obs}

    Esta generalización nos permitirá hacer la construcción de la homología singular de forma similiar a como se hizo anteriormente.

    \begin{mydef}
        Sean $X$ espacio topológico y $n\in\mathbb{N}^*$. El conjunto $Q_n(X)$ denota al grupo abeliano libre generado por el conjunto de todos los $n$-cubos singulares en $X$.
    \end{mydef}

    \begin{obs}
        Un elemento de $Q_n(X)$ (dados como en la definición anterior) es de la forma:
        \begin{equation*}
            n_1T_1+n_2T_2+\cdots+n_kT_k
        \end{equation*}
        donde $n_i\in\mathbb{Z}$, $\cf{T_i}{I^n}{X}$ es función continua, para todo $i\in\natint{1,n}$ y $T_j\neq T_i$, para todo $i,j\in\natint{1,n}$ con $i\neq j$. La suma NO es la suma usual de funciones (pues puede que $X$ no tenga estructura que le permita realizar esta suma de funciones, y aunque tuviese, ignoraríamos este hecho), es únicamente usada para expresar a los elementos del grupo abeliano libre.
    \end{obs}

    \begin{mydef}
        Sean $X$ espacio topológico y $n\in\mathbb{N}$. Un $n$-cubo singular $\cf{T}{I^n}{X}$ es \textbf{degenerado} si existe $i\in\natint{1,n}$ tal que:
        \begin{equation*}
            T(x_1,\dots,x_{ i_1},\dots,x_n)=T(x_1,\dots,x_{ i_2},\dots,x_n)
        \end{equation*}
        para todo $x_{i_1},x_{i_2}\in I$. En otras palabras, $T$ no depende de la $i$-ésima entrada.
    \end{mydef}

    Notemos que admitimos que ningún $0$-cubo singular puede ser degenerado. Además, un $1$-cubo singular es dengerado si y sólo si es la función constante.

    \begin{mydef}
        Sean $X$ espacio topológico y $n\in\mathbb{N}$. $D_n(X)$ denota al subgrupo de $Q_n(X)$ generado por el conjunto de todos los $n$-cubos singulares degenerados. Este es un subgrupo normal de $Q_n(X)$ (ya que es un subgrupo de un grupo abeliano), denotamos por:
        \begin{equation*}
            C_n(X)=Q_n(X)/D_n(X)
        \end{equation*}
        este es llamado \textbf{grupo de todas las cadenas singulares de $n$-cubos} o simplemente \textbf{$n$-cadenas de $X$}.
    \end{mydef}

    En otras palabras, lo que estamos haciendo es quitar al grupo de todos los $n$-cubos singulares a aquellos que deberían tener \textit{menos volumen} que los demás dentro de esta lista.

    Convenimos en que si $X=\emptyset$, entonces $Q_n(X)=D_n(X)=C_n(X)=\langle0\rangle$ para todo $n\in\mathbb{N}^*$.

    \begin{exa}
        Si $X=\left\{*\right\}$, entonces sólo puede haber un único cubo singular en $X$ para todo $n\in\mathbb{N}^*$. Además, este siempre es degenerado si $n\in\mathbb{N}$.

        Por tanto, $C_0(X)$ es cíclico infinito ya que $Q_0(X)$ lo es y $D_0(X)=\gen{0}$. Si $n\in\mathbb{N}$ se tiene que $C_n(X)=\gen{0}$ ya que $Q_n(X)=D_n(X)$.
    \end{exa}

    \begin{exa}
        Podemos ir más allá en la primer parte del ejemplo anterior, ya que para cualquier espacio $X$ se tiene que $D_0(X)=\gen{0}$, por lo que $C_0(X)=Q_0(X)$.
    \end{exa}

    \begin{exa}
        Para cualquier espacio $X$, $C_n(X)$ (con $n\in\mathbb{N}$) es grupo abeliano libre generado en el conjunto de todos los $n$-cubos singulares no dengenerados en $X$.
    \end{exa}

    \subsection{Caras de un cubo singular}

    Nuestro objetivo ahora es el de definir las caras de estos $n$-cubos singulares, de forma análoga a como se hizo con los simplejos.

    \begin{mydef}
        Sean $X$ espacio topológico y $n\in\mathbb{N}$. Sea $T$ un $n$-cubo singular en $X$, para cada $i\in\natint{1,n}$ se definen los $n$-cubos singulares:
        \begin{equation*}
            \cf{A_i(T),B_i(T)}{I^{ n-1}}{X}
        \end{equation*}
        dados por:
        \begin{equation*}
            \begin{split}
                A_iT(x_1,\dots,x_n)&=T(x_1,\dots,x_{ i-1},1,x_i,\dots,x_n)\\
                B_iT(x_1,\dots,x_n)&=T(x_1,\dots,x_{ i-1},0,x_i,\dots,x_n)\\
            \end{split}
        \end{equation*}
        $A_iT$ es llamada la \textbf{$i$-ésima cara frontal de $T$} y $B_iT$ es la \textbf{$i$-ésima cara trasera de $T$}.
    \end{mydef}

    \begin{obs}
        Podemos ver a $\cf{A_i}{Q_n(X)}{Q_{ n-1}(X)}$ y $\cf{B_i}{Q_n(X)}{Q_{ n-1}(X)}$ como funciones para cada $i\in\natint{1,n}$.
    \end{obs}

    Rápidamente (ejercicio), es posible verificar para cada $i,j\in\natint{1,n}$, $i< j$ y $n>1$:
    \begin{equation}
        \label{faceConmutativeIdentities}
        \begin{split}
            A_iA_j(T)&=A_{ j-1}A_i(T)\\
            B_iB_j(T)&=B_{ j-1}B_i(T)\\
            A_iB_j(T)&=B_{ j-1}A_i(T)\\
            B_iA_j(T)&=A_{ j-1}B_i(T)\\
        \end{split}
    \end{equation}

    \newcommand{\bound}[1]{\textup{\partial\left(#1\right)}}

    \begin{mydef}
        Sean $X$ espacio topológico y $n\in\mathbb{N}$. Para cada $n$-cubo singular $T$ se define:
        \begin{equation*}
            \partial_n(T)=\sum_{ i=1}^n (-1)^i\left[A_iT-B_iT \right]
        \end{equation*}
        Extendemos de forma natural $\partial_n$ al $Q_n$, definiéndolo en función de su valor en los elementos generadores. Denotamos esta extensión de igual manera y es tal que $\cf{\partial_n}{Q_n}{Q_{n-1}}$. Este es un homomorfismo llamado \textbf{operador frontera}.

        Por comodidad, de ahora en adelante denotaremos a este operador simplemente por $\partial$.
    \end{mydef}

    \begin{propo}
        Sea $n\in\mathbb{N}$ con $n>1$ y $T$ un $n$-cubo singular. Entonces:
        \begin{equation*}
            \partial_{ n-1}\left(\partial_n\left(T\right) \right)=0
        \end{equation*}
        Además, si $T$ es degenerado, entonces $\partial_n\left(T\right)\in D_{ n-1}(X)$. En otras palabras, $\partial_n(D_n(X))\subseteq D_{n-1}$. 
    \end{propo}

    \begin{proof}
        Probaremos primero la primera identidad. Veamos que:
        \begin{equation*}
            \begin{split}
                \partial_{ n-1}\left(\partial_n\left(T\right) \right)&=\partial_{ n-1}\left(\sum_{ i=1}^n (-1)^i\left[A_iT-B_iT \right] \right)\\
                &=\sum_{ i=1}^n (-1)^i\left[\partial_{ n-1}\left(A_iT\right)-\partial_{ n-1}\left(B_iT\right)\right]\\
                &=\sum_{ i=1}^n (-1)^i\left[\sum_{ j=1}^{n-1} (-1)^j\left[A_jA_iT-B_jA_iT \right]-\sum_{ j=1}^{n-1} (-1)^j\left[A_jB_iT-B_jB_iT \right]\right]\\
                &=\sum_{ i=1}^n \sum_{ j=1}^{n-1}(-1)^{i+j} \left[A_jA_iT-B_jA_iT-A_jB_iT+B_jB_iT\right]\\
            \end{split}
        \end{equation*}
        Analizaremos las sumas una por una:
        \begin{equation*}
            \begin{split}
                \sum_{ i=1}^n \sum_{ j=1}^{n-1}(-1)^{i+j}A_jA_iT&=\sum_{ i=1}^{ n-1}\sum_{ j=1}^{n-1}(-1)^{i+j}A_jA_iT+\sum_{ j=1}^{n-1}(-1)^{n+j}A_jA_nT\\
                &=\sum_{ j=1}^{n-1}(-1)^{1+j}A_jA_1T+\sum_{ j=1}^{n-1}(-1)^{2+j}A_jA_2T+\cdots+\\
                &+\sum_{ j=1}^{n-1}(-1)^{n-1+j}A_jA_{ n-1}T+\sum_{ j=1}^{n-1}(-1)^{n+j}A_jA_nT\\
            \end{split}
        \end{equation*}
        Hay dos casos, $n$ es par o $n$ es impar. Analicemos por casos:
        \begin{itemize}
            \item Suponga que $n$ es impar, entonces existe $k\in\mathbb{N}$ tal que $n=2k+1$, luego hay $2k$ sumandos. Se tiene:
            \begin{equation*}
                \begin{split}
                    \sum_{ i=1}^n \sum_{ j=1}^{n-1}(-1)^{i+j}A_jA_iT&=\sum_{ j=1}^{n-1}(-1)^{1+j}A_jA_1T+\sum_{ j=1}^{n-1}(-1)^{2+j}A_jA_2T+\cdots+\sum_{ j=1}^{n-1}(-1)^{k+j}A_jA_{k}T+ \\
                    &+\sum_{ j=1}^{n-1}(-1)^{k+1+j}A_jA_{k+1}T+\cdots+\sum_{ j=1}^{n-1}(-1)^{n+j}A_jA_nT\\
                \end{split}
            \end{equation*}
            Veamos que todo elemento de una suma de las primeras $k$-sumas se cancela con alguno de las otras $k$-sumas que siguen. En efecto, considere el sumando $(-1)^{l+j}A_jA_lT$ con $j\in\natint{1,n-1}$ y $l\in\natint{1,n}$ que es el $j$-ésimo sumando que se encuentra en la $l$-ésima suma. Se tienen dos casos:
            \begin{itemize}
                \item Si $1\leq j< l$, entonces al tenerse que $A_jA_lT=A_{l-1}A_{j}T$ por (\ref{faceConmutativeIdentities}) se sigue que el elemento $(-1)^{ l-1+j}A_{l-1}A_{j}T$ es el $(l-1)$-ésimo sumando que se encuentra en la $j$-ésima suma. Por ende, los sumandos $(-1)^{l+j}A_jA_lT=(-1)^{l+j}A_{l-1}A_{j}T$ y $(-1)^{ l-1+j}A_{l-1}A_{j}T$ se encuentran en diferentes sumas, así que cuando se efectúa su suma obtenemos:
                \begin{equation*}
                    \begin{split}
                        (-1)^{l+j}A_jA_lT+(-1)^{ l-1+j}A_{l-1}A_{j}T&=(-1)^{ l+j}\left(A_{l-1}A_{j}T-A_{l-1}A_{j}T\right)\\
                        &=0\\
                    \end{split}
                \end{equation*}
                \item Si $l\leq j<n$, entonces al tenerse que $A_jA_lT=A_{(j+1)-1}A_lT$, se sigue de la ecuación (\ref{faceConmutativeIdentities}) que $(-1)^{l+j}A_jA_lT=(-1)^{l+j }A_lA_{ j+1}T$. Como en el $j+1$-ésimo sumando se encuentra $l$-ésimo término (ya que $l\leq n-1$ este término existe) $(-1)^{j+1+l}A_lA_{ j+1}T$, se tiene que los términos $(-1)^{l+j }A_lA_{ j+1}T$ y $(-1)^{j+1+l}A_lA_{ j+1}T$ se encuentran en diferentes sumas, así que cuando se efectúa la suma obtenemos que:
                \begin{equation*}
                    \begin{split}
                        (-1)^{l+j }A_lA_{ j+1}T+(-1)^{j+1+l}A_lA_{ j+1}T&=(-1)^{l+j }(A_lA_{ j+1}T-A_lA_{ j+1}T)\\
                        &=0\\
                    \end{split}
                \end{equation*}
            \end{itemize}
            en ambos casos, se sigue que:
            \begin{equation*}
                \sum_{ i=1}^n \sum_{ j=1}^{n-1}(-1)^{i+j}A_jA_iT=0
            \end{equation*}
            \item Suponga que $n$ es par, entonces existe $k\in\mathbb{N}$ tal que $n=2k$. Entonces, hay $2k-1$ sumandos. El procedimiento para ver que se cancelan las sumas es análogo al anterior.
        \end{itemize}
        Por ambos incisos, se sigue que:
        \begin{equation*}
            \sum_{ i=1}^n \sum_{ j=1}^{n-1}(-1)^{i+j}A_jA_iT=0
        \end{equation*}
        es decir:
        \begin{equation*}
            \partial_n(\partial_{ n-1})(T)=0
        \end{equation*}

        Para la otra parte, suponga que $T$ es degenerado, entonces existe $j\in\natint{1,n}$ tal que:
        \begin{equation*}
            T(x_1,\dots,x_{j_1},\dots,x_n)=T(x_1,\dots,x_{j_2},\dots,x_n)
        \end{equation*}
        para todo $x_1,\dots,x_{j_1},x_{j_2},\dots,x_n\in I$. En particular, notemos que:
        \begin{equation*}
            T(x_1,\dots,0,\dots,x_n)=T(x_1,\dots,1,\dots,x_n)
        \end{equation*}
        por lo cual,
        \begin{equation*}
            B_jT=A_jT\Rightarrow A_jT-B_jT=0
        \end{equation*}
        Veamos entonces que:
        \begin{equation*}
            \begin{split}
                \partial_nT&=\sum_{ i=1}^n (-1)^i\left[A_iT-B_iT \right]\\
                &=\sum^n_{\substack{i=1 \\ i \neq j}} (-1)^i\left[A_iT-B_iT \right]\\
            \end{split}
        \end{equation*}
        donde todos los demás $(n-1)$-cubos en esta suma son degenerados, por lo que $\partial_nT\in D_{n-1}(X)$.
    \end{proof}

    \begin{cor}
        Sea $n\in\mathbb{N}$ con $n>1$. Entonces, $\cf{\partial_n}{Q_n(X)}{Q_{ n-1}(X)}$ induce un homomorfismo que denotamos por el mismo símbolo $\cf{\partial_n}{C_n(X)}{C_{n-1}(X)}$, tal que:
        \begin{equation*}
            \partial_{ n-1}\circ\partial_n=0
        \end{equation*}
    \end{cor}

    \begin{proof}
        En efecto, definamos $\cf{\partial_n}{C_n(X)}{C_{n-1}(X)}$ por:
        \begin{equation*}
            \partial_n([C]_{D_n(X)})=[\partial_n(C)]_{D_{n-1}(X)}
        \end{equation*}
        donde $C\in Q_n(X)$ y los corchetes denotan la clase de equivalencia, es decir que $[C]_{D_n(X)}\in C_n(X)$. Veamos que este es un homomorfismo que está bien definido.
        \begin{itemize}
            \item \textbf{$\partial_n$ está bien definido}. Sean $C_1,C_2\in Q_n(X)$ tales que $C_1+D_n(X)=C_2+D_n(X)$, en otras palabras:
            \begin{equation*}
                [C_1]_{D_n(X)}=[C_2]_{D_n(X)}
            \end{equation*}
            por comodidad, esto simplemente lo denotaremos por $[C_1]=[C_2]$. Por tanto, existen $T_1,\dots,T_n\in D_n(X)$ y $n_1,\dots,n_m\in\mathbb{Z}$ tales que:
            \begin{equation*}
                C_1-C_2=n_1T_1+\cdots+n_mT_m\Rightarrow C_1=C_2+n_1T_1+\cdots+n_mT_m
            \end{equation*}
            así que:
            \begin{equation*}
                \begin{split}
                    \partial_n([C_1])&=[\partial_n(C_1)]\\
                    &=[\partial_n(C_2+n_1T_1+\cdots+n_mT_m)]\\
                    &=[\partial_n(C_2)+n_1\partial_n(T_1)+\cdots+n_m\partial_n(T_m)]\\
                    &=[\partial_n(C_2)]\\
                    &=\partial_n([C_2])\\
                \end{split}
            \end{equation*}
            por lo que $\partial_n$ está bien definido.
            \item \textbf{$\partial_n$ es homomorfismo}. Es inmediato.
        \end{itemize}
    \end{proof}

    \begin{obs}
        Nuevamente, al operador $\partial_n$ lo denotaremos simplemente por $\partial$.
    \end{obs}

    \begin{obs}
        De forma análoga, resulta que para el homomorfismo $\cf{\partial_n}{C_n(X)}{C_{n-1}(X)}$ es tal que:
        \begin{equation*}
            \partial_{n-1}\circ\partial_n=0
        \end{equation*}
        para $n>1$.
    \end{obs}

    \begin{mydef}
        Sea $X$ espacio topológico y $n\in\mathbb{N}$. Definimos:
        \begin{equation*}
            Z_n(X)=\ker\partial_n=\left\{u\in C_n(X)\Big|\partial_n(u)=0 \right\}\subseteq C_n(X)
        \end{equation*}
        y, para $n\in\mathbb{N}^*$:
        \begin{equation*}
            B_n(X)=\textup{im}\:\partial_{n+1}=\partial_{n+1}(C_{n+1}(X))\subseteq C_n(X)
        \end{equation*}
        En consecuencia de que $\partial_n\circ\partial_{n+1} = 0$, se tiene que $B_n(X)\subseteq Z_n(X)$ para todo $n\in\mathbb{N}$. Con lo que se define el \textbf{$n$-ésimo grupo de homología singular de $X$} o el \textbf{$n$-ésimo grupo de homología de $X$}:
        \begin{equation*}
            H_n(X)=Z_n(X)/B_n(X),\quad\forall n\in\mathbb{N}
        \end{equation*}

        $Z_n(X)$ es llamado el \textbf{$n$-ésimo grupo de ciclos singulares de $X$}, o \textbf{grupo de $n$-ciclos}. $B_n(X)$ es llamado \textbf{grupo de froteras $n$-dimensionales} o \textbf{grupo de los $n$-ciclos delimitadores}.
    \end{mydef}

    Nuestro objeto principal de estudio será el grupo $H_n(X)$.

    \section{El $0$-ésimo grupo de homología, $H_0(X)$}

    En la parte anterior hemos definido los grupos de homología para $n\in\mathbb{N}$, pero para poder obtener aún más información (y hacer más álgebra con esta estructura algebraica), necesitamos alguna forma de definir a $H_0(X)$. Para ello, primero deberíamos intentar definir a $Z_0(X)$.

    Daremos dos definiciones, donde una de ellas resultará más sencilla en algunos casos, y la otra en otros. No existe casi diferencia entre ambas será tan sencilla que no habrá problemas en los resultados que obtengamos.

    \subsection{Primera definición de $H_0(X)$}

    \begin{mydef}
        Sea $X$ espacio topológico. Definimos $Z_0(X)=C_0(X)$ y hacemos:
        \begin{equation*}
            H_0(X)=Z_0(X)/B_0(X)=C_0(X)/B_0(X)
        \end{equation*}
    \end{mydef}

    Otra forma de llegar a este mismo resultado es de la siguiente forma: hacemos $C_n(X)=\left\{0\right\}$, para $n\in\mathbb{Z}$ con $n<0$ y hacemos:
    \begin{equation*}
        \cf{\partial_n}{C_n(X)}{C_{ n-1}(X)}\textup{ como }\partial_n=0
    \end{equation*}
    nuevamente, para todo $n\in\mathbb{Z}$ con $n<0$. De esta forma:
    \begin{equation*}
        Z_n(X)=\ker\partial_n,\quad\forall n\in\mathbb{N}
    \end{equation*}
    y $B_n(X)=\partial_{n+1}\left(C_{n+1}(X)\right)\subseteq Z_n(X)$, con lo cual:
    \begin{equation*}
        H_n(X)=Z_n(X)/B_n(X),\quad\forall n\in\mathbb{N}
    \end{equation*}
    En particular se tiene que $H_n(X)=\left\{0\right\}$ para todo $n\in\mathbb{Z}$ con $n<0$.

    De esta forma este grupo está definido aunque $X=\emptyset$.

    \subsection{Opcional: EL Grupo de Homología 0-dimensional Reducido $\tilde{H}_0(X)$}
    
    Para este caso, lo que haremos será definir un homomorfismo $\cf{\varepsilon}{C_0(X)}{\mathbb{Z}}$ donde $\mathbb{Z}$ denota el anillo de los enteros. Este homomorfismo es llamado usualmente \textbf{aumentación}.

    \begin{obs}
        Dado que $C_0(X)=Q_0(X)/D_0(X)$ y $D_0(X)\cong\left\{0\right\}$ (pues no hay 0-cubos singulares degenerados), se sigue que $C_0(X)\cong Q_0(X)$, por lo que podemos establecer naturalmente la igualdad $C_0(X)=Q_0(X)$.
    \end{obs}

    Por la observación anterior, basta definir $\cf{\varepsilon}{C_0(X)}{\mathbb{Z}}$ sobre 0-cubos, lo cual lo hacemos de forma natural:
    \begin{equation*}
        \varepsilon(T)=1,\quad\forall T\textup{ 0-cubo singular.}
    \end{equation*}
    Así que si $u=\sum_{ i=1}^k n_iT_i\in Q_0(X)$ podemos extender $\varepsilon$ a $Q_0(X)$ naturalmente:
    \begin{equation*}
        \varepsilon(u)=\sum_{ i=1}^k n_i
    \end{equation*}
    Con lo que tenemos el homomorfismo deseado.

    \begin{lema}
        Se cumple que:
        \begin{equation*}
            \varepsilon\circ\partial_1=0
        \end{equation*}
    \end{lema}

    \begin{proof}
        Sea $T$ un $1$-cubo singular. Entonces:
        \begin{equation*}
            \begin{split}
                \partial_1(T)&=\sum_{ i=1}^1(-1)^i(A_iT-B_iT)\\
                &=B_1T-A_1T\\
                &=T_0-T_1\\
            \end{split}
        \end{equation*}
        Con $\cf{T_j}{\left\{*\right\}}{X}$ la función tal que $*\mapsto T(j)$ para $j=0,1$. Por tanto:
        \begin{equation*}
            \varepsilon(\partial_1(T))=\varepsilon(T_0)-\varepsilon(T_1)=1-1=0
        \end{equation*}
        Usando el hecho de que $\partial_1$ es homomorfismo se sigue que:
        \begin{equation*}
            \varepsilon\circ\partial_1=0
        \end{equation*}
    \end{proof}
    
    Justamente este lema nos permite hacer la siguiente definición:

    \begin{mydef}
        Definimos $\tilde{Z}_0(X)=\ker\varepsilon$ y, dado que $B_0(X)=\textup{im}\partial_1\subseteq\tilde{Z}_0(X)$ por el lema anterior, podemos definir sin problemas el \textbf{grupo 0-dimensional de homología reducido de $X$} como:
        \begin{equation*}
            \tilde{H}_0(X)=\tilde{Z}_0(X)/B_0(X)
        \end{equation*}
    \end{mydef}

    \begin{obs}
        Para evitar problemas más adelante, definimos $\tilde{H}_n(X)=H_n(X)$ para todo $n\in\mathbb{N}$.
    \end{obs}

    Resulta que existe una relación entre el grupo de homología 0-reducido y el que hemos definido anteriormente.

    Primero, notemos que $\tilde{Z}_0(X)\subseteq Z_0(X)=C_0(X)$, luego de la definición se sigue que $\tilde{H}_0(X)$ es un subgrupo de $H_0(X)$.

    Tomemos $\cf{\zeta}{\tilde{H}_0(X)}{H_0(X)}$ el homomorfismo inclusión. Ahora, como $B_0(X)\subseteq\tilde{Z}_0(X)\subseteq Z_0(X)=C_0(X)$, se tiene que el homomorfismo aumentación $\cf{\varepsilon}{C_0(X)}{\mathbb{Z}}$ induce un homomorfismo $\cf{\varepsilon_*}{H_0(X)}{\mathbb{Z}}$ dado por:
    \begin{equation*}
        \varepsilon_*\left(B_0(X)+\sum_{ i=1}^k n_iT_i\right)=\varepsilon\left(\sum_{ i=1}^k n_i\right)=\sum_{ i=1}^k n_i
    \end{equation*}

    Resulta que estos homomorfismos guardan una relación muy particular entre sí. Antes de ver dicha, relación, veamos la siguiente definición:

    \begin{mydef}[\textbf{Secuencia Exacta}]
        Sean $G_0,\dots,G_n$ una secuencia de grupos y $\left(\cf{f_i}{G_{ i-1}}{G_i}\right)_{ i=1}^n$ una secuencia de homomorfismos. Decimos que la secuencia es \textbf{exacta} en $G_i$ si:
        \begin{equation*}
            \textup{im}\:f_i=\ker f_{ i+1}
        \end{equation*}
        y, decimos que la secuencia es \textbf{exacta} si es exacta en cada $G_i$ para $i\in\natint{1,n-1}$.
    \end{mydef}

    \begin{obs}
        Normalmente denotaremos la secuencia de grupos y homomorfismos de la siguiente manera:
        \begin{equation*}
            G_0\overset{f_1}{\longrightarrow}G_1\overset{f_2}{\longrightarrow}G_2\overset{f_3}{\longrightarrow}\cdots\overset{f_n}{\longrightarrow}G_n
        \end{equation*}
    \end{obs}

    \begin{propo}
        La siguiente secuencia de grupos y homomorfismos:
        \begin{equation*}
            0\longrightarrow \tilde{H}_0(X)\overset{\zeta}{\longrightarrow}H_0(X)\overset{\varepsilon_*}{\longrightarrow}\mathbb{Z}\longrightarrow0
        \end{equation*}
        es exacta. Entonces, podemos identificar a $\tilde{H}_0(X)$ con el kernel de $\varepsilon_*$ (asumiendo que el espacio $X$ ahora es no vacío).
    \end{propo}

    \begin{proof}
        Veamos que la secuencia es exacta:
        \begin{itemize}
            \item \textbf{Exacta en $\tilde{H}_0(X)$}. En efecto, $\textup{im}\:0=\left\{0\right\}=\ker\zeta$, ya que $\zeta$ es el homomorfismo inclusión (en particular, es inyectivo).
            \item \textbf{Exacta en $H_0(X)$}. Se tiene que la imagen de $\zeta$ es $\tilde{H}_0(X)$, ahora, sea $B_0(X)+\sum_{ i=1}^k n_iT_i\in \tilde{H}_0(X)$ elemento de $\ker\varepsilon_*$, entonces:
            \begin{equation*}
                \begin{split}
                    \varepsilon_*\left(B_0(X)+\sum_{ i=1}^k n_iT_i \right)&=0\\
                    \Rightarrow\varepsilon\left(\sum_{ i=1}^k n_i \right)&=0\\
                \end{split}
            \end{equation*}
            por tanto, $\sum_{ i=1}^k n_i=0$, así que $B_0(X)+\sum_{ i=1}^k n_iT_i\in \tilde{H}_0(X)$, es decir, $B_0(X)+\sum_{ i=1}^k n_iT_i\in \textup{im}\:\zeta$. Se sigue que $\textup{im}\:\zeta=\ker\varepsilon_*$.
            \item \textbf{Exacta} en $\mathbb{Z}$. Es inmediato (más adelante se verifica que $\varepsilon_*$ es epimorfismo).
        \end{itemize}
        De los tres incisos se sigue que la secuencia es exacta.
    \end{proof}

    \begin{exa}
        \label{ejemplo:espacio_punto_0_homotopia}
        Sea $X=\left\{*\right\}$ el espacio de un solo punto. Entonces:
        \begin{equation*}
            \begin{split}
                H_0(X)&\cong\mathbb{Z}\\
                H_n(X)&=\left\{0\right\}\\
                \tilde{H}_0(X)&=\left\{0\right\}\\
            \end{split}
        \end{equation*}
        en este caso $\cf{\varepsilon_*}{H_0(X)}{\mathbb{Z}}$ es un isomorfismo.
    \end{exa}

    Este ejemplo se puede generalizar de forma natural a la siguiente proposición:

    \begin{propo}
        Sea $X$ un espacio no vacío, arco-conexo. Entonces, $\cf{\varepsilon_*}{H_0(X)}{\mathbb{Z}}$ es un isomorfismo, y $\tilde{H}_0(X)=\left\{0\right\}$.
    \end{propo}

    \begin{proof}
        Para probar que $\cf{\varepsilon_*}{H_0(X)}{\mathbb{Z}}$ es un isomorfismo, basta con ver que es epimorfismo y monomorfismo.
        \begin{itemize}
            \item \textbf{Epimorfismo}: Es inmediato, pues si $T$ es un $0$-cubo singular, entonces para todo $n\in\mathbb{Z}$ se tiene que:
            \begin{equation*}
                \varepsilon_*(B_0(X)+nT)=n
            \end{equation*}
            \item \textbf{Monomorfismo}: Dado que por la proposición anterior la secuencia:
            \begin{equation*}
                0\longrightarrow \tilde{H}_0(X)\overset{\zeta}{\longrightarrow}H_0(X)\overset{\varepsilon_*}{\longrightarrow}\mathbb{Z}\longrightarrow0
            \end{equation*}
            es exacta, basta con ver que $\ker\varepsilon_*=\textup{im}\:\zeta=\left\{0\right\}$, en otras palabras, hay que probar que $B_0(X)=\tilde{Z}_0(X)=\ker\varepsilon$. Ya se tiene que $B_0(X)\subseteq\tilde{Z}_0(X)$, por lo que veamos la otra contención. Sea $u=\sum_{i=1}^k n_iT_i\in\ker\varepsilon$, esto es que:
            \begin{equation*}
                \varepsilon(u)=\sum_{ i=1}^k n_i=0
            \end{equation*}
            Ahora, como $X$ es arco-conexo, para cada par de $0$-cubo singulares $T_i$ y $T_j$ ($i,j\in\natint{1,k}$ con $i\neq j$) existe un $1$-cubo singular $T_{i,j}$ tal que:
            \begin{equation*}
                A_1T_{i,j}=T_i\quad\textup{y}\quad B_1T_{i,j}=T_j
            \end{equation*}
            Por tanto, notemos que:
            \begin{equation*}
                \begin{split}
                    \partial_1(T_{i,j})&=B_1T_{i,j}-A_1T_{i,j}\\
                    &=T_j-T_i\\
                \end{split}
            \end{equation*}
            Luego, podemos escribir a $u$ como:
            \begin{equation*}
                \begin{split}
                    \partial_1&\left(n_1T_{2,1}+(n_1+n_2)T_{3,2}+(n_1+n_2+n_3)T_{4,3}+\cdots+\sum_{ i=1}^j n_iT_{j+1,j}+\dots+\sum_{i=1}^{k-1}n_iT_{ k,k-1} \right)\\
                    &=n_1(T_1-T_2)+(n_1+n_2)(T_2-T_3)+\cdots+\left(\sum_{ i=1}^{k-1} n_i\right)(T_ {k-1}-T_k)\\
                    &=n_1T_1+\left(-n_1+n_1+n_2\right)T_2+\cdots+\left(-\sum_{ i=1}^{k-2} n_i+\sum_{ i=1}^{k-1} n_i\right)T_{ k-1}-\left(\sum_{ i=1}^{k-1} n_i\right)T_k\\
                    &=n_1T_1+n_2T_2+\dots+n_{k-1}T_{ k-1}-\left(\sum_{ i=1}^{k-1} n_i\right)T_k\\
                \end{split}
            \end{equation*}
            Pero, como $\sum_{ i=1}^kn_i=0$, se sigue que $n_k=-\sum_{ i=1}^{ k-1}n_i$, por lo cual lo anterior es igual a:
            \begin{equation*}
                n_1T_1+n_2T_2+\cdots+n_{k-1}T_{ k-1}+n_kT_k=u
            \end{equation*}
            se sigue así que $u\in\textup{im}\:\partial_1$. Por tanto, $\ker\varepsilon\subseteq B_0(X)$.
        \end{itemize}
    \end{proof}

    \begin{propo}
        Sea $X$ un espacio topológico. Si:
        \begin{equation*}
            X=\bigsqcup_{ \gamma\in\Gamma} X_\gamma,
        \end{equation*}
        (siendo esta la unión disjunta) donde $\left(X_\gamma\right)_{ \gamma\in\Gamma}$ es una sucesión de subconjuntos arco-conexos maximales de $X$ ajenos a pares, entonces:
        \begin{equation*}
            H_n(X)\cong\bigoplus_{ \gamma\in\Gamma} H_n(X_\gamma),
        \end{equation*}
        para todo $n\in\mathbb{N}$.
    \end{propo}

    \begin{proof}
        %TODO
    \end{proof}

    \begin{obs}
        En las hipótesis de la proposición anterior, los conjuntos arco-conexos maximales son exactamente las componentes arco-conexas de $X$.
    \end{obs}

    \begin{cor}[\textbf{Caracterización de $H_0(X)$}]
        Para cualquier espacio topológico $X$, $H_0(X)$ es suma directa de grupos cíclicos infinitos, uno por cada componente arco-conexa de $X$.

        En otras palabras, $H_0(X)$ es un grupo abeliano libre cuyo rango es igual al númeor de componentes arco-conexas de $X$.
    \end{cor}

    Resulta que el corolario anterior no es para nada válido con el grupo reducido $\tilde{H}_0(X)$, como lo muestra el Ejemplo (\ref{ejemplo:espacio_punto_0_homotopia}).  

    \section{El Homomorfismo Inducido por una Función Continua}

    La Homología asocia a cada espacio topológico una sucesión $H_n(X)$ de grupos. De manera igualmente importante, a cada función continua $\cf{f}{X}{Y}$ entre dos espacios le podemos asociar una secuencia de homomorfismos:
    \begin{equation*}
        \cf{f_*}{H_n(X)}{H_n(Y)},\quad\forall n\in\mathbb{N}^*
    \end{equation*}

    \begin{obs}
        Sucederá que ciertas propiedades topológicas se traducirán en propiedades del homomorfismo $f_*$.
    \end{obs}

    \begin{mydef}[\textbf{Homomorfismo entre $Q_n(X)$ y $Q_n(Y)$}]
        Sean $X$ y $Y$ espacios topológicos, y $\cf{f}{X}{Y}$ una función continua. Entonces, definimos, para cada $n\in\mathbb{N}^*$ el homomorfismo $\cf{f_\#}{Q_n(X)}{Q_n(Y)}$ dado por:
        \begin{equation*}
            f_\#(T)=f\circ T,\quad\forall \:T\textup{ n-cubo singular}
        \end{equation*}
    \end{mydef}

    Resulta que este homomorfismo tiene propiedades interesantes que se enlistan a continuación:

    \begin{propo}[\textbf{Propiedades de $f_\#$}]
        Sean $X$ y $Y$ espacios topológicos y $\cf{f}{X}{Y}$ una función continua. Entonces:
        \begin{enumerate}[label = \textit{(\arabic*)}]
            \item Si $T$ es un $n$-cubo singular degenerado en $X$, entonces $f_\#(T)$ lo es en $Y$. Esto induce un homomorfismo $\cf{f_\#}{C_n(X)}{C_n(Y)}$.
            \item El siguiente diagrama:
            
            \begin{minipage}{\textwidth}
                \begin{center}
                    \includegraphics[scale=1]{images/fig_2.pdf}\\
                    Figura \thefigcount. Conmutatividad de $f_\#$ con $\partial_n$ en $Q_n$.
                    \stepcounter{figcount}
                \end{center}
            \end{minipage}

            es conmtuativo, para todo $n\in\mathbb{N}$. En otras palabras:
            \begin{equation*}
                \partial_n\circ f_\#=f_\#\circ\partial_n,\quad\forall n\in\mathbb{N}
            \end{equation*}

            Además, el diagrama:

            \begin{minipage}{\textwidth}
                \begin{center}
                    \includegraphics[scale=1]{images/fig_4.pdf}\\
                    Figura \thefigcount. Conmutatividad de $f_\#$ con $\partial_n$ en $C_n$.
                    \stepcounter{figcount}
                \end{center}
            \end{minipage}

            es conmutativo, para todo $n\in\mathbb{N}$.
            \item El diagrama:
            
            \begin{minipage}{\textwidth}
                \begin{center}
                    \includegraphics[scale=1]{images/fig_3.pdf}\\
                    Figura \thefigcount. Conmutatividad de $f_\#$ con $\varepsilon$.
                    \stepcounter{figcount}
                \end{center}
            \end{minipage}

            es conmutativo.
        \end{enumerate}
    \end{propo}

    \begin{proof}
        De \textit{(1)}: Sea $T$ un $n$-cubo singular degenerado en $X$, entonces existe $j\in\natint{1,n}$ tal que:
        \begin{equation*}
            T(x_1,\dots,x_{j_1},\dots,x_n)=T(x_1,\dots,x_{j_2},\dots,x_n)
        \end{equation*}
        para todo $x_1,\dots,x_{j_1},x_{j_2},\dots,x_n\in I$. Por lo cual:
        \begin{equation*}
            \begin{split}
                f_\#(T)(x_1,\dots,x_{ j_1},\dots,x_n)&=f(T(x_1,\dots,x_{ j_1},\dots,x_n))\\
                &=f(T(x_1,\dots,x_{ j_2},\dots,x_n))\\
                &=f_\#(T)(x_1,\dots,x_{ j_2},\dots,x_n),\\
            \end{split}
        \end{equation*}
        para todo $x_1,\dots,x_{j_1},x_{j_2},\dots,x_n\in I$. Por tanto, $f_\#(T)$ es degenerado.

        Ahora, el homomorfismo inducido $\cf{f_\#}{C_n(X)}{C_n(Y)}$ está dado por:
        \begin{equation*}
            f_\#\left(\sum_{ i=1}^k n_iT_k+D_n(X)\right)=\sum_{ i=1}^k n_if_\#(T_k)+D_n(Y)
        \end{equation*}
        Esté está siempre bien definido, pues si tenemos dos elementos que son degenerados en $X$, su diferencia es degenerada, y al aplicar $f_\#$ se mantiene la degeneración pero ahora en $Y$. El hecho de que sea homomorfismo se sigue de la definición del $f_\#$ original.

        De \textit{(2)}: Sea $T$ un $n$-cubo singular en $X$, notemos que:
        \begin{equation*}
            \begin{split}
                f_\#\circ A_i (T)(t_1,\dots,t_{ n-1})&=f(A_iT(t_1,\dots,t_{ n-1}))\\
                &=f(T(t_1,\dots,t_{ i-1},1,t_i,\dots,t_{ n-1}))\\
                &=f\circ T(t_1,\dots,t_{ i-1},1,t_i,\dots,t_{ n-1})\\
                &=A_i(f_\#(T))(t_1,\dots,t_{ n-1})\\
                &=A_i\circ f_\#(T)(t_1,\dots,t_{ n-1})\\
            \end{split}
        \end{equation*}
        para todo $(t_1,\dots,t_{ n-1})\in I^{ n-1}$. Por ende, se tiene que:
        \begin{equation}
            \label{conmutatividad_f_A}
            f_\#\circ A_i=A_i\circ f_\#
        \end{equation}
        De forma análoga se prueba que:
        \begin{equation}
            \label{conmutatividad_f_B}
            f_\#\circ B_i=B_i\circ f_\#
        \end{equation}
        De (\ref{conmutatividad_f_A}) y (\ref{conmutatividad_f_B}) se sigue que para todo $T$ un $n$-cubo singular en $X$:
        \begin{equation*}
            \begin{split}
                f_\#\circ\partial_n(T)&=f_\#\left(\sum_{ i=1}^n (-1)^i\left[A_iT-B_iT \right]\right)\\
                &=\sum_{ i=1}^n (-1)^i\left[f_\#(A_iT)-f_\#(B_iT) \right]\\
                &=\sum_{ i=1}^n (-1)^i\left[A_i(f_\#T)-B_i(f_\#T) \right]\\
                &=\partial_n\circ f_\#(T)\\
            \end{split}
        \end{equation*}
        Dado que $f$ y $\partial_n$ son homomorfismos, se sigue que:
        \begin{equation*}
            \partial_n\circ f_\#\left(\sum_{ i=1}^kn_iT_i\right) =f_\#\circ\partial_n\left(\sum_{ i=1}^kn_iT_i\right)
        \end{equation*}
        para todo $\sum_{ i=1}^kn_iT_i\in Q_n(X)$. Por tanto:
        \begin{equation*}
            \partial_n\circ f_\#=f_\#\circ\partial_n
        \end{equation*}
        Ahora, dado que $f_\#$ induce un homomorfismo $\cf{f_\#}{C_n(X)}{C_n(Y)}$, este también conmuta con $\cf{\partial_n}{C_n}{C_{n-1}}$, pues si $\sum_{ i=1}^k n_iT_k+D_n(X)\in C_n(X)$:
        \begin{equation*}
            \begin{split}
                f_\#\circ\partial_n\left(\sum_{ i=1}^k n_iT_k+D_n(X)\right)&=f_\#\left(\sum_{ i=1}^k n_i\partial_n(T_k)+D_{n-1}(X)\right)\\
                &=\sum_{ i=1}^k n_if_\#\left(\partial_n(T_k)\right)+D_{n-1}(Y)\\
                &=\sum_{ i=1}^k n_i\partial_n\left(f_\#(T_k)\right)+D_{n-1}(Y)\\
                &=\partial_n\left(\sum_{ i=1}^k n_if_\#(T_k)+D_n(Y)\right)\\
                &=\partial_n\circ f_\#\left(\sum_{ i=1}^k n_iT_k+D_n(X)\right)\\
            \end{split}
        \end{equation*}

        De \textit{(3)}: Es análogo a lo hecho en el inciso \textit{(2)}.
    \end{proof}
    
    \begin{obs}
        En el inciso \textit{(2)} de la proposición anterior, notemos que $f_\#$ mapea a $Z_n(X)\subseteq C_n(X)$ en $Z_n(Y)\subseteq C_n(Y)$, pues si $\sum_{ i=1}^k n_iT_i+D_n(X)\in Z_n(X)$, se tiene que:
        \begin{equation*}
            \begin{split}
                \partial_n\circ f_\#\left(\sum_{ i=1}^k n_iT_i+D_n(X)\right)&=f_\#\circ\partial_n\left(\sum_{ i=1}^k n_iT_i+D_n(X)\right)\\
                &=f_\#(0+D_{ n-1}(X))\\
                &=0+D_{ n-1}(Y)\\
                &=0\\
            \end{split}
        \end{equation*}
        pues, dado que $\sum_{ i=1}^k n_iT_i+D_n(X)\in Z_n(X)$, se tiene que $\partial_n\left(\sum_{ i=1}^k n_iT_i+D_n(X)\right)=0+D_{ n-1}(X)$. Así que:
        \begin{equation*}
            f_\#(Z_n(X))\subseteq Z_n(Y)
        \end{equation*}

        Ahora, de forma análoga, se tiene que $f_\#$ mapea a $B_n(X)\subseteq C_n(X)$ en $B_n(Y)\subseteq C_n(Y)$, pues si $\sum_{ i=1}^k n_iT_i+D_n(X)\in B_n(X)$, existe $\sum_{ j=1}^m m_jS_j+D_{ n+1}(X)\in C_{ n+1}(X)$ tal que:
        \begin{equation*}
            \sum_{ i=1}^k n_iT_i+D_n(X)=\partial_{ n+1}\left(\sum_{ j=1}^m m_jS_j+D_{ n+1}(X)\right)
        \end{equation*}
        Por tanto:
        \begin{equation*}
            \begin{split}
                f_\#\left(\sum_{ i=1}^k n_iT_i+D_n(X)\right)&=f_\#\circ\partial_{ n+1}\left(\sum_{ j=1}^m m_jS_j+D_{ n+1}(X)\right)\\
                &=\partial_{ n+1}\circ f_\#\left(\sum_{ j=1}^m m_jS_j+D_{ n+1}(X)\right)\\
            \end{split}
        \end{equation*}
        es decir, que $f_\#\left(\sum_{ i=1}^k n_iT_i+D_n(X)\right)\in B_n(Y)$, pues $f_\#\left(\sum_{ j=1}^m m_jS_j+D_{ n+1}(X)\right)\in C_{ n+1}(Y)$. De aquí se sigue la otra contención.
    \end{obs}

    La observación anterior nos permite definir un homomofismo $\cf{f_*}{H_n(X)}{H_n(Y)}$ de forma natural, dado por:
    \begin{equation*}
        f_*\left(\sum_{ i=1}^kn_iT_i+B_n(X)\right)=\sum_{ i=1}^kn_if_\#(T_i)+B_n(Y),
    \end{equation*}
    para todo $\sum_{ i=1}^kn_iT_i+B_n(X)\in H_n(X)$.

    \begin{mydef}[\textbf{Homomorfismo Inducido}]
        Sean $X$ y $Y$ espacios topológicos, y $\cf{f}{X}{Y}$ una función continua. Entonces, el homomorfismo $\cf{f_*}{H_n(X)}{H_n(Y)}$ es llamado el \textbf{homomorfismo inducido por $f$}.
    \end{mydef}

    De manera análoga podemos hacer esta definición con el homomorfismo inducido en homología reducida: $\cf{f_*}{\tilde{H}_0(X)}{\tilde{H}_0(Y)}$. Este homomorfismo cumple además que el diagrama:

    \begin{minipage}{\textwidth}
        \begin{center}
            \includegraphics[scale=1]{images/fig_5.pdf}\\
            Figura \thefigcount. Conmutatividad de $f_*$ con $\zeta$ y $\varepsilon_*$.
            \stepcounter{figcount}
        \end{center}
    \end{minipage}    

    es conmutativo.

    \begin{lema}
        Sea $X$ espacio topológico y $\cf{\bbm{1}_X}{X}{X}$ la función identidad. Entonces los siguientes homomorfismos inducidos:
        \begin{equation*}
            \begin{split}
                &\cf{\bbm{1}_\#}{Q_n(X)}{Q_n(X)},\\
                &\cf{\bbm{1}_\#}{C_n(X)}{C_n(X)},\\
                &\cf{\bbm{1}_*}{H_n(X)}{H_n(X)},\\
            \end{split}
        \end{equation*}
        y,
        \begin{equation*}
            \cf{\bbm{1}_*}{\tilde{H}_n(X)}{\tilde{H}_n(X)}
        \end{equation*}
        Coinciden con los homomorfismos identidad en cada uno de los grupos respectivos.
    \end{lema}

    \begin{proof}
        %TODO
    \end{proof}

    \begin{lema}
        Sean $X$, $Y$ y $Z$ espacios topológicos y, $\cf{f}{X}{Y}$ y $\cf{g}{Y}{Z}$ funciones continuas. Entonces:
        \begin{equation*}
            (g\circ f)_*=g_*\circ f_*,\quad\forall n\in\mathbb{N}
        \end{equation*}
        Donde $\cf{(g\circ f)_*}{H_n(X)}{H_n(Z)}$ es el homomorfismo inducido por $g\circ f$.
    \end{lema}

    \begin{proof}
        %TODO
    \end{proof}

    \begin{obs}
        Si $\cf{f}{X}{Y}$ es una función continua entre dos espacios topológicos tal que es inyectiva, ¿entonces $f_*$ es inyectiva? la respuesta es que no necesariamente. Esto se ilustrará más adelante.
    \end{obs}

    \section{}

    \newpage

    \section{Ejercicios}

    \subsection{Homología Simplicial}

    \subsection{Homología Singular}

    \begin{excer}
        Compute $\partial_n(T)$ para el caso en que $n=1,2$.
    \end{excer}

    \begin{excer}
        Determine la estructura del grupo de homotopía $H_n(X)$ con $n\geq0$ si $X$ es:
        \begin{itemize}
            \item $X=(\mathbb{Q},\tau_u)$ con $\tau_u$ la topología usual heredada de la topología usual de $\mathbb{R}$.
            \item $X$ es un espacio numerable con la topología discreta.
        \end{itemize}
    \end{excer}

    \begin{excer}
        3.1.\; Let $X_y$ be an arccomponent of $X$, and $f : X_y \to X$ the inclusion map.  
        Prove that $f_* : H_n(X_y) \to H_n(X)$ is a monomorphism, and the image is the direct summand of $H_n(X)$ corresponding to $X_y$, as described in Proposition 2.6.  
        Consequence: the direct sum decomposition of Proposition 2.6 can be described completely in terms of such homomorphisms which are induced by inclusion maps.
    \end{excer}

    \begin{excer}
    3.2.\; Let $X$ and $Y$ be spaces having a finite number of arcwise-connected components, and $f : X \to Y$ a continuous map.  
    Describe the induced homomorphism $f_* : H_0(X) \to H_0(Y)$.  
    Generalize to the case where $X$ or $Y$ have an infinite number of arc components.
    \end{excer}

    \begin{excer}
    3.3.\; Let $A$ be a retract of $X$ with retracting map $r : X \to A$, and let $i : A \to X$ denote the inclusion map.  
    Prove that $r_* : H_n(X) \to H_n(A)$ is an epimorphism, $i_* : H_n(A) \to H_n(X)$ is a monomorphism, and that $H_n(X)$ is the direct sum of the image of $i_*$ and the kernel of $r_*$.
    \end{excer}




\end{document}