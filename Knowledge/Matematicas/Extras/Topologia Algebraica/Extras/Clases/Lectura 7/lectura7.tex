\documentclass{article}
\usepackage[spanish]{babel}
\usepackage[utf8]{inputenc}
\usepackage{amsmath}
\usepackage{amssymb}
\usepackage{amsthm}
\usepackage{graphics}
\usepackage{subfigure}
\usepackage{lipsum}
\usepackage{array}
\usepackage{multicol}
\usepackage{enumerate}
\usepackage[framemethod=TikZ]{mdframed}
\usepackage[a4paper, margin = 1.5cm]{geometry}
\usepackage{fullpage}

%En esta parte se hacen redefiniciones de algunos comandos para que resulte agradable el verlos%

\renewcommand{\theenumii}{\roman{enumii}}

\def\proof{\paragraph{Demostración:\\}}
\def\endproof{\hfill$\blacksquare$\\}

\def\sol{\paragraph{Solución:\\}}
\def\endsol{\hfill$\square$\\}

%En esta parte se definen los comandos a usar dentro del documento para enlistar%

\newtheoremstyle{largebreak}
  {}% use the default space above
  {}% use the default space below
  {\normalfont}% body font
  {}% indent (0pt)
  {\bfseries}% header font
  {}% punctuation
  {\newline}% break after header
  {}% header spec

\theoremstyle{largebreak}

\newmdtheoremenv[
    leftmargin=0em,
    rightmargin=0em,
    innertopmargin=-2pt,
    innerbottommargin=8pt,
    hidealllines = true,
    roundcorner = 5pt,
    backgroundcolor = gray!60!red!30
]{exa}{Ejemplo}[section]

\newmdtheoremenv[
    leftmargin=0em,
    rightmargin=0em,
    innertopmargin=-2pt,
    innerbottommargin=8pt,
    hidealllines = true,
    roundcorner = 5pt,
    backgroundcolor = gray!50!blue!30
]{obs}{Observación}[section]

\newmdtheoremenv[
    leftmargin=0em,
    rightmargin=0em,
    innertopmargin=-2pt,
    innerbottommargin=8pt,
    rightline = false,
    leftline = false
]{theor}{Teorema}[section]

\newmdtheoremenv[
    leftmargin=0em,
    rightmargin=0em,
    innertopmargin=-2pt,
    innerbottommargin=8pt,
    rightline = false,
    leftline = false
]{propo}{Proposición}[section]

\newmdtheoremenv[
    leftmargin=0em,
    rightmargin=0em,
    innertopmargin=-2pt,
    innerbottommargin=8pt,
    rightline = false,
    leftline = false
]{cor}{Corolario}[section]

\newmdtheoremenv[
    leftmargin=0em,
    rightmargin=0em,
    innertopmargin=-2pt,
    innerbottommargin=8pt,
    rightline = false,
    leftline = false
]{lema}{Lema}[section]

\newmdtheoremenv[
    leftmargin=0em,
    rightmargin=0em,
    innertopmargin=-2pt,
    innerbottommargin=8pt,
    roundcorner=5pt,
    backgroundcolor = gray!30,
    hidealllines = true
]{mydef}{Definición}[section]

\newmdtheoremenv[
    leftmargin=0em,
    rightmargin=0em,
    innertopmargin=-2pt,
    innerbottommargin=8pt,
    roundcorner=5pt
]{excer}{Ejercicio}[section]

%En esta parte se colocan comandos que definen la forma en la que se van a escribir ciertas funciones%
\newcommand{\norm}[1]{\ensuremath{\|#1\|}}
\newcommand\subtitle[1]{\textit{\large #1}\\}
\newcommand\abs[1]{\ensuremath{\left|#1\right|}}
\newcommand\divides{\ensuremath{\bigm|}}
\newcommand\cf[3]{\ensuremath{#1:#2\rightarrow#3}}
\newcommand\natint[1]{\ensuremath{\left[\!\left[ #1\right]\!\right]}}
\newcommand{\afa}{\:
    \begin{tikzpicture}
        \draw [line width = 0.17 mm, black] (0,0) -- (-0.115,0.29);
        \draw [line width = 0.17 mm, black] (0,0) -- (0.115,0.29);
        \draw [line width = 0.17 mm, black] (-0.12,0) arc (190:-10:0.12cm);
    \end{tikzpicture}
    \:
}
%Este símvolo es para casi todo salvo una cantidad finita

%recuerda usar \clearpage para hacer un salto de página

\begin{document}

    \title{Taller de Topología Algebraica, Lectura 7: El Grupo Fundamental de $\mathbb{S}^1$}
    \author{Cristo Alvarado}
    \setcounter{section}{7}
    \maketitle

    \subtitle{$\pi(\mathbb{S}^1)$}

    Lo hecho en la parte anterior sobre espacios recubridores será utilizado para caracterizar en su totalidad el grupo fundamental del círculo $\mathbb{S}^1$.

    \begin{mydef}
        Sea $\cf{\omega}{I}{\mathbb{S}^1}$ el bucle con base $(1,0)$ que va alrededor del circulo exactamente una vez, es decir:
        \begin{equation*}
            \omega(s)=(\cos 2\pi s,\sin 2\pi s),\quad\forall s\in I
        \end{equation*}
        Para cada $n\in\mathbb{Z}$ se define:
        \begin{equation*}
            \omega_n(s)=(\cos 2\pi ns,\sin 2\pi ns),\quad\forall s\in I
        \end{equation*}
        es decir, este bucle da $n$-vueltas alrededor del círculo cuando $t$ varía de $0$ a $1$.
    \end{mydef}

    \begin{excer}
        Pruebe que:
        \begin{equation*}
            [\omega]^n=[\omega_n],\quad\forall n\in\mathbb{Z}
        \end{equation*}
        donde $[\omega]^n$ es el producto de la clase con representante $\omega$ $n$-veces.
    \end{excer}

    \begin{proof}
        Ejercicio.
    \end{proof}

    Para la demostración del teorema, lo se se hará es que dado un camino $\cf{f}{I}{\mathbb{S}^1}$, se comparará al mismo con los caminos que genera la función recubridora $\cf{p}{\mathbb{R}}{\mathbb{S}^1}$
    \begin{equation*}
        p(s)=(\cos 2\pi s,\sin 2\pi s),\quad\forall s\in I
    \end{equation*}
    este mapeo puede ser visualizado como una helicoide en $\mathbb{R}^3$.

    \begin{obs}
        Veamos que
        \begin{equation*}
            \omega_n=p\circ\widetilde{\omega}_n,\quad\forall n\in\mathbb{Z}
        \end{equation*}
        donde $\cf{\widetilde{\omega}_n}{I}{\mathbb{R}}$ es la función tal que $s\mapsto ns$. Básicamente esta función controla el número de giros adicionales que va a dar la helicoide en un solo intervalo de longitud 1. El signo de $n$ determina el sentido de giro de la helicoide.

        Note que $\widetilde{\omega}_n$ es un levantamiento de $\omega_n$, para todo $n\in\mathbb{Z}$.
    \end{obs}
    
    Para determinar el grupo fundamental del círculo, se estudiará como es que los caminos en $\mathbb{S}^1$ se levantan a $\mathbb{R}$.

    \begin{theor}
        El grupo fundamental $\pi(\mathbb{S}^1,(1,0))$ es cíclico infinito generado por $[\omega]$.
    \end{theor}

    \begin{proof}
        Sea $\cf{f}{I}{\mathbb{S}^1}$ un bucle con punto base $x_0=(1,0)$. Por el lema \ref{lemaF1} existe un levantamiento $\widetilde{f}$ empezando en $0$. Este camino termina en algún entero $n\in\mathbb{Z}$ pues al ser levantamiento se cumple que
        \begin{equation*}
            f = p\circ \widetilde{f}
        \end{equation*}
        luego, para $s=1$:
        \begin{equation*}
            p(\widetilde{f}(1))=p\circ\widetilde{f}(1)=f(1)=x_0
        \end{equation*}
        y,
        \begin{equation*}
            p^{-1}(x_0)=\mathbb{Z}\subseteq\mathbb{R}
        \end{equation*}
        Por tanto, debe existir $n\in\mathbb{Z}$ tal que $\widetilde{f}(1)=n$. Así $\widetilde{f}$ es un camino que va de $0$ a $n$. Otro camino que también hace lo mismo es $\widetilde{\omega}_n$. Además
        \begin{equation*}
            \widetilde{f}\sim \widetilde{\omega}_n
        \end{equation*}
        tomando la función continua $\cf{\widetilde{F}}{I\times I}{\mathbb{R}}$, $\widetilde{F}(t,s)=(1-t)\widetilde{f}(s)+t\widetilde{\omega}_n(s)$, para todo $s,t\in I$. Tomando $F(s,t)=p\circ \widetilde{F}(s,t)$ resulta en que $f\sim\omega_n$ (si le quedan dudas, es fácil de verificar que se cumple lo anterior), luego $[f]=[\omega_n]=[\omega]^n$. Se sigue entonces que $\pi(\mathbb{S}^1,(1,0))$ es generado por $[\omega]$.

        Para ver que es cíclico infinito, veamos que para el bucle anterior el entero $n$ es único. Suponga que $[f]$ está determinado por $[\omega_n]$ y $[\omega_m]$, es decir que
        \begin{equation*}
            f\sim\omega_n\quad\textup{y}\quad f\sim\omega_m
        \end{equation*}
        con $m\in\mathbb{Z}$. Se sigue que $\omega_n\sim\omega_m$. Por el lema \ref{lemaF2} se tiene que $\widetilde{\omega}_n\sim\widetilde{\omega}_m$. En particular, tienen el mismo punto terminal, por lo que
        \begin{equation*}
            m=\widetilde{\omega}_m(1)=\widetilde{\omega}_n(1)=n
        \end{equation*}
        así, $n=m$. Se sigue entonces que el orden de $[\omega]$ no puede ser finito ya que en caso contrario $f$ tendría al menos dos representaciones con este mismo generador de $\pi(\mathbb{S}^1,(1,0))$.

        Por lo tanto $\pi(\mathbb{S}^1,(1,0))$ es un grupo cíclico infinito generado por $[\omega]$, luego es isomorfo a $\mathbb{Z}$. Así:
        \begin{equation*}
            \pi(\mathbb{S}^1,(1,0))\cong\mathbb{Z}
        \end{equation*}
    \end{proof}

    \subtitle{Aplicaciones}

    \begin{obs}
        Podemos ver el círculo como subconjunto de $\mathbb{C}$ dado por:
        \begin{equation*}
            \mathbb{S}^1=\left\{x+iy\in\mathbb{C}\Big|x^2+y^2=1 \right\}
        \end{equation*}
        en particular, $1\in\mathbb{S}^1$ y el espacio $\mathbb{S}^1$ es arco-conexo, por lo que la elección del punto para calcular el grupo fundamental es independiente del elemento de $\mathbb{S}^1$.
    \end{obs}

    Como aplicación de los resultados anteriores, tenemos el siguiente teorema:

    \begin{theor}[\textbf{Teorema fundamental del álgebra}]
        Todo polinomio no constante con coeficientes en $\mathbb{C}$ tiene una raíz en $\mathbb{C}$.
    \end{theor}

    \begin{proof}
        Sea $\cf{p}{\mathbb{C}}{\mathbb{C}}$ un polinomio. Sin pérdida de generalidad, podemos asumir que
        \begin{equation*}
            p(z)=z^{ n}+a_1z^{ n-1}+...+a_n
        \end{equation*}
        donde $a_1,...,a_n\in\mathbb{C}$, con $n\geq0$. Si $n=0$ entonces $p(z)=1,\forall z\in\mathbb{C}$ (estamos diciendo que el polinomio $p$ es mónico).

        Supongamos que $p$ no tiene raíces en $\mathbb{C}$. Para cada $r\geq 0$ se define la función $\cf{f_r}{I}{\mathbb{C}}$ por:
        \begin{equation*}
            f_r(s)=\frac{p(re^{ 2\pi is})/p(r)}{\abs{p(re^{ 2\pi is})/p(r)}},\quad\forall s\in I
        \end{equation*}
        la cual es un bucle en el círculo unitario $\mathbb{S}^1\subseteq\mathbb{C}$ con base en 1. Se tiene que $f_0$ es el bucle constante con base en 1, por lo que $[f_0]$ es la identidad de $\pi(\mathbb{S}^1,1)$. Afirmamos que
        \begin{equation*}
            f_r\sim f_0
        \end{equation*}
        para todo $r>0$. En efecto, sea $r>0$ y $\cf{F}{I\times I}{\mathbb{S}^1}$ dada por:
        \begin{equation*}
            F(t,s)=f_{ rt}(s),\quad\forall  s,t\in I
        \end{equation*}
        Es claro de la definición de $f_r$ que $F(t,s)$ es continua. Veamos que:
        \begin{equation*}
            F(0,s)=f_0(s)\quad\textup{y}\quad F(1,s)=f_r(s),\quad\forall s\in I
        \end{equation*}
        Y además:
        \begin{equation*}
            F(t,0)=f_{ rt}(0)=1\quad\textup{y}\quad F(t,1)=f_{ rt}(1)=1,\quad\forall t\in I
        \end{equation*}
        (pues todos los bucles tienen como punto base a $1$). Por tanto, $f_0\sim f_r$ para todo $r\geq0$. Se sigue pues que
        \begin{equation*}
            [f_0]=[f_r],\quad\forall r\geq0
        \end{equation*}
        es decir que $[f_r]$ es la identidad de $\pi(\mathbb{S}^1,1)$ para todo $r\geq0$.

        Fijemos ahora $r>0$ tal que
        \begin{equation*}
            r>\max\left\{\abs{a_1}+...+\abs{a_n},1\right\}
        \end{equation*}
        Entonces si $z\in\mathbb{C}$ es tal que $\|z\|=r$, tenemos que
        \begin{equation*}
            \begin{split}
                t\abs{a_1z^{ n-1}+...+a_n}&\leq\abs{a_1z^{ n-1}+...+a_n}\\
                \abs{a_1z^{ n-1}+...+a_n}&\leq\abs{a_1z^{ n-1}}+...+\abs{a_n}\\
                &<(\abs{a_1}+...+\abs{a_n})\abs{z}^{n-1}\\
                &<\abs{z}^n\\
                \Rightarrow \abs{a_1z^{ n-1}+...+a_n}&<\abs{z}^n\\
                \Rightarrow 0 &< \abs{z}^n-\abs{a_1z^{ n-1}+...+a_n}\\
            \end{split}
        \end{equation*}
        para todo $t\in I$, lo cual implica que
        \begin{equation*}
            t\abs{a_1z^{ n-1}+...+a_n}<\abs{z}^n,\quad\forall t\in I
        \end{equation*}
        Por tanto, el polinomio
        \begin{equation*}
            p_t(z)=z^n+t\left(a_1z^{ n-1}+...+a_n \right),\quad\forall z\in\mathbb{C}
        \end{equation*}
        no tiene raíces cuando $\|z\|=r$ (círculo centrado en 0 de radio $r$) y cuando $t\in I$, pues si $z$ y $t$ cumplen estas condiciones se sigue que:
        \begin{equation*}
            \begin{split}
                \abs{p_t(z)}&=\abs{z^n+t\left(a_1z^{ n-1}+...+a_n \right)}\\
                &\geq\abs{z}^n-t\abs{a_1z^{ n-1}+...+a_n}\\
                &>0\\
            \end{split}
        \end{equation*}
        Definimos la función $\cf{G}{I\times I}{\mathbb{C}}$ dada por:
        \begin{equation*}
            G(t,s)=\frac{p_t(re^{ 2\pi is})/p_t(r)}{\abs{p_t(re^{ 2\pi is})/p_t(r)}},\quad\forall s,t\in I
        \end{equation*}
        Esta función es tal que $f_r\sim\omega_n$. En efecto, esta función es continua y cumple que:
        \begin{equation*}
            \begin{split}
                F(0,s)&=\frac{p_0(re^{ 2\pi is})/p_0(r)}{\abs{p_0(re^{ 2\pi is})/p_0(r)}}\\
                &=\frac{r^n e^{ 2\pi ins}/r^n}{\abs{r^n e^{ 2\pi ins}/r^n}}\\
                &=e^{ 2\pi ins}\\
                &=\cos(2\pi ns) +i\sin(2\pi ns)\\
                &=\omega_n(s),\quad\forall s\in I\\ 
            \end{split}
        \end{equation*}
        y,
        \begin{equation*}
            \begin{split}
                F(1,s)&=\frac{p_1(re^{ 2\pi is})/p_1(r)}{\abs{p_1(re^{ 2\pi is})/p_1(r)}}\\
                &=\frac{p(re^{ 2\pi is})/p(r)}{\abs{p(re^{ 2\pi is})/p(r)}}\\
                &=f_r(s),\quad\forall s\in I\\
            \end{split}
        \end{equation*}
        (el hecho ed que los extremos que se quedan fijos se verifica rápidamente). Anteriormente se probó que $\omega_n$ es tal que
        \begin{equation*}
            [\omega_n]=[\omega]^n
        \end{equation*}
        por ende,
        \begin{equation*}
            [f_r]=[\omega]^n
        \end{equation*}
        pero $[f_r]=[f_0]$ es la identidad del grupo, así que $n=0$. Luego
        \begin{equation*}
            p(z)=1,\quad\forall z\in\mathbb{C}
        \end{equation*}
        esto es, el único polinomio que no tiene raíces en $\mathbb{C}$ es el polinomio constante no cero.
    \end{proof}

    Ahora un resultado familiar enunciado al inicio del taller:

    \begin{theor}[\textbf{Teorema del punto fijo de Brower para dimensión 2}]
        Toda función continua $\cf{f}{\mathbb{D}^2}{\mathbb{D}^2}$ tiene un punto fijo, es decir existe $z\in\mathbb{D}^2$ tal que $f(z)=z$, donde:
        \begin{equation*}
            \mathbb{D}^2=\left\{(x,y)\in\mathbb{R}^2\Big|x^2+y^2\leq1 \right\}
        \end{equation*}
    \end{theor}

    \begin{proof}
        Suponga que para todo $z\in\mathbb{D}^2$ se tiene que $f(z)\neq z$. Definimos la función $\cf{r}{\mathbb{D}^2}{\mathbb{S}^1}$ dada como sigue:

        \begin{itemize}
            \item Sea $z=(x,y)\in\mathbb{D}^2$. Considere la función $\cf{l_x}{]0,\infty[}{\mathbb{R}^2}$ dada por:
            \begin{equation*}
                l_z(t)=(1-t)f(z)+tz=((1-t)f_1(x,y)+tx,(1-t)f_2(x,y)+ty)
            \end{equation*}
            (es la recta que comienza en $f(z)$ y va en dirección de $z$) donde $f(z)=(f_1(z),f_2(z))$. Por el teorema del valor medio se tiene que existe un único $t_x\in]0,\infty[$ tal que
            \begin{equation*}
                \|l_z(t_z)\|=1
            \end{equation*}
            En efecto, como $l_z\left(\frac{1}{2}\right)=\frac{f(z)+z}{2}$, este punto es tal que no puede tener norma igual a $1$, ya que en caso contrario:
            \begin{equation*}
                \begin{split}
                    \|\frac{f(z)+z}{2}\|=1&\Rightarrow\|f(z)+z\|=2\\
                    &\Rightarrow2\leq\|f(z)\|+\|z\|\leq2 \\
                    &\Rightarrow\|f(z)\|=\|z\|=1 \\
                \end{split}
            \end{equation*}
            por ende, $\|f(z)\|+\|z\|=\|f(z)+z\|$ así que deben ser coolineales. Como tienen la misma norma, entonces deben ser el mismo, lo cual no puede suceder ya que $f(z)\neq z$. Por tanto
            \begin{equation*}
                l_z\left(\frac{1}{2}\right)<1
            \end{equation*}
            y,
            \begin{equation*}
                \lim_{  t\rightarrow\infty}\|l_z(t)\|=\infty
            \end{equation*}
            del teorema del valor medio se sigue la existencia de tal $t_x\in]0,\infty[$. Este elemento está determinado por una de las raíces con $t\in]0,\infty[$ de la ecuación:
            \begin{equation*}
                \begin{split}
                    \|l_z(t)\|=1&\iff \|l_x(t)\|^2=1\\
                    &\iff\|((1-t)f_1(x,y)+tx,(1-t)f_2(x,y)+ty)\|^2=1\\
                    &\iff((1-t)f_1(x,y)+tx)^2+(1-t)f_2(x,y)+ty)^2=1\\
                \end{split}
            \end{equation*}
            el cual es un polinomio de grado 2. Por tanto, la aplicación $z\mapsto t_z$ es una función continua. Así, la aplicación $z\mapsto l_z(t_z)$ es una función continua (por ser composición de funciones continuas). Hacemos entonces $r(z)=l_z(t_z)$ la cual es continua y es tal que $t_z>0$.
            \item Si $z\in\mathbb{D}^2$ es tal que $\|z\|=1$, entonces $r(z)=z$ (pues la función $l_z$ solo pasa por un $t\in]0,\infty[$ tal que $\|l_z(t)\|=1$, en particular debe pasar por $z$).
        \end{itemize}

        Por los dos incisos anteriores, $\cf{r}{\mathbb{D}^2}{\mathbb{S}^1}$ es una función continua siendo $\mathbb{S}^1\subseteq\mathbb{D}^2$. Se tiene que $r$ es una retracción de $\mathbb{D}^2$. En particular, se sabe que la función
        \begin{equation*}
            \cf{r_*}{\pi(\mathbb{D}^2,(1,0))}{\pi(\mathbb{S}^1,(1,0))}
        \end{equation*}
        es un epimorfismo. Pero como $\mathbb{D}^2$ tiene forma de estrella respecto a $(0,0)$ se tiene que $\mathbb{D}^2$ es contraíble a un punto, luego es simplemente conexo. Así que
        \begin{equation*}
            \pi(\mathbb{D}^2,(1,0))=\langle e\rangle
        \end{equation*}
        y,
        \begin{equation*}
            \pi(\mathbb{S}^1,(1,0))\cong\mathbb{Z}
        \end{equation*}
        por ende, $r_*$ no puede ser epimorfismo, lo cual es una contradicción. De esta forma se sigue que debe existir $z\in\mathbb{D}^2$ tal que $f(z)=z$.
    \end{proof}
    
    El hecho de que $\pi(\mathbb{S}^1)\cong\mathbb{Z}$ puede ser usado para probar el siguiente resultado:

    \begin{theor}[\textbf{Teorema de Borsuk-Ulam en dimensión 2}]
        Para cada función continua $\cf{f}{\mathbb{S}^2}{\mathbb{R}^2}$ existen un par de puntos antipodales $x,-x\in\mathbb{S}^2$ tales que
        \begin{equation*}
            f(x)=f(-x)
        \end{equation*}
        donde
        \begin{equation*}
            \mathbb{S}^2=\left\{(x,y,z)\in\mathbb{R}^3\Big|x^2+y^2+z^2=1\right\}
        \end{equation*}
    \end{theor}

    \begin{proof}
        Sea $\cf{f}{\mathbb{S}^2}{\mathbb{R}^2}$ una función continua tal que
        \begin{equation*}
            f(x)\neq f(-x),\quad\forall x\in\mathbb{S}^2
        \end{equation*}
        Podemos definir entonces una función $\cf{g}{\mathbb{S}^2}{\mathbb{S}^1}$ tal que
        \begin{equation*}
            g(x)=\frac{f(x)-f(-x)}{\|f(x)-f(-x) \|},\quad\forall x\in\mathbb{S}^2
        \end{equation*}
        considerando a $\mathbb{S}^1$ como subconjunto de $\mathbb{R}^2$ es decir:
        \begin{equation*}
            \mathbb{S}^1=\left\{(x,y)\in\mathbb{R}^3\Big|x^2+y^2=1 \right\}
        \end{equation*}
        Definimos ahora el bucle $\cf{\eta}{I}{\mathbb{S}^2}$ como:
        \begin{equation*}
            \eta(s)=(\cos 2\pi s,\sin 2\pi s,0)
        \end{equation*}
        y hacemos
        \begin{equation*}
            h=g\circ \eta
        \end{equation*}
        éste es un bucle en $\mathbb{S}^1$. Como $-g(x)=g(-x)$ para todo $x\in\mathbb{S}^2$, se tiene que
        \begin{equation*}
            \begin{split}
                h(s+1/2)&=g\circ\eta(s+1/2)\\
                &=g(\cos (2\pi s+\pi),\sin (2\pi s+\pi),0)\\
                &=g(-\cos 2\pi s,-\sin 2\pi s,0)\\
                &=g(-(\cos 2\pi s,\sin 2\pi s,0))\\
                &=-g\circ\eta(s)\\
                &=-h(s),\quad\forall s\in[0,1/2]\\
            \end{split}
        \end{equation*}
        El camino $h$ puede ser levantado a un camino $\cf{\widetilde{h}}{I}{\mathbb{R}}$ empezando en cero, para el que se cumple que:
        \begin{equation*}
            p\circ \widetilde{h}=h
        \end{equation*}
        siendo $p$ la función tal que $s\mapsto(\cos 2\pi s,\sin 2\pi s)$. En particular,
        \begin{equation*}
            p(\widetilde{h}(s+1/2))+p(\widetilde{h}(s))=h(s+1/2)+h(s)=0
        \end{equation*}
        para todo $s\in[0,1/2]$. Esto es:
        \begin{equation*}
            \begin{split}
                0&=(\cos 2\pi \widetilde{h}(s+1/2),\sin 2\pi \widetilde{h}(s+1/2))+(\cos 2\pi \widetilde{h}(s),\sin 2\pi \widetilde{h}(s))\\
                &=(\cos 2\pi \widetilde{h}(s+1/2)+\cos 2\pi \widetilde{h}(s),\sin 2\pi \widetilde{h}(s+1/2)+\sin 2\pi \widetilde{h}(s))\\
                &=\left(2\cos\frac{2\pi(\widetilde{h}(s+1/2)+\widetilde{h}(s))}{2}\cos\frac{2\pi(\widetilde{h}(s+1/2)-\widetilde{h}(s))}{2}\right.\\
                &\quad\quad,\left.2\sin\frac{2\pi(\widetilde{h}(s+1/2)+\widetilde{h}(s))}{2}\cos\frac{2\pi(\widetilde{h}(s+1/2)-\widetilde{h}(s))}{2}\right)\\
                &=\left(2\cos\pi(\widetilde{h}(s+1/2)+\widetilde{h}(s))\cos\pi(\widetilde{h}(s+1/2)-\widetilde{h}(s))\right.\\
                &\quad\quad,\left.2\sin\pi(\widetilde{h}(s+1/2)+\widetilde{h}(s))\cos\pi(\widetilde{h}(s+1/2)-\widetilde{h}(s))\right)\\
            \end{split}
        \end{equation*}
        la única forma de que esto sea cero, es que
        \begin{equation*}
            \begin{split}
                \cos\pi(\widetilde{h}(s+1/2)-\widetilde{h}(s))=0&\iff \widetilde{h}(s+1/2)-\widetilde{h}(s)=\frac{q(s)}{2}
            \end{split}
        \end{equation*}
        donde $q(s)\in\mathbb{Z}$ es un impar, para todo $s\in I$. Pero la función $\cf{q}{[0,1/2]}{\mathbb{Z}}$ es continua, por lo que debe ser constante. Así que:
        \begin{equation*}
            \widetilde{h}(s+1/2)-\widetilde{h}(s)=\frac{q}{2}
        \end{equation*}
        para algún $q\in\mathbb{Z}$. En particular:
        \begin{equation*}
            \begin{split}
                \widetilde{h}(1)&=\widetilde{h}(1/2)+\frac{q}{2}\\
                &=\left(\widetilde{h}(0)+\frac{q}{2}\right)+\frac{q}{2}\\
                &=\widetilde{h}(0)+q\\
                &=q\\
            \end{split}
        \end{equation*}
        por ende, $h\sim\omega_q$. En particular esto implica que $h$ no es homotópica a una función contante ya que $q$ es impar (la única forma en que fuera homotópica a una función constante es si $q=0$).

        Pero $\eta$ es homotópica a la función $g$, $s\mapsto (0,0,1)$ que es constante. Para ello, tome la función $\cf{F}{I\times I}{\mathbb{S}^2}$ dada por:
        \begin{equation*}
            F(s,t)=\left(\cos 2\pi s\cos\frac{\pi t}{2},\sin 2\pi s\cos\frac{\pi t}{2},\sin\frac{\pi t}{2}\right)
        \end{equation*}
        (claramente esta función está bien definida y es continua), para la cual se cumple que:
        \begin{equation*}
            \begin{split}
                F(s,0)&=\left(\cos 2\pi s\cos0,\sin 2\pi s\cos0,0\right)\\
                &=\left(\cos 2\pi s\cos,\sin 2\pi s,0\right)\\
                &=\eta(s),\quad\forall s\in I\\
            \end{split}
        \end{equation*}
        y,
        \begin{equation*}
            \begin{split}
                F(s,1)&=\left(\cos 2\pi s\cos\frac{\pi}{2},\sin 2\pi s\cos\frac{\pi}{2},\sin\frac{\pi}{2}\right)\\
                &=\left(0,0,1\right)\\
            \end{split}
        \end{equation*}
        con lo que se tiene la homotopía deseada. Así, también debe suceder que $g\circ\eta$ sea homotópica a una función constante, es decir que $h$ es homotópica a una función constante, lo cual contradice lo probado anteriormente.

        Así que, existe $x\in\mathbb{S}^2$ tal que
        \begin{equation*}
            f(x)=f(-x)
        \end{equation*}
    \end{proof}

    \begin{cor}
        Si $\mathbb{S}^2$ es escrito como la unión de tres conjuntos cerrados $A_1,A_2,A_3\subseteq\mathbb{S}^2$, entonces uno de estos debe tener un par de puntos antipodales $\left\{x,-x\right\}$.
    \end{cor}

    \begin{proof}
        Sean $A_1,A_2,A_3\subseteq\mathbb{S}^2$ conjuntos tales que
        \begin{equation*}
            \mathbb{S}^2=A_1\cup A_2\cup A_3
        \end{equation*}
        para cada $i=1,2$ definimos la función $\cf{d_i}{\mathbb{S}^2}{\mathbb{R}}$ dada por:
        \begin{equation*}
            d_i(x)=\inf\left\{\|x-y\|\Big|y\in A_i \right\}
        \end{equation*}
        se sabe que esta es una función continua por cursos de análisis matemático. Por el teorema de Borsuk-Ulam, para la función $\cf{f}{\mathbb{S}^2}{\mathbb{R}^2}$ continua dada por:
        \begin{equation*}
            f(x)=(d_1(x),d_2(x))
        \end{equation*}
        existen un par de puntos antipodales $x,-x\in\mathbb{S}^2$ tales que:
        \begin{equation*}
            f(x)=f(-x)\Rightarrow d_1(x)=d_1(-x)\textup{ y }d_2(x)=d_2(-x)
        \end{equation*}
        Se tienen dos casos:
        \begin{itemize}
            \item $d_1(x)=d_1(-x)=0$ o $d_2(x)=d_2(-x)=0$, en el primer caso se sigue que al ser $A_1$ cerrado, entonces $x,-x\in A_1$. De forma análoga en el segundo caso se sigue que $x,-x\in A_2$.
            \item $d_1(x)=d_1(-x)>0$ y $d_2(x)=d_2(-x)>0$. Como los conjuntos $A_1$ y $A_2$ son cerrados, forzosamente debe suceder que $x,-x\notin A_1,A_2$. Por tanto, $x,-x\in A_3$ ya que $\mathbb{S}^2$ es la unión de los $A_i$.
        \end{itemize}
        por los dos incisos anteriores se tiene el resultado.
    \end{proof}

\end{document}