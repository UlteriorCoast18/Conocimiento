\documentclass[12pt]{report}
\usepackage[spanish]{babel}
\usepackage[utf8]{inputenc}
\usepackage{amsmath}
\usepackage{amssymb}
\usepackage{amsthm}
\usepackage{graphics}
\usepackage{subfigure}
\usepackage{lipsum}
\usepackage{array}
\usepackage{multicol}
\usepackage{enumerate}
\usepackage[framemethod=TikZ]{mdframed}
\usepackage[a4paper, margin = 1.5cm]{geometry}

%En esta parte se hacen redefiniciones de algunos comandos para que resulte agradable el verlos%

\renewcommand{\theenumii}{\roman{enumii}}

\def\proof{\paragraph{Demostración:\\}}
\def\endproof{\hfill$\blacksquare$}

\def\sol{\paragraph{Solución:\\}}
\def\endsol{\hfill$\square$}

%En esta parte se definen los comandos a usar dentro del documento para enlistar%

\newtheoremstyle{largebreak}
  {}% use the default space above
  {}% use the default space below
  {\normalfont}% body font
  {}% indent (0pt)
  {\bfseries}% header font
  {}% punctuation
  {\newline}% break after header
  {}% header spec

\theoremstyle{largebreak}

\newmdtheoremenv[
    leftmargin=0em,
    rightmargin=0em,
    innertopmargin=-2pt,
    innerbottommargin=8pt,
    hidealllines = true,
    roundcorner = 5pt,
    backgroundcolor = gray!60!red!30
]{exa}{Ejemplo}[section]

\newmdtheoremenv[
    leftmargin=0em,
    rightmargin=0em,
    innertopmargin=-2pt,
    innerbottommargin=8pt,
    hidealllines = true,
    roundcorner = 5pt,
    backgroundcolor = gray!50!blue!30
]{obs}{Observación}[section]

\newmdtheoremenv[
    leftmargin=0em,
    rightmargin=0em,
    innertopmargin=-2pt,
    innerbottommargin=8pt,
    rightline = false,
    leftline = false
]{theor}{Teorema}[section]

\newmdtheoremenv[
    leftmargin=0em,
    rightmargin=0em,
    innertopmargin=-2pt,
    innerbottommargin=8pt,
    rightline = false,
    leftline = false
]{propo}{Proposición}[section]

\newmdtheoremenv[
    leftmargin=0em,
    rightmargin=0em,
    innertopmargin=-2pt,
    innerbottommargin=8pt,
    rightline = false,
    leftline = false
]{cor}{Corolario}[section]

\newmdtheoremenv[
    leftmargin=0em,
    rightmargin=0em,
    innertopmargin=-2pt,
    innerbottommargin=8pt,
    rightline = false,
    leftline = false
]{lema}{Lema}[section]

\newmdtheoremenv[
    leftmargin=0em,
    rightmargin=0em,
    innertopmargin=-2pt,
    innerbottommargin=8pt,
    roundcorner=5pt,
    backgroundcolor = gray!30,
    hidealllines = true
]{mydef}{Definición}[section]

\newmdtheoremenv[
    leftmargin=0em,
    rightmargin=0em,
    innertopmargin=-2pt,
    innerbottommargin=8pt,
    roundcorner=5pt
]{excer}{Ejercicio}[section]

%En esta parte se colocan comandos que definen la forma en la que se van a escribir ciertas funciones%

\newcommand\abs[1]{\ensuremath{\left|#1\right|}}
\newcommand\divides{\ensuremath{\bigm|}}
\newcommand\cf[3]{\ensuremath{#1:#2\rightarrow#3}}
\newcommand\natint[1]{\ensuremath{\left[\!\left[ #1\right]\!\right]}}
\newcommand{\afa}{\:
    \begin{tikzpicture}
        \draw [line width = 0.17 mm, black] (0,0) -- (-0.115,0.29);
        \draw [line width = 0.17 mm, black] (0,0) -- (0.115,0.29);
        \draw [line width = 0.17 mm, black] (-0.12,0) arc (190:-10:0.12cm);
    \end{tikzpicture}
    \:
}
%Este símvolo es para casi todo salvo una cantidad finita

%recuerda usar \clearpage para hacer un salto de página

\begin{document}
    \setlength{\parskip}{5pt} % Añade 5 puntos de espacio entre párrafos
    \setlength{\parindent}{12pt} % Pone la sangría como me gusta
    \title{Un Curso Introductorio en Topología Algebraica}
    \author{Cristo Daniel Alvarado}
    \maketitle

    \tableofcontents %Con este comando se genera el índice general del libro%

    %\setcounter{chapter}{3} %En esta parte lo que se hace es cambiar la enumeración del capítulo%
    
    \chapter{Introducción}
    
    A lo largo del curso (y estudiando temas de topología) llega a resultar de útilidad analizar el siguiente problema:

    \begin{center}
        \textbf{¿Cuándo dos espacios topológicos $X$ e $Y$ son homeomorfos?}
    \end{center}

    Desafortunadamente, esta pregunta resulta en extremo compleja de analizar. Analicemos por ejemplo los siguientes subespacios de $\mathbb{R}^2$:

    %poner el ejemplo que dejó quintín en sus notas

    \chapter{El Grupo Fundamental}

    El concepto de grupo fundamental 

    \section{Conceptos Fundamentales}

    De ahora en adelante, $I$ denotará al intervalo $[0,1]$.

    \begin{mydef}
        Un \textbf{camino} o \textbf{arco} en un espacio topológico $X$, es una función continua $\cf{f}{[a,b]}{X}$ de un intervalo cerrado en $X$.

        Las imágenes $f(a)$ y $f(b)$ son llamadas \textbf{puntos finales del camino o arco}. $f(a)$ es llamado \textbf{punto inicial} y $f(b)$ \textbf{punto final}.
    \end{mydef}

    \begin{obs}
        Por comodidad, dado a que existe un homeomorfismo lineal entre $[0,1]$ y $[a,b]$ (vistos como subespacios de $\mathbb{R}$ dotado de la topología usual) siendo $a,b\in\mathbb{R}$ con $a<b$ podemos ver a todos los caminos o arcos de un espacio topológico $X$ como funciones continuas de $I$ en $X$. Cuando sea más conveniente de esta manera, se usará esta convención.
    \end{obs}

    \begin{mydef}
        Un espacio topológico $X$ es llamado \textbf{conexo por arcos} o \textbf{arco-conexo} si cualesquiera dos puntos de $X$ pueden ser unidos mediante un arco, es decir tales que los puntos finales del arco coincidan con estos dos puntos.
    \end{mydef}

    \begin{theor}
        Todo espacio topológico arco-conexo es conexo.
    \end{theor}

    \begin{proof}
        Ejercicio.
    \end{proof}

    Como una sugerencia para la demostración del teorema anterior, recuerde el \textit{teorema del cactus} (la unión de una familia de conjuntos conexos tales que la intersección de la familia es no vacía, es un conjunto conexo).

    El recíproco del teorema anterior no es cierto como se ha visto en varios cursos pasados (recuerde Cálculo III).

    \begin{mydef}
        Sea $X$ un espacio topológico. Para cada $x\in X$ se define:
        \begin{equation*}
            \mathcal{A}(x)=\left\{y\in X\Big|\textup{ existe una función continua }\cf{f}{I}{X} \textup{ tal que }f(0)=x \textup{ y }f(1)=y \right\}
        \end{equation*}
        Se construye así la familia $\left\{\mathcal{A}(x) \right\}_{ x\in X}$. Esta familia forma una partición de $X$ y se denomina como \textbf{las componentes arco-conexas de $X$}.
    \end{mydef}

    En el sentido de la definición anterior, estamos obteniendo los subconjuntos de $X$ que son arco conexos más \textit{grandes} que tiene. Si $X$ es arco-conexo, entonces $\mathcal{A}(x)=X$, para todo $x\in X$.

    Las componentes arco-conexas de $X$ no necesariamente son conjuntos cerrados o abiertos.

    \begin{mydef}
        Un espacio topológico $X$ es \textbf{localmente arco-conexo} si cada punto tiene una familia básica de vecindades arco-conexas.
    \end{mydef}

    %TODO Explicar el concepto más profundamente.

    \begin{mydef}
        Sean $\cf{f,g}{[a,b]}{X}$ dos arcos en $X$ tales que $f(a)=g(a)$ y $f(b)=g(b)$ (esto es, que ambos arcos tienen los mismos puntos terminales). Decimos que estos dos arcos son \textbf{equivalentes}, denotándolo por $f\sim g$, si existe una función continua
        \begin{equation*}
            \cf{F}{[a,b]\times I}{X}
        \end{equation*}
        tal que
        \begin{equation*}
            F(t,0)=f(t)\quad\textup{y}\quad F(t,1)=g(t)
        \end{equation*}
        para todo $t\in[a,b]$ y,
        \begin{equation*}
            F(a,s)=f(a)=g(a)\quad\textup{y}\quad F(b,s)=f(b)=g(b)
        \end{equation*}
        para todo $s\in I$.
    \end{mydef}

    \begin{propo}
        La relación de la definición anterior es una relación de equivalencia en el conjunto de todos los arcos con mismos puntos terminales de un espacio topológico $X$.
    \end{propo}

    Notemos que el concepto anterior es casi el mismo que el de homotopía, considerando de forma adicional que en esta definición se dejen fijos los puntos terminales de ambos arcos.

    \begin{mydef}
        Sean $X$ e $Y$ dos espacios topológicos. Se dice que dos funciones $\cf{f,g}{X}{Y}$ son \textbf{homotópicas} si existe una función continua $\cf{F}{X\times I}{Y}$ tal que
        \begin{equation*}
            F(x,0)=f(x)\quad\textup{y}\quad F(x,1)=g(x)
        \end{equation*}
        para todo $x\in X$.
    \end{mydef}

    \begin{obs}
        En los dos casos de las definiciones anteriores, se dota a los espacios de la topología producto.
    \end{obs}

    Intuitivamente lo que uno hace es defomar, sin perder la continuidad, un arco en el otro en el espacio $X$, dejando fijos los puntos terminales fijos en todo momento de la deformación.

    Además de esta relación inducida, queremos definir una operación para dos arcos con ciertas propiedades:

    \begin{mydef}
        Sea $X$ espacio topológico y $\cf{f}{[a,b]}{X}$ y $\cf{g}{[b,c]}{X}$ arcos tales que $f(b)=g(b)$ (siendo $a<b<c$). Entonces el producto $f\cdot g$ se define como:
        \begin{equation*}
            (f\cdot g)(t)=\left\{
                \begin{array}{lcr}
                    f(t) & \textup{ si } & t\in[a,b]\\
                    g(t) & \textup{ si } & t\in[b,c]\\
                \end{array}
            \right.
        \end{equation*}
        para todo $t\in[a,c]$.
    \end{mydef}

    \begin{obs}
        Notemos que esta operación genera otro arco, es decir otra función continua $\cf{f\cdot g}{[a,c]}{X}$.
    \end{obs}

    En lo que sigue, se usará el intervalo $I=[0,1]$.

    \section{El Grupo Fundamental de un Espacio Topológico}

    De ahora en adelante, todo camino tendrá como dominio a $I$.

    \begin{mydef}[Nombre]
        
    \end{mydef}




    \newpage

    \section{Ejercicios}

    \begin{excer}
        Pruebe que un espacio conexo y localmente arco-conexo es arco conexo.
    \end{excer}

    \begin{excer}
        Construya una deformación de retracción de $\mathbb{R}^n$ en $S^{ n-1}$.
    \end{excer}

    \begin{sol}
        
    \end{sol}

\end{document}