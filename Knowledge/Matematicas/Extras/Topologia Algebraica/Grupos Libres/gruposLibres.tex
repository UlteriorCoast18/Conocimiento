\documentclass[12pt]{report}
\usepackage[spanish]{babel}
\usepackage[utf8]{inputenc}
\usepackage{amsmath}
\usepackage{amssymb}
\usepackage{amsthm}
\usepackage[mathscr]{euscript}
\usepackage{graphics}
\usepackage{wrapfig}
\usepackage{subfigure}
\usepackage{lipsum}
\usepackage{array}
\usepackage{multicol}
\usepackage{enumerate}
\usepackage[framemethod=TikZ]{mdframed}
\usepackage[a4paper, margin = 1.5cm]{geometry}
\usepackage{bbm}
\usepackage{float}

%En esta parte se hacen redefiniciones de algunos comandos para que resulte agradable el verlos%

\renewcommand{\theenumii}{\roman{enumii}}

\def\proof{\paragraph{Demostración:\\}}
\def\endproof{\hfill$\blacksquare$}

\def\sol{\paragraph{Solución:\\}}
\def\endsol{\hfill$\square$}

%En esta parte se definen los comandos a usar dentro del documento para enlistar%

\newtheoremstyle{largebreak}
  {}% use the default space above
  {}% use the default space below
  {\normalfont}% body font
  {}% indent (0pt)
  {\bfseries}% header font
  {}% punctuation
  {\newline}% break after header
  {}% header spec

\theoremstyle{largebreak}

\newmdtheoremenv[
    leftmargin=0em,
    rightmargin=0em,
    innertopmargin=-2pt,
    innerbottommargin=8pt,
    hidealllines = true,
    roundcorner = 5pt,
    backgroundcolor = gray!60!red!30
]{exa}{Ejemplo}[section]

\newmdtheoremenv[
    leftmargin=0em,
    rightmargin=0em,
    innertopmargin=-2pt,
    innerbottommargin=8pt,
    hidealllines = true,
    roundcorner = 5pt,
    backgroundcolor = gray!50!blue!30
]{obs}{Observación}[section]

\newmdtheoremenv[
    leftmargin=0em,
    rightmargin=0em,
    innertopmargin=-2pt,
    innerbottommargin=8pt,
    rightline = false,
    leftline = false
]{theor}{Teorema}[section]

\newmdtheoremenv[
    leftmargin=0em,
    rightmargin=0em,
    innertopmargin=-2pt,
    innerbottommargin=8pt,
    rightline = false,
    leftline = false
]{propo}{Proposición}[section]

\newmdtheoremenv[
    leftmargin=0em,
    rightmargin=0em,
    innertopmargin=-2pt,
    innerbottommargin=8pt,
    rightline = false,
    leftline = false
]{cor}{Corolario}[section]

\newmdtheoremenv[
    leftmargin=0em,
    rightmargin=0em,
    innertopmargin=-2pt,
    innerbottommargin=8pt,
    rightline = false,
    leftline = false
]{lema}{Lema}[section]

\newmdtheoremenv[
    leftmargin=0em,
    rightmargin=0em,
    innertopmargin=-2pt,
    innerbottommargin=8pt,
    roundcorner=5pt,
    backgroundcolor = gray!30,
    hidealllines = true
]{mydef}{Definición}[section]

\newmdtheoremenv[
    leftmargin=0em,
    rightmargin=0em,
    innertopmargin=-2pt,
    innerbottommargin=8pt,
    roundcorner=5pt
]{excer}{Ejercicio}[section]

%En esta parte se colocan comandos que definen la forma en la que se van a escribir ciertas funciones%

\newcommand\abs[1]{\ensuremath{\left|#1\right|}}
\newcommand\divides{\ensuremath{\bigm|}}
\newcommand\cf[3]{\ensuremath{#1:#2\rightarrow#3}}
\newcommand\natint[1]{\ensuremath{\left[\!\left[ #1\right]\!\right]}}
\newcommand{\afa}{\:
    \begin{tikzpicture}
        \draw [line width = 0.17 mm, black] (0,0) -- (-0.115,0.29);
        \draw [line width = 0.17 mm, black] (0,0) -- (0.115,0.29);
        \draw [line width = 0.17 mm, black] (-0.12,0) arc (190:-10:0.12cm);
    \end{tikzpicture}
    \:
}
\newcommand{\bbm}[1]{\mathbbm{#1}}
\newcounter{figcount}
\setcounter{figcount}{1}
%Este símvolo es para casi todo salvo una cantidad finita

%recuerda usar \clearpage para hacer un salto de página

\begin{document}
    \setlength{\parskip}{5pt} % Añade 5 puntos de espacio entre párrafos
    \setlength{\parindent}{12pt} % Pone la sangría como me gusta
    \title{Notas Taller Topología Algebraica}
    \author{Cristo Daniel Alvarado}
    \maketitle

    \tableofcontents %Con este comando se genera el índice general del libro%

    \setcounter{chapter}{1} %En esta parte lo que se hace es cambiar la enumeración del capítulo%
    
    \chapter{Grupos Libres y Productos de Grupos Libres}
    
    En los capítulos siguientes será indispensable el tratar con este tipo de grupos dada la naturaleza del grupo fundamental de los espacios topológicos.
    
    \section{Producto Débil de Grupos}
    
    \begin{obs}
        De ahora en adelante, el símbolo $\afa$ significa \textit{para casi todo salvo una cantidad finita de elementos}.
    \end{obs}

    \begin{obs}
        En esta parte, $I$ no denotará al intervalo $[0,1]$, sino a una indexación de una familia.
    \end{obs}

    \begin{mydef}
        Sea $\mathcal{G}=\left\{G_i \right\}_{ i\in I}$ una familia arbitraria no vacía de grupos. Se define el \textbf{producto directo de la familia $\mathcal{G}$} por:
        \begin{equation*}
            \prod\mathcal{G}=\left\{\cf{x}{I}{\prod_{ i\in I}G_i}\Big|x\textup{ es función} \right\}
        \end{equation*}
        y en ocasiones se denotará simplemente por $\prod_{ i\in I}G_i$. Se dota a este conjunto de la siguiente operación: si $x,y\in\prod\mathcal{G}$, entonces $\cf{x\cdot y}{I}{\prod_{ i\in I}G_i}$ es la función tal que
        \begin{equation*}
            (x\cdot y)(i)=x(i)\cdot y(i)
        \end{equation*}
        para todo $i\in I$, siendo la multiplicación respectiva en cada grupo.
    \end{mydef}

    En caso de que no lo haya hecho, queda como ejercicio al lector probar que el producto directo de una familia de grupos $\mathcal{G}$ es un grupo dotado de la operación de la definición anterior.

    \begin{mydef}
        Sea $\mathcal{G}=\left\{G_i \right\}_{ i\in I}$ una familia arbitraria no vacía de grupos. Se define el \textbf{producto débil de la familia $\mathcal{G}$} como el subgrupo de $\prod\mathcal{G}$ dado por:
        \begin{equation*}
            \prod\mathcal{G}^*=\left\{x\in\prod\mathcal{G}\Big|x(i)=e_i,\afa i\in I \right\}
        \end{equation*}
        donde $e_i$ denota la identidad de $G_i$ para cada $i\in I$.
    \end{mydef}

    \begin{propo}
        Si $\mathcal{G}$ es una familia arbitraria no vacía de grupos, entonces
        \begin{equation*}
            \prod\mathcal{G}^*<\prod\mathcal{G}
        \end{equation*}
        es decir, que $\prod\mathcal{G}^*$ es un subgrupo de $\prod\mathcal{G}$.
    \end{propo}

    \begin{proof}
        Ejercicio.
    \end{proof}

    \begin{mydef}
        Sea $\mathcal{G}=\left\{G_i \right\}_{ i\in I}$ una familia no vacía de grupos. Si $G_i$ es abeliano para cada $i\in I$, entonces llamaremos a $\prod\mathcal{G}^*$ la \textbf{suma directa de los grupos $G_i$}. En este caso, se denotará la operación del grupo de forma aditiva y se le denotará por:
        \begin{equation*}
            \prod\mathcal{G}^*=\bigoplus_{ i\in I}G_i=\bigoplus\mathcal{G}
        \end{equation*}
    \end{mydef}

    \begin{obs}
        Note que ambas definiciones coinciden si $I$ es un conjunto finito.
    \end{obs}

    \begin{mydef}
        En las condiciones de la definición anterior, para cada índice $i\in I$ definimos un \textbf{monomorfismo natural} $\cf{\varphi_i}{G_i}{\prod\mathcal{G}^*}$ definido como sigue: $\forall g\in G_i$ y para todo $j\in I$:
        \begin{equation*}
            \varphi_i(g)_j(\varphi_i(g))(j)=\left\{
                \begin{array}{lcr}
                    g & \textup{ si } & i = j\\
                    e_j & \textup{ si } & i\neq j\\
                \end{array}
            \right.
        \end{equation*}
    \end{mydef}

    En el caso en que cada $G_i$ sea un grupo abeliano, el siguiente teorema da una caracterización importante de su producto débil y de los monomorfismos $\varphi_i$.

    \begin{theor}
        \label{caractSumDir_1}
        Si $\left\{G_i \right\}_{ i\in I}$ es una familia no vacía de grupos abelianos y,
        \begin{equation*}
            G=\bigoplus_{ i\in I}G_i
        \end{equation*}
        entonces para cualquier grupo abeliano $A$ y cualquier familia de homomorfismos $\left\{\psi_i\right\}_{ i\in I}$ tales que
        \begin{equation*}
            \cf{\psi_i}{G_i}{A},\quad\forall i\in I
        \end{equation*}
        Existe un único homomorfismo $\cf{f}{G}{A}$ tal que para todo $i\in I$ el siguiente diagrama es conmutativo:

        \begin{minipage}{\textwidth}
            \begin{center}
                \includegraphics[scale=1.5]{images/fig_1.pdf}\\
                Figura \thefigcount. Diagrama conmutativo de $G$ y $A$.
                \stepcounter{figcount}
            \end{center}
        \end{minipage}

        esto es, $f\circ\varphi_i=\psi_i$ para todo $i\in I$.
    \end{theor}

    \begin{proof}
        Sean $A$ un grupo abeliano y $\left\{\psi_i\right\}_{ i\in I}$ una familia de homomorfismos tales que
        \begin{equation*}
            \cf{\psi_i}{G_i}{A},\quad\forall i\in I
        \end{equation*}
        sea ahora $x\in G$, como $x_i=e_i,\afa i\in I$ se tiene pues al ser $\psi_i$ homomorfismos debe suceder que $\psi_i(x_i)=e_A,\afa i\in I$ (siendo $e_A$ la identidad de $A$). Por lo cual la suma
        \begin{equation*}
            \sum_{ i\in I}\psi_i(x_i)
        \end{equation*}
        está bien definido (pues solo una cantidad finita de estos elementos es diferente de la identidad). Hacemos
        \begin{equation*}
            f(x)=\sum_{ i\in I}\psi_i(x_i),\quad\forall x\in G
        \end{equation*}
        Veamos que esta función está bien definida, ya se tiene por lo anterior que $\cf{f}{G}{A}$. Sea $x\in G$, si $x$ se expresa como
        \begin{equation*}
            x=\sum_{ j=1}^n y_j \quad\textup{y}\quad x=\sum_{ k=1}^m z_k
        \end{equation*}
        con $y_j,z_k\in G$ para todo $j\in\natint{1,n}$ y para todo $k\in\natint{1,m}$, sea
        \begin{equation*}
            I_0=\left\{i_1,...,i_r \right\}
        \end{equation*}
        el subconjunto de $I$ tal que si $i\in I_0$, entonces
        \begin{equation*}
            {y_j}_i\neq e_i\quad\textup{o}\quad{ z_k}_i\neq e_i 
        \end{equation*}
        para algún $j\in\natint{1,n}$ o algún $k\in\natint{1,m}$. Veamos que
        \begin{equation*}
            \begin{split}
                f\left(\sum_{ j=1}^n y_j \right)&=\sum_{ i\in I}\psi_i\left(\left(\sum_{ j=1}^n y_j\right)_i\right)\\
                &=\sum_{ i\in I_0}\psi_i\left(\sum_{ j=1}^n {y_j}_i\right)\\
                &=\sum_{ i\in I_0}\sum_{ j=1}^n \psi_{ i}\left( {y_j}_{i}\right)\\
            \end{split}
        \end{equation*}
        y,
        \begin{equation*}
            \begin{split}
                f\left(\sum_{ k=1}^m z_k \right)&=\sum_{ i\in I}\psi_i\left(\left(\sum_{ k=1}^m z_k\right)_i\right)\\
                &=\sum_{ i\in I_0}\psi_i\left(\sum_{ k=1}^m {z_k}_i\right)\\
                &=\sum_{ i\in I_0}\sum_{ k=1}^m \psi_{ i}\left( {z_k}_{i}\right)\\
            \end{split}
        \end{equation*}
        por ende,
        \begin{equation*}
            \begin{split}
                f\left(\sum_{ j=1}^n y_j \right)-f\left(\sum_{ k=1}^m z_k \right)&=\sum_{ i\in I_0}\sum_{ j=1}^n \psi_{ i}\left( {y_j}_{i}\right)-\sum_{ i\in I_0}\sum_{ k=1}^m \psi_{ i}\left( {z_k}_{i}\right)\\
                &=\sum_{ i\in I_0}\left(\sum_{ j=1}^n \psi_{ i}\left( {y_j}_{i}\right)-\sum_{ k=1}^m \psi_{ i}\left( {z_k}_{i}\right)\right)\\
                &=\sum_{ i\in I_0}\left(\psi_{ i}\left(\sum_{ j=1}^n {y_j}_{i}\right)-\psi_{ i}\left(\sum_{ k=1}^m{z_k}_{i}\right)\right)\\
                &=\sum_{ i\in I_0}\left(\psi_{ i}\left(\sum_{ j=1}^n {y_j}_{i}\right)-\psi_{ i}\left(\sum_{ k=1}^m{z_k}_{i}\right)\right)\\
                &=\sum_{ i\in I_0}\left(\psi_{ i}\left(\sum_{ j=1}^n {y_j}_{i}-\sum_{ k=1}^m{z_k}_{i}\right)\right)\\
            \end{split}
        \end{equation*}
        %TODO Algo falta aquí sobre la conmutatividad de los $G_i$
        pero $x_i=\sum_ { j=1}^n {y_j}_i$ y $x_i=\sum_ { k=1}^m {z_k}_i$, por lo cual
        \begin{equation*}
            \begin{split}
                f\left(\sum_{ j=1}^n y_j \right)-f\left(\sum_{ k=1}^m z_k \right)&=0\\
                \Rightarrow f\left(\sum_{ j=1}^n y_j \right)&=f\left(\sum_{ k=1}^m z_k \right)\\
            \end{split}
        \end{equation*}
        Se sigue que $f$ está bien definida. Veamos que es homomorfismo. Sean $x,y\in G$, entonces
        \begin{equation*}
            \begin{split}
                f(x+y)&=\sum_{ i\in I}\psi_i(x_i+y_i)\\
                &=\sum_{ i\in I}\psi_i(x_i)+\psi_i(y_i)\\
            \end{split}
        \end{equation*}
        como $A$ es abeliano, esta suma se puede reordenar de cualquier forma, en particular:
        \begin{equation*}
            \begin{split}
                f(x+y)&=\sum_{ i\in I}\psi_i(x_i)+\psi_i(y_i)\\
                &=\sum_{ i\in I}\psi_i(x_i)+\sum_{ i\in I}\psi_i(y_i)\\
                &=f(x)+f(y)\\
            \end{split}
        \end{equation*}
        por lo que $f$ es homomorfismo. Sea ahora $i\in I$, entonces
        \begin{equation*}
            \begin{split}
                f\circ\varphi_i(x_i)&=f(\varphi_i(x_i))\\
                &=\sum_{ j\in I}\psi_j({\varphi_i(x_i)}_j)\\
                &=\psi_i({\varphi_i(x_i)}_i)\\
                &=\psi_i(x_i),\quad\forall x_i\in G_i\\
            \end{split}
        \end{equation*}
        Luego, $f\circ\varphi_i=\psi_i$ para todo $i\in I$. Veamos la unicidad. Para ello, recordemos antes que si $x\in G$, entonces
        \begin{equation*}
            x=\sum_{ i\in I}\varphi_i(x_i)
        \end{equation*}
        (básicamente $x$ se expresa como la suma de sus componentes una por una vistas como elementos de $G$) siendo esta suma finita y por ende, está bien definida. Si $\cf{g}{G}{A}$ es otro homomorfismo tal que
        \begin{equation*}
            g\circ\varphi_i=\psi_i,\quad\forall i\in I
        \end{equation*}
        se tiene que
        \begin{equation*}
            \begin{split}
                g(x)&=g\left(\sum_{ i\in I}\varphi_i(x_i)\right)\\
                &=\sum_{ i\in I} g\circ\varphi_i(x_i)\\
                &=\sum_{ i\in I}\psi_i(x_i)\\
                &=f(x),\quad\forall x\in G\\
            \end{split}
        \end{equation*}
        por ende, $f$ es único.
    \end{proof}

    Este teorema caracteriza la suma directa de grupos abelianos, como lo muestra la siguiente proposición.

    \begin{propo}
        \label{caractSumDir_2}
        Sea $\left\{G_i \right\}_{ i\in I}$ una familia de grupos abelianos y $G=\bigoplus_{ i\in I}G_i$; sea $G'$ un subgrupo abeliano y consideremos para cada $i\in I$ las funciones $\cf{\varphi_i'}{G_i}{G'}$ tales que la conclusión del Teorema \ref{caractSumDir_1} cambiando a $G'$  y $\varphi_i'$ por $G$ y $\varphi_i$, respectivamente. Entonces, existe un único isomorfismo $\cf{h}{G}{G'}$ tal que el siguiente diagrama:
        
        \begin{minipage}{\textwidth}
            \begin{center}
                \includegraphics[scale=1.5]{images/fig_3.pdf}\\
                Figura \thefigcount. Diagrama conmutativo de $G$ y $G'$.
                \stepcounter{figcount}
            \end{center}
        \end{minipage}

        es conmutativo, para todo $i\in I$.
    \end{propo}

    \begin{proof}
        La existencia de un único homomrfismo $\cf{h}{G}{G'}$ que haga que el diagrama anterior sea conmutativo es inmediata del Teorema \ref{caractSumDir_1}. Por este mismo teorema, existe un único homomorfismo $\cf{k}{G'}{G}$ tal que el siguiente diagrama es conmutativo, para todo $i\in I$ (cambiando los papeles de $G'$ por $G$ y de $\varphi'$ por $\varphi_i$):
        
        \begin{minipage}{\textwidth}
            \begin{center}
                \includegraphics[scale=1.5]{images/fig_4.pdf}\\
                Figura \thefigcount. Diagrama conmutativo de $G'$ y $G$.
                \stepcounter{figcount}
            \end{center}
        \end{minipage}

        Se sigue de estos dos diagramas que:
        \begin{equation*}
            (k\circ h)\circ\varphi_i=\varphi_i\quad\textup{y}\quad(h\circ k)\circ\varphi_i'=\varphi_i'
        \end{equation*}
        para todo $i\in I$, estamos diciendo que los diagramas:
        
        \begin{minipage}{\textwidth}
            \begin{center}
                \includegraphics[scale=1.5]{images/fig_5.pdf}\\
                Figura \thefigcount. Diagramas conmutativos de $G$ y $G'$.
                \stepcounter{figcount}
            \end{center}
        \end{minipage}

        son conmutativos. Notemos que estos también son conmutativos cambiando a $k\circ h $ por $\bbm{1}_G$ y lo mismo cambiando a $h\circ k$ por $\bbm{1}_{G'}$. Como estos homomorfismos son únicos (tanto $h$ como $k$), debe suceder que:
        \begin{equation*}
            k\circ h=\bbm{1}_G\quad\textup{y}\quad h\circ k=\bbm{1}_{G'}
        \end{equation*}
        por ende, $h$ es isomorfismo.
    \end{proof}

    La importancia de este teorema, es que si consideremos un grupo abeliano $A$ como un \textit{producto} de grupos abelianos $G_i$, entonces el teorema anterior asegura que $G$ (el producto débil de los $G_i$) es el \textit{más libre} de entre todos los candidatos en el sentido de que existe un homomorfismo de $G$ en $A$ que conmuta con $\varphi_i$ y $\psi_i$, para todo $i\in I$.

    En este sentido, usamos la palabra \textit{más libre} como \textit{la menor cantidad de relaciones impuestas} (realmente podemos hablar en un sentido más particular, aterrizando la noción de objetos libres en una categoría, pero resulta complicado establecerla sin amplio conocimiento previo de Teoría de Categorías), y la filosofía general es es que si ciertas relaciones se cumplen para el grupo $G$, entonces éstas también deberan cumplirse en cualquier imagen homomorfa de $G$.

    Este mismo tipo de filosofía se matiene para otras estructuras algebraicas, como lo son los anillos, módulos, etc\dots

    Como el producto débil $G$ de subgrupos esta totalmente caracterizado por los monomomorfismos $\varphi_i$ de $G_i$ en $G$, podemos dejar de pensar que el producto débil es un subgrupo del grupo
    \begin{equation*}
        \prod_{ i\in I}G_i
    \end{equation*}
    y más aún, podemos pensar a los grupos $G_i$ como conjuntos en $G$ bajo la imagen de $\varphi_i$.

    \section{Grupos Abelianos Libres}

    Recordemos que si $G$ es un grupo, decimos que un conjunto $S\subseteq G$ \textbf{genera a $G$}, si
    \begin{equation*}
        G=\left\{s_1^{\epsilon_1}\cdot\dots\cdot s_m^{\epsilon_m}\Big|s_i\in S,\epsilon_i\in\left\{-1,1\right\}, \forall i\in\natint{1,m}; m\in\mathbb{N} \right\}
    \end{equation*}
    y en tal caso, se denota $G=\langle S\rangle$. 

    \begin{exa}
        Si $G$ es un grupo cíclico de orden $n\in\mathbb{N}$, entonces existe $x\in G$ tal que $G=\langle x\rangle$, así que
        \begin{equation*}
            G=\left\{x,x^2,...,x^n=e\right\}
        \end{equation*}
    \end{exa}

    Si un conjunto $S$ genera a un grupo, entonces ciertos productos de elementos de $S$ pueden coincidir con la identidad de $G$, por ejemplo:
    \renewcommand{\theenumi}{\alph{enumi}}
    \begin{enumerate}
        \item Si $x\in S$, entonces $xx^{-1}=e_G$.
        \item Si $G$ es un grupo cíclico de orden $n$ generado por $\left\{x\right\}$, entonces $x^n=1$.
    \end{enumerate}

    Cualquier producto de elementos de $S$ que sea igual a la identidad del grupo es llamado una \textbf{relación}.

    Distinguiremos dos tipos de relaciones: \textbf{relaciones triviales} (como las del inciso (a)) y \textbf{relaciones no triviales}.

    Estas nociones dan lugar a la siguiente definición:

    \begin{mydef}
        Sea $S\subseteq G$ un conjunto de generadores del grupo $G$. Decimos que $G$ es \textbf{libremente generado} por $S$, o \textbf{grupo libre en $S$}, si no hay relaciones no triviales entre los elementos de $S$.
    \end{mydef}

    \begin{exa}
        Si $G$ es un grupo cíclico infinito, entonces $G$ es un grupo libre en el conjunto $S=\left\{x \right\}$.
    \end{exa}

    Estas nociones dan lugar a la idea de que podemos caracterizar totalmente a un grupo $G$ enlistando sus elementos de un conjunto generador $S$, y las relaciones no triviales que se cumplen entre ellos.
    
    El problema de caracterizar estas ideas de esta forma, es que carecen de rigor matemático, pues ¿qué significa ser una \textit{relación no trivial}?

    Con la formación de la teoría de categorías, se fueron formalizando estas ideas y actualmente no se describe a un grupo libre de la forma en que se hace en la definición anterior. Para describirlo de una forma rigurosa, se hace uso de las siguientes dos observaciones:

    \renewcommand{\theenumi}{\arabic{enumi}}
    \begin{enumerate}
        \item Sea $S$ un conjunto de generadores de $S$, y sea $\cf{f}{G}{G'}$ un epimomorfismo, esto es que $G'$ es la imagen homomorfa de $G$. Entonces, el conjunto $f(S)$ es un conjunto de generadores de $G'$. Más aún, \textit{cualquier relación entre los elementos de $S$ se mantiene entre los elementos de $f(S)$ correspondientes}. Entonces, el grupo $G'$ satisface las mismas relaciones (o inclusive más) que el grupo $G$.
        \item Sea $S$ un conjunto de generadores de $G$, y sea $\cf{f}{G}{G'}$ un homomorfismo arbitrario. Entonces, $f$ está completamente determinado por su reestricción a $S$. Pero, no cualquier función $\cf{g}{S}{G'}$ puede ser extendida a un homomorfismo. La razón intuitiva es que dada la función $g$, puede que las relaciones que se cumplían en $S$ no se siguan cumpliendo en $g(S)$.
    \end{enumerate}

    \begin{exa}
        Considere $\mathbb{Z}$ y sea $\cf{g}{\left\{1\right\}}{\mathbb{Z}}$ dada por:
        \begin{equation*}
            g(1)=2
        \end{equation*}
        entonces, $g$  no puede ser extendida a un homomorfismo de $\mathbb{Z}$ en sí mismo.
    \end{exa}

    Con estas condiciones en mente, daremos una definición formal de lo que significa que un grupo \textit{abeliano} (se hará primero este caso, pues es el más sencillo de entender) sea libre.


    \begin{mydef}
        Sea $S$ un conjunto arbitrario. Un \textbf{grupo abeliano libre} en el conjunto $S$ es un grupo abeliano $F$ junto con una función $\cf{f}{S}{F}$ tal que se cumple la siguiente condición:
        \begin{itemize}
            \item Para cualquier grupo abeliano y cualquier función $\cf{\psi}{S}{A}$, existe un único homomorfismo $\cf{f}{F}{A}$ tal que el diagrama:
            
            \begin{minipage}{\textwidth}
                \begin{center}
                    \includegraphics[scale=1.5]{images/fig_2.pdf}\\
                    Figura \thefigcount. Diagrama conmutativo de $G$ y $A$.
                    \stepcounter{figcount}
                \end{center}
            \end{minipage}
            
            es conmutativo.
        \end{itemize}
    \end{mydef}

    Veamos que esta definición caracteriza a los grupos abelianos libres en un conjunto dado $S$.

    \begin{propo}
        Sean $F$ y $F'$ grupos abelianos libres en un conjunto $S$ con respecto a las funciones $\cf{\varphi}{S}{F}$ y $\cf{\varphi'}{S}{F'}$, respectivamente. Entonces, existe un único isomorfismo $\cf{h}{F'}
        {F}$ tal que el diagrama
        
        \begin{minipage}{\textwidth}
            \begin{center}
                \includegraphics[scale=1.5]{images/fig_6.pdf}\\
                Figura \thefigcount. Isomorfismo entre $F$ y $F'$.
                \stepcounter{figcount}
            \end{center}
        \end{minipage}

        es conmutativo
    \end{propo}

    \begin{proof}
        El procedimiento es análogo al de la proposición \ref{caractSumDir_2}.
    \end{proof}

    De momento, no hemos dicho que dado un conjunto $S$, existe un grupo abeliano libre $F$ en $S$. Nuestra tarea será de probar la existencia de este grupo abeliano libre en $S$.

    %TODO

    \newpage

    \section{Ejercicios}

    \begin{excer}
        Pruebe directo de la definición que $\varphi(S)$ genera a $F$.

        \textit{Sugerencia.} Suponga que no, considere el subgrupo $F'$ generado por $\varphi(S)$.
    \end{excer}

    \begin{proof}
        Sea
        \begin{equation*}
            F'=\langle\varphi(S)\rangle
        \end{equation*}
        se tiene que $F'\subseteq F$ y $F'$ es un grupo abeliano. Considere la función
        \begin{equation*}
            \cf{\psi}{S}{F'},\psi(s)=\varphi(s),\quad\forall s\in S
        \end{equation*}
        claramente esta función está bien definida. Por la definición de grupo abeliano libre existe un único homomorfismo $\cf{f}{F}{F'}$ tal que
        \begin{equation*}
            f\circ\varphi=\psi
        \end{equation*}
        esto es que:
        \begin{equation*}
            f\circ\varphi=\bbm{1}_{F'}\circ\varphi
        \end{equation*}
        por tanto, de la unicidad de $f$ se sigue que $f=\bbm{1}_{F'}$, es decir que $F=F'$.

        Por tanto, $\varphi(S)$ genera a $F$.
    \end{proof}

\end{document}