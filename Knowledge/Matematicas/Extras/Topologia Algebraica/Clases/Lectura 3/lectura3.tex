\documentclass{article}
\usepackage[spanish]{babel}
\usepackage[utf8]{inputenc}
\usepackage{amsmath}
\usepackage{amssymb}
\usepackage{amsthm}
\usepackage{graphics}
\usepackage{subfigure}
\usepackage{lipsum}
\usepackage{array}
\usepackage{multicol}
\usepackage{enumerate}
\usepackage[framemethod=TikZ]{mdframed}
\usepackage[a4paper, margin = 1.5cm]{geometry}
\usepackage{fullpage}
\usepackage[mathscr]{euscript}
\usepackage{bbm}

%En esta parte se hacen redefiniciones de algunos comandos para que resulte agradable el verlos%

\renewcommand{\theenumii}{\roman{enumii}}

\def\proof{\paragraph{Demostración:\\}}
\def\endproof{\hfill$\blacksquare$\\}

\def\sol{\paragraph{Solución:\\}}
\def\endsol{\hfill$\square$\\}

%En esta parte se definen los comandos a usar dentro del documento para enlistar%

\newtheoremstyle{largebreak}
  {}% use the default space above
  {}% use the default space below
  {\normalfont}% body font
  {}% indent (0pt)
  {\bfseries}% header font
  {}% punctuation
  {\newline}% break after header
  {}% header spec

\theoremstyle{largebreak}

\newmdtheoremenv[
    leftmargin=0em,
    rightmargin=0em,
    innertopmargin=-2pt,
    innerbottommargin=8pt,
    hidealllines = true,
    roundcorner = 5pt,
    backgroundcolor = gray!60!red!30
]{exa}{Ejemplo}[section]

\newmdtheoremenv[
    leftmargin=0em,
    rightmargin=0em,
    innertopmargin=-2pt,
    innerbottommargin=8pt,
    hidealllines = true,
    roundcorner = 5pt,
    backgroundcolor = gray!50!blue!30
]{obs}{Observación}[section]

\newmdtheoremenv[
    leftmargin=0em,
    rightmargin=0em,
    innertopmargin=-2pt,
    innerbottommargin=8pt,
    rightline = false,
    leftline = false
]{theor}{Teorema}[section]

\newmdtheoremenv[
    leftmargin=0em,
    rightmargin=0em,
    innertopmargin=-2pt,
    innerbottommargin=8pt,
    rightline = false,
    leftline = false
]{propo}{Proposición}[section]

\newmdtheoremenv[
    leftmargin=0em,
    rightmargin=0em,
    innertopmargin=-2pt,
    innerbottommargin=8pt,
    rightline = false,
    leftline = false
]{cor}{Corolario}[section]

\newmdtheoremenv[
    leftmargin=0em,
    rightmargin=0em,
    innertopmargin=-2pt,
    innerbottommargin=8pt,
    rightline = false,
    leftline = false
]{lema}{Lema}[section]

\newmdtheoremenv[
    leftmargin=0em,
    rightmargin=0em,
    innertopmargin=-2pt,
    innerbottommargin=8pt,
    roundcorner=5pt,
    backgroundcolor = gray!30,
    hidealllines = true
]{mydef}{Definición}[section]

\newmdtheoremenv[
    leftmargin=0em,
    rightmargin=0em,
    innertopmargin=-2pt,
    innerbottommargin=8pt,
    roundcorner=5pt
]{excer}{Ejercicio}[section]

%En esta parte se colocan comandos que definen la forma en la que se van a escribir ciertas funciones%
\newcommand{\norm}[1]{\ensuremath{\|#1\|}}
\newcommand\subtitle[1]{\textit{\large #1}\\}
\newcommand\abs[1]{\ensuremath{\left|#1\right|}}
\newcommand\divides{\ensuremath{\bigm|}}
\newcommand\cf[3]{\ensuremath{#1:#2\rightarrow#3}}
\newcommand\natint[1]{\ensuremath{\left[\!\left[ #1\right]\!\right]}}
\newcommand{\afa}{\:
    \begin{tikzpicture}
        \draw [line width = 0.17 mm, black] (0,0) -- (-0.115,0.29);
        \draw [line width = 0.17 mm, black] (0,0) -- (0.115,0.29);
        \draw [line width = 0.17 mm, black] (-0.12,0) arc (190:-10:0.12cm);
    \end{tikzpicture}
    \:
}
\newcommand{\bbm}[1]{\ensuremath{\mathbbm{#1}}}
%Este símvolo es para casi todo salvo una cantidad finita

%recuerda usar \clearpage para hacer un salto de página

\begin{document}

    \title{Taller Topología Algebraica, Lectura 3: }
    \author{Cristo Alvarado}
    \setcounter{section}{3}
    \maketitle

    \textit{Nota:} Debido a que la clase pasada se recortó el tiempo, algunas cosas de la lectura pasada están repetidas en esta lectura y se continua con temas adicionales.\\

    \subtitle{El Grupo Fundamental}

    \begin{mydef}
        Para cualquier camino $\cf{f}{I}{X}$, $\overline{f}$ denota al camino definido por:
        \begin{equation*}
            \overline{f}(t)=f(1-t),\quad\forall t\in I
        \end{equation*}
        El camino $\overline{f}$ se obtiene recorriendo el camino $f$ en sentido contrario.
    \end{mydef}

    \begin{lema}
        Sea $f$ un camino y denotemos por $\mathscr{F}=[f]$ y $\overline{\mathscr{F}}=[\overline{f}]$, entonces:
        \begin{equation*}
            \mathscr{F}\cdot\overline{\mathscr{F}}=\mathscr{I}_x\quad\textup{y}\quad\overline{\mathscr{F}}\cdot\mathscr{F}=\mathscr{I}_y
        \end{equation*}
        donde $x\in X$ y $y\in X$ son los puntos inicial y terminal de $f$, respectivamnete.
    \end{lema}

    \begin{proof}
        Sólo se probará la primera igualdad, para ello es suficiente con probar que $f\cdot\overline{f}\sim i_x$. Definimos la función $\cf{F}{I\times I}{X}$ por:
        \begin{equation*}
            F(t,s)=\left\{
                \begin{array}{lcr}
                    f(2t) & \textup{ si } & 0\leq t\leq\frac{s}{2}\\
                    f(s) & \textup{ si } & \frac{s}{2}\leq t\leq1-\frac{s}{2}\\
                    f(2-2t) & \textup{ si } & 1-\frac{s}{2}\leq t\leq1\\
                \end{array}
            \right.
        \end{equation*}
        para todo $s,t\in I$. Entonces,
        \begin{equation*}
            \begin{split}
                F(t,0)&=\left\{
                    \begin{array}{lcr}
                        f(2t) & \textup{ si } & 0\leq t\leq0\\
                        f(s) & \textup{ si } & 0\leq t\leq1-0\\
                        f(2-2t) & \textup{ si } & 1-0\leq t\leq1\\
                    \end{array}
                \right.\\
                &=\left\{
                    \begin{array}{lcr}
                        f(0) & \textup{ si } & t=0\\
                        f(0) & \textup{ si } & 0\leq t\leq1\\
                        f(2-2t) & \textup{ si } & t=1\\
                    \end{array}
                \right.\\
                &=\left\{
                    \begin{array}{lcr}
                        x & \textup{ si } & 0\leq t\leq1\\
                        f(0) & \textup{ si } & t=1\\
                    \end{array}
                \right.\\
                &=x\\
            \end{split}
        \end{equation*}
        para todo $t\in I$. Además,
        \begin{equation*}
            \begin{split}
                F(t,1)&=\left\{
                    \begin{array}{lcr}
                        f(2t) & \textup{ si } & 0\leq t\leq\frac{1}{2}\\
                        f(1) & \textup{ si } & \frac{1}{2}\leq t\leq1-\frac{1}{2}\\
                        f(2-2t) & \textup{ si } & 1-\frac{1}{2}\leq t\leq1\\
                    \end{array}
                \right.\\
                &=\left\{
                    \begin{array}{lcr}
                        f(2t) & \textup{ si } & 0\leq t\leq\frac{1}{2}\\
                        f(1-(2t-1)) & \textup{ si } & \frac{1}{2}\leq t\leq1\\
                    \end{array}
                \right.\\
                &=\left\{
                    \begin{array}{lcr}
                        f(2t) & \textup{ si } & 0\leq t\leq\frac{1}{2}\\
                        \overline{f}(2t-1) & \textup{ si } & \frac{1}{2}\leq t\leq1\\
                    \end{array}
                \right.\\
                &=f\cdot\overline{f}(t)\\
            \end{split}
        \end{equation*}
        para todo $t\in I$. La función $F$ es continua... %TODO

        Por tanto, $\mathscr{F}\cdot\overline{\mathscr{F}}$.
    \end{proof}

    En visata de estas propiedades de la clase $\overline{\mathscr{F}}$, de ahora en adelante la denotaremos por $\mathscr{F}^{-1}$.

    Podemos resumir todos los lemas antes probados diciendo que el conjunto de todas las clases de caminos en un espacio $X$ satisfacen los axiomas de grupo, excepto que el producto de dos caminos no siempre está definido. Solventamos este problema con la siguiente definición:

    \begin{mydef}
        Un camino o una clase de camino es llamada \textbf{cerrada} o un \textbf{bucle}, si el punto inicial y terminal son el mimso. El bucle se dice que tiene \textbf{base} en el punto inicial o terminal.
    \end{mydef}

    \begin{theor}
        Sea $X$ un espacio topológico y $x\in X$ un punto fijo. Entonces, el conjunto de todas las clases de caminos cerradas que tienen como punto base a $x$ dotado por la operación $\cdot$, denotado por $\pi(X,x)$ es un grupo llamado \textbf{grupo fundamental} o \textbf{grupo de Poincaré} de $X$ con punto base $x$.
    \end{theor}

    \begin{proof}
        Es un resumen de todos los lemas anteriores.
    \end{proof}

    \begin{obs}
        Para un espacio topológico dado $X$ y $x\in X$, dotamos el grupo fundamental $\pi(X,x)$ de una operación binaria que lo hace de grupo, de ahora en adelante tal operación se denotará al producto de dos clases $[f]$ y $[g]$ por $[f]\cdot[g]$ o por yuxtaposición como $[f][g]=[f\cdot g]$ (no confundir la operación dentro de los paréntesis cuadrados con la composición usual de funciones).

        Si $[f]\in\pi(X,x)$, se denotará a su inverso por $[f]^{-1}$ y, al elemento identidad por $\mathscr{I}$
    \end{obs}

    \begin{propo}
        Sea $X$ un espacio y $x,y\in X$ dos puntos distintos. Si $\cf{\gamma}{I}{X}$ es un camino con punto inicial $x$ y terminal $y$, entonces $\pi(X,x)\cong\pi(X,y)$ (es decir, son grupos isomorfos).
    \end{propo}

    \begin{proof}
        En efecto, defina la función $\cf{u}{\pi(X,x)}{\pi(X,y)}$ dada por:
        \begin{equation*}
            u([f])=[\gamma]^{-1}[f][\gamma]
        \end{equation*}
        Por cursos anteriores de teoría de Grupos, se ve de forma inmediata que esta función es un isomorfismo entre los grupos $\pi(X,x)$ y $\pi(X,y)$.
    \end{proof}

    \begin{cor}
        Sea $X$ un espacio topológico arco-conexo, entonces los grupos $\pi(X,x)$ y $\pi(X,y)$ son isomorfos para todo $x,y\in X$.
    \end{cor}

    La importancia del teorema anterior radica en que el grupo $\pi(X,x)$ tiene propiedades como grupo (es decir, es abeliano, finito, nilpotente, libre, etc...) no debido al punto elegido $x\in X$, sino al espacio mismo $X$, suponiendo que $X$ es arco-conexo.

    En general, no hay un mapeo canónico o isomorfismo natural entre $\pi(X,x)$ y $\pi(X,y)$, ya que a cada elección de camino entre $x$ y $y$ le corresponderá un isomorfismo.\\

    \subtitle{Efecto de una función continua en el grupo fundamental}

    \begin{obs}
        Para esta sección resultará de utilidad definir el siguiente conjunto, para todo espacio topológico $X$ se define
        \begin{equation*}
            \wp_X = \left\{[f]\Big|\cf{f}{I}{X}\textup{ es una función continua} \right\}
        \end{equation*}
        es decir, estamos tomando todas las clases de caminos de un espacio topológico $X$ (note que no tiene nada que ver con el grupo fundamental, más que con el hecho de que usa las clases de caminos en su definición).
    \end{obs}

    Considere dos espacios topológicos $X$ y $Y$ y sea $\cf{\varphi}{X}{Y}$ una función continua. Si $\cf{f_0,f_1}{I}{X}$ son caminos en $X$, ¿también lo son $\varphi\circ f_0$ y $\varphi\circ f_1$?

    \begin{propo}
        Sean $X$ y $Y$ espacios topológicos, $\cf{f_0,f_1}{I}{X}$ caminos equivalentes. Entonces, $\varphi\circ f_0\sim \varphi\circ f_1$.
    \end{propo}

    \begin{proof}
        Como $f_1\sim f_0$, existe pues una función continua $\cf{F}{I\times I}{X}$ tal que
        \begin{equation*}
            F(x,0)=f_0(x),\quad F(x,1)=f_1(x)
        \end{equation*}
        para todo $x\in I$ y,
        \begin{equation*}
            F(0,t)=f_0(0)=f_1(0),\quad F(1,t)=f_0(1)=f_1(1)
        \end{equation*}
        Considere la función $\cf{G}{I\times I}{Y}$ dada por:
        \begin{equation*}
            G(x,t)=\varphi\circ F(x,t)
        \end{equation*}
        Es claro que esta funciónes continua por ser composición de funciones continuas, además se cumple que
        \begin{equation*}
            \begin{split}
                G(x,0)&=\varphi\circ F(x,0)\\
                &=\varphi (F(x,0))\\
                &=\varphi (f_0(x))\\
                &=\varphi\circ f_0(x)\\
            \end{split}
        \end{equation*}
        para todo $x\in I$. De forma análoga
        \begin{equation*}
            G(x,1)=\varphi\circ f_1(x)
        \end{equation*}
        La otra condición se verifica de forma inmediata, con lo que se concluye que $\varphi\circ f_0\sim \varphi\circ f_1$.
    \end{proof}

    Con la proposición anterior, podemos definir sin problemas una función que mapee clases de caminos en $X$ a clases de caminos en $Y$, a partir de la función continua $\varphi$. Esto se hará con el objetivo de ver qué sucede con el grupo fundamental bajo esta función continua $\varphi_*$.

    \begin{mydef}
        Sean $X$ y $Y$ espacios topológicos y $\cf{\varphi}{X}{Y}$ una función continua. Sea $\cf{f}{I}{X}$ un camino que une a los puntos $x,y\in X$, se define la función $\cf{\varphi_*}{\wp_X}{\wp_Y}$ por
        \begin{equation*}
            \varphi_*([f])=[\varphi\circ f]
        \end{equation*}
        por la proposición anterior, esta función está bien definida.
    \end{mydef}

    Ahora, analizaremos las propiedades de la función $\varphi_*$.
    
    \renewcommand{\theenumi}{\roman{enumi}}

    \begin{propo}
        Sean $X$ y $Y$ espacios topológicos y $\cf{\varphi}{X}{Y}$ una función continua.
        \begin{enumerate}
            \item Si $\cf{f_0,f_1}{I}{X}$ son caminos en $X$ tales que $f_0\cdot f_1$ está definido (por ende, $[f_0]\cdot [f_1]$ lo está), entonces $\varphi_*([f_0]\cdot[f_1])=\varphi_*([f_0])\cdot\varphi_*([f_1])$.
            \item Para cualquier punto $x\in X$, $\varphi_*(\mathscr{I}_x)=\mathscr{I}_{\varphi(x)}$.
            \item Si $\cf{f}{I}{X}$ es un camino, entonces $\varphi_*([f]^{-1})=(\varphi_*([f]))^{-1}$.
        \end{enumerate}
    \end{propo}

    \begin{proof}
        De (i): Veamos primero que el producto $\varphi_*([f_0])\cdot\varphi_*([f_1])$ está bien definido. En efecto, dado a que el producto $[f_0]\cdot [f_1]$ lo está, entonces
        \begin{equation*}
            f_0(1)=f_1(0)
        \end{equation*}
        luego,
        \begin{equation*}
            \varphi\circ f_0(1)=\varphi\circ f_1(0)
        \end{equation*}
        donde $\varphi_*([f_0])=[\varphi\circ f_0]$ y $\varphi_*([f_1])=[\varphi\circ f_1]$, por tanto el producto de ambas clases está definido. Probemos ahora la igualdad. Se tiene que
        \begin{equation*}
            \begin{split}
                \varphi_*([f_0]\cdot[f_1])&=\varphi_*([f_0\cdot f_1])\\
                &=[\varphi\circ (f_0\cdot f_1)]\\
            \end{split}
        \end{equation*}
        siendo
        \begin{equation*}
            f_0\cdot f_1(t)=\left\{
                \begin{array}{lcr}
                    f_0(2t) & \textup{ si } 0\leq t\leq \frac{1}{2}\\
                    f_1(2t-1) & \textup{ si } \frac{1}{2}\leq t\leq 1\\
                \end{array}
            \right.,\quad\forall t\in I
        \end{equation*}
        por ende,
        \begin{equation*}
            \begin{split}
                \varphi\circ (f_0\cdot f_1)(t)&=\left\{
                    \begin{array}{lcr}
                        \varphi(f_0(2t)) & \textup{ si } 0\leq t\leq \frac{1}{2}\\
                        \varphi(f_1(2t-1)) & \textup{ si } \frac{1}{2}\leq t\leq 1\\
                    \end{array}
                \right.\\
                &=\left\{
                    \begin{array}{lcr}
                        \varphi\circ f_0(2t) & \textup{ si } 0\leq t\leq \frac{1}{2}\\
                        \varphi\circ f_1(2t-1) & \textup{ si } \frac{1}{2}\leq t\leq 1\\
                    \end{array}
                \right.\\
                &=(\varphi\circ f_0)\cdot (\varphi\circ f_1)(t),\quad\forall t\in I\\
            \end{split}
        \end{equation*}
        luego entonces
        \begin{equation*}
            \begin{split}
                [\varphi\circ (f_0\cdot f_1)]&=[(\varphi\circ f_0)\cdot (\varphi\circ f_1)]\\
                &=[\varphi\circ f_0]\cdot [\varphi\circ f_1]\\
                &=\varphi_*([f_0])\cdot \varphi_*([f_1])\\
            \end{split}
        \end{equation*}
        lo que prueba el resultado.

        De (ii) y (iii): Ejercicio.
    \end{proof}

    Por estas razones, llamaremos a $\varphi_*$ un \textit{homomorfismo} u \textit{homomorfismo inducido por $\varphi$}.

    \begin{propo}
        En el contexto de la proposición anterior, si $Z$ es un espacio topológico y $\cf{\psi}{Y}{Z}$ es una función continua, entonces
        \begin{equation*}
            (\psi\circ\varphi)_*=\psi_*\circ\varphi_*
        \end{equation*}
    \end{propo}

    \begin{proof}
        Es claro que la composición de ambas funciones está bien definida. Probaremos ahora la igualdad, sea $\cf{f}{I}{X}$ un camino, entonces:
        \begin{equation*}
            \begin{split}
                (\psi\circ\varphi)_*([f])&=[\psi\circ\varphi\circ f]\\
                &=[\psi\circ(\varphi\circ f)]\\
                &=\psi_*([\varphi\circ f])\\
                &=\psi_*(\varphi_*([f]))\\
                &=\psi_*\circ \varphi_*([f])\\
            \end{split}
        \end{equation*}
        lo que prueba la igualdad.
    \end{proof}

    \begin{obs}
        En otras palabras, lo que nos dice la proposición anterior es que el mapeo $i_*$ reestringido a $\pi(X,x)$ coincide con la identidad de $\bbm{1}_{\pi(X,x)}$, y que
        \begin{equation*}
            i_*=\bbm{1}_{\wp_X}
        \end{equation*}
    \end{obs}
    
    \begin{cor}
        Sean $X$ y $Y$ espacios topológicos y $\cf{\varphi}{X}{Y}$ una función continua y $x\in X$. La función $\varphi_*$ restringida a $\pi(X,x)\subseteq\wp_X$ es un homomorfismo entre $\pi(X,x)$ y $\pi(Y,\varphi(x))$. Más aún, si $\varphi$ es homeomorfismo, entonces $\varphi_*$ reestringida a $\pi(X,x)$ es isomorfismo.
    \end{cor}

    \begin{proof}
        El hecho de que sea homomorfismo es inmediato de la proposición anterior y de que si $f$ es un bucle con base en $x$, entonces
        \begin{equation*}
            \varphi_*([f])=[\varphi\circ f]
        \end{equation*}
        es la clase del camino $\varphi\circ f$, el cual es un bucle con base en $\varphi(x)$.

        Veamos ue si $\varphi$ es homeomorfismo entonces $\varphi_*$ es isomorfismo. En efecto, como es homeomorfismo existe una función $\cf{\varphi^{-1}}{Y}{X}$ continua que es la inversa de $\varphi$.

        Se sigue de la proposición anterior que
        \begin{equation*}
            (\bbm{1}_X)_*=(\varphi\circ\varphi^{-1})_*=\varphi_*\circ\varphi_*^{-1}
        \end{equation*}
        donde $\bbm{1}_X$ es la identidad de $X$, luego
        \begin{equation*}
            \varphi_*\circ\varphi_*^{-1}=\bbm{1}_{\wp_X}
        \end{equation*}
        de forma análoga se sigue que
        \begin{equation*}
            \varphi_*^{-1}\circ\varphi_*=\bbm{1}_{\wp_Y}
        \end{equation*}
        por tanto, la función $\varphi_*$ es invertible, luego $\varphi_*$ reestringida a $\pi(X,x)$ es invertible, con inversa la reestricción de $\varphi_*^{-1}$ a $\pi(Y,\varphi(x))$. Así, $\varphi_*$ es isomorfismo.
    \end{proof}

\end{document}