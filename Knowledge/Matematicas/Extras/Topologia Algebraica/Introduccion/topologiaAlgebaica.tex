\documentclass[12pt]{report}
\usepackage[spanish]{babel}
\usepackage[utf8]{inputenc}
\usepackage{amsmath}
\usepackage{amssymb}
\usepackage{amsthm}
\usepackage[mathscr]{euscript}
\usepackage{graphics}
\usepackage{wrapfig}
\usepackage{subfigure}
\usepackage{lipsum}
\usepackage{array}
\usepackage{multicol}
\usepackage{enumerate}
\usepackage[framemethod=TikZ]{mdframed}
\usepackage[a4paper, margin = 1.5cm]{geometry}
\usepackage{bbm}

%En esta parte se hacen redefiniciones de algunos comandos para que resulte agradable el verlos%

\renewcommand{\theenumii}{\roman{enumii}}

\def\proof{\paragraph{Demostración:\\}}
\def\endproof{\hfill$\blacksquare$}

\def\sol{\paragraph{Solución:\\}}
\def\endsol{\hfill$\square$}

%En esta parte se definen los comandos a usar dentro del documento para enlistar%

\newtheoremstyle{largebreak}
  {}% use the default space above
  {}% use the default space below
  {\normalfont}% body font
  {}% indent (0pt)
  {\bfseries}% header font
  {}% punctuation
  {\newline}% break after header
  {}% header spec

\theoremstyle{largebreak}

\newmdtheoremenv[
    leftmargin=0em,
    rightmargin=0em,
    innertopmargin=-2pt,
    innerbottommargin=8pt,
    hidealllines = true,
    roundcorner = 5pt,
    backgroundcolor = gray!60!red!30
]{exa}{Ejemplo}[section]

\newmdtheoremenv[
    leftmargin=0em,
    rightmargin=0em,
    innertopmargin=-2pt,
    innerbottommargin=8pt,
    hidealllines = true,
    roundcorner = 5pt,
    backgroundcolor = gray!50!blue!30
]{obs}{Observación}[section]

\newmdtheoremenv[
    leftmargin=0em,
    rightmargin=0em,
    innertopmargin=-2pt,
    innerbottommargin=8pt,
    rightline = false,
    leftline = false
]{theor}{Teorema}[section]

\newmdtheoremenv[
    leftmargin=0em,
    rightmargin=0em,
    innertopmargin=-2pt,
    innerbottommargin=8pt,
    rightline = false,
    leftline = false
]{propo}{Proposición}[section]

\newmdtheoremenv[
    leftmargin=0em,
    rightmargin=0em,
    innertopmargin=-2pt,
    innerbottommargin=8pt,
    rightline = false,
    leftline = false
]{cor}{Corolario}[section]

\newmdtheoremenv[
    leftmargin=0em,
    rightmargin=0em,
    innertopmargin=-2pt,
    innerbottommargin=8pt,
    rightline = false,
    leftline = false
]{lema}{Lema}[section]

\newmdtheoremenv[
    leftmargin=0em,
    rightmargin=0em,
    innertopmargin=-2pt,
    innerbottommargin=8pt,
    roundcorner=5pt,
    backgroundcolor = gray!30,
    hidealllines = true
]{mydef}{Definición}[section]

\newmdtheoremenv[
    leftmargin=0em,
    rightmargin=0em,
    innertopmargin=-2pt,
    innerbottommargin=8pt,
    roundcorner=5pt
]{excer}{Ejercicio}[section]

%En esta parte se colocan comandos que definen la forma en la que se van a escribir ciertas funciones%

\newcommand\abs[1]{\ensuremath{\left|#1\right|}}
\newcommand\divides{\ensuremath{\bigm|}}
\newcommand\cf[3]{\ensuremath{#1:#2\rightarrow#3}}
\newcommand\natint[1]{\ensuremath{\left[\!\left[ #1\right]\!\right]}}
\newcommand{\afa}{\:
    \begin{tikzpicture}
        \draw [line width = 0.17 mm, black] (0,0) -- (-0.115,0.29);
        \draw [line width = 0.17 mm, black] (0,0) -- (0.115,0.29);
        \draw [line width = 0.17 mm, black] (-0.12,0) arc (190:-10:0.12cm);
    \end{tikzpicture}
    \:
}
\newcommand{\bbm}[1]{\ensuremath{\mathbbm{#1}}}
%Este símvolo es para casi todo salvo una cantidad finita

%recuerda usar \clearpage para hacer un salto de página

\begin{document}
    \setlength{\parskip}{5pt} % Añade 5 puntos de espacio entre párrafos
    \setlength{\parindent}{12pt} % Pone la sangría como me gusta
    \title{Un Curso Introductorio en Topología Algebraica}
    \author{Cristo Daniel Alvarado}
    \maketitle

    \tableofcontents %Con este comando se genera el índice general del libro%

    %\setcounter{chapter}{3} %En esta parte lo que se hace es cambiar la enumeración del capítulo%
    
    \chapter{Introducción}
    
    A lo largo del curso (y estudiando temas de topología) llega a resultar de útilidad analizar el siguiente problema:

    \begin{center}
        \textbf{¿Cuándo dos espacios topológicos $X$ e $Y$ son homeomorfos?}
    \end{center}

    Desafortunadamente, esta pregunta resulta en extremo compleja de analizar. Analicemos por ejemplo los siguientes subespacios de $\mathbb{R}^2$:

    %poner el ejemplo que dejó quintín en sus notas

    \chapter{El Grupo Fundamental}

    El concepto de grupo fundamental 

    \section{Conceptos Fundamentales}

    De ahora en adelante, $I$ denotará al intervalo $[0,1]$.

    \begin{mydef}
        Un \textbf{camino} o \textbf{arco} en un espacio topológico $X$, es una función continua $\cf{f}{[a,b]}{X}$ de un intervalo cerrado en $X$.

        Las imágenes $f(a)$ y $f(b)$ son llamadas \textbf{puntos finales del camino o arco}. $f(a)$ es llamado \textbf{punto inicial} y $f(b)$ \textbf{punto final}.
    \end{mydef}

    \begin{obs}
        Por comodidad, dado a que existe un homeomorfismo lineal entre $[0,1]$ y $[a,b]$ (vistos como subespacios de $\mathbb{R}$ dotado de la topología usual) siendo $a,b\in\mathbb{R}$ con $a<b$ podemos ver a todos los caminos o arcos de un espacio topológico $X$ como funciones continuas de $I$ en $X$. Cuando sea más conveniente de esta manera, se usará esta convención.
    \end{obs}

    \begin{mydef}
        Un espacio topológico $X$ es llamado \textbf{conexo por arcos} o \textbf{arco-conexo} si cualesquiera dos puntos de $X$ pueden ser unidos mediante un arco, es decir tales que los puntos finales del arco coincidan con estos dos puntos.
    \end{mydef}

    \begin{theor}
        Todo espacio topológico arco-conexo es conexo.
    \end{theor}

    \begin{proof}
        Ejercicio.
    \end{proof}

    Como una sugerencia para la demostración del teorema anterior, recuerde el \textit{teorema del cactus} (la unión de una familia de conjuntos conexos tales que la intersección de la familia es no vacía, es un conjunto conexo).

    El recíproco del teorema anterior no es cierto como se ha visto en varios cursos pasados (recuerde Cálculo III).

    \begin{mydef}
        Sea $X$ un espacio topológico. Para cada $x\in X$ se define:
        \begin{equation*}
            \mathcal{A}(x)=\left\{y\in X\Big|\textup{ existe una función continua }\cf{f}{I}{X} \textup{ tal que }f(0)=x \textup{ y }f(1)=y \right\}
        \end{equation*}
        Se construye así la familia $\left\{\mathcal{A}(x) \right\}_{ x\in X}$. Esta familia forma una partición de $X$ y se denomina como \textbf{las componentes arco-conexas de $X$}.
    \end{mydef}

    En el sentido de la definición anterior, estamos obteniendo los subconjuntos de $X$ que son arco conexos más \textit{grandes} que tiene. Si $X$ es arco-conexo, entonces $\mathcal{A}(x)=X$, para todo $x\in X$.

    Las componentes arco-conexas de $X$ no necesariamente son conjuntos cerrados o abiertos.

    \begin{mydef}
        Un espacio topológico $X$ es \textbf{localmente arco-conexo} si cada punto tiene una familia básica de vecindades arco-conexas.
    \end{mydef}

    %TODO Explicar el concepto más profundamente.

    \begin{mydef}
        Sean $\cf{f,g}{[a,b]}{X}$ dos arcos en $X$ tales que $f(a)=g(a)$ y $f(b)=g(b)$ (esto es, que ambos arcos tienen los mismos puntos terminales). Decimos que estos dos arcos son \textbf{equivalentes}, denotándolo por $f\sim g$, si existe una función continua
        \begin{equation*}
            \cf{F}{[a,b]\times I}{X}
        \end{equation*}
        tal que
        \begin{equation*}
            F(t,0)=f(t)\quad\textup{y}\quad F(t,1)=g(t)
        \end{equation*}
        para todo $t\in[a,b]$ y,
        \begin{equation*}
            F(a,s)=f(a)=g(a)\quad\textup{y}\quad F(b,s)=f(b)=g(b)
        \end{equation*}
        para todo $s\in I$.
    \end{mydef}

    \begin{propo}
        La relación de la definición anterior es una relación de equivalencia en el conjunto de todos los arcos con mismos puntos terminales de un espacio topológico $X$.
    \end{propo}

    Notemos que el concepto anterior es casi el mismo que el de homotopía, considerando de forma adicional que en esta definición se dejen fijos los puntos terminales de ambos arcos.

    \begin{mydef}
        Sean $X$ e $Y$ dos espacios topológicos. Se dice que dos funciones $\cf{f,g}{X}{Y}$ son \textbf{homotópicas} si existe una función continua $\cf{F}{X\times I}{Y}$ tal que
        \begin{equation*}
            F(x,0)=f(x)\quad\textup{y}\quad F(x,1)=g(x)
        \end{equation*}
        para todo $x\in X$.
    \end{mydef}

    \begin{obs}
        En los dos casos de las definiciones anteriores, se dota a los espacios de la topología producto.
    \end{obs}

    Intuitivamente lo que uno hace es defomar, sin perder la continuidad, un arco en el otro en el espacio $X$, dejando fijos los puntos terminales fijos en todo momento de la deformación.

    Además de esta relación inducida, queremos definir una operación para dos arcos con ciertas propiedades:

    \begin{mydef}
        Sea $X$ espacio topológico y $\cf{f}{[a,b]}{X}$ y $\cf{g}{[b,c]}{X}$ arcos tales que $f(b)=g(b)$ (siendo $a<b<c$). Entonces el producto $f\cdot g$ se define como:
        \begin{equation*}
            (f\cdot g)(t)=\left\{
                \begin{array}{lcr}
                    f(t) & \textup{ si } & t\in[a,b]\\
                    g(t) & \textup{ si } & t\in[b,c]\\
                \end{array}
            \right.
        \end{equation*}
        para todo $t\in[a,c]$.
    \end{mydef}

    \begin{obs}
        Notemos que esta operación genera otro arco, es decir otra función continua $\cf{f\cdot g}{[a,c]}{X}$.
    \end{obs}

    En lo que sigue, se usará el intervalo $I=[0,1]$.

    \section{El Grupo Fundamental de un Espacio Topológico}

    \begin{mydef}
        Sean $f$ y $g$ caminos en un espacio $X$ tales que el punto final de $f$ es el punto inicial de $g$. Se define el producto $f\cdot g$ por:
        \begin{equation*}
            f\cdot g(t)=\left\{
                \begin{array}{lcr}
                    f(2t) & \textup{ si } & 0\leq t\leq\frac{1}{2}\\
                    g(2t-1) & \textup{ si } & \frac{1}{2}\leq t\leq1\\
                \end{array}
            \right.
        \end{equation*}
        también, decimos que dos caminos $f_0$ y $f_1$ son \textbf{equivalentes} (denotado por $f_0\sim f_1$) si lo son en el sentido de la sección anterior.
    \end{mydef}

    \begin{lema}
        La relación de equivalencia y el producto definido anteriormente son compatibles en el sentido siguiente: si $X$ es un espacio, $f_0\sim f_1$ y $g_0\sim g_1$ son caminos equivalentes, entonces $f_0\cdot g_0\sim f_1\cdot g_1$ (asumiendo que el punto terminal de $f_0$ es el punto inicial de $g_0$).
    \end{lema}

    \begin{proof}
        Como el punto terminal de $f_0$ es el punto terminal de $g_0$, al tenerse las relaciones $f_0\sim f_1$ y $g_0\sim g_1$ se sigue que el producto $f_1\cdot g_1$ está bien definido.

        Como $f_0\sim f_1$ y $g_0\sim g_1$, existen dos funciones continuas $\cf{F,G}{I\times I}{X}$ tales que
        \begin{equation*}
            F(t,0)=f_0(t),\quad F(t,1)=f_1(t),\quad G(t,0)=g_0(t),\quad G(t,1)=g_1(t)
        \end{equation*}
        para todo $t\in I$, y
        \begin{equation*}
            F(0,s)=f_0(0)=f_1(0),\quad F(1,s)=f_0(1)=f_1(1)
        \end{equation*}
        y,
        \begin{equation*}
            G(0,s)=g_0(0)=g_1(0),\quad G(1,s)=g_0(1)=g_1(1)
        \end{equation*}
        para todo $s\in I$. Definamos la función $\cf{H}{I\times I}{X}$ dada por:
        \begin{equation*}
            H(t,s)=\left\{
                \begin{array}{lcr}
                    F(2t,s) & \textup{ si } & 0\leq t\leq\frac{t}{2}\\
                    G(2t-1,s) & \textup{ si } & \frac{t}{2}\leq t\leq1\\
                \end{array}
            \right.
        \end{equation*}
        para todo $t,s\in I$. Claramente $H$ es continua por el lema de pegado y dado que $F$ y $G$ son continuas, además se cumple que:
        \begin{equation*}
            \begin{split}
                H(t,0)&=\left\{
                    \begin{array}{lcr}
                        F(2t,0) & \textup{ si } & 0\leq t\leq\frac{t}{2}\\
                        G(2t-1,0) & \textup{ si } & \frac{t}{2}\leq t\leq1\\
                    \end{array}
                \right.\\
                &=\left\{
                    \begin{array}{lcr}
                        f_0(2t) & \textup{ si } & 0\leq t\leq\frac{t}{2}\\
                        g_0(2t-1) & \textup{ si } & \frac{t}{2}\leq t\leq1\\
                    \end{array}
                \right.\\
                &=f_0\cdot g_0(t)\\
            \end{split}
        \end{equation*}
        y,
        \begin{equation*}
            \begin{split}
                H(t,1)&=\left\{
                    \begin{array}{lcr}
                        F(2t,1) & \textup{ si } & 0\leq t\leq\frac{t}{2}\\
                        G(2t-1,1) & \textup{ si } & \frac{t}{2}\leq t\leq1\\
                    \end{array}
                \right.\\
                &=\left\{
                    \begin{array}{lcr}
                        f_1(2t) & \textup{ si } & 0\leq t\leq\frac{t}{2}\\
                        g_1(2t-1) & \textup{ si } & \frac{t}{2}\leq t\leq1\\
                    \end{array}
                \right.\\
                &=f_1\cdot g_1(t)\\
            \end{split}
        \end{equation*}
        para todo $t\in I$. Ademśa, para todo $s\in I$ se cumple:
        \begin{equation*}
            \begin{split}
                H(0,s)&=F(0,s)\\
                &=f_0(0)\\
                &=f_0(2\cdot0)\\
                &=f_0\cdot g_0(0)\\
            \end{split}
        \end{equation*}
        y,
        \begin{equation*}
            \begin{split}
                H(0,s)&=F(0,s)\\
                &=f_1(0)\\
                &=f_1(2\cdot0)\\
                &=f_1\cdot g_1(0)\\
            \end{split}
        \end{equation*}
        por lo cual
        \begin{equation*}
            H(0,s)=f_0\cdot g_0(0)=f_1\cdot g_1(0)
        \end{equation*}
        de forma análoga se deduce que
        \begin{equation*}
            H(1,s)=f_0\cdot g_0(1)=f_1\cdot g_1(1)
        \end{equation*}
        Se sigue de lo anterior que $f_0\cdot g_0\sim f_1\cdot g_1$.
    \end{proof}

    Como resultado de este lema, se puede definir la multiplicación de clases de equivalencia de caminos (de tal suerte que el punto terminal del primer camino coincida con el inicial del segundo camino).
    
    \begin{mydef}
        Sea $X$ espacio y $f,g$ caminos en $X$ tales que el punto terminal de $f$ es el punto inicial de $g$. $\left[f\right]$ denota a la \textbf{clase de equivalencia con representante $f$} y $\left[g\right]$ a la de $g$.

        Se define el producto de las clases $[f]$ y $[g]$ por:
        \begin{equation*}
            [f]\cdot[g]=[f\cdot g]
        \end{equation*}
    \end{mydef}

    Debido al lema anterior, el producto de clases de equivalencia está bien definido.

    Este es el tipo de multiplicación que nos va a concerner en lo que sigue. En general, la multiplicación de caminos no es asociativa.

    \begin{excer}[\textbf{La multiplicación de caminos no es asociativa.}]
        Considere en $\mathbb{R}^3$ dotado de la topología usual los caminos
        \begin{equation*}
            f(t)=(t,0,0),\quad g(t)=(0,t,0)\quad\textup{y}\quad h(t)=(0,0,t)
        \end{equation*}
        para todo $t\in I$. Compute $f\cdot (g\cdot h)$ y $(f\cdot g)\cdot h$.
    \end{excer}

    Sin embargo, se tiene el siguiente resultado:

    \begin{lema}
        La multiplicación de clases de equivalencia de caminos es asociativa.
    \end{lema}

    \begin{proof}
        Sean $f,g,h$ caminos en $X$ tales que el punto terminal de $f$ es el punto inicial de $g$ y el punto terminal de $g$ es el punto inicial de $h$. Se tiene entonces que los productos:
        \begin{equation*}
            f\cdot (g\cdot h)\quad\textup{y}\quad (f\cdot g)\cdot h
        \end{equation*}
        están bien definidos (\textit{verificar!}). Para probar el resultado, debemos ver que
        \begin{equation*}
            [f]\cdot \left([g]\cdot [h]\right)=\left([f]\cdot[g] \right)\cdot [h]
        \end{equation*}
        lo que es equivalente a probar que
        
        \begin{equation*}
            [f\cdot (g\cdot h)]=[(f\cdot g)\cdot h]
        \end{equation*}
        es decir, que $f\cdot (g\cdot h)\sim(f\cdot g)\cdot h$. Considere la función $\cf{F}{I\times I}{X}$ dada por:
        \begin{equation*}
            F(t,s)=\left\{
                \begin{array}{lcr}
                    f\left(\frac{4t}{1+s}\right) & \textup{ si } & 0\leq t\leq\frac{s+1}{4}\\
                    g(4t-1-s) & \textup{ si } & \frac{s+1}{4}\leq t\leq\frac{s+2}{4}\\
                    h\left(1-\frac{4(1-t)}{2-s}\right) & \textup{ si } & \frac{s+2}{4}\leq t\leq 1\\
                \end{array}
            \right.
        \end{equation*}

        \begin{figure}
            \begin{center}
                \includegraphics[scale=1]{images/fig_1.pdf}
            \end{center}
            \caption{Dominio de la función $F$.}
        \end{figure}

        para todo $s,t\in I$. Veamos que $F$ es continua... %TODO
    \end{proof}

    Para cualquier punto $x\in X$, denotemos por $\mathscr{I}_x=[i_x]$ a la clase de equivalencia del mapeo identidad, es decir $\cf{i_x}{I}{X}$ es tal que $i_x(t)=x$ para todo $y\in I$. Esta clase tiene la siguiente propiedad fundamental:

    \begin{lema}
        Sea $\mathscr{F}=[f]$ una clase de equivalencia de caminos con punto inicial $x\in X$ y terminal $y\in X$. Entonces, $\mathscr{I}_x\cdot\mathscr{F}=\mathscr{F}$ y $\mathscr{F}\cdot\mathscr{I}_y=\mathscr{F}$.
    \end{lema}

    \begin{proof}
        Solo se probará la primera igualdad, para ello, basta con probar que $i_x\cdot f\sim f$. En efecto, defina la función $\cf{F}{I\times I}{X}$ dada por:
        \begin{equation*}
            F(t,s)=\left\{
                \begin{array}{lcr}
                    x & \textup{ si } & 0\leq t\leq\frac{s}{2}\\
                    f\left(\frac{2t-s}{2-s}\right) & \textup{ si } & \frac{s}{2}\leq t\leq 1\\
                \end{array}
            \right.
        \end{equation*}
        para todo $s,t\in I$. Entonces, $F(t,0)=f(t)$ y $F(t,1)=(e\cdot f)(1)$ siendo $F$ continua. Por tanto, $\mathscr{I}_x\cdot\mathscr{F}=\mathscr{F}$.
    \end{proof}

    En cierto modo, lo que decimos es que la clase $\mathscr{I}_x$ actúa como elemento identidad (\textit{¿De dónde?}).

    \begin{mydef}
        Para cualquier camino $\cf{f}{I}{X}$, $\overline{f}$ denota al camino definido por:
        \begin{equation*}
            \overline{f}(t)=f(1-t),\quad\forall t\in I
        \end{equation*}
        El camino $\overline{f}$ se obtiene recorriendo el camino $f$ en sentido contrario.
    \end{mydef}

    \begin{lema}
        Sea $f$ un camino y denotemos por $\mathscr{F}=[f]$ y $\overline{\mathscr{F}}=[\overline{f}]$, entonces:
        \begin{equation*}
            \mathscr{F}\cdot\overline{\mathscr{F}}=\mathscr{I}_x\quad\textup{y}\quad\overline{\mathscr{F}}\cdot\mathscr{F}=\mathscr{I}_y
        \end{equation*}
        donde $x\in X$ y $y\in X$ son los puntos inicial y terminal de $f$, respectivamnete.
    \end{lema}

    \begin{proof}
        Sólo se probará la primera igualdad, para ello es suficiente con probar que $f\cdot\overline{f}\sim i_x$. Definimos la función $\cf{F}{I\times I}{X}$ por:
        \begin{equation*}
            F(t,s)=\left\{
                \begin{array}{lcr}
                    f(2t) & \textup{ si } & 0\leq t\leq\frac{s}{2}\\
                    f(s) & \textup{ si } & \frac{s}{2}\leq t\leq1-\frac{s}{2}\\
                    f(2-2t) & \textup{ si } & 1-\frac{s}{2}\leq t\leq1\\
                \end{array}
            \right.
        \end{equation*}
        para todo $s,t\in I$. Entonces,
        \begin{equation*}
            \begin{split}
                F(t,0)&=\left\{
                    \begin{array}{lcr}
                        f(2t) & \textup{ si } & 0\leq t\leq0\\
                        f(s) & \textup{ si } & 0\leq t\leq1-0\\
                        f(2-2t) & \textup{ si } & 1-0\leq t\leq1\\
                    \end{array}
                \right.\\
                &=\left\{
                    \begin{array}{lcr}
                        f(0) & \textup{ si } & t=0\\
                        f(0) & \textup{ si } & 0\leq t\leq1\\
                        f(2-2t) & \textup{ si } & t=1\\
                    \end{array}
                \right.\\
                &=\left\{
                    \begin{array}{lcr}
                        x & \textup{ si } & 0\leq t\leq1\\
                        f(0) & \textup{ si } & t=1\\
                    \end{array}
                \right.\\
                &=x\\
            \end{split}
        \end{equation*}
        para todo $t\in I$. Además,
        \begin{equation*}
            \begin{split}
                F(t,1)&=\left\{
                    \begin{array}{lcr}
                        f(2t) & \textup{ si } & 0\leq t\leq\frac{1}{2}\\
                        f(1) & \textup{ si } & \frac{1}{2}\leq t\leq1-\frac{1}{2}\\
                        f(2-2t) & \textup{ si } & 1-\frac{1}{2}\leq t\leq1\\
                    \end{array}
                \right.\\
                &=\left\{
                    \begin{array}{lcr}
                        f(2t) & \textup{ si } & 0\leq t\leq\frac{1}{2}\\
                        f(1-(2t-1)) & \textup{ si } & \frac{1}{2}\leq t\leq1\\
                    \end{array}
                \right.\\
                &=\left\{
                    \begin{array}{lcr}
                        f(2t) & \textup{ si } & 0\leq t\leq\frac{1}{2}\\
                        \overline{f}(2t-1) & \textup{ si } & \frac{1}{2}\leq t\leq1\\
                    \end{array}
                \right.\\
                &=f\cdot\overline{f}(t)\\
            \end{split}
        \end{equation*}
        para todo $t\in I$. La función $F$ es continua... %TODO

        Por tanto, $\mathscr{F}\cdot\overline{\mathscr{F}}$.
    \end{proof}

    En visata de estas propiedades de la clase $\overline{\mathscr{F}}$, de ahora en adelante la denotaremos por $\mathscr{F}^{-1}$.

    Podemos resumir todos los lemas antes probados diciendo que el conjunto de todas las clases de caminos en un espacio $X$ satisfacen los axiomas de grupo, excepto que el producto de dos caminos no siempre está definido. Solventamos este problema con la siguiente definición:

    \begin{mydef}
        Un camino o una clase de camino es llamada \textbf{cerrada} o un \textbf{bucle}, si el punto inicial y terminal son el mimso. El bucle se dice que tiene \textbf{base} en el punto inicial o terminal.
    \end{mydef}

    \begin{theor}
        Sea $X$ un espacio topológico y $x\in X$ un punto fijo. Entonces, el conjunto de todas las clases de caminos cerradas que tienen como punto base a $x$ dotado por la operación $\cdot$, denotado por $\pi(X,x)$ es un grupo llamado \textbf{grupo fundamental} o \textbf{grupo de Poincaré} de $X$ con punto base $x$.
    \end{theor}

    \begin{proof}
        Es un resumen de todos los lemas anteriores.
    \end{proof}

    \begin{obs}
        Para un espacio topológico dado $X$ y $x\in X$, dotamos el grupo fundamental $\pi(X,x)$ de una operación binaria que lo hace de grupo, de ahora en adelante tal operación se denotará al producto de dos clases $[f]$ y $[g]$ por $[f]\cdot[g]$ o por yuxtaposición como $[f][g]=[f\cdot g]$ (no confundir la operación dentro de los paréntesis cuadrados con la composición usual de funciones).

        Si $[f]\in\pi(X,x)$, se denotará a su inverso por $[f]^{-1}$ y, al elemento identidad por $\mathscr{I}$
    \end{obs}

    \begin{propo}
        Sea $X$ un espacio y $x,y\in X$ dos puntos distintos. Si $\cf{\gamma}{I}{X}$ es un camino con punto inicial $x$ y terminal $y$, entonces $\pi(X,x)\cong\pi(X,y)$ (es decir, son grupos isomorfos).
    \end{propo}

    \begin{proof}
        En efecto, defina la función $\cf{u}{\pi(X,x)}{\pi(X,y)}$ dada por:
        \begin{equation*}
            u([f])=[\gamma]^{-1}[f][\gamma]
        \end{equation*}
        Por cursos anteriores de teoría de Grupos, se ve de forma inmediata que esta función es un isomorfismo entre los grupos $\pi(X,x)$ y $\pi(X,y)$.
    \end{proof}

    \begin{cor}
        Sea $X$ un espacio topológico arco-conexo, entonces los grupos $\pi(X,x)$ y $\pi(X,y)$ son isomorfos para todo $x,y\in X$.
    \end{cor}

    La importancia del teorema anterior radica en que el grupo $\pi(X,x)$ tiene propiedades como grupo (es decir, es abeliano, finito, nilpotente, libre, etc...) no debido al punto elegido $x\in X$, sino al espacio mismo $X$, suponiendo que $X$ es arco-conexo.

    En general, no hay un mapeo canónico o isomorfismo natural entre $\pi(X,x)$ y $\pi(X,y)$, ya que a cada elección de camino entre $x$ y $y$ le corresponderá un isomorfismo.

    \section{Efecto de una función continua en el grupo fundamental}

    \begin{obs}
        Para esta sección resultará de utilidad definir el siguiente conjunto, para todo espacio topológico $X$ se define
        \begin{equation*}
            \wp_X = \bigcup \left\{[f]\Big|\cf{f}{I}{X}\textup{ es una función continua} \right\}
        \end{equation*}
        es decir, estamos tomando todas las clases de caminos de un espacio topológico $X$ (note que no tiene nada que ver con el grupo fundamental, más que con el hecho de que usa las clases de caminos en su definición).
    \end{obs}

    Considere dos espacios topológicos $X$ y $Y$ y sea $\cf{\varphi}{X}{Y}$ una función continua. Si $\cf{f_0,f_1}{I}{X}$ son caminos en $X$, ¿también lo son $\varphi\circ f_0$ y $\varphi\circ f_1$?

    \begin{propo}
        Sean $X$ y $Y$ espacios topológicos, $\cf{f_0,f_1}{I}{X}$ caminos equivalentes. Entonces, $\varphi\circ f_0\sim \varphi\circ f_1$.
    \end{propo}

    \begin{proof}
        Como $f_1\sim f_0$, existe pues una función continua $\cf{F}{I\times I}{X}$ tal que
        \begin{equation*}
            F(x,0)=f_0(x),\quad F(x,1)=f_1(x)
        \end{equation*}
        para todo $x\in I$ y,
        \begin{equation*}
            F(0,t)=f_0(0)=f_1(0),\quad F(1,t)=f_0(1)=f_1(1)
        \end{equation*}
        Considere la función $\cf{G}{I\times I}{Y}$ dada por:
        \begin{equation*}
            G(x,t)=\varphi\circ F(x,t)
        \end{equation*}
        Es claro que esta funciónes continua por ser composición de funciones continuas, además se cumple que
        \begin{equation*}
            \begin{split}
                G(x,0)&=\varphi\circ F(x,0)\\
                &=\varphi (F(x,0))\\
                &=\varphi (f_0(x))\\
                &=\varphi\circ f_0(x)\\
            \end{split}
        \end{equation*}
        para todo $x\in I$. De forma análoga
        \begin{equation*}
            G(x,1)=\varphi\circ f_1(x)
        \end{equation*}
        La otra condición se verifica de forma inmediata, con lo que se concluye que $\varphi\circ f_0\sim \varphi\circ f_1$.
    \end{proof}

    Con la proposición anterior, podemos definir sin problemas una función que mapee clases de caminos en $X$ a clases de caminos en $Y$, a partir de la función continua $\varphi$. Esto se hará con el objetivo de ver qué sucede con el grupo fundamental bajo esta función continua $\varphi_*$.

    \begin{mydef}
        Sean $X$ y $Y$ espacios topológicos y $\cf{\varphi}{X}{Y}$ una función continua. Sea $\cf{f}{I}{X}$ un camino que une a los puntos $x,y\in X$, se define la función $\cf{\varphi_*}{\wp_X}{\wp_Y}$ por
        \begin{equation*}
            \varphi_*([f])=[\varphi\circ f]
        \end{equation*}
        por la proposición anterior, esta función está bien definida.
    \end{mydef}

    Ahora, analizaremos las propiedades de la función $\varphi_*$.
    
    \renewcommand{\theenumi}{\roman{enumi}}

    \begin{propo}
        Sean $X$ y $Y$ espacios topológicos y $\cf{\varphi}{X}{Y}$ una función continua.
        \begin{enumerate}
            \item Si $\cf{f_0,f_1}{I}{X}$ son caminos en $X$ tales que $f_0\cdot f_1$ está definido (por ende, $[f_0]\cdot [f_1]$ lo está), entonces $\varphi_*([f_0]\cdot[f_1])=\varphi_*([f_0])\cdot\varphi_*([f_1])$.
            \item Para cualquier punto $x\in X$, $\varphi_*(\mathscr{I}_x)=\mathscr{I}_{\varphi(x)}$.
            \item Si $\cf{f}{I}{X}$ es un camino, entonces $\varphi_*([f]^{-1})=(\varphi_*([f]))^{-1}$.
        \end{enumerate}
    \end{propo}

    \begin{proof}
        De (i): Veamos primero que el producto $\varphi_*([f_0])\cdot\varphi_*([f_1])$ está bien definido. En efecto, dado a que el producto $[f_0]\cdot [f_1]$ lo está, entonces
        \begin{equation*}
            f_0(1)=f_1(0)
        \end{equation*}
        luego,
        \begin{equation*}
            \varphi\circ f_0(1)=\varphi\circ f_1(0)
        \end{equation*}
        donde $\varphi_*([f_0])=[\varphi\circ f_0]$ y $\varphi_*([f_1])=[\varphi\circ f_1]$, por tanto el producto de ambas clases está definido. Probemos ahora la igualdad. Se tiene que
        \begin{equation*}
            \begin{split}
                \varphi_*([f_0]\cdot[f_1])&=\varphi_*([f_0\cdot f_1])\\
                &=[\varphi\circ (f_0\cdot f_1)]\\
            \end{split}
        \end{equation*}
        siendo
        \begin{equation*}
            f_0\cdot f_1(t)=\left\{
                \begin{array}{lcr}
                    f_0(2t) & \textup{ si } 0\leq t\leq \frac{1}{2}\\
                    f_1(2t-1) & \textup{ si } \frac{1}{2}\leq t\leq 1\\
                \end{array}
            \right.,\quad\forall t\in I
        \end{equation*}
        por ende,
        \begin{equation*}
            \begin{split}
                \varphi\circ (f_0\cdot f_1)(t)&=\left\{
                    \begin{array}{lcr}
                        \varphi(f_0(2t)) & \textup{ si } 0\leq t\leq \frac{1}{2}\\
                        \varphi(f_1(2t-1)) & \textup{ si } \frac{1}{2}\leq t\leq 1\\
                    \end{array}
                \right.\\
                &=\left\{
                    \begin{array}{lcr}
                        \varphi\circ f_0(2t) & \textup{ si } 0\leq t\leq \frac{1}{2}\\
                        \varphi\circ f_1(2t-1) & \textup{ si } \frac{1}{2}\leq t\leq 1\\
                    \end{array}
                \right.\\
                &=(\varphi\circ f_0)\cdot (\varphi\circ f_1)(t),\quad\forall t\in I\\
            \end{split}
        \end{equation*}
        luego entonces
        \begin{equation*}
            \begin{split}
                [\varphi\circ (f_0\cdot f_1)]&=[(\varphi\circ f_0)\cdot (\varphi\circ f_1)]\\
                &=[\varphi\circ f_0]\cdot [\varphi\circ f_1]\\
                &=\varphi_*([f_0])\cdot \varphi_*([f_1])\\
            \end{split}
        \end{equation*}
        lo que prueba el resultado.

        De (ii) y (iii): Ejercicio.
    \end{proof}

    Por estas razones, llamaremos a $\varphi_*$ un \textit{homomorfismo} u \textit{homomorfismo inducido por $\varphi$}.

    \begin{propo}
        En el contexto de la proposición anterior, si $Z$ es un espacio topológico y $\cf{\psi}{Y}{Z}$ es una función continua, entonces
        \begin{equation*}
            (\psi\circ\varphi)_*=\psi_*\circ\varphi_*
        \end{equation*}
    \end{propo}

    \begin{proof}
        Es claro que la composición de ambas funciones está bien definida. Probaremos ahora la igualdad, sea $\cf{f}{I}{X}$ un camino, entonces:
        \begin{equation*}
            \begin{split}
                (\psi\circ\varphi)_*([f])&=[\psi\circ\varphi\circ f]\\
                &=[\psi\circ(\varphi\circ f)]\\
                &=\psi_*([\varphi\circ f])\\
                &=\psi_*(\varphi_*([f]))\\
                &=\psi_*\circ \varphi_*([f])\\
            \end{split}
        \end{equation*}
        lo que prueba la igualdad.
    \end{proof}

    \begin{excer}
        Sea $X$ espacio topológico y $\cf{i}{X}{X}$ la función identidad, entonces
        \begin{equation*}
            i_*([f])=[f]
        \end{equation*}
        para todo camino $\cf{f}{I}{X}$ en $X$.
    \end{excer}

    \begin{proof}
        Ejercicio.
    \end{proof}

    \begin{cor}
        Sean $X$ y $Y$ espacios topológicos y $\cf{\varphi}{X}{Y}$ una función continua y $x\in X$. La función $\varphi_*$ restringida a $\pi(X,x)\subseteq\wp_X$ es un homomorfismo entre $\pi(X,x)$ y $\pi(Y,\varphi(x))$. Más aún, si $\varphi$ es homeomorfismo, entonces $\varphi_*$ es isomorfismo.
    \end{cor}

    \begin{proof}
        
    \end{proof}

    \section{Nociones geométricas subyacentes}

    Para continuar con el estudio de la función inducida $\varphi_*$, es necesario introducir algunos conceptos geométricos relevantes,

    \begin{mydef}
        Sean $X$ y $Y$ espacios topológicos. Dos funciones continuas $\cf{\varphi_0,\varphi_1}{X}{Y}$ son \textbf{homotópicas} si existe una función continua $\cf{\Phi}{X\times I}{Y}$ tal que para todo $x\in X$:
        \begin{equation*}
            \Phi(x,0)=\varphi_0(x)\quad\textup{y}\quad\Phi(x,1)=\varphi_1(x)
        \end{equation*}
        además, para denotar que son homotópicas, se usará el símbolo $\varphi_0\simeq\varphi_1$.
    \end{mydef}

    \begin{obs}
        En ciertos casos, será más conveniente denotar a la homotopía como la familia de funciones $\left\{\varphi_t \right\}_{ t\in I}$ tal que cada una es continua y que el mapeo $t\mapsto \varphi_t$ es continuo en el espacio de funciones continuas, donde
        \begin{equation*}
            \Phi(x,t)=\varphi_t(x),\quad\forall x\in X,\forall t\in I
        \end{equation*}
        y por esta razón, se denotará por $\varphi_t$ a $\Phi$.
    \end{obs}

    \begin{propo}
        Sean $X$ y $Y$ espacios topológicos. Considere el conjunto:
        \begin{equation*}
            \mathcal{F}=\left\{\cf{\varphi}{X}{Y}\Big|f\textup{ es una función continua} \right\}
        \end{equation*}
        entonces, $\simeq$ es una relación de equivalencia sobre $\mathcal{F}$.
    \end{propo}

    \begin{proof}
        Ejercicio.
    \end{proof}

    \begin{obs}
        Para aquellos que han tomado algún curso en teoría de categorías, verán de forma casi inmediata que esta relación de equivalencia induce una partición en la clase de todos los morfismos entre espacios topológicos.
    \end{obs}

    La idea detrás de la homotopía es intentar deformar de forma continua una función en la otra, conservando la continuidad de las funciones, veamos que si
    \begin{equation*}
        \varphi_t(x)=\Phi(x,t)
    \end{equation*}
    para todo $x\in X$ y todo $t\in I$, entonces la función $\cf{\varphi_t}{X}{Y}$ es continua.

    Por esta razón es que comúnmente se habla de homotopía como la deformación continua de una función.

    \begin{obs}
        En otros contextos, resulta más familiar decir que dos funciones son homotópicas si pueden ser unidas con un arco en el espacio de todas las funciones continuas que van de $X$ en $Y$.
    \end{obs}

    \begin{mydef}
        Dos funciones $\cf{\varphi_0,\varphi_1}{X}{Y}$ entre los espacios topológicos $X$ y $Y$ son \textbf{homotópicas relativas al subconjunto $A$ de $X$} si existe una función continua $\cf{\Phi}{X\times I}{Y}$ tal que
        \begin{equation*}
            \begin{split}
                \varphi(x,0)=\varphi_0(x) & \quad\forall x\in X\\
                \varphi(x,1)=\varphi_1(x) & \quad\forall x\in X\\
                \varphi(a,t)=\varphi_0(a)=\varphi_1(a) & \quad\forall a\in A\textup{ y }\forall t\in I \\
            \end{split}
        \end{equation*}
    \end{mydef}

    Básicamente la deformación continua es tal que deja al subconjunto $A$ sin modificarse en el proceso de deformación.

    \begin{theor}
        Sean $\cf{\varphi_0,\varphi_1}{X}{Y}$ funciones entre dos espacios topológicos y $x\in X$. Suponga que son homotópicas $\varphi_0$ y $\varphi_1$ relativas al conjunto $\left\{u \right\}$, entonces
        \begin{equation*}
            \cf{{\varphi_0}_*={\varphi_1}_*}{\pi(X,u)}{\pi(Y,\varphi_0(u))}
        \end{equation*}
        esto es, los homomorfismos inducidos son el mismo.
    \end{theor}

    \begin{proof}
        Sea $\cf{f}{I}{X}$ un bucle que une a $x$ consigo mismo. Como $\varphi_0$ y $\varphi_1$ son homotópicas relativas al conjunto $\left\{u\right\}$, entonces existe una función continua $\cf{\Phi}{X\times I}{Y}$ tal que
        \begin{equation*}
            \begin{split}
                \Phi(x,0)=\varphi_0(x) & \quad\forall x\in X\\
                \Phi(x,1)=\varphi_1(x) & \quad\forall x\in X\\
                \Phi(u,t)=\varphi_0(u)=\varphi_1(u) & \quad\forall t\in I
            \end{split}
        \end{equation*}
        por tanto, se tiene para el camino $f$ que la función $\cf{G}{I\times I}{Y}$ dada por
        \begin{equation*}
            G(s,t)=\Phi(f(s),t),\quad\forall s,t\in I
        \end{equation*}
        es continua, y cumple que
        \begin{equation*}
            F(s,0)=\varphi_0\circ f(s)\quad\textup{ y }F(s,1)=\varphi_1\circ f(s)
        \end{equation*}
        para todo $s\in I$. Además,
        \begin{equation*}
            \begin{split}
                F(0,t)&=\Phi(f(0),t)=\Phi(u,t)=\varphi_0(u)=\varphi_1(u)\\
                F(1,t)&=\Phi(f(1),t)=\Phi(u,t)=\varphi_0(u)=\varphi_1(u)\\
            \end{split}
        \end{equation*}
        para todo $t\in I$. Por tanto, los caminos
        \begin{equation*}
            \varphi_0\circ f\quad\textup{ y }\varphi_1\circ f
        \end{equation*}
        son equivalentes, luego:
        \begin{equation*}
            \begin{split}
                {\varphi_0}_*([f])&=[\varphi_0\circ f]\\
                &=[\varphi_1\circ f]\\
                &={\varphi_1}_*([f])\\
            \end{split}
        \end{equation*}
        dado a que el bucle $f$ fue arbitrario, se sigue que
        \begin{equation*}
            {\varphi_0}_*={\varphi_1}_*
        \end{equation*}
    \end{proof}

    Ahora nos dedicaremos a aplicar estos resultados.

    \begin{mydef}
        Un subconjunto $A\subseteq X$ de un espacio topológico es un \textbf{repliegue} de $X$ si existe una función continua $\cf{r}{X}{A}$ (llamada \textbf{retración}) tal que $r(a)=a$ para todo $a\in A$.
    \end{mydef}

    La condición de la definición antes mencionada, es una condición muy fuerte, ya que no todo espacio la cumple para cualquier conjunto $A$ arbitrario. Vea estos dos ejemplos:

    \begin{exa}
        Considere $X=\mathbb{R}^2\backslash\left\{(0,0)\right\}$ y tomemos $A=\mathbb{R}+(1,0)$. ¿Existe un repliegue de $X$ en $A$?
    \end{exa}

    \begin{exa}
        Considere el espacio $X$ como la cinta de Möbius y sea $A$ el círculo central de la cinta. ¿Es $A$ una retracción de $X$? En caso de que sea ¿cuál es una posible retracción?
    \end{exa}

    Sea ahora $\cf{r}{X}{A}$ una retración e $\cf{i}{A}{X}$ el mapeo inclusión. Para cualquier punto $a\in A$ se consideran los homomorfismos inducidos:
    \begin{equation*}
        \begin{split}
            \cf{i_*}{\pi(A,a)}{\pi(X,a)}\\
            \cf{r_*}{\pi(X,a)}{\pi(A,a)}\\
        \end{split}
    \end{equation*}
    siendo éstos tales que $r\circ i=\bbm{1}_{A}$, debe suceder entonces que $r_*\circ i_*=\bbm{1}_{\pi(A,a)}$ es el homomorfismo identidad.
    
    Se sigue entonces que
    \begin{itemize}
        \item $i_*$ es monomorfismo.
        \item $r_*$ es epimorfismo.
    \end{itemize}

    Estos resultados se usarán más adelante para probar que ciertos subespacios no son repliegues del espacio original.

    \begin{mydef}
        Un subconjunto $A$ de $X$ es un \textbf{repliegue de deformación} de $X$ si existe una retracción $\cf{r}{X}{A}$ y una homotopía $\cf{F}{X\times I}{X}$ tal que
        \begin{equation*}
            \begin{split}
                \left.
                \begin{array}{rcl}
                    F(x,0) & = & x\\
                    F(x,1) & = & r(x)\\
                \end{array}
            \right\} & \quad\forall x\in X\\
            F(a,t)=a,\quad&\forall a\in A, \forall t\in I\\
            \end{split}
        \end{equation*}
    \end{mydef}

    En otras palabras, la definición anterior es equivalente a decir que $r\simeq \bbm{1}_X$ y es tal que $F(A\times I)=A$.
    
    \begin{theor}
        Si $A$ es un repliegue de deformación de $X$, entonces el mapeo inclusión $\cf{i}{A}{X}$ induce un isomorfismo entre $\pi(A,a)$ y $\pi(X,a)$ para todo $a\in A$.
    \end{theor}

    \begin{proof}
        Sea $a\in A$. Ya se sabe de la parte anterior que $i_*$ es un monomorfismo. Para probar que es isomorfismo, basta con probar que
        \begin{equation*}
            \cf{i_*\circ r_*}{\pi(X,a)}{\pi(X,a)}
        \end{equation*}
        coincide con $\bbm{1}_{\pi(X,a)}$. Afirmamos que $i\circ r$ es homotópico a $\bbm{1}_X$ respecto a $\left\{a\right\}$. En efecto, como $A$ es repliegue de deformación de $X$, entonces existe una homotopía $\cf{F}{X\times I}{Y}$ tal que
        \begin{equation*}
            \begin{split}
                \left.
                \begin{array}{rcl}
                    F(x,0) & = & \bbm{1}_X(x)\\
                    F(x,1) & = & r(x)\\
                \end{array}
            \right\} & \quad\forall x\in X\\
            F(a',t)=a',\quad&\forall a'\in A, \forall t\in I\\
            \end{split}
        \end{equation*}
        y, como $i\circ r(x)=x$, para todo $x\in X$ (por ser $i$ el mapeo inclusión), se sigue que
        \begin{equation*}
            \begin{split}
                \left.
                \begin{array}{rcl}
                    F(x,0) & = & \bbm{1}_X(x)\\
                    F(x,1) & = & i\circ r(x)\\
                \end{array}
            \right\} & \quad\forall x\in X\\
            F(a,t)=a,\quad&\forall t\in I\\
            \end{split}
        \end{equation*}
        lo cual prueba la afirmación. Se sigue del teorema anterior que
        \begin{equation*}
            i_*\circ r_*=(i\circ r)_*=\bbm{1}_{\pi(X,a)}
        \end{equation*}
    \end{proof}

    Este teorema que acabamos de probar nos va a servir de dos cosas:
    \begin{itemize}
        \item Se usará para probar que dos espacios tienen grupos fundamentales isomorfos.
        \item Un subespacio $A$ no es un repliegue de deformación de $X$ si los grupos fundamentales no son isomorfos.
    \end{itemize}

    En particular se usará el segundo punto para probar que ciertos repliegues no son repliegues de deforamción.

    \begin{mydef}
        Un espacio topológico $X$ es \textbf{contraíble a un punto} si existe $x_0\in X$ tal que $\left\{x_0\right\}$ es un repliegue de deformación de $X$.
    \end{mydef}

    \begin{mydef}
        Un espacio topológico es \textbf{simplemente conexo} si es arco-conexo y $\pi(X,x)=\langle e\rangle$ para algún (y por ende para cualquier) $x\in X$.
    \end{mydef}

    \begin{cor}
        Si un espacio es contraíble a un punto, entonces es simplemente conexo.
    \end{cor}

    \begin{proof}
        Es inmediato del hecho que el grupo fundamental del espacio topológico consistente de un solo elemento es tal que su grupo fundamental es trivial.
    \end{proof}

    \section{Ejemplos}

    \begin{mydef}
        Un subconjunto $X$ del espacio $\mathbb{R}^n$ es llamado \textbf{convexo} si la línea uniendo cualesquiera dos puntos de $X$ está contenida en $X$.
    \end{mydef}

    Afirmamos que todo subconjunto convexo $X$ de $\mathbb{R}^n$ es contraíble a un punto. En efecto, sea $x_0\in X$ arbitrario fijo. Considere la función $\cf{f}{X\times I}{X}$ dada por:
    \begin{equation*}
        F(x,t)=(1-t)x+tx_0
    \end{equation*}

    \begin{exa}
        Todo subconjunto convexo de $\mathbb{R}^n$ es contraíble a un punto.
    \end{exa}

    \begin{proof}
        Sea $X$ un subconjunto convexo de $\mathbb{R}^n$ y $x_0\in X$. Primero debemos dar una retracción de $X$ en $\left\{ x_0\right\}$. Sea
        \begin{equation*}
            r(x)=x_0\quad\forall x\in X
        \end{equation*}
        es claro que esta función es continua. Para ver que $X$ es contraíble a $\left\{x_0\right\}$ veamos que la función $\cf{F}{X\times I}{X}$ dada por:
        \begin{equation*}
            F(x,t)=(1-t)x+tx_0\quad\forall x\in X,t\in I
        \end{equation*}
        (esta función es es básicamente para $x\in X$ fijo el segmento que une a $x$ con $x_0$, y al variar $x$ se pasan por todos los posibles segmentos que unen con $x_0$). Esta función es continua y efectivamente tiene como contradiminio $X$, ya que el espacio $X$ es convexo. Además:
        \begin{equation*}
            \begin{split}
                \left.
                    \begin{array}{rcl}
                        F(x,0) & = & x \\
                        F(x,1) & = & x_0 = r(x) \\
                    \end{array}
                \right\},\quad&\forall x\in X\\
                F(x_0,t)=x_0,\quad&\forall t\in I\\
            \end{split}
        \end{equation*}
        por tanto, $\left\{x_0\right\}$ es repliegue de deformación de $X$, i.e. $X$ es contraíble a un punto.
    \end{proof}

    \begin{mydef}
        Un subconjunto no vacío $X$ de $\mathbb{R}^n$ se dice que tiene \textbf{forma de estrella respecto a $x_0\in X$} si el segmento que une a $x_0$ con $x$ está contenido en $X$, para todo $x\in X$.
    \end{mydef}    

    Como en el ejemplo anterior, se prueba de forma análoga que cualquier conjunto con forma de estrella es contraíble a un punto.

    \begin{exa}
        Afirmamos que la $(n-1)$-esfera unitaria $\mathbb{S}^{ n-1}$
        \begin{equation*}
            \mathbb{S}^{ n-1}=\left\{x\in\mathbb{R}^n\Big|\|x\|=s1 \right\}
        \end{equation*}
        es una deformación de repliegue de $\mathbb{R}^n-\left\{0\right\}$ para todo $n\in\mathbb{N}$.
    \end{exa}

    \begin{proof}
        En efecto, sea $n\in\mathbb{N}$ y considere el repliegue $\cf{r}{\mathbb{R}^n-\left\{0\right\}}{\mathbb{S}^{ n-1}}$ dado por
        \begin{equation*}
            r(x)=\frac{x}{\|x\|},\quad\forall x\in\mathbb{R}^n-\left\{0\right\}
        \end{equation*}
        Claramente esta es una función continua. Construímos la homotopía $\cf{F}{(\mathbb{R}^n-\left\{0\right\})\times I}{\mathbb{R}^n-\left\{0\right\}}$ dada por:
        \begin{equation*}
            F(x,t)=(1-t)x+t\cdot\frac{x}{\|x\|},\quad\forall x\in \mathbb{R}^n-\left\{0\right\}, \forall t\in I
        \end{equation*}
        Es claro que esta función es continua, para la que se cumple    que
        \begin{equation*}
            \begin{split}
                \left.
                    \begin{array}{rcl}
                        F(x,0) & = & x\\
                        F(x,0) & = & r(x)\\
                    \end{array}
                \right\},&\quad\forall x\in\mathbb{R}^n-\left\{0\right\}\\
            \end{split}
        \end{equation*}
        y,
        \begin{equation*}
            \begin{split}
                F(s,t)&=(1-t)s+t\cdot\frac{s}{\|s\|}\\
                &=(1-t)s+ts\\
                &=s,\quad\forall s\in\mathbb{S}^{ n-1},\quad\forall t\in I \\
            \end{split}
        \end{equation*}
        Por tanto, $\mathbb{S}^{ n-1}$ es una deformación de retracción de $\mathbb{R}^n-\left\{0\right\}$.
    \end{proof}

    %TODO: Ver como redactarla adecuadamente.

    \begin{propo}
        Un espacio $X$ es simplemente conexo si y sólo si existe una única clase de equivalencia de caminos que conecten cualesquiera dos puntos de $X$.
    \end{propo}

    \begin{proof}
        Es más fácil ver la suficiencia del teorema para entender que es lo que está sucediendo.

        $\Leftarrow$): Sea $x_0\in X$. Suponga que existe una única clase de equivalencia de caminos que conecten cualesquiera dos puntos de $X$, digamos $[i]$, donde $\cf{i}{I}{X}$ es tal que $i(t)=x_0$ para todo $t\in I$. Entonces si dado $x\in X$, $\cf{f_x}{I}{X}$ es un camino tal que $f_x(t)=x$ para todo $t\in I$, debe suceder que
        \begin{equation*}
            [i]= [f_x]\iff i\simeq f_x
        \end{equation*}
        (en particular si dos caminos son equivalentes, son homotópicos) Por ende, existe una función continua $\cf{F}{X\times I}{X}$ tal que
        \begin{equation*}
            F(y,0)=i(y)=x_0,\quad\forall y\in X
        \end{equation*}
        y
        \begin{equation*}
            F(y,1)=f_x(y)=x,\quad\forall y\in X
        \end{equation*}
        Es decir, la función $\cf{g}{I}{X}$ dada por:
        \begin{equation*}
            g(t)=F(x_0,t),\quad\forall t\in I
        \end{equation*}
        dadas las condiciones anteriores es un camino que une a $x_0$ con $x$ (por ser continua). Luego el espacio $X$ es arco-conexo. Si $\cf{f}{I}{X}$ es un bucle con centro en $x_0$, entonces $f\simeq i$, es decir que es homotópico a la identidad
    \end{proof}

    \section{El grupo fundamental del circulo}

    Procederemos a calcular este grupo fundamental para dar algunos resultados relevantes en las siguientes lecturas. Para calcular este grupo se tienen que hacer antes algunas cosas adicionales.

    \begin{mydef}
        Sea $\cf{\omega}{I}{\mathbb{S}^1}$ el bucle con base $(1,0)$ que va alrededor del circulo exactamente una vez, es decir:
        \begin{equation*}
            \omega(s)=(\cos 2\pi s,\sin 2\pi s),\quad\forall s\in I
        \end{equation*}
        Y, para cada $n\in\mathbb{Z}$ se define:
        \begin{equation*}
            \omega_n(s)=(\cos 2\pi ns,\sin 2\pi ns),\quad\forall s\in I
        \end{equation*}
    \end{mydef}

    \begin{excer}
        Pruebe que:
        \begin{equation*}
            [\omega]^n=[\omega_n],\quad\forall n\in\mathbb{Z}
        \end{equation*}
    \end{excer}

    \begin{proof}
        Ejercicio.
    \end{proof}

    Para la demostración del teorema, lo se se hará es que dado un camino $\cf{f}{I}{\mathbb{S}^1}$, se comparará al mismo con el camino
    \begin{equation*}
        p(s)=(\cos 2\pi s,\sin 2\pi s),\quad\forall s\in I
    \end{equation*}
    este mapeo puede ser visualizado como una helicoide en $\mathbb{R}^3$ (con una parametrización del tipo $s\mapsto (\cos 2\pi s,\sin 2\pi s, s)$, luego para completar a $p$ se proyecta esta helicode en el plano $xy$).

    \begin{obs}
        Veamos que
        \begin{equation*}
            \omega_n=p\circ\widetilde{\omega}_n,\quad\forall n\in\mathbb{Z}
        \end{equation*}
        donde $\cf{\widetilde{\omega}_n}{I}{\mathbb{R}}$ es la función tal que $s\mapsto ns$. Básicamente esta función controla el número de giros adicionales que va a dar la helicoide en un solo intervalo de longitud 1. El signo de $n$ determina el sentido de giro de la helicoide.
    \end{obs}

    \begin{mydef}
        En el contexto de la definición anterior, diremos que $\widetilde{\omega}_n$ es un \textbf{levantamiento} de $\omega_n$.
    \end{mydef}

    La prueba del teorema se hará estudiando como es que los caminos en $\mathbb{S}^1$ se levantan a $\mathbb{R}$.

    \begin{mydef}
        Sea $X$ un espacio. Un \textbf{espacio recubridor} de $X$ consiste de un espacio $\widetilde{X}$ y una función $\cf{p}{\widetilde{X}}{X}$ que satisface lo siguiente:
        \begin{itemize}
            \item Para cada punto $x\in X$ existe una vecidad abierta $U$ de $X$ tal que $p^{-1}(U)$ es la unión disjunta de abiertos tales que cada uno de estos conjuntos es homeomorfo a $U$ con respecto a la función $p$.
        \end{itemize}
        el conjunto $U$ será denominado \textbf{recubierto uniformemente}.
    \end{mydef}

    \begin{propo}
        La función $\cf{p}{\mathbb{R}}{\mathbb{S}^1}$ definida anteriormente hace de la tupla $(\mathbb{R},p)$ un espacio recubridor de $\mathbb{S}^1$. Más aún, todo arco abierto en $\mathbb{S}^1$ es recubierto uniformemente.
    \end{propo}
    
    \begin{proof}
        Para la primera parte, sea $x=(\cos2\pi s,\sin2\pi s)\in\mathbb{S}^1$ (con $s\in[0,1[$). La función $p$ es continua, por lo que el abierto
        \begin{equation*}
            U=\left\{(\cos2\pi(s+r),\sin2\pi (s+r))\Big|r\in[0,1/4[ \right\}
        \end{equation*}
        en $\mathbb{S}^1$ es tal que $p^{-1}(U)$ es abierto. Es claro que
        \begin{equation*}
            p^{-1}(U)=\bigcup_{ s\in\mathbb{Z}}]s-1/4,s+1/4[
        \end{equation*}
        siendo cada uno de los intervalos disjuntos en la unión homeomorfo a $U$.

        Notemos que podemos cambiar $U$ por un arco abierto y el resultado sigue siendo válido. 
    \end{proof}

    Para probar el teorema necesitaremos de estos dos resultados preliminares concernientes a espacios recubridores con $\cf{p}{\widetilde{X}}{X}$:

    \begin{enumerate}
        \item Para cada camino $\cf{f}{I}{X}$ comenzando en $x_0\in X$ y para cada $\widetilde{x}_0\in p^{-1}(x_0)$ existe un único levantamiento $\cf{\widetilde{f}}{I}{\widetilde{X}}$ empezando en $\widetilde{x}_0$.
        \item Para cada homotopía $\cf{F}{I\times I}{X}$ de caminos comenzando en $x_0\in X$ y para cada $\widetilde{x}_0\in p^{-1}(x_0)$ existe un único levantamiento homotópico (es decir, que cada función $f_t(x)=F(t,x)$ es un levantamiento) $\cf{\widetilde{F}(t,x)}{I}{\widetilde{X}}$ empezando en $\widetilde{x}_0$.
    \end{enumerate}

    Para probar estos resultados, basta con probar lo siguiente:

    \begin{propo}
        Dadas funciones $\cf{F}{Y\times I}{X}$ y $\cf{\widetilde{F}}{Y\times\left\{0\right\}}{\widetilde{X}}$ un levantamiento de $F\big|_{Y\times\left\{0\right\}}$ existe una única función $\cf{\widetilde{F}}{Y\times I}{\widetilde{I}}$ levantamiento de $F$ tal que reestringida a $Y\times\left\{0\right\}$ resulta en $\widetilde{F}$
    \end{propo}

    Ya que es un caso particular cuando $Y=\left\{0\right\}$ consiste de un solo punto (en (i)). Para (ii), basta con tomar $Y=I$ y aplicar (i) de la siguiente forma: 

    %TODO

    Cuando se vea la parte de espacios recubridores se hará una prueba formal de este resultado. En esencia este resultado se usa para probar que $\pi(\mathbb{S}^1,(1,0))$ es ínfinito.

    \begin{theor}
        El grupo fundamental $\pi(\mathbb{S}^1,(1,0))$ es cíclico infinito generado por $[\omega]$.
    \end{theor}

    \begin{proof}
        Sea $\cf{f}{I}{\mathbb{S}^1}$ un bucle con punto base $x_0=(1,0)$. Por (i) existe un levantamiento $\widetilde{f}$ empezando en $0$. Este camino termina en algún entero $n\in\mathbb{Z}$ pues al ser levantamiento se cumple que
        \begin{equation*}
            f = p\circ \widetilde{f}
        \end{equation*}
        luego, para $s=1$:
        \begin{equation*}
            p(\widetilde{f}(1))=p\circ\widetilde{f}(1)=f(1)=x_0
        \end{equation*}
        y,
        \begin{equation*}
            p^{-1}(x_0)=\mathbb{Z}\subseteq\mathbb{R}
        \end{equation*}
        Por tanto, debe existir $n\in\mathbb{Z}$ tal que $\widetilde{f}(1)=n$. Así $\widetilde{f}$ es un camino que va de $0$ a $n$. Otro camino que también hace lo mismo es $\widetilde{\omega}_n$. Además
        \begin{equation*}
            \widetilde{f}\simeq \widetilde{\omega}_n
        \end{equation*}
        dada la función continua $\cf{F}{I\times I}{\mathbb{R}}$, $F(t,s)=(1-t)\widetilde{f}(s)+t\widetilde{\omega}_n(s)$, para todo $s,t\in I$. Tomando $G(s,t)=p\circ F(s,t)$ resulta en que $f\sim\omega_n$, luego $[f]=[\omega_n]$.

        Para ver la unicidad de $n$, suponga que $[f]$ está determinado por $[\omega_n]$ y $[\omega_m]$, es decir que
        \begin{equation*}
            f\sim\omega_n\quad\textup{y}\quad f\sim\omega_m
        \end{equation*}
        con $m\in\mathbb{Z}$. Se sigue que $\omega_n\sim\omega_m$. Sea $\cf{F}{I\times I}{\mathbb{S}^1}$ la homotopía tal que
        \begin{equation*}
            f_0=\omega_n\quad\textup{y}\quad f_1=\omega_m
        \end{equation*}
        Por (ii) se tiene que...
        %TODO
    \end{proof}

    Como aplicación tenemos el siguente teorema:

    \begin{theor}[\textbf{Teorema fundamental del álgebra}]
        Todo polinomio no constante con coeficientes en $\mathbb{C}$ tiene una raíz en $\mathbb{C}$.
    \end{theor}

    \begin{proof}
        Sea $\cf{p}{\mathbb{C}}{\mathbb{C}}$ un polinomio. Sin pérdida de generalidad, podemos asumir que
        \begin{equation*}
            p(z)=z^{ n}+a_1z^{ n-1}+...+a_n
        \end{equation*}
        donde $a_1,...,a_n\in\mathbb{C}$, con $n\geq0$. Si $n=0$ entonces $p(z)=1,\forall z\in\mathbb{C}$ (normalizando al polinomio).

        Supongamos que $p$ no tiene raíces en $\mathbb{C}$, entonces para todo $r\geq 0$, y para todo:
        \begin{equation*}
            f_r(s)=\frac{p(re^{ 2\pi is})/p(r)}{\abs{p(re^{ 2\pi is})/p(r)}},\quad\forall s\in I
        \end{equation*}
        define un bucle en el círculo unitario $\mathbb{S}^1\subseteq\mathbb{C}$ con base en 1. Se tiene que $f_0$ es el bucle constante con base en 1, por lo que $[f_0]$ es la identidad de $\pi(\mathbb{S}^1,1)$. Afirmamos que
        \begin{equation*}
            f_r\sim f_0
        \end{equation*}
        para todo $r>0$. En efecto, sea $\cf{F}{I\times I}{\mathbb{S}^1}$ dada por:
        \begin{equation*}
            F(t,s)=f_{ rt}(s),\quad\forall  s,t\in I
        \end{equation*}
        Es claro de la definición de $f_r$ que $F(t,s)$ es continua. Veamos que:
        \begin{equation*}
            F(0,s)=f_0(s)\quad\textup{y}\quad F(1,s)=f_r(s),\quad\forall s\in I
        \end{equation*}
        Y además:
        \begin{equation*}
            F(t,0)=f_{ rt}(0)=1\quad\textup{y}\quad F(t,1)=f_{ rt}(1)=1,\quad\forall t\in I
        \end{equation*}
        (pues todos los bucles tienen como punto base a $1$). Por tanto, $f_0\sim f_r$ para todo $r\geq0$. Se sigue pues que
        \begin{equation*}
            [f_0]=[f_r],\quad\forall r\geq0
        \end{equation*}
        es decir que $[f_r]$ es la identidad de $\pi(\mathbb{S}^1,(1,0))$ para todo $r\geq0$.

        Fijemos ahora $r>0$ tal que
        \begin{equation*}
            r>\max\left\{\abs{a_1}+...+\abs{a_n},1\right\}
        \end{equation*}
        Entonces para $\abs{z}=r$ se tiene que
        \begin{equation*}
            \begin{split}
                \abs{a_1z^{ n-1}+...+a_n}&\leq\abs{a_1z^{ n-1}}+...+\abs{a_n}\\
                &<(\abs{a_1}+...+\abs{a_n})\abs{z}^{n-1}\\
                &<\abs{z}^n\\
                \Rightarrow \abs{a_1z^{ n-1}+...+a_n}&<\abs{z}^n\\
                \Rightarrow 0 &< \abs{z}^n-\abs{a_1z^{ n-1}+...+a_n}\\
            \end{split}
        \end{equation*}
        para todo $t\in I$, lo cual además implica que
        \begin{equation*}
            t\abs{a_1z^{ n-1}+...+a_n}<\abs{z}^n,\quad\forall t\in I
        \end{equation*}
        Por tanto, el polinomio
        \begin{equation*}
            p_t=z^n+t\left(a_1z^{ n-1}+...+a_n \right)
        \end{equation*}
        no tiene raíces cuando $\|z\|=r$ (círculo centrado en 0 de radio $r$) y cuando $t\in I$, pues:
        \begin{equation*}
            \begin{split}
                \abs{p_t(z)}&=\abs{z^n+t\left(a_1z^{ n-1}+...+a_n \right)}\\
                &\geq\abs{z}^n-t\abs{a_1z^{ n-1}+...+a_n}\\
                &>0\\
            \end{split}
        \end{equation*}
        para todo $\abs{z}=r$ y $t\in I$. Cambiando a $p$ por $p_t$ en la fórmula de $f_r$ y haciendo
        \begin{equation*}
            F(t,s)=\frac{p_t(re^{ 2\pi is})/p_t(r)}{\abs{p_t(re^{ 2\pi is})/p_t(r)}}
        \end{equation*}
        obtenemos que $f_r\sim\omega_n$. En efecto, pues
        \begin{equation*}
            \begin{split}
                F(0,s)&=\frac{p_0(re^{ 2\pi is})/p_0(r)}{\abs{p_0(re^{ 2\pi is})/p_0(r)}}\\
                &=\frac{r^n e^{ 2\pi ins}/r^n}{\abs{r^n e^{ 2\pi ins}/r^n}}\\
                &=e^{ 2\pi ins},\quad\forall s\in I \\
            \end{split}
        \end{equation*}
        y,
        \begin{equation*}
            \begin{split}
                F(1,s)&=\frac{p_1(re^{ 2\pi is})/p_1(r)}{\abs{p_1(re^{ 2\pi is})/p_1(r)}}\\
                &=\frac{p(re^{ 2\pi is})/p(r)}{\abs{p(re^{ 2\pi is})/p(r)}},\quad\forall s\in I\\
            \end{split}
        \end{equation*}
        (lo de los extremos que se quedan fijos se verifica rápidamente). Anteriormente se probó que $\omega_n$ es tal que
        \begin{equation*}
            [\omega_n]=[\omega]^n
        \end{equation*}
        por ende,
        \begin{equation*}
            [f_r]=[\omega]^n
        \end{equation*}
        pero $[f_r]=[f_0]$ es la identidad del grupo, por ende debe suceder que $n=0$. Por tanto,
        \begin{equation*}
            p(z)=1,\quad\forall z\in\mathbb{C}
        \end{equation*}
        el único polinomio que no tiene raíces en $\mathbb{C}$ es el polinomio constante no cero.
    \end{proof}

    Ahora un resultado familiar enunciado al inicio del taller:

    \begin{theor}[\textbf{Teorema del punto fijo de Brower para dimensión 2}]
        Toda función continua $\cf{f}{\mathbb{D}^2}{\mathbb{D}^2}$ tiene un punto fijo, es decir existe $z\in\mathbb{D}^2$ tal que $f(z)=z$, donde:
        \begin{equation*}
            \mathbb{D}^2=\left\{(x,y)\in\mathbb{R}^2\Big|x^2+y^2\leq1 \right\}
        \end{equation*}
    \end{theor}

    \begin{proof}
        %TODO
        Suponga que para todo $x\in\mathbb{D}^2$ se tiene que $f(x)\neq x$. Definimos la función $\cf{r}{\mathbb{D}^2}{\mathbb{S}^1}$ dada como sigue:
        \begin{itemize}
            \item Sea $x\in\mathbb{D}^2$. Considere la función $\cf{l_x}{]0,\infty[}{\mathbb{R}^2}$ dada por:
            \begin{equation*}
                l_x(t)=(1-t)f(x)+tx
            \end{equation*}
            (es la recta que comienza en $f(x)$ y va en dirección de $x$). Por el teorema del valor medio se tiene que existe un único $t_x\in]0,\infty[$ tal que
            \begin{equation*}
                \|l_x(t_x)\|=1
            \end{equation*}
            En efecto, sea $\cf{L}{]0,\infty[\times\mathbb{D}^2}{\mathbb{R}}$ dada por:
            \begin{equation*}
                L(x,t)=\|l_x(t)\|=\|(1-t)f(x)+tx\|
            \end{equation*}
            (la prueba no es muy complicada, salvo la unicidad todo se puede hacer rápidamente). Hacemos entonces
            \begin{equation*}
                r(x)=l_x(t_0)
            \end{equation*}
            \item En la función anterior, considere el mapeo $\cf{\gamma}{\mathbb{D}^2}{]0,\infty[}$ tal que
            \begin{equation*}
                x\mapsto t_0
            \end{equation*}
            es continua. En efecto, sea $x\in\mathbb{D}^2$ y $\left\{x_n \right\}_{ n=1}^{\infty}$ una sucesión que converge a $x$.
        \end{itemize}
        Afirmamos que $r$ es continua. En efecto, veamos que
        \begin{equation*}
            r(x)=(1-\gamma(x))f(x)+\gamma(x)x
        \end{equation*}
        como todas las funciones involucradas son continuas, se sigue que $r$ es continua.
    \end{proof}
    

    \newpage

    \section{Ejercicios}

    \begin{excer}
        Pruebe que un espacio conexo y localmente arco-conexo es arco conexo.
    \end{excer}

    \begin{proof}
        
    \end{proof}

    \begin{excer}
        ¿Bajo qué condiciones dos clases de caminos que unen a $x$ y $y$ se tendrá el mismo isomorfismo entre de $\pi(X,x)$ y $\pi(X,y)$?
    \end{excer}

    \begin{sol}
        
    \end{sol}

    \begin{excer}
        Sea $X$ un espacio arco conexo. ¿Bajo qué condiciones es la siguiente proposición válida? Para cualesquiera dos puntos $x,y\in X$ todas las clases de caminos de $x$ a $y$ dan el mismo isomorfismo entre $\pi(X,x)$ y $\pi(X,y)$. 
    \end{excer}

    \begin{sol}
        
    \end{sol}

    \begin{excer}
        Sean $\cf{f,g}{I}{X}$ dos caminos con punto inicial $x_0$ y final $x_1$. Pruebe que $f\sim g$ si y sólo si $f\cdot\overline{g}$ es equivalente al camino constante en $x_0$ (recordando que $\overline{g}$ es el camino que invierte la forma de recorrer a $g$).
    \end{excer}

    \begin{proof}
        
    \end{proof}

    \begin{excer}
        Sean $\cf{\varphi}{X}{Y}$ una función continua y $[f]$ una clase de camino en $X$ que va de $x_0$ a $x_1$. Pruebe que el siguiente diagrama es conmutativo:
        \begin{equation*}
            \begin{array}{rcccl}
              & \pi(X,x_0) & \overset{\varphi_*}{\longrightarrow} & \pi(Y,\varphi(x_0)) & \\
              u & \downarrow & & \downarrow & v \\
               & \pi(X,x_1) & \overset{\varphi_*}{\longrightarrow} & \pi(Y,\varphi(x_1)) & \\
            \end{array}
        \end{equation*}
        donde $u$ es el homomorfismo definido como: $u([g])=[f]^{-1}\cdot[g]\cdot[f]$ y $v$ se define de forma similar usando $\varphi_*([f])$ en lugar de $[f]$. ¿Qué sucede si $\varphi(x_0)=\varphi(x_1)$?
    \end{excer}

    \begin{proof}
        
    \end{proof}

    \begin{excer}
        Construya una deformación de retracción de $\mathbb{R}^n$ en $S^{ n-1}$.
    \end{excer}

    \begin{sol}
        
    \end{sol}

    \begin{excer}
        Pruebe que un repliegue de un espacio Hausdorff debe ser un conjunto cerrado.
    \end{excer}

    \begin{proof}
        
    \end{proof}

    \begin{excer}
        Pruebe que si $A$ es un repliegue de $X$ y $\cf{r}{X}{A}$ es una retracción, $\cf{i}{A}{X}$ es el mapeo inclusión, $a\in A$ es arbitrario fijo y $i_*(\pi(A,a))$ es un subgrupo normal de $\pi(X,a)$, entonces $\pi(X,a)$ es el producto directo de los subgrupos
        \begin{equation*}
            i_*(\pi(A,a))\quad\textup{y}\quad \ker(r_*)
        \end{equation*}
    \end{excer}

    \begin{proof}
        
    \end{proof}

    \begin{excer}
        Sea $A$ un subespacio de $X$, y sea $Y$ un espacio topológico no vacío. Pruebe que $A\times Y$ es un repliegue de $X\times Y$ si y sólo si $A$e s una repliegue de $X$.
    \end{excer}

    \begin{proof}
        
    \end{proof}

    \begin{excer}
        Pruebe que la relación \textbf{ser repliegue de} es transitiva, esto es, si $A$ es un repliegue de $B$ y $B$ es un repliegue de $C$, entonces $A$ es un repliegue de $C$.
    \end{excer}

    \begin{proof}
        
    \end{proof}



\end{document}