%% Inicio del archivo `template.tex'.
%% Copyright 2006-2013 Xavier Danaux (xdanaux@gmail.com).
%
% Este trabajo puede ser distribuido o modificado bajo las 
% condiciones de la LaTeX Project Public License V1.3c, 
% disponible en http://www.latex-project.org/lppl/.
%
% Traducción al Español por Fausto M. Lagos (piratax007@protonmail.ch), 2016.


\documentclass[11pt,a4paper,sans]{moderncv}        % posibles opciones de tamaño de fuente ('10pt', '11pt' and '12pt'), papel ('a4paper', 'letterpaper', 'a5paper', 'legalpaper', 'executivepaper' and 'landscape') y familia de fuentes ('sans' and 'roman')

% temas moderncv
\moderncvstyle{casual}                             % Los estilos disponibles son 'casual' (default), 'classic', 'oldstyle' and 'banking'
\moderncvcolor{blue}                               % las opciones de color son 'blue' (default), 'orange', 'green', 'red', 'purple', 'grey' and 'black'
%\renewcommand{\familydefault}{\sfdefault}         % descomentar al inicio de la línea para definir la fuente por defecto; use '\sfdefault' para sans serif por defecto, '\rmdefault' para roman, o cualquier otro nombre de fuente instalada en sus sistema
%\nopagenumbers{}                                  % descomente para eliminar el numerado automático de las páginas en cartas de más de una página

% Codificación de carácteres
\usepackage[utf8]{inputenc}                        % Si no esta usando xelatex o lualatex, remplace por la codificación que este usando
%\usepackage{CJKutf8}                              % descomente si necesita usar CJK para escribir su carta en Chino, Japones or Koreano
\usepackage[spanish, english]{babel}			   % comentar si su carta esta escrita en un idioma diferente del Español

% Configuración de márgenes
\usepackage[scale=0.75]{geometry}
%\setlength{\hintscolumnwidth}{3cm}                % descomente si quiere modificar el ancho de columna para la fecha
%\setlength{\makecvtitlenamewidth}{10cm}           % para el estilo 'classic', si quiere forzar el ancho del nombre. la longitud es normalmente calculada para evitar sobrelapamientos con su información personal; descomente esta línea bajo su propio riesgo

% Información personal
\name{Cristo Daniel}{Alvarado}
\title{Resumé title}                               % opcional, remover o comentar si no quiere que aparezca su título personal
\phone{55 6437 4266}
\email{calvarado1700@alumno.ipn.mx}                               % opcional, remover o comentar si no quiere incluir su dirección de email
\quote{La técnica al servicio de la patria}                                 % opcional, remover o comentar si no quiere una frase o cita

% para mostrar etiquetas numéricas en la bibliografía (por defecto no se muestran etiquecas); descomente las siguientes líneas solo si usa referencias bibliográficas en su carta
%\makeatletter
%\renewcommand*{\bibliographyitemlabel}{\@biblabel{\arabic{enumiv}}}
%\makeatother
%\renewcommand*{\bibliographyitemlabel}{[\arabic{enumiv}]} % Considere reemplazar la línea 44 con esta

% bibliografía con múltiples entradas
%\usepackage{multibib}
%\newcites{book,misc}{{Books},{Others}}
%----------------------------------------------------------------------------------
%            contenido
%----------------------------------------------------------------------------------
\begin{document}
%-----       carta       ---------------------------------------------------------
% Datos del destinatario
\recipient{Dr. Erick Lee Guzmán}{Ciudad de México}
\date{21 de agosto de 2024}
\opening{Estimado Dr. Erick Lee Guzmán}
\closing{Esperando una respuesta favorable, le envío un cordial saludo.}

\makelettertitle

Espero que esta carta le encuentre bien. Me dirijo a usted con el propósito de solicitar la asignación de un salón para los seminarios que impartiré sobre Topología Algebraica, que se llevarán a cabo los días lunes y miércoles, de 15:00 a 17:00.

El curso titulado "Seminarios de Topología Algebraica" abarcará los siguientes temas:

\begin{enumerate}
    \item Grupo Fundamental
    \item Grupos Libres y Producto directo de Grupos
    \item Teorema de Seifert y Van Kampen
    \item Aplicaciones
    \item Espacios Recubridores
\end{enumerate}

Actualmente, contamos con un grupo de 22 alumnos interesados en asistir a estos seminarios. Para poder ofrecer una experiencia de aprendizaje óptima y acomodar adecuadamente a todos los participantes, es crucial disponer de un salón adecuado que pueda albergar a este grupo de estudiantes.

Le agradecería mucho si pudiera facilitar la asignación de un salón apropiado para el desarrollo de estos seminarios. La disposición de un espacio adecuado no solo contribuirá a la efectividad del curso, sino que también garantizará que se cubran las necesidades logísticas y pedagógicas del mismo.

Estoy a su disposición para coordinar cualquier detalle relacionado con la organización de los seminarios y para proporcionar información adicional si es necesario. Puede encontrar mi información de contacto en la parte de abajo de esta carta para cualquier consulta adicional o para coordinar la asignación del salón.

Agradezco de antemano su colaboración y apoyo en este asunto. Estoy seguro de que con su ayuda será posible ofrecer un curso exitoso y enriquecedor para todos los participantes.

\makeletterclosing

\end{document}