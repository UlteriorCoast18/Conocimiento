\documentclass[12pt]{report}
\usepackage[spanish]{babel}
\usepackage[utf8]{inputenc}
\usepackage{amsmath}
\usepackage{amssymb}
\usepackage{amsthm}
\usepackage{graphics}
\usepackage{subfigure}
\usepackage{lipsum}
\usepackage{array}
\usepackage{multicol}
\usepackage{enumerate}
\usepackage{hyperref}
\usepackage[framemethod=TikZ]{mdframed}
\usepackage[a4paper, margin = 1.5cm]{geometry}

%En esta parte se hacen redefiniciones de algunos comandos para que resulte agradable el verlos%

\renewcommand{\theenumii}{\roman{enumii}}

\def\proof{\paragraph{Demostración:\\}}
\def\endproof{\hfill$\blacksquare$}

\def\sol{\paragraph{Solución:\\}}
\def\endsol{\hfill$\square$}

%En esta parte se definen los comandos a usar dentro del documento para enlistar%

\newtheoremstyle{largebreak}
  {}% use the default space above
  {}% use the default space below
  {\normalfont}% body font
  {}% indent (0pt)
  {\bfseries}% header font
  {}% punctuation
  {\newline}% break after header
  {}% header spec

\theoremstyle{largebreak}

\newmdtheoremenv[
    leftmargin=0em,
    rightmargin=0em,
    innertopmargin=-2pt,
    innerbottommargin=8pt,
    hidealllines = true,
    roundcorner = 5pt,
    backgroundcolor = gray!60!red!30
]{exa}{Ejemplo}[section]

\newmdtheoremenv[
    leftmargin=0em,
    rightmargin=0em,
    innertopmargin=-2pt,
    innerbottommargin=8pt,
    hidealllines = true,
    roundcorner = 5pt,
    backgroundcolor = gray!50!blue!30
]{obs}{Observación}[section]

\newmdtheoremenv[
    leftmargin=0em,
    rightmargin=0em,
    innertopmargin=-2pt,
    innerbottommargin=8pt,
    rightline = false,
    leftline = false
]{theor}{Teorema}[section]

\newmdtheoremenv[
    leftmargin=0em,
    rightmargin=0em,
    innertopmargin=-2pt,
    innerbottommargin=8pt,
    rightline = false,
    leftline = false
]{propo}{Proposición}[section]

\newmdtheoremenv[
    leftmargin=0em,
    rightmargin=0em,
    innertopmargin=-2pt,
    innerbottommargin=8pt,
    rightline = false,
    leftline = false
]{cor}{Corolario}[section]

\newmdtheoremenv[
    leftmargin=0em,
    rightmargin=0em,
    innertopmargin=-2pt,
    innerbottommargin=8pt,
    rightline = false,
    leftline = false
]{lema}{Lema}[section]

\newmdtheoremenv[
    leftmargin=0em,
    rightmargin=0em,
    innertopmargin=-2pt,
    innerbottommargin=8pt,
    roundcorner=5pt,
    backgroundcolor = gray!30,
    hidealllines = true
]{mydef}{Definición}[section]

\newmdtheoremenv[
    leftmargin=0em,
    rightmargin=0em,
    innertopmargin=-2pt,
    innerbottommargin=8pt,
    roundcorner=5pt
]{excer}{Ejercicio}[section]

%En esta parte se colocan comandos que definen la forma en la que se van a escribir ciertas funciones%

\newcommand\abs[1]{\ensuremath{\left|#1\right|}}
\newcommand\divides{\ensuremath{\bigm|}}
\newcommand\cf[3]{\ensuremath{#1:#2\rightarrow#3}}
\newcommand\natint[1]{\ensuremath{\left[\!\left[ #1\right]\!\right]}}
\newcommand{\afa}{\:
    \begin{tikzpicture}
        \draw [line width = 0.17 mm, black] (0,0) -- (-0.115,0.29);
        \draw [line width = 0.17 mm, black] (0,0) -- (0.115,0.29);
        \draw [line width = 0.17 mm, black] (-0.12,0) arc (190:-10:0.12cm);
    \end{tikzpicture}
    \:
}
%Este símvolo es para casi todo salvo una cantidad finita

%recuerda usar \clearpage para hacer un salto de página

\begin{document}
    \setlength{\parskip}{5pt} % Añade 5 puntos de espacio entre párrafos
    \setlength{\parindent}{12pt} % Pone la sangría como me gusta
    \title{Notas sobre ideas de interés}
    \author{Cristo Daniel Alvarado}
    \maketitle

    \tableofcontents %Con este comando se genera el índice general del libro%

    %\setcounter{chapter}{3} %En esta parte lo que se hace es cambiar la enumeración del capítulo%
    
    \chapter{???}
    
    \section{Mecánica Clásica y Cuántica}
    
    Desconozco en que momento empezó a llamarme tanto la atención la mecánica cuántica, así que me dejo unos recursos para visitarlos en un futuro.

    \begin{itemize}
        \item Sobre el Hamiltoniano en la mecánica clásica: \href{https://es.wikipedia.org/wiki/Hamiltoniano_(mecánica_clásica)}{Hamiltoniano}.
        \item Ecuaciones de movimiento: \href{https://es.wikipedia.org/wiki/Ecuación_de_movimiento}{Ecuación de Movimiento}.
    \end{itemize}

    \section{Variable Compleja}

    En algún congreso (en el de 3-variedades hiperbólicas) se habló un poco sobre esta clase de mapeos, llamados \textit{transformaciones de Möbius}, lo cual parece ser muy fundamental pero desconozco del tema. Algunas fuentes:

    \begin{itemize}
        \item Transformaciones de Möbius: \href{https://math.libretexts.org/Bookshelves/Geometry/Geometry_with_an_Introduction_to_Cosmic_Topology_(Hitchman)/03\%3A_Transformations/3.04\%3A_Mobius_Transformations}{Möbius Transformations}.
    \end{itemize}
    
    dentro de todas estas ideas, surgió el análisis del hiperplano:

    \begin{itemize}
        \item Geometría Hiperbólica: \href{https://en.wikipedia.org/wiki/Hyperbolic_geometry}{Hyperbolic Geometry}.
        \item Curvas sobre el plano hiperbólico: \href{https://math.stackexchange.com/questions/2430495/curvature-of-curves-on-the-hyperbolic-plane}{Curvature of curves on the hyperbolic plane}.
        \item Espacio hiperbólico: \href{https://en.wikipedia.org/wiki/Hyperbolic_space}{Hyperbolic Space}.
        \item Modelos del plano hiperbólico: \href{https://en.wikipedia.org/wiki/Poincar%C3%A9_half-plane_model}{Poincaré Half-Plane Model}.
    \end{itemize}

    \section{}

\end{document}