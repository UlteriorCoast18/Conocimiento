\documentclass[../algebraic_geometry.tex]{subfiles}

\begin{document}

    \chapter{Basic Notions}

    Throughout this document, we work with a fixed algebraically closed field $k$ (or sometimes denoted $K$), called \textbf{ground field}.

    \section{Algebraic Closed Fields}

    Let $k$ be an algebraic closed field, then for every $f\in k[x]$ (where $k[x]$ denotes the set of all polynomials over the field $k$), we have that if $f$ is non-constant, then $f$ has a root on $k$.

    \begin{exa}
        $\bbm{C}$ is an algebraic closed extension of $\bbm{R}$, and $\bbm{Q}$.
    \end{exa}

    Turns out algebraic closed fields are important due to the following fact:

    \begin{center}
        \textit{There is a one to one correspondence between geometric and algebraic objects using algebraic closed fields}.
    \end{center}

    We will first deal with simple objects and then we'll scalate to more complex in order to generalize certain notions defined in algebraic geometry.

    \begin{mydef}[\textbf{Affine Space}]
        Let $k$ be a field. We denote by $\bbm{A}^n$ the \textbf{$n$-dimensional affine space over the field $k$}, that is:
        \begin{equation*}
            \bbm{A}^n=\left\{(\alpha_1,\dots,\alpha_n)\Big|\alpha_i\in k,\quad\forall i\in\natint{1,n} \right\}
        \end{equation*}
    \end{mydef}

    All the geometric concepts we will be dealing with are within this space, this is due to the following fact:

    \begin{mydef}[\textbf{Closed Subsets of Affine Space}]
        Let $X\subseteq\bbm{A}^n$ be a subset, then $X$ is \textbf{closed} if there exists $f_1,\dots,f_m\in k[x_1,\dots,x_m]$ polynomials over the field $k$ such that:
        \begin{equation*}
            f_i(u)=0,\quad\forall i\in\natint{1,m} \iff u\in X
        \end{equation*}
    \end{mydef}

    From now on, we will write $F(T)$ to denote a polynomial in $n$-variables, allowing $T$ to stand for the set of variables $T_1,\dots,T_n$.

    \begin{obs}[\textbf{Equations of a Set}]
        If a closed subset $X$ consists of all common zeros of polynomials $F_1(T),\dots,F_m(T)$, then we refer to:
        \begin{equation*}
            F_1(T)=\dots=F_n(T)=0        
        \end{equation*}
        as the \textbf{equations of the set $X$}.
    \end{obs}
    
    One really useful definition and fact that are proved in the notes of Algebra Moderna III are the following:

    \begin{mydef}[\textbf{Noetherian Ring}]
        Let $R$ be a ring. We say that $R$ is \textbf{noetherian} if for all $I$ ideal of $R$ there exists $a_1,\dots,a_n\in R$ such that:
        \begin{equation*}
            I=(a_1,\dots,a_n),
        \end{equation*}
        where $(a_1,\dots,a_n)$ denotes the ideal generated by the set $\left\{a_1,\dots,a_n\right\}$.
    \end{mydef}

    \begin{obs}[\textbf{Ideal Generated by a Set $S$}]
        \label{ideal_generated_finite_family}
        Let $R$ be a commutative ring with identity, then the ideal generated by $S\subseteq R$ is:
        \begin{equation*}
            (S)=\left\{\sum_{ i=1}^n r_is_i\Big|r_i\in R,s_i\in S,\forall i\in\natint{1,n};n\in\mathbb{N} \right\}
        \end{equation*}
    \end{obs}

    \begin{theor}[\textbf{Hilbert Basis Theorem}]
        Let $R$ be a noetherian ring, then $R[x]$ is noetherian.
    \end{theor}

    A useful fact about the Hilbert basis theorem is that it generalizes neatly to an arbitrary number of indeterminates:

    \begin{cor}
        Let $R$ be a noetherian ring, then $R[T]$ is noetherian.
    \end{cor}

    This useful fact is fundamental in the proof of the following result:

    \begin{propo}
        Let $X$ be a set defined by an infinite system of equations $\left\{F_\alpha(T)\right\}_{\alpha\in I}$ with $I\neq\emptyset$. Then $X$ is closed.
    \end{propo}

    \begin{proof}
        Let $\mathfrak{U}$ be the ideal generated by the system of equations $\left\{F_\alpha(T)\right\}_{\alpha\in I}$. Since $k[T]$ is noetherian due to the fact that $k$ is a field (in particular, every field is noetherian), then there exists $G_1,\dots,G_m\in k[T]$ such that:
        \begin{equation*}
            \left(\left\{F_\alpha(T)\right\}_{\alpha\in I}\right)=\left(G_1(T),\dots,G_m(T) \right)
        \end{equation*}
        We claim that $u\in X$ iff $G_i(u)=0$, for all $i\in\natint{1,m}$. If $u\in X$, then $F_\alpha(u)=0$ for all $\alpha\in I$, so in particular by Observation (\ref{ideal_generated_finite_family}):
        \begin{equation*}
            F(u)=0,\quad\forall F\in\left(\left\{F_\alpha(T)\right\}_{\alpha\in I}\right)
        \end{equation*}
        which is the same as:
        \begin{equation*}
            F(u)=0,\quad\forall F\in\left(G_1(T),\dots,G_m(T) \right) 
        \end{equation*}
        so $G_i(u)=0$, for all $i\in\natint{1,m}$.
    \end{proof}

    It follows from this proposition that the arbitrary intersection of closed subsets of $\bbm{A}^n$ is closed. Also, it happens that $\emptyset$ and $\bbm{A}^n$ are closed (using the polynomials $F=1$ and $F=0$).

    \begin{propo}
        If $X_1$ and $X_2$ are closed subsets of $\bbm{A}^n$, then $X_1\cup X_2$ is also a closed subset of $\bbm{A}^n$.
    \end{propo}

    \begin{proof}
        Let $F_1,\dots,F_n$ and $G_1,\dots,G_m$ polinomials over the ring $k[T]$ such that:
        \begin{equation*}
            F_i(u)=0,\quad\forall u\in X_1\textup{ and }G_j(v)=0,\quad\forall v\in X_2
        \end{equation*}
        for all $(i,j)\in\natint{1,n}\times\natint{1,m}$. We define $H_{ i,j}\in k[T]$ as:
        \begin{equation*}
            H_{ i,j}=F_i G_j,\quad\forall (i,j)\in\natint{1,n}\times\natint{1,m}
        \end{equation*}
        Then:
        \begin{equation*}
            H_{ i,j}(w)=0,\quad\forall w\in X_1\cup X_2
        \end{equation*} 
        for all $(i,j)\in\natint{1,n}\times\natint{1,m}$. It follows that $X_1\cup X_2$ is closed.
    \end{proof}

    By all this it follows that family of all the complements of closed subsets of $\bbm{A}^n$ are a topology over $\bbm{A}^n$.

    \begin{exa}[\textbf{Closed Subsets of $\bbm{A}^1$}]
        Let $X\subseteq\bbm{A}^1$ be a closed subset of $\bbm{A}^1$, then there exists $f_1,\dots,f_m\in k[x]$ (polynomials in one variable) such that:
        \begin{equation*}
            f_i(u)=0,\quad\forall i\in\natint{1,m}\iff u\in X
        \end{equation*}
        Let $d\in [x]$ the highest degree polynomial with leading coefficient one such that:
        \begin{equation*}
            f_i=u_id,\quad\forall i\in\natint{1,n}
        \end{equation*}
        where $u_i\in k[x]$. If $u_1=1$, then $X=\emptyset$ if one of the polynomials is non zero and $X=\bbm{A}^1$ if all of them are equal to zero, otherwise it follows that $X$ is the family of all the roots of $d$, which is finite.

        If $X=\left\{\alpha_1,\dots,\alpha_n \right\}$, then $X$ is closed because $X$ is the family of zeros of the polynomial:
        \begin{equation*}
            f(x)=(x-\alpha_1)\cdots(x-\alpha_n)
        \end{equation*}
    \end{exa}

    \begin{exa}[\textbf{Closed Subsets of $\bbm{A}^2$}]
        %TODO
    \end{exa}

    \begin{mydef}[\textbf{Hypersurface}]
        A set $X\subseteq\bbm{A}^n$ defined by one equation $F(T_1,\dots,T_n)=0$ is called a \textbf{hypersurface}.
    \end{mydef}

    \section{Regular Functions on Closed Subsets}

    \begin{mydef}[Nombre]
        Let $X$ be a closed subset of $\bbm{A}^n$ over the ground field $k$. A function $\cf{f}{X}{k}$ is called \textbf{regular} if there exists a polynomial $F\in k[T]$ such that:
        \begin{equation*}
            f(u)=F(u),\quad\forall u\in X
        \end{equation*}
    \end{mydef}

    %TODO

\end{document}