\documentclass[../algebraic_geometry.tex]{subfiles}

\begin{document}

    \chapter{Basic Notions}

    Throughout this document, we work with a fixed algebraically closed field $k$ (or sometimes denoted $K$), called \textbf{ground field}.

    \section{Algebraic Closed Fields}

    Let $k$ be an algebraic closed field, then for every $f\in k[x]$ (where $k[x]$ denotes the set of all polynomials over the field $k$), we have that if $f$ is non-constant, then $f$ has a root on $k$.

    \begin{exa}
        $\bbm{C}$ is an algebraic closed extension of $\bbm{R}$, and $\bbm{Q}$.
    \end{exa}

    Turns out algebraic closed fields are important due to the following fact:

    \begin{center}
        \textit{There is a one to one correspondence between geometric and algebraic objects using algebraic closed fields}.
    \end{center}

    We will first deal with simple objects and then we'll scalate to more complex in order to generalize certain notions defined in algebraic geometry.

    \begin{mydef}[\textbf{Affine Space}]
        Let $k$ be a field. We denote by $\bbm{A}^n$ the \textbf{$n$-dimensional affine space over the field $k$}, that is:
        \begin{equation*}
            \bbm{A}^n=\left\{(\alpha_1,\dots,\alpha_n)\Big|\alpha_i\in k,\quad\forall i\in\natint{1,n} \right\}
        \end{equation*}
    \end{mydef}

    All the geometric concepts we will be dealing with are within this space, this is due to the following fact:

    \begin{mydef}[\textbf{Closed Subsets of Affine Space}]
        Let $X\subseteq\bbm{A}^n$ be a subset, then $X$ is \textbf{closed} if there exists $f_1,\dots,f_m\in k[x_1,\dots,x_m]$ polynomials over the field $k$ such that:
        \begin{equation*}
            f_i(u)=0,\quad\forall i\in\natint{1,m} \iff u\in X
        \end{equation*}
    \end{mydef}

    From now on, we will write $F(T)$ to denote a polynomial in $n$-variables, allowing $T$ to stand for the set of variables $T_1,\dots,T_n$.

    \begin{obs}[\textbf{Equations of a Set}]
        If a closed subset $X$ consists of all common zeros of polynomials $F_1(T),\dots,F_m(T)$, then we refer to:
        \begin{equation*}
            F_1(T)=\dots=F_n(T)=0        
        \end{equation*}
        as the \textbf{equations of the set $X$}.
    \end{obs}
    
    One really useful definition and fact that are proved in the notes of Algebra Moderna III are the following:

    \begin{mydef}[\textbf{Noetherian Ring}]
        Let $R$ be a ring. We say that $R$ is \textbf{noetherian} if for all $I$ ideal of $R$ there exists $a_1,\dots,a_n\in R$ such that:
        \begin{equation*}
            I=(a_1,\dots,a_n),
        \end{equation*}
        where $(a_1,\dots,a_n)$ denotes the ideal generated by the set $\left\{a_1,\dots,a_n\right\}$.
    \end{mydef}

    \begin{obs}[\textbf{Ideal Generated by a Set $S$}]
        \label{ideal_generated_finite_family}
        Let $R$ be a commutative ring with identity, then the ideal generated by $S\subseteq R$ is:
        \begin{equation*}
            (S)=\left\{\sum_{ i=1}^n r_is_i\Big|r_i\in R,s_i\in S,\forall i\in\natint{1,n};n\in\mathbb{N} \right\}
        \end{equation*}
    \end{obs}

    \begin{theor}[\textbf{Hilbert Basis Theorem}]
        Let $R$ be a noetherian ring, then $R[x]$ is noetherian.
    \end{theor}

    A useful fact about the Hilbert basis theorem is that it generalizes neatly to an arbitrary number of indeterminates:

    \begin{cor}
        Let $R$ be a noetherian ring, then $R[T]$ is noetherian.
    \end{cor}

    This useful fact is fundamental in the proof of the following result:

    \begin{propo}
        Let $X$ be a set defined by an infinite system of equations $\left\{F_\alpha(T)\right\}_{\alpha\in I}$ with $I\neq\emptyset$. Then $X$ is closed.
    \end{propo}

    \begin{proof}
        Let $\mathfrak{U}$ be the ideal generated by the system of equations $\left\{F_\alpha(T)\right\}_{\alpha\in I}$. Since $k[T]$ is noetherian due to the fact that $k$ is a field (in particular, every field is noetherian), then there exists $G_1,\dots,G_m\in k[T]$ such that:
        \begin{equation*}
            \left(\left\{F_\alpha(T)\right\}_{\alpha\in I}\right)=\left(G_1(T),\dots,G_m(T) \right)
        \end{equation*}
        We claim that $u\in X$ iff $G_i(u)=0$, for all $i\in\natint{1,m}$. If $u\in X$, then $F_\alpha(u)=0$ for all $\alpha\in I$, so in particular by Observation (\ref{ideal_generated_finite_family}):
        \begin{equation*}
            F(u)=0,\quad\forall F\in\left(\left\{F_\alpha(T)\right\}_{\alpha\in I}\right)
        \end{equation*}
        which is the same as:
        \begin{equation*}
            F(u)=0,\quad\forall F\in\left(G_1(T),\dots,G_m(T) \right) 
        \end{equation*}
        so $G_i(u)=0$, for all $i\in\natint{1,m}$.
    \end{proof}

    It follows from this proposition that the arbitrary intersection of closed subsets of $\bbm{A}^n$ is closed. Also, it happens that $\emptyset$ and $\bbm{A}^n$ are closed (using the polynomials $F=1$ and $F=0$).

    \begin{propo}
        If $X_1$ and $X_2$ are closed subsets of $\bbm{A}^n$, then $X_1\cup X_2$ is also a closed subset of $\bbm{A}^n$.
    \end{propo}

    \begin{proof}
        Let $F_1,\dots,F_n$ and $G_1,\dots,G_m$ polinomials over the ring $k[T]$ such that:
        \begin{equation*}
            F_i(u)=0,\quad\forall u\in X_1\textup{ and }G_j(v)=0,\quad\forall v\in X_2
        \end{equation*}
        for all $(i,j)\in\natint{1,n}\times\natint{1,m}$. We define $H_{ i,j}\in k[T]$ as:
        \begin{equation*}
            H_{ i,j}=F_i G_j,\quad\forall (i,j)\in\natint{1,n}\times\natint{1,m}
        \end{equation*}
        Then:
        \begin{equation*}
            H_{ i,j}(w)=0,\quad\forall w\in X_1\cup X_2
        \end{equation*} 
        for all $(i,j)\in\natint{1,n}\times\natint{1,m}$. It follows that $X_1\cup X_2$ is closed.
    \end{proof}

    By all this it follows that family of all the complements of closed subsets of $\bbm{A}^n$ are a topology over $\bbm{A}^n$.

    \begin{exa}[\textbf{Closed Subsets of $\bbm{A}^1$}]
        Let $X\subseteq\bbm{A}^1$ be a closed subset of $\bbm{A}^1$, then there exists $f_1,\dots,f_m\in k[x]$ (polynomials in one variable) such that:
        \begin{equation*}
            f_i(u)=0,\quad\forall i\in\natint{1,m}\iff u\in X
        \end{equation*}
        Let $d\in [x]$ the highest degree polynomial with leading coefficient one such that:
        \begin{equation*}
            f_i=u_id,\quad\forall i\in\natint{1,n}
        \end{equation*}
        where $u_i\in k[x]$. If $u_1=1$, then $X=\emptyset$ if one of the polynomials is non zero and $X=\bbm{A}^1$ if all of them are equal to zero, otherwise it follows that $X$ is the family of all the roots of $d$, which is finite.

        If $X=\left\{\alpha_1,\dots,\alpha_n \right\}$, then $X$ is closed because $X$ is the family of zeros of the polynomial:
        \begin{equation*}
            f(x)=(x-\alpha_1)\cdots(x-\alpha_n)
        \end{equation*}
    \end{exa}

    \begin{exa}[\textbf{Closed Subsets of $\bbm{A}^2$}]
        %TODO
    \end{exa}

    \begin{mydef}[\textbf{Hypersurface}]
        A set $X\subseteq\bbm{A}^n$ defined by one equation $F(T_1,\dots,T_n)=0$ is called a \textbf{hypersurface}.
    \end{mydef}

    \section{Regular Functions on Closed Subsets}

    \begin{mydef}[\textbf{Regular Functions}]
        Let $X$ be a closed subset of $\bbm{A}^n$ over the ground field $k$. A function $\cf{f}{X}{k}$ is called \textbf{regular} if there exists a polynomial $F\in k[T]$ such that:
        \begin{equation*}
            f(u)=F(u),\quad\forall u\in X
        \end{equation*}
    \end{mydef}

    In general, there's not a unique polynomial that defines a regular function.

    \begin{obs}
        We can add to $F$ any polynomial entering in the system of equations that defines the set $X$, this doesn't always result in an alteration of $F$.
    \end{obs}

    For the next proposition, we need to remember the definition of an algebra:

    \begin{mydef}[\textbf{$k$-Algebra}]
        A $k$-algebra is a cuatruplet $(A,k,+,\cdot)$ such that:
        \begin{enumerate}[label = \textit{(\arabic*)}]
            \item $(A,+,\cdot)$ is a ring.
            \item $(A,+)$ is a vector space over the field $k$.
            \item The ring multiplication is $k$-bilinear, that is:
            \begin{equation*}
                \alpha(a b)=(\alpha a)b=a(\alpha b),\quad\forall \alpha\in k,a,b\in A
            \end{equation*}
        \end{enumerate}
    \end{mydef}

    We will usually denote a $k$-algebra simply by $A$.

    \begin{propo}[\textbf{Algebra of Regular Functions}]
        Let $X$ be a closed subset of $\bbm{A}^n$ over the ground field $k$. The set of all regular functions over $X$ is a $k$-algebra with the usual addition and multiplication of functions.
        
        The ring obtained is denoted by $k[X]$ and is called the \textbf{ring of regular functions over $X$}.
    \end{propo}

    \begin{proof}
        Let $k[X]$ be the set of all regular functions over $X$. We will prove that $k[X]$ is a $k$-algebra by checking the three conditions of the definition:
        \begin{enumerate}[label = \textit{(\arabic*)}]
            \item \textbf{$(k[X],+,\cdot)$ is a ring}. This is obvious since the sum and product of polynomials is a polynomial.
            \item \textbf{$(k[X],+)$ is a vector space over the field $k$}. This is also obvious since the sum of polynomials and the multiplication of a polynomial by a scalar is a polynomial.
            \item \textbf{The ring multiplication is $k$-bilinear}. This is also immediate from the fact that the multiplication of elements in the field $k$ is commutative.
        \end{enumerate}
        It follows that $k[X]$ is a $k$-algebra.
    \end{proof}

    \begin{mydef}[\textbf{Coordinate Ring}]
        Let $X$ be a closed subset of $\bbm{A}^n$ over the ground field $k$. The $k$-algebra $k[X]$ is called the \textbf{coordinate ring of $X$}.
    \end{mydef}

    We will only deal with the ring structure of the algebra of regular functions, leaving the vector space structure aside (for the moment). We write $k[T]$ for the polynomial ring in $T_1,\dots,T_n$-variables over the field $k$.

    \begin{obs}
        Let $\cf{ap}{k[T]}{k[X]}$ be the function defined as:
        \begin{equation*}
            ap(F)=f,\quad\forall F\in k[T]
        \end{equation*}
        where $\cf{f}{X}{k}$ is defined as $f(u)=F(u)$ for all $u\in X$. It's immediate that $ap$ is a epimorfism of rings.
    \end{obs}

    Due to the latter obsevation and using the first isomorphism theorem, it follows that:
    \begin{equation*}
        k[X]\cong k[T]/\ker(ap)
    \end{equation*}
    where:
    \begin{equation*}
        \ker(ap)=\left\{F\in k[T]\Big|F(u)=0,\quad\forall u\in X \right\}
    \end{equation*}
    is an ideal of $k[T]$.

    \begin{mydef}[\textbf{Ideal of a Closed Set}]
        Let $X$ be a closed subset of $\bbm{A}^n$ over the ground field $k$. The ideal $\ker(ap)$ of $k[T]$ is called the \textbf{ideal of the closed set $X$} and is denoted by $\mathfrak{I}_X$.
    \end{mydef}
    From the above, it follows that:
    \begin{equation*}
        k[X]\cong k[T]/\mathfrak{I}_X
    \end{equation*}
    Thus, the ring is fully determined by the ideal $\mathfrak{I}_X$. We will now focus on studying this ideal in order to understand better the structure of the ring of regular functions over a closed set.

    \begin{exa}
        If $X\subseteq\bbm{A}^n$ is such that $X=\left\{(x_1,\dots,x_n)\right\}$, then:
        \begin{equation*}
            k[X]\cong k[T]/\mathfrak{I}_X
        \end{equation*}
        where $\mathfrak{I}_X$ is the ideal of all polynomials vanishing at the point $x=(x_1,\dots,x_n)$, which is given by:
        \begin{equation*}
            \mathfrak{I}_X=\left\{F\Big|\textup{ where }F(T)=\alpha(T-x);\alpha\in k \right\}
        \end{equation*}
        So, $k[T]/\mathfrak{I}_X\cong k$. It follows that $k[X]\cong k$.
    \end{exa}

    \begin{exa}
        If $X=\bbm{A}^n$, then $\mathfrak{I}_X=\left\{0\right\}$, where $0$ is the $0$ polynomial, so $k[X]\cong k[T]$.
    \end{exa}

    \begin{exa}
        Let $X\subseteq\bbm{A}^2$ be given by the equation:
        \begin{equation*}
            X=\left\{(x_1,x_2)\in\bbm{A}^2\Big|x_1x_2-1=0\right\}
        \end{equation*}
        Then, the ideal $\mathfrak{I}_X$ is given by:
        \begin{equation*}
            \mathfrak{I}_X=\left\{F\in k[T_1,T_2]\Big|F(T_1,T_2)=(T_1T_2-1)G(T_1,T_2);G\in k[T_1,T_2] \right\}
        \end{equation*}
        It is not so difficult to show that there is a ring isomorphism between $k[X]$ and $k[T_1,T_1^{-1}]$, the ring of Laurent polynomials in one variable over the field $k$.
    \end{exa}

    \begin{obs}
        Let $R$ be a ring and $I$ and ideal of $R$. We know that there is a correspondence between the ideals of $R/I$ and the ideals of $R$ that contain $I$. This correspondence is given by:
        \begin{equation*}
            J\mapsto J/I,\textup{ were $J$ is an ideal of $R$ such that }I\subseteq J
        \end{equation*}
        Since $k[T]$ is Noetherian, it follows that $k[X]$ is also Noetherian, no matter the closed subset $X$ of $\bbm{A}^n$.
    \end{obs}

    And furthermore, we have that $k[X]$ satisfies the following analogue of the Nullstellensatz Theorem. Before that, we shall give the following definition:

    \begin{mydef}[\textbf{Radical Ideal}]
        Let $R$ be a ring and $I$ an ideal of $R$. The radical of the ideal $I$, denoted by $\rad{I}$ or $\sqrt{I}$, is given by:
        \begin{equation*}
            \rad{I}=\left\{a\in R\Big|a^n\in I,\textup{ for some }n\in\bbm{N}\right\}
        \end{equation*}  
    \end{mydef}

    \begin{propo}
        Let $X$ be a closed subset of $\bbm{A}^n$ over the ground field $k$. Then, if $f,g_1,\dots,g_m \in k[X]$ are functions such that $f$ vanishes at all points where $g_1,\dots,g_m$ vanish, then there exists $r\in\bbm{N}$ such that:
        \begin{equation*}
            f^r\in (g_1,\dots,g_m)
        \end{equation*}
        where $(g_1,\dots,g_m)$ denotes the ideal of $k[X]$ generated by the set $\left\{g_1,\dots,g_m \right\}$.
    \end{propo}

    \begin{proof}
        Let $F$ and $G_1,\dots,G_m\in k[T]$ be polynomials such that:
        \begin{equation*}
            i(F)=f\textup{ and }i(G_j)=g_j,\quad\forall j\in\natint{1,m}
        \end{equation*}
        where $\cf{i}{k[T]/\mathfrak{I}_X}{k[X]}$ is given by:
        \begin{equation*}
            i(H+\mathfrak{I}_X)=h,\quad\forall H\in k[T]
        \end{equation*}
        being $\cf{h}{X}{k}$ the function defined as $h(u)=H(u)$ for all $u\in X$.

        Let $F_1,\dots, F_l$ be the equations of $X$, that is:
        \begin{equation*}
            F_i(u)=0,\quad\forall u\in X,\quad\forall i\in\natint{1,l}
        \end{equation*}
        If $u\in X$ is such that $G_j(u)=0$ for all $j\in\natint{1,m}$, then $F_i(u)=0$ for all $i\in\natint{1,l}$ and $F(u)=0$. It follows from the Nullstellensatz Theorem that there exists $r\in\bbm{N}$ such that:
        \begin{equation*}
            F^r\in (G_1,\dots,G_m,F_1,\dots,F_l)
        \end{equation*}
        Therefore, since $F_1,\dots,F_l\in\mathfrak{I}_X$, it follows that $f^r\in\left(g_1,\dots,g_m\right)$.
    \end{proof}

    \begin{preg}
        What's the relation between the ideal $\mathfrak{I}_X$ of a closed set $X$ and a system $F_1=\dots=F_m=0$ of defining equations of $X$?
    \end{preg}

    \begin{sol}
        Clearly, by definition we have that $F_i\in\mathfrak{I}_X$, for all $i\in\natint{1,m}$. Therefore, the ideal generated by the set $\left\{F_1,\dots,F_m \right\}$ is contained in $\mathfrak{I}_X$.

        However, it is not always true that $\mathfrak{I}_X$ is equal to the ideal generated by the set $\left\{F_1,\dots,F_m \right\}$.
    \end{sol}

    \begin{exa}
        Let $X\subseteq\bbm{A}^2$ be given by the equation:
        \begin{equation*}
            T^2=0
        \end{equation*}
        Then, $X=\left\{0\right\}$, so $\mathfrak{I}_X$ consists of all the polynomials vanishing at $0$, that is, all polynomials without constant term. However, the ideal generated by the polynomial $T^2$ doesn't contain the polynomial $T$, so they are not equal.
    \end{exa}

    However, we can always define the same closed set $X$ with a system of equations $G_1=\dots=G_l=0$ whose generated ideal is equal to $\mathfrak{I}_X$. This follows from the Hilbert's Basis Theorem, since $\mathfrak{I}_X$ is finitely generated.

    \begin{obs}[\textbf{Generating System of Equations}]
        Let $X$ be a closed subset of $\bbm{A}^n$, and let $G_1,\dots,G_m\in k[T]$ be polynomials such that:
        \begin{equation*}
            \mathfrak{I}_X=(G_1,\dots,G_m)
        \end{equation*}
        then, for all $F\in\mathcal{I}_X$ we have that there exists $H_1,\dots,H_m\in k[T]$ such that:
        \begin{equation*}
            F=H_1G_1+\dots+H_mG_m
        \end{equation*}
        Since $G_i\in\mathfrak{I}_X$ it follows that $X$ is defined by the system of equations $G_1=\dots=G_m=0$.

        Therefore, we can always find a system of equations defining $X$ whose generated ideal is equal to $\mathfrak{I}_X$.
    \end{obs}

    \begin{idea}
        It is sometimes even convenient to consider a closed set as defined by the infinite system of equations $F = 0$ for all polynomials $F\in\mathfrak{I}_X$ . Indeed, if $(F1 , . . . , Fm ) = \mathfrak{I}_X$ then these equations are all consequences of $F_1=\dots=F_m=0$.
    \end{idea}

    It turns out that relations between closed sets can be translated to relations between their corresponding ideals.

    \begin{exa}
        Let $X$ and $Y$ be closed subsets of $\bbm{A}^n$. Then, $Y\subseteq X$ if and only if $\mathfrak{I}_X\subseteq\mathfrak{I}_Y$.
    \end{exa}

    This follows directly from the definitions. The latter example allows us to asociate to every closed subset of a closed set $X$ of the afine space $\bbm{A}^n$ an ideal of the coordinate ring $k[X]$.
        
    Indeed, if $Y\subseteq X$ is a closed subset of $X$, then we can asociate to $Y$ the ideal $\mathfrak{a}_Y=\mathfrak{I}_Y/\mathfrak{I}_X$ of the ring $k[X]=k[T]/\mathfrak{I}_X$ (using correspondence theorem).

    Conversely, if $\mathfrak{a}$ is an ideal of $k[X]$, then we can asociate to $\mathfrak{a}$ the closed subset $Y$ of $X$ defined by the equations $F=0$ for all polynomials $F\in\mathfrak{a}$.

    \begin{obs}
        $Y=\emptyset$ if and only if $\mathfrak{I}_Y=k[X]$
    \end{obs}

    Something interesting happens when we consider isolated points of $X$. Indeed, if $x\in X$ is an isolated point of $X$, then the ideal $\mathfrak{m}_x=\mathfrak{I}_{\left\{x\right\}}/\mathfrak{I}_X$ has to be a maximal ideal of $k[X]$.

    \begin{obs}
        By definition, this ideal is the kernel of the homomorphism $\cf{ap}{k[T]}{k[X]}$, so:
        \begin{equation*}
            \mathfrak{m}_x=\left\{F\in k[T]\Big|F(x)=0 \right\}
        \end{equation*}
        Since $k$ is an algebraically closed field, then:
        \begin{equation*}
            \mathfrak{m}_x=\left\{\alpha(x-T)\Big|\alpha\in k \right\}
        \end{equation*}
        So, $k[T]/\mathfrak{m}_x=k$. It follows that $\mathfrak{m}_x$ has to be a maximal ideal of $k[X]$.
    \end{obs}

    We can do the converse too, and to each maximal ideal of $k[X]$ there is an isolated point of $X$ associated.

    \begin{mydef}[\textbf{Hypersurface}]
        Let $X$ be a closed subset of $\bbm{A}^n$ over the ground field $k$. We say that $X$ is an \textbf{hypersurface} if there exists $F\in k[T]$ such that:
        \begin{equation*}
            X=\left\{u\in\bbm{A}^n\Big|F(u)=0 \right\}
        \end{equation*}
    \end{mydef}

    \section{Regular Maps}

    From now on, $X$ will denote a closed subset of $\bbm{A}^n$ and $Y$ a closed subset of $\bbm{A}^m$.

    \begin{mydef}[\textbf{Regular Maps}]
        A map $\cf{f}{X}{Y}$ is called \textbf{regular} if there exists $f_1,\dots,f_m\in X$ regular functions on $X$ such that:
        \begin{equation*}
            f(x)=(f_1(x),\dots,f_m(x)),\quad\forall x\in X
        \end{equation*}
    \end{mydef}

    Any regular map $\cf{f}{X}{\bbm{A}^m}$ is always given by $m$-functions $f_1,\dots,f_m\in k[X]$.

    \begin{obs}
        In order to know that this maps into the closed subset $Y\subseteq\bbm{A}^n$, we need to check that:
        \begin{equation*}
            G(f_1(x),\dots,f_m(x))=0,\quad\forall x\in X
        \end{equation*}
        for all $G\in\mathfrak{I}_Y$.
    \end{obs}

    \begin{exa}
        A regular function is the same thing as a regular map $\cf{f}{X}{\bbm{A}^1=k}$.
    \end{exa}

    \begin{exa}
        A linear map $\cf{L}{\bbm{A}^n}{\bbm{A}^m}$ is a regular map.
    \end{exa}

    \begin{proof}
        It suffices to show that the components of $L$, let say $L=(L_1,\dots,L_m)$ are regular functions. Indeed, since $L$ is linear, it follows that $L_j$ is $k$-linear for all $j\in\natint{1,m}$, so there exists $\alpha_1,\dots,\alpha_m\in\bbm{A}^n$ such that:
        \begin{equation*}
            L_j(x)=\alpha_j\cdot x,\quad\forall x\in\bbm{A}^n
        \end{equation*}
        where $\alpha_j\cdot x$ denotes the usual dot product. Since each $L_j$ is a polynomial of degree 1, it follows that $L_j$ is a polynomial function, so it's regular.
    \end{proof}

    \begin{exa}
        The projection map $(x,y)\mapsto x$ defines a regular map of the curve defined by $xy=1$ to $\bbm{A}^1$.
    \end{exa}

    \begin{exa}
        The map $f(t)=(t^2,t^3)$ is a regular map of the line $\bbm{A}^1$ to the curve defined by $x^3-y^2=0$ in $\bbm{A}^2$.
    \end{exa}

    \subsection{The Zeta Function of a Variety over $\bbm{F}_p$}

    This example is very important to number theorists, since it relates algebraic geometry with number theory in a very deep way.

    Let $p$ be a prime number and $\bbm{F}_p$ the finite field with $p$ elements. Let $X$ be a closed subset of $\bbm{A}^n$ defined over $\bbm{F}_p$, that is, the polynomials defining $X$ have coefficients in $\bbm{F}_p$.

    \begin{obs}
        Let's consider the closed set $X$ and $F_1=\dots=F_m$ the polynomials defining $X$. If $x\in X$ then:
        \begin{equation*}
            F_i(x)=0,\quad\forall i\in\natint{1,m}
        \end{equation*}
        So it follows that: %TODO: Congruences mod p
    \end{obs}
    Let's consider the map $\cf{\varphi}{\bbm{A}^n}{\bbm{A}^n}$ given by:
    \begin{equation*}
        \varphi(x_1,\dots,x_n)=(x_1^p,\dots,x_n^p)
    \end{equation*}
    It's immediate to see that $\varphi$ is a regular map. Furthermore, if $x\in X$, then $\varphi(x)\in X$. Let $x\in X$, so:
    \begin{equation*}
        F_i(x)=0,\quad\forall i\in\natint{1,m}
    \end{equation*}

    \newpage 

    \section{Excersies}

    \begin{excer}
        If $X$ and $Y$ are closed subsets of $\bbm{A}^n$ and $\bbm{A}^m$, then $k[X\times Y]$ is isomorphic to $k[X]\otimes_k k[Y]$, where:
        \begin{equation*}
            k[X]\otimes_k k[Y]=\left\{f\otimes_k g\Big|f\in k[X]\textup{ and }g\in k[Y] \right\}
        \end{equation*}
        where $\cf{f\otimes_kg}{X\times Y}{k}$ is defined as:
        \begin{equation*}
            f\otimes_kg(x,y)=f(x)g(y),\quad\forall (x,y)\in X\times Y
        \end{equation*}
    \end{excer}

\end{document}