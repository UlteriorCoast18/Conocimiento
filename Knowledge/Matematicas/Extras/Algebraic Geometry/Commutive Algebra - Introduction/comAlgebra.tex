\documentclass[12pt]{report}
\usepackage[spanish]{babel}
\usepackage[utf8]{inputenc}
\usepackage{amsmath}
\usepackage{amssymb}
\usepackage{amsthm}
\usepackage{graphics}
\usepackage{subfigure}
\usepackage{lipsum}
\usepackage{array}
\usepackage{multicol}
\usepackage{enumerate}
\usepackage[framemethod=TikZ]{mdframed}
\usepackage[a4paper, margin = 1.5cm]{geometry}
\usepackage{enumitem}
\usepackage{tikz}
\usepackage{pgffor}
\usepackage{ifthen}
\usepackage{bbm}
\usepackage{hyperref}

\usetikzlibrary{shapes.multipart}

\newcounter{it}
\newcommand*\watermarktext[1]{\begin{tabular}{c}
    \setcounter{it}{1}%
    \whiledo{\theit<10}{%
    \foreach \col in {0,...,4}{#1\ \ } \\ \\ \\ \\
    \stepcounter{it}%
    }
    \end{tabular}
    }

\AddToHook{shipout/foreground}{
    \begin{tikzpicture}[remember picture,overlay, every text node part/.style={align=center}]
        \node[rectangle,black,rotate=30,scale=2,opacity=0.08] at (current page.center) {\watermarktext{Cristo Daniel Alvarado ESFM\quad}}; 
  \end{tikzpicture}
}

%En esta parte se hacen redefiniciones de algunos comandos para que resulte agradable el verlos%

\def\proof{\paragraph{Demostración:\\}}
\def\endproof{\hfill$\blacksquare$}

\def\sol{\paragraph{Solución:\\}}
\def\endsol{\hfill$\square$}

%En esta parte se definen los comandos a usar dentro del documento para enlistar%

\newtheoremstyle{largebreak}
  {}% use the default space above
  {}% use the default space below
  {\normalfont}% body font
  {}% indent (0pt)
  {\bfseries}% header font
  {}% punctuation
  {\newline}% break after header
  {}% header spec

\theoremstyle{largebreak}

\newmdtheoremenv[
    leftmargin=0em,
    rightmargin=0em,
    innertopmargin=0pt,
    innerbottommargin=5pt,
    hidealllines = true,
    roundcorner = 5pt,
    backgroundcolor = gray!60!red!30
]{exa}{Ejemplo}[section]

\newmdtheoremenv[
    leftmargin=0em,
    rightmargin=0em,
    innertopmargin=0pt,
    innerbottommargin=5pt,
    hidealllines = true,
    roundcorner = 5pt,
    backgroundcolor = gray!50!blue!30
]{obs}{Observación}[section]

\newmdtheoremenv[
    leftmargin=0em,
    rightmargin=0em,
    innertopmargin=0pt,
    innerbottommargin=5pt,
    rightline = false,
    leftline = false
]{theor}{Teorema}[section]

\newmdtheoremenv[
    leftmargin=0em,
    rightmargin=0em,
    innertopmargin=0pt,
    innerbottommargin=5pt,
    rightline = false,
    leftline = false
]{propo}{Proposición}[section]

\newmdtheoremenv[
    leftmargin=0em,
    rightmargin=0em,
    innertopmargin=0pt,
    innerbottommargin=5pt,
    rightline = false,
    leftline = false
]{cor}{Corolario}[section]

\newmdtheoremenv[
    leftmargin=0em,
    rightmargin=0em,
    innertopmargin=0pt,
    innerbottommargin=5pt,
    rightline = false,
    leftline = false
]{lema}{Lema}[section]

\newmdtheoremenv[
    leftmargin=0em,
    rightmargin=0em,
    innertopmargin=0pt,
    innerbottommargin=5pt,
    roundcorner=5pt,
    backgroundcolor = gray!30,
    hidealllines = true
]{mydef}{Definición}[section]

\newmdtheoremenv[
    leftmargin=0em,
    rightmargin=0em,
    innertopmargin=0pt,
    innerbottommargin=5pt,
    roundcorner=5pt
]{excer}{Ejercicio}[section]

%En esta parte se colocan comandos que definen la forma en la que se van a escribir ciertas funciones%

\newcommand\abs[1]{\ensuremath{\left|#1\right|}}
\newcommand\divides{\ensuremath{\bigm|}}
\newcommand\cf[3]{\ensuremath{#1:#2\rightarrow#3}}
\newcommand\contradiction{\ensuremath{\#_c}}
\newcommand\natint[1]{\ensuremath{\left[\!\left[ #1\right]\!\right]}}
\newcommand{\gen}[1]{\ensuremath{\langle#1\rangle}}
\newcommand{\bbm}[1]{\mathbbm{#1}}

\begin{document}
    \setlength{\parskip}{5pt} % Añade 5 puntos de espacio entre párrafos
    \setlength{\parindent}{12pt} % Pone la sangría como me gusta
    \title{Notas Introduction to Commutative Algebra}
    \author{Cristo Daniel Alvarado}
    \maketitle

    \tableofcontents %Con este comando se genera el índice general del libro%

    \chapter{Anillos e Ideales}

    Muchos de los resultados que se usarán se han visto en el curso de Álgebra Moderna II, por lo que solo se incluirán resultados nuevos.

    A lo largo de todo el documento, todo anillo será un anillo conmutativo con identidad.

  \section{Nilradical y Radical de Jacobson}

    \begin{mydef}
        Sea $A$ un anillo. Un elemento $x\in A$ es llamado \textbf{nilpotente} si $x^n=0$ para algún $n\in\mathbb{N}$
    \end{mydef}

    \begin{obs}
        Todo elemento nilpotente es divisor de cero, sin embargo el converso no es cierto.
    \end{obs}

    \begin{propo}
        El conjunto $\mathfrak{N}$ de todos los elementos nilpotentes de un anillo $A$ es un ideal, y el ideal cociente $A/\mathfrak{N}$ no tiene elmentos nilpotentes distintos de cero.
    \end{propo}

    \begin{proof}
        Veamos que $\mathfrak{N}$ es un ideal.
        \begin{enumerate}[label = \textit{(\arabic*)}]
            \item Sean $x,y\in\mathfrak{N}$, entonces existen $m,n\in\mathbb{N}$ tales que $x^n=y^m=0$. Por ende:
            \begin{equation*}
                (x+y)^{ n+m}=0
            \end{equation*}
            pues, en el desarrollo binomial de esta expresión, todo término es de la forma $c_{(r,s)}x^ry^s$ con $c_{(r,s)}\in\mathbb{N}$ el cual además cumple que
            \begin{equation*}
                r+s=n+m
            \end{equation*}
            con $r,s\in\natint{0,n+m}$. Si $r<n$ entonces debe suceder que $s>m$, luego $y^s=0$. Si $r>n$ se sigue que $x^r=0$. En cualquier caso, todos los coeficientes de la forma $x^ry^s=0$, lo cual prueba lo enunciado.
            \item Sea $x\in\mathfrak{N}$, entonces existe $n\in\mathbb{N}$ tal que $x^n=0$, entonces:
            \begin{equation*}
                (ax)^n=a^nx^n=0
            \end{equation*}
            por lo que $ax\in\mathfrak{N}$.
        \end{enumerate}
        Por los dos incisos anteriores se sigue que $\mathfrak{N}$ es un ideal de $A$. Sea $\mathfrak{N}+x\in A/\mathfrak{N}$ con $x\in A$ tal que
        \begin{equation*}
            (\mathfrak{N}+x)^n=\mathfrak{N}
        \end{equation*}
        para algún $n\in\mathbb{N}$, entonces:
        \begin{equation*}
            \mathfrak{N}+x^n=\mathfrak{N}\Rightarrow x^n\in\mathfrak{N}
        \end{equation*}
        luego existe $k\in\mathbb{N}$ tal que $(x^n)^k=0$, esto es que $x\in\mathfrak{N}$, por lo que
        \begin{equation*}
            \mathfrak{N}+x=\mathfrak{N}
        \end{equation*}
    \end{proof}

    \begin{mydef}
        El ideal de la proposición anterior es llamado el \textbf{nilradical de $A$} cuando se trabaje con varios anillos, será denotado por $\mathfrak{N}_A$.
    \end{mydef}

    Resulta que podemos caracterizar de otra manera al nilradical $\mathfrak{N}$:

    \begin{propo}
        El nilradical $\mathfrak{N}$ de $A$ es la intersección de todos los ideales primos de $A$.
    \end{propo}

    \begin{proof}
        Sea $\mathfrak{N}'$ la intersección de todos los ideales primos de $A$. Se tiene que este es un ideal de $A$.
        \begin{itemize}
            \item Si $x\in A$ es tal que $x\in\mathfrak{N}$, entonces existe $n\in\mathbb{N}$ tal que $x^n=0$. Como $0\in\mathfrak{N}'$ y $\mathfrak{N}'$, entonces
            \begin{equation*}
                x^n\in\mathfrak{p}
            \end{equation*}
            para todo ideal primo $\mathfrak{p}$ de $A$, luego por inducción se tiene que $x\in\mathfrak{p}$, es decir que $x\in\mathfrak{N}'$.
            \item Sea $x\in A$ tal que $x\notin\mathfrak{N}$, probaremos que $x\notin\mathfrak{N}'$. Sea
            \begin{equation*}
                \Sigma=\left\{\mathfrak{a}\Big|\mathfrak{a}\textup{ es ideal de $A$ tal que }n\in\mathbb{N}\Rightarrow x^n\notin\mathfrak{a} \right\}
            \end{equation*}
            el conjunto $\Sigma$ es no vacío, pues $\langle 0\rangle\in\Sigma$. Sea $\mathcal{C}$ una cadena de elementos de $\Sigma$. Como
            \begin{equation*}
                \mathcal{C}=\left\{I_n \right\}_{ n=1}^\infty
            \end{equation*}
            es tal que $I_1\subseteq I_2\subseteq\cdots\subseteq I_m\subseteq\cdots$, se sigue de una proposición que
            \begin{equation*}
                I=\bigcup_{ n=1}^\infty I_n
            \end{equation*} 
            es un ideal de $A$, el cual debe estar en $\Sigma$. Por el Lema de Zorn, este conjunto tiene elementos maximales, digamos $\mathfrak{p}$. Es claro que $x^n\notin\mathfrak{p}$ para todo $n\in\mathbb{N}$. Veamos que $\mathfrak{p}$ es primo.

            En efecto, sean $y,z\notin\mathfrak{p}$, entonces los ideales
            \begin{equation*}
                \mathfrak{p}+\langle y\rangle\textup{ y }\mathfrak{p}+\langle z\rangle
            \end{equation*}
            son dos ideales de $A$ que contienen propiamente a $\mathfrak{p}$, por lo que $x\in\mathfrak{p}+\langle y\rangle$ y $x\in\mathfrak{p}+\langle z\rangle$, luego existen $m,n\in\mathbb{N}$ tales que
            \begin{equation*}
                x^n\in\mathfrak{p}+\langle y\rangle,\quad\textup{y}\quad x^m\in\mathfrak{p}+\langle z\rangle
            \end{equation*}
            por ende,
            \begin{equation*}
                x^{ n+m}\in\left(\mathfrak{p}+\langle y\rangle \right)\left(\mathfrak{p}+\langle z\rangle \right)=\mathfrak{p}+\langle yz\rangle
            \end{equation*}
            por tanto, $\mathfrak{p}+\langle yz\rangle$ contiene propiamente a $\mathfrak{p}$, luego no puede estar en $\Sigma$, así que $yz\notin\mathfrak{p}$. Se sigue entonces que $\mathfrak{p}$ es primo. Por tanto, $x\notin\mathfrak{N}'$.
        \end{itemize}
        Por los dos incisos anteriores se sigue que $\mathfrak{N}=\mathfrak{N}'$.
    \end{proof}

    \begin{mydef}
        Sea $A$ un anillo, el \textbf{radical de Jacobson $\mathfrak{R}$ de $A$} se define como la intersección de todos los ideales maximales de $A$. Cuando se trabaje con varios anillos, será denotado por $\mathfrak{R}_A$.
    \end{mydef}

    El radical de Jacobson (que claramente es un ideal), se caracteriza de la siguiente manera:

    \begin{propo}
        Sea $A$ un anillo. Entonces, $x\in\mathfrak{R}$ si y sólo si $1-xy$ es una unidad en $A$ para todo $y\in A$.
    \end{propo}

    \begin{proof}
        $\Rightarrow):$ Suponga que existe $y\in A$ tal que $1-xy$ no es una unidad de $A$, entonces existe un ideal maximal que contiene a $1-xy$, digamos $\mathfrak{m}$, pero $x\in\mathcal{R}$, en particular $x\in\mathfrak{m}$ por lo que $xy\in\mathfrak{m}$ lo cual implica que $1\in\mathfrak{m}$, lo cual no puede suceder pues $\mathfrak{m}$ es ideal maximal.

        $\Leftarrow):$ Suponga que existe un ideal maximal $\mathfrak{m}$ tal que $x\notin\mathfrak{m}$. Entonces,
        \begin{equation*}
            \mathfrak{m}+\langle x\rangle=\langle\mathfrak{m}+x\rangle=A=\langle 1\rangle
        \end{equation*}
        es decir, existe $u\in\mathfrak{m}$ y $y\in A$ tales que
        \begin{equation*}
            u+xy=1
        \end{equation*}
        luego, $1-xy$ no puede ser unidad de $A$.
    \end{proof}

    \begin{mydef}
        Sea $A$ anillo y $\mathfrak{a}$ un ideal de $A$. Se define el \textbf{radical de $\mathfrak{a}$} como el conjunto
        \begin{equation*}
            r(\mathfrak{a})=\left\{x\in A\Big|x^n\in\mathfrak{a}\textup{ para algún }n\in\mathbb{N} \right\}
        \end{equation*}
    \end{mydef}

    \begin{propo}
        Dado un anillo $A$ un ideal $\mathfrak{a}$ de $A$, se tiene que $r(\mathfrak{a})$ es un ideal de $A$.
    \end{propo}

    \begin{proof}
        Considere el homomorfismo natural $\cf{\pi}{A}{A/\mathfrak{a}}$, afirmamos que
        \begin{equation*}
            r(A)=\pi^{-1}\left(\mathfrak{N}_{ A/\mathfrak{a}} \right)
        \end{equation*}
        donde $\mathfrak{N}_{ A/\mathfrak{a}}$ es el nilradical de $A/\mathfrak{a}$. En efecto, veamos que:
        $x\in r(\mathfrak{a})$ si y sólo si existe $n\in\mathbb{N}$ tal que $x^n\in\mathfrak{a}$, lo cual ocurre si y sólo si
        \begin{equation*}
            \left(\mathfrak{a}+x\right)^n=\mathfrak{a}+x^n=\mathfrak{a}
        \end{equation*}
        si y sólo si $\mathfrak{a}+x$ es un elemento nilpotente de $A/\mathfrak{a}$, si y sólo si $\mathfrak{a}+x\in\mathfrak{N}_{ A/\mathfrak{a}}$, si y sólo si $x\in\pi^{-1}\left(\mathfrak{N}_{ A/\mathfrak{a}} \right)$.

        Lo anterior prueba la igualdad.
    \end{proof}

    \begin{excer}
        Sea $A$ un anillo y $\mathfrak{a},\mathfrak{b}$ ideales de $A$. Se cumple lo siguiente:
        \begin{enumerate}[label = \textit{(\arabic*)}]
            \item $\mathfrak{a}\subseteq r(\mathfrak{a})$.
            \item $r(r(a))=r(\mathfrak{a})$.
            \item $r(\mathfrak{a}\mathfrak{b})=r(\mathfrak{a}\cap\mathfrak{b})=r(\mathfrak{a})\cap r(\mathfrak{b})$.
            \item $r(\mathfrak{a})=\gen{1}$ si y sólo si $\mathfrak{a}=\gen{1}$.
            \item $r(\mathfrak{a}+\mathfrak{b})=r(r(\mathfrak{a})+r(\mathfrak{b}))$.
            \item Si $\mathfrak{p}$ es un ideal primo de $A$, entonces $r(\mathfrak{p}^n)=\mathfrak{p}$ para todo $n>0$.
        \end{enumerate}
    \end{excer}

    \begin{proof}
        De \textit{(1)}: Tomemos $x\in\mathfrak{a}$, entonces $x^1\in\mathfrak{a}$, así que $x\in r(\mathfrak{a})$.

        De \textit{(2)}: Por el inciso anterior ya se tiene que $r(r(\mathfrak{a}))\subseteq r(\mathfrak{a})$. Ahora, si $x\in r(\mathfrak{a})$, entonces existe $1\in\bbm{N}$ tal que $x^1\in r(\mathfrak{a})$, así que $x\in r(r(\mathfrak{a}))$. Se sigue así la igualdad.

        De \textit{(3)}: Haremos la demostración por partes:
        \begin{itemize}
            \item Primero, probaremos que $r(\mathfrak{a}\mathfrak{b})=r(\mathfrak{a}\cap\mathfrak{b})$. En efecto, sea $x\in r(\mathfrak{a}\mathfrak{b})$, entonces existe $n\in\mathbb{N}$ tal que:
            \begin{equation*}
                x^n\in\mathfrak{a}\mathfrak{b}
            \end{equation*}
            Por ser $\mathfrak{a}$ y $\mathfrak{b}$ ideales, se sigue que $x^n\in\mathfrak{a}$ y $x^n\in\mathfrak{b}$, por lo cual $x\in r(\mathfrak{a})\cap r(\mathfrak{b})$ y $x\in r(\mathfrak{a}\cap\mathfrak{b})$.
            \item Si $x\in r(\mathfrak{a})\cap r(\mathfrak{b})$, entonces existen $n,m\in\bbm{N}$ tales que $x^n\in\mathfrak{a}$ y $x^m\in\mathfrak{b}$, luego $x^{ nm}\in \mathfrak{ab}$, lo cual implica que $x\in r(\mathfrak{ab})$.
            
            Si $x\in r(\mathfrak{a}\cap\mathfrak{b})$, entonces existe $n\in\bbm{N}$ tal que $x^n\in \mathfrak{a}\cap\mathfrak{b}$, así que $x^{2n}\in\mathfrak{ab}$, esto es que $x\in r(\mathfrak{ab})$.
        \end{itemize}
        Por ambos casos se sigue la doble igualdad.

        De \textit{(6)}: Sea $\mathfrak{p}$ un ideal primo de $A$. Procederemos por inducción sobre $n$.
        \begin{itemize}
            \item Por \textit{(1)} se tiene que $\mathfrak{p}\subseteq r(\mathfrak{p})$. Sea $x\in r(\mathfrak{p})$, entonces existe $m\in\bbm{N}$ tal que $x^m\in\mathfrak{p}$. Esto implica que $x\in\mathfrak{p}$ (tal hecho se verifica usando inducción). De esta forma se sigue la igualdad.
            \item Supongamos que lo anterior se cumple para algún $n\in\bbm{N}$. Sea $x\in r(\mathfrak{p}^{ n+1})$, entonces existe $m\in\mathbb{N}$ tal que $x^m\in\mathfrak{p}^{ n+1}\subseteq\mathfrak{p}$, así que $x\in\mathfrak{p}$.
        \end{itemize}
        %TODO: no se uso induccion, corregir
        De ambos incisos se sigue la igualdad.
    \end{proof}

    \begin{obs}
        Si $I$ y $J$ son ideales, entonces $IJ$ es un ideal, siendo este último definido por:
        \begin{equation*}
            IJ=\left\{\sum_{ k=1}^n i_kj_k\Big|i_k\in I\textup{ and } j_k\in J,\forall k\in\natint{1,n};n\in\bbm{N} \right\}
        \end{equation*}
    \end{obs}
    
    \begin{propo}
        El radical de un ideal $\mathfrak{a}$ es la intersección de todos ideales primos que contienen a $\mathfrak{a}$.
    \end{propo}

    \begin{proof}
        Considere $A/\mathfrak{a}$.%TODO
    \end{proof}

    \newpage

    \section{Ejercicios}

    \begin{propo}
        Todo ideal maximal $\mathfrak{m}$ de $A$ es ideal primo de $A$.
    \end{propo}

    \begin{proof}
        En efecto, se tiene que $A/\mathfrak{m}$ es campo, en particular es dominio entero (por no tener divisores de cero), luego $\mathfrak{m}$ es ideal primo.
    \end{proof}

    \begin{excer}
        En el anillo $A[x]$, el radical de Jacobson es igual al nilradical.
    \end{excer}

    \begin{proof}
        Como todo ideal maximal es un ideal primo, se tiene la contención:
        \begin{equation*}
            \mathfrak{N}\subseteq\mathfrak{R}
        \end{equation*}
        sea ahora $f(x)\in \mathfrak{R}$ se tiene que
        \begin{equation*}
            1-f(x)g(x) \textup{ es unidad de $A[x]$ para todo }g(x)\in A[x]
        \end{equation*}
        en particular, $1-xf(x)$ es unidad de $A[x]$, luego si $f(x)=a_nx^n+a_{ n-1}x^{ n-1}+\cdots+a_0$. Debe suceder entonces que los coeficientes de:
        \begin{equation*}
            1-xf(x)=-a_nx^{n+1}-a_{ n-1}x^{ n}-\cdots-a_0x+1
        \end{equation*}
        sean tales que $a_i$ es nilpotente para todo $i\in\natint{0,n}$, luego $f(x)$ es elemento nilpotente de $A[x]$.

        Se sigue entonces la igualdad.
    \end{proof}

    \begin{excer}
        Sea $A$ un anillo y sea $A[[x]]$ el anillo de series de potencias formales con coeficientes en $A$. Pruebe que:
        \begin{enumerate}[label =\textit{\arabic*}]
            \item $f$ es unidad de ...%TODO
        \end{enumerate}
    \end{excer}
    
    \begin{proof}
        
    \end{proof}

    \begin{excer}
        
    \end{excer}

    \begin{proof}
        
    \end{proof}

    \begin{excer}
        Sea $A$ un anillo tal que para todo $x\in A$ existe $n\in\mathbb{N}$, $n>1$ tal que $x^n=x$. Pruebe que todo ideal primo de $A$ es maximal.
    \end{excer}

    \begin{proof}
        Sea $\mathfrak{p}$ un ideal propio primo de $A$. Probaremos que $A/\mathfrak{p}$ es campo. En efecto, no tiene divisores de cero, pues si
        \begin{equation*}
            \mathfrak{p}+xy=\left(\mathfrak{p}+x\right)\left(\mathfrak{p}+y\right)=\mathfrak{p}
        \end{equation*}
        con $x,y\notin\mathfrak{p}$, entonces $xy\in\mathfrak{p}$\contradiction. Por tanto, no tiene divisores de cero. Hay que ver que todo elemento no cero es invertible. Sea $x\in A\setminus\mathfrak{p}$. Se tiene que existe $n\in\mathbb{N}$ con $n>1$ tal que
        \begin{equation*}
            \begin{split}
                \mathfrak{p}+x^{n}&=\mathfrak{p}+x\\
                \Rightarrow \left(\mathfrak{p}+x\right)\left(\left(\mathfrak{p}+x^{ n-1}\right)-\left(\mathfrak{p}+1\right) \right)&=\mathfrak{p}\\
            \end{split}
        \end{equation*}
        como no hay divisores de cero, debe suceder que
        \begin{equation*}
            \mathfrak{p}+x^{ n-1}=\mathfrak{p}+1
        \end{equation*}
        por ende, al ser $n>1$, se tiene que $n-1>0$, así que:
        \begin{equation*}
            \left(\mathfrak{p}+x\right)\left(\mathfrak{p}+x^{ n-2}\right)=\mathfrak{p}+1
        \end{equation*}
        con $n-2\geq 0$. Luego $\mathfrak{p}+x$ es invertible. Así que $A/\mathfrak{p}$ es campo, es decir que $\mathfrak{p}$ es ideal maximal.
    \end{proof}

    \begin{excer}
        Sea $x$ un elemento nilpotente de un anillo A. Muestre que $1+x$ es una unidad de $A$. Deduzca que la suma de elementos nilpotentes con una unidad es una unidad. 
    \end{excer}

    \begin{proof}
        Dado que $x$ es nilpotente, existe $n\in\bbm{N}$ tal que:
        \begin{equation*}
            x^n=0
        \end{equation*}
        Entonces:
        \begin{equation*}
            \begin{split}
                &(1+x)\left(1-x+\dots+(-1)^{ n-1}x^{ n-1}\right)\\
                &=\left[1-x+\dots+(-1)^{ n-1}x^{ n-1}\right]+[x-x^2+\dots+(-1)^{ n-2}x^{n-1}+(-1)^{ n-1}x^n]\\
                &=1\\
            \end{split}
        \end{equation*}
        Por ende, $1+x$ es unidad. Si $a$ y $b$ son nilpotentes, entonces existen $n,m\in\bbm{N}$ tal que:
        \begin{equation*}
            a^n=b^m=0
        \end{equation*}
        Rápidamente se verifica que $(a+b)^{ nm}=0$. Se sigue así que $a+b$ es nilpotente. En particular, usando inducción se generaliza que la suma de elementos nilpotentes es nilpotente. Por ende, de lo anterior se sigue que la suma de 1 mas un elemento nilpotente es una unidad.
    \end{proof}

    \begin{excer}
        Sea $A$ un anillo en el cual todo elemento satisface que $x^n=x$ para algún $n\in\mathbb{N}$. Pruebe que todo ideal primo en $A$ es maximal.
    \end{excer}

    \begin{proof}
        Sea $I$ un ideal primo en $A$. Entonces, $A/I$ es dominio entero. Afirmamos que $A/I$ es campo. En efecto, para ello basta con mostrar que todo elemento de este anillo posee inverso. Sea $I+x\in A/I$, entonces existe $n\in\bbm{N}$ tal que $x^n=x$, luego:
        \begin{equation*}
            (I+x)(I+x^{ n-1})=I+x^n=I
        \end{equation*}
        Se sigue así que $A/I$ es campo, luego $I$ es ideal maximal.
    \end{proof}

    \begin{excer}
        Sea $A$ un anilo tal que cada ideal no contenido en el nilradical contiene un elemento idempotente no cero (esto es, un elemento $e$ tal que $e^2=e\neq0$). Pruebe que el nilradical y el radical de Jacobson de $A$ son iguales.
    \end{excer}

    \begin{proof}
        Recordemos que el nilradical es la itnerseccion de todos los ideales primos de $A$ y, el radical de Jacobson es el la intersección de todos los ideales maximales de $A$.

        Denotamos a los primer y segundo ideales mencionados anteriormente por $\mathfrak{N}$ y $\mathfrak{R}$. Dado que todo ideal maximal es en particular un ideal primo, se sigue que:
        \begin{equation*}
            \mathfrak{N}\subseteq\mathfrak{R}
        \end{equation*}
        Supongamos que la contención es propia, se sigue por hipótesis que $\mathfrak{R}$ contiene un elemento idempotente no cero, digamos $e=e^2\neq0$. Observemos que:
        \begin{equation*}
            (1-e)^2=1-e-e+e^2=1-2e+e=1-e
        \end{equation*}
        por lo cual $1-e$ también es idempotente. Por la caracterización del radical de Jacobson se tiene que $1-e$ es una unidad en $A$.

        Ahora, dado que $1-e$ es unidad en $A$, existe $y\in A$ tal que $(1-e)y=1$, luego:
        \begin{equation*}
            1=(1-e)^2y^2=(1-e)yy=y
        \end{equation*}
        Por lo cual $1-e=1$, lo cual contradice la elección de $e$ como elemento no cero. Se sigue así que $\mathfrak{N}=\mathfrak{R}$.
    \end{proof}

    \begin{obs}
        Créditos a la demostración del ejercicio anterior: \href{https://crypto.stanford.edu/pbc/notes/commalg/jacobson.html}{The Jacobson Radical}.
    \end{obs}

    \begin{excer}
        Sea $A$ un anillo no cero. Muestre que el conjunto de ideales primos de $A$ tiene elemento mínimo con respecto a la relación inclusión.
    \end{excer}

    \begin{proof}
        
    \end{proof}

    \begin{excer}
        Sea $\mathfrak{a}\neq(1)$ un ideal en un anillo $A$. Muestre que $\mathfrak{a}=r(\mathfrak{a})$ si y solo si $\mathfrak{a}$ es la intersección de ideales primos.
    \end{excer}

    \begin{proof}
        %TODO
    \end{proof}

    \newpage

    \section{Referencias}

    \begin{itemize}
        \item \textit{Introduction to Commutative Algebra}, M. F. Atiyah y I. G. MacDonald, University of Oxford.
    \end{itemize}
     
\end{document}