\documentclass[12pt]{report}
\usepackage[spanish]{babel}
\usepackage[utf8]{inputenc}
\usepackage{amsmath}
\usepackage{amssymb}
\usepackage{amsthm}
\usepackage{graphics}
\usepackage{subfigure}
\usepackage{lipsum}
\usepackage{array}
\usepackage{multicol}
\usepackage{enumerate}
\usepackage[framemethod=TikZ]{mdframed}
\usepackage[a4paper, margin = 1.5cm]{geometry}
\usepackage{bbm}
\usepackage{amsfonts}
\usepackage{enumitem}
\usepackage{tikz}
\usepackage{pgffor}
\usepackage{ifthen}

\usetikzlibrary{shapes.multipart}

\newcounter{it}
\newcommand*\watermarktext[1]{\begin{tabular}{c}
    \setcounter{it}{1}%
    \whiledo{\theit<100}{%
    \foreach \col in {0,...,15}{#1\ \ } \\ \\ \\
    \stepcounter{it}%
    }
    \end{tabular}
    }

\AddToHook{shipout/foreground}{
    \begin{tikzpicture}[remember picture,overlay, every text node part/.style={align=center}]
        \node[rectangle,black,rotate=30,scale=2,opacity=0.08] at (current page.center) {\watermarktext{Cristo Daniel Alvarado ESFM\quad}}; 
  \end{tikzpicture}
}

%En esta parte se hacen redefiniciones de algunos comandos para que resulte agradable el verlos%

\renewcommand{\theenumii}{\textit{\alph{enumii}})}

\def\proof{\paragraph{Demostración:\\}}
\def\endproof{\hfill$\blacksquare$}

\def\sol{\paragraph{Solución:\\}}
\def\endsol{\hfill$\square$}

%En esta parte se definen los comandos a usar dentro del documento para enlistar%

\newtheoremstyle{largebreak}
  {}% use the default space above
  {}% use the default space below
  {\normalfont}% body font
  {}% indent (0pt)
  {\bfseries}% header font
  {}% punctuation
  {\newline}% break after header
  {}% header spec

\theoremstyle{largebreak}

\newmdtheoremenv[
    leftmargin=0em,
    rightmargin=0em,
    innertopmargin=-2pt,
    innerbottommargin=8pt,
    hidealllines = true,
    roundcorner = 5pt,
    backgroundcolor = gray!60!red!30
]{exa}{Ejemplo}[section]

\newmdtheoremenv[
    leftmargin=0em,
    rightmargin=0em,
    innertopmargin=-2pt,
    innerbottommargin=8pt,
    hidealllines = true,
    roundcorner = 5pt,
    backgroundcolor = gray!50!blue!30
]{obs}{Observación}[section]

\newmdtheoremenv[
    leftmargin=0em,
    rightmargin=0em,
    innertopmargin=-2pt,
    innerbottommargin=8pt,
    rightline = false,
    leftline = false
]{theor}{Teorema}[section]

\newmdtheoremenv[
    leftmargin=0em,
    rightmargin=0em,
    innertopmargin=-2pt,
    innerbottommargin=8pt,
    rightline = false,
    leftline = false
]{propo}{Proposición}[section]

\newmdtheoremenv[
    leftmargin=0em,
    rightmargin=0em,
    innertopmargin=-2pt,
    innerbottommargin=8pt,
    rightline = false,
    leftline = false
]{cor}{Corolario}[section]

\newmdtheoremenv[
    leftmargin=0em,
    rightmargin=0em,
    innertopmargin=-2pt,
    innerbottommargin=8pt,
    rightline = false,
    leftline = false
]{lema}{Lema}[section]

\newmdtheoremenv[
    leftmargin=0em,
    rightmargin=0em,
    innertopmargin=-2pt,
    innerbottommargin=8pt,
    roundcorner=5pt,
    backgroundcolor = gray!30,
    hidealllines = true
]{mydef}{Definición}[section]

\newmdtheoremenv[
    leftmargin=0em,
    rightmargin=0em,
    innertopmargin=-2pt,
    innerbottommargin=8pt,
    roundcorner=5pt
]{excer}{Ejercicio}[section]

%En esta parte se colocan comandos que definen la forma en la que se van a escribir ciertas funciones%

\newcommand\abs[1]{\ensuremath{\left|#1\right|}}
\newcommand\divides{\ensuremath{\bigm|}}
\newcommand\cf[3]{\ensuremath{#1:#2\rightarrow#3}}
\newcommand\natint[1]{\ensuremath{\left[\!\left[ #1\right]\!\right]}}
\newcommand{\afa}{\:
    \begin{tikzpicture}
        \draw [line width = 0.17 mm, black] (0,0) -- (-0.115,0.29);
        \draw [line width = 0.17 mm, black] (0,0) -- (0.115,0.29);
        \draw [line width = 0.17 mm, black] (-0.12,0) arc (190:-10:0.12cm);
    \end{tikzpicture}
    \:
}
\newcommand{\contradiction}{\ensuremath{\#_c}}
\newcommand{\bbm}[1]{\ensuremath{\mathbbm{#1}}}
\newcommand{\fk}[1]{\ensuremath{\mathfrak{#1}}}
\newcommand{\gen}[1]{\ensuremath{\langle#1\rangle}}

\renewcommand{\theenumi}{\roman{enumi}}

%Este símvolo es para casi todo salvo una cantidad finita

%recuerda usar \clearpage para hacer un salto de página

\begin{document}
    \setlength{\parskip}{5pt} % Añade 5 puntos de espacio entre párrafos
    \setlength{\parindent}{12pt} % Pone la sangría como me gusta
    \title{Introducción a la Geometría Algebraica
    
    Notas}
    \author{Cristo Daniel Alvarado}
    \maketitle

    \tableofcontents %Con este comando se genera el índice general del libro%

    %\setcounter{chapter}{3} %En esta parte lo que se hace es cambiar la enumeración del capítulo%
    
    \chapter{Variedades}
    
    \textit{Nota}: No confundir el tipo de variedades que estaremos trabajando en esta primera parte con aquellas variedades en el ámbito topológico. Es deseable tener conocimiento básico sobre anillos, campos, extensiones de campos, teoría de grupos y demás temas concernientes al álgebra abstracta para entender los conceptos que se presentarán a lo largo del documento.
    
    De ahora en adelante el símbolo $\natint{a,b}$, denotará a todos los números naturales contenidos en el intervalo $[a,b]$, $a,b\in\mathbb{R}$.

    \section{Variedades Afines}
    
    Nuestra tarea fundmanetal será enunciar los conceptos básicos con los que se trabajarán.

    De ahora en adelante, $K$ denotará a un campo (generalmente algebraicamente cerrado).

    \begin{mydef}
        Sea $K$ un campo algebraicamente cerrado. Definimos el \textbf{espacio afín $n$-dimensional sobre $K$}, denotado por $\bbm{A}_K^n$, o simplemente $\bbm{A}^n$ como el conjunto de todas las $n$-tuplas de elementos de $K$, esto es:
        \begin{equation*}
            \bbm{A}_K^n=\underset{n-\textup{veces}}{\underbrace{K\times\cdots\times K}}
        \end{equation*}
        Un elemento $P\in\bbm{A}_K^n$ será llamado \textbf{punto}, y si $P=(a_1,...,a_n)$, con $a_i\in K$ para todo $i\in\natint{1,n}$, entonces $a_i$ será llamada la \textbf{$i$-ésima coordenada de $P$}.
    \end{mydef}

    Sea $A=K[x_1,...,x_n]$ el anillo de polinomios en $n$-entradas sobre el campo $K$ (considerado como algebraicamente cerrado). Interpretaremos los elementos de $A$ como funciones del espacio afín $\bbm{A}^n$ a $K$, definiendo
    \begin{equation*}
        f(P)=f(a_1,...,a_n)
    \end{equation*}
    para todo $P\in\bbm{A}^n$, con $f\in A$.

    \begin{mydef}
        Sea $f\in K[x_1,...,x_n]$. Se definen los \textbf{ceros de $f$}, como el conjunto
        \begin{equation*}
            Z(f)=\left\{P\in\bbm{A}^n\Big|f(P)=0 \right\}
        \end{equation*}
        Si $T\subseteq\bbm{A}^n$, definimos el \textbf{conjunto cero de $T$}, como el conjunto:
        \begin{equation*}
            Z(T)=\left\{P\in\bbm{A}^n\Big|f(P)=0\textup{, para todo }f\in T \right\}
        \end{equation*}
    \end{mydef}

    \begin{propo}
        Si $\fk{a}$ es el ideal de $K[x_1,...,x_n]$ generado por $T$, entonces:
        \begin{equation*}
            Z(T)=Z(\fk{a})
        \end{equation*}
    \end{propo}

    \begin{proof}
        Ejercicio.
    \end{proof}

    \begin{obs}
        Como $K[x_1,...,x_n]$ es un anillo Noetheriano (por ser $K[x_1,...,x_n]$ un Dominio Euclideano, se sigue que es Dominio de Ideales Principales, en particular, todo DIP es Noetheriano). Entonces, el ideal $\fk{a}$ es finitamente generado, esto es:
        \begin{equation*}
            \fk{a}=\gen{f_1,...,f_n}
        \end{equation*}
        donde $f_1,...,f_n\in K[x_1,...,x_n]$. Por tanto, $Z(T)$ puede ser expresado como todos los ceros comunes del conjunto finito de polinomios $\left\{f_1,...,f_n \right\}$.
    \end{obs}

    \begin{mydef}
        Un subconjunto $Y\subseteq\bbm{A}^n$ es llamado \textbf{algebraico} si existe $T\subseteq K[x_1,...,x_n]$ tal que $Y=Z(T)$.
    \end{mydef}

    Para la siguiente proposición, recuerde que dados $A,B\subseteq K[x_1,...,x_n]$, se define el producto de $A$ por $B$ como el conjunto:
    \begin{equation*}
        AB=\left\{fg\Big|f\in A\textup{ y }g\in B \right\}
    \end{equation*}
    
    \begin{propo}
        Se cumple lo siguiente:
        \begin{enumerate}[label=(\textit{\alph*})]
            \item La unión de dos conjuntos algebraicos es un conjunto algebraico.
            \item La interseccioń de cualquier familia de conjuntos algebraicos es un conjunto algebraico.
            \item El conjunto vacío y todo el espacio son conjuntos algebraicos.
        \end{enumerate}
    \end{propo}

    \begin{proof}
        De $(a)$: Sean $Y_1$ y $Y_2$ subconjuntos de $\bbm{A}^n$ algebraios, entonces existen $T_1,T_2\subseteq K[x_1,...,x_n]$ finitos tales que
        \begin{equation*}
            Y_i=Z(T_i),\quad\forall i=1,2
        \end{equation*}
        Afirmamos que
        \begin{equation*}
            Y_1\cup Y_2=Z(T_1T_2)
        \end{equation*}
        En efecto:
        \begin{itemize}
            \item Si $P\in Y_1\cup Y_2$, entonces $f(P)=0$ para todo $f\in T_1$ o para todo $f\in T_2$. Se sigue entonces que:
            \begin{equation*}
                fg(P)=f(P)g(P)=0
            \end{equation*}
            para todo $fg\in T_1T_2$. Luego, $P\in Z(T_1T_2)$.
            \item Si $P\in Z(T_1T_2)$, suponiendo que $P\notin Y_2$, entonces existe $g\in T_2$ tal que $g(P)\neq 0$. Pero, se tiene que:
            \begin{equation*}
                fg(P)=0
            \end{equation*}
            para todo $f\in T_1$, luego $f(P)=0$ para todo $f\in T_1$. Así que $P\in Z(T_1)=Y_1$. Se sigue así que $P\in Y_1\cup Y_2$.
        \end{itemize}
        por los dos incisos anteriores se tiene la doble contención. Por tanto, la unión de dos conjuntos algebraicos sigue siendo algebraico.

        De $(b)$: Sea $\left\{Y_i=Z(T_i) \right\}_{ i\in I}$ una familia no vacía de conjuntos algebraicos. Afirmamos que:
        \begin{equation*}
            \bigcap_{ i\in I}Y_I=Z\left(\bigcap_{ i\in I}T_i \right)
        \end{equation*}
        En efecto, es inmediata de la definición de conjunto algebraico. Por tanto, la intersección de conjuntos algebraicos es algebraicos.

        De $(c)$: Finalmente, se tiene que
        \begin{itemize}
            \item $\emptyset=Z(1)$, donde $1$ es el polinomio con valor constante 1 en $K[x_1,...,x_n]$.
            \item $\bbm{A}^n=Z(0)$, donde 0 es el polinomio con valor constante 0 en $K[x_1,...,x_n]$.
        \end{itemize}
    \end{proof}
    
    Note que de esta definción anterior, si tomamos la familia los complementos de los conjuntos algebraicos, obtenemos que la unión arbitraria, la intersección finita y el vacío y todo el conjunto están en la familia, es decir, hemos construído una topología sobre $\bbm{A}^n$.

    \begin{mydef}
        Definimos la \textbf{topología de Zariski en $\bbm{A}^n$}, formada por la familia de los complementos de todos los conjuntos algebraicos sobre $\bbm{A}^n$.
    \end{mydef}

    \begin{exa}
        Consideremos la topología de Zariski en la recta afín $\bbm{A}^1$. Como todo ideal de $A=K[x]$ es principal, se tiene que todo conjunto algebraico es el conjunto de ceros de un sólo polinomio en $A$, digamos si $Y$ es algebraico, entonces
        \begin{equation*}
            Y=Z(\left\{f\right\})=Z(f)
        \end{equation*}

        Asumiendo que $K$ es algebraicamente cerrado se tiene que todo polinomio no cero se puede escribir como:
        \begin{equation*}
            f(x)=c(x-a_1)\cdots(x-a_n)
        \end{equation*}
        con $c\in K^*$ y $a_1,...,a_n\in K$. Por tanto,
        \begin{equation*}
            Z(f)=\left\{a_1,...,a_n\right\}
        \end{equation*}
        Por tanto, los conjuntos algebraicos en $\bbm{A}^1$ son los conjuntos finitos, el total y el vacío. Por tanto, la topología de Zariski en $\bbm{A}^1$ coincide con la topología de los complementos finitos.
    \end{exa}

    \begin{mydef}
        Sea $X$ espacio topológico. Un subconjunto no vacío $Y\subseteq X$ es llamado \textbf{irreducible}, si no puede ser expresado como la unión de dos subconjuntos propios $Y_1,Y_2\subseteq Y$, cada uno de los cuales es cerrado en $Y$.

        Convenimos en que el conjunto vacío no es irreducible.
    \end{mydef}

    \begin{exa}
        $\bbm{A}^1$ es irreducible, pues sus únicos subconjuntos propios cerrados son los algebraicos, que son finitos y $\bbm{A}^1$ es infinito.
    \end{exa}

    \begin{exa}
        Cualquier abierto no vacío de un espacio irreducible de un espacio irreducible es irreducible y denso.
    \end{exa}

    \begin{proof}
        Sea $X$ un espacio irreducible y $Y\subseteq X$ abierto no vacío en $X$. Si $Y=Y_1\cup Y_2$ siendo $Y_1,Y_2\subseteq X$ cerrados en $Y$, como $Y_2$ es el complemento de $Y_1$ en $Y$, por ser $Y_1$ cerrado en $Y$, se sigue que $Y_2$ es abierto en $Y$, en particular es abierto en $X$, de manera análoga $Y_1$ es abierto en $X$. Por tanto, tomando
        \begin{equation*}
            Z_i=X-Y_i,\quad i=1,2
        \end{equation*}
        se tiene que
        \begin{equation*}
            X=Z_1\cup Z_2
        \end{equation*}
        siendo $Z_1,Z_2$ subconjuntos de $X$ cerrados. Al ser $X$ irreducible se tiene que $Z_1=X$ o $Z_2=X$, en cuyo caso se seguiría que $Y_1=\emptyset$ o $Y_2=\emptyset$, esto es que $Y_2=Y$ o $Y_1=Y$. Por tanto, $Y$ es irreducible.

        Ahora, veamos que $Y$ es denso. Podemos suponer que $Y\neq X$, tomemos
        \begin{equation*}
            Z_1=X-Y\quad\textup{y}\quad Z_2=\overline{Y}
        \end{equation*}
        $Z_1$ y $Z_2$ son cerrados no vacíos tales que $X=Z_1\cup Z_2$. Como $X$ es irreducible, entonces $Z_1=X$ o $Z_2=X$, pero $Z_1\neq X$ ya que $Y$ es subconjunto propio de $X$. Por ende, $Z_2=X$, es decir que
        \begin{equation*}
            \overline{Y}=X
        \end{equation*}
        por tanto, $Y$ es denso en $X$.
    \end{proof}

    \begin{exa}
        Si $Y$ es un subconjunto irreducible de $X$, entonces $\overline{Y}$ es irreducible. 
    \end{exa}

    \begin{proof}
        Si $\overline{Y}=Y_1\cup Y_2$ con $Y_1$ y $Y_2$ subconjuntos cerrados de $\overline{Y}$ en el mismo espacio, se tiene que en particular son cerrados en $X$. Luego,
        \begin{equation*}
            Y=(Y_1\cap Y)\cup(Y_2\cap Y)
        \end{equation*}
        donde los dos conjuntos de la unión son cerrados en $Y$. Así que $Y_1\cap Y=Y$ o $Y_2\cap Y=Y$ por ser $Y$ irreducible, luego $Y\subseteq Y_1$ o $Y\subseteq Y_2$, es decir que $\overline{Y}\subseteq Y_1$ o $\overline{Y}\subseteq Y_2$, esto es que $Y_1=\overline{Y}$ o $Y_2=\overline{Y}$.
        
        Por tanto, $\overline{Y}$ es irreducible.
    \end{proof}

    \begin{mydef}
        Una \textbf{variedad algebraica afín} (o simplemente llamada \textbf{variedad afín}), es un subconjunto cerrado irreducible de $\bbm{A}^n$ (con la topología inducida). Un subconjunto abierto de una variedad afín es una \textbf{variedad quasi-afín}.
    \end{mydef}

    Por el ejemplo 1.1.3, tiene sentido nombrar a los subconjuntos abiertos de una variedad afín como variedades quasi-afínes, ya que son densos en la variedad afín.

    Nuestro objetivo ahora será estudiar las variedades afines y quasi-afines. Antes de ello, debemos hablar un poco de la relación entre los ideales de $K[x_1,...,x_n]$ y los subconjuntos de $\bbm{A}^n$.

    \begin{mydef}
        Sea $Y\subseteq\bbm{A}^n$, definimos el \textbf{ideal de $Y$ en $K[x_1,...,x_n]$} como el conjunto:
        \begin{equation*}
            I(Y)=\left\{f\in K[x_1,...,x_n]\Big|f(P)=0\textup{ para todo }P\in Y \right\}
        \end{equation*}
    \end{mydef}

    De la definición de ideal es inmediato que $I(Y)$ es un ideal, para todo $Y\subseteq\bbm{A}^n$.

    \begin{theor}[\textbf{Hillbert's Nullstellensatz}]
        Sea $K$ un campo algebraicamente cerrado, $\fk{a}$ un ideal en $A=K[x_1,...,x_n]$ y sea $f\in A$ un polinomio tal que
        \begin{equation*}
            f(P)=0,\quad\forall P\in Z(\fk{a})
        \end{equation*}
        entonces, $f^r\in\fk{a}$ para algún entero positivo $r$.
    \end{theor}

    \begin{proof}
        Podría escribir la prueba (aunque me llevaría un ratote), pero es más fácil consultar la demostración en \textit{Algebra} de Serge Lang, pag. 378, por lo que daremos el resultado como cierto.
    \end{proof}

    \begin{propo}
        Se cumple lo siguiente:
        \begin{enumerate}[label=(\textit{\alph*})]
            \item Si $T_1\subseteq T_2$ son subconjuntos de $K[x_1,...,x_n]$, entonces $Z(T_1)\supseteq Z(T_2)$.
            \item Si $Y_1\subseteq Y_2$ son subconjuntos de $\bbm{A}^n$, entonces $I(Y_1)\supseteq I(Y_2)$.
            \item Para cualesquiera dos subconjuntos $Y_1,Y_2$ de $\bbm{A}^n$, se tiene que $I(Y_1\cup Y_2)=I(Y_1)\cap I(Y_2)$.
            \item Para un ideal $\fk{a}\subseteq K[x_1,...,x_n]$, $I(Z(\fk{a}))=\sqrt{\fk{a}}$, donde
            \begin{equation*}
                \sqrt{\fk{a}}=\left\{f\in K[x_1,...,x_n]\Big|f^r\in\fk{a}\textup{ para algún entero positivo }r \right\}
            \end{equation*}
            \item Para todo subconjunto $Y\subseteq\bbm{A}^n$, $Z(I(Y))=\overline{Y}$.
        \end{enumerate}
    \end{propo}

    \begin{proof}
        De $(a)$, $(b)$ y $(c)$: Es inmediata de la definición.

        De $(d)$: Asumiendo que $K$ es algebraicamente cerrado, veamos la doble contención:
        \begin{itemize}
            \item Si $f\in I(Z(\fk{a}))$, entonces $f(P)=0$ para todo $P\in Z(\fk{a})$. Por el Teorema Hillbert's Nullstellensatz, es sigue que existe un entero positivo $r$ tal que $f^r\in\fk{a}$, esto es que $f\in\sqrt{\fk{a}}$.
            \item Si $f\in\sqrt{\fk{a}}$, entonces existe un entero positivo $r$ tal que $f^r\in A$. Por tanto:
            \begin{equation*}
                f^r(P)=\left(f(P)\right)^r=0,\quad\forall P\in Z(\fk{a})
            \end{equation*}
            como $K[x_1,...,x_n]$ es dominio entero, entonces $f(P)=0$, para todo $P\in Z(\fk{a})$. Por tanto, $f\in I(Z(\fk{a}))$.
        \end{itemize}
        de los dos incisos se sigue la doble contención.
        
        De $(e)$: Sea $Y\subseteq\bbm{A}^n$. Si $P\in Y$, entonces $f(P)=0$ para todo $f\in I(P)$, luego se sigue de forma inmediata que $P\in Z(I(Y))$. Así que $Y\subseteq Z(I(Y))$. Como el conjunto
        \begin{equation*}
            Z(I(Y))
        \end{equation*}
        (pues por definición $Z(I(Y))$ es algebraico, en particular es cerrado en la topología de Zariski inducida en $\bbm{A}^n$). Por ende
        \begin{equation*}
            \overline{Y}\subseteq Z(I(Y))
        \end{equation*}
        Sea ahora $W$ un conjunto cerrado tal que $Y\subseteq W$. Se tiene que existe un ideal $\fk{a}$ de $K[x_1,...,x_n]$ tal que
        \begin{equation*}
            W=Z(\fk{a})
        \end{equation*}
        por ende,
        \begin{equation*}
            Y\subseteq Z(\fk{a})
        \end{equation*}
        Por $(b)$ se sigue que
        \begin{equation*}
            I(Z(\fk{a}))\subseteq I(Y)
        \end{equation*}
        pero, $I(Z(\fk{a}))=\sqrt{\fk{a}}$, donde $\fk{a}\subseteq\sqrt{\fk{a}}$. Por ende:
        \begin{equation*}
            \fk{a}\subseteq I(Y)
        \end{equation*}
        luego,
        \begin{equation*}
            Z(I(Y))\subseteq Z(\fk{a})=W
        \end{equation*}
        tomando $W=\overline{Y}$ se tiene que
        \begin{equation*}
            Z(I(Y))\subseteq\overline{Y}
        \end{equation*}
        Por las dos contenciones obtenemos
        \begin{equation*}
            Z(I(Y))=\overline{Y}
        \end{equation*}
    \end{proof}

    \begin{cor}
        Existe una correspondencia biyectiva entre los conjuntos algebraicos en $\bbm{A}^n$ y los ideales radicales (esto es, ideales que coinciden con su ideal radical) en $K[x_1,...,x_n]$, dada por $Y\mapsto I(Y)$ y $\fk{a}\mapsto Z(\fk{a})$.

        Más aún, un conjunto algebraico es irreducible si y sólo si su ideal es ideal primo.
    \end{cor}

    \begin{proof}
        La primera parte es inmediata del teorema anterior. Para la otra,
        \begin{itemize}
            \item Sea $Y\subseteq\bbm{A}^n$ irreducible, veamos que $I(Y)$ es ideal primo. Sea $fg\in I(Y)$, entonces $\left\{fg \right\}\subseteq I(Y)$, por lo cual
            \begin{equation*}
                Y=Z(I(Y))\subseteq Z(fg)
            \end{equation*}
            donde $Z(fg)=Z(f)\cup Z(g)$. Por tanto,
            \begin{equation*}
                Y=(Y\cap Z(f))\cup (Y\cap Z(g))
            \end{equation*}
            siendo los dos conjuntos de la unión cerrados en $Y$. Como $Y$ es irreducible, debe suceder que $Y=Y\cap Z(f)$ o $Y=Y\cap Z(g)$, esto es que $Y\subseteq Z(f)$ o $Y\subseteq Z(g)$, es decir que
            \begin{equation*}
                \left\{f \right\}=I(Z(f))\subseteq I(Y)\textup{ o }\left\{g \right\}=I(Z(g))\subseteq I(Y)
            \end{equation*}
            por ende, $f\in I(Y)$ o $g\in I(Y)$. Se sigue entonces que $I(Y)$ es ideal primo de $K[x_1,...,x_n]$.
            \item Sea $\fk{p}$ un ideal primo de $K[x_1,...,x_n]$ y suponga que $Z(\fk{p})=Y_1\cup Y_2$ siendo los dos conjuntos de la derecha cerrados y contenidos en $Z(\fk{p})$ (que es cerrado), luego lo son en $\bbm{A}^n$, entonces:
            \begin{equation*}
                I(Y_1\cup Y_2)=I(Z(\fk{p}))=\fk{p}
            \end{equation*}
            pero, $I(Y_1\cup Y_2)=I(Y_1)\cap I(Y_2)$ (por la proposición anterior). Luego,
            \begin{equation*}
                \fk{p}=I(Y_1)\cap I(Y_2)
            \end{equation*}
            siendo $I(Y_1),I(Y_2)$ ideales primos de $K[x_1,...,x_n]$ (por el inciso anterior). Por ende, $\fk{p}=I(Y_1)=I(Y_2)$. Así entonces $Y_1=Z(\fk{p})$ y $Y_2=Z(\fk{p})$. Se sigue así que $Z(\fk{p})$ es un conjunto irreducible en $\bbm{A}^n$.
        \end{itemize}
        Por los dos incisos, se tiene la segunda parte del corolario.
    \end{proof}

    \begin{exa}
        $\bbm{A}^n$ es irreducible, pues se corresponde con el ideal $\gen{0}$ en $K[x_1,...,x_n]$, el cual es irreducible.
    \end{exa}

    \begin{exa}
        Sea $f$ un polinomio irreducible en $A=K[x,y]$. Entonces, $f$ genera un ideal primo en $A$, esto es que $\gen{f}$ es ideal primo de $A$. Como $A$ es DFU, el conjunto cero
        \begin{equation*}
            Y=Z(f)=Z(\gen{f})
        \end{equation*}
        es irreducible (por ser algebraico). Llamamos al conjunto $Y$ la \textbf{curva afín} definida por la ecuación $f(x,y)=0$. Si $f$ tiene grado $d$, entonces decimos que $Y$ es una \textbf{curva de grado $d$}.
    \end{exa}

    El ejemplo anterior motiva la siguiente definición.

    \begin{mydef}
        Sea $f$ un polinomio irreducible en $K[x_1,...,x_n]$, la variedad afín
        \begin{equation*}
            Y=Z(f)=Z(\gen{f})
        \end{equation*}
        es llamada \textbf{superficie} si $n=3$ y, si $n>3$ es llamada \textbf{hipersuperficie}.
    \end{mydef}

    \begin{exa}
        Un ideal maximal $\fk{m}$ de $K[x_1,...,x_n]$ corresponde (ya que en particular $\fk{m}$ es ideal primo), a un subconjunto cerrado minimal de $\bbm{A}^n$, el cual forzosamente debe ser un punto (ya que $\fk{m}\neq K[x_1,...,x_n]$), digamos $\left\{P \right\}$ con $P=(a_1,...,a_n)$.

        Por ende, todo ideal maximal de $K[x_1,...,x_n]$ debe ser de la forma:
        \begin{equation*}
            \fk{m}=\gen{x_1-a_1,...,x_n-a_n}
        \end{equation*}
    \end{exa}

    \begin{exa}
        Los resultados anteriores no son válidos si $K$ no es algebraicamente cerrado. Por ejemplo, si $K=\bbm{R}$, entonces la curva $Y$:
        \begin{equation*}
            x^2+y^2+1=0
        \end{equation*}
        en $\bbm{A}^2_{\bbm{R}}$ es un conjunto vacío. Por ende, el inciso $(d)$ de la proposicioń 1.1.3 es falso (ya que si $f=x^2+y^2+1$, entonces $f$ es irreducible sobre $\mathbb{R}[x,y]$, luego debería suceder que $\gen{f}=I(Y)$ siendo $\gen{0}\neq\gen{f}$ y $I(Y)=I(\emptyset)=\gen{0}$, lo cual es un absurdo).
    \end{exa}

    \begin{mydef}
        Si $Y\subseteq\bbm{A}^n$ es una sub-variedad algebraica afín, definimos el \textbf{anillo afín de coordenadas}, denotado por $A(Y)$, como el cociente:
        \begin{equation*}
            A(Y)=K[x_1,...,x_n]/I(Y)
        \end{equation*}
    \end{mydef}

    \begin{obs}
        Si $Y\subseteq\bbm{A}^n$ es sub-variedad algebraica afín, entonces $I(Y)$ es ideal primo, luego $A(Y)$ es dominio entero. Más aún, $A(Y)$ es una $K$-álgebra finitamente generada.

        Recíprocamente, si $B$ es una $K$-álgebra finitamente generada y que sea dominio entero, es el cociente de $K[x_1,...,x_n]$ y $\fk{a}$ un ideal primo de este anillo, luego tomando $Y=Z(\fk{a})$ se sigue que:
        \begin{equation*}
            B=K[x_1,...,x_n]/\fk{a}=K[x_1,...,x_n]/I(Y)=A(Y)
        \end{equation*}
        es decir, $B$ es el anillo afín de coordenadas de $Y$.
    \end{obs}

    Ahora, estudiaremos las propiedades topológicas de nuestras variedad. Para ello, introduciremos una clase importante de espacios topológicos, mismos que incluyen a todas las variedades.

    \begin{mydef}
        Un espacio topológico $X$ es llamado \textbf{Noetheriano}, si satisface la \textit{condición de cadena descendiente} para subconjuntos cerrados:
        \begin{itemize}
            \item Para toda sucesión $\left\{Y_n \right\}_{ n=1}^\infty$ de subconjuntos cerrados de $X$ tales que
            \begin{equation*}
                Y_1\supseteq Y_2\supseteq\cdots\supseteq Y_m\supseteq\cdots
            \end{equation*}
            existe $r\in\mathbb{N}$ tal que
            \begin{equation*}
                Y_r=Y_{ r+n},\quad\forall n\in\mathbb{N}
            \end{equation*}
        \end{itemize}
    \end{mydef}

    \begin{exa}
        $\bbm{A}^n$ es un espacio Noetheriano. En efecto, si $\left\{Y_n \right\}_{ n=1}^\infty$ es una sucesión de subconjuntos cerrados de $X$ tales que
        \begin{equation*}
            Y_1\supseteq Y_2\supseteq\cdots\supseteq Y_m\supseteq\cdots
        \end{equation*}
        entonces, $\left\{I(Y_n) \right\}_{ n=1}^\infty$ es una sucesión creciente de ideales de $K[x_1,...,x_n]$. Como $K[x_1,...,x_n]$  es Noetheriano, existe $r\in\mathbb{N}$ tal que 
        \begin{equation*}
            I(Y_r)=Y(Y_{ n+r}),\quad\forall n\in\mathbb{N}
        \end{equation*}
        Por ende,
        \begin{equation*}
            Y_r=Y_{ r+n},\quad\forall n\in\mathbb{N}
        \end{equation*}
        Así, $\bbm{A}^n$ es Noetheriano.
    \end{exa}

    \begin{obs}
        Notemos que si $X$ es Noetheriano, toda familia no vacía de conjuntos cerrados $\mathcal{C}=\left\{C_i\Big|i\in I \right\}$ tiene elemento minimal. En efecto, $\mathcal{C}$ es parcialmente ordenado por $\supseteq$ y toda cadena tiene elemento minimal, luego por el Lema de Zorn, se sigue que $\mathcal{C}$ tiene elemento minimal.
    \end{obs}

    \begin{propo}
        En un espacio Noetheriano $X$, todo subconjunto no vacío $Y$ de $X$ puede ser expresado como la unioń finita $Y=Y_1\cup\cdots\cup Y_r$ de subconjuntos cerrados irreducibles $Y_i$, para $i\in\natint{1,r}$.

        Si se requiere que $Y_i\nsubseteq Y_j$ para todo $i\neq j$, $i,j\in\natint{1,r}$, entonces los $Y_i$ están determinados de forma única y son llamados las \textbf{componentes irreducibles de $Y$}.
    \end{propo}

    \begin{proof}
        Primero, probaremos la existencia de esta representación de $Y$. Sea
        \begin{equation*}
            \mathcal{C}=\left\{Z\subseteq X\Big|Z\neq\emptyset\textup{ es cerrado y no se escribe como unión finita de conjuntos cerrados irreducibles} \right\}
        \end{equation*}
        Si $\mathcal{C}$ fuese no vacío, al ser $X$ Noetheriano entonces $\mathcal{C}$ tendría elemento minimal (recuerde la observación anterior), digamos $C\subseteq X$ cerrado. Este $C$ no es irreducible, por lo que puede ser escrito como:
        \begin{equation*}
            C=C_1\cup C_2
        \end{equation*}
        donde $C_1,C_2$ son subconjuntos propios cerrados en $C$, en particular lo son en $X$. Luego, por minimalidad de $C$ debe suceder que $C_1,C_2\in\mathcal{C}$, así que $C\in\mathcal{C}$\contradiction. Por tanto, $\mathcal{C}$ es vacío. Se sigue que $Y$ admite esta representación.

        Ahora, supongamos que los $Y_i$ son tales que $Y_i\nsubseteq Y_j$ para todo $i,j\in\natint{1,r}$, $i\neq j$. Consideremos otra representación de $Y$ con las mismas caracaterísticas, digamos
        \begin{equation*}
            Y=Y_1'\cup\cdots\cup Y_s'
        \end{equation*}
        Entonces,
        \begin{equation*}
            Y_1'=\bigcup_{ i=1}^r \left(Y_i\cap Y_j'\right)
        \end{equation*}
        pero, $Y_1'$ es cerrado irreducible, luego debe suceder que exista $i_1\in\natint{1,r}$ tal que
        \begin{equation*}
            Y_1'=Y_{i_1}\cap Y_1'
        \end{equation*}
        esto es, que $Y_1'\subseteq Y_{i_1}$. Similarmente se prueba    que tiene que $Y_{i_1}\subseteq Y_{ j}'$, para algún $j\in\natint{1,r}$, luego $Y_1'=Y_j'$, así que $Y_1'=Y_{i_1}$. Sea ahora
        \begin{equation*}
            Z=Y_2\cup\cdots\cup Y_r=Y_2'\cup\cdots\cup Y_s'
        \end{equation*}
        aplicando inducción se sigue el resultado.
    \end{proof}

    \begin{cor}
        Todo subconjunto algebraico de $\bbm{A}^n$ puede ser escrita de forma única como unión de variedades algebraicas afines, ninguna conteniendo a la otra. 
    \end{cor}

    \begin{proof}
        Como $\bbm{A}^n$ es Noetheriando, el resultado es inmediato del hecho de que en $\bbm{A}^n$ los conjuntos cerrados son los conjuntos algebraicos, y que las variedades algebraicas afines son las conjuntos cerrados irreducibles de $\bbm{A}^n$.
    \end{proof}

    \begin{mydef}
        
    \end{mydef}

    \newpage

    \section{Ejercicios}

    \begin{excer}
        Haga lo siguiente:
        \begin{enumerate}[label=(\textit{\alph*})]
            \item Sea $Y$ la curva plana $y=x^2$ (esto es, $Y$ es el conjunto de los ceros del polinomio $f(x,y)=y-x^2$). Muestre que $A(Y)$ es isomorfo a 
        \end{enumerate}
    \end{excer}

    \newpage

    \section{Referencias}

    \begin{itemize}
        \item \textit{Algebraic Geometry} de Robin Hartshorne, ed. Springer.
        \item \textit{Algebra} de Serge Lang, ed. Springer.
    \end{itemize}

\end{document}