\documentclass[12pt]{report}
\usepackage[spanish]{babel}
\usepackage[utf8]{inputenc}
\usepackage{amsmath}
\usepackage{amssymb}
\usepackage{amsthm}
\usepackage{graphics}
\usepackage{subfigure}
\usepackage{lipsum}
\usepackage{array}
\usepackage{multicol}
\usepackage{enumerate}
\usepackage[framemethod=TikZ]{mdframed}
\usepackage[a4paper, margin = 1.5cm]{geometry}
\usepackage{tikz}
\usepackage{pgffor}
\usepackage{ifthen}
\usepackage{enumitem}
\usepackage{hyperref}

\usepackage{listings}

%Gestión de Hipervínculos

\hypersetup{
    colorlinks=true,
    linkcolor=black,
    filecolor=magenta,      
    urlcolor=cyan
}

%Gestión de Código de Programación

\definecolor{listing-background}{HTML}{F7F7F7}
\definecolor{listing-rule}{HTML}{B3B2B3}
\definecolor{listing-numbers}{HTML}{B3B2B3}
\definecolor{listing-text-color}{HTML}{000000}
\definecolor{listing-keyword}{HTML}{435489}
\definecolor{listing-keyword-2}{HTML}{1284CA} % additional keywords
\definecolor{listing-keyword-3}{HTML}{9137CB} % additional keywords
\definecolor{listing-identifier}{HTML}{435489}
\definecolor{listing-string}{HTML}{00999A}
\definecolor{listing-comment}{HTML}{8E8E8E}

\lstdefinestyle{myStyle}{
    language         = C++,
    alsolanguage     = scala,
    numbers          = left,
    xleftmargin      = 2.7em,
    framexleftmargin = 2.5em,
    backgroundcolor  = \color{gray!15},
    basicstyle       = \color{listing-text-color}\linespread{1.0}\ttfamily,
    breaklines       = true,
    frameshape       = {RYR}{Y}{Y}{RYR},
    rulecolor        = \color{black},
    tabsize          = 2,
    numberstyle      = \color{listing-numbers}\linespread{1.0}\small\ttfamily,
    aboveskip        = 1.0em,
    belowskip        = 0.1em,
    abovecaptionskip = 0em,
    belowcaptionskip = 1.0em,
    keywordstyle     = {\color{listing-keyword}\bfseries},
    keywordstyle     = {[2]\color{listing-keyword-2}\bfseries},
    keywordstyle     = {[3]\color{listing-keyword-3}\bfseries\itshape},
    sensitive        = true,
    identifierstyle  = \color{listing-identifier},
    commentstyle     = \color{listing-comment},
    stringstyle      = \color{listing-string},
    showstringspaces = false,
    label            = lst:bar,
    captionpos       = b,
}

\lstset{style = myStyle}

%Estilo del capítulo y sección

\makeatletter
\def\thickhrulefill{\leavevmode \leaders \hrule height 1ex \hfill \kern \z@}
\def\@makechapterhead#1{%
  {\parindent \z@ \raggedright
    \reset@font
    \hrule
    \vspace*{10\p@}%
    \par
    \center \LARGE \scshape \@chapapp{} \huge \thechapter
    \vspace*{10\p@}%
    \par\nobreak
    \vspace*{10\p@}%
    \par
    \vspace*{1\p@}%
    \hrule
    %\vskip 40\p@
    \vspace*{60\p@}
    \Huge #1\par\nobreak
    \vskip 50\p@
  }}

\def\section#1{%
  \par\bigskip\bigskip
  \hrule\par\nobreak\noindent
  \refstepcounter{section}%
  \addcontentsline{toc}{chapter}{#1}%
  \reset@font
  { \large \scshape
    \strut\S \thesection \quad
    #1}% 
    \hrule   
  \par
  \medskip
}

%Gestión marca de agua

\usetikzlibrary{shapes.multipart}

\newcounter{it}
\newcommand*\watermarktext[1]{\begin{tabular}{c}
    \setcounter{it}{1}%
    \whiledo{\theit<100}{%
    \foreach \col in {0,...,15}{#1\ \ } \\ \\ \\
    \stepcounter{it}%
    }
    \end{tabular}
    }

\AddToHook{shipout/foreground}{
    \begin{tikzpicture}[remember picture,overlay, every text node part/.style={align=center}]
        \node[rectangle,black,rotate=30,scale=2,opacity=0.04] at (current page.center) {\watermarktext{Cristo Daniel Alvarado ESFM\quad}};
  \end{tikzpicture}
}

%En esta parte se hacen redefiniciones de algunos comandos para que resulte agradable el verlos%

\def\proof{\paragraph{Demostración:\\}}
\def\endproof{\hfill$\blacksquare$}

\def\sol{\paragraph{Solución:\\}}
\def\endsol{\hfill$\square$}

%En esta parte se definen los comandos a usar dentro del documento para enlistar%

\newtheoremstyle{largebreak}
  {}% use the default space above
  {}% use the default space below
  {\normalfont}% body font
  {}% indent (0pt)
  {\bfseries}% header font
  {}% punctuation
  {\newline}% break after header
  {}% header spec

\theoremstyle{largebreak}

\newmdtheoremenv[
    leftmargin=0em,
    rightmargin=0em,
    innertopmargin=0pt,
    innerbottommargin=5pt,
    hidealllines = true,
    roundcorner = 5pt,
    backgroundcolor = gray!60!red!30
]{exa}{Ejemplo}[section]

\newmdtheoremenv[
    leftmargin=0em,
    rightmargin=0em,
    innertopmargin=0pt,
    innerbottommargin=5pt,
    hidealllines = true,
    roundcorner = 5pt,
    backgroundcolor = gray!50!blue!30
]{obs}{Observación}[section]

\newmdtheoremenv[
    leftmargin=0em,
    rightmargin=0em,
    innertopmargin=0pt,
    innerbottommargin=5pt,
    rightline = false,
    leftline = false
]{theor}{Teorema}[section]

\newmdtheoremenv[
    leftmargin=0em,
    rightmargin=0em,
    innertopmargin=0pt,
    innerbottommargin=5pt,
    rightline = false,
    leftline = false
]{propo}{Proposición}[section]

\newmdtheoremenv[
    leftmargin=0em,
    rightmargin=0em,
    innertopmargin=0pt,
    innerbottommargin=5pt,
    rightline = false,
    leftline = false
]{cor}{Corolario}[section]

\newmdtheoremenv[
    leftmargin=0em,
    rightmargin=0em,
    innertopmargin=0pt,
    innerbottommargin=5pt,
    rightline = false,
    leftline = false
]{lema}{Lema}[section]

\newmdtheoremenv[
    leftmargin=0em,
    rightmargin=0em,
    innertopmargin=0pt,
    innerbottommargin=5pt,
    roundcorner=5pt,
    backgroundcolor = gray!30,
    hidealllines = true
]{mydef}{Definición}[section]

\newmdtheoremenv[
    leftmargin=0em,
    rightmargin=0em,
    innertopmargin=0pt,
    innerbottommargin=5pt,
    roundcorner=5pt
]{excer}{Ejercicio}[section]

%En esta parte se colocan comandos que definen la forma en la que se van a escribir ciertas funciones%

\newcommand\abs[1]{\ensuremath{\left|#1\right|}}
\newcommand\divides{\ensuremath{\bigm|}}
\newcommand\cf[3]{\ensuremath{#1:#2\rightarrow#3}}
\newcommand\contradiction{\ensuremath{\#_c}}
\newcommand\natint[1]{\ensuremath{\left[\big|#1\big|\right]}}
\newcounter{figcount}
\setcounter{figcount}{1}

\begin{document}

\chapter{Posibles Salidas Laborales}

¡Qué interesante tu perfil! Con una formación en física y matemáticas, especialmente en matemáticas, tienes muchas opciones laborales en una variedad de áreas. Te haré una lista de posibles puestos y qué podrías estudiar o aprender para prepararte para esos roles:

\section{Analista de Datos / Científico de Datos (Data Scientist)}

Puestos posibles:

\begin{itemize}
    \item Data Analyst
    \item Data Scientist
    \item Research Scientist
\end{itemize}


Qué puedes aprender:

\begin{itemize}
    \item Lenguajes de programación: Python, R, SQL
    \item Herramientas de análisis de datos: Pandas, Numpy, SciPy
    \item Machine Learning: scikit-learn, TensorFlow, Keras, PyTorch
    \item Estadística avanzada (Modelos de regresión, probabilidades, etc.)
    \item Big Data: Apache Hadoop, Spark
\end{itemize}

Libros y recursos recomendados:

\begin{itemize}
    \item "Python for Data Analysis" por Wes McKinney
    \item "Introduction to Statistical Learning" por Gareth James
    \item Cursos en plataformas como Coursera (por ejemplo, el de Data Science de Johns Hopkins).
\end{itemize}


\section{Desarrollador de Software}

Puestos posibles:
\begin{itemize}
    \item Software Engineer
    \item Backend Developer
    \item Frontend Developer
\end{itemize}


Qué puedes aprender:

\begin{itemize}
    \item Lenguajes de programación: Python, C++, Java, JavaScript, Rust
    \item Estructuras de datos y algoritmos (tema clave en matemáticas)
    \item Bases de datos y SQL
    \item Desarrollo de software en la nube: AWS, Azure
\end{itemize}


Libros y recursos recomendados:

\begin{itemize}
    \item "Clean Code" por Robert C. Martin
    \item "Cracking the Coding Interview" por Gayle Laakmann McDowell
    \item Cursos en plataformas como Udemy o edX.
\end{itemize}


\section{Consultoría Cuantitativa / Analista Financiero Cuantitativo (Quantitative Analyst)}

Puestos posibles:

\begin{itemize}
    \item Quantitative Analyst
    \item Quantitative Researcher
\end{itemize}


Qué puedes aprender:

\begin{itemize}
    \item Modelos estadísticos y financieros (Modelos de valoración de opciones, derivados, etc.)
    \item Lenguajes de programación: Python, R, C++, MATLAB
    \item Teoría de probabilidad avanzada y estadística
\end{itemize}

Econometría y teoría financiera


Libros y recursos recomendados:

\begin{itemize}
    \item "Quantitative Finance For Dummies" por Steve Bell
    \item "The Concepts and Practice of Mathematical Finance" por Mark S. Joshi
    \item Cursos de Finanzas Cuantitativas en plataformas como Coursera o Khan Academy.
\end{itemize}


\section{Investigador en Matemáticas Aplicadas}

Puestos posibles:

\begin{itemize}
    \item Investigador en Institutos de Investigación
    \item Profesor universitario
\end{itemize}


Qué puedes aprender:

\begin{itemize}
    \item Teoría de grupos, álgebra abstracta, topología, etc.
    \item Métodos numéricos y simulación computacional
    \item Investigación y publicaciones académicas
\end{itemize}


Libros y recursos recomendados:
\begin{itemize}
    \item "Numerical Methods for Engineers" por Steven C. Chapra
    \item "A Course in Mathematical Methods for Physicists" por Peter Arfken
\end{itemize}


\section{Ingeniero de Investigación Operativa}

Puestos posibles:

\begin{itemize}
    \item Operations Research Analyst
    \item Optimization Engineer
\end{itemize}


Qué puedes aprender:

\begin{itemize}
    \item Optimización matemática y programación lineal
    \item Teoría de colas y simulación
    \item Herramientas de software: Gurobi, CPLEX, MATLAB
\end{itemize}


Libros y recursos recomendados:

\begin{itemize}
    \item "Introduction to Operations Research" por Frederick S. Hillier
    \item Cursos en Coursera sobre optimización y investigación operativa.
\end{itemize}


\section{Desarrollador de Algoritmos y Matemáticas Computacionales}

Puestos posibles:

\begin{itemize}
    \item Algorithm Engineer
    \item Research Engineer en empresas tecnológicas
\end{itemize}


Qué puedes aprender:

\begin{itemize}
    \item Estructuras de datos y algoritmos avanzados
    \item Computación en paralelo y optimización
    \item Teoría de grafos y complejidad computacional
\end{itemize}


Libros y recursos recomendados:

\begin{itemize}
    \item "Introduction to Algorithms" por Cormen, Leiserson, Rivest y Stein (conocido como CLRS)
    \item Cursos de algoritmos en edX o Coursera (por ejemplo, los de Stanford).
\end{itemize}


\section{Ingeniero de Machine Learning / IA}

Puestos posibles:

\begin{itemize}
    \item Machine Learning Engineer
    \item AI Researcher
\end{itemize}


Qué puedes aprender:

\begin{itemize}
    \item Redes neuronales profundas, aprendizaje supervisado y no supervisado
    \item Deep Learning y redes neuronales convolucionales (CNN)
    Frameworks: TensorFlow, PyTorch
\end{itemize}


Libros y recursos recomendados:

\begin{itemize}
    \item "Hands-On Machine Learning with Scikit-Learn, Keras, and TensorFlow" por Aurélien Géron
    \item "Deep Learning" por Ian Goodfellow, Yoshua Bengio, y Aaron Courville
\end{itemize}


\section{Ingeniero en Criptografía}

Puestos posibles:

\begin{itemize}
    \item Cryptography Engineer
    \item Security Researcher
\end{itemize}


Qué puedes aprender:

\begin{itemize}
    \item Teoría de números y álgebra abstracta (importante para criptografía)
    \item Protocolos de seguridad y cifrado
    \item Lenguajes de programación: C++, Python, Rust
\end{itemize}


Libros y recursos recomendados:

\begin{itemize}
    \item "Introduction to Modern Cryptography" por Jonathan Katz y Yehuda Lindell
    \item Cursos en Coursera o edX sobre criptografía.
\end{itemize}


\section{Matemático en el sector tecnológico o de investigación}

Puestos posibles:

\begin{itemize}
    \item Researcher en empresas tecnológicas como Google, Facebook, etc.
    \item Matemático computacional
\end{itemize}


Qué puedes aprender:

\begin{itemize}
    \item Teoría avanzada de computación (computación cuántica, computación algorítmica)
    \item Métodos numéricos y simulaciones en física y matemáticas
\end{itemize}

Libros y recursos recomendados:

\begin{itemize}
    \item "Mathematics for Computer Science" de Eric Lehman, F. Thomson Leighton y Albert R. Meyer
    \item "The Art of Computer Programming" de Donald E. Knuth
\end{itemize}

\section{Sugerencias adicionales}

Networking: Participar en conferencias, foros académicos, y en eventos de la industria te ayudará a conocer posibles oportunidades de trabajo.

Proyectos personales y portafolio: Trabajar en proyectos que puedas mostrar a posibles empleadores (por ejemplo, en GitHub si te interesa la programación).

Con tus conocimientos previos y estas herramientas/cursos, ¡sin duda puedes acceder a muchas de estas áreas! Si tienes alguna preferencia o te gustaría que profundice en alguna de estas áreas, estaré encantado de ayudarte.

\end{document}