\documentclass[../../comAlgebra.tex]{subfiles}

\begin{document}

    \chapter{Anillos e Ideales}

    Muchos de los resultados que se usarán se han visto en el curso de Álgebra Moderna II, por lo que solo se incluirán resultados nuevos.

    A lo largo de todo el documento, todo anillo será un anillo conmutativo con identidad.

  \section{Nilradical y Radical de Jacobson}

    \begin{mydef}
        Sea $A$ un anillo. Un elemento $x\in A$ es llamado \textbf{nilpotente} si $x^n=0$ para algún $n\in\mathbb{N}$
    \end{mydef}

    \begin{obs}
        Todo elemento nilpotente es divisor de cero, sin embargo el converso no es cierto.
    \end{obs}

    \begin{propo}
        El conjunto $\mathfrak{N}$ de todos los elementos nilpotentes de un anillo $A$ es un ideal, y el ideal cociente $A/\mathfrak{N}$ no tiene elmentos nilpotentes distintos de cero.
    \end{propo}

    \begin{proof}
        Veamos que $\mathfrak{N}$ es un ideal.
        \begin{enumerate}[label = \textit{(\arabic*)}]
            \item Sean $x,y\in\mathfrak{N}$, entonces existen $m,n\in\mathbb{N}$ tales que $x^n=y^m=0$. Por ende:
            \begin{equation*}
                (x+y)^{ n+m}=0
            \end{equation*}
            pues, en el desarrollo binomial de esta expresión, todo término es de la forma $c_{(r,s)}x^ry^s$ con $c_{(r,s)}\in\mathbb{N}$ el cual además cumple que
            \begin{equation*}
                r+s=n+m
            \end{equation*}
            con $r,s\in\natint{0,n+m}$. Si $r<n$ entonces debe suceder que $s>m$, luego $y^s=0$. Si $r>n$ se sigue que $x^r=0$. En cualquier caso, todos los coeficientes de la forma $x^ry^s=0$, lo cual prueba lo enunciado.
            \item Sea $x\in\mathfrak{N}$, entonces existe $n\in\mathbb{N}$ tal que $x^n=0$, entonces:
            \begin{equation*}
                (ax)^n=a^nx^n=0
            \end{equation*}
            por lo que $ax\in\mathfrak{N}$.
        \end{enumerate}
        Por los dos incisos anteriores se sigue que $\mathfrak{N}$ es un ideal de $A$. Sea $\mathfrak{N}+x\in A/\mathfrak{N}$ con $x\in A$ tal que
        \begin{equation*}
            (\mathfrak{N}+x)^n=\mathfrak{N}
        \end{equation*}
        para algún $n\in\mathbb{N}$, entonces:
        \begin{equation*}
            \mathfrak{N}+x^n=\mathfrak{N}\Rightarrow x^n\in\mathfrak{N}
        \end{equation*}
        luego existe $k\in\mathbb{N}$ tal que $(x^n)^k=0$, esto es que $x\in\mathfrak{N}$, por lo que
        \begin{equation*}
            \mathfrak{N}+x=\mathfrak{N}
        \end{equation*}
    \end{proof}

    \begin{mydef}
        El ideal de la proposición anterior es llamado el \textbf{nilradical de $A$} cuando se trabaje con varios anillos, será denotado por $\mathfrak{N}_A$.
    \end{mydef}

    Resulta que podemos caracterizar de otra manera al nilradical $\mathfrak{N}$:

    \begin{propo}
        El nilradical $\mathfrak{N}$ de $A$ es la intersección de todos los ideales primos de $A$.
    \end{propo}

    \begin{proof}
        Sea $\mathfrak{N}'$ la intersección de todos los ideales primos de $A$. Se tiene que este es un ideal de $A$.
        \begin{itemize}
            \item Si $x\in A$ es tal que $x\in\mathfrak{N}$, entonces existe $n\in\mathbb{N}$ tal que $x^n=0$. Como $0\in\mathfrak{N}'$ y $\mathfrak{N}'$, entonces
            \begin{equation*}
                x^n\in\mathfrak{p}
            \end{equation*}
            para todo ideal primo $\mathfrak{p}$ de $A$, luego por inducción se tiene que $x\in\mathfrak{p}$, es decir que $x\in\mathfrak{N}'$.
            \item Sea $x\in A$ tal que $x\notin\mathfrak{N}$, probaremos que $x\notin\mathfrak{N}'$. Sea
            \begin{equation*}
                \Sigma=\left\{\mathfrak{a}\Big|\mathfrak{a}\textup{ es ideal de $A$ tal que }n\in\mathbb{N}\Rightarrow x^n\notin\mathfrak{a} \right\}
            \end{equation*}
            el conjunto $\Sigma$ es no vacío, pues $\langle 0\rangle\in\Sigma$. Sea $\mathcal{C}$ una cadena de elementos de $\Sigma$. Como
            \begin{equation*}
                \mathcal{C}=\left\{I_n \right\}_{ n=1}^\infty
            \end{equation*}
            es tal que $I_1\subseteq I_2\subseteq\cdots\subseteq I_m\subseteq\cdots$, se sigue de una proposición que
            \begin{equation*}
                I=\bigcup_{ n=1}^\infty I_n
            \end{equation*} 
            es un ideal de $A$, el cual debe estar en $\Sigma$. Por el Lema de Zorn, este conjunto tiene elementos maximales, digamos $\mathfrak{p}$. Es claro que $x^n\notin\mathfrak{p}$ para todo $n\in\mathbb{N}$. Veamos que $\mathfrak{p}$ es primo.

            En efecto, sean $y,z\notin\mathfrak{p}$, entonces los ideales
            \begin{equation*}
                \mathfrak{p}+\langle y\rangle\textup{ y }\mathfrak{p}+\langle z\rangle
            \end{equation*}
            son dos ideales de $A$ que contienen propiamente a $\mathfrak{p}$, por lo que $x\in\mathfrak{p}+\langle y\rangle$ y $x\in\mathfrak{p}+\langle z\rangle$, luego existen $m,n\in\mathbb{N}$ tales que
            \begin{equation*}
                x^n\in\mathfrak{p}+\langle y\rangle,\quad\textup{y}\quad x^m\in\mathfrak{p}+\langle z\rangle
            \end{equation*}
            por ende,
            \begin{equation*}
                x^{ n+m}\in\left(\mathfrak{p}+\langle y\rangle \right)\left(\mathfrak{p}+\langle z\rangle \right)=\mathfrak{p}+\langle yz\rangle
            \end{equation*}
            por tanto, $\mathfrak{p}+\langle yz\rangle$ contiene propiamente a $\mathfrak{p}$, luego no puede estar en $\Sigma$, así que $yz\notin\mathfrak{p}$. Se sigue entonces que $\mathfrak{p}$ es primo. Por tanto, $x\notin\mathfrak{N}'$.
        \end{itemize}
        Por los dos incisos anteriores se sigue que $\mathfrak{N}=\mathfrak{N}'$.
    \end{proof}

    \begin{mydef}
        Sea $A$ un anillo, el \textbf{radical de Jacobson $\mathfrak{R}$ de $A$} se define como la intersección de todos los ideales maximales de $A$. Cuando se trabaje con varios anillos, será denotado por $\mathfrak{R}_A$.
    \end{mydef}

    El radical de Jacobson (que claramente es un ideal), se caracteriza de la siguiente manera:

    \begin{propo}
        Sea $A$ un anillo. Entonces, $x\in\mathfrak{R}$ si y sólo si $1-xy$ es una unidad en $A$ para todo $y\in A$.
    \end{propo}

    \begin{proof}
        $\Rightarrow):$ Suponga que existe $y\in A$ tal que $1-xy$ no es una unidad de $A$, entonces existe un ideal maximal que contiene a $1-xy$, digamos $\mathfrak{m}$, pero $x\in\mathcal{R}$, en particular $x\in\mathfrak{m}$ por lo que $xy\in\mathfrak{m}$ lo cual implica que $1\in\mathfrak{m}$, lo cual no puede suceder pues $\mathfrak{m}$ es ideal maximal.

        $\Leftarrow):$ Suponga que existe un ideal maximal $\mathfrak{m}$ tal que $x\notin\mathfrak{m}$. Entonces,
        \begin{equation*}
            \mathfrak{m}+\langle x\rangle=\langle\mathfrak{m}+x\rangle=A=\langle 1\rangle
        \end{equation*}
        es decir, existe $u\in\mathfrak{m}$ y $y\in A$ tales que
        \begin{equation*}
            u+xy=1
        \end{equation*}
        luego, $1-xy$ no puede ser unidad de $A$.
    \end{proof}

    \begin{mydef}
        Sea $A$ anillo y $\mathfrak{a}$ un ideal de $A$. Se define el \textbf{radical de $\mathfrak{a}$} como el conjunto
        \begin{equation*}
            r(\mathfrak{a})=\left\{x\in A\Big|x^n\in\mathfrak{a}\textup{ para algún }n\in\mathbb{N} \right\}
        \end{equation*}
    \end{mydef}

    \begin{propo}
        Dado un anillo $A$ un ideal $\mathfrak{a}$ de $A$, se tiene que $r(\mathfrak{a})$ es un ideal de $A$.
    \end{propo}

    \begin{proof}
        Considere el homomorfismo natural $\cf{\pi}{A}{A/\mathfrak{a}}$, afirmamos que
        \begin{equation*}
            r(A)=\pi^{-1}\left(\mathfrak{N}_{ A/\mathfrak{a}} \right)
        \end{equation*}
        donde $\mathfrak{N}_{ A/\mathfrak{a}}$ es el nilradical de $A/\mathfrak{a}$. En efecto, veamos que:
        $x\in r(\mathfrak{a})$ si y sólo si existe $n\in\mathbb{N}$ tal que $x^n\in\mathfrak{a}$, lo cual ocurre si y sólo si
        \begin{equation*}
            \left(\mathfrak{a}+x\right)^n=\mathfrak{a}+x^n=\mathfrak{a}
        \end{equation*}
        si y sólo si $\mathfrak{a}+x$ es un elemento nilpotente de $A/\mathfrak{a}$, si y sólo si $\mathfrak{a}+x\in\mathfrak{N}_{ A/\mathfrak{a}}$, si y sólo si $x\in\pi^{-1}\left(\mathfrak{N}_{ A/\mathfrak{a}} \right)$.

        Lo anterior prueba la igualdad.
    \end{proof}

    \begin{excer}
        Sea $A$ un anillo y $\mathfrak{a},\mathfrak{b}$ ideales de $A$. Se cumple lo siguiente:
        \begin{enumerate}[label = \textit{(\arabic*)}]
            \item $\mathfrak{a}\subseteq r(\mathfrak{a})$.
            \item $r(r(a))=r(\mathfrak{a})$.
            \item $r(\mathfrak{a}\mathfrak{b})=r(\mathfrak{a}\cap\mathfrak{b})=r(\mathfrak{a})\cap r(\mathfrak{b})$.
            \item $r(\mathfrak{a})=\gen{1}$ si y sólo si $\mathfrak{a}=\gen{1}$.
            \item $r(\mathfrak{a}+\mathfrak{b})=r(r(\mathfrak{a})+r(\mathfrak{b}))$.
            \item Si $\mathfrak{p}$ es un ideal primo de $A$, entonces $r(\mathfrak{p}^n)=\mathfrak{p}$ para todo $n>0$.
        \end{enumerate}
    \end{excer}

    \begin{proof}
        De \textit{(1)}: Tomemos $x\in\mathfrak{a}$, entonces $x^1\in\mathfrak{a}$, así que $x\in r(\mathfrak{a})$.

        De \textit{(2)}: Por el inciso anterior ya se tiene que $r(r(\mathfrak{a}))\subseteq r(\mathfrak{a})$. Ahora, si $x\in r(\mathfrak{a})$, entonces existe $1\in\bbm{N}$ tal que $x^1\in r(\mathfrak{a})$, así que $x\in r(r(\mathfrak{a}))$. Se sigue así la igualdad.

        De \textit{(3)}: Haremos la demostración por partes:
        \begin{itemize}
            \item Primero, probaremos que $r(\mathfrak{a}\mathfrak{b})=r(\mathfrak{a}\cap\mathfrak{b})$. En efecto, sea $x\in r(\mathfrak{a}\mathfrak{b})$, entonces existe $n\in\mathbb{N}$ tal que:
            \begin{equation*}
                x^n\in\mathfrak{a}\mathfrak{b}
            \end{equation*}
            Por ser $\mathfrak{a}$ y $\mathfrak{b}$ ideales, se sigue que $x^n\in\mathfrak{a}$ y $x^n\in\mathfrak{b}$, por lo cual $x\in r(\mathfrak{a})\cap r(\mathfrak{b})$ y $x\in r(\mathfrak{a}\cap\mathfrak{b})$.
            \item Si $x\in r(\mathfrak{a})\cap r(\mathfrak{b})$, entonces existen $n,m\in\bbm{N}$ tales que $x^n\in\mathfrak{a}$ y $x^m\in\mathfrak{b}$, luego $x^{ nm}\in \mathfrak{ab}$, lo cual implica que $x\in r(\mathfrak{ab})$.
            
            Si $x\in r(\mathfrak{a}\cap\mathfrak{b})$, entonces existe $n\in\bbm{N}$ tal que $x^n\in \mathfrak{a}\cap\mathfrak{b}$, así que $x^{2n}\in\mathfrak{ab}$, esto es que $x\in r(\mathfrak{ab})$.
        \end{itemize}
        Por ambos casos se sigue la doble igualdad.

        De \textit{(6)}: Sea $\mathfrak{p}$ un ideal primo de $A$. Procederemos por inducción sobre $n$.
        \begin{itemize}
            \item Por \textit{(1)} se tiene que $\mathfrak{p}\subseteq r(\mathfrak{p})$. Sea $x\in r(\mathfrak{p})$, entonces existe $m\in\bbm{N}$ tal que $x^m\in\mathfrak{p}$. Esto implica que $x\in\mathfrak{p}$ (tal hecho se verifica usando inducción). De esta forma se sigue la igualdad.
            \item Supongamos que lo anterior se cumple para algún $n\in\bbm{N}$. Sea $x\in r(\mathfrak{p}^{ n+1})$, entonces existe $m\in\mathbb{N}$ tal que $x^m\in\mathfrak{p}^{ n+1}\subseteq\mathfrak{p}$, así que $x\in\mathfrak{p}$.
        \end{itemize}
        %TODO: no se uso induccion, corregir
        De ambos incisos se sigue la igualdad.
    \end{proof}

    \begin{obs}
        Si $I$ y $J$ son ideales, entonces $IJ$ es un ideal, siendo este último definido por:
        \begin{equation*}
            IJ=\left\{\sum_{ k=1}^n i_kj_k\Big|i_k\in I\textup{ and } j_k\in J,\forall k\in\natint{1,n};n\in\bbm{N} \right\}
        \end{equation*}
    \end{obs}
    
    \begin{propo}
        \label{propo:caracterizacion_radical_ideal}
        El radical de un ideal $\mathfrak{a}$ es la intersección de todos ideales primos que contienen a $\mathfrak{a}$.
    \end{propo}

    \begin{proof}
        Consideremos el anillo $A/\mathfrak{a}$. El nilradical $\mathfrak{N}$ de $A/\mathfrak{a}$ es la intersección de todos los ideales primos de $A/\mathfrak{a}$. Ahora, por el teorema de correspondencia, para cada ideal primo $\mathfrak{p}$ de $A/\mathfrak{a}$ existe un único ideal primo $P$ en $A$ tal que $\mathfrak{a}\subseteq P$.

        Afirmamos que:
        \begin{equation*}
            \mathfrak{N}=r(\mathfrak{a})/\mathfrak{a}
        \end{equation*}
        En efecto, sea $\mathfrak{a}+x\in\mathfrak{N}$, entonces existe $n\in\bbm{N}$ tal que $\mathfrak{a}+x^n=\mathfrak{a}$, esto es que $x^n\in r(\mathfrak{a})$. 
        
        Si $\mathfrak{a}+x\in r(\mathfrak{a})/\mathfrak{a}$, entonces $x\in r(\mathfrak{a})$, luego existe $n\in\bbm{N}$ tal que $x^n\in\mathfrak{a}$, esto es que $\mathfrak{a}+x^n=\mathfrak{a}$, luego $\mathfrak{a}+x^n=\mathfrak{a}$, es decir que $\mathfrak{a}+x\in\mathfrak{N}$. Se sigue así la contención.
    \end{proof}

    \section{Ideales Primos}

    Sean $A$ un anillo y $\mathfrak{a}_1,\dots,\mathfrak{a}_n$ ideales de $A$. Se define el homomorfismo:
    \begin{equation*}
        \cf{\phi}{A}{\prod_{ i=1}^{n}(A/\mathfrak{a}_i)}
    \end{equation*}
    Dado por: $\phi(x)=(x+\mathfrak{a}_1,\dots,x+\mathfrak{a}_n)$.

    \begin{propo}
        Sean $A$ un ideal y $\mathfrak{a}_1,\dots,\mathfrak{a}_n$ ideales de $A$. Entonces:
        \begin{equation*}
            s
        \end{equation*}
    \end{propo}
    

\end{document}