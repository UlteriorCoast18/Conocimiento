\documentclass[../../introduccion.tex]{subfiles}

\begin{document}

    \chapter*{Ejercicios}

    \section{Ejercicios}

    Ejercicios de cada una de las secciones.

    \subsection{Series de Potencias}

    \begin{excer}
        Pruebe que si $z_1,z_2\in\mathbb{C}$, entonces:
        \begin{equation*}
            e^{z_1+z_2}=e^{ z_1}e^{ z_1}
        \end{equation*}
    \end{excer}

    \begin{proof}
        En efecto, veamos que:
        \begin{equation*}
            \begin{split}
                e^{ x+iz_2}&=\sum_{ n=0}^\infty\frac{(z_1+z_2)^n}{n!}\\
                &=\sum_{ n=0}^\infty\frac{1}{n!}\sum_{ k=0}^n\binom{n}{k}z_1^kz_2^{ n-k}\\
                &=\sum_{ n=0}^\infty\sum_{ k=0}^n\frac{1}{n!}\cdot\frac{n!}{k!(n-k)!}z_1^kz_2^{ n-k}\\
                &=\sum_{ n=0}^\infty\sum_{ k=0}^n\frac{z_1^k}{k!}\cdot\frac{z_2^{n-k}}{(n-k)!}\\
            \end{split}
        \end{equation*}
        donde, recordemos que:
        \begin{equation*}
            e^z_1=\sum_{ n=0}^\infty\frac{z_1^n}{n!}\quad\textup{y}\quad e^ {z_2}=\sum_{ n=0}^\infty\frac{z_2^n}{n!}
        \end{equation*}
        por tanto, de la Proposición \ref{MultSumSeriesPot} se sigue que:
        \begin{equation*}
            e^{ z_1+z_2}=e^z_1e^{ z_2}
        \end{equation*}
    \end{proof}

    \begin{excer}
        Pruebe que
        \begin{equation*}
            e^z=e^x(\cos y+i\sin y)
        \end{equation*}
        donde $z=x+iy$.
    \end{excer}

    \begin{proof}
        Sea $z\in\mathbb{C}$. Se tiene que:
        \begin{equation*}
            \begin{split}
                e^z&=e^{x}e^{iy}
            \end{split}
        \end{equation*}
        Veamos que:
        \begin{equation*}
            \begin{split}
                e^{ iy}&=\sum_{n=0}^\infty\frac{(iy)^n}{n!}\\
                &=\sum_{n=0}^\infty\frac{(iy)^{2k}}{(2k)!}+\sum_{k=0}^\infty\frac{(iy)^{2k+1}}{(2k+1)!}\\
            \end{split}
        \end{equation*}
    \end{proof}

    \begin{excer}
        \label{productoSucesionesLimSupLim}
        Pruebe que si $\left\{a_n\right\}_{ n=1}^\infty$ y $\left\{b_n\right\}_{ n=1}^\infty$ son dos sucesiones de números no negativos tales que $0\leq b=\lim_{ n\rightarrow\infty}b_n$ y $a=\limsup_{ n\rightarrow\infty}a_n$, entonces:
        \begin{equation*}
            \limsup_{ n\rightarrow\infty}(a_nb_n)=ab
        \end{equation*}
    \end{excer}

    \begin{proof}
        Antes, notemos que al tenerse:
        \begin{equation*}
            \limsup_{ n\rightarrow\infty}a_n=a
        \end{equation*}
        se tiene que el siguiente límite existe:
        \begin{equation*}
            \lim_{ n\rightarrow\infty}\left(\sup_{ k\geq n}a_k \right)
        \end{equation*}
        por tanto, la sucesión $\left\{\sup_{ k\geq n}a_k\right\}_{ n=1}^\infty$ es acotada. Así que, para cada $n\in\mathbb{N}$, se tiene que el supremo:
        \begin{equation*}
            \sup_{ k\geq n}a_k
        \end{equation*}
        existe. Veamos ahora que:
        \begin{equation*}
            \begin{split}
                \limsup_{ n\rightarrow\infty}(a_nb_n)&=\lim_{ n\rightarrow\infty}\left(\sup_{ k\geq n}a_kb_k \right)\\
                &=\lim_{ n\rightarrow\infty}\left(\left(\sup_{ k\geq n}a_k\right)\left(\sup_{ k\geq n}b_k\right) \right)\\
            \end{split}
        \end{equation*}
        donde el supremo se puede separar ya que ambos supremos existen y ser las dos sucesiones acotadas y de números no negativos. Por ende:
        \begin{equation*}
            \begin{split}
                \limsup_{ n\rightarrow\infty}(a_nb_n)&=\lim_{ n\rightarrow\infty}\left(\left(\sup_{ k\geq n}a_k\right)\left(\sup_{ k\geq n}b_k\right) \right)\\
                &=\lim_{ n\rightarrow\infty}\left(\sup_{ k\geq n} a_k\right)\cdot\lim_{ n\rightarrow\infty}\left(\sup_{ k\geq n}b_k\right)\\
                &=\left(\limsup_{ n\rightarrow\infty}a_n\right)\cdot\left(\limsup_{ n\rightarrow\infty}b_n\right)\\
                &=\left(\limsup_{ n\rightarrow\infty}a_n\right)\cdot\left(\lim_{ n\rightarrow\infty}b_n\right)\\
                &=ab\\
            \end{split}
        \end{equation*}
    \end{proof}

    \begin{excer}
        \label{normalDerivadaComplejaDerivadaCompos}
        Sea $\cf{f}{G}{\bbm{C}}$ analítica en $G$ y $\cf{h}{[0,1]}{G}$ función real diferenciable. Entonces la función $\cf{f\circ h}{[0,1]}{\bbm{C}}$ cumple que:
        \begin{equation*}
            \lim_{ t\rightarrow s}\frac{f\circ h(s)-f\circ h(t)}{s-t}=f'(h(s))\cdot h'(s)
        \end{equation*}
        para todo $s\in[0,1]$.
    \end{excer}

    \begin{proof}
        %TODO
    \end{proof}

    \begin{excer}
        Pruebe que la función $u(x,y)=\log(x^2+y^2)^{\frac{1}{2}}$ es armónica en $G=\bbm{C}\setminus\left\{0\right\}$.
    \end{excer}

    \begin{proof}
        %TODO
    \end{proof}

    \subsection{Funciones Analíticas}

    \begin{excer}
        Pruebe que la función $\cf{f(z)}{\bbm{C}}{\bbm{R}}$ dada por $f(z)=\abs{z}^2$ tiene derivada solamente en cero.
    \end{excer}

    \begin{proof}
        Sea $z_0=x_0+iy_0\in\bbm{C}$. Observemos que:
        \begin{equation*}
            f(z_0)=x_0^2+y_0^2=u(x_0,y_0)+iv(x_0,y_0)
        \end{equation*}
        Donde $u(x,y)=x^2+y^2$ y $v(x,y)=0$. Ahora, notemos también que:
        \begin{equation*}
            \begin{split}
                u_x=2x\quad&\textup{y}\quad u_y=2y\\
                v_x=0\quad&\textup{y}\quad v_y=0\\
            \end{split}
        \end{equation*}
        Estas derivadas parciales cumplen (\ref{ecuacionesCauchyRiemann}) si y solo si $x,y=0$, esto es que $f$ es diferenciable en $z_0$ si y solo si $z_0=0$.
    \end{proof}

    \begin{excer}[]
        
    \end{excer}

    \chapter*{Bibliografía}

    \begin{itemize}
        \item A. Markusevich, \textit{Teoría de las funciones analíticas}, Ed. Mir Moscu.
        \item J. Conway, \textit{Complex Analysis}, Ed. Mir Moscu.
    \end{itemize}

\end{document}