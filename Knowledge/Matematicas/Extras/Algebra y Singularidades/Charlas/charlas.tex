\documentclass[12pt]{report}
\usepackage[spanish]{babel}
\usepackage[utf8]{inputenc}
\usepackage{amsmath}
\usepackage{amssymb}
\usepackage{amsthm}
\usepackage{graphics}
\usepackage{subfigure}
\usepackage{lipsum}
\usepackage{array}
\usepackage{multicol}
\usepackage{enumerate}
\usepackage[framemethod=TikZ]{mdframed}
\usepackage[a4paper, margin = 1.5cm]{geometry}
\usepackage{tikz}
\usepackage{pgffor}
\usepackage{ifthen}
\usepackage{enumitem}

\usetikzlibrary{shapes.multipart}

\newcounter{it}
\newcommand*\watermarktext[1]{\begin{tabular}{c}
    \setcounter{it}{1}%
    \whiledo{\theit<100}{%
    \foreach \col in {0,...,15}{#1\ \ } \\ \\ \\
    \stepcounter{it}%
    }
    \end{tabular}
    }

\AddToHook{shipout/foreground}{
    \begin{tikzpicture}[remember picture,overlay, every text node part/.style={align=center}]
        \node[rectangle,black,rotate=30,scale=2,opacity=0.04] at (current page.center) {\watermarktext{Cristo Daniel Alvarado ESFM\quad}};
  \end{tikzpicture}
}
%En esta parte se hacen redefiniciones de algunos comandos para que resulte agradable el verlos%

\def\proof{\paragraph{Demostración:\\}}
\def\endproof{\hfill$\blacksquare$}

\def\sol{\paragraph{Solución:\\}}
\def\endsol{\hfill$\square$}

%En esta parte se definen los comandos a usar dentro del documento para enlistar%

\newtheoremstyle{largebreak}
  {}% use the default space above
  {}% use the default space below
  {\normalfont}% body font
  {}% indent (0pt)
  {\bfseries}% header font
  {}% punctuation
  {\newline}% break after header
  {}% header spec

\theoremstyle{largebreak}

\newmdtheoremenv[
    leftmargin=0em,
    rightmargin=0em,
    innertopmargin=0pt,
    innerbottommargin=5pt,
    hidealllines = true,
    roundcorner = 5pt,
    backgroundcolor = gray!60!red!30
]{exa}{Ejemplo}[section]

\newmdtheoremenv[
    leftmargin=0em,
    rightmargin=0em,
    innertopmargin=0pt,
    innerbottommargin=5pt,
    hidealllines = true,
    roundcorner = 5pt,
    backgroundcolor = gray!50!blue!30
]{obs}{Observación}[section]

\newmdtheoremenv[
    leftmargin=0em,
    rightmargin=0em,
    innertopmargin=0pt,
    innerbottommargin=5pt,
    rightline = false,
    leftline = false
]{theor}{Teorema}[section]

\newmdtheoremenv[
    leftmargin=0em,
    rightmargin=0em,
    innertopmargin=0pt,
    innerbottommargin=5pt,
    rightline = false,
    leftline = false
]{propo}{Proposición}[section]

\newmdtheoremenv[
    leftmargin=0em,
    rightmargin=0em,
    innertopmargin=0pt,
    innerbottommargin=5pt,
    rightline = false,
    leftline = false
]{cor}{Corolario}[section]

\newmdtheoremenv[
    leftmargin=0em,
    rightmargin=0em,
    innertopmargin=0pt,
    innerbottommargin=5pt,
    rightline = false,
    leftline = false
]{lema}{Lema}[section]

\newmdtheoremenv[
    leftmargin=0em,
    rightmargin=0em,
    innertopmargin=0pt,
    innerbottommargin=5pt,
    roundcorner=5pt,
    backgroundcolor = gray!30,
    hidealllines = true
]{mydef}{Definición}[section]

\newmdtheoremenv[
    leftmargin=0em,
    rightmargin=0em,
    innertopmargin=0pt,
    innerbottommargin=5pt,
    roundcorner=5pt
]{excer}{Ejercicio}[section]

%En esta parte se colocan comandos que definen la forma en la que se van a escribir ciertas funciones%

\newcommand\abs[1]{\ensuremath{\left|#1\right|}}
\newcommand\divides{\ensuremath{\bigm|}}
\newcommand\cf[3]{\ensuremath{#1:#2\rightarrow#3}}
\newcommand\contradiction{\ensuremath{\#_c}}
\newcommand\natint[1]{\ensuremath{\left[\big|#1\big|\right]}}

\begin{document}
    \setlength{\parskip}{5pt} % Añade 5 puntos de espacio entre párrafos
    \setlength{\parindent}{12pt} % Pone la sangría como me gusta
    \title{Charlas CIMAT 2024}
    \author{Cristo Daniel Alvarado}
    \maketitle

    \tableofcontents %Con este comando se genera el índice general del libro%

    %\setcounter{chapter}{3} %En esta parte lo que se hace es cambiar la enumeración del capítulo%

    \newpage

    \chapter{Hilbert, Fronenius y los cuadrados mágicos}

    Denotaremos por $K$ un campo algebraicamente cerrado de característica $p>0$. Sea $S=K[x_1,...,x_n]$ el anillo graduado por todos los polinomios homogéneos de grado $n$, esto es:
    \begin{equation*}
        [S]_n=\bigoplus_{ \abs{\alpha}=n}Kx^{\alpha}
    \end{equation*}
    Consideremos $I\subseteq S$ el deal homogéneo:
    \begin{equation*}
        [I]_n=I\cap[S]_n
    \end{equation*}
    podemos considerar así al ideal graduado dado por:
    \begin{equation*}
        I=\bigoplus_{ n=0}^\infty[I]_n
    \end{equation*}
    sea $R=S/I$, entonces:
    \begin{equation*}
        [R]_n=[S]_n/[I]_n
    \end{equation*}

    \begin{mydef}
        Se define la función de Hilbert de $R$, dada por:
        \begin{equation*}
            HF_R(n)=\dim_K([R]_n)
        \end{equation*}
        y su \textbf{serie de Hilbert} es:
        \begin{equation*}
            HS_R(t)=\sum_{ n=0}^\infty HF_R(n)t^n\in\mathbb{Q}[t]
        \end{equation*}
    \end{mydef}

    \begin{exa}
        Considere $R=K[x]$, se tiene que:
        \begin{equation*}
            HS_R(t)=\sum_{ n=0}^\infty t^n=\frac{1}{1-t}
        \end{equation*}
        pues,
        \begin{equation*}
            \left(\sum_{ n=0}^\infty t^n\right)\cdot(1-t)=1
        \end{equation*}
    \end{exa}

    \begin{excer}
        En el anillo $R=K[x_1,...,x_l]$, pruebe que:
        \begin{equation*}
            HF_R(n)=\left(\begin{array}{c}
                n+l-1\\
                n\\
            \end{array} \right)
        \end{equation*}
        por lo que,
        \begin{equation*}
            HS_R(t)=\frac{1}{(1-t)^l}
        \end{equation*}
    \end{excer}

    \begin{proof}
        
    \end{proof}

    \begin{theor}
        Sea $d=\dim(R)$, entonces:
        \begin{itemize}
            \item Existe $q(t)\in\mathbb{Q}[t]$ tal que $\deg(q)=d-1$ y $HF_R(n)=q(n)$ para todo $n>>0$.
            \item Existe $h(t)\in\mathbb{Q}[t]$ tal que
            \begin{equation*}
                HS_R(t)=\frac{h(n)}{(1-t)^d}
            \end{equation*}
        \end{itemize}
    \end{theor}

    \begin{excer}
        Se tiene que:
        \begin{equation*}
            HF_R(n)=q(n)\forall n\iff \deg(h)<d
        \end{equation*}
    \end{excer}

    \begin{obs}
        En el ejercicio anterior, veamos que podemos expresar a $h$ por:
        \begin{equation*}
            h(t)=\sum_{ n=0}^\infty c_n(t-1)^n
        \end{equation*}
    \end{obs}

    \section{Cuadrados Mágicos}

    \begin{mydef}
        Una matriz $A\in\mathcal{M}_{l\times l}(\mathbb{Z})$ es un \textbf{cuadrado mágico} si existe $n\in\mathbb{Z}$ tal que:
        \begin{equation*}
            \sum_{ j=1}^la_{ i,j}=\sum_{ i=1}^la_{ i,j}=n
        \end{equation*}
        para todo $i,j\in\left\{1,...,l\right\}$
    \end{mydef}

    \begin{theor}[\textbf{Teormea de Birkhoff-Von Neumann}]
        Si $A$ es un cuadrado mágico que suma $n$, entonces $A$ es combinación lineal entera $\geq0$ de matrizes de permutación.
    \end{theor}

    \begin{obs}
        Una permutación se ve como el vector columna:
        \begin{equation*}
            P_\sigma=[e_{\sigma(1)},...,e_{\sigma(l)}]
        \end{equation*}
        con $\sigma\in S_l$ y $e_1,...,e_l\in\mathbb{Z}^l$ son vectores columna.
    \end{obs}

    \begin{center}
        \textit{¿Cuántos cuadrados mágicos (denotado por $\square_l(newline)$) de $l\times l$ que suman $n$ existen?}
    \end{center}

    De forma inmediata uno deduce que:
    \begin{equation*}
        \square_l(0)=1,\quad\textup{y}\quad\square_l(1)=l!
    \end{equation*}

    \section{Vuelta al álgebra}

    Consideremos el anillo de polinomios $K[x_{ i,j}\Big|i,j\in\left\{1,...,l \right\}]$ (la idea es hacer una especie de anillo de matrices). Dada una $A\in\mathcal{M}_{ l\times l}(\mathbb{Z}_{\geq0})$, tenemos que:
    \begin{equation*}
        y^A=\prod_{ i,j}y_{ i,j}^{a_{i,j}}
    \end{equation*}

    \begin{exa}
        Se tine que:
        \begin{equation*}
            y^I=y_{ 1,1}y_{ 2,2}\cdots y_{ l,l}
        \end{equation*}
        y,
        \begin{equation*}
            y^{\left(
                \begin{array}{cc}
                    0 & 1 \\
                    1 & 0 \\
                \end{array}
            \right)}=y_{ 1,2}y_{ 2,1}
        \end{equation*}
    \end{exa}

    Sea ahora $T(l)=K[y^A\Big|A\textup{ es matriz de permutación}]$. Se tiene pues que:
    \begin{equation*}
        T(2)=K[ y_{ 1,1}y_{ 2,2},y_{ 1,2}y_{ 2,1}]
    \end{equation*}
    con esta nueva noción, se tiene el siguiente resultado:
    \begin{propo}
        $A$ es una matriz cuadrada si y sólo si $A=\sum_{ \sigma\in S_l}c_\sigma P_\sigma$, si y sólo si $y^A=\prod_{\sigma\in S_l}(y^{ P_\sigma})^{ c_\sigma}$ (siendo $c_\sigma\geq0$).
    \end{propo}

    Con esta nueva noción, se verifica rápidamente que:
    \begin{equation*}
        \square_l(n)=\dim_K([T(l)]_{ nl})
    \end{equation*}

    \begin{exa}
        Podemos ver en $T(2)$ simplemente al anillo
        \begin{equation*}
            K[x_1,x_2]\overset{\alpha}{\rightarrow}T(2)=K[y_{ 1,1}y_{ 2,2},y_{ 1,2}y_{ 2,1}]
        \end{equation*}
        tal que $x_1\mapsto y_{ 1,1}y_{ 2,2}$ y $x_2\mapsto y_{ 1,2}y_{ 2,1}$. De forma inmediata por el primer teorema de isomorfismos se sigue que:
        \begin{equation*}
            K[x_1,x_2]/\ker\alpha\cong T(2)
        \end{equation*}
        con lo que nos hemos quitado un montón de ceros que no nos sirven.
    \end{exa}

    \begin{obs}
        Generalizando este proceso, hacemos:
        \begin{equation*}
            K[x_\sigma\Big|\sigma\in S_l]\overset{\alpha_l}{\rightarrow}T(l)
        \end{equation*}
        tal que $x_\sigma\mapsto y^{ P_\sigma}$, lo que resulta en el isomorfismo:
        \begin{equation*}
            R(l)=K[x_\sigma\Big|\sigma\in S_l]/\ker(\alpha_l)\cong T(l)
        \end{equation*}
    \end{obs}

    Por esta razón, se simplifica el problema de cálculo simplemente a hacer:
    \begin{equation*}
        \square_l(n)=\dim_K([T(l)]_{ nl})=\dim_K[R(l)]_n
    \end{equation*}

    \begin{mydef}
        Sea $\underline{f}=f_1,...,f_u\in A$ una sucesión de elementos del anillo $A$. El \textbf{complejo de Cech de $\underline{f}$} (denotado por $\check{C}(\underline{f})$) se define como:
        \begin{equation*}
            0\rightarrow A\rightarrow\bigoplus_{ i=1}^u A\left[1/f_i \right]\rightarrow\bigoplus_{ i<j}A\left[1/f_i,1/f_j \right]\rightarrow\cdots\rightarrow A[1/f_i,1/f_j]\rightarrow 0
        \end{equation*}
        La \textbf{$i$-ésima cohomología local de $A$} en $\underline{f}$ es
        \begin{equation*}
            H_{\underline{f}}^i(A)=H^i(\check{C}(\underline{f}))
        \end{equation*}
    \end{mydef}

    Si ocupan saber más, mandar correo al Dr. que dió la plática.
    
    \begin{propo}
        Se tiene que:
        \begin{equation*}
            H_{\underline{f}}^i(A)=H_{\underline{g}}^i(A)
        \end{equation*}
        si $(\underline{f})=(\underline{g})$.
    \end{propo}

    \begin{propo}
        Sea $A=K[x]/I$ y $m=(\underline{x})$.
        \begin{itemize}
            \item $H_m^i(A)=0$ para todo $i>\dim A=d$.
            \item $H_m^j(A)\neq0$.
        \end{itemize}
    \end{propo}

    \begin{mydef}
        Decimos que $A$ es de \textbf{Cohen-Maculay} si
        \begin{equation*}
            H_m^i(A)\neq0\quad\forall i\neq d 
        \end{equation*}
    \end{mydef}

    \begin{center}
        \textit{¿Para qué se ocupa lo anterior?}
    \end{center}

    \begin{itemize}
        \item $V(I)$.
        \item Tiene procesos inductivos.
    \end{itemize}

    \begin{theor}
        $R(l)$ es Cohen-Maculay.        
    \end{theor}

    \begin{proof}
        La idea es que $R(l)$ es un sumando directo de $K[y^A\Big|a_{ i,j}\geq0]$.

        Luego se usa un teorema de Coxen-Maculay.
    \end{proof}

    Recordemos que el homomorfismo de Fröbenius:
    \begin{equation*}
        \begin{split}
            F:R(l)&\rightarrow R(l)\\
            f&\mapsto f^p\\
        \end{split}
    \end{equation*}
    (en campos de característica $p$).

    \begin{propo}
        El homomorfismo de Fröbenius $\cf{F}{R(l)}{R(l)}$ induce un morfismo en la cohomología:
        \begin{equation*}
            \cf{F}{H_m^i(R(l))}{H_m^i(R(l))}
        \end{equation*}
    \end{propo}

    \begin{theor}
        $q(t)\in\mathbb{Q}[t]$ es tal que $\square_l(n)=q(n)$, para todo $n\geq0$.
    \end{theor}

    \begin{proof}
        \begin{itemize}
            \item $\square_l(n)=HF_{ R(l)}(n)$.
            \item $\dim(R(l))=(l-1)^2+1$.
            \item $F$ en $H_m^i(R(l))$ es inyectivo.
        \end{itemize}
        Como $R(l)$ es Cohen-Maculay con proceso inductivos demuestra que $h$...
    \end{proof}

    \chapter{Un contraejemplo a la resolución de singularidades vía explosiones de Nash}

    \chapter{Sobre los teoremas de Cayley-Bacharach}

    \chapter{Título por anunciar}

\end{document}