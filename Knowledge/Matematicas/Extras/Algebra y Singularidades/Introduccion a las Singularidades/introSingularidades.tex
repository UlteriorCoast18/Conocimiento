\documentclass[12pt]{report}
\usepackage[spanish]{babel}
\usepackage[utf8]{inputenc}
\usepackage{amsmath}
\usepackage{amssymb}
\usepackage{amsthm}
\usepackage{graphics}
\usepackage{subfigure}
\usepackage{lipsum}
\usepackage{array}
\usepackage{multicol}
\usepackage{enumerate}
\usepackage[framemethod=TikZ]{mdframed}
\usepackage[a4paper, margin = 1.5cm]{geometry}
\usepackage{tikz}
\usepackage{pgffor}
\usepackage{ifthen}
\usepackage{enumitem}

\usetikzlibrary{shapes.multipart}

\newcounter{it}
\newcommand*\watermarktext[1]{\begin{tabular}{c}
    \setcounter{it}{1}%
    \whiledo{\theit<100}{%
    \foreach \col in {0,...,15}{#1\ \ } \\ \\ \\
    \stepcounter{it}%
    }
    \end{tabular}
    }

\AddToHook{shipout/foreground}{
    \begin{tikzpicture}[remember picture,overlay, every text node part/.style={align=center}]
        \node[rectangle,black,rotate=30,scale=2,opacity=0.04] at (current page.center) {\watermarktext{Cristo Daniel Alvarado ESFM\quad}};
  \end{tikzpicture}
}
%En esta parte se hacen redefiniciones de algunos comandos para que resulte agradable el verlos%

\def\proof{\paragraph{Demostración:\\}}
\def\endproof{\hfill$\blacksquare$}

\def\sol{\paragraph{Solución:\\}}
\def\endsol{\hfill$\square$}

%En esta parte se definen los comandos a usar dentro del documento para enlistar%

\newtheoremstyle{largebreak}
  {}% use the default space above
  {}% use the default space below
  {\normalfont}% body font
  {}% indent (0pt)
  {\bfseries}% header font
  {}% punctuation
  {\newline}% break after header
  {}% header spec

\theoremstyle{largebreak}

\newmdtheoremenv[
    leftmargin=0em,
    rightmargin=0em,
    innertopmargin=0pt,
    innerbottommargin=5pt,
    hidealllines = true,
    roundcorner = 5pt,
    backgroundcolor = gray!60!red!30
]{exa}{Ejemplo}[section]

\newmdtheoremenv[
    leftmargin=0em,
    rightmargin=0em,
    innertopmargin=0pt,
    innerbottommargin=5pt,
    hidealllines = true,
    roundcorner = 5pt,
    backgroundcolor = gray!50!blue!30
]{obs}{Observación}[section]

\newmdtheoremenv[
    leftmargin=0em,
    rightmargin=0em,
    innertopmargin=0pt,
    innerbottommargin=5pt,
    rightline = false,
    leftline = false
]{theor}{Teorema}[section]

\newmdtheoremenv[
    leftmargin=0em,
    rightmargin=0em,
    innertopmargin=0pt,
    innerbottommargin=5pt,
    rightline = false,
    leftline = false
]{propo}{Proposición}[section]

\newmdtheoremenv[
    leftmargin=0em,
    rightmargin=0em,
    innertopmargin=0pt,
    innerbottommargin=5pt,
    rightline = false,
    leftline = false
]{cor}{Corolario}[section]

\newmdtheoremenv[
    leftmargin=0em,
    rightmargin=0em,
    innertopmargin=0pt,
    innerbottommargin=5pt,
    rightline = false,
    leftline = false
]{lema}{Lema}[section]

\newmdtheoremenv[
    leftmargin=0em,
    rightmargin=0em,
    innertopmargin=0pt,
    innerbottommargin=5pt,
    roundcorner=5pt,
    backgroundcolor = gray!30,
    hidealllines = true
]{mydef}{Definición}[section]

\newmdtheoremenv[
    leftmargin=0em,
    rightmargin=0em,
    innertopmargin=0pt,
    innerbottommargin=5pt,
    roundcorner=5pt
]{excer}{Ejercicio}[section]

%En esta parte se colocan comandos que definen la forma en la que se van a escribir ciertas funciones%

\newcommand\abs[1]{\ensuremath{\left|#1\right|}}
\newcommand\divides{\ensuremath{\bigm|}}
\newcommand\cf[3]{\ensuremath{#1:#2\rightarrow#3}}
\newcommand\contradiction{\ensuremath{\#_c}}
\newcommand\natint[1]{\ensuremath{\left[\big|#1\big|\right]}}

\begin{document}
    \setlength{\parskip}{5pt} % Añade 5 puntos de espacio entre párrafos
    \setlength{\parindent}{12pt} % Pone la sangría como me gusta
    \title{Introducción a las singularidades}
    \author{Cristo Daniel Alvarado}
    \maketitle

    \tableofcontents %Con este comando se genera el índice general del libro%  

    \chapter{Nociones Básicas}

    \section{Preeliminares Algebraicos}

    \begin{mydef}
        Un anillo $R$ es \textbf{graduado (por $\mathbb{N}$)} si $R$ puede ser escrito como la suma directa (como grupo abeliano):
        \begin{equation*}
            R=\bigoplus_{ n=0}^\infty R_n
        \end{equation*}
        tal que para todos $m,n\in\mathbb{Z}_{\geq0}$ tenemos que $A_nA_m\subseteq A_{ n+m}$. Se sigue en particular que $A_0$ es un subanillo y que cada componente $A_n$ es un $A_0$-módulo.
    \end{mydef}

    
    
    \section{Variedades Algebraicas}

    En síntesis, las singularidades abarcan muchas ramas de las matemáticas, como son la geometría algebraica, el álgebra conmutativa, el análisis compleo, la topología algebraica y cosas sobre teoría de nudos.
    
    Considremos a $K$ un campo (o cuerpo), en ocasiones este puede ser considerado simplemente como un anillo, el cuál siempre será de característica 0.

    En el anillo de polinomios $K[x_1,...,x_n]$ tenemos los monomios
    \begin{equation*}
        x^d=x_1^{d_1}\cdots x_n^{d_n}
    \end{equation*}
    donde $d_1+...+d_n=d$. Así que todo polinomio $f$ se puede ver como:
    \begin{equation*}
        f=\sum_{\textup{finita}}c_dx^d
    \end{equation*}
    donde $c_d\in K\setminus\left\{0 \right\}$. Se define el \textbf{grado de $f$} por:
    \begin{equation*}
        \deg f=\max\left\{d_1+...+d_n\Big|c_d\neq 0 \right\}
    \end{equation*}

    \begin{exa}
        El anillo de polinomios $K[x_1,...,x_n]$ es graduado, a saber los subgrupos abelianos que lo graduano son aquellas polinomios con todas sus componentes de mismo grado. En este caso, 
    \end{exa}

    Consideramos el \textbf{espacio afín} $K^n$ de todas las tuplas $(a_1,...,a_k)$. Podemos también ver el \textbf{espacio proyectivo} $\mathbb{P}^n_k$, con coordenadas homogéneas $[x_0:x_1:...:x_n]$.

    \begin{obs}
        En las coordenadas homogéneas, $[x_0:x_1:...:x_n]$ es tal que $x_i$ no es cero para todo $i$. En particular también se tiene que:
        \begin{equation*}
            [x]=[\lambda x]=[\lambda x_0:\lambda x_1:...:\lambda x_n]
        \end{equation*}
        con $\lambda\in K\setminus\left\{ 0\right\}$
    \end{obs}

    \begin{obs}
        Podemos descomponer a la variedad proyectiva $\mathbb{P}^n_k$ como:
        \begin{equation*}
            \mathbb{P}^n_k=K^n\cup \mathbb{P}^{ n-1}_k
        \end{equation*}
        donde la primera parte es una variedad afín y la segunda es un hiperplano en el infinito (no sé a qué se refiera esto). Repitiendo este proceso podemos verlo como:
        \begin{equation*}
            \mathbb{P}^n_k=K^n\cup K^{ n-1}\cup\cdots\cup K\cup p^t
        \end{equation*}
    \end{obs}

    \begin{obs}
        Podemos también descomponer al espacio proyectivo como:
        \begin{equation*}
            \mathbb{P}^n_k=\bigcup_{ i=0}^n U_i
        \end{equation*}
        donde
        \begin{equation*}
            U_i=\left\{[x]\Big|x_i\neq0 \right\}
        \end{equation*}
        cada uno de estos $U_i$ es isomorfo a $K^n$, con isomorfismo dado por:
        \begin{equation*}
            [x]=[x_0:x_1:...:x_{ i-1}:x_i:x_{ i+1}:...:x_n]\mapsto \left(\frac{x_0}{x_i},\frac{x_1}{x_i},\cdots,\frac{x_{ i-1}}{x_i},\frac{x_{ i+1}}{x_i},...,\frac{x_n}{x_i} \right)
        \end{equation*}
    \end{obs}

    Consideraremos variedades algebraicas:
    \begin{equation*}
        V(f)=\left\{x\in K^n\Big|f(x)=0 \right\}
    \end{equation*}

    \begin{mydef}
        Decimos que un polinomio $F\in K[x_0,x_1,...,x_n]$ es \textbf{homogéneo}, si todos sus monomios tienen el mismo grado.
    \end{mydef}

    \begin{obs}
        La definición anterior es equivalente a que para todo $\lambda\in K$:
        \begin{equation*}
            F(\lambda x)=\lambda^{\deg F}F(x)
        \end{equation*}
        para todo $x=(x_0,x_1,...,x_n)\in K^{ n+1}$.
    \end{obs}

    \begin{mydef}
        Si $F$ es homogéneo, entonces $V(F)$ es una \textbf{hipersuperficie}.
    \end{mydef}

    Podemos hacer un proceso para deshomogeneizar un polinomio homogéneo, de la siguiente manera:

    \begin{equation*}
        F\left(1,\frac{x_1}{x_0},...,\frac{x_n}{x_0} \right)=f(x_1,...,x_n)
    \end{equation*}

    y, podemos homogeneizar un polinomio haciendo:
    \begin{equation*}
        F(x_0,x_1,...,x_n)=x^{\deg f}f\left(\frac{x_1}{x_0},...,\frac{x_n}{x_0} \right)
    \end{equation*}

    \begin{exa}
        Considere el polinomio $f=3+x_1+x_2$, entonces $F$ homogéneo sería:
        \begin{equation*}
            \begin{split}
                F(x_0,x_1,x_2)&=x_0^1f\left(\frac{x_1}{x_0},\frac{x_2}{x_0}\right)\\
                &=3x_0+x_1+x_2\\
            \end{split}
        \end{equation*}
    \end{exa}

    \begin{obs}
        En ocasiones interesa que $K$ sea algebraicamente cerrado. En este caso, se nos permite escribir un polinomio como:
        \begin{equation*}
            f=c\cdot(x-a_1)(x-a_2)\cdots (x-a_d),\quad a_i\in K
        \end{equation*}
        donde $d$ es el grado del polinomio, esto para polinomios en una variable.
    \end{obs}

    \begin{obs}
        En el caso en que $F$ sea un polinomio homogéneo en varias variables, podemos escribirlo como:
        \begin{equation*}
            F=c\cdot( b_1x-a_1y)\cdots( bd_x-a_dy),\quad a_i,b_i\in K
        \end{equation*}
        por lo que resulta importante tener la noción de polinomio homogéneo.
    \end{obs}

    \begin{mydef}
        Dados $f=a_0x^m+a_1x^{ m-1}+\cdots+a_{ m-1}x+a_m$ y $g=b_0x^n+b_1x^{ n-1}+\cdots+b_{ n-1}x+b_n$. Se define el \textbf{resultante de $f$ y $g$}, como:
        \begin{equation*}
            Res(f,g)=\det A_{ m+n}(a_i,b_j)
        \end{equation*}
    \end{mydef}

    Esta matriz se vería de esta manera:
    \begin{equation*}
        \left(
            \begin{array}{cccccccccc}
                a_0 & a_1 & a_2 & \cdots & a_m & 0 & 0 & \cdots & 0 \\
                0 & a_0 & a_1 &  \cdots & a_{ m-1} & a_m & 0 & \cdots & 0 \\
                0 & 0 & a_0 & \cdots & a_{ m-2} & a_{ m-1} & a_m & \cdots & 0 \\
                \vdots & \vdots & \vdots & \vdots  & \vdots & \vdots & \vdots & \cdots & \vdots \\
                0 & 0 & 0 &  \underset{(n-1)\textup{-veces recorrido}}{\underbrace{\cdots}} & a_0 & a_1 & a_2 & \cdots & a_m \\
                b_0 & b_1 & b_2 & \cdots & b_n & 0 & 0 & \cdots & 0 \\
                0 & b_0 & b_1 &  \cdots & b_{ n-1} & b_n & 0 & \cdots & 0 \\
                0 & 0 & b_0 & \cdots & b_{ n-2} & b_{ n-1} & b_n& \cdots & 0 \\
                \vdots & \vdots & \vdots & \vdots  & \vdots & \vdots & \vdots & \cdots & \vdots \\
                0 & 0 & 0 &  \underset{(m-1)\textup{-veces recorrido}}{\underbrace{\cdots}} & b_0 & b_1 & b_2 & \cdots & b_n \\
            \end{array}
            \right)
    \end{equation*}

    \begin{propo}
        $Res(f,g)=0$ si y sólo si $f$ y $g$ tienen una raíz común.
    \end{propo}

    \begin{proof}
        $\Rightarrow):$

        $\Leftarrow):$ Suponga que existe $r\in K$ tal que $f(r)=g(r)$, entonces:
        \begin{equation*}
            f(x)=(x-r)p(x)\quad\textup{y}\quad g(x)=(x-r)q(x)
        \end{equation*}
        donde $\deg p=m-1$ y $\deg q=n-1$. Se cumple además la igualdad:
        \begin{equation*}
            fq-gp=0
        \end{equation*}
        la ecuación anterior, la podemos ver como la matriz cuadrada $B_{m+n}(a_i,b_j)$ de tamaño $m+n$. Si hacemos
        \begin{equation*}
            p(x)=\alpha_0x^{ m-1}+\cdots+\alpha_{ m-1}
        \end{equation*}
        y,
        \begin{equation*}
            q(x)=\beta_0x^{ n-1}+\cdots+\beta_{ n-1}
        \end{equation*}
        Se reduciría todo a un sistema:
        \begin{equation*}
            B_{ n+m}(a_i,b_j)\left[ 
                \begin{array}{c}
                    \alpha_i \\
                    \beta_j \\
                \end{array}
            \right]=\left[ 
                \begin{array}{c}
                    0 \\
                    \vdots \\
                    0 \\
                \end{array}
            \right]
        \end{equation*}
        (completar la demostración).
    \end{proof}

    \begin{excer}
        Hacer lo de la proposición anterior cuando $f_1=f_2=x^2-3x+2$ y $g_1=x-1$ (calcular los sistemas necesarios).
    \end{excer}

    \begin{sol}
        
    \end{sol}

    \begin{exa}
        Considere los polinomios $f=x^3-3x^2+2x+1$ y $g=x^2-x+2$. Entonces $m=3$ y $n=2$, por lo que:
        \begin{equation*}
            A_{5}=\left(
                \begin{array}{ccccc}
                    1 & -3 & 2 & 1 & 0 \\
                    0 & 1 & -3 & 2 & 1 \\
                    1 & -1 & 2 & 0 & 0 \\
                    0 & 1 & -1 & 2 & 0 \\
                    0 & 0 & 1 & -1 & 2 \\
                \end{array}
            \right)
        \end{equation*}
        sería la matriz asociada al resultante de los polinomios $f$ y $g$.
    \end{exa}

    Para la siguiente proposición, $K$ es un campo algebraicamente cerrado.

    \begin{propo}
        Sean $f,g\in K[\underline{x}]$ (anillo de polinomios en varias variables). Entonces:
        \begin{enumerate}
            \item $V(f)=V(g)$ si y sólo si $f$ y $g$ tienen las mismas componentes irreducibles.
            \item $V(f)\neq\emptyset$ si y sólo si $f\in K\setminus\left\{0\right\}$.
        \end{enumerate}
    \end{propo}

    \begin{proof}
        
    \end{proof}

    \begin{mydef}
        Sea $p\in V(f)\subseteq K^n$. Decimos que $p$ es un \textbf{punto singular de $V(f)$}, si
        \begin{equation*}
            f(p)=\frac{\partial f}{\partial x_i}(p)=0
        \end{equation*}
        para todo $i=1,...,n$. El conjunto de puntos singulares de $f$ se denota por $Sing(V(f))$. Si $p\notin Sing(V(f))$, se dice que $p$ es \textbf{no singular} o \textbf{liso}.

        Si $V(f)$ es tal que $Sing(V(f))=\emptyset$, se dice que $V(f)$ es \textbf{no singular}.
    \end{mydef}

    \begin{exa}
        Considere el polinomio $f=ax+by$, $a,b\in K$ no ambas nulas. Entonces, $V(f)$ es no singular.
    \end{exa}

    \begin{exa}
        Considere $f=xy$. Entonces:
        \begin{equation*}
            Sing(V(f))=\left\{(0,0,*,*,...,*)\in K^n \right\}
        \end{equation*}
        En el caso de $K^n=\mathbb{C}^2$, se tiene que:
        \begin{equation*}
            Sing(V(f))=\left\{(0,0) \right\}\subseteq\mathbb{C}^2
        \end{equation*}
        se dice \textbf{singularidad aislada}.

        Si estamos en $\mathbb{C}^3$, entonces
        \begin{equation*}
            Sing(V(f))=\left\{(0,0,*) \right\}\subseteq\mathbb{C}^3
        \end{equation*}
        es \textbf{no aislada}.
    \end{exa}

    \begin{exa}
        En el caso en que $f=f_1\cdot f_2$, se tiene que $V(f_1)\cap V(f_2)\subseteq Sing(V(f))$.
    \end{exa}

    \begin{exa}
        Los siguientes tienen puntos singulares de diferentes tipos:
        \begin{itemize}
            \item $g=y^2-x^3$.
            \item $h=y^2-x^2(x+1)$.
            \item $k=z^2-xy^3$.
        \end{itemize}
    \end{exa}

    \section{Geometría y Topología de Curvas Algebraicas en $\mathbb{P}^2_{\mathbb{C}}$ (o en $\mathbb{C}^2$).}

    En esta parte, tendremos como objetivos dos cosas:

    \begin{enumerate}[label = \textit{(\arabic*)}]
        \item Entender la topología abstracta de $C=V(F)\subseteq\mathbb{P}^2_{\mathbb{C}}$.
        \item Entender la geometría de $C=V(F)\subseteq\mathbb{P}^2_{\mathbb{C}}$.
    \end{enumerate}

    \begin{theor}[\textbf{Teorema de Bezout}]
        Sean $C=V(P)$ y $D=V(Q)$ curvas contenidas en $\mathbb{P}^2_{\mathbb{C}}$ con $\deg P=n$ y $\deg Q=m$. Entonces, $C\cap D$ es un conjunto de $n\cdot m$ puntos (contando multiplicidades).
    \end{theor}

    \begin{theor}[\textbf{Fórmula de género-grado}]
        Sea $C=V(P)\subseteq\mathbb{P}^2_{\mathbb{C}}$ no singular y de grado $n$ irreducible. Entonces, $C$ es topológicamente una superficie (dimensión 2 sobre $\mathbb{R}$) conexa, compacta, orientable y sin borde con $\chi=2-(n-1)(n-2)$ (siendo $\chi$ la característica de Euler de la superficie).
    \end{theor}

    Luego hubo una explicación sobre la característica de Euler para superficices (en particular, algunas triangulaciones de la 2-esfera).

    \begin{theor}[\textbf{Teorema de Clasificación de Superficies}]
        La característica de Euler de toda superficie compacta, orientable, conexa y sin borde es:
        \begin{equation*}
            \chi=2-2g
        \end{equation*}
        donde $g$ es el género de la superficie.
    \end{theor}

    Notemos que:
    \begin{equation*}
        \begin{tabular}{ c | c }
            $n$ & $g=\frac{(n-1)(n-2)}{2}$ \\
            \hline
            1 & 0 \\
            2 & 0 \\
            3 & 1 \\
            4 & 3 \\
            5 & 6 \\
            6 & 10 \\
        \end{tabular}
    \end{equation*}
    por lo que no todos los géneros se pueden obtener a partir de curvas $C\subseteq\mathbb{P}_{\mathbb{C}}^2$.

    Uno puede construir todas las superifices orientables, conexas, compactas y sin borde a partir de la identificación usual que se hacía con la esfera, el toro, el 2-toro, etc...

    Hablaremos del teorema de Bezout pero desde el punto de vista de resultantes con polinomios en varias variables. Recordemos que si $f,g\in\mathbb{C}[x]$, entonces
    \begin{equation*}
        Res(f,g)=\det A_{ m+n}(a_i,b_j)
    \end{equation*}
    siendo
    \begin{equation*}
        \left(
            \begin{array}{cccccccccc}
                a_0 & a_1 & a_2 & \cdots & a_m & 0 & 0 & \cdots & 0 \\
                0 & a_0 & a_1 &  \cdots & a_{ m-1} & a_m & 0 & \cdots & 0 \\
                0 & 0 & a_0 & \cdots & a_{ m-2} & a_{ m-1} & a_m & \cdots & 0 \\
                \vdots & \vdots & \vdots & \vdots  & \vdots & \vdots & \vdots & \cdots & \vdots \\
                0 & 0 & 0 &  \underset{(n-1)\textup{-veces recorrido}}{\underbrace{\cdots}} & a_0 & a_1 & a_2 & \cdots & a_m \\
                b_0 & b_1 & b_2 & \cdots & b_n & 0 & 0 & \cdots & 0 \\
                0 & b_0 & b_1 &  \cdots & b_{ n-1} & b_n & 0 & \cdots & 0 \\
                0 & 0 & b_0 & \cdots & b_{ n-2} & b_{ n-1} & b_n& \cdots & 0 \\
                \vdots & \vdots & \vdots & \vdots  & \vdots & \vdots & \vdots & \cdots & \vdots \\
                0 & 0 & 0 &  \underset{(m-1)\textup{-veces recorrido}}{\underbrace{\cdots}} & b_0 & b_1 & b_2 & \cdots & b_n \\
            \end{array}
            \right)
    \end{equation*}
    con $\deg f=m$ y $\deg g = n$ (siendo $a_i$ los coeficientes de $f$ y $b_j$ los de $g$). Si consideramos ahora polinomios en 3 variables:
    \begin{equation*}
        f(x,y,z)=a_0(x,y)z^m+a_1(x,y)z^{ m-1}+\cdots+a_m(x,y)
    \end{equation*}
    y,
    \begin{equation*}
        g(x,y,z)=b_0(x,y)z^n+b_1(x,y)z^{ n-1}+\cdots+b_n(x,y)
    \end{equation*}
    tomamos a los polinomios $f,g\in\mathbb{C}[x,y][z]=\mathbb{C}[x,y,z]$ homogéneos. En este caso, los grados de $f$ y $g$ son $m$ y $n$, respectivamente, por lo que $a_i(x,y)$ y $b_j(x,y)$ son polinomios homogéneos de grado $i$ y $j$, respectivamente.

    \begin{mydef}
        Sean $F,G\in\mathbb{C}[x,y][z]$ polinomios homogéneos de grados $m$ y $n$, respectivamente. Entonces:
        \begin{equation*}
            Res_{z}(F,G)=\det A_{ m+n}(a_i(x,y),b_j(x,y))
        \end{equation*}
        con el $A$ dado como se hizo anteriormente.
    \end{mydef}

    \begin{obs}
        Se tiene que $Res_{z}(F,G)\in\mathbb{C}[x,z]$ es un polinomio homogéneo de grado $n\cdot m$ (a lo más ya que puede ser cero). Por tanto,
        \begin{equation*}
            Res_z(F,G)=\prod_{ i=n}^{ n\cdot m}(b_ix+a_iy)
        \end{equation*}
        (por ser $\mathbb{C}$ algebraicamente cerrado).
    \end{obs}

    \begin{obs}
        Dados $a,b\in\mathbb{C}$, hacemos:
        \begin{equation*}
            F(a,b,z)=f(z)\quad\textup{y}\quad G(a,b,z)=g(z)
        \end{equation*}
        entonces,
        \begin{equation*}
            Res_z(F,G)(a,b)=Res(f,g)
        \end{equation*}
        
        Recordemos que $Res(f,g)=0$ si existe $c\in\mathbb{C}$ tal que $f(c)=g(c)$.
    \end{obs}

    Por tanto, de las observaciones anteriores, se tiene que para cada $i=1,...,n\cdot m$ existen $c_i$ tales que
    \begin{equation*}
        f(c_i)=g(c_i)=0
    \end{equation*}
    esto es que
    \begin{equation*}
        F(a,b,c_i)=G(a,b,c_i)
    \end{equation*}

    \begin{obs}
        Se tiene que $C\cap D\neq\emptyset$ ¿?.
    \end{obs}

    \begin{propo}
        $Res_z(F,G)=0\in\mathbb{C}[x,y]$ si y sólo si $F$ y $G$ tienen una componente común.
    \end{propo}

    \begin{proof}
        Procederemos por reducción al absurdo. Supongamos que
        \begin{itemize}
            \item $Res_z(F,G)\in\mathbb{C}[x,y]$ y es no constante.
            \item $F$ y $G$ tienen al menos $n\cdot m+1$ puntos comunes.
        \end{itemize}
        Podemos tomar coordeadas de modo que cada punto común a $F$ y $G$ induce un factor lineal $b_ix+a_iy$ de $Res_z(F,G)$ y son no proporcionales dos a dos. Por lo cual al menos hay $n\cdot m+1$ factores lineales\contradiction.
    \end{proof}

    \begin{mydef}
        Sean $C,D$ dos curvas en $\mathbb{P}_\mathbb{C}^2$ tales que:
        \begin{itemize}
            \item $[0:0:1]\notin C\cup D$.
            \item $[0:0:1]$ no pertenece a una recta por dos puntos de $C\cap D$.
            \item $[0:0:1]$ no pertenece a la tangente de $C$ ni a la tangente a $D$ por un punto común $Q\in C\cap D$.
        \end{itemize}
        y, sea $O=[a:b:c]\in\mathbb{P}_{\mathbb{C}}^2$. Definimos la \textbf{multiplicidad de intersección de $O$}
        \begin{equation*}
            I_O(C,D)=\left\{
                \begin{array}{lcr}
                    0 & \textup{ si } & 0\notin C\cap D\\
                    \max\left\{k \right\} & \textup{ t.q. } & (bx-ay)^k\divides Res_z(F,G) \\
                    \infty & \textup{ si } & O\textup{ pertenece a una componente común a $C$ y $D$} \\
                \end{array}
            \right.
        \end{equation*}
    \end{mydef}

    Con la definición anterior se sigue que:
    \begin{equation*}
        \begin{split}
            Res_z(F,G)&=\prod_{ O=[a,b,c]\in C\cap D}(bx-ay)^k\\
        \end{split}
    \end{equation*}
    por lo cual:
    \begin{equation*}
        \sum_{ I\in C\cap D}I_O(F,G)=n\cdot m
    \end{equation*}

    Un puede definir la multiplicidad de un punto en una curva $C$, a partir de ver la mínima derivada parcial donde el punto no se anula, denotada por $mult_O(C)$. Se tiene que:
    \begin{equation*}
        I_O(C,D)\geq mult_O(C)+mult_O(D)
    \end{equation*}

    Ahora hablaremos de la fórumla de grado-género.

    \begin{theor}[\textbf{Teorema de la función implícita}]
        Sea $\underline{0}=(0,0,z)\in C=V(f)\subseteq\mathbb{C}^2$ tal que $f_y(\underline{0})\neq0$, entonces existen entornos abiertos $O_1\in V\subseteq \mathbb{C}$ y $\underline{O}\in U\subseteq C$ y una función analítica $\cf{g}{V}{U}$ tal que
        \begin{equation*}
            f(x,g(x))=0,\quad\forall x\in V
        \end{equation*}
    \end{theor}

    Por tanto, una curva lisa $C$ y $p\in C$ implica que $\frac{\partial f}{\partial x}(p)\neq 0$ o bien $\frac{\partial f}{\partial y}(p)\neq0$, por lo que del teorema de la función implícita se sigue que un entorno de $p$ en $C$ es homeomorfo a un entnro de $\mathbb{C}$ (de $\mathbb{R}^2$).
    
    Por lo que, $C$ es efectivamente una superficie.

    Se sabe de cursos de topología de conexión por arcos implica conexión.

    Considere puntos en la curva $p,q\in C$. Se tiene que existe $\cf{\gamma}{[0,1]}{C}$ continua tal que $\gamma(0)=q$ y $\gamma(1)=p$.

    Tenemos la curva $C=V(f)$. Podemos suponer que
    \begin{equation*}
        f=y^n+a_1(x)y^{ n-1}+\cdots+a_{ n-1}(x)y+a_n(x)
    \end{equation*}
    se tiene que $f(x_0,y)\in\mathbb{C}[y]$ es tal que $\deg f(x_0,y)=n$. Si consideramos la proyección $\cf{\pi_1}{C}{\mathbb{C}}$, entonces $\pi^{-1}(x_0)$ nos dará las raíces distinas de $f(x_0,y)$.

    Se define:
    \begin{equation*}
        Disc(f(x_0,y))=Res(f(x_0,y),f_y(x_0,y))
    \end{equation*}
    entonces
    \begin{equation*}
        \abs{\pi^{-1}(x_0)}=n\iff Disc(f(x_0,y))\neq0
    \end{equation*}

    En conclusión, $C\setminus\bigcup_{ q\in\mathbb{C} }\pi_1^{-1}(q)$ donde $q$ es raíz del discriminante, es un recubrimiento de $n$-hojas.

    Lo demás se deduce de forma más sencilla.

\end{document}