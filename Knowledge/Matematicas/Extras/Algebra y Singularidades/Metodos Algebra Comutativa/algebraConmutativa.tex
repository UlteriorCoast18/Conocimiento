\documentclass[12pt]{report}
\usepackage[spanish]{babel}
\usepackage[utf8]{inputenc}
\usepackage{amsmath}
\usepackage{amssymb}
\usepackage{amsthm}
\usepackage{graphics}
\usepackage{subfigure}
\usepackage{lipsum}
\usepackage{array}
\usepackage{multicol}
\usepackage{enumerate}
\usepackage[framemethod=TikZ]{mdframed}
\usepackage[a4paper, margin = 1.5cm]{geometry}
\usepackage{tikz}
\usepackage{pgffor}
\usepackage{ifthen}
\usepackage{enumitem}

\usetikzlibrary{shapes.multipart}

\newcounter{it}
\newcommand*\watermarktext[1]{\begin{tabular}{c}
    \setcounter{it}{1}%
    \whiledo{\theit<100}{%
    \foreach \col in {0,...,15}{#1\ \ } \\ \\ \\
    \stepcounter{it}%
    }
    \end{tabular}
    }

\AddToHook{shipout/foreground}{
    \begin{tikzpicture}[remember picture,overlay, every text node part/.style={align=center}]
        \node[rectangle,black,rotate=30,scale=2,opacity=0.04] at (current page.center) {\watermarktext{Cristo Daniel Alvarado ESFM\quad}};
  \end{tikzpicture}
}
%En esta parte se hacen redefiniciones de algunos comandos para que resulte agradable el verlos%

\def\proof{\paragraph{Demostración:\\}}
\def\endproof{\hfill$\blacksquare$}

\def\sol{\paragraph{Solución:\\}}
\def\endsol{\hfill$\square$}

%En esta parte se definen los comandos a usar dentro del documento para enlistar%

\newtheoremstyle{largebreak}
  {}% use the default space above
  {}% use the default space below
  {\normalfont}% body font
  {}% indent (0pt)
  {\bfseries}% header font
  {}% punctuation
  {\newline}% break after header
  {}% header spec

\theoremstyle{largebreak}

\newmdtheoremenv[
    leftmargin=0em,
    rightmargin=0em,
    innertopmargin=0pt,
    innerbottommargin=5pt,
    hidealllines = true,
    roundcorner = 5pt,
    backgroundcolor = gray!60!red!30
]{exa}{Ejemplo}[section]

\newmdtheoremenv[
    leftmargin=0em,
    rightmargin=0em,
    innertopmargin=0pt,
    innerbottommargin=5pt,
    hidealllines = true,
    roundcorner = 5pt,
    backgroundcolor = gray!50!blue!30
]{obs}{Observación}[section]

\newmdtheoremenv[
    leftmargin=0em,
    rightmargin=0em,
    innertopmargin=0pt,
    innerbottommargin=5pt,
    rightline = false,
    leftline = false
]{theor}{Teorema}[section]

\newmdtheoremenv[
    leftmargin=0em,
    rightmargin=0em,
    innertopmargin=0pt,
    innerbottommargin=5pt,
    rightline = false,
    leftline = false
]{propo}{Proposición}[section]

\newmdtheoremenv[
    leftmargin=0em,
    rightmargin=0em,
    innertopmargin=0pt,
    innerbottommargin=5pt,
    rightline = false,
    leftline = false
]{cor}{Corolario}[section]

\newmdtheoremenv[
    leftmargin=0em,
    rightmargin=0em,
    innertopmargin=0pt,
    innerbottommargin=5pt,
    rightline = false,
    leftline = false
]{lema}{Lema}[section]

\newmdtheoremenv[
    leftmargin=0em,
    rightmargin=0em,
    innertopmargin=0pt,
    innerbottommargin=5pt,
    roundcorner=5pt,
    backgroundcolor = gray!30,
    hidealllines = true
]{mydef}{Definición}[section]

\newmdtheoremenv[
    leftmargin=0em,
    rightmargin=0em,
    innertopmargin=0pt,
    innerbottommargin=5pt,
    roundcorner=5pt
]{excer}{Ejercicio}[section]

%En esta parte se colocan comandos que definen la forma en la que se van a escribir ciertas funciones%

\newcommand\abs[1]{\ensuremath{\left|#1\right|}}
\newcommand\divides{\ensuremath{\bigm|}}
\newcommand\cf[3]{\ensuremath{#1:#2\rightarrow#3}}
\newcommand\contradiction{\ensuremath{\#_c}}
\newcommand\natint[1]{\ensuremath{\left[\big|#1\big|\right]}}
\newcommand{\gen}[1]{\ensuremath{\mathcal{L}\left(#1 \right)}}
\newcommand{\Ass}[1]{\ensuremath{\textup{Ass}\left(#1\right)}}
\newcommand{\height}[1]{\ensuremath{\textup{ht}\left(#1\right)}}

\begin{document}
    \setlength{\parskip}{5pt} % Añade 5 puntos de espacio entre párrafos
    \setlength{\parindent}{12pt} % Pone la sangría como me gusta
    \title{Métodos del Álgebra Conmutativa en la Teoría de Códigos}
    \author{Cristo Daniel Alvarado}
    \maketitle

    \tableofcontents %Con este comando se genera el índice general del libro%

    \chapter{Preeliminares}

    \section{Teoría de Códigos}

    \begin{mydef}
        Un \textbf{código} $\mathcal{C}$ es un subconjunto de $\mathbb{K}^n$, donde $\mathbb{K}$ es un campo. Si $\mathcal{C}$ es un subespacio vectorial, decimos que $\mathcal{C}$ es un \textbf{código lineal}.

        Los elementos de $\mathcal{C}$ son llamados \textbf{palabras}.
    \end{mydef}

    \begin{obs}
        En la definición anterior, el campo $\mathbb{K}$ puede ser finito o infinito, y generalmente se usará $\mathbb{K}=\mathbb{F}_q$.
    \end{obs}

    \begin{obs}
        Un código lineal puede ser considerado como:
        \begin{equation*}
            \mathcal{C}=\gen{w_1,...,w_n}
        \end{equation*}
        si consideramos la base estándar de $\mathbb{K}^n$, entonces podemos pensar en $\mathcal{C}$ como la imagen de una función lineal lineal $\cf{\phi}{\mathbb{K}^k}{\mathbb{K}^n}$, con matriz $k\times n$, denotada por $G$.

        En particular:
        \begin{equation*}
            G=\left(
                \begin{array}{c}
                    w_1 \\
                    \vdots \\
                    w_k \\
                \end{array}
            \right)
        \end{equation*}
        además, $\mathcal{C}=\phi(\mathbb{K}^k)$.
    \end{obs}

    \begin{mydef}
        En el ejemplo anterior, $G$ es llamada \textbf{matriz generadora de $\mathcal{C}$}.
    \end{mydef}

    En cierto sentido, la matriz $\mathcal{C}$ es la que genera al código.

    \textit{¿Cómo podemos determinar si $v\in\mathcal{K}^n$ es una palabra de $\mathcal{C}$?}

    Pues en el caso en que $v\in\mathcal{C}$, se tiene que
    \begin{equation*}
        \phi(v)=vG=vw_1+\cdots+v_kw_k
    \end{equation*}
    y, $\dim_{\mathbb{K}}\mathcal{C}=\dim\mathcal{C}$. $n$ es llamado \textbf{longitud (bloque)} de $\mathcal{C}$.

    Notemos que si $\mathcal{C}$ es un subespacio vectorial, podemos considerar al espacio ortogonal:
    \begin{equation*}
        \mathcal{C}^\perp=\left\{w\in\mathbb{K}^n\Big|w\cdot c=0,\quad\forall c\in\mathcal{C} \right\}
    \end{equation*}

    \begin{mydef}
        El espacio $\mathcal{C}^\perp$ es llamado \textbf{código dual}. Este es un subespacio de $\mathbb{K}^n$ y hacemos que $H$ sea la \textbf{matriz generadora de $\mathcal{C}^\perp$}.
    \end{mydef}

    $H$ también es la matriz sde chequeo de paridad de $\mathcal{C}$, pues tenemos que $\mathcal{C}$ es el espacio nulo de $H$, pues:
    \begin{equation*}
        \dim\mathcal{C}+\dim\mathcal{C}^\perp=n
    \end{equation*}

    \begin{mydef}
        Si $x=(x_1,...,x_n),y=(y_1,...,y_n)\in\mathbb{K}^n$, se define la \textbf{distancia de Hamming} entre ambos vectores por:
        \begin{equation*}
            d(x,y)=\abs{\left\{i\in\left\{1,...,n \right\}\Big|x_i\neq y_i \right\}}
        \end{equation*}
    \end{mydef}

    \begin{mydef}
        El \textbf{peso de Hamming} de un vector $v\in\mathbb{K}^n$ es el número de entradas diferentes de 0 que tiene $v$:
        \begin{equation*}
            w(v)=d(v,0)
        \end{equation*}
        La \textbf{distancia mínima de $\mathcal{C}\subseteq\mathbb{K}^n$} se define por:
        \begin{equation*}
            d(\mathcal{C})=\min\left\{w(v)\Big|v\in\mathcal{C}\setminus\left\{0\right\} \right\}
        \end{equation*}

        Una palabra $v\in\mathcal{C}$ tal que $w(v)=d(\mathcal{C})$ será llamada \textbf{palabra de peso mínimo}.
    \end{mydef}

    Se tienen los siguientes parámetros básicos de un código lineal $\mathcal{C}$:
    \begin{itemize}
        \item Dimensión ($\dim_{\mathbb{K}}(\mathcal{C})$).
        \item Longitud (tamaño de las palabras, $n$).
        \item Distancia mínima ($d(\mathcal{C})$).
    \end{itemize}

    \begin{obs}
        $d(\mathcal{C})$ mide la capacidad de detección y corrección de errores en un código.
    \end{obs}

    \begin{propo}
        Se tiene lo siguiente:
        \begin{enumerate}[label = \textit{(\arabic*)}]
            \item La distancia de Hamming es una métrica en $\mathbb{K}^n$.
            \item Para cualquier código lineal $[n,\dim_{\mathbb{K}}(\mathcal{C},d(\mathcal{C}))]$ se satisface que:
            \begin{equation*}
                d(\mathcal{C})\leq n<\dim\mathcal{C}+1
            \end{equation*}
            la cota superior es llamada \textbf{cota de Singleton}. Cuando $d(\mathcal{C})=n-\dim\mathcal{C}+1$, el código es llamado \textbf{MDS (maximum distance separable)}.
            \item Si $H$ es la matriz de chequeo de paridad, entonces $c\in\mathcal{C}$ si y sólo si $Hc^T=0$.
        \end{enumerate}
    \end{propo}

    \begin{proof}
        
    \end{proof}

    \begin{mydef}
        Diremos que dos códigos son \textbf{equivalentes}, se tienen los mismos parámetros.
    \end{mydef}

    \begin{mydef}
        Un código es \textbf{no degenerado}, si para cualquier matriz generadora $M$, todas sus columnas son no nulas.
    \end{mydef}

    \begin{exa}
        Considere $\mathbb{K}=\mathbb{F}_2$. Tomemos el código lineal con matriz generadora:
        \begin{equation*}
            G=\left(
                \begin{array}{lcccccr}
                    1 & 0 & 0 & 0 & 1 & 1 & 0 \\
                    0 & 1 & 0 & 0 & 0 & 1 & 1 \\
                    1 & 0 & 1 & 0 & 1 & 1 & 1 \\
                    0 & 0 & 0 & 1 & 1 & 0 & 1 
                \end{array}
            \right)
        \end{equation*}
        tiene los parámetros $[7,4,3]$. Si multiplicamos todos los vectores de $\mathbb{F}_2^4$, se tiene que $\mathcal{C}$ consta de 16 palabras, tiene 7 con peso de Haming igual a 3, 7 con peso de Hamming igual a 4 y 1 con peso de Hamming igual a 7, más el vector 0. Por lo que $d(\mathcal{C})=3$.
    \end{exa}

    \begin{obs}
        La equivalencia es en que estamos olvidando la información sobre la estructura de espacio vectorial del código $\mathcal{C}$, pero estamos preservando información que tiene que ver con la longitud de la palabra, dimensión del espacio y la distancia mínima (lo que sea para lo que sirva).
    \end{obs}

    \begin{excer}
        Sea $\mathcal{C}$ un $[n,k]$ código lineal no degenerado con matriz generadora $G$ de tamaño $k\times n$. Entonces:
        \begin{equation*}
            d(\mathcal{C})=n-h
        \end{equation*}
        donde $h$ es el máximo del número de columnas de la matriz generadora que generan a un subespacio $k-1$ dimensional.
    \end{excer}

    \begin{proof}
        Ejercicio.
    \end{proof}

    \section{Álgebra}

    \begin{obs}
        De ahora en adelante, $R$ será un anillo conmutativo con identidad.
    \end{obs}

    \begin{mydef}
        Sea $I$ un ideal de $R$, se define el \textbf{ideal radical de $I$} (denotado por $\sqrt{I}$) por:
        \begin{equation*}
            \sqrt{I}=\left\{r\in R\Big|r^n\in I,\textup{ para algún }n\in\mathbb{N} \right\}
        \end{equation*}
    \end{mydef}

    \begin{mydef}
        Un ideal $Q$ de $R$ es \textbf{ideal primario de $R$}, si siempre que $f\cdot g\in Q$ con $f\notin Q$, enotnces $q\in\sqrt{Q}$.
    \end{mydef}

    \begin{obs}
        Si $Q$ es primario, entonces $\sqrt{Q}=P$ es ideal primo
    \end{obs}

    \begin{mydef}
        Si:
        \begin{equation*}
            \min\left(Q\right)=\left\{P \right\}
        \end{equation*}
        (es decir, el mínimo de los ideales primos que contienen a $Q$ es $P$), decimos que $Q$ es \textbf{$P$-primario.}
    \end{mydef}

    \begin{mydef}
        Un ideal $I\subseteq R$ es \textbf{Noetheriano}, si es finitamente generado, esto es que existen $f_1,...,f_l\in R$ tales que
        \begin{equation*}
            I=\langle f_1,...,f_l\rangle
        \end{equation*}
        por ende, todo $g\in I$ puede ser expresado como:
        \begin{equation*}
            g=r_1f_1+\cdots+r_lf_l
        \end{equation*}
        con $r_i\in R$.
    \end{mydef}

    \begin{theor}
        Podemos expresar a $I\subseteq R$ como:
        \begin{equation*}
            I=Q_1\cap\cdots\cap Q_s
        \end{equation*}
        donde cada $Q_i$ es $P_i$-primario y a esta descomposición le llamamos \textbf{descomposición primaria minimal}, para la cual se cumple que el ideal
        \begin{equation*}
            P_i=\sqrt{Q_i}
        \end{equation*}
        es primo.
    \end{theor}

    \begin{proof}
        
    \end{proof}

    \begin{mydef}
        Dado un ideal $I\subseteq R$, el conjunto formado a partir del teorema anterior:
        \begin{equation*}
            \Ass{I}=\left\{P_1,...,P_s \right\}
        \end{equation*}
        es llamado el \textbf{conjunto de primos asociados a $I$}.
    \end{mydef}

    \begin{mydef}
        La altura de un ideal primo $P\subseteq R$ es el supremo de las longitudes $n$, de las cadenas de ideales primos:
        \begin{equation*}
            P_0\subsetneq P_1\subsetneq\cdots\subsetneq P_n=P
        \end{equation*}
        denotado por $ht(P)$.
    \end{mydef}

    \begin{mydef}
        Sea $I\subseteq R$. Se define la \textbf{altura de $I$}, como:
        \begin{equation*}
            ht(I)=\min\left\{ht(P)\Big|I\subseteq P \right\}
        \end{equation*}
    \end{mydef}

    \begin{mydef}
        La \textbf{dimensión de Krull de $R$} es el máximo de $ht(P)$ tal que $P$ es ideal primo de $R$. En este caso, la dimensión de Krull de $R$ se denota por $\dim R$.
    \end{mydef}

    Cuando $R=K[x_1,...,x_n]$, se tiene que:
    \begin{equation*}
        \dim R/I=\dim R-ht(I)
    \end{equation*}
    y, $\dim R=n$. Para la prueba de esto, vea algún libro de álgebra conmutativa.

    \begin{exa}
        Considere $A=K[x,y,z]/I$ donde $I=\langle xy,xz\rangle$. Una descomposición primaria sería:
        \begin{equation*}
            I=\langle x\rangle\cap\langle y,z\rangle
        \end{equation*}
        Además,
        \begin{equation*}
            \begin{split}
                ht(I)&=\min\left\{ht(\langle x\rangle),ht(\langle y,z\rangle)\right\}\\
                &=\min\left\{1,2 \right\}\\
                &=1\\
            \end{split}
        \end{equation*}
        Por el hecho anterior, se tiene que
        \begin{equation*}
            \dim K[x,y,z]/\langle xy,yz\rangle=3-1=2
        \end{equation*}
    \end{exa}

    \section{Puntos Gordos (Fat-points)}

    En el espacio afín $\mathbb{A}^k$ (producto de $\mathbb{K}$ consigo mismo $k$-veces), podemos tomar el conjunto $\mathbb{A}^k\setminus\left\{(0,...,0) \right\}$ y definir una equivalencia sobre este conjunto dada como sigue:
    \begin{equation*}
        (a_1,...,a_k)\sim(b_1,...,b_k)
    \end{equation*}
    si y sólo si existe $\lambda\in\mathbb{K}$ tal que:
    \begin{equation*}
        (a_1,...,a_k)=(\lambda b_1,...,\lambda b_k)
    \end{equation*}
    Se prueba rápidamente que $\sim$ es relación de equivalencia sobre este conjunto.
    
    Denotamos por $[a_1,...,a_k]=[a_1:...:a_k]$ a la clase de equivalencia con representante $(a_1,...,a_k)$.

    \begin{mydef}
        El conjunto de todas las clases de equivalencia anteriores es denotado por $\mathbb{P}^{ k-1}$ y es llamado \textbf{espacio proyectivo}.

        Un elemento $[a_1,...,a_k]\in\mathbb{P}^{ k-1}$ es un \textbf{punto proyectivo}, $a_1,...,a_k$ son llamadas \textbf{coordenadas homogéneas}. Un \textbf{representante estándar} de un punto proyectivo es un representante con la primer coordenada homogénea igual a 1, esto es, que es de la forma:
        \begin{equation*}
            [0,...,0,1,a_i,...,a_k]
        \end{equation*}
    \end{mydef}

    \begin{mydef}
        Consideremos el anillo $R=K[x_1,...,x_n]$. Una \textbf{variedad proyectiva}, es el conjunto de ceros comunes de un conjunto de polinomios homogéneos en $R$.

        Si $X\subseteq\mathbb{P}^{ k-1}$, se define el \textbf{ideal de anulación} o \textbf{ideal de definición de $X$}, denotado por $I(X)$, es el conjunto de todos los polinomios que se anulan en todos los puntos de $X$.
    \end{mydef}

    \begin{mydef}
        Una \textbf{hipersuperficie} es una variedad generada por una sola variedad polinomial.
    \end{mydef}

    \begin{mydef}
        La \textbf{dimensión} de una variedad proyectiva $X\subseteq\mathbb{P}^{ k-1}$ es $m$ si $k-1-m$ es el número más pequeño de hiperplanos genéricos que tienen intersección con $X$ en un conjunto finito de puntos.

        El número de este conjunto finito de puntos es llamado el \textbf{grado de $X$}, denotado por $\deg(X)$.
    \end{mydef}

    \begin{exa}
        En particular, si $Q\in\mathbb{P}^{ k-1}$ con $Q=[a_1,...,a_k]$, entonces
        \begin{equation*}
            I(Q)=\langle\left\{a_ix_j-a_jx_i\Big|1\leq i\leq j\leq k \right\} \rangle
        \end{equation*}
    \end{exa}

    Si $X=\left\{P_1,...,P_m \right\}$ es un conjunto finito de puntos de $\mathbb{P}^{ k-1}$ y su ideal de definición
    \begin{equation*}
        I(X)=I(P_1)\cap\cdots\cap I(P_m) 
    \end{equation*}
    para cada punto $P_i\in X$.

    \begin{mydef}
        En lo anterior, tenemos que se denomina un \textbf{esquema de puntos reducidos}.
    \end{mydef}

    \begin{mydef}
        Sea $X=\left\{P_1,...,P_m \right\}\subseteq\mathbb{P}^{ k-1}$ un conjunto finito y $n_1,...,n_m$ enteros positivos. Un \textbf{esquema de puntos gordos}, es un esquema de puntos proyectivos con ideal de definición:
        \begin{equation*}
            I(X)=I(P_1)^{ n_1}\cap\cdots\cap I(P_m)^{ n_m}
        \end{equation*}
    \end{mydef}

    \begin{obs}
        En notación de \textit{divisores}, escribimos:
        \begin{equation*}
            \mathcal{Z}=n_1P_1+\cdots+n_mP_m
        \end{equation*}
        El \textbf{soporte de $\mathcal{Z}$} es $\sup\left(\mathcal{Z} \right)=X$ y los enteros $n_i$ representan la multiplicidad de $p_i$.
    \end{obs}

    \begin{mydef}
        Sea $X=\left\{P_1,...,P_n \right\}\subseteq\mathbb{P}^{ k-1}$. Si consideramos a $G$ como la matriz $k\times n$ con columnas las coordenadas homogéneas de algún representante de $P_i$.

        Decimos que $X$ están en \textbf{posición general} si y sólo si para cualquier $1\leq c\leq \min\left\{n,k \right\}$. cualesquiera $c$ columnas de $G$ generan un espacio $c$-dimensional de $\mathbb{K}^k$.
    \end{mydef}

    \begin{mydef}
        Si $p=[a_1,...,a_k]\in\mathbb{P}^k$ es un punto, entonces podemos asociar una forma lineal $l_p=a_1x_1+\cdots+a_kx_k\in R=K[x_1,...,x_n]$.
        
        Inversamente, para cualquier forma lineal $l=b_1x_1+\cdots+b_kx_k$ podemos asociar un punto $l^\nu=[b_1,...,b_k]\in\mathbb{P}^k$. 
    \end{mydef}

    De momento, esto es todo lo que ocupamos de álgebra.

    Consideremos $\mathcal{C}$ un $[n,k,d]$ código lineal con matriz generadora $G$ de tamaño $k\times n$, rango $k\geq 1$ y además el código $\mathcal{C}$ es no degenerado. Considere el conjunto de $n$ puntos reducidos (esto es, que la multiplicidad de cada uno es uno), digamos:
    \begin{equation*}
        x_c=\left\{p_1,...,p_n \right\}\subseteq\mathbb{P}^{ k-1}
    \end{equation*}
    donde las coordenadas homogéneas de los puntos $p_i$ corresponden a las entradas de la $i$-ésima columna de $G$, como el rango de $G$ es $k$, entonces el conjunto de puntos $x_c$ no están todos incluídos en un hiperplano de $\mathbb{P}^{ k-1}$.

    \begin{mydef}
        En el caso anterior, el \textbf{esquema reducido}, es el ideal de definición de $x_c$:
        \begin{equation*}
            I(x_c)=I(p_1)\cap\cdots\cap I(p_n)
        \end{equation*}
    \end{mydef}

    Si la matriz $G$ tiene columnas proporcionales, consideremos:
    \begin{equation*}
        \mathcal{Z}_c=m_1p_q+\cdots+m_sp_s
    \end{equation*}
    donde $m_i$ es la respectiva multiplicidad de dos columnas proporcionales.

    \begin{mydef}
        En el caso anterior, el esquema de puntos gordos asociado a $\mathcal{C}$ es el ideal de definición:
        \begin{equation*}
            I(\mathcal{Z}_c)=I(p_1)^{ m_1}\cap\cdots\cap I(p_s)^{ m_s}
        \end{equation*}
    \end{mydef}

    Ahora analizaremos la conexión entre $\mathcal{C}$ y todas las ideas algebraicas que hemos introducido anteriormente.

    \begin{mydef}
        Para un esquema de puntos gordos $\mathcal{Z}\subseteq\mathbb{P}^{ k-1}$, denotaremos por $hyp(\mathcal{Z})$ al máximo número de puntos de $\mathcal{Z}$ (contando multiplicidad) que están contenidos en un hiperplano de $\mathbb{P}^{ k-1}$.
    \end{mydef}

    \begin{exa}
        Imaginemos que el esquema de puntos gordos está dado por:
        \begin{equation*}
            \mathcal{Z}=3p_1+2p_2+p_3+p_4\subseteq\mathbb{P}^2
        \end{equation*}
        con $p_1,...,p_4$ no colineales y $p_2,...,p_4$ colineales, entonces:
        \begin{equation*}
            hyp(\mathcal{Z})=5
        \end{equation*}
        ya que recuerde que estamos contando multiplicidades.
    \end{exa}

    \begin{mydef}
        Decimos que el código $\mathcal{C}$ es \textbf{definido por el esquema de puntos gordos} $\mathcal{Z}_c$.
    \end{mydef}

    \begin{propo}
        Sea $\mathcal{C}$ un código lineal, $[n,k,d]$ un código lineal con esquema de puntos gordos $\mathcal{Z}_c$ en $\mathbb{P}^{ k-1}$. Entonces:
        \begin{equation*}
            d=n-hyp(\mathcal{Z}_c)
        \end{equation*}
    \end{propo}

    \begin{proof}
        Por definición, la distancia de mínima $d$ de código lineal, existe una palabra no cero $v=(v_1,...,v_n)\in\mathcal{C}$ tal que
        \begin{equation*}
            w(v)=d=d(\mathcal{C})
        \end{equation*}
        y cualquier otra palabra de $\mathcal{C}$ tiene peso de Hamming mayor o igual a $d$. En particular, cualquier otra palabra con peso menor que $d$ es la palabra cero.

        Como $v\in\mathcal{C}$, entonces existe $\alpha=(\alpha_1,...,\alpha_k)\in \mathbb{K}^k$ tal que
        \begin{equation*}
            l_i(\alpha)=v_i,\quad\forall i=1,...,n
        \end{equation*}
        donde $l_i$ son las formas lineales de $\mathcal{C}$ en $\mathbb{K}[x_1,...,x_n]$.

        También tenemos que existen índices $i_1,...,i_{ n-d}\in\left\{1,...,n \right\}$ tales que
        \begin{equation*}
            l_{ i_a}(\alpha)=0
        \end{equation*}
        con $a=1,...,n-d$ y si $j\in\left\{1,...,n \right\}\setminus\left\{i_1,...,i_{ n-d} \right\}$, entonces:
        \begin{equation*}
            l_j(\alpha)\neq0
        \end{equation*}

        Pero, como
        \begin{equation*}
            l_i=a_{ 1i}x_1+\cdots+a_{ ki}x_k
        \end{equation*}
        donde $a_{ ji}$ son entradas de la matriz generadora $G$. Entonces:
        \begin{equation*}
            l_i(\alpha)=a_{ 1i}\alpha_1+\cdots+a_{ ki}\alpha_k=L(p_i)
        \end{equation*}
        donde
        \begin{equation*}
            L(x_1,...,x_k)=\alpha_1x_1+\cdots+\alpha_kx_k
        \end{equation*}
        siendo $P_i=[a_{ 1i},...,a_{ ki}]\in\mathbb{P}^{ k-1}$ es la $i$-ésima columna de $G$. ¿Cuál es el máximo número de puntos del esquema $\mathcal{Z}_c$ del esquema que contando multiplicidades están en un mismo hiperplano? Pues debe ser $n-d$, ya que esos son los únicos números de puntos que le pegan al cero en las transformaciones $l_i$ en su respectiva $i$-ésima entrada, luego son ceros de $L$, por lo que están en el hiperplano generado por la imagen del endomorfismo $L$. Por lo que:
        \begin{equation*}
            hyp(\mathcal{Z}_c)=n-d\Rightarrow d=n-hyp(\mathcal{Z}_c)
        \end{equation*}
        lo que prueba el resultado.
    \end{proof}

    \begin{obs}
        En la proposición anterior, estamos clasificando cosas en función de si nos da algo o no nos da nada.
    \end{obs}

    \begin{mydef}
        Para un esquema de puntos gordos $\mathcal{Z}_c=m_1p_1+\cdots+m_sp_s\subseteq\mathbb{P}^{k-1}$ no todos colineales, si
        \begin{equation*}
            m_1+\cdots+m_s=n
        \end{equation*}
        entonces, el valor $n-hyp(\mathcal{Z})$ se llama la \textbf{distancia mínima de $\mathcal{Z}$} y la denotaremos por $d(\mathcal{Z})$ ($n$ es el número de puntos).
    \end{mydef}

    Sea $\mathcal{Z}=m_1p_1+\cdots+m_sp_s\subseteq\mathbb{P}^{ k-1}$ un esquema de puntos gordos no contenidos todos en un hiperplano con $m_1\geq m_2\geq\cdots\geq m_s$. Se tiene que:
    \begin{equation*}
        \sup(\mathcal{Z})=\left\{p_1,...,p_s \right\}=X
    \end{equation*}
    para $i=1,...,s$ supongamos que $c_i$ es el vector columna del punto $p_i$. Consideremos
    \begin{equation*}
        A(X)=(c_1,...,c_s)
    \end{equation*}
    y,
    \begin{equation*}
        A(Z)=(\underset{ m_1}{\underbrace{ c_1,...,c_1}},\cdots,\underset{ m_s}{\underbrace{ c_s,...,c_s}})
    \end{equation*}

    \begin{theor}
        Si $d=d(X)$, entonces
        \begin{equation*}
            m_1+...+m_d\geq d(Z)\geq m_{ s-d+1}+\cdots+m_s
        \end{equation*}
        además, si $m_1=\cdots=m_s=m$, entonces
        \begin{equation*}
            d(\mathcal{Z})=md(X)
        \end{equation*}
    \end{theor}

    \begin{exa}
        Considere
        \begin{equation*}
            \mathcal{Z}=3p_1+2p_2+p_3+p_4
        \end{equation*}
        como en el ejemplo anterior. Se determinó que $hyp(\mathcal{Z})=5$. Se tiene además que $d(\mathcal{Z})=n-5=7-2=2$. Se tiene que $X=\left\{p_1,p_2,p_3,p_4\right\}$, por lo que $d(X)=4-3=1$ y $hyp(X)=3$.
    \end{exa}

    \section{La distancia mínima y el grado inicial}

    \begin{mydef}
        Sea $\mathcal{C}$ un $[n,k,d]$ código lineal y $\mathcal{Z}_c$ su esquema de puntos gordos. El \textbf{grado inicial del ideal $I(\mathcal{Z}_c)$} denotado como
        \begin{equation*}
            \alpha(\mathcal{Z})=\alpha(I(\mathcal{Z}))=\min\left\{t\Big|I(\mathcal{Z})_t\neq0 \right\}
        \end{equation*}
    \end{mydef}
    (ver anillos graduados).

    En el caso anterior, se tiene que:
    \begin{equation*}
        I(\mathcal{Z})=\bigoplus_{ t\geq0}I(\mathcal{Z})_t
    \end{equation*}

    Una familia \textit{especial} de códigos es la llamada familia de \textbf{códigos evaluación}: sea $X=\left\{p_1,...,p_n \right\}\subseteq\mathbb{P}^{ k-1}$ un conjunto finito. Se tiene que
    \begin{equation*}
        R=\mathbb{K}[x_1,...,x_k]=\bigoplus_{ t\geq0}R_t
    \end{equation*}
    donde $R_a$ es un espacio vectorial de polinomio homogéneos de grado $t$.

    Definimos un mapeo lineal
    \begin{equation*}
        \cf{ev_t}{R_t}{\mathbb{K}^n}
    \end{equation*}
    dado por:
    \begin{equation*}
        ev_t(f)=(f(p_1),...,f(p_n))
    \end{equation*}

    \begin{mydef}
        El código \textbf{evaluación} es la imagen de $ev_t(R_t)\subseteq\mathbb{K}^n$.
    \end{mydef}

    \begin{obs}
        El código evaluación es un código lineal de orden/grado $t$ en el conjunto $X$.
    \end{obs}

    \begin{theor}
        Todo código lineal es un código evaluación.
    \end{theor}

    \begin{proof}
        
    \end{proof}

    \begin{propo}
        El código evaluacón tiene los parámetros básicos que cumplen las igualdades:
        \begin{itemize}
            \item Longitud, $n=\abs{X}=\deg\left(R/I(X) \right)=e$, donde $e$ es llamado la multiplicidad de Hulbert-Samuel.
            \item Dimensión es igual a $H(R/I(X),t)$
            \item Distancia mínima, $d(ev_t(R_t))=d$, y
            \begin{equation*}
                d_t=n-\max_{ X'\subseteq X}\left\{\abs{X'}\Big|\dim_{\mathbb{K}}(I(X')_t)>\dim_{\mathbb{K}}(I(X)_t) \right\}
            \end{equation*}
        \end{itemize}
    \end{propo}

    Nos dedicaremos a probar esta proposición.

    \begin{mydef}
        Se define la función de Hilbert de un ideal graduado $I\subseteq R$:
        \begin{equation*}
            H_I(n)=\dim_\mathbb{K}\left(R/I \right)_n
        \end{equation*}
    \end{mydef}

    \begin{mydef}
        con lo que se define la serie de Hilbert:
        \begin{equation*}
            \begin{split}
                Hilbert_I(t)&=\sum_{ i\geq0}\dim_{\mathbb{K}}(R/I)_it^i\\
                &=\sum_{ i\geq0}H_i(t)^i\\
            \end{split}
        \end{equation*}        
    \end{mydef}

    \begin{theor}[\textbf{Teorema de Hilbert-Serre}]
        \begin{equation*}
            Hilbert_I(t)=\frac{h(t)}{(1-t)^s}
        \end{equation*}
        donde $s=\dim(R/I)-1$ y $h(t)\in\mathbb{Q}[t]$ y $\deg(h)\leq n$.
    \end{theor}

    \begin{mydef}
        Se define el grado de multiplicidad de $R/I$ por:
        \begin{equation*}
            \deg(R/I)=e(R/I)=h(1)
        \end{equation*}
        con $h$ dada en la función anterior.
    \end{mydef}

    \begin{exa}
        Considere el anillo $R=K[x,y]$ y el ideal $I=\langle x^2,xy^2,y^4\rangle$. Queremos calcular $e(R/I)$. Se tiene el anillo graduado:
        \begin{equation*}
            R/I=\bigoplus_{ n\geq 1}(R/I)_n
        \end{equation*}
        se tiene que:
        \begin{center}
            \begin{tabular}{c | c | c | c}
                Grado & Graduado & Base & Valor Hilbert $H_I(n)$ \\
                \hline
                $n=0$ & $(R/I)_0=\mathbb{K}$ & $\left\{1 \right\}$ & 1 \\
                $n=1$ & $(R/I)_1$ & $\left\{x,y \right\}$ & 2 \\
                $n=2$ & $(R/I)_2$ & $\left\{xy,y^2 \right\}$ & 2 \\
                $n=3$ & $(R/I)_3$ & $\left\{y^3 \right\}$ & 1 \\
                $n\geq4$ & $(R/I)_4=\langle 0\rangle$ & $\emptyset$ & 0 \\
            \end{tabular}
        \end{center}
        (todas estas son bases sobre el campo $\mathbb{K}$).
        Por lo cual tenemos las anteriores funciones de Hilbert, con lo que se genera la siguiente serie de Hilbert:
        \begin{equation*}
            Hilbert_I(t)=1t^0+2t^1+2t^3+1t^3
        \end{equation*}
        por lo cual:
        \begin{equation*}
            e(R/I)=h(1)=6
        \end{equation*}
    \end{exa}

    \begin{exa}
        Considere el anillo $R=\mathbb{K}[x,y,z]$ y al ideal $I=\langle x^2,xy^2,y^3\rangle$. Tomamemos:
        \begin{equation*}
            M=R/I=\mathbb{K}[x,y,z]/\langle x^2,xy^2,y^3\rangle
        \end{equation*}
        se tiene:
        \begin{center}
            \begin{tabular}{c | c | c | c}
                Grado & Graduado & Base & Valor Hilbert $H_M(n)$ \\
                \hline
                $n=0$ & $M_0=\mathbb{K}$ & $\left\{1 \right\}$ & 1 \\
                $n=1$ & $M_1$ & $\left\{x,y,z \right\}$ & 3 \\
                $n=2$ & $M_2$ & $\left\{y^2,z^2,xz,yz,xy \right\}$ & 5 \\
                $n=3$ & $M_3$ & $\left\{z^3,xz^2,yz^2,y^2z,xyz, \right\}$ & 5 \\
                $n=4$ & $M_4$ & $\left\{z^4,xz^3,yz^3,xyz^2,y^2z^2\right\}$ & 5 \\
                $\vdots$ & $\vdots$ & $\vdots$ & $\vdots$ \\
                $n=i$ & $M_i$ & $\left\{z^i,xz^{ i-1},yz^{ i-1},xyz^{ i-2},y^2z^{ i-2}\right\}$ & 5 \\
                $\vdots$ & $\vdots$ & $\vdots$ & $\vdots$ \\
            \end{tabular}
        \end{center}
        entonces, la serie de Hilbert es infinita y es:
        \begin{equation*}
            Hilbert_M(t)=1+3t+5t^2+5t^3+5t^4+\cdots=\frac{1+2t+2t^2}{1-t}
        \end{equation*}
        (ejercicio) por tanto:
        \begin{equation*}
            \deg(M)=e(M)=1+2+2=5
        \end{equation*}
    \end{exa}

    En general, todo lo anterior que hemos hecho se puede definir para cualquier $R$-módulo graduado.

    \begin{propo}[\textbf{Propiedad Aditiva de $e(R/I)$}]
        Si $I=q_1\cap\cdots\cap q_l$ es una descomposición primaria de $I$, entonces
        \begin{equation*}
            e(R/I)=\sum_{height(Q_i)=ht(I)}e(R/q_i)
        \end{equation*}
    \end{propo}

    Continúamos ahora con los códigos evaluación. Recordemos que fijamos puntos en el espacio:
    \begin{equation*}
        x=\left\{p_1,...,p_n \right\}\subseteq\mathbb{P}^{ k-1}
    \end{equation*}
    y formábamos una función $\cf{ev_t}{R_t}{\mathbb{K}^n}$ dada por:
    \begin{equation*}
        f\mapsto(f(p_1),\cdots,f(p_n))
    \end{equation*}
    hacíamos $\mathcal{C}=ev_t(R_t)$ el código evaluación. Se tien que $ev_t$ es un mapeo lineal. Se tiene:
    \begin{equation*}
        \begin{split}
            \ker(ev_t)&=\left\{f\in R_t\Big|ev_t(f)=0 \right\}\\
            &=\left\{f\in R_t\Big|f(p_i)=0,\quad\forall i=1,...,n \right\}\\
            &=I(X)_t\\
        \end{split}
    \end{equation*}
    siendo $I(X)_t$ el ideal de anulación de todos los polinomios de grado $t$.

    Por el primer teorema de isomorfismo se sigue que:
    \begin{equation*}
        R_t/I(X)_t\cong ev_t(R_t)=\mathcal{C}
    \end{equation*}
    como todo código lineal es un código evaluacíon.

    Recordemos que para un código $\mathcal{C}$ se tiene que su longitud es $\abs{X}=e(R/I(X))$. En la clase pasada se dijo que:
    \begin{equation*}
        I(X)=I(p_1)\cap\cdots\cap I(p_n)
    \end{equation*}
    es una descomposición primaria del ideal $I(X)$. Se puede probar que:
    \begin{itemize}
        \item $ht(I(X))=ht(I(p_i))$ para todo $i$.
        \item $e(R/I(p_i))=1$.
    \end{itemize}
    en particular, por la propiedad aditiva de $e(R/I)$ se sigue que:
    \begin{equation*}
        e(R/I(X))=\sum_{ ht(I(X))=ht(I(p_i))} e(R/I(p_i))=n=\abs{X}
    \end{equation*}
    por lo que tenemos ahora una forma únicamente algebraica para interpretar la lóngitud de un código.

    Para el código $\mathcal{C}$, se tiene que:
    \begin{equation*}
        \dim_{\mathbb{K}}(\mathcal{C})=\dim\left([R/I(X)]_t \right)=H_{ I(X)}(t)
    \end{equation*}
    considerando el anillo graduado $R/I=\bigoplus_{ t\geq1}(R/I)_t$.

    \begin{propo}
        La distancia mínima del código $\mathcal{C}(X,t)$ está dada por:
        \begin{equation*}
            d(X)_t=m-\max_{ X'\subseteq X}\left\{\abs{X'}\Big|\dim_{\mathbb{K}}(I(X'))>\dim_{\mathbb{K}}(I(X)_t) \right\}
        \end{equation*}
    \end{propo}

    \begin{proof}
        Sea $w\in\mathcal{C}(X,t)$ palabra de peso de Hamming $s\geq1$ que alcanza la distancia mínima, luego $w$ es de la forma:
        \begin{equation*}
            w=(f(p_1),...,f(p_m))
        \end{equation*}
        donde el grado del polinomio es $t$ (todo esto ya hace sentido). En particular, podemos elegir $i_1,...,i_s$ tales que
        \begin{equation*}
            f(p_{ i_1}),...,f(p_{ i_s})\neq0
        \end{equation*}
        y, $f(p_j)=0$ para todo $j\in\left\{1,...,n \right\}\setminus\left\{i_1,...,i_s \right\}$. Sea
        \begin{equation*}
            X'=\left\{p_j\Big|j\in\left\{1,...,n \right\}\setminus\left\{i_1,...,i_s \right\}\right\}
        \end{equation*}
        es claro que $X'\subseteq X$, en particular $I(X)_t\subseteq I(X')_t$, luego
        \begin{equation*}
            \dim_{\mathbb{K}}(I(X')_t)>\dim_{\mathbb{K}}(I(X)_t)
        \end{equation*}
        y es estricta ya que $f\in I(X')_t\setminus I(X)_t$. Se tiene también que
        \begin{equation*}
            s=m-\abs{X'}
        \end{equation*}
        por lo que $X'$ es tal que:
        \begin{equation*}
            d(X)_t=m-\max_{ X'\subseteq X}\left\{\abs{X'}\Big|... \right\}
        \end{equation*}
        lo cual prueba el resultado.
    \end{proof}

    \begin{mydef}
        El \textbf{grado inicial} de un ideal homogéneo $\alpha(I(\mathcal{Z}_c))=\min\left\{t\big|(I(\mathcal{Z}_c))_t\neq0 \right\}$. Lo denotaremos por
        \begin{equation*}
            \alpha(\mathcal{Z})=\alpha(I(\mathcal{Z}_c))
        \end{equation*}
    \end{mydef}

    \begin{propo}
        Sea $1\leq t\leq\alpha(X)-1$, entonces
        \begin{equation*}
            d(X)_t+u\geq(k-1)(\alpha(X)-t)
        \end{equation*}
        para algún $u\in\left\{0,...,k-2 \right\}$, con $X\subseteq\mathbb{P}^{ k-1}$.
    \end{propo}

    \begin{proof}
        Supongamos que $X=\left\{p_1,...,p_m \right\}$. Denotemos por:
        \begin{equation*}
            d=d(X)_t\geq1
        \end{equation*}
        Sea $X'\subseteq X$ tal que
        \begin{equation*}
            d=m-\abs{X'}\Rightarrow\abs{X'}=m-d
        \end{equation*}
        (por la proposición anterior). Sea
        \begin{equation*}
            Y=X\setminus X'=\left\{Q_1,...,Q_d \right\}
        \end{equation*}
        y sea $f\in I(X')_t$.
        \begin{equation*}
            f(Q_i)\neq0
        \end{equation*}
        para todo $j=1,...,d$. Si $d\leq k-1$, entonces $Y$ están contenido en un hiperplano definido por $V(L)$ (ceros de $L$), donde $L$ es una forma lineal (esto es que $\deg(L)=1$) y entonces $f\cdot L\in I(X)$, entonces
        \begin{equation*}
            \alpha(X)\leq t+1
        \end{equation*}
        por lo que:
        \begin{equation*}
            \alpha(X)-t\leq 1\Rightarrow (k-1)(\alpha(X)-t)\leq k-1=d+(k-1)-d
        \end{equation*}
        por lo que tomando $u=(k-1)-d$ se tiene el resultado (esto si $d\leq k-1$).

        Si $d\geq k$, tomemos $\delta=\lceil\frac{d}{k-1}\rceil$, entonces,
        \begin{equation*}
            \delta\cdot(k-1)=d\cdot u
        \end{equation*}
        con $u\in\left\{0,...,k-2 \right\}$, entonces cualquier conjunto $Y$ de $k-1$ puntos pertenece a un hiperplano. Podemos considerar la unión de $\delta$-hiperplanos:
        \begin{equation*}
            V(L_1),\cdots,V(L_\delta)\subseteq\mathbb{P}^{k-1}
        \end{equation*}
        que contine a los puntos de $Y$ siendo cada $L_i$ una forma lineal de grado uno, entonces
        \begin{equation*}
            L_1\cdots L_\delta\cdot f\in I(X)
        \end{equation*}
        por lo que $\alpha(X)\leq \delta+t$. De forma inmediata se sigue que:
        \begin{equation*}
            d+u\geq(k-1)(\alpha(X)-t)
        \end{equation*}
    \end{proof}

    \begin{exa}
        Sea $\mathcal{C}$ un código con matriz generadora
        \begin{equation*}
            G=\left( 
                \begin{array}{ccccc}
                    1 & 0 & 0 & 1 & 0 \\
                    0 & 1 & 0 & -1 & 1 \\
                    0 & 0 & 1 & 0 & -1 \\
                \end{array}
            \right)
        \end{equation*}
        el esquema de puntos (reducido/gordos) está dado por:
        \begin{equation*}
            X=\left\{p_1=[1,0,0],p_2=[0,1,0],p_3=[0,0,1],p_4=[1,-1,0],p_5=[1,0,-1] \right\}\subseteq\mathbb{P}^2
        \end{equation*}
        Se tiene el ideal de definición:
        \begin{equation*}
            \begin{split}
                I(X)&=I(p_1)\cap\cdots\cap I(p_5)\\
                &=\langle x_2,x_3 \rangle\cap\langle x_1,x_3 \rangle\cap\langle x_1,x_2 \rangle\cap\langle x_1+x_2,x_3 \rangle\cap\langle x_1+x_3,x_2 \rangle
            \end{split}
        \end{equation*}
        (nomás basta ver como son los puntos para ver como se generan los ideales). Notemos que $p_1,p_2,p_4\in V(x_3)$ y $p_1,p_3,p_5\in V(x_2)$. Por ende, $x_2x_3\in I(X)$.

        Así, $\alpha(X)\leq 2$. Afirmamos que $\alpha(X)=2$ ya que no hay una forma lineal que se vaya a anular en todos.
        \begin{equation*}
            hyp(X)=3
        \end{equation*}
        Por lo cual,
        \begin{equation*}
            d(X)=5-hyp(X)=5-3=2
        \end{equation*}
        Si $t=1$ con $u=0$, tenemos que:
        \begin{equation*}
            \begin{split}
                d(X)_1+u&=2+0\\
                &=(3-1)(2-1)\\
                &=(k-1)(\alpha(X)-t)\\
            \end{split}
        \end{equation*}
        para $t=2$, necesitamos exhibir un polinomio de grado 2 que anule a la mayor cantidad de puntos de $X$, pero no a todos, elegimos $x_1x_3$ ya que $p_1,p_2,p_3,p_4\in V(x_1x_3)$. En este caso,
        \begin{equation*}
            d(X)_t=\abs{X}-\max_{ X'\subseteq X}\left\{\abs{X'}\Big|\cdots \right\}
        \end{equation*}
        note que $X'=\left\{p_1,...,p_4 \right\}$ y $p_5\notin V(x_1x_3)$, por lo que
        \begin{equation*}
            d(X)_2=5-4=1
        \end{equation*}
        (esto por el hecho de que $\alpha(X)=2$).
    \end{exa}

    \begin{cor}
        Sea $X\subseteq\mathbb{P}^{ k-1}$ con $k\geq 3$ es un conjunto de $n$ puntos reducidos no todos contenidos en un mismo hiperplano. Sea $t\geq 1$ un entero, entonces:
        \begin{equation*}
            d(X)_t\geq(k-1)(\alpha(X)-1-t)+1
        \end{equation*} 
    \end{cor}

    \begin{proof}
        No se verá.
    \end{proof}

    \begin{cor}
        Sea $X\subseteq\mathbb{P}^{ k-1}$ con $k\geq 3$ es un conjunto de $n$ puntos reducidos no todos contenidos en un mismo hiperplano. Sea $d(X)$ la distancia mínima del esquema $X$. Entonces,
        \begin{equation*}
            d(X)=\alpha(X)-1\iff hyp(X)=n-1\iff d=1
        \end{equation*}
    \end{cor}

    \begin{proof}
        Si $t=1$, del corolario anteriror tenemos que
        \begin{equation*}
            d(X)_t\geq(k-1)(\alpha(X)-2)+1
        \end{equation*}
        \begin{itemize}
            \item Si $d(X)=\alpha(X)-1$, se tiene que:
            \begin{equation*}
                \begin{split}
                    \alpha(X)-1&\geq(k-1)(\alpha(X)-2)+1\\
                    \Rightarrow 0&\geq(k-1)(\alpha(X)-2)+2-\alpha(X)\\
                    \Rightarrow 0&\geq (k-2)(\alpha(X)-2)\\
                \end{split}
            \end{equation*}
            como $k\geq 3$, entonces $\alpha(X)-2\leq 0$, luego al tenerse que el grado inicial es un número positivo, debe tenerse que $1\leq\alpha(X)\leq 2$, pero como no todos están en un mismo hiperplano, se sigue que $\alpha(X)=2$ (en caso que fuese 1, habría una forma lineal que se anulase en todos, luego todos están en un mismo hiperplano, cosa que no puede suceder).

            Luego, $d(X)=\alpha(X)-1=1$, lo cual implica que:
            \begin{equation*}
                hyp(X)=n-d(X)=n-1
            \end{equation*}
            \item Si $d(X)=1$, entonces del corolario anterior se sigue que:
            \begin{equation*}
                \begin{split}
                    1&\leq(k-1)(\alpha(X)-2)+1\\
                    \Rightarrow 0\geq(k-1)(\alpha(X)-2)\\
                \end{split}
            \end{equation*}
            como en el caso anterior, se sigue que $\alpha(X)=2$. Luego $d(X)=2-1=\alpha(X)-1\iff hyp(X)=n-1$.
        \end{itemize}
    \end{proof}

    Todas estás preguntas las hemos respondido en un esquema de puntos reducidos, pero ¿qué sucede en el caso de esquema de puntos gordos (es decir, con multiplicidad)?

    \section{El caso de puntos gordos}

    Sea $\mathcal{Z}=m_1p_1+\cdots+m_sp_s$ y $m_1,...,m_s\geq1$ es un esquema de puntos gordos no todos contenidos en un hiperplano y además,
    \begin{equation*}
        m_1+\cdots+m_s=n
    \end{equation*}
    sea $X=supp(\mathcal{Z})=\left\{p_1,...,p_s \right\}$. Denotemos por
    \begin{equation*}
        m=m(\mathcal{Z})=\max\left\{m_1,...,m_s \right\}
    \end{equation*}

    \begin{theor}
        Sea $\mathcal{Z}$ un esquema de puntos gordos, entonces
        \begin{equation*}
            d(\mathcal{Z})\geq\alpha(\mathcal{Z})-m
        \end{equation*}
    \end{theor}

    \begin{proof}
        No se hará.
    \end{proof}

    Ahora hablaremos de más parámetros sobre los códigos, en particular de lo siguiente:
    \begin{itemize}
        \item Pesos generalizados de Hamming.
        \item Códigos evaluación.
        \item Como una función sobre ideales graduados/función de Hilbert.
    \end{itemize}

    \begin{mydef}
        Sea $\mathcal{C}\subseteq\mathbb{K}^n$ un $[n,k,d]$ un código lineal. Considere $\mathcal{D}\subseteq\mathcal{C}$ un subcódigo (subespacio) de $\mathcal{C}$.

        El \textbf{soporte de $\mathcal{D}$} se define por:
        \begin{equation*}
            supp(\mathcal{D})=\left\{i\in\left\{1,...,n \right\}\Big|\exists (x_1,...,x_n)\in D\textup{ con }x_i\neq 0 \right\}
        \end{equation*}
        y sea $m(D)=\abs{supp(D)}$.
    \end{mydef}

    \begin{obs}
        Para $c=(c_1,...,c_n)\in\mathcal{C}$:
        \begin{equation*}
            supp(c)=\left\{i\in\left\{1,...,n \right\}\Big|c_i\neq0 \right\}=w(c)
        \end{equation*}
        recordemos que esto es el peso de Hamming de una palabra.
    \end{obs}

    \begin{mydef}
        Sea $\mathcal{C}\subseteq\mathbb{K}^n$ un $[n,k,d]$ un código lineal. Para $r\in\left\{1,...,k \right\}$, definimos el \textbf{$r$-ésimo peso generalizado de Hamming de $\mathcal{C}$} por:
        \begin{equation*}
            d_r(\mathcal{C})=\min_{\mathcal{D}\subseteq\mathcal{C}}\left\{m(\mathcal{D})\Big|\dim_{\mathbb{K}}(\mathcal{D})=r \right\}
        \end{equation*}
    \end{mydef}

    \begin{obs}
        Veamos quién es uno por uno:
        \begin{equation*}
            d_1(\mathcal{C})=\min_{ \mathcal{D}\subseteq\mathcal{C}}\left\{m(\mathcal{D})\Big|\dim_{\mathbb{K}}(\mathcal{D})=1 \right\}
        \end{equation*}
        entonces, $\mathcal{D}$ es un subespacio de $\mathcal{C}$ generado por un único elemento, digamos $D=\gen{c}$ con $c\in\mathcal{C}$. Por lo que:
        \begin{equation*}
            \begin{split}
                d_1(\mathcal{C})&=\min_{ c\in\mathcal{C}}\left\{w(c)\Big|c\in\mathcal{C}\setminus\left\{0\right\} \right\}\\
                &=d(\mathcal{C})\\
            \end{split}
        \end{equation*}
        es decir que recuperamos el peso de Hamming original.
    \end{obs}

    Consideremos ahora $X0\left\{p_1,...,p_m \right\}\subseteq\mathbb{P}^{ k-1}$ y tomemos el anillo de polinomios $R=\mathbb{K}[x_1,...,x_m]=\bigoplus_{ t\geq1}R_t$ con la graduación estándar.

    Se definió la función evaluacíon:
    \begin{equation*}
        \begin{split}
            ev_t:R_t&\rightarrow\mathbb{K}^{m}\\
            f&\mapsto(f(p_1),...,f(p_m))\\ 
        \end{split}
    \end{equation*}
    se tomó el código evaluación con parámetro $t$ por $\mathcal{C}(X,t)=ev_t(R_t)$. Se sabe además que:
    \begin{equation*}
        \mathcal{C}(X,t)\cong R_t/\ker(ev_t)=R_t/I(X)_t
    \end{equation*}

    El peso de Hamming de una palabra $c$ cuenta el número de entradas no cero de una palabra, lo cual lo podemos ver coom elnúmero de no raíces de un polinomio $f$ en $X$, el cual es
    \begin{equation*}
        \abs{X}-\abs{\left\{\textup{raíces de }f \right\}}=\abs{X}-\abs{V_X(f)}
    \end{equation*}
    entonces,
    \begin{equation*}
        \begin{split}
            d(\mathcal{C})&=\min\left\{\abs{X}-\abs{V_X(f)}\Big|f\in\mathbb{K}[x_1,...,x_m]\textup{ es tal que }f\notin I(X)_t \right\}\\
            &=\deg(R_t/I(X)_t)-\max\left\{\abs{V_X(f)}\Big|f\notin I(X)_t \right\}\\
        \end{split}
    \end{equation*}

    \begin{center}
        \textit{¿Cómo calcular la cardinalidad de $\abs{V_X(f)}$ donde $f$ es un polinomio homogéneo de grado $t$ y $X\subseteq\mathbb{P}^{k-1}$?}
    \end{center}

    En 2018 se encontró lo siguiente: bajo las condiciones anteriores:
    \begin{equation*}
        \abs{V_X(f)}=\left\{
            \begin{array}{lcr}
                \deg\left(R/(I(X),f) \right) & \textup{ si } & (I(X):f)\neq I(X) \\
                0 & \textup{ si } & (I(X):f)= I(X) \\
            \end{array}
        \right.
    \end{equation*}
    el ideal $(I:f)$ es llamado \textbf{ideal colon/cociente}, formado por:
    \begin{equation*}
        (I:f)=\left\{h\in R\Big|hf\in I \right\}
    \end{equation*}
    por lo cual,
    \begin{equation*}
        \begin{split}
            d(\mathcal{C})&=\deg(R_t/I(X)_t)-\max\left\{\deg(R_t/(I(X)_t:f))\Big|(I(X):f)\neq I(X) \right\}\\
        \end{split}
    \end{equation*}

    Ahora, para los pesos generalizados. En el caso en que $r=1$, solamente necesitamos un polinomio.

    Considere el conjunto de polinomios:
    \begin{equation*}
        \mathcal{F}_{ t,r}=\left\{\left\{f_1,...,f_r \right\}\subseteq R_t \Big|((I(X)_t:\langle f_1,...,f_r\rangle))\neq I(X)_t \right\}
    \end{equation*}
    donde ahora el ideal colon/cociente es:
    \begin{equation*}
        (I:J)=\left\{h\in R\Big|hJ\subseteq I \right\}
    \end{equation*}

    Con lo cual, la forma de calcular el $r$-ésimo grado generalizado se convierte en computar:
    \begin{equation*}
        d_r(\mathcal{C})=\deg\left(R_t/I(X)_t \right)-\max\left\{\deg\left(R_t/(I(X)_t,f_1,...,f_r) \right)\Big|\left\{f_1,...,f_r \right\}\in\mathcal{F}_{ t,r}\neq\emptyset \right\}
    \end{equation*}

    \newpage

    \section{Ejercicios}

    

\end{document}