%% Inicio del archivo `template.tex'.
%% Copyright 2006-2013 Xavier Danaux (xdanaux@gmail.com).
%
% Este trabajo puede ser distribuido o modificado bajo las 
% condiciones de la LaTeX Project Public License V1.3c, 
% disponible en http://www.latex-project.org/lppl/.
%
% Traducción al Español por Fausto M. Lagos (piratax007@protonmail.ch), 2016.


\documentclass[11pt,a4paper,sans]{moderncv}        % posibles opciones de tamaño de fuente ('10pt', '11pt' and '12pt'), papel ('a4paper', 'letterpaper', 'a5paper', 'legalpaper', 'executivepaper' and 'landscape') y familia de fuentes ('sans' and 'roman')

% temas moderncv
\moderncvstyle{casual}                             % Los estilos disponibles son 'casual' (default), 'classic', 'oldstyle' and 'banking'
\moderncvcolor{blue}                               % las opciones de color son 'blue' (default), 'orange', 'green', 'red', 'purple', 'grey' and 'black'
%\renewcommand{\familydefault}{\sfdefault}         % descomentar al inicio de la línea para definir la fuente por defecto; use '\sfdefault' para sans serif por defecto, '\rmdefault' para roman, o cualquier otro nombre de fuente instalada en sus sistema
%\nopagenumbers{}                                  % descomente para eliminar el numerado automático de las páginas en cartas de más de una página

% Codificación de carácteres
\usepackage[utf8]{inputenc}                        % Si no esta usando xelatex o lualatex, remplace por la codificación que este usando
%\usepackage{CJKutf8}                              % descomente si necesita usar CJK para escribir su carta en Chino, Japones or Koreano
\usepackage[spanish, english]{babel}			   % comentar si su carta esta escrita en un idioma diferente del Español

% Configuración de márgenes
\usepackage[scale=0.75]{geometry}

\usepackage{ragged2e}
%\setlength{\hintscolumnwidth}{3cm}                % descomente si quiere modificar el ancho de columna para la fecha
%\setlength{\makecvtitlenamewidth}{10cm}           % para el estilo 'classic', si quiere forzar el ancho del nombre. la longitud es normalmente calculada para evitar sobrelapamientos con su información personal; descomente esta línea bajo su propio riesgo

% Información personal
\name{Cristo Daniel Alvarado, estudiante del 8vo semestre de la Licenciatura en Física y Matemáticas de la ESFM}{}

% para mostrar etiquetas numéricas en la bibliografía (por defecto no se muestran etiquecas); descomente las siguientes líneas solo si usa referencias bibliográficas en su carta
%\makeatletter
%\renewcommand*{\bibliographyitemlabel}{\@biblabel{\arabic{enumiv}}}
%\makeatother
%\renewcommand*{\bibliographyitemlabel}{[\arabic{enumiv}]} % Considere reemplazar la línea 44 con esta

% bibliografía con múltiples entradas
%\usepackage{multibib}
%\newcites{book,misc}{{Books},{Others}}
%----------------------------------------------------------------------------------
%            contenido
%----------------------------------------------------------------------------------
\begin{document}
%-----       carta       ---------------------------------------------------------
% Datos del destinatario
\recipient{A QUIEN CORRESPONDA}{Ciudad de México, México}
\date{1 de enero de 2025}
\opening{}
\closing{Esperando una respuesta favorable, le envío un cordial saludo.}

\makelettertitle

\justifying

Por la presente, me dirijo a ustedes con el propósito de expresar mi interés en formar parte del proyecto: Desarrollo de Herramientas y Análisis de Datos con Inteligencia Artificial ofrecido por ustedes. Como estudiante de la licenciatura en física y matemáticas, con especialización en matemáticas abstractas en el último semestre de mi formación universitaria y Técnico en Programación, estoy convencido/a de que este programa es el siguiente paso natural en mi desarrollo académico y profesional.

Durante mi carrera, he adquirido una sólida formación en matemáticas y programación. Mi experiencia incluye el conocimiento de diversos lenguajes de programación como Python, C, Java y JavaScript. He desarrollado algunos proyectos de programación como el desarrollo de Cyclon, una aplicación web desarrollada con el objetivo de dar seguimiento de huracanes en la zona de la República Mexicana para las personas en la costa de México. Fue desarrollada una aplicación móvil y una página web donde es posible consultar predicciones sobre la trayectoria de huracanes en la república. El proyecto fue desarrollado en NodeJS, Python, Javascript, Java y MySQL, mismo que fue acreedor del segundo lugar en el vigésimo noveno concurso: \textbf{Premio a los mejores prototipos de nivel superior: 2020}, oraganizado por el IPN.
ssssssssssss
Estoy seguro/a de que mi formación académica, mi experiencia en proyectos de programación y mi entusiasmo por aprender serán valiosos para aportar al éxito del programa y enriquecer el entorno de aprendizaje colaborativo que ustedes promueven. A la vez, estoy ansioso/a por adquirir nuevos conocimientos y habilidades que me permitan continuar creciendo profesionalmente y generar un impacto positivo en el campo de la inteligencia artificial y el análisis de datos.

Agradezco de antemano la oportunidad de ser considerado/a para este programa. Estoy disponible para proporcionar cualquier información adicional que consideren necesaria. Espero tener la oportunidad de demostrar mi compromiso y potencial como parte de esta iniciativa.

Atte.ss
sss
Cristo Daniel Alvarado.

\end{document}