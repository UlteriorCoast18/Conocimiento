%% Inicio del archivo `template.tex'.
%% Copyright 2006-2013 Xavier Danaux (xdanaux@gmail.com).
%
% Este trabajo puede ser distribuido o modificado bajo las 
% condiciones de la LaTeX Project Public License V1.3c, 
% disponible en http://www.latex-project.org/lppl/.
%
% Traducción al Español por Fausto M. Lagos (piratax007@protonmail.ch), 2016.


\documentclass[11pt,a4paper]{moderncv}        % posibles opciones de tamaño de fuente ('10pt', '11pt' and '12pt'), papel ('a4paper', 'letterpaper', 'a5paper', 'legalpaper', 'executivepaper' and 'landscape') y familia de fuentes ('sans' and 'roman')
\usepackage{cfr-lm}
\usepackage{fontawesome5}

% temas moderncv
\moderncvstyle{casual}                             % Los estilos disponibles son 'casual' (default), 'classic', 'oldstyle' and 'banking'
\moderncvcolor{blue}                               % las opciones de color son 'blue' (default), 'orange', 'green', 'red', 'purple', 'grey' and 'black'
%\renewcommand{\familydefault}{\sfdefault}         % descomentar al inicio de la línea para definir la fuente por defecto; use '\sfdefault' para sans serif por defecto, '\rmdefault' para roman, o cualquier otro nombre de fuente instalada en sus sistema
%\nopagenumbers{}                                  % descomente para eliminar el numerado automático de las páginas en cartas de más de una página

% Codificación de carácteres
\usepackage[utf8]{inputenc}                        % Si no esta usando xelatex o lualatex, remplace por la codificación que este usando
%\usepackage{CJKutf8}                              % descomente si necesita usar CJK para escribir su carta en Chino, Japones or Koreano
\usepackage[spanish, english]{babel}			   % comentar si su carta esta escrita en un idioma diferente del Español

% Configuración de márgenes
\usepackage[scale=0.75]{geometry}

\usepackage{ragged2e}
%\setlength{\hintscolumnwidth}{3cm}                % descomente si quiere modificar el ancho de columna para la fecha
%\setlength{\makecvtitlenamewidth}{10cm}           % para el estilo 'classic', si quiere forzar el ancho del nombre. la longitud es normalmente calculada para evitar sobrelapamientos con su información personal; descomente esta línea bajo su propio riesgo


% Información personal
\name{Cristo Daniel Alvarado, pasante de la Licenciatura en Física y Matemáticas de la ESFM}{}

% para mostrar etiquetas numéricas en la bibliografía (por defecto no se muestran etiquecas); descomente las siguientes líneas solo si usa referencias bibliográficas en su carta
%\makeatletter
%\renewcommand*{\bibliographyitemlabel}{\@biblabel{\arabic{enumiv}}}
%\makeatother
%\renewcommand*{\bibliographyitemlabel}{[\arabic{enumiv}]} % Considere reemplazar la línea 44 con esta

% bibliografía con múltiples entradas
%\usepackage{multibib}
%\newcites{book,misc}{{Books},{Others}}
%----------------------------------------------------------------------------------
%            contenido
%----------------------------------------------------------------------------------
\begin{document}
%-----       carta       ---------------------------------------------------------
% Datos del destinatario
\recipient{Dirección General de Administración Escolar
Universidad Nacional Autónoma de México}{Ciudad de México, México}
\date{10 de febrero de 2025}
\opening{}
\closing{Esperando una respuesta favorable, envío un cordial saludo.}

\makelettertitle

\justifying

Estimados miembros del comité:

Me permito la presente para expresar mi interés en ser admitido al programa de Maestría en Ciencias Matemáticas que ofrece el Instituto de Matemáticas de la Universidad Nacional Autónoma de México. Mi nombre es Cristo Daniel Alvarado, soy egresado de la Licenciatura en Física y Matemáticas con especialización en Matemáticas de la Escuela Superior de Física y Matemáticas del Instituto Politécnico Nacional, y a lo largo de mi formación académica, he adquirido un profundo interés por las áreas de Álgebra, Topología, Geometría y Análisis.

En el transcurso de mi carrera universitaria, me he formado de manera rigurosa en diversos campos de las matemáticas, consolidando una base sólida tanto en teoría como en resolución de problemas complejos. Esta formación ha despertado en mí una fascinación particular por las áreas mencionadas, sobre todo el área de Álgebra, que desde sus inicios ha sido el foco de mi atención, llegando al punto en que mi tesis de Licenciatura analiza un problema de consistencia de Álgebra Topológica.

Uno de los momentos clave que reafirmó mi decisión de continuar mis estudios en matemáticas fue mi participación en la 10° Escuela Oaxaqueña de Matemáticas, donde tuve la oportunidad de profundizar en temas como lo son la Topología Algebraica y la Teoría Geométrica de Grupos. Interactuar con académicos y estudiantes de alto nivel y, sobre todo, apreciar la relevancia que tiene la investigación matemática tanto a nivel local como global ayudo a convencerme de que debo continuar mi formación en un entorno académico de excelencia, y la UNAM, con su prestigioso programa de posgrado y su comunidad de matemáticos de primer nivel, representa el lugar ideal para desarrollar mis conocimientos y habilidades en el área.

A través de este posgrado, espero poder adquirir una formación más especializada en los campos que me apasionan, contribuyendo con mi esfuerzo y dedicación al desarrollo de nuevas ideas que puedan enriquecer la disciplina matemática. Además, considero que la interacción con los profesores y compañeros de la UNAM será invaluable para ampliar mis perspectivas y fortalecer mi capacidad crítica y creativa. Agradezco de antemano la atención prestada a mi solicitud y quedo a su disposición para cualquier duda o información adicional que necesiten.

Atentamente,

Cristo Daniel Alvarado.

\end{document}