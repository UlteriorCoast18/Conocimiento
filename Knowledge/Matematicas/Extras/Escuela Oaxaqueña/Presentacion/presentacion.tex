%----------------------------------------------------------------------------------------
%    PACKAGES AND THEMES
%----------------------------------------------------------------------------------------

\documentclass[xcolor=dvipsnames]{beamer}
\usetheme{SimpleDarkBlue}

\usepackage{hyperref}
\usepackage{graphicx}
\usepackage{booktabs}
\usepackage[spanish]{babel}
\usepackage[utf8]{inputenc}
\usepackage{amsmath}
\usepackage{amssymb}
\usepackage{amsthm}
\usepackage{graphics}
\usepackage{subfigure}
\usepackage{lipsum}
\usepackage{array}
\usepackage{multicol}
\usepackage{enumerate}
\usepackage[framemethod=TikZ]{mdframed}
\usepackage{geometry}
\usepackage{tikz}
\usepackage{pgffor}
\usepackage{ifthen}
\usepackage{enumitem}
\usepackage{hyperref}
\usepackage{listings}
\usepackage{bbm}

%Gestión de Hipervínculos

\hypersetup{
    colorlinks=true,
    linkcolor=black,
    filecolor=magenta,      
    urlcolor=cyan
}

%En esta parte se hacen redefiniciones de algunos comandos para que resulte agradable el verlos%

\def\proof{Demostración:\\}
\def\endproof{\hfill$\blacksquare$}

\def\sol{Solución:\\}
\def\endsol{\hfill$\square$}

%En esta parte se definen los comandos a usar dentro del documento para enlistar%

\newtheoremstyle{largebreak}
  {}% use the default space above
  {}% use the default space below
  {\normalfont}% body font
  {}% indent (0pt)
  {\bfseries}% header font
  {}% punctuation
  {\newline}% break after header
  {}% header spec

\theoremstyle{largebreak}

\newmdtheoremenv[
    hidealllines = true
]{exa}{Ejemplo}[section]

\newmdtheoremenv[
    hidealllines = true
]{obs}{Observación}[section]

\newmdtheoremenv[
    hidealllines = true
]{theor}{Teorema}[section]

\newmdtheoremenv[
    hidealllines = true
]{propo}{Proposición}[section]

\newmdtheoremenv[
    hidealllines = true
]{cor}{Corolario}[section]

\newmdtheoremenv[
    hidealllines = true
]{lema}{Lema}[section]

\newmdtheoremenv[
    hidealllines = true
]{mydef}{Definición}[section]

\newmdtheoremenv[
    hidealllines = true
]{excer}{Ejercicio}[section]

%En esta parte se colocan comandos que definen la forma en la que se van a escribir ciertas funciones%

\newcommand\abs[1]{\ensuremath{\left|#1\right|}}
\newcommand\divides{\ensuremath{\bigm|}}
\newcommand\cf[3]{\ensuremath{#1:#2\rightarrow#3}}
\newcommand\contradiction{\ensuremath{\#_c}}
\newcommand\natint[1]{\ensuremath{\left[\big|#1\big|\right]}}
\newcounter{figcount}
\setcounter{figcount}{1}
\newcommand{\bbm}[1]{\ensuremath{\mathbbm{#1}}}
\newcommand{\pint}[2]{\langle#1\big|#2 \rangle}
\newcommand{\norm}[1]{\|#1\|}
\newcommand{\Isom}[1]{\ensuremath{\textup{Isom}\left(#1\right)}}
\newcommand{\SO}[1]{\ensuremath{\textup{SO}\left(#1\right)}}
\newcommand\Aut[1]{\ensuremath{\textup{Aut}\left(#1\right)}}
\newcommand{\Cay}[1]{\ensuremath{\textup{Cay}\left(#1\right)}}
\newcommand{\gen}[1]{\ensuremath{\langle#1\rangle}}
\newcommand{\qisom}{\ensuremath{\underset{C.I.}{\sim}}}
\newcommand{\SL}[1]{\ensuremath{\textup{SL}\left(#1\right)}}
\newcommand{\PSL}[1]{\ensuremath{\textup{PSL}\left(#1\right)}}
\newcommand{\Tr}[1]{\ensuremath{\textup{Tr}\left(#1\right)}}
\newcommand{\im}[1]{\ensuremath{\textup{im}\left(#1\right)}}
\newcommand{\Diam}[1]{\ensuremath{\textup{diam}\left(#1\right)}}
\newcommand{\arcosh}[1]{\ensuremath{\textup{arcosh}\left(#1\right)}}

%----------------------------------------------------------------------------------------
%    TITLE PAGE
%----------------------------------------------------------------------------------------

\title{Grupos y Geometría: Acciones, Dimensión y Dualidad}
\subtitle{Plano Hiperbólico, Aplicaciones del Teorema de Svarc-Milnor y Grupos Hiperbólicos}

\author{Cristo Daniel Alvarado}

\institute
{
    Escuela Superior de Física y Matemáticas \\
    Instituto Politécnico Nacional
}
\date{\today} % Date, can be changed to a custom date

%----------------------------------------------------------------------------------------
%    PRESENTATION SLIDES
%----------------------------------------------------------------------------------------

\begin{document}

\begin{frame}
    \titlepage
\end{frame}

\begin{frame}{Index}
    \tableofcontents
\end{frame}

%----------------------------------------------------------------------------------------
\section{Plano Hiperbólico}
%----------------------------------------------------------------------------------------

\begin{frame}
    \begin{center}
        \LARGE{El Plano Hiperbólico $\bbm{H}^2$}
    \end{center}
    %TODO: Colocar imágenes del plano y el disco de poincaré
\end{frame}

\begin{frame}
    \begin{center}
        Construcción del Plano Hiperbólico
    \end{center}
    %TODO: ColocAR HERRAMIENTAS DE CONSTRUCCIÓN
\end{frame}

\subsection{Construcción del Plano Hiperbólico}

\begin{frame}{Construcción del Plano Hiperbólico}
    Hablaremos un poco sobre el plano hiperbólico y sus propiedades.
    \begin{mydef}[\textbf{Plano superior}]
        Escribimos:
        \begin{equation*}
            H=\left\{(x,y)\in\bbm{R}^2\Big|y>0 \right\}\subseteq\bbm{R}^2
        \end{equation*}
        para el \textbf{plano superior}.
    \end{mydef}
    \begin{obs}
        Dependiendo del contexto, veremos a $H$ como subconjunto de $\bbm{C}$, haciendo las identificaciones:
        \begin{equation*}
            H\rightarrow\left\{z\in\bbm{C}\Big|\Im z>0 \right\}
        \end{equation*}
        con la aplicación biyectiva $(x,y)\mapsto x+iy$.
    \end{obs}
\end{frame}

\begin{frame}{Construcción del Plano Hiperbólico}
    \begin{mydef}[\textbf{Haz tangente}]
        Sea $M$ una variedad $C^k$-diferenciable. El \textbf{fibrado tangente} o \textbf{haz tangente} es la unión disjunta de los espacios tangentes a cada punto de la variedad, dado por:
        \begin{equation*}
            TM=\bigsqcup_{ p\in M}T_pM=\bigcup_{ p\in M}\left\{p\right\}\times T_pM
        \end{equation*}
        donde $T_pM$ denota el espacio tangente a $M$ en el punto $p\in M$.
    \end{mydef}

    Como el conjunto $H$ es abierto y subconjunto de $\bbm{R}^2$, entonces este hereda la estructura de variedad suave de $\bbm{R}^2$. Además, como el haz tangente a $p\in\bbm{R}^2$ es trivial, se sigue también que el haz tangente a $H$ es trivial y por ende, podemos identificar de forma natural al espacio $T_zH$ como el espacio tangente de $x\in H$. Además, como $T_zH\cong\bbm{R}^2$, haremos la identificación de estos dos espacios como el mismo.
\end{frame}

\begin{frame}{Construcción del Plano Hiperbólico}
    \begin{mydef}[\textbf{Métrica Riemanniana}]
        Una \textbf{métrica Riemanniana} en una variedad $C^k$-diferenciable $M$ es una aplicación bilineal simétrica $\cf{g_p}{T_pM\times T_pM}{\bbm{R}}$ en cada uno de los espacios tangentes $T_pM$ de $M$.
    \end{mydef}

    \begin{obs}
        De la definición anterior se sigue que para cada $p\in M$ se satisface:
        \begin{enumerate}[label = \textit{(\arabic*)}]
            \item $g_p(u,v)=g_p(v,u)$ para todo $u,v\in T_pM$.
            \item $g_p(u,u)\geq0$ para todo $u\in T_pM$.
            \item $g_p(u,u)=0$ si y sólo si $u=0$.
        \end{enumerate}
    \end{obs}
\end{frame}

\begin{frame}{Construcción del Plano Hiperbólico}
    \begin{mydef}[\textbf{Plano Hiperbólico}]
        El \textbf{plano hiperbólico} $\bbm{H}^2$ es la variedad Riemanniana $(H,g_H)$, donde:
        \begin{itemize}
            \item $H\subseteq\bbm{R}^2$ hereda la estructura suave de $\bbm{R}^2$.
            \item Consideramos la métrica Riemanniana $\cf{g_{H,p}}{T_pH\times T_pH=\bbm{R}^2\times \bbm{R}^2}{\bbm{R}}$ dada por:
            \begin{equation*}
                g_{H,(x,y)}(u,v)=\frac{1}{y^2}\langle u,v\rangle,\quad\forall u,v\in\bbm{R}^2
            \end{equation*}
            para todo $(x,y)\in H$, donde $\langle\cdot\big|\cdot \rangle$ denota el producto interno usual de $\bbm{R}^2$. Más aún, escribiremos $\langle\cdot\big|\cdot \rangle_{ H,z}$ en vez de $g_{H,z}$ y a la norma inducida se le denotará por $\norm{\cdot}_{H,z}$.
        \end{itemize}
    \end{mydef}
\end{frame}

\begin{frame}{Construcción del Plano Hiperbólico}
    Nuestro interés ahora será hablar de las isometrías de $\bbm{H}^2$, para lo cual tendremos que construír una métrica en este espacio.
    
    \begin{mydef}[\textbf{Longitud hiperbólica de una curva}]
        Sea $\cf{\gamma}{[a,b]}{H}$ una curva suave. Se define la \textbf{longitud hiperbólica de $\gamma$} por:
        \begin{equation*}
            L_{\bbm{H}^2}(\gamma)=\int_{a}^{b}\norm{\dot{\gamma}(t)}_{ H,\gamma(t)}\:dt=\int_{a}^{b}\frac{\sqrt{\dot{\gamma_1}^2(t)+\dot{\gamma_2}^2(t)}}{\gamma_2(t)}\:dt
        \end{equation*}
        siendo $\gamma=(\gamma_1,\gamma_2)$.
    \end{mydef}
\end{frame}

\subsection{Isometrías del Plano Hiperbólico}

\begin{frame}
    \begin{center}
        Isometrías de $\bbm{H}^2$
    \end{center}
\end{frame}

\begin{frame}{Isometrías del Plano Hiperbólico}
    Resulta que con la definición anterior de longitud hiperbólica de una curva es posible inducir una métrica en el espacio $H$:

    \begin{propo}
        La función $\cf{d_H}{H\times H}{\bbm{R}_{\geq0}}$ dada por:
        \begin{equation*}
            (z,z')\mapsto\inf\left\{L_{\bbm{H}^2}(\gamma)\Big|\gamma\textup{ es una curva suave en $H$ que une a $z$ con $z'$} \right\}
        \end{equation*}
        es una métrica en $H$.
    \end{propo}
\end{frame}

\begin{frame}{Isometrías del Plano Hiperbólico}
    \begin{propo}
        Sea $\cf{\gamma}{[a,b]}{H}$ una curva suave. Entonces:
        \begin{equation*}
            L_{\bbm{H}^2}(\gamma)=L_{(H,d_H)}(\gamma)
        \end{equation*}
        donde $L_{(H,d_H)}$ es llamada la \textbf{longitud métrica} y está dada por:
        \begin{equation*}
            \begin{split}
                L_{(H,d_H)}=&\sup\left\{\sum_{ j=0}^{k-1}d_H(\gamma(t_j),\gamma(t_{j+1}))\Big| \right.\\
                &\quad\left.k\in\bbm{N}, t_0,t_1,...,t_k\in[a,b], t_0<t_1<\cdots <t_k \right\}\\
            \end{split}
        \end{equation*}
    \end{propo}

    Conociendo la métrica de este espacio, nos interesa conocer ahora las geodésicas del mismo. Para ello, primero veremos quiénes son sus isometrías.
\end{frame}

\begin{frame}{Isometrías del Plano Hiperbólico}
    \begin{mydef}[\textbf{Grupo de isometrías Riemanniano}]
        Una \textbf{isometría Riemanniana de $\bbm{H}^2$} es un difeomorfismo suave $\cf{f}{H}{H}$ que satisface:
        \begin{equation*}
            \forall z\in H, \forall v,v'\in T_zH,\quad\pint{(Df)_z(v)}{(Df)_z(v')}_{H,f(z)}=\pint{v}{v'}_{H,z}
        \end{equation*}
    \end{mydef}

    \begin{propo}[\textbf{Isometrías Riemannianias son isometrías}]
        Toda isometría Riemanniana de $\bbm{H}^2$ es una isometría métrica de $(H,d_H)$. En particular, existe un monomorfismo de grupos:
        \begin{equation*}
            \Isom{\bbm{H}^2}\rightarrow \Isom{H,d_H}
        \end{equation*}
    \end{propo}
\end{frame}

\subsection{El Grupo $\PSL{2,\bbm{R}}$}

\begin{frame}
    \begin{center}
        El Grupo $\PSL{2,\bbm{R}}$
    \end{center}
\end{frame}

\begin{frame}{El Grupo $\PSL{2,\bbm{R}}$}
    \begin{mydef}
        $\SL{n,\bbm{A}}$ denota al espacio de todas las matrices $2\times 2$ con entradas en $\bbm{A}\subseteq\bbm{C}$ tales que:
        \begin{equation*}
            \det(A)=1,\quad\forall A\in\bbm{A}
        \end{equation*}
    \end{mydef}
\end{frame}

\begin{frame}{El Grupo $\PSL{2,\bbm{R}}$}
    \begin{mydef}[\textbf{Transformaciones de Möbius}]
        Para la matriz $2\times 2$:
        \begin{equation*}
            \left(
                \begin{array}{cc}
                    a & b \\
                    c & d \\
                \end{array}
             \right)\in\SL{2,\bbm{R}}
        \end{equation*}
        definimos la \textbf{transformación de Möbius asociada} $\cf{f_A}{H}{H}$, dada por:
        \begin{equation*}
            z\mapsto\frac{a\cdot z+b}{c\cdot z+d}
        \end{equation*}
    \end{mydef}
\end{frame}

\begin{frame}{El Grupo $\PSL{2,\bbm{R}}$}
    \begin{obs}
        Toda transformación de Möbius está bien definida, ya que como $H$ es el plano superior, entonces la parte real de $z$ nunca será un número con parte imaginaria cero, así que $c\cdot z+d\neq 0$ para todo $z\in H$.
    \end{obs}

    \begin{exa}
        La función $z\mapsto z$ es una transformación de Möbius. Al igual que la función $z\mapsto\frac{1}{z}$. En particular, todas las funciones lineales de $H$ en $H$ son transformaciones de Möbius.
    \end{exa}
\end{frame}

\begin{frame}{El Grupo $\PSL{2,\bbm{R}}$}
    \begin{propo}
        Se tiene lo siguiente:
        \begin{enumerate}[label = \textit{(\arabic*)}]
            \item $f_A$ está bien definido y es un difeomorfismo $C^\infty$ (o suave).
            \item Para todo $A,B\in\SL{2,\bbm{R}}$ se tiene que $f_{A\cdot B}=f_A\circ f_B$.
            \item $f_A=f_{-A}$ para todo $A\in\SL{2,\bbm{R}}$.
        \end{enumerate}
    \end{propo}

    \begin{proof}
        De \textit{(1)} y \textit{(2)}: Son inmediatas.

        De \textit{(3)}: Si
        \begin{equation*}
            A=\left(
                \begin{array}{cc}
                    a & b \\
                    c & d \\
                \end{array}
             \right)\in\SL{2,\bbm{R}}
        \end{equation*}
        entonces,
        \begin{equation*}
            f_A(z)=\frac{a\cdot z+b}{c\cdot z+d}=\frac{-a\cdot z+-b}{-c\cdot z+-d}=f_{-A}(z)
        \end{equation*}
        para todo $z\in H$.
    \end{proof}

\end{frame}


\begin{frame}{El Grupo $\PSL{2,\bbm{R}}$s}
    \begin{exa}[\textbf{Generadores $\SL{2,\bbm{R}}$}]
        Tenemos los siguientes dos tipos de transformaciones de Möbius:
        \begin{itemize}
            \item Sea $b\in\bbm{R}$. Entonces, la transformación de Möbius asociada a la matriz:
            \begin{equation*}
                \left(\begin{array}{cc}
                    1 & b \\
                    0 & 1 \\
                \end{array} \right)\in\SL{2,\bbm{R}}
            \end{equation*}
            es la traslación horizontal $z\mapsto z+b$ en $H$ por un factor $b$ se denotará por $T_b$.
            \item La transformación de Möbius asociada a la matriz:
            \begin{equation*}
                \left(\begin{array}{cc}
                    0 & 1 \\
                    -1 & 0 \\
                \end{array} \right)\in\SL{2,\bbm{R}}
            \end{equation*}
            es la función $z\mapsto\frac{1}{z}$ se denotará por $In$.
        \end{itemize}
    \end{exa}
\end{frame}

\begin{frame}{El Grupo $\PSL{2,\bbm{R}}$}
    \begin{exa}[\textbf{Generadores $\SL{2,\bbm{R}}$}]
        Se tiene que el grupo $\SL{2,\bbm{R}}$ es generado por:
        \begin{equation*}
            \left\{\left(\begin{array}{cc}
                0 & 1 \\
                -1 & 0 \\
            \end{array} \right)\right\}\cup\left\{\left(\begin{array}{cc}
                1 & b \\
                0 & 1 \\
            \end{array} \right)\Big|b\in\bbm{R} \right\}
        \end{equation*}
    \end{exa}
\end{frame}

\begin{frame}{El Grupo $\PSL{2,\bbm{R}}$}
    \begin{proof}
        Notemos que:
        \begin{equation*}
            \left(\begin{array}{cc}
                0 & 1 \\
                -1 & 0 \\
            \end{array} \right)\cdot\left(\begin{array}{cc}
                1 & b \\
                0 & 1 \\
            \end{array} \right)\cdot\left(\begin{array}{cc}
                0 & 1 \\
                -1 & 0 \\
            \end{array} \right)=\left(\begin{array}{cc}
                1 & 0 \\
                -b & 1 \\
            \end{array} \right)
        \end{equation*}
        para todo $b\in\bbm{R}$. Así que todas las matrices de la forma:
        \begin{equation*}
            \left(\begin{array}{cc}
                1 & 0 \\
                a & 1 \\
            \end{array} \right)
        \end{equation*}
        está en el grupo generado por el conjunto anterior. Para terminar, basta notar que toda matriz en $\SL{2,\bbm{R}}$ admite una descomposición $LU$ o $UL$, dependiendo del caso.
        %TODO justificar
    \end{proof}
\end{frame}

\begin{frame}{El Grupo $\PSL{2,\bbm{R}}$}
    \begin{propo}[\textbf{Transformaciones de Möbius son isometrías}]
        Si $A\in\SL{2,\bbm{R}}$, entonces la transformación de Möbius asociada $\cf{f_A}{H}{H}$ es una isometría Riemanniana de $\bbm{H}^2$. En particular, tenemos un monomorfismo de grupos:
        \begin{equation*}
            \PSL{2,\bbm{R}}=\SL{2,\bbm{R}}/\left\{I,-I \right\}\rightarrow\Isom{H,d_H}
        \end{equation*}
        dado por $[A]\mapsto f_A$.
    \end{propo}

    \begin{proof}
        Por el ejemplo anterior basta con ver que $T_b$ y $In$ son isometrías Riemannianas de $\bbm{H}^2$, ya que la composición de isometrías Riemannianias sigue siendo una isometría Riemanniana.
    \end{proof}
\end{frame}

\begin{frame}{El Grupo $\PSL{2,\bbm{R}}$}
    \begin{theor}[\textbf{El grupo de isometrías hiperbólicas}]
        El grupo $\Isom{H,d_H}$ es generado por:
        \begin{equation*}
            \left\{f_A\Big|A\in\SL{2,\bbm{R}} \right\}\cup\left\{z\mapsto-\overline{z} \right\}
        \end{equation*}
        En particular, toda isometría de $(H,d_H)$ es una isometría Riemanniana suave y, $\Isom{H,d_H}=\Isom{\bbm{H}^2}$. Además, la función:
        \begin{equation*}
            \begin{split}
                \PSL{2,\bbm{R}}&\rightarrow\Isom{H,d_H}^+\\
                A&\mapsto f_A\\
            \end{split}
        \end{equation*}
        es un isomorfismo, siendo $\Isom{H,d_H}^+$ al grupo de todas las isometrías que preservan orientación de $\Isom{H,d_H}$.
    \end{theor}
\end{frame}

\begin{frame}{El Grupo $\PSL{2,\bbm{R}}$}
    \begin{theor}[\textbf{Caracterización de las geodésicas}]
        \label{caracterizacionGeodesicas}
        Sean $z,z'\in H$ distintos.
        \begin{enumerate}[label = \textit{(\arabic*)}]
            \item Existe una única geodésica en $(H,d_H)$ que une a $z$ con $z'$. En particular, el espacio métrico es geodésico.
            \item Hasta reparametrizaciones en $\bbm{R}$, existe una única linea geodésica en $(H,d_H)$ que contiene a $z$ y $z'$. 
        \end{enumerate}
        Más precisamente, si $A\in\SL{2,\bbm{R}}$ con $\Re(f_A(z))=0=\Re(f_A(z'))$, entonces la función $f_A\circ t\mapsto i\cdot e^{ t}$ es una línea geodésica que une a $z$ con $z'$ y la geodésica que va de $z$ a $z'$ genera esta línea.
    \end{theor}
\end{frame}

\begin{frame}{El Grupo $\PSL{2,\bbm{R}}$}
    \begin{obs}
        Usando la descripción anterior de las geodésicas nos permite obtener una fórmula explícita para la métrica $d_H$ en $H$:
        \begin{equation*}
            d_H(z,z')=\arcosh{1+\frac{\abs{z-z'}^2}{2\cdot\Im{z}\cdot\Im{z}}}
        \end{equation*}
        siendo $\cf{\textup{arcosh}}{\bbm{R}_{\geq1}}{\bbm{R}}$ la función:
        \begin{equation*}
            x\mapsto\ln\left(x+\sqrt{x^2-1} \right)
        \end{equation*}
    \end{obs}
\end{frame}

\subsection{Acción de $\SL{2,\bbm{R}}$ en $\bbm{H}^2$}

\begin{frame}
    \begin{center}
        Acción de $\SL{2,\bbm{R}}$ en $\bbm{H}^2$
    \end{center}
\end{frame}

\begin{frame}{Acción de $\SL{2,\bbm{R}}$ en $\bbm{H}^2$}
    \begin{propo}[\textbf{Acción de $\SL{2,\bbm{R}}$ en $H$}]
        \label{accionSL2RenH}
        Se tiene lo siguiente:
        \begin{enumerate}[label = \textit{(\arabic*)}]
            \item El grupo $\SL{2,\bbm{R}}$ actúa en $H$ vía transformaciones de Möbius, más aún, esta acción es transitiva.
            \item El grupo estabilizador de $i$ respecto a esta acción es $\SO{2}$.
            \item Para todo $z,z'\in H$ existe $A\in\SL{2,\bbm{R}}$ tal que:
            \begin{equation*}
                f_A(z)=i\quad\textup{y}\quad\Re(f_A(z'))=0, \Im(f_A(z'))>1
            \end{equation*}
        \end{enumerate}
    \end{propo}
\end{frame}

\begin{frame}{Acción de $\SL{2,\bbm{R}}$ en $\bbm{H}^2$}
    \textit{Demostración}:

    De \textit{(1)}: Es inmediato que el grupo actúa via transformaciones de Möbius con la acción dada por:
    \begin{equation*}
        (A,z)\mapsto A\cdot z = f_A(z),\quad\forall A\in \SL{2,\bbm{R}},\forall z\in H
    \end{equation*}
    Veamos que esta acción es transitiva. Basta probar que para todo $z\in H$ existe un $A_z\in\SL{2,\bbm{R}}$ tal que:
    \begin{equation*}
        f_{A_z}(z)=i
    \end{equation*}
    Tomemos $x=\Re(z)$ y $y=\Im(z)$. Entonces la transformación de Möbius asociada a la matriz:
    \begin{equation*}
        A_z=\left(
            \begin{array}{cc}
                \frac{1}{\sqrt{y}} & -\frac{x}{\sqrt{y}} \\
                0 & \sqrt{y} \\
            \end{array}
        \right)\in\SL{2,\bbm{R}}
    \end{equation*}
    es tal que:
    \begin{equation*}
        A_z\cdot z=f_{ A_z}(z)=\frac{\frac{x+iy}{\sqrt{y}}-\frac{x}{\sqrt{y}}}{\sqrt{y}}=i
    \end{equation*}
\end{frame}

\begin{frame}{Acción de $\SL{2,\bbm{R}}$ en $\bbm{H}^2$}
    Con lo que la acción es transitiva.
    
    De \textit{(2)}: Se tiene que:
    \begin{equation*}
        \begin{split}
            \SL{2,\bbm{R}}_i&=\left\{A\in\SL{2,\bbm{R}}\Big|A\cdot i=i \right\}\\
            &=\left\{\left(\begin{array}{cc}
                a & b \\
                c & d \\
            \end{array} \right) \in\SL{2,\bbm{R}}\Big|a = d\textup{ y }c=-b \right\}\\
            &=\left\{\left(\begin{array}{cc}
                a & -c \\
                c & a \\
            \end{array} \right) \in\mathcal{M}_{2\times2}(\bbm{R}) \Big|a^2+c^2=1 \right\}\\
            &=\SO{2}\\
        \end{split}
    \end{equation*}

    De \textit{(3)}: Inmediato del inciso \textit{(1)}.
\end{frame}

\begin{frame}{Acción de $\SL{2,\bbm{R}}$ en $\bbm{H}^2$}
    Resulta que podemos dotar al grupo $\PSL{2,\bbm{R}}$ con una topología. Para ello, notemos que la función:
    \begin{equation*}
        f_A\mapsto (a,b,c,d)
    \end{equation*}
    es una función suprayectiva de $\PSL{2,\bbm{R}}$ en el subconjunto:
    \begin{equation*}
        \left\{(a,b,c,d)\in\bbm{R}^4\Big|ad-bc=1 \right\}
    \end{equation*}
    y, es una función biyectiva al espacio cociente:
    \begin{equation*}
        \left\{(a,b,c,d)\in\bbm{R}^4\Big|ad-bc=1 \right\}/\left\{(a,b,c,d)\sim(-a,-b,-c,-d) \right\}
    \end{equation*}
\end{frame}

\begin{frame}{Acción de $\SL{2,\bbm{R}}$ en $\bbm{H}^2$}
    Dotando al subespacio $\left\{(a,b,c,d)\in\bbm{R}^4\Big|ad-bc=1 \right\}$ con la norma usual de $\bbm{R}^4$ resulta que el cociente también se puede dotar de una norma, así que el grupo $\PSL{2,\bbm{R}}$ tiene una norma inducida por la norma del espacio cociente, a saber:
    \begin{equation*}
        \norm{f_A}=\sqrt{a^2+b^2+c^2+d^2}
    \end{equation*}
    donde
    \begin{equation*}
        A=\left(\begin{array}{cc}
            a & b \\
            c & d \\
        \end{array}\right)
    \end{equation*}
\end{frame}

\begin{frame}{Acción de $\SL{2,\bbm{R}}$ en $\bbm{H}^2$}
    \begin{propo}
        $\PSL{2,\bbm{R}}$ es un grupo topológico con la métrica inducida por la norma:
        \begin{equation*}
            \norm{f_A}=\sqrt{a^2+b^2+c^2+d^2}
        \end{equation*}
    \end{propo}

    \begin{proof}
        %TODO
    \end{proof}
\end{frame}

\subsection{Švarc-Milnor y el plano hiperbólico}

\begin{frame}
    \begin{center}
        Švarc-Milnor y el plano hiperbólico
    \end{center}
\end{frame}

\begin{frame}{Švarc-Milnor y el plano hiperbólico}
    \begin{lema}[\textbf{Švarc-Milnor}]
        Sea $G$ un grupo actuando en un espacio métrico no vacío $(X,d)$ por isometrías. Suponga que existen constantes $c,b\in\bbm{R}_{>0}$ tales que $(X,d)$ es $(c,d)$-cuasi-geodésico y además que existe un conjunto $B\subseteq X$ con las siguientes propiedades:
        \begin{itemize}
            \item El diámetro de $B$ es finito.
            \item Las traslaciones de $G$ cubren a todo $X$, esto es: $\bigcup_{g\in G}g\cdot B=X$.
            \item El conjunto $S=\left\{g\in G\Big|g\cdot B'\cap B'\neq\emptyset \right\}$ es finito, donde:
            \begin{equation*}
                B'=B_{2\cdot b}^{(X,d)}(B)=\left\{x\in X\Big|\exists y\in B\textup{ tal que }d(x,y)\leq 2b \right\}
            \end{equation*}
        \end{itemize}
    \end{lema}
\end{frame}

\begin{frame}{Švarc-Milnor y el plano hiperbólico}
    \begin{lema}[\textbf{Švarc-Milnor}]
        Entonces:
        \begin{enumerate}[label = \textit{(\arabic*)}]
            \item El grupo $G$ es generado por $S$; en particular, $G$ es finitamente generado.
            \item Para todo $x\in X$, la función:
            \begin{equation*}
                \begin{split}
                    G&\rightarrow X\\
                    g&\mapsto g\cdot x\\
                \end{split}
            \end{equation*}
            es una cuasi-isometría (con respecto a la métrica de palabras en $G$).
        \end{enumerate}
    \end{lema}
\end{frame}

\begin{frame}{Švarc-Milnor y el plano hiperbólico}
    \begin{obs}
        Notemos que para la acción de $\SL{2,\bbm{R}}$ en el espacio métrico $(H,d_H)$ estamos en la posibilidad de aplicar el lema anterior, solo basta verificar algunas condiciones. Las que ya se tienen son las siguientes:
        \begin{enumerate}[label = \textit{(\arabic*)}]
            \item El espacio $(H,d_H)$ es no vacío $(1,0)$-cuasi-geodésico, por ser geodésico.
            \item $\SL{2,\bbm{R}}$ actúa por isometrías en $(H,d_H)$.
        \end{enumerate}
        Para aplicar el lema, debemos ver que las tres condiciones del lema se cumplen para un conjunto $B\subseteq H$.
    \end{obs}
\end{frame}

\begin{frame}{Švarc-Milnor y el plano hiperbólico}
    Sea $B=\left\{z\right\}\subseteq H$. Se tiene que:
    \begin{enumerate}[label = \textit{(\arabic*)}]
        \item El diámetro de $B$ es cero, es decir que es finito.
        \item $\bigcup_{A\in\SL{2,\bbm{R}}}A\cdot B=\bigcup_{A\in\SL{2,\bbm{R}}}A\cdot\left\{z\right\}=H$, pues la acción de $\SL{2,\bbm{R}}$ en $H$ es transitiva.
        \item Como el espacio es $(1,0)$-cuasi-geodésico, solo hay que ver si el conjunto:
        \begin{equation*}
            \left\{A\cdot B\cap B\neq\emptyset \Big|A\in\SL{2,\bbm{R}} \right\}=\left\{A\cdot z=z\Big|A\in\SL{2,\bbm{R}} \right\}
        \end{equation*}
        es finito o no. Esto ya que en este caso se tiene:
        \begin{equation*}
            B'=B_{2\cdot 0}^{(X,d)}(B)=\left\{x\in X\Big|d(x,z)=0\right\}=B
        \end{equation*}
    \end{enumerate}
\end{frame}

\begin{frame}{Švarc-Milnor y el plano hiperbólico}
    Resulta que tal conjunto no es finito, pues si $A_z\in\SL{2,\bbm{R}}$ es tal que:
    \begin{equation*}
        f_{A_z}(z)=i
    \end{equation*}
    se tiene que:
    \begin{equation*}
        A_z^{-1}\cdot\SO{2}\cdot A_z\subseteq\left\{A\cdot B\cap B\neq\emptyset \Big|A\in\SL{2,\bbm{R}} \right\}
    \end{equation*}
    pues:
    \begin{equation*}
        f_{A_z^{-1}\cdot O\cdot A_z}(z)=f_{A_z^{-1}}\circ f_{O}\circ f_{A_z}(z)=f_{A_z^{-1}}\circ f_{O}(i)=f_{A_z^{-1}}(i)=z
    \end{equation*}
    con lo que el conjunto de la derecha no es finito, pues $A_z^{-1}\cdot\SO{2}\cdot A_z$ no lo es.
\end{frame}

\begin{frame}{Švarc-Milnor y el plano hiperbólico}
    \begin{center}
        \textit{¿Cómo solucionamos este problema?}

        Usando subgrupos de $\SL{2,\bbm{R}}$ o equivalentemente, de $\PSL{2,\bbm{R}}$.
    \end{center}
\end{frame}

%TODO: Seguir con Grupos Fuchsianos

\subsection{Grupos Fuchsianos}

\begin{frame}
    \begin{center}
        Grupos Fuchsianos
    \end{center}
\end{frame}

\begin{frame}{Grupos Fuchsianos}
    \begin{mydef}
        Un subgrupo $H<\Isom{2,\bbm{R}}^+$ es llamado \textbf{discreto} si el grupo $H$ visto como subgrupo de $\PSL{2,\bbm{R}}$ es discreto.
        
        En tal caso, $H$ es llamado \textbf{grupo Fuchsiano}.
    \end{mydef}

    En otras palabras, un grupo Fuchsiano es un subgrupo discreto de $\PSL{2,\bbm{R}}$.
\end{frame}

\begin{frame}{Grupos Fuchsianos}
    \begin{exa}
        El \textbf{grupo modular} $\PSL{2,\bbm{Z}}$ es un subgrupo discreto de $\PSL{2,\bbm{R}}$, por lo que es Fuchsiano.
    \end{exa}

    \begin{exa}
        El grupo $\PSL{2,\bbm{Q}}$ es un subgrupo de $\PSL{2,\bbm{R}}$ que no es discreto, luego no es Fuchsiano.
    \end{exa}
\end{frame}

\begin{frame}{Grupos Fuchsianos}
    \begin{exa}
        El conjunto de todas las traslaciones reales enteras $\left\{T_n\Big|n\in\bbm{Z} \right\}$ es un grupo Fuchsiano.
    \end{exa}

    \begin{exa}
        El conjunto de todas las traslaciones reales $\left\{T_r\Big|r\in\bbm{R} \right\}$ no es un grupo Fuchsiano.
    \end{exa}
\end{frame}

\begin{frame}{Grupos Fuchsianos}
    \begin{mydef}[\textbf{Clasificación de los elementos de $\PSL{2,\bbm{R}}$}]
        Sea $[A]\in\PSL{2,\bbm{R}}$.
        \begin{itemize}
            \item Si $\abs{\Tr{A}}<2$ decimos que $A$ es una \textbf{transformacioń elíptica}.
            \item Si $\abs{\Tr{A}}=2$ decimos que $A$ es una \textbf{transformacioń parabólica}.
            \item Si $\abs{\Tr{A}}>2$ decimos que $A$ es una \textbf{transformacioń hiperbólica}.
        \end{itemize}
    \end{mydef}
\end{frame}

\begin{frame}{Grupos Fuchsianos}
    
\end{frame}

\begin{frame}{Grupos Fuchsianos}
    \begin{mydef}[\textbf{Familias localmente finitas}]
        Una familia $\left\{F_i\subseteq X\Big|i\in I \right\}$ de subconjuntos un espacio métrico $(X,d)$ es \textbf{localmente finita} si el conjunto:
        \begin{equation*}
            \left\{i\in I \Big|F_i\cap C\neq\emptyset \right\}
        \end{equation*}
        es un conjunto finito para todo $C\subseteq X$ compacto.
    \end{mydef}

    \begin{mydef}[\textbf{Acciones propiamente discontinuas}]
        Decimos que un grupo $G$ actuando en un espacio métrico $(X,d)$ \textbf{actúa propiamente de forma discontinua} si la familia $\left\{G\cdot x\Big|x\in X \right\}$ es localmente finita.
    \end{mydef}
\end{frame}

\begin{frame}{Grupos Fuchsianos}
    \begin{theor}[\textbf{Caracterización de las acciones propiamente discontinuas}]
        Un grupo $G$ actúa propiamente de forma discontinua sobre un espacio métrico $(X,d)$ si y sólo si para todo $x\in X$ existe $\epsilon>0$ tal que:
        \begin{equation*}
            \left\{g\in G\Big|g\cdot B_\varepsilon^{(X,d)}(x)\cap B_\varepsilon^{(X,d)}(x)\neq\emptyset \right\}
        \end{equation*}
        es un conjunto finito.
    \end{theor}
\end{frame}

\begin{frame}{Grupos Fuchsianos}
    \begin{theor}[\textbf{Caracterización de los grupos Fuchsianos}]
        Sea $\Gamma<\PSL{2,\bbm{R}}$. Entonces, $\Gamma$ es Fuchsiano si y sólo si actúa propiamente de forma discontinua en $(H,d_H)$.
    \end{theor}
\end{frame}

\begin{frame}{Grupos Fuchsianos}
    
\end{frame}

\begin{frame}{Grupos Fuchsianos}
    Sea $\Gamma$ un grupo Fuchsiano. Este grupo actúa por isometrías en $(H,d_H)$. 
\end{frame}

\begin{frame}{Grupos Fuchsianos}
    
\end{frame}

\begin{frame}{Grupos Fuchsianos}
    
\end{frame}

\begin{frame}{Grupos Fuchsianos}
    
\end{frame}

\begin{frame}{Grupos Fuchsianos}
    
\end{frame}

\begin{frame}{Grupos Fuchsianos}
    
\end{frame}

\subsection{Superficies de Género $g\geq0$}

\begin{frame}{Superficies de Género $g\geq0$}
    Resulta que existe una relación profunda entre los subgrupos de isometrías del plano hiperbólico y el grupo fundamental de superficies de género $g$.

    \begin{theor}
        Sea $X$ un espacio conexo, localmente arco-conexo y semilocalmente simplemente conexo. Entonces $X$ tiene admite una cubierta universal.
    \end{theor}
\end{frame}

\begin{frame}{Superficies de Género $g\geq0$}
    \begin{mydef}
        Una \textbf{superficie de Riemann} es un espacio topológico conexo Hausdorff $M$ junto con una colección de cartas $\left\{(U_\alpha,\phi_\alpha) \right\}_{\alpha\in I}$ tales que:
        \begin{itemize}
            \item $\left\{U_\alpha \right\}_{\alpha\in I}$ es una cubierta abierta de $M$.
            \item Para todo $\alpha\in I$, $\cf{\phi_{\alpha}}{U_\alpha}{V_\alpha\subseteq\bbm{C}}$ es un homeomorfismo, donde $V_\alpha$ es un abierto de $\bbm{C}$.
            \item Si $U_\alpha\cap U_\beta$ para algunos $\alpha,\beta\in I$, entonces la función $\cf{\phi_{\alpha\beta}=\phi_\beta\circ\phi_\alpha^{-1}}{\phi_\alpha(U_\alpha\cap U_\beta)}{\phi_\beta(U_\alpha\cap U_\beta)}$ es una homeomorfismo analítico complejo.
        \end{itemize}
    \end{mydef}
\end{frame}

\begin{frame}{Superficies de Género $g\geq0$}
    \begin{exa}
        $\bbm{C}$ es una superficie de Riemann con carta $\left\{(\bbm{C},\bbm{1}_{\bbm{C}})\right\}$.
    \end{exa}

    \begin{exa}
        La esfera $\bbm{S}^2\cong\hat{\bbm{C}}=\bbm{C}\cup\left\{\infty \right\}$ es una superficie de Riemann (recuerde la proyección estereográfica).
    \end{exa}
\end{frame}

\begin{frame}{Superficies de Género $g\geq0$}
    \begin{exa}
        El plano hiperbólico $\bbm{H}^2$ es una superficie de Riemann. En efecto, basta con ver que el plano hiperbólico es un subconjunto de $\bbm{C}$, por lo que hereda toda la estructura de variedad de Riemann.
    \end{exa}

    \begin{exa}
        Toda superficie de género $g\geq0$ es una superficie de Riemann. 
    \end{exa}
\end{frame}

\begin{frame}{Superficies de Género $g\geq0$}
    \begin{theor}[\textbf{Teorema de uniformización de Riemann}]
        Toda superficie de Riemann simplemente conexa es conformemente equivalente a alguna de las tres:
        \begin{itemize}
            \item El plano complejo: $\bbm{C}$.
            \item La esfera de Riemann; $\hat{\bbm{C}}$.
            \item El plano hiperbólico: $\bbm{H}^2$.
        \end{itemize}
    \end{theor}
\end{frame}

\begin{frame}{Superficies de Género $g\geq0$}
    Con el teorema anterior resulta que podemos caracterizar los cubrientes universales de todas las superficies de género $g\geq0$:

    \begin{propo}
        Toda superficie de género $g\geq0$ tiene como cubriente universal a alguno de los siguientes:
        \begin{itemize}
            \item El plano complejo: $\bbm{C}$.
            \item La esfera de Riemann; $\hat{\bbm{C}}$.
            \item El plano hiperbólico: $\bbm{H}^2$.
        \end{itemize}
        En particular, si $g\geq2$ entonces $S_g$ tiene como cubriente universal a $\bbm{H}^2$.
    \end{propo}
\end{frame}

\begin{frame}{Superficies de Género $g\geq0$}
    
\end{frame}

%----------------------------------------------------------------------------------------
\section{Espacios Hiperbólicos}
%----------------------------------------------------------------------------------------

\begin{frame}{Espacios Hiperbólicos}
    \begin{center}
        \LARGE{Espacios Hiperbólicos}
    \end{center}
    %TODO: Poner imagen
\end{frame}

\subsection{Hiperbólicidad y $\delta$-hiperbolicidad}

\begin{frame}{Hiperbólicidad y $\delta$-hiperbolicidad}
    Hablaremos ahora de una propiedad importante definida sobre espacios métricos geodésicos y cuasi-geodésicos.

    \begin{mydef}
        Sea $(X,d)$ un espacio métrico. Para cada $\delta>0$ y para cada $A\subseteq X$ se define el conjunto:
        \begin{equation*}
            B_\delta^{(X,d)}(A)=\left\{x\in X\Big|\exists a\in A\textup{ tal que }d(x,a)\leq\delta \right\}
        \end{equation*}
    \end{mydef}
\end{frame}

\begin{frame}{Hiperbólicidad y $\delta$-hiperbolicidad}
    \begin{mydef}[\textbf{Triángulos geodésicos $\delta$-delgados}]
        Sea $(X,d)$ un espacio métrico. Un \textbf{triángulo geodésico en $X$} es una tripleta $(\gamma_0,\gamma_1,\gamma_2)$ de geodésicas $\cf{\gamma_i}{[0,L_i]}{X}$ en $X$ tales que:
        \begin{equation*}
            \gamma_0(L_0)=\gamma_1(0),\quad \gamma_1(L_1)=\gamma_2(0),\quad \gamma_2(L_2)=\gamma_0(0)
        \end{equation*}
    \end{mydef}

    \begin{mydef}[\textbf{Triángulos geodésicos $\delta$-delgados}]
        Un triángulo geodésico es \textbf{$\delta$-delgado} si:
        \begin{equation*}
            \begin{array}{cc}
                \im{\gamma_0}&\subseteq B_{\delta}^{(X,d)}(\im{\gamma_1}\cup\im{\gamma_2}),\\
                \im{\gamma_1}&\subseteq B_{\delta}^{(X,d)}(\im{\gamma_0}\cup\im{\gamma_2}),\\
                \im{\gamma_2}&\subseteq B_{\delta}^{(X,d)}(\im{\gamma_0}\cup\im{\gamma_1})
            \end{array}
        \end{equation*}
    \end{mydef}
\end{frame}

\begin{frame}{Hiperbólicidad y $\delta$-hiperbolicidad}
    \begin{exa}
        %TODO: COlocar triángulo geodésico $\delta$ delgado
    \end{exa}
\end{frame}

\begin{frame}{Hiperbólicidad y $\delta$-hiperbolicidad}
    \begin{mydef}[\textbf{Espacios hiperbólicos}]
        Sea $(X,d)$ un espacio métrico.
        \begin{enumerate}[label = \textit{(\arabic*)}]
            \item Sea $\delta\bbm{R}_{\geq0}$. Decimos que $(X,d)$ es \textbf{$\delta$-hiperbólico} si $X$ es geodésico y todos los triángulos geodésicos de $X$ son $\delta$-delgados.
            \item $(X,d)$ es \textbf{hiperbólico} si existe $\delta\in\bbm{R}_{\geq0}$ tal que $(X,d)$ es $\delta$-hiperbólico.
        \end{enumerate}
    \end{mydef}

    \begin{exa}
        Todo espacio métrico geodésico $X$ de diámetro finito es $\Diam{X}$-hiperbólico. 
    \end{exa}
\end{frame}

\begin{frame}{Hiperbólicidad y $\delta$-hiperbolicidad}
    \begin{exa}
        La recta real $\bbm{R}$ es $0$-hiperbólico ya que cada triángulo geodésico en $\bbm{R}$ es degenerado, pues estos se ven simplemente como líneas rectas.
    \end{exa}

    \begin{exa}
        El plano euclideano $\bbm{R}^2$ no es hiperbólico.
    \end{exa}
\end{frame}

\begin{frame}{Hiperbólicidad y $\delta$-hiperbolicidad}
    %TODO: Hacer triángulo par que no sea hiperbólico
\end{frame}

\subsection{Hiperbolicidad del Plano $\bbm{H}^2$}

\begin{frame}
    \begin{center}
        ¿Coincide esta nueva noción de hiperbolicidad de espacios métricos con la definición sobre superficies?
    \end{center}
\end{frame}

\begin{frame}
    \begin{center}
        Hiperbolicidad del Plano $\bbm{H}^2$
    \end{center}
\end{frame}

\begin{frame}{Hiperbolicidad del Plano $\bbm{H}^2$}
    Nuestro objetivo en esta subsección será probar el siguiente resultado:

    \begin{propo}
        El plano hiperbólico $\bbm{H}^2$ (visto como espacio métrico) es un espacio métrico hiperbólico en el sentido de la definición de la sección anterior.
    \end{propo}

    Antes de llegar a ello, probaremos algunos resultados adicionales y enunciaremos algunas definciones fundamentales.
\end{frame}

\begin{frame}{Hiperbolicidad del Plano $\bbm{H}^2$}
    \begin{mydef}[\textbf{Área hiperbólica}]
        Sea $\cf{f}{H}{\bbm{R}_\geq0}$ una función Lebesgue integrable. Se define la \textbf{integral de $f$ sobre $\bbm{H}^2$} como:
        \begin{equation*}
            \begin{split}
                \int_{H}f\:dV_H&=\int_{H}f(x,y)\sqrt{\det(G_{H,(x,y)})}\:dxdy\\
                &=\int_{H}\frac{f(x,y)}{y^2}\:dxdy\\
            \end{split}
        \end{equation*}
        donde:
        \begin{equation*}
            G_{ H,,(x,y)}=\left(\begin{array}{cc}
                g_{ H,(x,y)}(e_1,e_1) & g_{ H,(x,y)}(e_1,e_2) \\
                g_{ H,(x,y)}(e_2,e_1) & g_{ H,(x,y)}(e_2,e_2) \\
            \end{array} \right)=\left(\begin{array}{cc}
                1/y^2 & 0 \\
                0 & 1/y^2 \\
            \end{array} \right)
        \end{equation*}
        siendo $e_1,e_2\in T_{(x,y)}H=\bbm{R}^2$ los vectores canónicos.
    \end{mydef}
\end{frame}

\begin{frame}{Hiperbolicidad del Plano $\bbm{H}^2$}
    \begin{mydef}[\textbf{Área hiperbólica}]
        Si $A\subseteq H$ es un conjunto Lebesgue medible, definimos el \textbf{área hiperbólica de $A$} por:
        \begin{equation*}
            \mu_{\bbm{H}^2}(A)=\int_{H}\chi_A\:dV_H
        \end{equation*}
        siendo $\chi_A$ la función característica de $A$.
    \end{mydef}
\end{frame}

\begin{frame}{Hiperbolicidad del Plano $\bbm{H}^2$}
    \begin{propo}[\textbf{Las isometrías preservan el área}]
        Sea $A\subseteq H$ un conjunto Lebesgue medible y tomemos $f\in\Isom{H,d_H}$. Entonces, $f(A)$ es medible y:
        \begin{equation*}
            \mu_{\bbm{H}^2}(A)=\mu_{\bbm{H}^2}(f(A))
        \end{equation*}
    \end{propo}
\end{frame}

\begin{frame}{Hiperbolicidad del Plano $\bbm{H}^2$}
    \begin{propo}[\textbf{Crecimiento exponencial del área hiperbólica}]
        Para todo $r\in\bbm{R}_{>10}$ tenemos que:
        \begin{equation*}
            \mu_{\bbm{H}^2}(B_r^{(H,d_H)}(i))\geq e^{\frac{r}{10}}(1-e^{-\frac{r}{2}})
        \end{equation*}
    \end{propo}
    \begin{proof}
        Sea $r\in\bbm{R}_{>10}$. Se tiene que el conjunto:
        \begin{equation*}
            Q_r=\left\{x+iy\Big|x\in[0,e^{ r/10}],y\in[1,e^{r/2}] \right\}
        \end{equation*}
        está contenido en $B_r^{(H,d_H)}(i)$. En particular, obtenemos que:
        \begin{equation*}
            \begin{split}
                \mu_{\bbm{H}^2}(B_r^{(H,d_H)}(i))&\geq\mu_{\bbm{H}^2}(Q_r)\\
                &=\int_{0}^{e^{r/10}}\int_{1}^{e^{r/2}}\frac{dxdy}{y^2}\\
                =&e^{\frac{r}{10}}(1-e^{-\frac{r}{2}})\\
            \end{split}
        \end{equation*}
    \end{proof}
\end{frame}

\begin{frame}{Hiperbolicidad del Plano $\bbm{H}^2$}
    \begin{mydef}[\textbf{Área de un triángulo geodésico}]
        Sea $\Delta$ un triángulo geodésico en $(H,d_H)$. Se define el \textbf{área de $\Delta$} como:
        \begin{equation*}
            \mu_{\bbm{H}^2}(\Delta)=\mu_{\bbm{H}^2}(A_\Delta)
        \end{equation*}
        siendo $A_\Delta\subseteq H$ el conjunto compacto encerrado por las geodésicas de $\Delta$.
    \end{mydef}
\end{frame}

\begin{frame}{Hiperbolicidad del Plano $\bbm{H}^2$}
    \begin{theor}[\textbf{Teorema de Gauß-Bonnet para triángulos hiperbólicos}]
        Sea $\Delta$ un triángulo geodésico en $(H,d_H)$ con ángulos $\alpha,\beta,\gamma$ y suponga que la imagen de $\Delta$ no está contenida en una sola línea geodésica. Entonces:
        \begin{equation*}
            \mu_{\bbm{H}^2}(\Delta)=\pi-(\alpha+\beta+\gamma)
        \end{equation*}
        En particular, la suma de los ángulos de un triángulo geodésico es menor que $\pi$ y el área hiperbólica está acotada por $\pi$.
    \end{theor}
\end{frame}

\begin{frame}{Hiperbolicidad del Plano $\bbm{H}^2$}
    \begin{theor}[\textbf{Triángulos son delgados}]
        Existe una constante $C\in\bbm{R}_{\geq0}$ tal que todo triángulo geodésico en $(H,d_H)$ es $C$-delgado.
    \end{theor}

    \textit{Demostración:}

    Por la proposición anterior, existe $C>0$ tal que:
    \begin{equation*}
        \mu_{\bbm{H}^2}(B_C^{(H,d_H)}(i))\geq 4\cdot\pi
    \end{equation*}
    (por ejemplo $C=26$). Tomemos $\Delta=(\gamma_0,\gamma_1,\gamma_2)$ un triángulo geodésico en $(H,d_H)$ y sea $x\in\im{\gamma_0}$.

    Sin pérdida de generalidad, podemos suponer que el triángulo geodésico $\Delta$ no está contenido en una sola línea geodésica. Por el inciso \textit{(3)} de la Proposición (\ref{accionSL2RenH}) se sigue que podemos trasladar los puntos $x$ a $i$ y el final de la geodésica a un punto tal que:
    \begin{equation*}
        f_A(z)=ci, \quad c>1
    \end{equation*}

\end{frame}

\begin{frame}{Hiperbolicidad del Plano $\bbm{H}^2$}
    Luego, del Teorema (\ref{caracterizacionGeodesicas}) y la Proposición () se sigue que la geodésica $\gamma_0$ es un segmento vertical que yace sobre el eje $y$.

    Supongamos que no existe $y\in\im{\gamma_1}\cup\im{\gamma_2}$ tal que $d_H(x,y)\leq C$. Se tiene entonces que:
    \begin{equation*}
        B_c^{(H,d_H)}(i)\subseteq A_\Delta\cup\im{\gamma_0}\cup f(A_\Delta)
    \end{equation*}
    siendo $A_\Delta$ el conjunto encerrado por las geodésicas de $\Delta$ y $\cf{f}{H}{H}$ la isometría $z\mapsto-\overline{z}$.

    %TODO Incluir imagen de qué está pasando.

    Por tanto:
    \begin{equation*}
        \begin{split}
            4\cdot\pi&\leq\mu_{\bbm{H}^2}(B_C^{(H,d_H)}(i))\\
            &\leq\mu_{\bbm{H}^2}(A_\Delta\cup\im{\gamma_0}\cup f(A_\Delta))\\
            &=\mu_{\bbm{H}^2}(A_\Delta)+\mu_{\bbm{H}^2}(\im{\gamma_0})+\mu_{\bbm{H}^2}(f(A_\Delta))\\
            &=\mu_{\bbm{H}^2}(\Delta)+\mu_{\bbm{H}^2}(D)\\
            &< 2\cdot\pi\\
        \end{split}
    \end{equation*}
\end{frame}

\begin{frame}{Hiperbolicidad del Plano $\bbm{H}^2$}
    Lo cual es una contradicción. Por lo cual existe $y\in\im{\gamma_1}\cup\im{\gamma_2}$ tal que $d(x,y)\leq C$. En particular se sigue que:
    \begin{equation*}
        \im{y_0}\subseteq \bigcup_{y\in\im{\gamma_1}\cup\im{\gamma_2}}B_C^{(H,d_H)}(y)\subseteq B_C^{(H,d_H)}( \im{\gamma_1}\cup\im{\gamma_2} )
    \end{equation*}
    el procedimiento anterior se puede repetir para las otras geodésicas, resultando en que:
    \begin{equation*}
        \begin{split}
            \im{\gamma_0}&\subseteq B_{C}^{(H,d_H)}(\im{\gamma_1}\cup\im{\gamma_2}),\\
            \im{\gamma_1}&\subseteq B_{C}^{(H,d_H)}(\im{\gamma_0}\cup\im{\gamma_2}),\\
            \im{\gamma_2}&\subseteq B_{C}^{(H,d_H)}(\im{\gamma_0}\cup\im{\gamma_1})\\
        \end{split}
    \end{equation*}
    así que $C$ es un triángulo geodésico $C$-delgado. Como el $\Delta$ triángulo geodésico fue arbitrario se sigue que el plano hiperbólico es $C$-hiperbólico, es decir que es hiperbólico en el sentido de espacio métrico.
\end{frame}

\begin{frame}{Hiperbolicidad del Plano $\bbm{H}^2$}
    Un resultado más general dice que... %TODO cosas sobre variedades de curvatura negativa.
\end{frame}

\begin{frame}
    \begin{center}
        \textit{Y, ¿para qué nos sirve la hiperbolicidad?}
    \end{center}
\end{frame}

\subsection{Hiperbolicidad es invariante cuasi-isométrico}

\begin{frame}
    \begin{center}
        La hiperbolicidad es un invariante cuasi-isométrico
    \end{center}
\end{frame}

\begin{frame}{Hiperbolicidad es invariante cuasi-isométrico}
    Para llegar a probar tal cosa, debemos debilitar la definición de hiperbolicidad:
    
    \begin{mydef}[\textbf{Triángulos cuasi-geodésicos $\delta$-delgados}]
        Sea $(X,d)$ un espacio métrico.
        \begin{enumerate}[label = \textit{\arabic*}]
            \item Un \textbf{triángulo cuasi-geodésico en $X$} es una tripleta $(\gamma_0,\gamma_1,\gamma_2)$ de $(c,b)-$cuasi-geodésicas $\cf{\gamma_i}{[0,L_i]}{X}$ en $X$ tales que:
            \begin{equation*}
                \gamma_0(L_0)=\gamma_1(0),\quad \gamma_1(L_1)=\gamma_2(0),\quad \gamma_2(L_2)=\gamma_0(0)
            \end{equation*}
            \item Un triángulo $(c,b)$-cuasi-geodésico es \textbf{$\delta$-delgado} si:
            \begin{equation*}
                \begin{split}
                    \im{\gamma_0}&\subseteq B_{\delta}^{(X,d)}(\im{\gamma_1}\cup\im{\gamma_2}),\\
                    \im{\gamma_1}&\subseteq B_{\delta}^{(X,d)}(\im{\gamma_0}\cup\im{\gamma_2}),\\
                    \im{\gamma_2}&\subseteq B_{\delta}^{(X,d)}(\im{\gamma_0}\cup\im{\gamma_1})\\
                \end{split}
            \end{equation*}
        \end{enumerate}
    \end{mydef}
\end{frame}

\begin{frame}{Hiperbolicidad es invariante cuasi-isométrico}
    %TODO Poner ejemplo de un triángulo cuasi-geodésico

    \begin{obs}
        De esta definición es inmediato que todo triángulo geodésico es triángulo cuasi-geodésico.
    \end{obs}
\end{frame}

\begin{frame}{Hiperbolicidad es invariante cuasi-isométrico}
    \begin{mydef}[\textbf{Espacios cuasi-hiperbólicos}]
        Sea $(X,d)$ un espacio métrico.
        \begin{enumerate}[label = \textit{(\arabic*)}]
            \item Sean $c,b\in\bbm{R}_{>0}$, $\delta\in\bbm{R}_{\geq0}$. Decimos que el espacio $(X,d)$ es \textbf{$(c,b,\delta)$-cuasi-hiperbólico} si $(X,d)$ es $(c,b)$-cuasi-geodésico y todos los triángulos $(c,b)$-cuasi-geodésicos en $X$ son $\delta$-delgados.
            \item Sean $c,b\in\bbm{R}_{>0}$. El espacio $(X,d)$ es llamado \textbf{$(c,b)$-cuasi-hiperbólico} si para todo $c',b'\in\bbm{R}_{>0}$ con $c'\geq c$ y $b'\geq b$ existe $\delta\in\bbm{R}_{\geq0}$ tal que $(X,d)$ es $(c',b',\delta)$-cuasi-hiperbólico.
            \item El espacio $(X,d)$ es \textbf{cuasi-hiperbólico} si existen $c,b\in\bbm{R}_{>0}$ tales que $(X,d)$ es $(c,b)$-cuasi-hiperbólico.
        \end{enumerate}
    \end{mydef}
\end{frame}

\begin{frame}{Hiperbolicidad es invariante cuasi-isométrico}
    \begin{exa}
        Todos los espacios métricos de diámetro finito son cuasi-hiperbólicos.
    \end{exa}

    \begin{obs}
        En general resultará muy complicado probar que un espacio es cuasi-hiperbólico usando la definición anterior, por el hecho de que pueden existir demasiadas geodésicas. Resulta que este proceso se puede hacer más sencillo usando unos resultados que se verán más adelante.
    \end{obs}
\end{frame}

\begin{frame}{Hiperbolicidad es invariante cuasi-isométrico}
    \begin{propo}[\textbf{Invariancia de la cuasi-hiperbolicidad bajo cuasi-isometrías}]
        Sean $(X,d)$ y $(Y,\rho)$ espacios métricos.
        \begin{enumerate}[label = \textit{(\arabic*)}]
            \item Si $(Y,\rho)$ es cuasi-geodésico y, $(X,d)$ y $(Y,\rho)$ son cuasi-isométricos, entonces $(X,d)$ es cuasi-geodésico.
            \item Si $(Y,\rho)$ es cuasi-hiperbólico, $(X,d)$ es cuasi-geodésico y existe un encaje cuasi-isométrico de $(X,d)$ en $(Y,\rho)$, entonces $(X,d)$ es cuasi-hiperbólico.
            \item Si $(X,d)$ y $(Y,\rho)$ son cuasi-isométricos, entonces $X$ es cuasi-hiperbólico si y sólo si $Y$ es cuasi-hiperbólico.
        \end{enumerate}
    \end{propo}
\end{frame}

\begin{frame}{Hiperbolicidad es invariante cuasi-isométrico}
    Resulta que no existe mucha diferencia entre la propiedad de hiperbolicidad y cuasi-hiperbolicidad, como lo muestra el siguiente resultado:

    \begin{theor}[\textbf{Hiperbolicidad y cuasi-hiperbolicidad}]
        Sea $(X,d)$ un espacio métrico geodésico. Entonces $(X,d)$ es hiperbólico si y sólo si es cuasi-hiperbólico.
    \end{theor}

    Si $(X,d)$ es cuasi-hiperbólico, entonces es hiperbólico (ya que en particular toda geodésica es una cuasi-geodésica y por ende, todo triángulo geodésico es cuasi-geodésico).

\end{frame}

\begin{frame}{Hiperbolicidad es invariante cuasi-isométrico}
    La idea para probar la otra parte de la demostración de este teorema radica en ver como podemos aproximar cuasi-geodésicas con geodésicas y por ende, aproximar cuasi-triángulos geodésicos con triángulos geodésicos.

    %TODO: colocar aproximación de estos triángulos cuasi-geodésicos,

    \begin{cor}[\textbf{Invariancia cuasi-isométrica de la hiperbolicidad}]
        Sean $(X,d)$ y $(Y,\rho)$ espacios métricos.
        \begin{enumerate}[label = \textit{(\arabic*)}]
            \item Si $(Y,\rho)$ es hiperbólico, $(X,d)$ es cuasi-geodésico y existe un encaje cuasi-isométrico de $(X,d)$ en $(Y,\rho)$, entonces $X$ es cuasi-hiperbólico.
            \item Si $(Y,\rho)$ es geodésico y $(X,d)$ es cuasi-isométrico a $(Y,\rho)$, entonces $(X,d)$ es cuasi-hiperbólico si y sólo si $(Y,\rho)$ es hiperbólico.
            \item Si $(X,d)$ y $(Y,\rho)$ son geodésicos y cuasi-isométricos, entonces $(X,d)$ es hiperbólico si y sólo si $(Y,\rho)$ es hiperbólico. 
        \end{enumerate}
    \end{cor}
\end{frame}

\begin{frame}{Hiperbolicidad es invariante cuasi-isométrico}
    Como algunos ejemplos de la aplicación del teorema anterior tenemos los siguientes:

    \begin{cor}[\textbf{Hiperbolicidad de gráficas}]
        Sea $X$ una gráfica conexa. Entonces $X$ es cuasi-hiperbólica si y sólo si su realización geométrica $\abs{X}$ es hiperbólica.
    \end{cor}

    \begin{proof}
        
    \end{proof}

\end{frame}

\begin{frame}{Hiperbolicidad es invariante cuasi-isométrico}
    \begin{propo}[\textbf{Hiperbolicidad de árboles}]
        Si $T$ es un árbol, entonces su realización geométrica $\abs{T}$ es $0$-hiperbólica. En particular, $T$ es cuasi-hiperbólico.
    \end{propo}

    \begin{proof}
        
    \end{proof}
\end{frame}

\begin{frame}
    \begin{center}
        \textit{¿Y PARA QUÉ SIRVE LA HIPERBOLICIDAD?}
    \end{center}
\end{frame}

\subsection{Grupos Hiperbólicos}

\begin{frame}
    \begin{center}
        Grupos Hiperbólicos
    \end{center}
\end{frame}

\begin{frame}{Grupos Hiperbólicos}
    Debido a que la hiperbolicidad (y cuasi-hiperbolicidad) es un invariante cuasi-isométrico, resulta que podemos extender la noción de hiperbolicidad a grupos:

    \begin{mydef}[\textbf{Grupos hiperbólicos}]
        Un grupo finitamente generado $G$ es \textbf{hiperbólico} si para algún conjunto generador $S$ de $G$ se tiene que la gráfica de Caley $\Cay{G,S}$ es cuasi-hiperbólica.
    \end{mydef}
\end{frame}

\begin{frame}{Grupos Hiperbólicos}
    \begin{obs}
        Como la gráfica de Caley de un grupo $G$ es un invariante cuasi-isométrico, es decir que si $S,S'\subseteq G$ son conjuntos finitos que generan a $G$, se tiene que:
        \begin{equation*}
            \Cay{G,S}\qisom\Cay{G,S'}
        \end{equation*}
    \end{obs}
\end{frame}

\begin{frame}{Grupos Hiperbólicos}
    \begin{propo}[\textbf{Hiperbolicidad es un invariante cuasi-isométrico}]
        Sean $G$ y $H$ grupos finitamente generados.
        \begin{enumerate}[label = \textit{(\arabic*)}]
            \item Si $H$ es hiperbólico y existen conjuntos finitos generadores $S$ y $T$, de $G$ y $H$, respectivamente tal que existe un encaje cuasi-isométrico entre $(G,d_S)$ y $(H,d_T)$, entonces $G$ es hiperbólico.
            \item Si $G$ y $H$ son cuasi-isométricos, entonces $G$ es hiperbólico si y sólo si $H$ es hiperbólico.
        \end{enumerate}
    \end{propo}

    \begin{proof}
        %TODO
    \end{proof}
\end{frame}

\begin{frame}{Grupos Hiperbólicos}
    \begin{exa}
        Todos los grupos finitos son hiperbólicos ya que la realización geométrica de su gráfica de Caley es de diámetro finito.
    \end{exa}

    \begin{exa}
        $\bbm{Z}$ es hiperbólico por ser cuasi-isométrico a $\bbm{R}$, que es un espacio métrico hiperbólico.
    \end{exa}
\end{frame}

\begin{frame}{Grupos Hiperbólicos}
    \begin{exa}
        $\bbm{Z}^2$ no es hiperbólico, ya que es cuasi-isométrico al plano euclideano $\bbm{R}^2$, el cual no es hiperbólico.
    \end{exa}
\end{frame}

\begin{frame}{Grupos Hiperbólicos}
    \begin{center}
        \textit{¿Y PARA QUÉ GENERALIZAR ESTA NOCIÓN A GRUPOS?}
    \end{center}
\end{frame}

\subsection{El problema de la palabra en Grupos Hiperbólicos}

\begin{frame}
    \begin{center}
        El problema de la palabra en Grupos Hiperbólicos
    \end{center}
\end{frame}

\begin{frame}{El problema de la palabra en Grupos Hiperbólicos}
    \begin{mydef}
        Sea $\gen{S|R}$ una presentación finita de un grupo. Decimos que \textbf{el problema de la palabra es soluble para la presentación $\gen{S|R}$}, si existe una función total computable que recibe como entrada una palabra en $(S\cup S^{-1})^{*}$ que decida si esta representa o no un elemento trivial en el grupo $\gen{S|R}$.
    \end{mydef}
\end{frame}

\begin{frame}{El problema de la palabra en Grupos Hiperbólicos}
    Al decir que exista una función total computable, en términos más simples estamos diciendo que existe un algoritmo que para cada entrada que demos, termina en un tiempo finito.

    \begin{obs}
        Otra forma de enunciar la definición anterior es que los conjuntos:
        \begin{equation*}
            \begin{split}
                &\left\{w\in(S\cup S^{-1})^*\Big|w\textup{ representa un elemento trivial de }\gen{S|R} \right\}\\
                &\left\{w\in(S\cup S^{-1})^*\Big|w\textup{ no representa un elemento trivial de }\gen{S|R} \right\}\\
            \end{split}
        \end{equation*}
        son conjuntos computablemente enumerables.
    \end{obs}
\end{frame}

\begin{frame}{El problema de la palabra en Grupos Hiperbólicos}
    Al decir que son computablemente enumerables, intuitivamente estamos diciendo que existe un algoritmo que va arrojando todos los elementos de este conjunto.

    \begin{exa}
        La presentación $\gen{x,y|\emptyset}$ tiene problema de la palabra soluble, al igual que $\gen{x,y|xyx^{-1}y^{-1}}$.
    \end{exa}
\end{frame}

\begin{frame}{El problema de la palabra en Grupos Hiperbólicos}
    A primera vista uno podría imaginar que todo grupo finitamente presentado tiene problema de la palabra soluble, cosa que no es cierta, como muestra el siguiente resultado:

    \begin{theor}
        Existen grupos finitamente presentados tales que ninguna presentación finita de ellos tiene problema de la palabra soluble.
    \end{theor}
\end{frame}

\begin{frame}{El problema de la palabra en Grupos Hiperbólicos}
    \begin{exa}
        El grupo:
        %TODO: https://en.wikipedia.org/wiki/Word_problem_for_groups
        no tiene problema de la palabra soluble.
    \end{exa}

    Más cosas que podemos decir sobre los grupos hiperbólicos es lo siguiente:
\end{frame}

\begin{frame}{El problema de la palabra en Grupos Hiperbólicos}
    \begin{theor}[\textbf{Grupos genéricos son hiperbólicos}]
        En un sentido estadístico bien definido, casi todos los grupos con presentación finita representan grupos hiperbólicos.
    \end{theor}

    Por lo que resulta relevante preguntarnos sobre propiedades de los grupos hiperbólicos.
\end{frame}

\begin{frame}{El problema de la palabra en Grupos Hiperbólicos}
    \begin{mydef}[\textbf{Presentaciones de Dehn}]
        Una presentación finita $\gen{S|R}$ es una \textbf{presentación de Dehn} si existe $n\in\bbm{N}$ y palabras $u_1,...,u_n,v_1,...,v_n$ tales que:
        \begin{itemize}
            \item $R=\left\{u_1v_1^{-1},...,u_nv_n^{-1} \right\}$.
            \item Para todo $j=1,...,n$, la palabra $v_j$ es más corta que $u_j$.
            \item Para topa palabra $w\in(S\cup S^{-1})^*\setminus\left\{e\right\}$ que representa un elemento neutro del grupo $\gen{S|R}$ existe $j=1,...,n$ tal que $u_j$ es subpalabra de $w$.
        \end{itemize}
    \end{mydef}
\end{frame}

\begin{frame}{El problema de la palabra en Grupos Hiperbólicos}
    \begin{exa}
        La presentación:
        \begin{equation*}
            \gen{x,y|xx^{-1}e,yy^{-1}e,x^{-1}xe,y^{-1}ye}
        \end{equation*}
        es una presentacion de Dehn del grupo libre de rango 2.
    \end{exa}

    \begin{exa}
        La presentación:
        \begin{equation*}
            \gen{x,y|[x,y]}
        \end{equation*}
        no es una presentación de Dehn de $\bbm{Z}^2$.
    \end{exa}
\end{frame}

\begin{frame}{El problema de la palabra en Grupos Hiperbólicos}
    \begin{propo}[\textbf{Algoritmo de Dehn}]
        Si $\gen{S|R}$ es una presentación de Dehn, entonces el problema de la palabra es soluble para $\gen{S|R}$.
    \end{propo}
    \textit{Demostración}:

    Escribimos:
    \begin{equation*}
        R=\left\{u_1v_1^{-1},...,u_nv_n^{-1} \right\}
    \end{equation*}
    como en la definición de presentación de Dehn. Tomemos $w\in(S\cup S^{-1})^*$ una palabra.
    \begin{itemize}
        \item Si $w=e$, entonces $w$ representa un elemento trivial del grupo $\gen{S|R}$.
        \item Si $w\neq e$, tenemos dos casos:
        \begin{itemize}
            \item Si ninguna de las palabras $u_1,...,u_n$ es una subpalabra de $w$, entonces $w$ no representa un elemento trivial del grupo $\gen{S|R}$ (por la tercera parte de la definción de presentaciones de Dehn).
        \end{itemize}
    \end{itemize}
\end{frame}

\begin{frame}{El problema de la palabra en Grupos Hiperbólicos}
    \begin{proof}
        \begin{itemize}
            \item Si $w\neq e$, tenemos dos casos:
            \begin{itemize}
                \item Existe $j=1,...,n$ tal que $u_j$ es subpalabra de $w$, en cuyo caso se sigue que existen palabras $w',w''$ tales que: $w=w'u_jw''$. Ahora, como $u_jv_j^{-1}\in R$ se sigue que los elementos:
                \begin{equation*}
                    w'u_jw''\quad\textup{y}\quad w'v_jw''
                \end{equation*}
                representan el mismo elemento en el grupo $\gen{S|R}$. Así que la palabra $w$ es trivial si y sólo si la palabra $w'v_jw''$ (que es más corta) es trivial. Aplicando recursivamente el algoritmo se llega a determinar si $w$ es la palabra trivial o no.
            \end{itemize}
        \end{itemize}
        Este algoritmo siempre determina si la palabra $w$ es trivial o no, por lo que el problema de la palabra es resoluble en $\gen{S|R}$.
    \end{proof}
\end{frame}

\begin{frame}{El problema de la palabra en Grupos Hiperbólicos}
    \begin{theor}[\textbf{Presentaciones de Dehn en grupos hiperbólicos}]
        Sea $G$ un grupo hiperbólico y $S$ un conjunto generador de $G$. Entonces existe un conjunto finito $R\subseteq(S\cup S^{-1})^*$ tal que $\gen{S|R}$ es una presentación de Dehn y $G\cong\gen{S|R}$.
    \end{theor}
\end{frame}

\begin{frame}{El problema de la palabra en Grupos Hiperbólicos}
    \begin{cor}[\textbf{Grupos hiperbólicos tienen problema de la palabra soluble}]
        Sea $G$ grupo hiperbólico y $S\subseteq G$ un conjunto generador finito. Entonces existe una presentación finita $\gen{S|R}$ de $G$ tal que el problema de la palabra es soluble.
    \end{cor}

    \begin{cor}
        Todo grupo hiperbólico admite una presentación finita.
    \end{cor}
\end{frame}

\begin{frame}{Referencias}
    \footnotesize
    \bibliography{reference.bib}
    \bibliographystyle{apalike}
\end{frame}

\begin{frame}
    \Huge{\centerline{\textbf{Fin}}}
\end{frame}

\end{document}