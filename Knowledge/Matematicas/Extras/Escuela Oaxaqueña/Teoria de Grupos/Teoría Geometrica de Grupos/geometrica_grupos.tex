\documentclass[12pt]{report}
\usepackage[spanish]{babel}
\usepackage[utf8]{inputenc}
\usepackage{amsmath}
\usepackage{amssymb}
\usepackage{amsthm}
\usepackage{graphics}
\usepackage{subfigure}
\usepackage{lipsum}
\usepackage{array}
\usepackage{multicol}
\usepackage{enumerate}
\usepackage[framemethod=TikZ]{mdframed}
\usepackage[a4paper, margin = 1.5cm]{geometry}
\usepackage{tikz}
\usepackage{pgffor}
\usepackage{ifthen}
\usepackage{enumitem}
\usepackage{hyperref}
\usepackage{bbm}
\usepackage{listings}
\usetikzlibrary{graphs}

%Gestión de Hipervínculos



\hypersetup{
    colorlinks=true,
    linkcolor=black,
    filecolor=magenta,      
    urlcolor=cyan
}

%Gestión de Código de Programación

\definecolor{listing-background}{HTML}{F7F7F7}
\definecolor{listing-rule}{HTML}{B3B2B3}
\definecolor{listing-numbers}{HTML}{B3B2B3}
\definecolor{listing-text-color}{HTML}{000000}
\definecolor{listing-keyword}{HTML}{435489}
\definecolor{listing-keyword-2}{HTML}{1284CA} % additional keywords
\definecolor{listing-keyword-3}{HTML}{9137CB} % additional keywords
\definecolor{listing-identifier}{HTML}{435489}
\definecolor{listing-string}{HTML}{00999A}
\definecolor{listing-comment}{HTML}{8E8E8E}

\lstdefinestyle{myStyle}{
    language         = C++,
    alsolanguage     = scala,
    numbers          = left,
    xleftmargin      = 2.7em,
    framexleftmargin = 2.5em,
    backgroundcolor  = \color{gray!15},
    basicstyle       = \color{listing-text-color}\linespread{1.0}\ttfamily,
    breaklines       = true,
    frameshape       = {RYR}{Y}{Y}{RYR},
    rulecolor        = \color{black},
    tabsize          = 2,
    numberstyle      = \color{listing-numbers}\linespread{1.0}\small\ttfamily,
    aboveskip        = 1.0em,
    belowskip        = 0.1em,
    abovecaptionskip = 0em,
    belowcaptionskip = 1.0em,
    keywordstyle     = {\color{listing-keyword}\bfseries},
    keywordstyle     = {[2]\color{listing-keyword-2}\bfseries},
    keywordstyle     = {[3]\color{listing-keyword-3}\bfseries\itshape},
    sensitive        = true,
    identifierstyle  = \color{listing-identifier},
    commentstyle     = \color{listing-comment},
    stringstyle      = \color{listing-string},
    showstringspaces = false,
    label            = lst:bar,
    captionpos       = b,
}

\lstset{style = myStyle}

%Estilo del capítulo y sección

\makeatletter
\def\thickhrulefill{\leavevmode \leaders \hrule height 1ex \hfill \kern \z@}
\def\@makechapterhead#1{%
  {\parindent \z@ \raggedright
    \reset@font
    \hrule
    \vspace*{10\p@}%
    \par
    \center \LARGE \scshape \@chapapp{} \huge \thechapter
    \vspace*{10\p@}%
    \par\nobreak
    \vspace*{10\p@}%
    \par
    \vspace*{1\p@}%
    \hrule
    %\vskip 40\p@
    \vspace*{60\p@}
    \Huge #1\par\nobreak
    \vskip 50\p@
  }}

\def\section#1{%
  \par\bigskip\bigskip
  \hrule\par\nobreak\noindent
  \refstepcounter{section}%
  \addcontentsline{toc}{chapter}{#1}%
  \reset@font
  { \large \scshape
    \strut\S \thesection \quad
    #1}% 
    \hrule   
  \par
  \medskip
}

%Gestión marca de agua

\usetikzlibrary{shapes.multipart}

\newcounter{it}
\newcommand*\watermarktext[1]{\begin{tabular}{c}
    \setcounter{it}{1}%
    \whiledo{\theit<100}{%
    \foreach \col in {0,...,15}{#1\ \ } \\ \\ \\
    \stepcounter{it}%
    }
    \end{tabular}
    }

\AddToHook{shipout/foreground}{
    \begin{tikzpicture}[remember picture,overlay, every text node part/.style={align=center}]
        \node[rectangle,black,rotate=30,scale=2,opacity=0.04] at (current page.center) {\watermarktext{Cristo Daniel Alvarado ESFM\quad}};
  \end{tikzpicture}
}

%En esta parte se hacen redefiniciones de algunos comandos para que resulte agradable el verlos%

\def\proof{\paragraph{Demostración:\\}}
\def\endproof{\hfill$\blacksquare$}

\def\sol{\paragraph{Solución:\\}}
\def\endsol{\hfill$\square$}

%En esta parte se definen los comandos a usar dentro del documento para enlistar%

\newtheoremstyle{largebreak}
  {}% use the default space above
  {}% use the default space below
  {\normalfont}% body font
  {}% indent (0pt)
  {\bfseries}% header font
  {}% punctuation
  {\newline}% break after header
  {}% header spec

\theoremstyle{largebreak}

\newmdtheoremenv[
    leftmargin=0em,
    rightmargin=0em,
    innertopmargin=0pt,
    innerbottommargin=5pt,
    hidealllines = true,
    roundcorner = 5pt,
    backgroundcolor = gray!60!red!30
]{exa}{Ejemplo}[section]

\newmdtheoremenv[
    leftmargin=0em,
    rightmargin=0em,
    innertopmargin=0pt,
    innerbottommargin=5pt,
    hidealllines = true,
    roundcorner = 5pt,
    backgroundcolor = gray!50!blue!30
]{obs}{Observación}[section]

\newmdtheoremenv[
    leftmargin=0em,
    rightmargin=0em,
    innertopmargin=0pt,
    innerbottommargin=5pt,
    rightline = false,
    leftline = false
]{theor}{Teorema}[section]

\newmdtheoremenv[
    leftmargin=0em,
    rightmargin=0em,
    innertopmargin=0pt,
    innerbottommargin=5pt,
    rightline = false,
    leftline = false
]{propo}{Proposición}[section]

\newmdtheoremenv[
    leftmargin=0em,
    rightmargin=0em,
    innertopmargin=0pt,
    innerbottommargin=5pt,
    rightline = false,
    leftline = false
]{cor}{Corolario}[section]

\newmdtheoremenv[
    leftmargin=0em,
    rightmargin=0em,
    innertopmargin=0pt,
    innerbottommargin=5pt,
    rightline = false,
    leftline = false
]{lema}{Lema}[section]

\newmdtheoremenv[
    leftmargin=0em,
    rightmargin=0em,
    innertopmargin=0pt,
    innerbottommargin=5pt,
    roundcorner=5pt,
    backgroundcolor = gray!30,
    hidealllines = true
]{mydef}{Definición}[section]

\newmdtheoremenv[
    leftmargin=0em,
    rightmargin=0em,
    innertopmargin=0pt,
    innerbottommargin=5pt,
    roundcorner=5pt
]{excer}{Ejercicio}[section]

%En esta parte se colocan comandos que definen la forma en la que se van a escribir ciertas funciones%

\newcommand\abs[1]{\ensuremath{\left|#1\right|}}
\newcommand\divides{\ensuremath{\bigm|}}
\newcommand\cf[3]{\ensuremath{#1:#2\rightarrow#3}}
\newcommand\contradiction{\ensuremath{\#_c}}
\newcommand\natint[1]{\ensuremath{\left[\big|#1\big|\right]}}
\newcommand{\bbm}[1]{\ensuremath{\mathbbm{#1}}}
\newcounter{figcount}
\setcounter{figcount}{1}
\newcommand\Aut[1]{\ensuremath{\textup{Aut}\left(#1\right)}}
\newcommand{\Cay}[1]{\ensuremath{\textup{Cay}\left(#1\right)}}
\newcommand{\gen}[1]{\ensuremath{\langle#1\rangle}}

\begin{document}
    \setlength{\parskip}{5pt} % Añade 5 puntos de espacio entre párrafos
    \setlength{\parindent}{12pt} % Pone la sangría como me gusta
    \title{Notas Teoría Geométrica de Grupos
    
    10° Escuela Oaxaqueña de Matemáticas}
    \author{Cristo Daniel Alvarado}
    \maketitle

    \tableofcontents %Con este comando se genera el índice general del libro%

    \newpage

    \chapter{Teoría Geométrica de Grupos}

    \section{Introducción}

    En los capítulos anteriores hemos hablado sobre 

    \begin{mydef}
        Un grupo discreto $G$ es tal que está dotado de la topología discreta.
    \end{mydef}

    \section{Aplicaciones de espacios cubrientes}

    \begin{theor}
        Dado un espacio cubriente de $X$, $\cf{p}{\widetilde{X}}{X}$, se tiene que $\pi_1(X)\curvearrowright\widetilde{X}$ vía transformaciones de Deck.
    \end{theor}

    \begin{theor}
        Un grupo $G$ actúa de manera libre en un árbol si y sólo si $G$ es grupo libre.
    \end{theor}

    \begin{proof}
        $\Rightarrow):$ Supongamso que $G$ actúa de manera libre en un árbol $T$. Considere la función proyección $\cf{p}{T}{T/G}$. Afirmamos que $p$ es una función cubriente, en efecto...

        Se tien que $T/G$ es una gráfica, por lo cual:
        \begin{equation*}
            \pi_1(T/G)\cong F_n
        \end{equation*}
        para algún $n\in\mathbb{N}\cup\left\{\infty\right\}$. Usando transormaciones de Deck deducimos que:
        \begin{equation*}
            \pi_1(T/G)\cong G
        \end{equation*}
        por lo cual $G$ es libre.

        $\Leftarrow):$ Supongamos que $G$ es grupo libre, entonces existe un conjunto $S$ tal que $G=F(S)$. Notemos que:
        \begin{equation*}
            F(S)\cong\pi_1\left(X=\bigvee_{ s\in S}\bbm{S}^1 \right)
        \end{equation*}
        Se tiene que este espacio admite un cubrente universal, digamos $\cf{p}{\widetilde{X}}{X}$. Por un teorema, $\widetilde{X}$ es una gráfica y, por ser una gráfica tal que su grupo fundamental es trivial, entonces debe ser un árbol. También se probó que:
        \begin{equation*}
            \textup{Deck}(\widetilde{X})\rightarrow\widetilde{X}
        \end{equation*}
        actúa sobre este árbol (mediante $f\in\textup{Deck}(\widetilde{X})$).
    \end{proof}

    \begin{obs}
        Completar lo anterior formalmente y unir todo con las notas anteriores.
    \end{obs}

    \begin{propo}
        El grupo libre $F_2$ contiene como subgrupo de índice finito a $F_n$, para todo $n\geq2$.
    \end{propo}

    \begin{proof}
        
    \end{proof}

    La explicación de la proposición anterior es que, 

    \begin{excer}
        Encontrar un cubriente $\widetilde{X}$ de $X$ con $\pi_1(X)=F_2$ tal que $\pi_1(\widetilde{X})=F_1$.
    \end{excer}

    \section{Producto Semidirecto de Grupos}

    \begin{mydef}
        Sean $G$ y $H$ grupos, y sea $\cf{\varphi}{H}{\textup{Aut}(G)}$ un morfismo de grupos.

        Definimos el \textbf{producto semidirecto $G\rtimes_{\varphi} H$}, como el grupo en $G\times H$ con la operación dada por:
        \begin{equation*}
            (g_1,h_1)(g_2,h_2)=\left(g_1\varphi_{ h_1}(g_1^{-1}),h_1h_2 \right)
        \end{equation*}
    \end{mydef}

    \begin{propo}
        Probar que dados dos grupos $G$ y $H$, su producto semidirecto es un grupo.
    \end{propo}

    \begin{proof}
        Ejercicio.
    \end{proof}

    \begin{exa}
        Sean $G$ y $H$ grupos, y tomemos $\cf{\varphi}{G}{\textup{Aut}(H)}$ el homomorfismo trivial. Entonces:
        \begin{equation*}
            G\rtimes_{\varphi} H\cong G\times H
        \end{equation*} 
    \end{exa}

    \begin{exa}
        Si $K$ es la botella de Klein, entonces:
        \begin{equation*}
            \pi_1(K)\cong\bbm{Z}\rtimes\bbm{Z}
        \end{equation*}
    \end{exa}

    \begin{proof}
        Recordemos que:
        \begin{equation*}
            \pi_1(K)\cong\langle a,b|aba=b^{-1}\rangle
        \end{equation*}
        y, observemos que todo homomorfismo $\cf{f}{\bbm{Z}}{\textup{Aut}(\bbm{Z})}$ solo tiene de dos:
        \begin{equation*}
            \varphi(1)=\pm\bbm{1}_{\bbm{Z}}
        \end{equation*}
        completar la demostración.
    \end{proof}

    \begin{excer}
        Pruebe que $D_\infty\cong\bbm{Z}\rtimes\bbm{Z}_2$.
    \end{excer}

    \begin{proof}
        Recordemos que:
        \begin{equation*}
            D_\infty=\textup{Aut}(\bbm{R})
        \end{equation*}
        tomando los automorfismos de $\bbm{R}$ como gráfica (con nodos los enteros $\bbm{Z}$).
    \end{proof}

    \section{Grupos finitamente generados y Gráfica de Caley}

    \begin{mydef}
        Un grupo $G$ se dice \textbf{finitamente generado} si existe un conjunto $S\subseteq G$ finito tal que $G=\langle S\rangle$.
    \end{mydef}

    \begin{exa}
        $\bbm{Z}=\langle1\rangle$ es finitamente generado.
    \end{exa}

    \begin{mydef}
        Sea $G$ un grupo finitamente generado, y sea $S\subseteq G$ un conjunto finito de generadores de $G$. La \textbf{gráfica de Caley}, denotada por $\textup{Cay}(G,S)$ se define como una gráfica en la que:
        \begin{itemize}
            \item Los vértices son elementos de $G$, es decir, $V(\textup{Cay}(G,S))=G$.
            \item Las aristas se construyen de la siguiente manera: para cada vértice $g\in G$ y cada $s\in S\cup S^{-1}\setminus\left\{1\right\}$, se dibuja una arista entre $g$ y $gs$.
        \end{itemize}
    \end{mydef}

    \begin{exa}
        Considere $G=\bbm{Z}$ y $S=\left\{1\right\}$. Entonces, la gráfica de Caley estará dada por la gráfica de $\bbm{R}$ con $\bbm{Z}$ los vértices de la gráfica.
    \end{exa}

    \begin{exa}
        Considere $G=\bbm{Z}$ y $S=\left\{2,3\right\}$ ¿cuál es la gráfica de Caley?
    \end{exa}

    \begin{sol}
        Resulta que:
        
        \begin{minipage}{\textwidth}
            \begin{center}
                \includegraphics[scale=0.3]{images/Caley_2_3.png}\\
                Figura \thefigcount. Caption.
                \stepcounter{figcount}
            \end{center}
        \end{minipage}

        es la gráfica de Caley de esta cosa, que es diferente de la gráfica del ejemplo anterior. 
    \end{sol}

    \begin{propo}
        La gráfica de Caley $\textup{Caley}(G,S)$, para un grupo generado $G$ y un conjunto generador $S$, tiene las siguientes propiedades:
        \begin{itemize}
            \item Es arco-conexa (o conexa).
            \item Es localmente finita, es decir que cada vértice tiene solamente un conjunto finito de puntos como vecinos.
        \end{itemize}
    \end{propo}

    \section{Gráfica de Caley como espacio métrico}

    \begin{mydef}
        Dado un grupo $G$ finitamente generado por un conjunto $S$, definimos la \textbf{métrica de palabras} $d_S(g,h)$ entre dos elementos $g,h\in G$ como la longitud mínima de una palabra en los generadores $S\cup S^{-1}$ que representa a $g^{-1}h$.
    \end{mydef}

    \begin{obs}
        De ahora en adelante consideraremos a $\bbm{R}^n$ con la métrica euclideana.
    \end{obs}

    \begin{mydef}
        Sean $X$ y $Y$ espacios métricos. Una función $\cf{f}{X}{Y}$ se dice que es un encaje \textbf{isométrico} si:
        \begin{equation*}
            d(x,y)=\rho(f(x),f(y)),\quad\forall x,y\in X
        \end{equation*}
    \end{mydef}

    \begin{mydef}
        Sean $X$ y $Y$ espacios métricos. Una función $\cf{f}{X}{Y}$ se dice que es un \textbf{encaje bilipschitz} si existe $L\in\bbm{R}_{\geq0}$ tal que:
        \begin{equation*}
            \frac{1}{L}d_X(x,y)\leq d_Y(f(x),f(y))\leq Ld_X(x,y)
        \end{equation*}

        Decimos que $f$ es una \textbf{equivalencia bilipschitz} si existe una función $\cf{g}{Y}{X}$ encaje bilipschitz tal que:
        \begin{equation*}
            g\circ f=\bbm{1}_X\quad\textup{y}\quad f\circ g=\bbm{1}_Y
        \end{equation*}
    \end{mydef}

    \begin{mydef}
        Sea $\cf{f}{X}{Y}$ una función. Decimos que $f$ es un \textbf{encaje quasi-isométrico} si:
        \begin{equation*}
            \frac{1}{c}d_X(x,y)-b\leq d_Y(f(x),f(y))\leq cd_X(x,y)+b
        \end{equation*}
        con $c\geq1$ y $d\geq0$.
    \end{mydef}

    \newcommand{\floor}[1]{\ensuremath{\lfloor #1\rfloor}}

    \begin{exa}
        Consideremos a $\bbm{R}$ con la métrica euclideana y tomemos el encaje $\cf{i}{\bbm{Z}}{\bbm{R}}$. Queremos encontrar constates $c\geq1$ y $b\geq0$ tales que:
        \begin{equation*}
            \frac{1}{c}\abs{n-m}-b\leq\abs{i(n)-i(m)}\leq c\abs{n-m}+b
        \end{equation*}
        para todo $n,m\in\bbm{Z}$. Podemos elegir $b=0$ y $c=1$, con lo cual sucede que:
        \begin{equation*}
            \abs{n-m}\leq\abs{i(n)-i(m)}\leq \abs{n-m}
        \end{equation*}
        para todo $n,m\in\bbm{Z}$.
    \end{exa}

    \begin{exa}
        Consideremos la función parte entera $x\mapsto \floor{x}=\max\left\{n\in\mathbb{Z}\Big|n\leq x \right\}$. Afirmamos que $\floor{\cdot}$ es un encaje quasi-isométrico.
    \end{exa}

    \begin{proof}
        En efecto, sean $x,y\in\bbm{R}$, debemos encontrar $c\geq1$ y $b\geq0$ tales que:
        \begin{equation*}
            \frac{1}{c}\abs{x-y}-b\leq\abs{\floor{x}-\floor{y}}\leq c\abs{x-y}+b
        \end{equation*}
        Tomemos $c=b=1$. En efecto, sean $x,y\in\bbm{R}$, se tiene que:
        \begin{equation*}
            \abs{x-y}\leq \abs{\floor{x}-\floor{y}}+1
        \end{equation*}
        y, de forma análoga:
        \begin{equation*}
            \abs{\floor{x}-\floor{y}}\leq\abs{x-y}+1
        \end{equation*}
        con lo que se tienen las desigualdades deseadas.
    \end{proof}

    \newcommand{\qisom}{\ensuremath{\underset{C.I.}{\sim}}}

    \begin{mydef}
        Sea $\cf{f}{X}{Y}$ un $(c,b)$-encaje quasi-isométrico.
        \begin{enumerate}[label = \textit{(\alph*)}]
            \item Decimos que una función $\cf{f'}{X}{Y}$ está a \textbf{distancia fintia de $f$} si existe una constante $k\in\bbm{R}_{\geq0}$ tal que:
            \begin{equation*}
                d_Y(f(x),f'(x))\leq k,\quad\forall x\in X
            \end{equation*}
            \item Decimos que $f$ es una \textbf{quasi-isometría} si existe un encaje quasi-isométrico $\cf{g}{Y}{X}$ tal que $f\circ g$ está a distancia finita de $\bbm{1}_Y$ y $g\circ f$ lo está de $\bbm{1}_X$.
            \item Si $X$ es quasi-isométrico a $Y$, escribimos $X\qisom Y$.
        \end{enumerate}
    \end{mydef}

    \begin{mydef}
        Sea $\cf{f}{X}{Y}$. Decimos que $f$ tiene una \textbf{imagen quasi-densa} si existe $k\in\bbm{R}_{\geq0}$ tal que:
        \begin{equation*}
            \forall y\in Y \exists x\in X d_Y(y,f(x))\leq k
        \end{equation*}
    \end{mydef}

    \begin{theor}[\textbf{Caracterización de Quasi-isometrías}]
        Una función $\cf{f}{X}{Y}$ es una quasi-isometría si y sólo si $f$ es un encaje isométrico con imagen quasi-densa.
    \end{theor}

    \newcommand{\im}[1]{\ensuremath{\textup{im}\left(#1\right)}}

    \begin{proof}
        $\Rightarrow):$ Supongamos que $\cf{f}{X}{Y}$ es un encaje C.I. Veamos que $f$ es encaje quasi-isométrico y que $\im{f}\subseteq Y$ es quasi-densa.

        Claramente es encaje quasi-isométrico. Ahorra, por ser encaje C.I. existe una función $\cf{g}{Y}{X}$ encaje quasi-isométrico tal que:
        \begin{equation*}
            d_Y(f\circ g,\bbm{1}_Y)\leq k
        \end{equation*}
        para aklgún $k\in\bbm{R}_{\geq0}$. Entonces, si $y\in Y$ existe $x=g(y)\in X$ tal que:
        \begin{equation*}
            d_Y(f(x),y)\leq k
        \end{equation*}
        por lo que la imagen de $f$ es quasi-densa.

        $\Leftarrow):$ Supongamos que $\cf{f}{X}{Y}$ es un $(c,b)$ encaje quasi-isométrico con imagen quasi-densa, es decir que:
        \begin{equation*}
            \frac{1}{c}d_X(x,y)-b\leq d_Y(f(x),f(y))\leq cd_X(x,y)+b
        \end{equation*}
        $\forall x,y\in X$. Y además, existe $c^*\in\bbm{R}_{\geq0}$ tal que $\forall y\in Y$ existe $x\in X$ tal que:
        \begin{equation*}
            d_Y(f(x),y)\leq c^*
        \end{equation*}
        Tomemos $k=\max\left\{c,c^*,b \right\}\geq 1$. Entonces:
        \begin{equation*}
            \frac{1}{k}d_X(x,y)-k\leq d_Y(f(x),f(y))\leq kd_X(x,y)+k
        \end{equation*}
        y,
        \begin{equation*}
            d_Y(f(x),y)\leq k
        \end{equation*}

        Sea $y\in Y$, tomemos un $x_y\in X$ tal que:
        \begin{equation*}
            d_Y(f(x_y),y)\leq k
        \end{equation*}
        hagamos $g(y)=x_y$ para todo $y\in Y$.
    \end{proof}

    \begin{exa}
        Todo espacio métrico $X$ de diámetro finito es quasi-isométrico al espacio de un punto.
    \end{exa}

    \begin{mydef}
        Sea $G$ un grupo actuando en una gráfica $(V,E)$, esto es, tenemos un homomorfismo $\cf{\varphi}{G}{\Aut{V,E}}$. La acción $\varphi$ es \textbf{libre} si para todo $g\in G\setminus\left\{e\right\}$:
        \begin{equation*}
            \forall v\in V, \varphi_g(v)\neq v
        \end{equation*}
        y,
        \begin{equation*}
            \forall\left\{v,w \right\}\in E,\varphi_g\left(\left\{v,w \right\}\right)=\left\{\varphi_g(v),\varphi_g(w) \right\}\neq \left\{v,w \right\}
        \end{equation*}
    \end{mydef}

    \begin{exa}
        Suponga que $G=\gen{S}$, entonces tenemos una acción:
        \begin{equation*}
            \cf{\varphi}{G}{\Aut{V,E}}
        \end{equation*}
        tal que $g\mapsto g\cdot h$.
    \end{exa}

    \begin{obs}
        La acción se reestringe a:
        \begin{equation*}
            \cf{\varphi}{G}{\Aut{G}}
        \end{equation*}
        tal que $g\mapsto\varphi_g$. Afirmamos que esta acción es libre.
    \end{obs}

    \begin{proof}
        En efecto, se tiene que:
        \begin{equation*}
            \begin{split}
                \varphi_g(h)=h&\iff g\cdot h=h\\
                &\iff g=e\\
            \end{split}
        \end{equation*}
        así que la acción es libre.
    \end{proof}

    \begin{mydef}
        Una acción $G\curvearrowright X$ es \textbf{transitiva} si $\forall x\in X$ tenemos que:
        \begin{equation*}
            \mathcal{O}_x=\left\{gx\Big|g\in X \right\}=X
        \end{equation*}
    \end{mydef}

    \begin{propo}
        Sea $G$ un grupo y $S\subseteq G$ un conjunto generador de $G$. Entonces, la acción $\cf{\varphi}{G}{\Aut{\Cay{G,S}}}$ dada por:
        \begin{equation*}
            g\mapsto(h\mapsto g\cdot h)
        \end{equation*}
        es libre si y sólo si no contiene involuciones, es decir que no existe $s\in S$ tal que $s^2=e$.
    \end{propo}

    \begin{proof}
        Sabemos que la acción $\varphi$ se reestringe a $G$ y es libre.

        $\Rightarrow):$ Supo

        $\Leftarrow):$ Supongamos que $\varphi$ no es libre, entonces existe una arista $\left\{v,w \right\}$ tal que:
        \begin{equation*}
            g\cdot\left\{v,w \right\}=\left\{v,w \right\}
        \end{equation*}
        para algún $g\in G\setminus\left\{e \right\}$. Tenemos dos casos:
        \begin{itemize}
            \item $g\cdot v=v$\contradiction, cosa que no puede suceder ya que la acción es libre sobre $G$.
            \item $g\cdot v=w$, por lo que $g\cdot w=v$, así que:
            \begin{equation*}
                v=g\cdot w=g(g\cdot v)=g^2\cdot v\Rightarrow g^2=e
            \end{equation*}
        \end{itemize}
        con lo que se sigue que $G$ tiene involuciones.
    \end{proof}

    \begin{propo}
        Sea $G$ un grupo finitamente generado. Entonces, la gráfica de Caley es única hasta quasi-isometrías.
    \end{propo}

    \begin{proof}
        Sean $S$ y $S'$ conjuntos finitos generadores de $G$. Probaremos que:
        \begin{equation*}
            \cf{\bbm{1}}{\Cay{G,S}}{\Cay{G,S'}}
        \end{equation*}
        es una equivalencia bilipschitz. Sea:
        \begin{equation*}
            n=\max\left\{d(e,s)\Big|s\in S' \right\}
        \end{equation*}
        Notemos que $n$ existe ya que el conjunto de la derecha es finito.
    \end{proof}

    \begin{mydef}
        Dos grupos $G$ y $H$ son \textbf{quasi-isométricos} si $\Cay{G,S}\qisom\Cay{H,T}$.
    \end{mydef}

    \begin{exa}
        Todos los grupos finitos son quasi-isométricos.
    \end{exa}

    ¿Cómo clasificar grupos hasta quasi-isometrías?

    La primera respuesta es construír invariantes quasi-isométricos.

    \section{Lema de Svarc-Milnor y algunas aplicaciones}

    \begin{mydef}
        Sea $X$ un espacio métrico.
        \begin{enumerate}[label = \textit{(\alph*)}]
            \item Una \textbf{geodésica} entre dos puntos $x,y\in X$ es un encaje isomérico $\cf{\eta}{[a,b]}{X}$ tal que $\eta(a)=x$ y $\eta(b)=y$, con la longitud de $\eta$ igual a $\abs{b-a}$.
            \item Decimos que $X$ es un \textbf{espacio geodésico} si para todo $x,y\in X$ existe una geodésica entre $x$ y $y$.
        \end{enumerate}
    \end{mydef}

    \begin{obs}
        Aquí habla de localidad, cuando dice que es el camino más corto es \textit{localmente}.
    \end{obs}

    \begin{exa}
        Considere a $(\bbm{R}^n,\|\cdot\|)$ con la métrica euclideana. Las geodésicas en este espacio son las líneas rectas.
    \end{exa}

    \begin{exa}
        El espacio $\bbm{R}^2\setminus\left\{0\right\}$ no es geodésico.
    \end{exa}

    \begin{exa}
        El espacio $\bbm{H}^2$ es geodésico.
    \end{exa}

    \begin{exa}
        En $\bbm{S}^2\subseteq\bbm{R}^3$ se tiene que las geodésicas son curvas maximales, dadas por planos que pasan por el origen.
    \end{exa}

    \begin{theor}
        Sea $S$ una superficie compacta, cerrada y conexa. Entonces:
        \begin{equation*}
            S=\left\{
                \begin{array}{l}
                    \bbm{S}^2 \\
                    \underset{n-\textup{veces}}{\underbrace{\bbm{T}\#\cdots\#\bbm{T}}}\\
                    \underset{n-\textup{veces}}{\underbrace{\bbm{R}P^2\#\cdots\#\bbm{R}P^2}}\\
                \end{array}
            \right.
        \end{equation*}
        con $n\in\bbm{N}\cup\left\{0\right\}$.
    \end{theor}

    Resulta que algunos de estos espacios tienen como cubriente universal a $\bbm{S}^2$, $\bbm{R}^2$ o $\bbm{H}^2$. Debido a que estos tienen métrica (y son geodésicos), puede bajarse la métrica localmente a los espacios que cubren.

    \begin{mydef}
        Sea $X$ un espacio métrico.
        \begin{enumerate}[label = \textit{(\alph*)}]
            \item Una \textbf{cuasi-geodésica} entre dos puntos $x,y\in X$ es un encaje cuasi-isométrico $\cf{\gamma}{[a,b]}{X}$ tal que $\gamma(a)=x$ y $\gamma(b)=y$.
            \item Decimos que $X$ es \textbf{cuasi-geodésico} si existen $a\geq 1$ y $b\geq 0$ tales que para todo $x,y\in X$ existe una $(a,b)$-cuasi-geodésica que los une.
        \end{enumerate}
    \end{mydef}

    \begin{exa}
        Todo espacio métrico geodésico es cuasi-geodésico.
    \end{exa}

    \begin{exa}
        El espacio $\bbm{R}^2\setminus\left\{0\right\}$ es cuasi-geodésico.
    \end{exa}

    Checar Clara Löph la página 129.

    \begin{propo}
        Una gráfica conexa es $(1,1)$-cuasi-geodésico.
    \end{propo}

    \begin{theor}
        Sea $G$ un grupo tal que actúa en un espacio métrico $(X,d)$ por isometrías. Supongamos que $X$ es $(c,b)$-cuasi-geodésico y que existe un conjunto $B\subseteq X$ con las siguientes propiedades:
        \begin{enumerate}[label = \textit{(\alph*)}]
            \item $\textup{diam}(B)<\infty$.
            \item $\bigcup_{ g\in G}g\cdot B=X$.
            \item El conjunto $S=\left\{g\in G\Big|g\cdot B'\cap B'\neq\emptyset \right\}$ es finito donde:
            \begin{equation*}
                B'=B_{ 2b}^{(X.d)}(B)=\left\{x\in X\Big|\exists y\in B\textup{ tal que }d(y,x)\leq 2b \right\}
            \end{equation*}
        \end{enumerate}
        Entonces:
        \begin{enumerate}[label = \textit{(\arabic*)}]
            \item $G=\gen{S}$.
            \item La función $\cf{\varphi}{G}{X}$ dada por:
            \begin{equation*}
                g\mapsto g\cdot x
            \end{equation*}
            es una cuasi-isometría. Aquí estamos considerando a $G$ con la métrica de palabras.
        \end{enumerate}
    \end{theor}

    \begin{cor}
        Sea $G$ un grupo y $H<G$ un subgrupo de índice finito. Si $G$ es finitmente generado, entonces $H\qisom G$. 
    \end{cor}

    \begin{mydef}
        Decimos que un grupo $G$ es \textbf{virtualmente $\bbm{Z}^n$} si $G$ tiene un subgrupo isomorfo a $\bbm{Z}^n$ de índice finito.
    \end{mydef}

    \begin{theor}
        Todo grupo virtualmente $\bbm{Z}$ es cuasi isométrico a $\bbm{Z}$.
    \end{theor}

    \begin{propo}
        Todo grupo libre de rango finito $\geq2$ es cuasi-isométrico a $F_2$
    \end{propo}

    \begin{proof}
        Sea $F_n$ el grupo libre en $n$-generadores, entonces $F_n<F_2$ es de índice finito, así que ambos son cuasi-isométricos.
    \end{proof}

    \begin{cor}
        $F_n\qisom F_2$.
    \end{cor}

    \begin{theor}
        Sea $M$ una variedad cerrada Riemmaniana, conexa y esférica (es decir que $\widetilde{M}$ el cubriente universal es contraíble). Entonces:
        \begin{equation*}
            \pi_1(M)\qisom\widetilde{ M}
        \end{equation*}
    \end{theor}

    \begin{mydef}
        Un espacio métrico $X$ es \textbf{propio} si las bolas cerradas son compactas.
    \end{mydef}

    \begin{mydef}
        Una acción $G\times X\rightarrow X$ de un grupo $G$ en un espacio toplógico es \textbf{propia} si para todo subconjunto compacto $B\subseteq X$, el conjunto:
        \begin{equation*}
            \left\{g\in G\Big|g\cdot B\cap B\neq\emptyset \right\}
        \end{equation*}
        es finito.
    \end{mydef}

    \begin{mydef}
        Una acción $G\times X\rightarrow X$ de un grupo $G$ en un espacio topológico $X$ es \textbf{cocompacta} si el espacio cociente $G/X$ es compacto respecto a la topología cociente. 
    \end{mydef}

    \begin{lema}[\textbf{Lema de Svar-Milnor Versión Topológica}]
        Sea $G$ un grupo actúando en un espacio métrico $X$ propio y geodésico. Además, supongamos que la acción es propia y cocompacta.

        Entonces, $G$ es finitamente generado y $G\qisom X$.
    \end{lema}

    \begin{exa}
        $\bbm{Z}$ actuando en $\bbm{R}$ por traslaciones es un conjunto finito.
    \end{exa}

    \begin{exa}
        Si $X$ es un CW-complejo conexo, entonces:
    \end{exa}

    \begin{center}
        \textit{¿Cómo clasificamos grupos hasta cuasi-isometrías?}
    \end{center}

    Esto es, si tenemos dos grupos $G$ y $H$, cómo sabemos si hay una cuasi-isometría entre ellos. En este caso, debemos mostrar la existencia o no de tal cosa.

    Es más fácil decir cuando son o no son cuasi-isométricos. La razón de esto es que a veces es más facil construir invariantes cuasi-isométricos.

    Para esto, debemos construir invariantes cuasi-isométricos.

    La idea es que vamos a tomar un espacio métrico y nos vamos a fijar en todas las formas de \textit{escapar} al \textit{infinito}, queriendo decir esto \textit{alejarnos} de la identidad lo más posible.

    Por ejemplo, en $\Cay{\bbm{Z},\left\{1\right\}}$ nos podemos escapar de dos maneras al infinito.

    Pero, en el espacio:

    \begin{minipage}{\textwidth}
        \begin{center}
            \includegraphics[scale=0.3]{images/F_2.png}\\
            Figura \thefigcount. Caption.
            \stepcounter{figcount}
        \end{center}
    \end{minipage}

    hay una cantidad infinita de maneras de escapar al infinito.

    \chapter{Ejercicios y Problemas}

    \section{Introducción a la teoría geométrica de grupos}
    
    \begin{excer}
        Utilizando la teoría de cubrientes demuestra que:
        \begin{enumerate}[label = \textit{(\arabic*)}]
            \item Todo subgrupo de índice finito en un grupo finitamente generado es finitamente generado.
            \item Todo subgrupo de índice finito en un grupo finitamente presentado es finitamente presentado.
        \end{enumerate}
    \end{excer}

    \begin{proof}
        De \textit{(1)}: Sea $G$ un grupo finitamente generado y $H<G$ tal que $[G:H]<\infty$.


    \end{proof}

    \begin{excer}
        Demuestra que en el producto semidirecto $N\rtimes_{\varphi}H$, $H$ es un subgrupo normal si y sólo si $\varphi$ es el homomorfismo trivial.
    \end{excer}

    \begin{proof}
        Recordemos que el producto semidirecto $N\rtimes_\varphi H$ es el grupo $N\times H$ dotado de la operación:
        \begin{equation*}
            (n,h)(n',h')=(n\varphi_h(n'),hh')
        \end{equation*}
        donde $\cf{\varphi}{H}{\Aut{N}}$ es un homomorfismo tal que $h\mapsto\varphi_h$. El elemento neutro de este grupo es $(e_N,e_H)$, donde cada elemento tiene como inverso:
        \begin{equation*}
            (n,h)^{-1}=\left((\varphi_{h^{-1}}(n))^{-1},h^{-1}\right)
        \end{equation*}

        Sean $(n_1,h_1)\in N\rtimes_{\varphi}H$ y $h\in H$, se tiene que:
        \begin{equation*}
            \begin{split}
                (n_1,h_1)(e_N,h)(n_1,h_1)^{-1}&=(n_1,h_1)(e_N,h)\left((\varphi_{h_1^{-1}}(n_1))^{-1},h_1^{-1}\right)\\
                &=(n_1\varphi_{h_1}(e_N),h_1h)\left((\varphi_{h_1^{-1}}(n_1))^{-1},h_1^{-1}\right)\\
                &=(n_1\varphi_{h_1}(e_N),h_1h)\left((\varphi_{h_1^{-1}}(n_1))^{-1},h_1^{-1}\right)\\
                &=(n_1e_N,h_1h)\left((\varphi_{h_1^{-1}}(n_1))^{-1},h_1^{-1}\right)\\
                &=(n_1,h_1h)\left((\varphi_{h_1^{-1}}(n_1))^{-1},h_1^{-1}\right)\\
                &=\left(n_1\varphi_{h_1h}\left((\varphi_{h_1^{-1}}(n_1))^{-1}\right),h_1hh_1^{-1} \right)\\
                &=\left(n_1\varphi_{h_1h}\left((\varphi_{h_1^{-1}}(n_1^{-1}))\right),h_1hh_1^{-1} \right)\\
                &=\left(n_1\varphi_{h_1h h_1^{-1}}\left(n_1^{-1}\right),h_1hh_1^{-1} \right)\\
            \end{split}
        \end{equation*}
        pues, $\varphi_{h_1}(e_N)=e_N$ y por ser $h\mapsto\varphi_h$ homomorfismo.

        $\Rightarrow)$: Suponga que $H$ es un subgrupo normal de $N\rtimes_\varphi H$, esto es que el grupo $H$ visto como subgrupo de $N\rtimes_\varphi H$:
        \begin{equation*}
            H=\left\{(e_N,h)\Big|h\in H \right\}
        \end{equation*}
        es subgrupo normal de $N\rtimes_\varphi H$. Como es normal, se sigue que:
        \begin{equation*}
            (n_1,h_1)(e_N,h)(n_1,h_1)^{-1}\in H
        \end{equation*}
        para todo $(n_1,h_1)\in N\rtimes_{\varphi}H$ y $h\in H$, por lo que:
        \begin{equation*}
            \left(n_1\varphi_{h_1h h_1^{-1}}\left(n_1^{-1}\right),h_1hh_1^{-1} \right)\in H
        \end{equation*}
        nuevamente, para todo $(n_1,h_1)\in N\rtimes_{\varphi}H$ y $h\in H$. En particular:
        \begin{equation*}
            n_1\varphi_{h_1h h_1^{-1}}\left(n_1^{-1}\right)=e_N
        \end{equation*}
        por lo que para todo $n\in N$ y $h\in H$:
        \begin{equation*}
            n^{-1}\varphi_{h}\left(n\right)=e_N\Rightarrow\varphi_h(n)=n
        \end{equation*}
        es decir, que $\varphi_h=\bbm{1}_H$, por lo que $h\mapsto \varphi_h$ es el homomorfismo trivial.
        
        $\Leftarrow):$ Suponga que $\varphi$ es trivial, se sigue que:
        \begin{equation*}
            \begin{split}
                (n_1,h_1)(e_N,h)(n_1,h_1)^{-1}&=\left(n_1\varphi_{h_1h h_1^{-1}}\left(n_1^{-1}\right),h_1hh_1^{-1} \right)\\
                &=(n_1\bbm{1}_H(n_1^{-1}),h_1hh_1^{-1})\\
                &=(n_1n_1^{-1},h_1hh_1^{-1})\\
                &=(e_N,h_1hh_1^{-1})\in H\\
            \end{split}
        \end{equation*}
        para todo $(n_1,h_1)\in N\rtimes_{\varphi}H$ y $h\in H$, por lo que $H$ es normal en $N\rtimes_{\varphi}H$.
    \end{proof}

    \begin{excer}
        Demuestra que el grupo diédrico infinito $D_\infty$ es isomorfo tanto al producto libre $\bbm{Z}/2\bbm{Z}*\bbm{Z}/2\bbm{Z}$ como al producto semidirecto $\bbm{Z}\rtimes\bbm{Z}/2\bbm{Z}$.
    \end{excer}

    \begin{proof}
        Recordemos que:
        \begin{equation*}
            D_\infty=\gen{r,s\Big|s^2=1,srs=r^{-1}}
        \end{equation*}
        Probaremos que es isomorfo a ambos grupos.
        \begin{itemize}
            \item Observemos que el producto semidirecto $\bbm{Z}\rtimes\bbm{Z}/2\bbm{Z}$ es tal que el homomorfismo $\cf{\varphi}{\bbm{Z}/2\bbm{Z}}{\Aut{\bbm{Z}}}$ tiene dos opciones, o es el trivial, ya que:
            \begin{equation*}
                \varphi(0)=\bbm{1}_{\bbm{Z}}
            \end{equation*}
            y $\varphi(1)=-\bbm{1}_\bbm{Z}$, o:
            \begin{equation*}
                \varphi(x)=\bbm{1}_{\bbm{Z}},\quad\forall x\in\bbm{Z}/2\bbm{Z}
            \end{equation*}
            Analicemos ambos casos:
            \begin{itemize}
                \item Si $\varphi$ no es trivial, se sigue que:
                \begin{equation*}
                    \begin{split}
                        (x_1,y_1)(x_2,y_2)&=(x_1+\varphi_{ y_1}(x_2),y_1+y_2)\\
                        &=\left\{
                            \begin{array}{lcr}
                                (x_1+\bbm{1}_\bbm{Z}(x_2),0+y_2) & \textup{ si } & y_1=0 \\
                                (x_1-\bbm{1}_\bbm{Z}(x_2),1+y_2) & \textup{ si } & y_1=1 \\
                            \end{array}
                        \right.\\
                        &=\left\{
                            \begin{array}{lcr}
                                (x_1+x_2,y_2) & \textup{ si } & y_1=0 \\
                                (x_1-x_2,1+y_2) & \textup{ si } & y_1=1 \\
                            \end{array}
                        \right.\\
                    \end{split}
                \end{equation*}
                Afirmamos que en este caso, $D_\infty\cong\bbm{Z}\rtimes\bbm{Z}/2\bbm{Z}$. En efecto, considere la función $\cf{f}{D_\infty}{\bbm{Z}\rtimes\bbm{Z}/2\bbm{Z}}$ dada por:
                \begin{equation*}
                    f(r)=(1,0)\quad\textup{y}\quad f(s)=(0,1)
                \end{equation*}
                Como $D_\infty$ admite una presentación, $f$ se puede extender a un homomorfismo. Veamos que este homomorfismo es inyectivo y suprayectivo.
                \begin{itemize}
                    \item \textbf{$f$ es inyectiva}: Sea $s^{\epsilon_1}r^ns^{\epsilon_2}\in D_\infty$ con $\epsilon_1,\epsilon_2\in\left\{0,1\right\}$ y $n\in\mathbb{N}\cup\left\{0\right\}$ tal que:
                    \begin{equation*}
                        f(s^{\epsilon_1}r^ns^{\epsilon_2})=0
                    \end{equation*}
                    entonces:
                    \begin{equation*}
                        \begin{split}
                            f(s^{\epsilon_1}r^ns^{\epsilon_2})&=(0,\epsilon_1)(n,0)(0,\epsilon_2)\\
                            &=(0,\epsilon_1)(n,\epsilon_2)\\
                            &=\left\{
                                \begin{array}{lcr}
                                    (n, \epsilon_2) & \textup{ si } & \epsilon_1=0 \\
                                    (-n, 1+\epsilon_2) & \textup{ si } & \epsilon_1=1 \\
                                \end{array}
                            \right.\\
                        \end{split}
                    \end{equation*}
                    en cualquier caso, debe tenrse que $n=0$, por lo cual $\epsilon_2=0$ si $\epsilon_1=0$ y $\epsilon_2=1=\epsilon_1$, en cualquier caso se sigue que:
                    \begin{equation*}
                        s^{\epsilon_1}r^ns^{\epsilon_2}=1
                    \end{equation*}
                    por lo que $\ker(f)=\gen{1}$.
                    \item \textbf{$f$ es suprayectiva}: Sea $(m,\epsilon)\in\bbm{Z}\rtimes\bbm{Z}/2\bbm{Z}$. Veamos que:
                    \begin{equation*}
                        \begin{split}
                            f(r^ms^{\epsilon})&=(m,\epsilon)
                        \end{split}
                    \end{equation*}
                    es el elemento deseado.
                \end{itemize}
                Por ambos incisos se sigue que $f$ es isomorfismo.
            \end{itemize}
            \item Veamos que el producto libre $\bbm{Z}/2\bbm{Z}*\bbm{Z}/2\bbm{Z}$ está dado por:
            \begin{equation*}
                \bbm{Z}/2\bbm{Z}*\bbm{Z}/2\bbm{Z}=\gen{x,y\Big|x^2=1\textup{ y }y^2=1}
            \end{equation*}
            pues, $\bbm{Z}/2\bbm{Z}=\gen{x\Big|x^2=1}$. Los elementos de este grupo son sucesiones alternantes $xyxy\cdots$ o $yxyx\cdots$ de los elementos $x$ y $y$. Considere la función $\cf{f}{\bbm{Z}/2\bbm{Z}*\bbm{Z}/2\bbm{Z}}{D_\infty}$ dada por:
            \begin{equation*}
                f(x)=r\quad\textup{y}\quad f(y)=s
            \end{equation*}
            esta es función y más aún, puede ser extendida a un homeomorfismo ya que como el producto libre admite presentación basta con definirlo sobre los elementos de la base (usando además las propiedades de grupos libres).

            Por ende, veamos que es inyectiva y sobreyectiva:
            \begin{itemize}
                \item \textbf{$f$ es inyectiva}: Sea $z\in\bbm{Z}/2\bbm{Z}*\bbm{Z}/2\bbm{Z}$ tal que $f(z)=0$. Se tienen cuatro casos:
                \begin{enumerate}[label = \textit{(\alph*)}]
                    \item $z=xy\cdots xy$ sea $m$ el número de veces que aparece $r$ en este producto. Se sigue que $f(z)=f(x)f(y)\cdots f(x)f(y)=rs\cdots rs$. En esencia, todos los elementos están alternando y, usando el hecho de que en $D_\infty$ se cumple:
                    \begin{equation*}
                        rs=sr^{-1}
                    \end{equation*}
                    podemos reducir todo lo anterior a algo de la forma $s^{\epsilon}(r^{-1})^n$ con $\epsilon\in\left\{0,1\right\}$ y $n\in\mathbb{N}$. Se sigue así que:
                    \begin{equation*}
                        \begin{split}
                            f(z)=s^{\epsilon}r^n=0
                        \end{split}
                    \end{equation*}
                    y eso es cero si y sólo si $\epsilon=0$ y $n=0$. Pero $n$ está dado por el número de veces que cambiamos $r$ a la derecha, que fueron $m$-veces, es decir que $m=n$. Así que $m=0$. Por tanto, $z=0$.
                    \item Los otros casos son $z=xy\cdots yx$, $z=yx\cdots yx$ y $z=yx\cdots xy$, donde se procede de forma análoga al ejemplo anterior.
                \end{enumerate}
                \textbf{$f$ es sobreyectiva}. Sea $u\in D_\infty$. Por las propiedades de este grupo este elemento se expresa como:
                \begin{equation*}
                    s^{\epsilon_1} r^ns^{\epsilon_2}
                \end{equation*}
                donde $\epsilon_1,\epsilon_2\in\left\{0,1\right\}$ y $n\in\mathbb{N}\cup\left\{0\right\}$. Veamos por casos:
                \begin{itemize}
                    \item 
                \end{itemize}
            \end{itemize}
        \end{itemize}
    \end{proof}

    \begin{mydef}
        Sea $G$ un grupo finitamente generado, y sea $S\subseteq G$ un conjunto finito de generadores de $G$. La \textbf{gráfica de Caley}, denotada por $\textup{Cay}(G,S)$ se define como una gráfica en la que:
        \begin{itemize}
            \item Los vértices son elementos de $G$, es decir, $V(\textup{Cay}(G,S))=G$.
            \item Las aristas se construyen de la siguiente manera: para cada vértice $g\in G$ y cada $s\in S\cup S^{-1}\setminus\left\{1\right\}$, se dibuja una arista entre $g$ y $gs$.
        \end{itemize}
    \end{mydef}

    \begin{excer}
        Esboza la gráfica de Caley del producto libre $\bbm{Z}/2\bbm{Z}*\bbm{Z}/2\bbm{Z}$
    \end{excer}

    \begin{sol}
        Primero, un conjunto de generadores de $\bbm{Z}/2\bbm{Z}*\bbm{Z}/2\bbm{Z}=\gen{x,y\Big|x^2=y^2=1}$ es:
        \begin{equation*}
            S=\left\{x,y\right\}=S\cup S^{-1}
        \end{equation*}
        Considere el elemento $e$. De este elemento parten dos aristas, una hacia $x$ y otra hacia $y$.
        
        Luego, de $x$ se dibuja una arista hacia $xy$ y otra hacia $e$, que ya estaba. De $y$ se dibuja una hacia $yx$ y otra hacia $e$, que ya estaba.

        En síntesis, esto se vería como una recta con centro $e$ y los elementos $x$ y $y$ sus vecinos, luego se ponen como vecinos los elementos $xy$ y $yx$, alternando.

        \begin{center}
            \begin{tikzpicture}
                \begin{scope}[every node/.style={draw,circle}]
                    \node (-3)[label = $yxy$] at (-6,0){};
                    \node (-2)[label = $yx$] at (-4,0){};
                    \node (-1)[label = $y$] at (-2,0){};
                    \node (0)[label = $e$] at (0,0){};
                    \node (1)[label = $x$] at (2,0){};
                    \node (2)[label = $xy$] at (4,0){};
                    \node (3)[label = $xyx$] at (6,0){};
                    \draw[-] (-3) to (-8,0);
                    \draw[-] (-3) to (-2);
                    \draw[-] (-2) to (-1);
                    \draw[-] (-1) to (0);
                    \draw[-] (0) to (1);
                    \draw[-] (1) to (2);
                    \draw[-] (2) to (3);
                    \draw[-] (3) to (8,0);
                \end{scope}
            \end{tikzpicture}
        \end{center}

        Otro conjunto de generadores es:
        \begin{equation*}
            S=\left\{x,xy \right\}\Rightarrow S\cup S^{-1}=\left\{x,xy,yx \right\}
        \end{equation*}
        en este caso, va a cambiar la gráfica y quedará de la siguiente manera:

        \begin{center}
            \begin{tikzpicture}
                \begin{scope}[every node/.style={draw,circle}]
                    \node (-2)[label = $xyx$] at (-4,0){};
                    \node (-1)[label = $xy$] at (-2,0){};
                    \node (0)[label = $e$] at (0,0){};
                    \node (1)[label = $x$] at (2,0){};
                    \node (2)[label = $y$] at (4,0){};
                    \node (3)[label = $yx$] at (6,0){};
                    \draw[-] (-3) to (-2);
                    \draw[-] (-2) to (-1);
                    \draw[-] (-1) to (0);
                    \draw[-] (0) to (1);
                    \draw[-] (1) to (2);
                    \draw[-] (2) to (3);
                    \draw[-][bend right = 20] (-2) to (3);
                    \draw[-][bend right = 20] (-2) to (1);
                \end{scope}
            \end{tikzpicture}
        \end{center}

    \end{sol}

    \begin{excer}
        Haga lo siguiente:
        \begin{enumerate}[label = \textit{(\arabic*)}]
            \item Demuestra que existen conjuntos generadores finitos $S$ de $\bbm{Z}$ y $T$ de $D_\infty$ tales que $\Cay{\bbm{Z},S}\cong\Cay{D_\infty,T}$.
            \item Demuestra que existen conjuntos generadores finitos $S$ de $\bbm{Z}\times\bbm{Z}/2\bbm{Z}$ y $T$ de $D_\infty$ tales que $\Cay{\bbm{Z}\times\bbm{Z}/2\bbm{Z}}\cong\Cay{D_\infty,T}$.
        \end{enumerate}
    \end{excer}

    \begin{proof}
        De \textit{(1)}: Como $D_\infty\cong\bbm{Z}/2\bbm{Z}*\bbm{Z}/2\bbm{Z}$, entonces por el ejercicio anterior un conjunto infinito de generadores que sirve es:
        \begin{equation*}
            S=\left\{1\right\}
        \end{equation*}
        y $T=\left\{x,y\right\}$.

        De \textit{(2)}: 
    \end{proof}

    \section{Quasi-isometrías}

    \begin{excer}
        Toda función a una distancia finita de un encaje quasi-isométrico es un encaje quasi-isométrico.
    \end{excer}

    \begin{proof}
        Sean $X$ y $Y$ espacios métricos y $\cf{f,g}{X}{Y}$ funciones tales que $f$ es una $(c,d)$-encaje quasi-isométrico y la distancia entre ambas funciones es finita, es decir que existe $k\in\bbm{R}_{\geq0}$ tal que:
        \begin{equation*}
            d_Y(f(x),g(x))\leq k,\quad\forall x\in X
        \end{equation*}
        Por ser $f$ encaje quasi-isométrico se cumple que:
        \begin{equation*}
            \frac{1}{c}d_X(x.y)-b\leq d_Y(f(x),f(y))\leq cd_X(x,y)+b
        \end{equation*}
        Ahora, veamos que:
        \begin{equation*}
            \begin{split}
                d_Y(f(x),f(y))&\leq d_Y(f(x),g(x))+d_Y(g(x),g(y))+d_Y(g(y),f(y))\\
                &\leq 2k+d_Y(g(x),g(y))\\
                \Rightarrow d_Y(f(x),f(y))&\leq d_Y(g(x),g(y))+2k\\
            \end{split}
        \end{equation*}
        de forma análoga:
        \begin{equation*}
            \begin{split}
                d_Y(f(x),f(y))&\geq d_Y(g(x),f(y))-d_Y(f(x),g(x))\\
                &\geq d_Y(g(x),g(y))-d_Y(g(y),f(y))-d_Y(f(x),g(x))\\
                &\geq d_Y(g(x),g(y))-2k\\
                \Rightarrow d_Y(f(x),f(y))&\geq d_Y(g(x),g(y))-2k\\
            \end{split}
        \end{equation*}

        Por ende,
        \begin{equation*}
            \begin{split}
                \frac{1}{c}d_X(x,y)-b&\leq d_Y(f(x),f(y))\\
                &\leq d_Y(g(x),g(y))+2k\\
                \Rightarrow  \frac{1}{c}d_X(x,y)-(b+2k)&\leq d_Y(g(x),g(y))+2k\\
            \end{split}
        \end{equation*}
        y, 
        \begin{equation*}
            \begin{split}
                d_Y(g(x),g(y))-2k&\leq d_Y(f(x),f(y))\\
                &\leq cd_X(x,y)+b\\
                \Rightarrow d_Y(g(x),g(y))&\leq cd_X(x,y)+(b+2k)\\
            \end{split}
        \end{equation*}
        para todo $x,y\in X$. Por tanto, $g$ es un $(c,b+2k)$-encaje quasi-isométrico. 
    \end{proof}

    \begin{excer}
        Toda función a distancia finita de una quasi-isometría es una quasi-isometría.
    \end{excer}

    \begin{proof}
        Sean $\cf{f,g}{X}{Y}$ funciones tales que $f$ es quasi-isometría y,
        \begin{equation*}
            d_Y(f(x),g(x))\leq k_2,\quad\forall x\in X
        \end{equation*}
        para algún $k_1\in\bbm{R}_{\geq0}$. Como $f$ es quasi-isometría, por un teorema $f$ es encaje quasi-isométrico y tiene imagen quasi densa. Veamos que $g$ también lo cumple. En efecto, por el ejercicio anterior $g$ es encaje quasi-isométrico. Veamos que tiene imagen quasi-densa.

        Sea $y\in Y$, entonces por tener $f$ imagen quasi-densa, existe $x\in X$ y $k_2\in\bbm{R}_{\geq0}$ tales que:
        \begin{equation*}
            d_Y(y,f(x))\leq k
        \end{equation*}
        por tanto:
        \begin{equation*}
            d_Y(y,g(x))\leq d_Y(y,f(x))+d(f(x),g(x))\leq k_1+k_2
        \end{equation*}
        donde $k_1,k_2\in\bbm{R}_{\geq0}$. Así que $g$ tiene imagen quasi-densa.

        Por un teorema se sigue que $g$ es quasi-isometría.
    \end{proof}

    \begin{excer}
        Sean $X,Y,Z$ espacios métricos y sean $\cf{f,f'}{X}{Y}$ funciones que están a distancia finita entre ellas.
        \begin{enumerate}[label = \textit{(\alph*)}]
            \item Si $\cf{g}{Z}{X}$ es función, entonces $f\circ g$ y $f'\circ g$ están a distancia finita entre sí.
            \item Si $\cf{g}{Y}{Z}$ es un encaje quasi-isométrico, entonce $g\circ f$ y $g\circ f'$ también están a distancia finita entre sí.
        \end{enumerate}
    \end{excer}

    \begin{proof}
        De \textit{(a)}: Suponga que $g$ es función. Como $f$ y $f'$ están a distancia finita entre ellas, existe $k\in\bbm{R}_{\geq0}$ tal que:
        \begin{equation*}
            d_Y(f(x),f'(x))\leq k,\quad\forall x\in X
        \end{equation*}
        por ende, para todo $z\in Z$:
        \begin{equation*}
            d_Y(f\circ g(z),f'\circ g(z))=d_Y(f(g(z)),f'(g(z)))\leq k
        \end{equation*}
        así que $f\circ g$ y $f'\circ g$ están a distancia finita entre ellas.

        De \textit{(b)}: Suponga que $g$ es $(c,d)$-encaje quasi-isométrico, entonces:
        \begin{equation*}
            \frac{1}{c}d_Y(y_1,y_2)-b\leq d_Z(g(y_1),g(y_2))\leq cd_Y(y_1,y_2)+b,\quad\forall y_1,y_2\in Y
        \end{equation*}
        entonces, se tiene que:
        \begin{equation*}
            d_Z(g\circ f(x),g\circ f'(x))=d_Z(g(f(x)),g(f'(x)))\leq cd_Y(f(x),f'(x))+b,\quad\forall x\in X
        \end{equation*}
        Como $f$ y $f'$ están a distancia finita entre ellas, existe $k\in\bbm{R}_{\geq0}$ tal que:
        \begin{equation*}
            d_Y(f(x),f'(x))\leq k,\quad\forall x\in X
        \end{equation*}
        así que:
        \begin{equation*}
            d_Z(g\circ f(x),g\circ f'(x))\leq ck+b
        \end{equation*}
        donde $ck+b\in\bbm{R}_{\geq0}$. Por tanto, las funciones $g\circ f$ y $g\circ f'$ están a distancia finita entre ellas.
    \end{proof}

    \begin{excer}
        La composición de encajes quasi-isométricos son encajes quasi-isométricos.    
    \end{excer}

    \begin{proof}
        Sean $X,Y,Z$ espacios métricos y $\cf{f}{X}{Y}$, $\cf{g}{Y}{Z}$ $(c,b)$ y $(c',b')$ encajes quasi-isométricos, respectivamente, es decir:
        \begin{equation*}
            \frac{1}{c}d_X(x_1,x_2)-b\leq d_Y(f(x_1),f(x_2))\leq cd_X(x_1,x_2)+b,\quad\forall x_1,x_2\in X
        \end{equation*}
        y,
        \begin{equation*}
            \frac{1}{c'}d_Y(y_1,y_2)-b'\leq d_Z(g(y_1),g(y_2))\leq c'd_Y(y_1,y_2)+b',\quad\forall y_1,y_2\in Y
        \end{equation*}
        Veamos que la composición también es encaje quasi-isométrico. En efecto, sean $x_1,x_2\in X$, se tiene que:
        \begin{equation*}
            \begin{split}
                d_Z(g\circ f(x_1),g\circ f(x_2))&=d_Z(g(f(x_1)),g(f(x_2)))\\
                &\leq c'd_Y(f(x_1),f(x_2))+b'\\
                &\leq (c'c)d_X(x_1,x_2)+(b+b')\\
            \end{split}
        \end{equation*}
        y, de forma análoga:
        \begin{equation*}
            \begin{split}
                d_Z(g\circ f(x_1),g\circ f(x_2))&=d_Z(g(f(x_1)),g(f(x_2)))\\
                &\geq \frac{1}{c'}d_Y(f(x_1),f(x_2))-b'\\
                &\geq \frac{1}{c'c}d_X(x_1,x_2)-(b+b')\\
            \end{split}
        \end{equation*}
        para todo $x_1,x_2\in X$. Por tanto:
        \begin{equation*}
            \frac{1}{c'c}d_X(x_1,x_2)-(b+b')\leq d_Z(g\circ f(x_1),g\circ f(x_2))\leq(c'c)d_X(x_1,x_2)+(b+b'),\quad\forall x_1,x_2\in X
        \end{equation*}
        así que, $g\circ f$ es un $(cc',b+b')$-encaje quasi-isométrico.
    \end{proof}

    \begin{excer}
        La composición de quasi-isometrías son quasi-isometrías.    
    \end{excer}

    \begin{proof}
        Sean $X,Y,Z$ espacios métricos y, $\cf{f}{X}{Y}$ y $\cf{g}{Y}{Z}$ quasi-isometrías, en particular por un teorema, estas son encajes quasi-isométricos y tienen imágenes quasi-densas.

        Por ser encajes quasi-isométricos, del ejercicio anterior se sigue que $\cf{g\circ f}{X}{Z}$ es un encaje quasi-isométrico. Veamos que tiene imagen quasi-densa. Como $f$ y $g$ tienen imagen quasi-densa, existen $k_1,k_2\in\bbm{R}_{\geq}$ tales que:
        \begin{equation*}
            \forall y\in Y\exists x\in X (d_Y(y,f(x))\leq k_1)
        \end{equation*}
        y,
        \begin{equation*}
            \forall z\in Z\exists y\in Y( d_Z(z,g(y))\leq k_2)
        \end{equation*}
        Por tanto, para $z\in Z$ existe $y\in Y$ tal que:
        \begin{equation*}
            d_Z(z,g(y))\leq k_2
        \end{equation*}
        y, para $y\in Y$ existe $x\in X$ tal que
        \begin{equation*}
            d_Y(y,f(x))\leq k_1
        \end{equation*}
        Así que,
        \begin{equation*}
            \begin{split}
                d_Z(z,g\circ f(x))&=d_Z(z,g(f(x)))\\
                &\leq d_Z(z,g(y))+d_Z(g(y),g(f(x)))\\
                &\leq k_1+\left(\frac{1}{c'}d_Y(y,f(x))+b'\right)\\
                &\leq k_1+\frac{k_2}{c'}+b'\\ 
            \end{split}
        \end{equation*}
        donde $k_1+\frac{k_2}{c'}+b'\in\bbm{R}_{\geq0}$ y siendo $g$ un $(c',b')$-encaje quasi-isométrico.

        Por ende, $g\circ f$ tiene imagen quasi-densa. Por un teorema anterior se sigue que $g\circ f$ es quasi-isometría.
    \end{proof}
    
    \section{Gráfica de Caley y Acciones de Grupos}

    \begin{excer}
        Demuestra que el grupo libre $F_n$ actúa libremente en su gráfica de Caley. Además, demuestra que todo subgrupo de $F_n$ es un grupo libre.
    \end{excer}

    \begin{proof}
        Considere el grupo libre $F_n$:
        \begin{equation*}
            F_n=\gen{x_1,...,x_n}
        \end{equation*}
        Tomemmos $S=\left\{x_1,...,x_n\right\}$. Así que $\Cay{F_n,S}$ es la gráfica de Caley de $F_n$. Veamos que la acción:
        \begin{equation*}
            (g,h)\mapsto gh
        \end{equation*}
        de $G$ en la gráfica $\Cay{F_n,S}$ es libre. Por una proposición basta con ver que el grupo $G$ no contiene involuciones, es decir que:
        \begin{equation*}
            s^2\neq e,\quad\forall s\in S
        \end{equation*}
        En efecto, si tal cosa ocurriese, existiría $i=1,...,n$ tal que:
        \begin{equation*}
            x_i^2=e\Rightarrow x_i=x_i^{-1}
        \end{equation*}
        pero, como $F_n$ es grupo libre en los elementos $x_1,..,x_n$, forzosamente se tiene que:
        \begin{equation*}
            x_i\neq x_i^{-1}
        \end{equation*}
        por tanto $x_i=x_i^{-1}$ y $x_i\neq x_i^{-1}$\contradiction. Así que $s^2\neq e$ para todo $s\in S$, es decir que $F_n$ no tiene involuciones, por lo que la acción de $F_n$ en $\Cay{F_n,S}$ es libre.

        Para la otra parte, sea $H$ un subgrupo de $F_n$. Afirmamos que $H$ es libre. En efecto, recordemos que un grupo $G$ actúa de manera libre en un árbol si y sólo si $G$ es grupo libre.
        
        En particular, $F_n$ es grupo libre, por lo que $F_n$ actúa de manera libre en un árbol, digamos $T$.

        En particular, como $F_n$ actúa libre sobre $T$ se tiene que:
        \begin{equation*}
            \forall v\in V, g\cdot v\neq v
        \end{equation*}
        y,
        \begin{equation*}
            \forall\left\{v,w \right\}\in E, g\cdot\left\{v,w \right\}\neq \left\{v,w \right\}
        \end{equation*}
        para todo $g\in F_n\setminus\left\{e\right\}$. Por ser $H$ subgrupo de $F_n$, se sigue que:
        \begin{equation*}
            \forall v\in V, h\cdot v\neq v
        \end{equation*}
        y,
        \begin{equation*}
            \forall\left\{v,w \right\}\in E, g\cdot\left\{v,w \right\}\neq \left\{v,w \right\}
        \end{equation*}
        para todo $g\in H\setminus\left\{e\right\}$. Por tanto, $H$ actúa de manera libre en el árbol $T$, así que $H$ es grupo libre.
    \end{proof}

    \begin{excer}
        Demuestra que los siguientes subgrupos son cuasi-isométricos: $D_\infty$, $\bbm{Z}\times\bbm{Z}/2\bbm{Z}$ y $\bbm{Z}$.
    \end{excer}

    \begin{proof}
        Los grupos $D_\infty$, $\bbm{Z}\times\bbm{Z}/2\bbm{Z}$ y $\bbm{Z}$ son finitamente generados, por lo que su gráfic ade Caley es única hasta cuasi-isometrías.

        Primero, veamos que $D_\infty\qisom\bbm{Z}$. Como $D_\infty\cong\bbm{Z}/2\bbm{Z}*\bbm{Z}/2\bbm{Z}$, entonces ya se sabe por un ejercicio anterior que los conjuntos:
        \begin{equation*}
            S=\left\{x,y \right\}\subseteq\bbm{Z}/2\bbm{Z}*\bbm{Z}/2\bbm{Z} \quad\textup{y}\quad T=\left\{1\right\}\subseteq\bbm{z}
        \end{equation*}
        con $\bbm{Z}/2\bbm{Z}*\bbm{Z}/2\bbm{Z}=\gen{x,y\Big|x^2=y^2=1}$, son tales que:
        \begin{equation*}
            \Cay{D_\infty,S}\cong\Cay{\bbm{Z},T}
        \end{equation*}
        en particular, como son isomorfos como gráficas, son cuasi-isométricos, es decir que:
        \begin{equation*}
            \Cay{D_\infty,S}\qisom\Cay{\bbm{Z},T}
        \end{equation*}
        por lo que $D_\infty\qisom\bbm{Z}$.

        Ahora, veamos que $D_\infty\qisom\bbm{Z}\times\bbm{Z}/2\bbm{Z}$. Para ello, recordemos que:
        \begin{equation*}
            D_\infty=\gen{r,s\Big|s^2=1\textup{ y }srs=r^{-1}}
        \end{equation*}
        Por otro ejercicio, se sabe que los conjuntos:
        \begin{equation*}
            S=\left\{(1,0),(0,1)\right\}\subseteq\bbm{Z}\times\bbm{Z}/2\bbm{Z}\quad\textup{y}\quad T=\left\{r,s \right\}\subseteq D_\infty
        \end{equation*}
        son tales que:
        \begin{equation*}
            \Cay{D_\infty,T}\cong\Cay{\bbm{Z}\times\bbm{Z}/2\bbm{Z},S}
        \end{equation*}
        por tanto, al igual que en la parte anterior, se sigue que:
        \begin{equation*}
            \Cay{D_\infty,T}\qisom\Cay{\bbm{Z}\times\bbm{Z}/2\bbm{Z},S}
        \end{equation*}
        es decir que $D_\infty\qisom\bbm{Z}\times\bbm{Z}/2\bbm{Z}$.
    \end{proof}

    \begin{excer}
        Demuestra que un grupo infinito no puede ser quasi-isométrico a un grupo finito.
    \end{excer}

    \begin{proof}
        Sea $G$ un grupo infinito y $H$ un grupo finito. Como es finito y todos los grupos finitos son quasi-isométricos, se tiene que:
        \begin{equation*}
            H\qisom\gen{e}
        \end{equation*}
        por lo que existe uno conjunto $S_1\subseteq H$ tal que:
        \begin{equation*}
            \Cay{H,S_1}\qisom\Cay{\gen{e},\left\{e\right\}}
        \end{equation*}

        Si $G\qisom H$, entonces existirían conjuntos $T\subseteq G$ y $S_2\subseteq H$ tales que:
        \begin{equation*}
            \Cay{G,T}\qisom\Cay{H,S_2}
        \end{equation*}
        Por ser $H$ finitamente generado se sigue que su gráfica de Caley es única hasta quasi-isometrías, es decir que:
        \begin{equation*}
            \Cay{H,S_2}\qisom \Cay{H,S_1}
        \end{equation*}
        por tanto, usando la transitividad de la relación $\qisom$ se sigue que:
        \begin{equation*}
            \Cay{G,T}\qisom\Cay{\gen{e},\left\{e\right\}}
        \end{equation*}
        Así que existe un encaje cuasi-isométrico $\cf{f}{\Cay{\gen{e},\left\{e\right\}}}{\Cay{G,T}}$ con imagen cuasi-densa. La gráfica de $\Cay{\gen{e},\left\{e\right\}}$ consta de un solo punto y ninguna arista.

        Por tener imagen cuasi-densa existe $k\in\bbm{R}_{\geq0}$ tal que para todo $g\in G$ existe $e\in\gen{e}$ que cumple:
        \begin{equation*}
            d_G(g,f(e))\leq k
        \end{equation*}
        Sea $m=\lceil k\rceil+1$. Considere $g\in G$ y tomemos:
        \begin{equation*}
            n_g=\min\left\{n\in\bbm{N}\Big|g=t_1^{\epsilon_1}\cdots t_n^{\epsilon_n};t_i\in T, \epsilon_i\in\left\{-1,1\right\} \right\}
        \end{equation*}
        Este mínimo existe ya que el conjunto es no vacío (pues $G$ es generado por $T$). Veamos que:
        \begin{equation*}
            \sup\left\{n_g\in\bbm{N}\Big|g\in G \right\}=\infty
        \end{equation*}
        En efecto, en caso contrario de que fuese finito, digamos $N$, todo elemento de $G$ estaría en el conjunto:
        \begin{equation*}
            G=\left\{t_1^{\epsilon_1}\cdot t_l^{\epsilon_l}\Big|t_i\in T, \epsilon_i\in\left\{-1,1\right\},l\leq N \right\}
        \end{equation*}
        siendo el conjunto de la derecha un conjunto finito (por ser $T$ finito) y el de la izquierda infinito (ya que $G$ es infinito). Por ende, $\sup\left\{n_g\in\bbm{N}\Big|g\in G \right\}=\infty$. En particular, para $m\in\bbm{N}$ existe un elemento $g\in G$ tal que:
        \begin{equation*}
            n_g\geq m
        \end{equation*}
        es decir, que la distancia mínima de $e$ a $g$ es mayor o igual a $m$. Veamos que:
        \begin{equation*}
            d(g,f(e))\geq d(g,e)-d(e,f(e))\geq d(g,e)=n_g\geq m=\lceil k\rceil+1>k
        \end{equation*}
        por tanto:
        \begin{equation*}
            d(g,f(e))>k
        \end{equation*}
        pues, $d(e,f(e))\geq 0$. Por ende, para $g\in G$ no existe $x\in\gen{e}$ tal que:
        \begin{equation*}
            d(g,f(x))\leq k
        \end{equation*}
        lo cual es una contradicción ya que $f$ tiene imagen cuasi-densa\contradiction. Por tanto, $G$ no puede ser cuasi-isométrico a $H$.
    \end{proof}

    \begin{lema}[\textbf{Lema de Svarc-Milnor}]
        Sea $G$ un grupo actuando por isometrías en un espacio métrico (no vacío) propio y geodésico $(X,d)$. Además, supongamos que esta acción es propia y cocompacta. Entonces, $G$ es finitamente geneardo y, para todo $x\in X$, el mapeo $\cf{\varphi}{G}{X}$ dado por:
        \begin{equation*}
            g\mapsto \varphi_g(x)
        \end{equation*}
        es una cuasi-isometría.
    \end{lema}

    \begin{obs}
        Recordemos que:
        \begin{equation*}
            SL(2,\bbm{Z})=\left\{A\in\mathcal{M}_{ 2\times 2}(\bbm{Z})\Big|\det(A)=1 \right\}
        \end{equation*}
        
        Un espacio métrico propio es aquel en el que todas las bolas cerradas son compactas.
    \end{obs}

    \begin{mydef}
        Una acción $G\times X\rightarrow X$ de un grupo $G$ en un espacio toplógico es \textbf{propia} si para todo subconjunto compacto $B\subseteq X$, el conjunto:
        \begin{equation*}
            \left\{g\in G\Big|g\cdot B\cap B\neq\emptyset \right\}
        \end{equation*}
        es finito.
    \end{mydef}

    \begin{mydef}
        Una acción $G\times X\rightarrow X$ de un grupo $G$ en un espacio topológico $X$ es \textbf{cocompacta} si el espacio cociente $G/X$ es compacto respecto a la topología cociente. 
    \end{mydef}

    \begin{obs}
        Analicemos los elementos de $SL(2\bbm{Z})$. Si:
        \begin{equation*}
            \left[\begin{array}{cc}
                a & b \\
                c & d \\
            \end{array} \right]\in SL(2,\bbm{Z})
        \end{equation*}
        entonces, se tiene que $a,b,c,d\in\bbm{Z}$ son tales que:
        \begin{equation*}
            ad-bc=1
        \end{equation*}
        Por lo que es tal que el máximo común divisor de todos los elementos involucrados es $1$ (salvo $a$ y $d$, y $b$ y $c$). Una matriz en este espacio es generada por:
        \begin{equation*}
            \left[\begin{array}{cc}
                1 & a \\
                0 & 1 \\
            \end{array} \right],\left[\begin{array}{cc}
                1 & 0 \\
                b & 1 \\
            \end{array} \right]
        \end{equation*}
        con $a,b\in\bbm{Z}$. Por tanto, todas las matrices son de la forma:
        \begin{equation*}
            \left[\begin{array}{cc}
                1 & a \\
                0 & 1 \\
            \end{array} \right]\cdot \left[\begin{array}{cc}
                1 & 0 \\
                b & 1 \\
            \end{array} \right]=\left[\begin{array}{cc}
                1+ab & a \\
                b & 1 \\
            \end{array} \right]
        \end{equation*}
        en general, un elemento $A$ de $SL(2,\bbm{Z})$ es de la forma:
        \begin{equation*}
            \left[\begin{array}{cc}
                1+s_1\cdot s_n\cdot l_1\cdots l_m & s_1\cdot s_n \\
                l_1\cdots l_m & 1 \\
            \end{array} \right]
        \end{equation*}
        con $s_i,l_j\in\bbm{Z}$.
    \end{obs}

    \begin{excer}
        Para cada una de las siguientes acciones de grupos, nombra una de las condiciones del lema de Svarc-Milnor que se cumple y una que no:
        \begin{enumerate}[label = \textit{(\alph*)}]
            \item La acción de $SL(2,\bbm{Z})$ en $\bbm{R}^2$ dada por la multiplicación de matrices.
            \item La acción de $\bbm{Z}$ en $X=\left\{(r^3,s)\Big|r,s\in\bbm{Z} \right\}$ (con respecto a la métrica inducida de la métrica euclideana en $\bbm{R}^2$) que está dada por:
            \begin{equation*}
                \begin{split}
                    \bbm{Z}\times X&\rightarrow X\\
                    (n,(r^3,s))&\mapsto (r^3,n+s)\\
                \end{split}
            \end{equation*}
        \end{enumerate}
    \end{excer}

    \begin{sol}
        De \textit{(a)}: Considere la acción de $SL(2,\bbm{Z})$ en $\bbm{R}^2$ dada por:
        \begin{equation*}
            (A,(x_1,x_2))\mapsto A\cdot\left[\begin{array}{c}
                x_1 \\
                x_2 \\
            \end{array} \right]
        \end{equation*}
        El espacio $\bbm{R}^2$ con la topología usual es geodésico y propio. Además, la acción es por isometrías (ya que las transformaciones con determinante 1 preservan las distancias). Veamos si la acción es propia y cocompacta.

        El espacio cociente está dado por:
        \begin{equation*}
            \begin{split}
                SL(2,\bbm{Z})/\bbm{R}^2&=\left\{SL(2,\bbm{Z})\cdot(x_1,x_2)\Big|x=(x_1,x_2)\in\bbm{R}^2 \right\}
            \end{split}
        \end{equation*}
        el espacio $SL(2,\bbm{Z})/\bbm{R}^2$ tiene como generadores:
        \begin{equation*}
            \left[\begin{array}{cc}
                1 & a \\
                0 & 1 \\
            \end{array} \right],\left[\begin{array}{cc}
                1 & 0 \\
                b & 1 \\
            \end{array} \right],a,b\in\bbm{Z}
        \end{equation*}
        por lo que basta con ver que pasa con estos generadores. Se tiene que:
        \begin{equation*}
            \begin{split}
                \left[\begin{array}{cc}
                    1 & a \\
                    0 & 1 \\
                \end{array} \right]\cdot\left[\begin{array}{c}
                    x_1 \\
                    x_2 \\
                \end{array} \right]&=\left[\begin{array}{c}
                    x_1+ax_2 \\
                    x_2 \\
                \end{array} \right]
            \end{split}
        \end{equation*}
        y,
        \begin{equation*}
            \begin{split}
                \left[\begin{array}{cc}
                    1 & 0 \\
                    b & 1 \\
                \end{array} \right]\cdot\left[\begin{array}{c}
                    x_1 \\
                    x_2 \\
                \end{array} \right]&=\left[\begin{array}{c}
                    x_1 \\
                    bx_1+x_2 \\
                \end{array} \right]
            \end{split}
        \end{equation*}
        para todo $a,b\in\bbm{Z}$ y para todo $(x_1,x_2)\in\bbm{R}^2$, en particular para $(0,0)$ toda transformación de este tipo lo deja invariante. Veamos que la acción no es propia. Tomemos el compacto $B=B(0,1)$, se tiene que:
        \begin{equation*}
            (0,0)\in A\cdot B\cap B,\quad\forall A\in SL(2,\bbm{Z})
        \end{equation*}
        ya que toda transformación de $SL(2,\bbm{Z})$ deja invariante al origen, por lo que el conjunto:
        \begin{equation*}
            \left\{A\in SL(2,\bbm{Z})\Big|A\cdot B\cap B\neq\emptyset \right\}=SL(2,\bbm{Z})
        \end{equation*}
        no es finito.

        De \textit{(b)}: Se tiene que el espacio $X$ es geodésico (con la métrica inducida de $\bbm{R}^2$) y es propio. Además, la acción es por isometrías (pues la traslación preserva la distancia). Veamos si la acción es propia y cocompacta.

        El espacio cociente está dado por:
        \begin{equation*}
            \begin{split}
                \bbm{Z}/X&=\left\{\bbm{Z}\cdot x\Big|x\in X \right\}\\
                &=\left\{\bbm{Z}\cdot (r^3,s)\Big|r,s\in\bbm{Z} \right\}\\
                &=\left\{(r^3,\bbm{Z}+s)\Big|r,s\in\bbm{Z} \right\}\\
                &=\left\{(r^3,\bbm{Z})\Big|r\in\bbm{Z} \right\}\\
                &\cong\bbm{Z}\\
            \end{split}
        \end{equation*}
        donde:
        \begin{equation*}
            (r^3,\bbm{Z})=\left\{(r^3,n)\Big|n\in\bbm{Z} \right\}
        \end{equation*}
        Por tanto, $\bbm{Z}/X$ no es compacto ya que $\bbm{Z}/X\cong\bbm{Z}$, el cual no es compacto. Así que la acción no es cocompacta.
    \end{sol}

    \section{Realización geométrica de gráficas}

    El objetivo de la realización geométrica de una gráfica es asociar a una gráfica un espacio métrico, para poder usar propiedades de las geodésicas y cuasi-geodésicas.

    La idea, hablando en términos poco técnicos y muy intuitivamente, es pegar un segmento unitario a dos vértices de una gráfica que se encuentren unidos por una arista. Un punto técnico a considerar es que el intervalo $[0,1]$ es dirigido, para lo cual en la construcción se toman estos dos intervalos dirigidos (de un vértice al otro y viceversa) y se identifican como si fuesen el mismo.

    \begin{mydef}
        Sea $X=(V,E)$ una gráfica conexa. La \textbf{realización geométrica de $X$} es el espacio métrico:
        \begin{equation*}
            (\abs{X},d_{\abs{X}})
        \end{equation*}
        definido como sigue:
        \begin{itemize}
            \item Si $E=\emptyset$, entonces al ser $X$ conexo se tiene que $\abs{V}\leq 1$; en este caso tomamos $\abs{X}=V$ y $d_{\abs{X}}:=0$.
            \item Si $E\neq\emptyset$,entonces todo vértice de $X$ cae en alguna arista, por ser $X$ conexo. Definimos así:
            \begin{equation*}
                \abs{X}:=\widetilde{E}\times[0,1]/\sim
            \end{equation*}
            siendo:
            \begin{equation*}
                \widetilde{E}:=\left\{(u,v)\Big|u,v\in V, \left\{u,v\right\}\in E \right\}
            \end{equation*}
            el conjunto de todas las aristas dirigidas. En este caso, para cada arista que no está orientada se tiene que $\left\{u,v\right\}=\left\{v,u\right\}$ obtenemos dos aristas dirigidas $(u,v)$ y $(v,u)$. La relación $\sim$ está dada como sigue: para cualesquiera dos elementos $((u,v),t)\sim((u',v').t')$ se tiene:
            \begin{itemize}
                \item Los elementos coinciden, es decir que $((u,v),t)=((u',v'),t')$.
                \item Los elementos describen el mismo vértice que yace sobre ambas aristas, esto es:
                
                \begin{minipage}{\textwidth}
                    \begin{center}
                        \includegraphics[scale=0.25]{images/relation_clara.png}\\
                        Figura \thefigcount. Ejemplo de algunas equivalencias.
                        \stepcounter{figcount}
                    \end{center}
                \end{minipage}

                \begin{itemize}
                    \item $u=u'$ y $t=t'=0$ (puede que $v$ y $v'$ sean diferentes), o
                    \item $u=v'$, $t=0$ y $t'=1$, o
                    \item $v=v'$ y $t=1=t'$, o
                    \item $v=u'$., $t=1$ y $t'=0$, o
                    \item Los elementos describen el mismo punto en la arista pero usando diferentes orientaciones, esto es que $(u,v)=(v',u')$ y $t=1-t'$.
                \end{itemize}
            \end{itemize}
            La métrica $d_{\abs{X}}$ está dada por:
            \begin{equation*}
                \begin{split}
                    &d_{\abs{X}}\left(\left[((u,v),t)\right],\left[((u',v'),t')\right] \right)\\
                    &=\left\{
                        \begin{array}{lcr}
                            \abs{t-t'} & \textup{ si } & (u,v)=(u',v')\\
                            \abs{t-(1-t')} & \textup{ si } & (u,v)=(v',u')\\
                            \min\left(
                                \begin{split}
                                    &t+d_X(u,u')+t',\\
                                    &t+d_X(u,v')+1-t',\\
                                    &1-t+d_X(v,u')+t',\\
                                    &1-t+d_X(v,v')+1-t'\\ 
                                \end{split}
                            \right) & \textup{ si } & \left\{u,v \right\}\neq\left\{v,u \right\}\\
                        \end{array}
                    \right.
                \end{split}
            \end{equation*}
            para todo $\left[((u,v),t)\right],\left[((u',v'),t')\right]\in \abs{X}$, en este caso $d_X$ es la métrica en $V$ inducida por la estructura de gráfica.
        \end{itemize}
    \end{mydef}

    \begin{exa}
        La realización geométrica de la gráfica consistente en dos vértices y una arista uniéndolas es isométrica al intervalo unitario.
    \end{exa}

    \begin{exa}
        La realización geométrica de $\Cay{\bbm{Z},\left\{1\right\}}$ es isométrico a la recta real con la métrica estándar.
    \end{exa}

    \begin{exa}
        La realización geométrica de $\Cay{\bbm{Z}^2,\left\{(1,0),(0,1) \right\}}$ es isométrica a la estructura de lattice $\left(\bbm{R}\times\bbm{Z}\right)\cup\left(\bbm{R}\times\bbm{Z}\right)\subseteq\bbm{R}^2$ con la métrica inducida de la métrica $l^1$ en $\bbm{R}^2$.
    \end{exa}

    \begin{minipage}{\textwidth}
        \begin{center}
            \includegraphics[scale=0.3]{images/realizaciones.png}\\
            Figura \thefigcount. Realizaciones geométricas de los ejemplos anteriores.
            \stepcounter{figcount}
        \end{center}
    \end{minipage}    

    \begin{excer}
        Haga lo siguiente:
        \begin{enumerate}[label = \textit{(\arabic*)}]
            \item Demuestra que la realización geométrica de una gráfica conexa es un espacio métrico geodésico.
            \item Sea $X=(V,E)$ una gráfica conexa. Demuestra que el mapa canónico $V\mapsto\abs{X}$ es un encaje cuasi-isométrico (donde equipamos a $V$ con la métrica inducida por la estructura de gráfica de $X$).
            \item Demuestra que todo espacio métrico cuasi-geodésico es cuasi-isométrico a un espacio métrico geodésico.
        \end{enumerate}
    \end{excer}

    \begin{proof}
        De \textit{(2)}: Considere el mapeo canónico $V\rightarrow\abs{X}$ dado por:
        \begin{equation*}
            v\mapsto ((v,w),0)
        \end{equation*}
        donde $w_v$ es un vértice de $V$ tal que $\left\{v,w_v \right\}\in E$. Afirmamos que esto es un $(1,1)$-encaje cuasi-isométrico. En efecto, el mapeo está bien definido (por como se definieron las clases de equivalencia). Si $v'\in V$ es otro vértice, se tiene que: 
    \end{proof}

    \begin{mydef}[\textbf{Conmensurabilidad}]
        Sean $G$ y $H$ grupos.
        \begin{itemize}
            \item Decimos que $G$ y $H$ son \textbf{conmensurables} si contienen subgrupos de índice finito $G'\subseteq G$ y $H'\subseteq H$ tales que $G'\cong H'$.
            \item Dos grupos $G$ y $H$ son \textbf{débilmente conmensurables} si contienen subgrupos de índice finito $G'\subseteq G$ y $H'\subseteq H$ tales que cumplen la siguiente condición: existen subgrupos normales finitos $N$ de $G'$ y $M$ de $H'$ tales que:
            \begin{equation*}
                G'/N\cong H'/M
            \end{equation*}
        \end{itemize}
    \end{mydef}

    \begin{excer}
        Sea $G$ un grupo.
        \begin{enumerate}[label = \textit{(\arabic*)}]
            \item Sea $G'$ un subgrupo de índice finito de $G$. Entonces, $G'$ es finitamente generado si y sólo si $G$ es finitamente generado. Si estos grupos son finitamente generados, entonces $G\qisom G'$.
            \item Sea $N$ un subgrupo normal finito. Entonces, $G/N$ es finitamente generado si y sólo si $G$ es finitamente generado. Si estos grupos son finitamente generados, entonces $G\qisom G/N$.
        \end{enumerate}
        En particular, si $G$ es finitamente generado, entonces todo grupo débilmente conmensurable de $G$ es finitamente generado y cuasi-isométrico a $G$.
    \end{excer}

    \begin{proof}
        De \textit{(1)}: Si $G$ es finitamente generado, entonces como $[G:G']<\infty$ se sigue de un ejercicio anterior que $G'$ es finitamente generado.

        Suponga que $G'$ es finitamente generado. Considere:
        \begin{equation*}
            G/_DG'=\left\{G'g\Big|g\in G \right\}
        \end{equation*}
        este conjunto es finito pues $[G:G']<\infty$. Así que existen $g_1,...,g_n\in G$ tales que:
        \begin{equation*}
            G/_DG'=\left\{G'g_1,...,G'g_n \right\}
        \end{equation*}
        Al ser $G'$ finitamente generado, existen $g_1',...,g_m'\in G$ tales que:
        \begin{equation*}
            G'=\gen{g_1',...,g_m'}
        \end{equation*}
        Veamos que $G=\gen{g_1,...,g_n,g_1',...,g_m'}$. En efecto, sea $g\in G$, entonces existe $i=1,...,n$ tal que:
        \begin{equation*}
            g\in G'g_i
        \end{equation*}
        luego, existe $g'\in G$ tal que:
        \begin{equation*}
            g=g'g_i
        \end{equation*}
        como $g'\in G'$, existen $s_1,...,s_k\in\left\{g_1',...,g_m' \right\}$ $\epsilon_1,...,\epsilon_k\in\left\{-1,1\right\}$ tales que:
        \begin{equation*}
            g'=g_1'^{\epsilon_1}\cdots g_k'^{\epsilon_k}
        \end{equation*}
        por lo cual:
        \begin{equation*}
            g=g_1'^{\epsilon_1}\cdots g_k'^{\epsilon_k}g_i
        \end{equation*}
        así que $g\in\gen{g_1,...,g_n,g_1',...,g_m'}$. Por ende, $G$ es finitamente generado.

        Para la última parte, si estos dos grupos son finitamente generados, en particular $G$ es finitamente generado, luego por el Corolario al Teorema de Svarc Milnor, se sigue que $G\qisom G'$.

        De \textit{(2)}: Suponga que $G/N$ es finitamente generado, entonces existen $g_1,...,g_n\in G$ tales que:
        \begin{equation*}
            G/N=\gen{Ng_1,...,Ng_n}
        \end{equation*}
        Sea $S=N\cup\left\{g_1,...,g_n\right\}$. Este conjunto es finito por ser $N$ finito. Afirmamos que $G=\gen{S}$. En efecto, sea $g\in G$, entonces existe $i=1,...,n$ tal que:
        \begin{equation*}
            g\in Ng_i
        \end{equation*}
        luego, existe $n\in N$ tal que:
        \begin{equation*}
            g=ng_i
        \end{equation*}
        por ende $g\in\gen{S}$.

        Suponga ahora que $G$ es finitamente generado, entonces existen $g_1,...,g_n\in G$ tales que:
        \begin{equation*}
            G=\gen{g_1,...,g_n}
        \end{equation*}
        Afirmamos que:
        \begin{equation*}
            G/N=\gen{Ng_1,...,Ng_n }
        \end{equation*}
        En efecto, tomemos una clase lateral de $G/N$, digamos $Ng$ con $g\in G$. Se tiene que existen $s_1,...,s_k\in G$ y $\epsilon_1,...,\epsilon_k\in\left\{-1,1\right\}$ tales que:
        \begin{equation*}
            g=s_1^{\epsilon_1}\cdots s_k^{\epsilon_k}
        \end{equation*}
        por lo cual:
        \begin{equation*}
            Ng=(Ng_1)^{\epsilon_1}\cdots (Ng_k)^{\epsilon_k}
        \end{equation*}
        con lo que $Ng\in\gen{Ng_1,...,Ng_n }$. Así que $G/N$ es finitamente generado.

        Para la última parte, si $G$ es finitamente generado se tiene que si $H$ débilmente conmensurable con $G$, entonces existen $G'\subseteq G$ y $H'\subseteq H$ subgrupos respectivos con índice finito tales que:
        \begin{equation*}
            G'\cong H'
        \end{equation*}
        En particular, $G'$ es finitamente generado por \textit{(1)}, con lo que $H'$ es finitamente generado y de índice finito con $H$, así que $H$ es finitamente generado. Además, por la segunda parte de \textit{(1)} se sigue que:
        \begin{equation*}
            G\qisom G'\quad\textup{y}\quad H\qisom H'
        \end{equation*}
        y, como $G'\cong H'$ entonces $G'\qisom H'$. Por tanto, $G\qisom H$.
    \end{proof}

    \begin{excer}
        Demuestra la siguiente afirmación: Sea
        \begin{equation*}
            1\rightarrow F\rightarrow G\rightarrow K\rightarrow1
        \end{equation*}
        una sucesión exacta corta de grupos finitamente generados. Si $F$ es finito, entonces $G$ es cuasi-isométrico a $K$.
    \end{excer}

    \begin{proof}
        Se tiene que:
        \begin{equation*}
            K\cong G/\im{F}
        \end{equation*}
        (viendo a $\im{F}$ como subgrupo de $G$). Entonces, $\im{F}$ es finito y normal en $G$, el cual es finitamente generado. Por el inciso \textit{(2)} del ejercicio anterior se sigue que $G/\im{F}$ es finitamente generado, luego $G\qisom G/\im{F}$. Por ende, como $K\cong G/\im{F}$ entonces $K\qisom G/\im{F}$ con lo que se sigue que:
        \begin{equation*}
            G\qisom K
        \end{equation*} 
    \end{proof}

\end{document}