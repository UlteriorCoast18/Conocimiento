\documentclass[12pt]{report}
\usepackage[spanish]{babel}
\usepackage[utf8]{inputenc}
\usepackage{amsmath}
\usepackage{amssymb}
\usepackage{amsthm}
\usepackage{graphics}
\usepackage{subfigure}
\usepackage{lipsum}
\usepackage{array}
\usepackage{multicol}
\usepackage{enumerate}
\usepackage[framemethod=TikZ]{mdframed}
\usepackage[a4paper, margin = 1.5cm]{geometry}

%En esta parte se hacen redefiniciones de algunos comandos para que resulte agradable el verlos%

\def\proof{\paragraph{Demostración:\\}}
\def\endproof{\hfill$\square$}
\renewcommand{\theenumii}{\roman{enumii}}

%En esta parte se definen los comandos a usar dentro del documento para enlistar%

\newtheoremstyle{largebreak}
  {}% use the default space above
  {}% use the default space below
  {\normalfont}% body font
  {}% indent (0pt)
  {\bfseries}% header font
  {}% punctuation
  {\newline}% break after header
  {}% header spec

\theoremstyle{largebreak}

\newmdtheoremenv[
    leftmargin=0em,
    rightmargin=0em,
    innertopmargin=-2pt,
    innerbottommargin=8pt,
    hidealllines = true,
    roundcorner = 5pt,
    backgroundcolor = gray!60!red!30
]{exa}{Ejemplo}[section]

\newmdtheoremenv[
    leftmargin=0em,
    rightmargin=0em,
    innertopmargin=-2pt,
    innerbottommargin=8pt,
    hidealllines = true,
    roundcorner = 5pt,
    backgroundcolor = gray!50!blue!30
]{obs}{Observación}[section]

\newmdtheoremenv[
    leftmargin=0em,
    rightmargin=0em,
    innertopmargin=-2pt,
    innerbottommargin=8pt,
    rightline = false,
    leftline = false
]{theor}{Teorema}[section]

\newmdtheoremenv[
    leftmargin=0em,
    rightmargin=0em,
    innertopmargin=-2pt,
    innerbottommargin=8pt,
    rightline = false,
    leftline = false
]{propo}{Proposición}[section]

\newmdtheoremenv[
    leftmargin=0em,
    rightmargin=0em,
    innertopmargin=-2pt,
    innerbottommargin=8pt,
    rightline = false,
    leftline = false
]{cor}{Corolario}[section]

\newmdtheoremenv[
    leftmargin=0em,
    rightmargin=0em,
    innertopmargin=-2pt,
    innerbottommargin=8pt,
    rightline = false,
    leftline = false
]{lema}{Lema}[section]

\newmdtheoremenv[
    leftmargin=0em,
    rightmargin=0em,
    innertopmargin=-2pt,
    innerbottommargin=8pt,
    roundcorner=5pt,
    backgroundcolor = gray!30,
    hidealllines = true
]{mydef}{Definición}[section]

\newmdtheoremenv[
    leftmargin=0em,
    rightmargin=0em,
    innertopmargin=-2pt,
    innerbottommargin=8pt,
    roundcorner=5pt
]{excer}{Ejercicio}[section]

%En esta parte se colocan comandos que definen la forma en la que se van a escribir ciertas funciones%

\newcommand\abs[1]{\ensuremath{\lvert#1\rvert}}
\newcommand\divides{\ensuremath{\bigm|}}

%recuerda usar \clearpage para hacer un salto de página

\begin{document}
    \title{Teoría de Grupos}
    \author{Cristo Daniel Alvarado}
    \maketitle

    \tableofcontents %Con este comando se genera el índice general del libro%

    %\setcounter{chapter}{3} %En esta parte lo que se hace es cambiar la enumeración del capítulo%
    
    \setcounter{chapter}{6}

    \chapter{Grupos de Sylow}
    
    \section{Los $p$-grupos}
    
    Cuando tenemos un grupo finito $G$, para analizar su estructura surgen problemas ya que, mientras más complicada sea su descomposición en primos, más complicado es entender su estructura (casi por norma general). Por ello, se estudiará en caso particular en que el orden de $G$ es una potencia de un número primo.

    \begin{propo}
        Sean $p$ un número primo, $G$ un grupo de orden $p^n$, con $n\in\mathbb{N}$, actuando $G$ sobre un conjunto no vacío. Entonces,
        \begin{equation*}
            |X|\equiv|X^G|\mod p
        \end{equation*}
    \end{propo}

    \begin{proof}
        Sea $R$ un conjunto completo de representantes bajo la relación de acción de $G$ sobre $X$. Claro que $X^G\subseteq R$, y por ende, la ecuación de clase generalizada está dada por:
        \begin{equation*}
            |X|=|X^G|+\sum_{x\in R\backslash X^G}|G\cdot x|
        \end{equation*}
        donde $p$ divide a $|G\cdot x|$, para cada $x\in R\backslash X^G$. Por tanto, se sigue que $|X|\equiv|X^G|\mod p$.
    \end{proof}

    A continuación, analizaremos un ejemplo que nos servirá para entender la teoría de más adelante.

    \begin{exa}
        Sean $G$ un grupo finito, $p$ un número primo, y $H$ un subgrupo de $G$ tal que $|H|=p^m$, donde $m\in\mathbb{Z}_{\geq0}$. Sea $X$ el conjunto
        \begin{equation*}
            X=\left\{gH| g\in G \right\}
        \end{equation*}
        es decir, $X$ es el conjunto de clases laterales izquierdas de $H$ en $G$. Tenemos que $H$ actúa sobre $X$ por traslación izquierda, dada la acción:
        \begin{equation*}
            h\cdot gH = hgH
        \end{equation*}
        para todo $h\in H$. Por definición, es inmediato que $\left[G:H\right]=|X|$, y además:
        \begin{equation*}
            X^H=\left\{gH|g\in N_G\left(H\right) \right\}
        \end{equation*}
        ya que, si $g\in N_G\left(H\right)$, entonces $gh=hg$, para todo $h\in H$. Luego:
        \begin{equation*}
            h\cdot gH = hgH=ghH=gH,\quad\forall h\in H
        \end{equation*}
        por tanto, $gH\in X^H$. De esta forma $|X^H|=[N_G(H):H]$. Por la proposición 7.1.1, se tiene que
        \begin{equation*}
            [G:H]\equiv[N_G(H):H]\mod p
        \end{equation*}
        en particular, si $p|[G:H]$, entonces $p|[N_G(H):H]$ lo cual implica que la contención de $H$ en $N_G(H)$ es propia.
    \end{exa}

    \begin{theor}[\textbf{Teorema de Cauchy}]
        Sean $p$ un número primo y $G$ un grupo finito tal que $p$ divide a $|G|$. Enotnces existe un elemento de orden $p$ en $G$.
    \end{theor}

    \begin{proof}
        
    \end{proof}

    Este teorema justifica la razón de la existencia de la siguiente definición y teorema.

    \begin{mydef}
        Sea $p$ un número primo. Todo grupo $G$ se dice que es \textbf{$p$-grupo}, si todo elemento de $G$ es del orden $p^t$, donde $t\in\mathbb{Z}_{\geq0}$.
    \end{mydef}

    

    \newpage

    \begin{theor}[Nombre]
        Teorema
    \end{theor}

    \begin{propo}[Nombre]
        Proposición
    \end{propo}

    \begin{cor}[Nombre]
        Corolario
    \end{cor}

    \begin{lema}[Nombre]
        Lema
    \end{lema}

    \begin{mydef}[Nombre]
        Definición
    \end{mydef}

    \begin{obs}[Nombre]
        Observación
    \end{obs}

    \begin{exa}[Nombre]
        Ejemplo
    \end{exa}

    \begin{excer}[Nombre]
        Ejercicio
    \end{excer}

\end{document}