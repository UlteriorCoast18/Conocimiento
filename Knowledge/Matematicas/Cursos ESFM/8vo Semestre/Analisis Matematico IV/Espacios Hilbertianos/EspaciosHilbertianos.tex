\documentclass[12pt]{report}
\usepackage[spanish]{babel}
\usepackage[utf8]{inputenc}
\usepackage{amsmath}
\usepackage{amssymb}
\usepackage{amsthm}
\usepackage{graphics}
\usepackage{subfigure}
\usepackage{lipsum}
\usepackage{array}
\usepackage{multicol}
\usepackage{enumerate}
\usepackage[framemethod=TikZ]{mdframed}
\usepackage[a4paper, margin = 1.5cm]{geometry}

%En esta parte se hacen redefiniciones de algunos comandos para que resulte agradable el verlos%

\renewcommand{\theenumii}{\roman{enumii}}
\renewcommand{\theenumi}{\arabic{enumi})}

\def\proof{\paragraph{Demostración:\\}}
\def\endproof{\hfill$\blacksquare$}

\def\sol{\paragraph{Solución:\\}}
\def\endsol{\hfill$\square$}

%En esta parte se definen los comandos a usar dentro del documento para enlistar%

\newtheoremstyle{largebreak}
  {}% use the default space above
  {}% use the default space below
  {\normalfont}% body font
  {}% indent (0pt)
  {\bfseries}% header font
  {}% punctuation
  {\newline}% break after header
  {}% header spec

\theoremstyle{largebreak}

\newmdtheoremenv[
    leftmargin=0em,
    rightmargin=0em,
    innertopmargin=-2pt,
    innerbottommargin=8pt,
    hidealllines = true,
    roundcorner = 5pt,
    backgroundcolor = gray!60!red!30
]{exa}{Ejemplo}[section]

\newmdtheoremenv[
    leftmargin=0em,
    rightmargin=0em,
    innertopmargin=-2pt,
    innerbottommargin=8pt,
    hidealllines = true,
    roundcorner = 5pt,
    backgroundcolor = gray!50!blue!30
]{obs}{Observación}[section]

\newmdtheoremenv[
    leftmargin=0em,
    rightmargin=0em,
    innertopmargin=-2pt,
    innerbottommargin=8pt,
    rightline = false,
    leftline = false
]{theor}{Teorema}[section]

\newmdtheoremenv[
    leftmargin=0em,
    rightmargin=0em,
    innertopmargin=-2pt,
    innerbottommargin=8pt,
    rightline = false,
    leftline = false
]{propo}{Proposición}[section]

\newmdtheoremenv[
    leftmargin=0em,
    rightmargin=0em,
    innertopmargin=-2pt,
    innerbottommargin=8pt,
    rightline = false,
    leftline = false
]{cor}{Corolario}[section]

\newmdtheoremenv[
    leftmargin=0em,
    rightmargin=0em,
    innertopmargin=-2pt,
    innerbottommargin=8pt,
    rightline = false,
    leftline = false
]{lema}{Lema}[section]

\newmdtheoremenv[
    leftmargin=0em,
    rightmargin=0em,
    innertopmargin=-2pt,
    innerbottommargin=8pt,
    roundcorner=5pt,
    backgroundcolor = gray!30,
    hidealllines = true
]{mydef}{Definición}[section]

\newmdtheoremenv[
    leftmargin=0em,
    rightmargin=0em,
    innertopmargin=-2pt,
    innerbottommargin=8pt,
    roundcorner=5pt
]{excer}{Ejercicio}[section]

%En esta parte se colocan comandos que definen la forma en la que se van a escribir ciertas funciones%

\newcommand\abs[1]{\ensuremath{\big|#1\big|}}
\newcommand\divides{\ensuremath{\bigm|}}
\newcommand\cf[3]{\ensuremath{#1:#2\rightarrow#3}}
\newcommand\norm[1]{\ensuremath{\|#1\|}}
\newcommand\ora[1]{\ensuremath{\vec{#1}}}
\newcommand\pint[2]{\ensuremath{\left(#1\big| #2\right)}}
\newcommand\conj[1]{\ensuremath{\overline{#1}}}

%recuerda usar \clearpage para hacer un salto de página

\begin{document}
    \title{Espacios Hilbertianos}
    \author{Cristo Daniel Alvarado}
    \maketitle

    \tableofcontents %Con este comando se genera el índice general del libro%

    \chapter{Espacios Hilbertianos}
    
    %apostol de análisis matemático, lang de análisis real

    \section{Conceptos básicos. Proyecciones ortogonales}

    \begin{mydef}
        Sea $H$ un espacio vectorial sobre el campo $\mathbb{K}$. Decimos que $H$ es un \textbf{espacio prehilbertiano} si está dotado de una aplicación $(\ora{x},\ora{y})\mapsto \pint{\ora{x}}{\ora{y}}$ con las propiedades siguientes:
        \begin{enumerate}
            \item $\forall \ora{y}\in H$ fijo, $\ora{x}\mapsto\pint{\ora{x}}{\ora{y}}$ es una aplicación lineal de $H$ en $\mathbb{K}$, o sea
            \begin{equation*}
                \begin{split}
                    \pint{\ora{x_1}+\ora{x_2}}{\ora{y}}=&\pint{\ora{x_1}}{\ora{y}}+\pint{\ora{x_2}}{\ora{y}}\\
                    \pint{\alpha\ora{x}}{\ora{y}}=&\alpha\cdot\pint{\ora{x}}{\ora{y}}\\
                \end{split}
            \end{equation*}
            para todo $\ora{x},\ora{x_1},\ora{x_2}\in H$ y $\alpha\in\mathbb{K}$.
            \item $(\ora{y}\big| \ora{x})=\conj{\pint{\ora{x}}{\ora{y}}}$, para todo $\ora{x}\in H$.
            \item $\pint{\ora{x}}{\ora{x}}\geq0$, para todo $\ora{x}\in H$.
            \item $\pint{\ora{x}}{\ora{x}}=0$ si y sólo si $\ora{x}=0$.
        \end{enumerate}
    \end{mydef}

    \begin{obs}
        Si $\mathbb{K}=\mathbb{R}$, entonces 1) y 2) implican que $\forall \ora{x}\in H$ fijo, la aplicación $\ora{y}\mapsto\pint{\ora{x}}{\ora{y}}$ de $H$ en $\mathbb{R}$ es lineal. En este caso se dice que $(\ora{x},\ora{y})\mapsto\pint{\ora{x}}{\ora{y}}$ es una \textbf{forma bilineal sobre $H$}.

        Si $\mathbb{K}=\mathbb{C}$, entonces
        \begin{equation*}
            \begin{split}
                \pint{\ora{x}}{\ora{y_1}+\ora{y_2}}=&\pint{\ora{x}}{\ora{y_1}}+\pint{\ora{x}}{\ora{y_2}}\\
                \pint{\ora{x}}{\alpha\ora{y}}=&\conj{\alpha}\pint{\ora{x}}{\ora{y}}\\
            \end{split}
        \end{equation*}
        Se dice que $\ora{y}\mapsto\pint{\ora{x}}{\ora{y}}$ es entonces \textbf{semilineal} y que $(\ora{x},\ora{y})\mapsto\pint{\ora{x}}{\ora{y}}$ es \textbf{sesquilineal} ($1\frac{1}{2}$-lineal).

        La aplicación $(\ora{x},\ora{y})\mapsto\pint{\ora{x}}{\ora{y}}$ se llama \textbf{producto escalar sobre $H$}.
    \end{obs}

    \begin{mydef}
        Para todo $\ora{x}\in H$ se define la \textbf{norma de $\ora{x}$} como: $\norm{\ora{x}}=\sqrt{\pint{\ora{x}}{\ora{x}}}$.
    \end{mydef}

    \begin{exa}
        Sea $H=\mathbb{K}^n$
        %El producto interior usual en ese espacio ya conocido y es prehilbertiano
    \end{exa}

    \begin{exa}
        Sea $S\subseteq\mathbb{R}^n$ medible y sea $H=L_2(S,\mathbb{K})$. Para todo $f,g\in H$ se define
        \begin{equation*}
            \pint{f}{g}=\int_Sf\conj{g}
        \end{equation*}
        La integral existe por Hölder con $p=p^*=2$. Este es un producto escalar sobre $H$ y, en este caso:
        \begin{equation*}
            \norm{f}=\left[\int_S\big|f\big|^2 \right]^{\frac{1}{2}}=\mathcal{N}_2(f),\quad \forall f\in H
        \end{equation*}
    \end{exa}

    \begin{exa}
        Sea $H=l_2(\mathbb{K})$ el espacio de sucesiones en $\mathbb{K}$ que son cuadrado sumables. Se sabe que $\ora{x}=(x_1,x_2,...)\in l_2(\mathbb{K})$ si y sólo si
        \begin{equation*}
            \sum_{i=1}^{\infty}|x_i|^2<\infty
        \end{equation*}
        $l_2(\mathbb{K})$ es un espacio prehilbertiano con el producto escalar:
        \begin{equation*}
            \pint{\ora{x}}{\ora{y}}=\sum_{i=1}^{\infty}x_i\conj{y_i}
        \end{equation*}
        donde la serie es convergente por Hölder. En este caso:
        \begin{equation}
            \norm{\ora{x}}=\left[\sum_{i=1}^{\infty}\abs{x_i}^2\right]^{\frac{1}{2}}=\mathcal{N}_2(\ora{x}),\quad\forall\ora{x}\in l_2(\mathbb{K})
        \end{equation}
    \end{exa}

    \newpage

    \begin{proof}
        Entorno de Prueba
    \end{proof}

    \begin{sol}
        Entorno de Solución
    \end{sol}

    \begin{theor}[Nombre]
        Teorema
    \end{theor}

    \begin{propo}[Nombre]
        Proposición
    \end{propo}

    \begin{cor}[Nombre]
        Corolario
    \end{cor}

    \begin{lema}[Nombre]
        Lema
    \end{lema}

    \begin{mydef}[Nombre]
        Definición
    \end{mydef}

    \begin{obs}[Nombre]
        Observación
    \end{obs}

    \begin{exa}[Nombre]
        Ejemplo
    \end{exa}

    \begin{excer}[Nombre]
        Ejercicio
    \end{excer}

\end{document}