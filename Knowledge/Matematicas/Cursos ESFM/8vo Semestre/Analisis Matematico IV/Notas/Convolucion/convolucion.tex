\documentclass[12pt]{report}
\usepackage[spanish]{babel}
\usepackage[utf8]{inputenc}
\usepackage{amsmath}
\usepackage{amssymb}
\usepackage{amsthm}
\usepackage{graphics}
\usepackage{subfigure}
\usepackage{lipsum}
\usepackage{array}
\usepackage{multicol}
\usepackage{enumerate}
\usepackage[framemethod=TikZ]{mdframed}
\usepackage[a4paper, margin = 1.5cm]{geometry}

%En esta parte se hacen redefiniciones de algunos comandos para que resulte agradable el verlos%

\renewcommand{\theenumii}{\roman{enumii}}

\def\proof{\paragraph{Demostración:\\}}
\def\endproof{\hfill$\blacksquare$}

\def\sol{\paragraph{Solución:\\}}
\def\endsol{\hfill$\square$}

%En esta parte se definen los comandos a usar dentro del documento para enlistar%

\newtheoremstyle{largebreak}
  {}% use the default space above
  {}% use the default space below
  {\normalfont}% body font
  {}% indent (0pt)
  {\bfseries}% header font
  {}% punctuation
  {\newline}% break after header
  {}% header spec

\theoremstyle{largebreak}

\newmdtheoremenv[
    leftmargin=0em,
    rightmargin=0em,
    innertopmargin=-2pt,
    innerbottommargin=8pt,
    hidealllines = true,
    roundcorner = 5pt,
    backgroundcolor = gray!60!red!30
]{exa}{Ejemplo}[section]

\newmdtheoremenv[
    leftmargin=0em,
    rightmargin=0em,
    innertopmargin=-2pt,
    innerbottommargin=8pt,
    hidealllines = true,
    roundcorner = 5pt,
    backgroundcolor = gray!50!blue!30
]{obs}{Observación}[section]

\newmdtheoremenv[
    leftmargin=0em,
    rightmargin=0em,
    innertopmargin=-2pt,
    innerbottommargin=8pt,
    rightline = false,
    leftline = false
]{theor}{Teorema}[section]

\newmdtheoremenv[
    leftmargin=0em,
    rightmargin=0em,
    innertopmargin=-2pt,
    innerbottommargin=8pt,
    rightline = false,
    leftline = false
]{propo}{Proposición}[section]

\newmdtheoremenv[
    leftmargin=0em,
    rightmargin=0em,
    innertopmargin=-2pt,
    innerbottommargin=8pt,
    rightline = false,
    leftline = false
]{cor}{Corolario}[section]

\newmdtheoremenv[
    leftmargin=0em,
    rightmargin=0em,
    innertopmargin=-2pt,
    innerbottommargin=8pt,
    rightline = false,
    leftline = false
]{lema}{Lema}[section]

\newmdtheoremenv[
    leftmargin=0em,
    rightmargin=0em,
    innertopmargin=-2pt,
    innerbottommargin=8pt,
    roundcorner=5pt,
    backgroundcolor = gray!30,
    hidealllines = true
]{mydef}{Definición}[section]

\newmdtheoremenv[
    leftmargin=0em,
    rightmargin=0em,
    innertopmargin=-2pt,
    innerbottommargin=8pt,
    roundcorner=5pt
]{excer}{Ejercicio}[section]

%En esta parte se colocan comandos que definen la forma en la que se van a escribir ciertas funciones%

\newcommand\divides{\ensuremath{\bigm|}}
\newcommand\cf[3]{\ensuremath{#1:#2\rightarrow#3}}
\newcommand\contradiction{\ensuremath{\#_c}}
\newcommand\abs[1]{\ensuremath{\big|#1\big|}}
\newcommand\norm[1]{\ensuremath{\|#1\|}}
\newcommand\ora[1]{\ensuremath{\vec{#1}}}
\newcommand\pint[2]{\ensuremath{\left(#1\big| #2\right)}}
\newcommand\conj[1]{\ensuremath{\overline{#1}}}
\newcommand{\N}[2]{\ensuremath{\mathcal{N}_{#1}\left(#2\right)}}
\newcommand{\GenSpace}[4]{\ensuremath{\mathcal{#1}_{#2}\left(#3,#4\right)}}

%recuerda usar \clearpage para hacer un salto de página

\begin{document}
    \setlength{\parskip}{5pt} % Añade 5 puntos de espacio entre párrafos
    \setlength{\parindent}{12pt} % Pone la sangría como me gusta
    \title{Notas de Análisis Matemático IV}
    \author{Cristo Daniel Alvarado}
    \maketitle

    \tableofcontents %Con este comando se genera el índice general del libro%

    \setcounter{chapter}{1} %En esta parte lo que se hace es cambiar la enumeración del capítulo%
 
    \chapter{Convolución}
    
    Se sabe que el producto puntual de dos funciones integrables no necesariamente es una función integrable (por ejemplo, $f(x)=g(x)=\frac{1}{\sqrt{x}}\chi_{]0,1[}$). Sin embargo, es posible definir un auténtico producto en $L_1(\mathbb{R}^n,\mathbb{K})$ que sea compatible con la adición y el producto por escalares, con el cual $L_1(\mathbb{R}^n,\mathbb{K})$ sea un \textbf{álgebra de Banach conmutativa sin elemento identidad}. Tal operación se llama \textbf{convolución}.

    \section{Preliminares}

    \begin{lema}
        Si $M$ es un subconjunto despreciable de $\mathbb{R}^n$, entonces $M\times\mathbb{R}^m$ es despreciable en $\mathbb{R}^{n+m}$.
    \end{lema}

    \begin{proof}
        Escriba a $\mathbb{R}^m$ como unión numerable de rectángulos acotados disjuntos. Basta probar que si $Q$ es un rectángulo acotado en $\mathbb{R}^m$, entonces $M\times Q$ es despreciable en $\mathbb{R}^{ n+m}$.

        Sea $\varepsilon>0$. Si $\textup{Vol}(Q)=0$, el resultado es inmediato, pues se sigue que $\textup{Vol}(P\times Q)=0$. Suponga que $\textup{Vol}(Q)>0$, se tiene para $M\subseteq\mathbb{R}^n$ que por definición de medida exterior existe $\left\{P_\nu \right\}_{\nu=1}^\infty$ sucesión de rectángulos acotados disjuntos tales que $M\subseteq \bigcup_{ \nu=1}^\infty P_\nu$ y:
        \begin{equation*}
            \sum_{ \nu=1}^{\infty}\textup{Vol}(P_\nu)<\frac{\varepsilon}{\textup{Vol}(Q)}
        \end{equation*}
        Entonces, $\left\{P_\nu\times Q \right\}_{\nu=1}^\infty$ es una sucesión de rectángulos acotados en $\mathbb{R}^{n+m}$ tales que $M\times Q\subseteq \bigcup_{ \nu=1}^\infty P_\nu\times Q$, y
        \begin{equation*}
            \begin{split}
                \sum_{ \nu=1}^{\infty}\textup{Vol}(P_\nu\times Q)&=\textup{Vol}(Q)\cdot\sum_{ \nu=1}^{\infty}\textup{Vol}(P_\nu)\\
                &<\textup{Vol}(Q)\cdot\frac{\varepsilon}{\textup{Vol}(Q)} \\
                &=\varepsilon\\
            \end{split}
        \end{equation*}
        luego, el conjunto $M\times Q$ es despreciable, con lo cual el conjunto $M\times\mathbb{R}^m$ también lo es.
    \end{proof}

    \begin{mydef}
        Si $\cf{f}{\mathbb{R}^p}{\mathbb{K}}$ y $\cf{g}{\mathbb{R}^q}{\mathbb{K}}$ son funciones, se define el \textbf{producto tensorial de $f$ y $g$} como la función: $\cf{f\otimes g}{\mathbb{R}^{ p+q}}{\mathbb{K}}$, dada por:
        \begin{equation*}
            f\otimes g(x,y)=f(x)g(y),\quad\forall (x,y)\in\mathbb{R}^{ p+q}
        \end{equation*}
    \end{mydef}

    \begin{propo}
        Si $\cf{f}{\mathbb{R}^p}{\mathbb{K}}$ y $\cf{g}{\mathbb{R}^q}{\mathbb{K}}$ son funciones medibles, entonces $\cf{f\otimes g}{\mathbb{R}^{ p+q}}{\mathbb{K}}$ es medible.
    \end{propo}

    \begin{proof}
        Se probarán dos casos:
        \begin{enumerate}
            \item Afirmamos que el resultado es cierto para funciones escalonadas $\cf{\varphi}{\mathbb{R}^p}{\mathbb{K}}$ y $\cf{\psi}{\mathbb{R}^q}{\mathbb{K}}$ escritas canónicamente como:
            \begin{equation*}
                \varphi=\sum_{ i=1}^{r}c_i\chi_{ P_i}\quad\textup{y}\quad\psi=\sum_{ j=1}^{s}d_j\chi_{ Q_j}
            \end{equation*}
            donde los $P_i$ y $Q_j$ son rectángulos acotados disjuntos. En efecto, en este caso:
            \begin{equation*}
                \begin{split}
                    \varphi\otimes \psi(x,y)&=\sum_{ i=1}^{r}\sum_{ j=1}^{s}c_id_j\chi_{ P_i}(x)\chi_{ Q_j}(y)\\
                    &=\sum_{ i=1}^{r}\sum_{ j=1}^{s}c_id_j\chi_{ P_i\times Q_j}(x,y)\\
                \end{split}
            \end{equation*}
            la cual es una función escalonada en $\mathbb{R}^{ p+q}$, luego medible.
            \item En el caso general, se sabe que existen $\left\{ \varphi_\nu\right\}_{ \nu=1}^\infty$ en $\mathcal{E}(\mathbb{R}^p,\mathbb{K})$ y $\left\{\psi_\nu \right\}_{ \nu=1}^\infty$ en $\mathcal{E}(\mathbb{R}^q,\mathbb{K})$ y conjuntos despreciables $M\subseteq \mathbb{R}^p$, $N\subseteq \mathbb{R}^q$ tales que:
            \begin{equation*}
                \lim_{ \nu\rightarrow\infty}\varphi_\nu(x)=f(x),\quad\forall x\in \mathbb{R}^p\backslash M
            \end{equation*}
            y,
            \begin{equation*}
                \lim_{ \nu\rightarrow\infty}\psi_\nu(x)=g(x),\quad\forall x\in \mathbb{R}^q\backslash N
            \end{equation*}
            luego, se tiene que:
            \begin{equation*}
                \begin{split}
                    \lim_{ \nu\rightarrow\infty}\varphi_\nu\otimes \psi_\nu(x,y)&=\lim_{ \nu\rightarrow\infty}\varphi_\nu(x)\psi_\nu(y)\\
                    &=f(x)g(y)\\
                \end{split}
            \end{equation*}
            para todo $(x,y)\in \mathbb{R}^{ p+q}\backslash\left[M\times\mathbb{R}^q\cup\mathbb{R}^p\times N \right]$. Por el lema anterior se tine que $M\times\mathbb{R}^q\cup\mathbb{R}^p\times N$ es despreciable en $\mathbb{R}^{ p+q}$. Como $\varphi_\nu\otimes \psi_\nu$ son medibles para todo $\nu\in\mathbb{N}$, entonces $f\otimes g$ es medible.
        \end{enumerate}
    \end{proof}

    \begin{cor}
        Si $\cf{f}{\mathbb{R}^p}{\mathbb{K}}$ es medible, entonces $\cf{F}{\mathbb{R}^{ p+q}}{\mathbb{K}}$ dada como:
        \begin{equation*}
            F(x,y)=f(x),\quad\forall (x,y)\in\mathbb{R}^{ p+q}
        \end{equation*}
        es medible.
    \end{cor}

    \begin{proof}
        Es inmediata de la proposición anterior tomando a $f$ y $g=\chi_{\mathbb{R}^q}$.
        
    \end{proof}

    \begin{cor}
        Si $f\in\mathcal{L}_1(\mathbb{R}^p,\mathbb{K})$, $g\in\mathcal{L}_1(\mathbb{R}^q,\mathbb{K})$, entonces $f\otimes g\in \mathcal{L}_1(\mathbb{R}^{p+q},\mathbb{K})$ y:
        \begin{equation*}
            \int_{\mathbb{R}^{p+q}}f\otimes g=\int_{\mathbb{R}^p}f\cdot\int_{\mathbb{R}^q}g
        \end{equation*}
    \end{cor}

    \begin{proof}
        Es inmediato del teorema de Tonelli.
    \end{proof}

    \section{Convolución}

    \begin{mydef}
        Sean $\cf{f,g}{\mathbb{R}^n}{\mathbb{K}}$ funciones medibles. La \textbf{convolución de $f$ por $g$} se define como la función de $\mathbb{R}^n$ en $\mathbb{K}$ tal que:
        \begin{equation*}
            f*g(x)=\int_{\mathbb{R}^n}f(y)g(x-y)dy
        \end{equation*}
        para toda $x\in\mathbb{R}^n$ tal que la integral exista.
    \end{mydef}

    \begin{exa}
        Considere la función:
        \begin{equation*}
            f(x)=\left\{ 
            \begin{array}{lcr}
                1 & \textup{ si } & 0\leq x\leq 1\\
                0 & & \textup{ en caso contrario} \\
            \end{array}
            \right.
        \end{equation*}
        y
        \begin{equation*}
            g(x)=\left\{ 
            \begin{array}{lcr}
                x & \textup{ si } & 0\leq x\leq 1\\
                0 & & \textup{ en caso contrario} \\
            \end{array}
            \right.
        \end{equation*}
        entonces,
        \begin{equation*}
            f*g(x)=\int_{- \infty}^\infty f(y)g(x-y)dx=\int_{0}^\infty f(y)g(x-y)dx
        \end{equation*}
        se tienen dos casos, por como están dadas las funciones $f$ y $g$:
        \begin{equation*}
            \begin{split}
                \int_{0}^\infty f(y)g(x-y)dx&=
                \left\{
                    \begin{array}{lcr}
                       0 & \textup{ si } & 0\leq x \\
                       \int_{0}^x f(y)g(x-y)dy & \textup{ si }& x>0\\
                    \end{array}
                \right. \\
                &=\left\{
                    \begin{array}{lcr}
                       0 & \textup{ si } & 0\leq x \\
                       \int_{0}^x f(y)g(x-y)dy & \textup{ si }& 0<x<1 \\
                       \int_{0}^1 f(y)g(x-y)dy & \textup{ si }& x\geq1\\
                    \end{array}
                \right. \\
                &=\left\{
                    \begin{array}{lcr}
                       0 & \textup{ si } & 0\leq x \\
                       \int_{0}^x g(x-y)dy & \textup{ si }& 0<x<1 \\
                       \int_{0}^1 g(x-y)dy & \textup{ si }& x\geq1\\
                    \end{array}
                \right. \\
                &=\left\{
                    \begin{array}{lcr}
                       0 & \textup{ si } & 0\leq x \\
                       \int_{0}^x g(x-y)dy & \textup{ si }& 0<x<1 \\
                       \int_{0}^1 g(x-y)dy & \textup{ si }& x\geq1\\
                    \end{array}
                \right. \\
                &=\left\{
                    \begin{array}{lcr}
                       0 & \textup{ si } & 0\leq x \\
                       \int_{0}^x g(x-y)dy & \textup{ si }& 0<x<1 \\
                       \int_{0}^1 g(x-y)dy & \textup{ si }& 1\leq x\leq 2 \\
                       \int_{0}^1 g(x-y)dy & \textup{ si }& x>2\\
                    \end{array}
                \right. \\
            \end{split}
        \end{equation*}

        \begin{equation*}
            \begin{split}
                \Rightarrow\int_{0}^\infty f(y)g(x-y)dx&=\left\{
                    \begin{array}{lcr}
                       0 & \textup{ si } & 0\leq x \\
                       \int_{0}^x (x-y) dy & \textup{ si }& 0<x<1 \\
                       \int_{x-1 }^1 g(x-y)dy & \textup{ si }& 1\leq x\leq 2 \\
                       0 & \textup{ si }& x>2\\
                    \end{array}
                \right. \\
                &=\left\{
                    \begin{array}{lcr}
                       0 & \textup{ si } & 0\leq x \\
                       -\frac{(x-y)^2}{2}\Big|_{0}^x & \textup{ si }& 0<x<1 \\
                       \int_{x-1 }^1 (x-y)dy & \textup{ si }& 1\leq x\leq 2 \\
                       0 & \textup{ si }& x>2\\
                    \end{array}
                \right. \\
                &=\left\{
                    \begin{array}{lcr}
                       0 & \textup{ si } & 0\leq x \\
                       \frac{x^2}{2} & \textup{ si }& 0<x<1 \\
                       -\frac{(x-y)^2}{2}\Big|_{ x-1}^1 & \textup{ si }& 1\leq x\leq 2 \\
                       0 & \textup{ si }& x>2\\
                    \end{array}
                \right. \\
                &=\left\{
                    \begin{array}{lcr}
                       0 & \textup{ si } & 0\leq x \\
                       \frac{x^2}{2} & \textup{ si }& 0<x<1 \\
                       -\frac{(x-1)^2}{2}+\frac{1}{2} & \textup{ si }& 1\leq x\leq 2 \\
                       0 & \textup{ si }& x>2\\
                    \end{array}
                \right. \\
                &=\left\{
                    \begin{array}{lcr}
                       0 & \textup{ si } & 0\leq x \\
                       \frac{x^2}{2} & \textup{ si }& 0<x<1 \\
                       -\frac{x^2}{2}+x & \textup{ si }& 1\leq x\leq 2 \\
                       0 & \textup{ si }& x>2\\
                    \end{array}
                \right. \\
            \end{split}
        \end{equation*}
    \end{exa}

    \begin{obs}
        Note que la función $f*g$ es continua. (esto servirá para ver que la convolución obtenida es correcta).
    \end{obs}

    \begin{exa}
        Recuerde la fórmula de Cauchy para la $n$-ésima integral reiterada:
        \begin{equation*}
            \int_{ 0}^xdx_1\int_0^{x_1}dx_2\cdots \int_0^{ x_{n-1}}f(x_n)dx_n=\frac{1}{(n-1)!}\int_0^x\frac{f(t)}{(x-t)^{n-1}}dt
        \end{equation*}
        la igualdad anterior es la misma que la de la función:
        \begin{equation*}
            \int_0^x\frac{f(t)dt}{\Gamma(n)(x-t)^{n-1}}=f*g(x)
        \end{equation*}
        donde
        \begin{equation*}
            g(x)=\left\{
                \begin{array}{lcr}
                    0 & \textup{ si }&x\leq 0\\
                    \frac{1}{\Gamma(n)x^{n-1}} & \textup{ si }& x>0\\
                \end{array}
            \right.
        \end{equation*}
        Si $0<\alpha\leq 1$, definimos:
        \begin{equation*}
            \int_0^x\frac{f(t)dt}{\Gamma(\alpha)(x-t)^{1-\alpha}}=I^{\alpha}_0[f](x)
        \end{equation*}
        llamada la \textbf{integral fraccional de orden $\alpha$ de $f$ en $x$}. Por ejemplo:
        \begin{equation*}
            I^{1/2}_0[t](x)=\frac{4}{3\sqrt{\pi}}x^{3/2}
        \end{equation*}
        \begin{equation*}
            I^{1/2}_0\left[\frac{4}{3\sqrt{\pi}}t^{3/2}\right](x)=\frac{x^2}{2}
        \end{equation*}
        que concuerda con la integral normal de $t$.
    \end{exa}

    Ahora estudiaremos algunas propiedades de este operador.

    \begin{propo}[\textbf{Asociatividad y conmutatividad de la convolución}]
        Sean $\cf{f,g,h}{\mathbb{R}^n}{\mathbb{K}}$ funciones medibles.
        \begin{enumerate}
            \item Si para algún $x\in\mathbb{R}^n$ existe la convolución $f*g(x)$, entonces también existe $g*f(x)$, y,
            \begin{equation*}
                f*g(x)=g*f(x)
            \end{equation*}
            \item Si la función $\abs{f}*\abs{g}$ está definida c.t.p. en $\mathbb{R}^n$ y, para algún $x\in\mathbb{R}^n$ existe $(\abs{f}*\abs{g})*\abs{h}(x)$, entonces existen $(f*g)*h(x)$, $f*(g*h)(x)$ y,
            \begin{equation*}
                (f*g)*h(x)=f*(g*h)(x)
            \end{equation*}
        \end{enumerate}
    \end{propo}

    \begin{proof}
        De (1): Se tiene que:
        \begin{equation*}
            f*g(x)=\int_{\mathbb{R}^n }f(y)g(x-y)dy=\int_{ \mathbb{R}^n}f(x-u)g(u)du=\int_{ \mathbb{R}^n}g(u)f(x-u)du=g*f(x)
        \end{equation*}
        por el cambio de variable $u=x-y$, de Jacobiano $\abs{(-1)^n}=1$. En particular, esto garantiza la existencia de $g*f(x)$.

        De (2): Se demostrará primero que la función
        \begin{equation*}
            (y,z)\mapsto f(z)g(y-z)h(x-y)
        \end{equation*} es medible como función de $\mathbb{R}^n\times\mathbb{R}^n$ en $\mathbb{K}$, para un $x\in\mathbb{R}^n$ fijo. Ya se sabe que $(y,z)\mapsto f(z)$ es medible (por una proposición sobre productos tensoriales).

        Se afirma que la función $(y,z)\mapsto h(x-y)$ es medible. En efecto, $u\mapsto h(u)$ es medible. Por el cambio de variable $u=x-y$, la función $y\mapsto h(x-y)$ también es medible (por el teorema de cambio de variable). Luego, como con $f$, se sigue que $(y,z)\mapsto h(x-y)$ es medible.

        También $(y,z)\mapsto g(y-z)$ es medible. Por productos tensoriales:
        \begin{equation*}
            G(u,v)=g(u)
        \end{equation*}
        es medible. La función $\Phi(r,s)=(r-s,s)$ es un isomorfismo $C^\infty$ de $\mathbb{R}^n\times\mathbb{R}^n$ sobre $\mathbb{R}^n\times\mathbb{R}^n$. Por el teorema de cambio de variable se sigue que es medible la función:
        \begin{equation*}
            G\circ \Phi(y,z)=g(y-z)
        \end{equation*}

        Por lo tanto, la función inicial es medible.

        Puesto que para $x\in\mathbb{R}^n$:
        \begin{equation*}
            \int_{ \mathbb{R}^n}\abs{h(x-y)}dy\int_{\mathbb{R}^n}\abs{f(z)}\abs{g(y-z)}dz=\int_{\mathbb{R}^n}\abs{h(x-y)}(\abs{f}*\abs{g})(y)dy=(\abs{f}*\abs{g})*\abs{h}(x)<\infty
        \end{equation*}
        (para los $x$ en que esté definida la función), entonces por Tonelli la función $(y,z)\mapsto f(z)g(y-z)h(x-y)$ es integrable y, por Fubini:
        \begin{equation*}
            (f*g)*h(x)=\int_{ \mathbb{R}^n}h(x-y)dy\int_{\mathbb{R}^n}f(z)g(y-z)dz
        \end{equation*}
        además,
        \begin{equation*}
            \begin{split}
                \int_{ \mathbb{R}^n\times \mathbb{R}^n}h(x-y)f(z)g(y-z)dydz&=\int_{ \mathbb{R}^n}f(z)dx\int_{ \mathbb{R}^n}h(x-y)g(y-z)dy\\
                &=\int_{ \mathbb{R}^n}f(z)dz\int_{ \mathbb{R}^n}h((x-z)-u)g(y-z)dy\\
                &=\int_{ \mathbb{R}^n}f(z)(g*h)(x-z)dz\\
                &=f*(g*h)(x)\\
            \end{split}
        \end{equation*}
        En particular, existen y son iguales $f*(g*h)(x)$ y $(f*g)*h(x)$.
    \end{proof}

    \begin{theor}
        Si $f,g\in\mathcal{L}_1(\mathbb{R}^n,\mathbb{K})$, se cumplen las afirmaciones siguientes.
        \begin{enumerate}
            \item Para casi toda $x\in\mathbb{R}^n$, existe $f*g(x)$.
            \item La función $f*g$, definida c.t.p. en $\mathbb{R}^n$, es integrable en $\mathbb{R}^n$.
            \item $\int_{ \mathbb{R}^n}f*g=\left(\int_{ \mathbb{R}^n}f\right)\left(\int_{ \mathbb{R}^n}g\right)$.
            \item $\N{1}{f*g}\leq\N{1}{\abs{f}*\abs{g}}=\N{1}{f}\N{1}{g}$.
        \end{enumerate}
    \end{theor}

    \begin{proof}
        De (1): Ya se sabe que la función $(x,y)\mapsto f(y)g(x-y)$ es medible (ver la proposición anterior). Como
        \begin{equation*}
            \int_{\mathbb{R}^n}\abs{f(y)}dy\int_{\mathbb{R}^n}\abs{g(x-y)}dx=\left(\int_{\mathbb{R}^n}\abs{f(y)}dy \right)\left(\int_{\mathbb{R}^n}\abs{g(z)}dz \right)<\infty
        \end{equation*}
        haciendo el cambio de variable $x=y+z$ y por ser $f,g$ integrables, entonces la función $(x,y)\mapsto f(y)g(x-y)$ es integrable en $\mathbb{R}^n\times \mathbb{R}^n$. Por el teorema de Fubini, la función $y\mapsto f(y)g(x-y)$ es integrable para casi toda $x\in\mathbb{R}^n$, lo cual prueba el primer inciso.

        De (2): Además, por Fubini nuevamente, la función $x\mapsto f*g(x)=\int_{\mathbb{R}^n}f(y)g(x-y)dy$ definida c.t.p. en $\mathbb{R}^n$ también es integrable, lo cual prueba el segundo inciso.

        De (3): Y, por Fubini:
        \begin{equation*}
            \begin{split}
                \int_{\mathbb{R}^n}(f*g)(x)dx&=\int_{ \mathbb{R}^n\times\mathbb{R}^n}f(y)g(x-y)dxdy\\
                &=\int_{\mathbb{R}^n}f(y)dy\int_{\mathbb{R}^n}g(x-y)dx\\
                &=\int_{\mathbb{R}^n}f(y)dy\int_{\mathbb{R}^n}g(u)du\\
                &=\left(\int_{\mathbb{R}^n}f(y)dy\right)\left(\int_{\mathbb{R}^n}g(u)du\right) \\
            \end{split}
        \end{equation*}
        lo cual prueba el tercer inciso.

        De (4): Aplicando (3) a $\abs{f},\abs{g}$, resulta que:
        \begin{equation*}
            \begin{split}
                \N{1}{f*g}&=\int_{\mathbb{R}^n}\abs{f*g}(x)dx\\
                &=\int_{\mathbb{R}^n}\abs{\int_{\mathbb{R}^n}f(y)g(x-y)}dx\\
                &\leq\int_{\mathbb{R}^n}\int_{\mathbb{R}^n}\abs{f(y)g(x-y)}dx\\
                &=\int_{\mathbb{R}^n}(\abs{f}*\abs{g})(x)dx\\
                &=\N{1}{\abs{f}*\abs{g}}\\
                &=\left(\int_{\mathbb{R}^n}\abs{f} \right)\left(\int_{\mathbb{R}^n}\abs{g} \right)\\
                &=\N{1}{f}\N{1}{g}\\
            \end{split}
        \end{equation*}
        lo cual prueba el cuarto inciso.
    \end{proof}

    \begin{obs}
        Se tiene lo siguiente:
        \begin{enumerate}
            \item La existencia y el valor de la convolución dependen solamente de las clases de equivalencia de $f$ y $g$, se puede pues considerar la convolución como una aplicación de $L_1(\mathbb{R}^n,\mathbb{K})\times L_1(\mathbb{R}^n,\mathbb{K})$ en $L_1(\mathbb{R}^n,\mathbb{K})$, tal que:
            \begin{equation*}
                \N{1}{f*g}\leq \N{1}{f}\N{1}{g}
            \end{equation*}
            \item Es claro que:
            \begin{equation*}
                (\alpha_1f_1+\alpha_2f_2)*g=\alpha_1(f_1*g)+\alpha_2(f_2*g)
            \end{equation*}
            y
            \begin{equation*}
                f*(\beta_1g_1+\beta_2g_2)=\beta_1(f*g_1)+\beta(f*g_2)
            \end{equation*}
            o sea, que la convolución es un aplicación bilineal y asociativa.
        \end{enumerate}
    \end{obs}

    \begin{mydef}
        Un \textbf{Álgebra de Banach} es un espacio de Banach $(E,\norm{\cdot})$ provisto de un producto $(x,y)\mapsto x\cdot y$. Este producto es bilineal y, además,
        \begin{equation*}
            \norm{x\cdot y}\leq \norm{x}\norm{y}
        \end{equation*}
        si el producto es conmutativo, se dice que el álgebra de Banach es \textbf{conmutativa}.
    \end{mydef}

    \begin{excer}
        En un álgebra de Banach, la función $(x,y)\mapsto x\cdot y$ es continua del espacio normado producto $E\times E$ en $E$.
    \end{excer}

    \begin{proof}
        Sean $\varepsilon>0$ y $(x_0,y_0)\in E\times E$. Tomemos $\delta=\min\left\{\frac{\varepsilon}{2(\norm{x_0}+1)},\frac{\varepsilon}{2(\norm{y_0}+1)},1\right\}>0$, entonces, si $(x,y)\in E\times E$ es tal que:
        \begin{equation*}
            \norm{(x_0,y_0)-(x,y)}<\delta
        \end{equation*}
        entonces,
        \begin{equation*}
            \norm{x_0-x}<\delta\quad\textup{y}\quad\norm{y_0-y}<\delta\Rightarrow\norm{y}<1+\norm{y_0}
        \end{equation*}
        luego, se tiene que:
        \begin{equation*}
            \begin{split}
                \norm{x_0\cdot y_0-x\cdot y}&=\norm{x_0\cdot y_0-x_0\cdot y+ x_0\cdot y-x\cdot y}\\
                &\leq\norm{x_0\cdot(y_0-y)}+\norm{(x_0-x)\cdot y}\\
                &\leq\norm{x_0}\norm{y_0-y}+\norm{x_0-x}\norm{y}\\
                &<\norm{y_0-y}(\norm{x_0}+1)+\norm{x_0-x}(\norm{y_0}+1)\\
                &<\frac{\varepsilon}{2(\norm{x_0}+1)}(\norm{x_0}+1)+\frac{\varepsilon}{2(\norm{y_0}+1)}(\norm{y_0}+1)\\
                &=\frac{\varepsilon}{2}+\frac{\varepsilon}{2}\\
                &=\varepsilon\\
            \end{split}
        \end{equation*}
        por tanto, $(x,y)\mapsto x\cdot y$ es continua en $(x_0,y_0)\in E\times E$. Por ser este elemento de $E\times E$ arbitrario, se sigue que es continua en todo $E\times E$.

    \end{proof}

    \begin{exa}
        Considere $\mathbb{K}$ como espacio vectorial sobre sí mismo con la norma usual y, provisto de la multiplicación usual en $\mathbb{K}$, es un álgebra de Banach conmutativa con elemento uno.
    \end{exa}
    
    \begin{exa}
        Sea $S$ un conjunto no vacío. El espacio vectorial $\mathcal{B}(S,\mathbb{K})$ de las funciones acotadas de $S$ en $\mathbb{K}$, provisto de la norma uniforme $\norm{\cdot}_\infty$ y con la multiplicación definida puntualmente, es un álgebra de Banach conmutativa con elemento uno (la función constante de valor uno).
    \end{exa}

    \begin{exa}
        Sea $S$ un espacio métrico. El subespacio $\mathcal{BC}(S,\mathbb{K})$ de las funciones continuas y acotadas de $S$ en $\mathbb{K}$ es una sub-álgebra de Banach del ejemplo anterior con elemento uno.
    \end{exa}

    \begin{exa}
        El subespacio $\mathcal{C}(\mathbb{R}^n,\mathbb{K})$ de $\mathcal{B}(\mathbb{R}^n,\mathbb{K})$ de las funciones continuas nulas en infinito es una sub-álgebra de Banach de $\mathcal{B}(\mathbb{R}^n,\mathbb{K})$ sin elemento uno.
    \end{exa}

    \begin{exa}
        Sea $E$ un espacio de Banach. El espacio normado $\textup{End}(E)$ de todos los endomorfismos continuos de $E$ provisto del producto $(A,B)\mapsto A\circ B $ es un álgebra de Banach no conmutativa con elemento uno.
    \end{exa}

    \begin{exa}
        $L_1(\mathbb{R}^n,\mathbb{K})$ provisto de la convolución también es un álgebra de Banach conmutativa (¿con elemento identidad?).
    \end{exa}

    \section{Convolución en $\mathcal{L}_p$}

    \begin{theor}[\textbf{Desigualdad de Hölder Generalizada}]
        Sean $p_1,...,p_m$ números positivos tales que:
        \begin{equation*}
            \frac{1}{p_1}+\frac{1}{p_2}+\cdots+\frac{1}{p_m}=1
        \end{equation*}
        entonces, si $f_1\in\mathcal{L}_{p_1}(\mathbb{R}^n,\mathbb{K}),f_2\in\mathcal{L}_{p_2}(\mathbb{R}^n,\mathbb{K}),...,f_m\in\mathcal{L}_{p_m}(\mathbb{R}^n,\mathbb{K})$, entonces $f_1\cdot f_2\cdots f_m\in \mathcal{L}_1(\mathbb{R}^n,\mathbb{K})$, y
        \begin{equation*}
            \N{1}{f_1\cdot f_2\cdots f_m}\leq \N{p_1}{f_1}\N{p_2}{f_2}\cdots\N{p_m}{f_m}
        \end{equation*}
    \end{theor}

    \begin{proof}
        Procederemos por inducción sobre $m\in\mathbb{N}$, $m\geq 2$. El caso $n=2$ es inmediato de la desigualdad de Hölder clásica.

        Suponga que el resultado se cumple para algún $m\in\mathbb{N}$, $m\geq 2$. Veamos que se cumple para $m+1$. En efecto, sean $f_1\in\mathcal{L}_{p_1}(\mathbb{R}^n,\mathbb{K}),f_2\in\mathcal{L}_{p_2}(\mathbb{R}^n,\mathbb{K}),...,f_{m+1}\in\mathcal{L}_{p_{m+1}}(\mathbb{R}^n,\mathbb{K})$ con $p_1,...,p_{m+1}$ números positivos tales que:
        \begin{equation*}
            \begin{split}
                \frac{1}{p_1}+\frac{1}{p_2}+\cdots+\frac{1}{p_{m+1}}&=1\\
                \Rightarrow \frac{1}{p_{m+1}^*}=1-\frac{1}{p_1}+\frac{1}{p_2}+\cdots+\frac{1}{p_{m}} \\
            \end{split}
        \end{equation*}
        afirmamos que $f_1\cdots f_m\in\mathcal{L}_{p_{m+1}^*}(\mathbb{R}^n,\mathbb{K})$. En efecto, observemos que:
        \begin{equation*}
            \int_{\mathbb{R}^n}\abs{}
        \end{equation*}
        %TODO
    \end{proof}

    \begin{propo}
        Si $\cf{f}{\mathbb{R}^{p+q}}{\mathbb{K}}$ es medible, se cumple lo siguiente:
        \begin{enumerate}
            \item Para casi toda $x\in\mathbb{R}^p$, la función $f_x(y)=f(x,y)$ de $\mathbb{R}^q$ en $\mathbb{K}$ es medible.
            \item Si para casi toda $x\in\mathbb{R}^p$, la función $f_x$ es integrable en $\mathbb{R}^q$, entonces:
            \begin{equation*}
                g(x)=\int_{\mathbb{R}^q}f_x=\int_{R^q}f(x,y)dy
            \end{equation*}
            definida c.t.p. es medible.
        \end{enumerate}
    \end{propo}

    \begin{theor}[\textbf{Teorema de Young}]
        Sean $p,q\in [1,\infty[$ tales que $\frac{1}{p}+\frac{1}{q}>1$ y defina $r$ como sigue:
        \begin{equation*}
            \frac{1}{r}=\frac{1}{p}+\frac{1}{q}-1
        \end{equation*}
        Entonces, si $f\in\mathcal{L}_p(\mathbb{R}^n,\mathbb{K})$ y $g\in\mathcal{L}_q(\mathbb{R}^n,\mathbb{K})$, se cumple lo siguiente:
        \begin{enumerate}
            \item Para casi toda $x\in\mathbb{R}^n$, existe la convolución $f*g$, es decir:
            \begin{equation*}
                f*g(x)=\int_{\mathbb{R}^n}f(y)g(x-y)dy
            \end{equation*}
            para casi toda $x\in\mathbb{R}^n$.
            \item $f*g\in\mathcal{L}_r(\mathbb{R}^n,\mathbb{K})$.
            \item $\N{r}{f*g}\leq \N{p}{f}\N{q}{g}$.
        \end{enumerate}
    \end{theor}

    \begin{proof}
        Observemos primero que los números $p,q,r$ satisfacen lo siguiente:
        \begin{equation*}
            r>1,\quad\frac{1}{p}-\frac{1}{r}\geq 0,\quad\frac{1}{q}-\frac{1}{r}\geq0
        \end{equation*}
        En efecto,
        \begin{equation*}
            \frac{1}{r}=\frac{1}{p}+\frac{1}{q}-1\leq 2-1=1\Rightarrow r\geq 1
        \end{equation*}
        las otras dos son inmediatas, ya que:
        \begin{equation*}
            \frac{1}{p}-\frac{1}{r}>1-\frac{1}{q}\geq0\quad\textup{y}\quad \frac{1}{q}-\frac{1}{r}>1-\frac{1}{p}\geq0
        \end{equation*}
        Se verá que para casi toda $x\in\mathbb{R}^n$, la función $y\mapsto f(y)g(x-y)$ es integrable en $\mathbb{R}^n$. Por un teorema anterior, ya se sabe que dicha función es medible.
        Escriba
        \begin{equation*}
            \abs{f(y)}\abs{g(x-y)}=\left(\abs{f(y)}^p\abs{g(x-y)}^q\right)^{\frac{1}{r}}\left(\abs{f(y)}^p \right)^{\frac{1}{p}-\frac{1}{r}}\left(\abs{g(x-y)}^q \right)^{\frac{1}{q}-\frac{1}{r}}
        \end{equation*}
        Para probar el resultado, se probarán dos casos:
        \begin{enumerate}
            \item $p>1$ y $q>1$ En este caso, $\frac{1}{p}-\frac{1}{r}>0$ y $\frac{1}{q}-\frac{1}{r}>0$. Si
            \begin{equation*}
                \frac{1}{\alpha}=\frac{1}{r},\quad \frac{1}{\beta}=\frac{1}{p}-\frac{1}{r},\quad \frac{1}{\gamma}=\frac{1}{q}-\frac{1}{r}
            \end{equation*}
            entonces,
            \begin{equation*}
                \frac{1}{\alpha}+\frac{1}{\beta}+\frac{1}{\gamma}=\frac{1}{p}+\frac{1}{q}-\frac{1}{r}=1
            \end{equation*}
            La función $y\mapsto \left(\abs{f(y)}^p\abs{g(x-y)}^q\right)^{\frac{1}{r}}$ está en $\mathcal{L}_\alpha(\mathbb{R}^n,\mathbb{K})$ (pues, existe la convolución $\abs{f}^p*\abs{g}^q(x)$ para casi toda $x\in\mathbb{R}^n$). También, $y\mapsto \left(\abs{f(y)}^p \right)^{\frac{1}{p}-\frac{1}{r}}$ está en $\mathcal{L}_\beta(\mathbb{R}^n,\mathbb{K})$ y $y\mapsto \left(\abs{g(x-y)}^q \right)^{\frac{1}{q}-\frac{1}{r}}$ está en $\mathcal{L}_\gamma(\mathbb{R}^n,\mathbb{K})$.

            Por Hölder generalizado, se tiene que $y\mapsto \abs{f(x)}\abs{g(x-y)}$ es integrable, en particular, existe la convolución $f*g$, lo que prueba (1). Además,
            \begin{equation*}
                \begin{split}
                    \abs{f*g}(x)&\leq \int_{\mathbb{R}^n}\abs{f(y)}\abs{g(x-y)}dy\\
                    &\leq \left[\int_{\mathbb{R}^n}\abs{f(y)}^p\abs{g(x-y)}^qdy\right]^{\frac{1}{r}} \left[\int_{\mathbb{R}^n}\abs{f(y)}^pdy \right]^{\frac{1}{p}-\frac{1}{r}}\left[\int_{\mathbb{R}^n}\abs{g(x-y)}^qdy \right]^{\frac{1}{q}-\frac{1}{r}}\\
                    &=\left[\abs{f}^p*\abs{g}^q(x) \right]^{\frac{1}{r}}\N{p}{f}^{1-\frac{p}{r}}\N{q}{g}^{1-\frac{q}{r}}\\
                \end{split}
            \end{equation*}
            luego,
            \begin{equation*}
                \abs{f*g}^r(x)\leq \N{p}{f}^{r-p}\N{q}{g}^{r-q}\left(\abs{f}^p*\abs{g}^q(x) \right)
            \end{equation*}
            por el teorema anterior (el cual asegura que $\abs{f}^p*\abs{g}^q$ es integrable), implica que $\abs{f}*\abs{g}\in\mathcal{L}_3(\mathbb{R}^n,\mathbb{K})$, lo cual prueba (2).

            Finalmente,
            \begin{equation*}
                \begin{split}
                    \N{r}{f*g}^r&=\int_{\mathbb{R}^n}\abs{f*g(x)}^rdx\\
                    &\leq \N{p}{f}^{r-q}\N{q}{g}^{r-p}\int_{\mathbb{R}^n}\abs{f}^p*\abs{g}^q(x)dx\\
                    &=\N{p}{f}^{r-q}\N{q}{g}^{r-p}\left(\int_{\mathbb{R}^n}\abs{f}^p\right)\left(\int_{\mathbb{R}^n}\abs{g}^q\right) \\
                    &=\N{p}{f}^{r-q}\N{q}{g}^{r-p}\N{p}{f}^{p}\N{q}{g}^q\\
                    &=\left(\N{p}{f}\N{q}{g}\right)^r \\
                    \Rightarrow \N{r}{f*g}&\leq \N{p}{f}\N{q}{g}\\
                \end{split}
            \end{equation*}

            \item $p>1$, $q=1$. En este caso, $r=p$, luego se sigue que:
            \begin{equation*}
                \frac{1}{\alpha}=\frac{1}{r}=\frac{1}{p},\quad \frac{1}{\beta}=\frac{1}{p}-\frac{1}{r}=0,\quad \frac{1}{\gamma}=\frac{1}{q}-\frac{1}{r}=1-\frac{1}{r}=\frac{1}{p^*}
            \end{equation*}
            Luego, si $x\in\mathbb{R}^n$, se tiene que:
            \begin{equation*}
                \begin{split}
                    \abs{f(y)}\abs{g(x-y)}&=\left(\abs{f(y)}^p\abs{g(x-y)}^q\right)^{\frac{1}{r}}\left(\abs{f(y)}^p \right)^{\frac{1}{p}-\frac{1}{r}}\left(\abs{g(x-y)}^q \right)^{\frac{1}{q}-\frac{1}{r}}\\
                    &=\left(\abs{f(y)}^p\abs{g(x-y)}\right)^{\frac{1}{p}}\left(\abs{f(y)}^p \right)^{0}\left(\abs{g(x-y)}^q \right)^{\frac{1}{p^*}}\\
                    &=\left(\abs{f(y)}^p\abs{g(x-y)}\right)^{\frac{1}{p}}\left(\abs{g(x-y)}^q \right)^{\frac{1}{p^*}}\\
                \end{split}
            \end{equation*}
            Como $y\mapsto \left(\abs{f(y)}^p\abs{g(x-y)}\right)^{\frac{1}{p}}$ está en $\mathcal{L}_p(\mathbb{R},\mathbb{K})$ (pues existe $\abs{f}^p*\abs{g}(x)$ para casi toda $x\in\mathbb{R}^n$) y $y\mapsto \left(\abs{g(x-y)}^q \right)^{\frac{1}{p^*}}$ está en $\mathcal{L}_{p^*}(\mathbb{R}^n,\mathbb{K})$, entonces por Hölder y la ecuación anterior, se sigue que $y\mapsto \abs{f(y)g(x-y)}$ es integrable en $\mathbb{R}^n$, luego existe $\abs{f}*\abs{g}(x)$ para casi toda $x\in\mathbb{R}^n$, lo que prueba (1). Además,
            \begin{equation*}
                \begin{split}
                    \abs{f*g}(x)&\leq\int_{\mathbb{R}^n}\abs{f(y)}\abs{g(x-y)}dy\\
                    &\leq\left[\int_{\mathbb{R}^n}\abs{f(y)}^p\abs{g(x-y)}dy \right]^{\frac{1}{p}}\left[\int_{\mathbb{R}^n}\abs{g(x-y)}dy \right]^{\frac{1}{p^*}}\\
                    &=\left[\abs{f}^p*\abs{g}(x) \right]^{\frac{1}{p}}\N{1}{g}^{\frac{1}{p^*}=1-\frac{1}{p^*}}\\
                    \Rightarrow \abs{f*g}^p(x)&\leq \left[\abs{f}^p*\abs{g}(x) \right]\N{1}{g}^{1-p}\\
                \end{split}
            \end{equation*}
            luego, $f*g\in\mathcal{L}_r(\mathbb{R}^n,\mathbb{K})$ (recuerde que $r=p$) lo cual prueba (2), y
            \begin{equation*}
                \begin{split}
                    \int_{\mathbb{R}^n}\abs{f*g}^p(x)dx&\leq\N{1}{g}^{p-1}\left(\int_{\mathbb{R}^n}\abs{f}^p \right)\left(\int_{\mathbb{R}^n}\abs{g} \right)\\
                    &\leq\N{1}{g}^{p-1}\left(\int_{\mathbb{R}^n}\abs{f}^p \right)\N{1}{q}\\
                    &\leq\N{p}{f}^p\N{1}{g}^{p} \\
                \end{split}
            \end{equation*}
            lo cual prueba (3).
        \end{enumerate}

        El caso $p=q=1$ es el teorema anterior, y por la conmutatividad de la convolución, no es necesario probar el caso $q=1$, $p>1$.
    \end{proof}

    \begin{obs}
        El caso $q=1$ y $r=p$ es importante, dice: Si $f\in\mathcal{L}_p(\mathbb{R}^n,\mathbb{K})$ y $g\in\mathcal{L}_1(\mathbb{R}^n,\mathbb{K})$ entonces, para casi toda $x\in\mathbb{R}^n$ existe $f*g(x)\in\mathcal{L}_p(\mathbb{R}^n,\mathbb{K})$ y $\N{p}{f*g}\leq\N{p}{f}\N{1}{g}$.
    \end{obs}

    \begin{theor}
        Fije $p\in[1,\infty]$. Si $f\in\mathcal{L}_p(\mathbb{R}^n,\mathbb{K})$ y $g\in\mathcal{L}_{ p^*}(\mathbb{R}^n,\mathbb{K})$ entonces, para toda $x\in\mathbb{R}^n$ (no solamente para casi toda $x$) existe $f*g(x)$ ,$f*g$ es medible acotada y:
        \begin{equation*}
            \sup_{ x\in\mathbb{R}^n}\abs{f*g(x)}\leq\N{p}{f}\N{p^*}{g}
        \end{equation*}
    \end{theor}

    \begin{proof}
        La función $y\mapsto f(y)$ está en $\mathcal{L}_p(\mathbb{R}^n,\mathbb{K})$ y, para cada $x\in\mathbb{R}^n$, $y\mapsto g(x-y)$ está en $\mathcal{L}_{ p^*}(\mathbb{R}^n,\mathbb{K})$. Entonces, $y\mapsto f(y)g(x-y)$ es integrable, luego existe $f*g(x)$ y, por Hölder:
        \begin{equation*}
            \begin{split}
                \abs{f*g(x)}=
                &\abs{\int_{\mathbb{R}^n}f(y)g(x-y)dx}\\
                &\leq\int_{\mathbb{R}^n}\abs{f(y)}\abs{g(x-y)}dy\\
                &=\N{p}{f}\left(\int_{\mathbb{R}^n}\abs{g(x-y)}^{ p^*}dy \right)^{1/p^*}\\
                &= \N{p}{f}\left(\int_{\mathbb{R}^n}\abs{g(z)}^{ p^*}dz \right)^{1/p^*}\textup{ por T.C.V. con }z=x-y \\
                &\leq\N{p}{f}\N{p^*}{g}\\
            \end{split}
        \end{equation*}
        Esto prueba que $f*g$ es acotada y, tomando supremos:
        \begin{equation*}
            \sup_{x\in\mathbb{R}^n}\abs{f*g(x)}\leq\N{p}{f}\N{p^*}{g}
        \end{equation*}
        además, por un resultado anterior, $f*g$ es medible.
    \end{proof}

    \begin{obs}
        Recuerde que si $\cf{f}{\mathbb{R}^n}{\mathbb{K}}$ entonces, para cada $h\in\mathbb{R}^n$ la función $\cf{f_h}{\mathbb{R}^n}{\mathbb{K}}$ dada por $f_h(x)=f(x+h)$ para todo $x\in\mathbb{R}^n$ es medible.
    \end{obs}

    \begin{lema}
        Sea $p\in[1,\infty[$, $f\in\mathcal{L}_p(\mathbb{R}^n,\mathbb{K})$. Entonces, para cada $h\in\mathbb{R}^n$, $f_h\in\mathcal{L}_p(\mathbb{R}^n,\mathbb{K})$ y $\N{p}{f_h}=\N{p}{f}$. Además, la aplicación $h\mapsto f_h$ de $\mathbb{R}^n$ en $\mathcal{L}_p(\mathbb{R}^n,\mathbb{K})$ es uniformemente continua en $\mathbb{R}^n$.
    \end{lema}

    \begin{proof}
        Se tienen que probar varias cosas:
        \begin{enumerate}
            \item Por el teorema de cambio de variable, para todo $h\in\mathbb{R}^n$, $f_h$ es medible y
            \begin{equation*}
                \int_{\mathbb{R}^n}\abs{f(y)}^pdy=\int_{\mathbb{R}^n}\abs{f(x+h)}^pdy=\int_{\mathbb{R}^n}\abs{f_h(y)}^pdy
            \end{equation*}
            por tanto, $f_h\in\mathcal{L}_p(\mathbb{R}^n,\mathbb{K})$ y, más aún, $\N{p}{f}=\N{p}{f_h}$.
            \item Se prueba que si $g\in\mathcal{C}_c(\mathbb{R}^n,\mathbb{K})$, entonces $h\mapsto g_h$ de $\mathbb{R}^n$ en el subespacio denso $\GenSpace{\mathcal{C}}{c}{\mathbb{R}^n}{\mathbb{K}}$ en $\mathcal{L}_p(\mathbb{R}^n,\mathbb{K})$ es uniformemente continua.
            
            Sea $\varepsilon>0$ y $K=\textup{Spt}(K)$. Entonces, $K$ es compacto en $\mathbb{R}^n$. Existe un rectángulo acotado con medida positiva $P\subseteq\mathbb{R}^n$ tal que $K\subseteq\mathring{P}$.

            Sea $\norm{\cdot}$ una norma de $\mathbb{R}^n$ y $d$ la correspondiente distancia inducida. Entonces, $d(K,\mathbb{R}^n\backslash\mathring{P})>0$. Como $g$ es uniformemente continua en $\mathbb{R}^n$ (pues es continua en un conjunto compacto, a saber, $\overline{P}$ y fuera de este conjunto es nula) existe $0<\delta<d(K,\mathbb{R}^n\backslash\mathring{P})$ tal que:
            \begin{equation*}
                x_1,y_1\in\mathbb{R}^n,\norm{x_1-y_1}<\delta\Rightarrow \abs{g(x_1)-g(y_1)}<\frac{\varepsilon}{(\textup{Vol}(P))^{1/p}}
            \end{equation*}
            Sean $s,t\in\mathbb{R}^n$ tales que $\norm{s-t}<\delta$. Entonces,
            \begin{equation*}
                \begin{split}
                    \N{p}{g_s-g_t}&=\left[\int_{\mathbb{R}^n}\abs{g(x+s)-g(x+t)}^pdx \right]^{1/p}\\
                    &=\left[\int_{\mathbb{R}^n}\abs{g(y+s-y)-g(y)}^pdy \right]^{1/p}\\
                \end{split}
            \end{equation*}
            haciendo el cambio de variable $x=y-t$ y, como para $y\in\mathbb{R}^n\backslash\mathring{P}$ se tiene que $y+s-k\notin K$ (pues, $\norm{s-t}<d(K,\mathbb{R}^n\backslash\mathring{P})$) luego, el integrando se anula fuera de $P$. Se sigue que:
            \begin{equation*}
                \begin{split}
                    \N{p}{g_s-g_t}&=\left[\int_{P}\abs{g(y+s-y)-g(y)}^pdy \right]^{1/p}\\
                    &=\left[\int_{P}\abs{\frac{\varepsilon}{(\textup{Vol}(P))^{1/p}}}^pdy \right]^{1/p}\\
                    &=\left[\int_{P}\frac{\varepsilon^p}{(\textup{Vol}(P))}dy \right]^{1/p}\\
                    &=\varepsilon
                \end{split}
            \end{equation*}
            lo que prueba el resultado.
            \item Sea $\varepsilon>0$. Existe $g\in\mathcal{C}_c(\mathbb{R}^n,\mathbb{K})$ tal que:
            \begin{equation*}
                \N{p}{f-g}z<\frac{\varepsilon}{3}
            \end{equation*}
            Por (2), existe $\delta>0$ tal que:
            \begin{equation*}
                s,t\in\mathbb{R}^n,\norm{s-t}<\delta\Rightarrow\N{p}{g_s-g_t}<\frac{\varepsilon}{3}
            \end{equation*}
            Dados $s,t\in\mathbb{R}^n$ tales que $\norm{s-t}<\delta$ se tiene que:
            \begin{equation*}
                \begin{split}
                    \N{p}{f_s-f_t}&\leq\N{p}{f_s-g_s}+\N{p}{g_s-g_t}+\N{p}{f_t-g_t}\\
                    &<\frac{\varepsilon}{3}+\frac{\varepsilon}{3}+\frac{\varepsilon}{3}\\
                    &=\varepsilon\\
                \end{split}
            \end{equation*}
            lo cual prueba la continuidad uniforme de $h\mapsto f_h$.
        \end{enumerate}
    \end{proof}

    \begin{propo}
        Fije $p\in[1,\infty]$. Si $f\in\mathcal{L}_p(\mathbb{R}^n,\mathbb{K})$ y $g\in\mathcal{L}_{ p^*}(\mathbb{R}^n,\mathbb{K})$, entonces $f*g$ es uniformemente continua en $\mathbb{R}^n$.
    \end{propo}

    \begin{proof}
        Se puede suponer que, por ejemplo, $p^*<\infty$. Por Hölder, para todo $s,t\in\mathbb{R}^n$:
        \begin{equation*}
            \begin{split}
                \abs{f*g(s)-f*g(t)}&=\int_{\mathbb{R}^n}\abs{f(y)[g(s-y)-g(t-y)]}dy\\
                &\leq\int_{\mathbb{R}^n}\abs{f(y)}\abs{g(s-y)-g(t-y)}dy\\
                &\leq\N{p}{f}\left[\int_{\mathbb{R}^n}\abs{g(s-y)-g(t-y)}^{ p^*}dy \right]^{1/p^*}\\
                &\leq\N{p}{f}\left[\int_{\mathbb{R}^n}\abs{g(s+x)-g(t+x)}^{ p^*}dx \right]^{1/p^*}\\
                &=\N{p}{f}\N{p^*}{g_s-g_t}\\
            \end{split}
        \end{equation*}
        haciendo el cambio de variable $y=-x$. Por la continuidad uniforme de $h\mapsto f_h$, se tiene que $f*g$ también debe ser uniformemente continua. En efecto, sea $\varepsilon>0$, como $h\mapsto g_h$ es uniformemente continua, (usando el teorema anterior y ya que $p^*<\infty$), existe $\delta>0$ tal que si $s,t\in\mathbb{R}^n$ son tales que:
        \begin{equation*}
            \norm{s-t}<\delta\Rightarrow\N{p^*}{g_s-g_t}<\frac{\varepsilon}{\N{p}{f}+1}
        \end{equation*}
        Luego,
        \begin{equation*}
            \norm{s-t}<\delta\Rightarrow\abs{f*g(s)-f*g(t)}<\left(\N{p}{f}+1 \right)\cdot\frac{\varepsilon}{\N{p}{f}+1}=\varepsilon
        \end{equation*}
        lo que prueba la continuidad uniforme de $f*g$.
    \end{proof}

    \begin{propo}
        Fije $p\in ]1,\infty[$. Si $f\in\mathcal{L}_p(\mathbb{R}^n)$ y $g\in\mathcal{L}_{ p^*}(\mathbb{R}^n,\mathbb{K})$, entonces:
        \begin{equation*}
            \lim_{x\rightarrow\infty}f*g(x)=0
        \end{equation*}
    \end{propo}

    \begin{proof}
        Fije una norma en $\mathbb{R}^n$, digamos $\norm{\cdot}$. Sea $\varepsilon>0$. Para cada $M>0$ se tiene lo siguiente:
        \begin{equation*}
            \begin{split}
                \abs{f*g(x)}&\leq\int_{\mathbb{R}^n}\abs{f(y)}\abs{g(x-y)}dy\\
                &\leq\int_{ \norm{y}\leq M}\abs{f(y)}\abs{g(x-y)}dy+\int_{ \norm{y}> M}\abs{f(y)}\abs{g(x-y)}dy\\
                &\leq\N{p}{f}\left[\int_{ \norm{y}\leq M}\abs{g(x-y)}^{ p^*}dy\right]^{ 1/p^*}+\N{p^*}{g}\left[\int_{ \norm{y}>M}\abs{f(y)}^pdy \right]^{1/p}\\
            \end{split}
        \end{equation*}
        para todo $x\in\mathbb{R}^n$. Por Lebesgue,
        \begin{equation*}
            \lim_{ M\rightarrow\infty}\int_{\norm{y}>M}\abs{f(y)}^pdy=0
        \end{equation*}
        Entonces, existe $M>0$ tal que
        \begin{equation*}
            \left[\int_{ \norm{y}>M}\abs{f(y)}^pdy \right]^{1/p}<\frac{\varepsilon}{1+\N{p}{f}+\N{p^*}{g}}
        \end{equation*}
        Por el cambio de variable $y=x-z$, resulta lo siguiente:
        \begin{equation*}
            \begin{split}
                \int_{\norm{y}\leq M }\abs{g(x-y)}^{ p^*}dy&=\int_{\norm{x-z}\leq M }\abs{g(z)}^{ p^*}dz\\
            \end{split}
        \end{equation*}
        Se sigue también del teorema de Lebesgue que
        \begin{equation*}
            \lim_{ R\rightarrow\infty}\int_{ \norm{z}>R}\abs{g(z)}^{p^*}dz=0
        \end{equation*}
        Entonces, para $\varepsilon>0$ existe $R>0$ tal que si $\norm{z}>R$, entonces:
        \begin{equation*}
            \int_{ \norm{z}>R}\abs{g(z)}^{p^*}dz<\frac{\varepsilon}{1+\N{p}{f}+\N{p^*}{g}}
        \end{equation*}
        Ahora, como 
        \begin{equation*}
            \left\{z\in\mathbb{R}^n\Big|\norm{x-z}\leq M \right\}\subseteq \left\{z\in\mathbb{R}^n\Big|\norm{x}-M\leq\norm{z} \right\}
        \end{equation*}
        tomando $x\in\mathbb{R}^n$ tal que $\norm{x}>R+M$, se sigue que:
        \begin{equation*}
            \int_{\norm{x-z}\leq M }\abs{g(z)}^{ p^*}dz\leq\int_{\norm{z}> R }\abs{g(z)}^{ p^*}dz<\frac{\varepsilon}{1+\N{p}{f}+\N{p^*}{g}}
        \end{equation*}
        Por tanto, tomando $\norm{x}>R+M$ se sigue que:
        \begin{equation*}
            \begin{split}
                \abs{f*g(x)}&\leq\left[\N{p}{f}+\N{p^*}{g} \right]\cdot\frac{\varepsilon}{1+\N{p}{f}+\N{p^*}{g}}\\
                &<\varepsilon\\
            \end{split}
        \end{equation*}
        por tanto:
        \begin{equation*}
            \lim_{x\rightarrow\infty}f*g(x)=0
        \end{equation*}
    \end{proof}

    \begin{obs}
        El resultado anterior no se generaliza al caso $p>1$ y $p^*=\infty$. En efecto, si $f\in\mathcal{L}_1(\mathbb{R}^n,\mathbb{K})$ con $\int_{\mathbb{R}^n}f\neq0$ y $g=\chi_{\mathbb{R}^n}$, entonces:
        \begin{equation*}
            f*g(x)=\int_{\mathbb{R}^n}f(y)g(x-y)dy=\int_{\mathbb{R}^n}f(y)dy\neq0
        \end{equation*}
        la cual no es nula en el infinito.
    \end{obs}

    \begin{propo}
        Si $f\in\mathcal{L}_1(\mathbb{R}^n,\mathbb{K})$ y $g\in\mathcal{L}_\infty(\mathbb{R}^n,\mathbb{K})$ es tal que
        \begin{equation*}
            \lim_{y\rightarrow\infty} g(y)=0
        \end{equation*}
        entonces,
        \begin{equation*}
            \lim_{x\rightarrow\infty} f*g(x)=0
        \end{equation*}
    \end{propo}
    
    \begin{proof}
        Por Hölder tenemos lo siguiente:
        \begin{equation*}
            \begin{split}
                \abs{f*g(x)}&\leq\int_{\mathbb{R}^n}\abs{f(x-y)}\abs{g(y)}dy\\
                &=\int_{\norm{y}\leq M }\abs{f(x-y)}\abs{g(y)}dy+\int_{\norm{y}>M}\abs{f(x-y)}\abs{g(y)}dy\\
                &\leq\N{\infty}{g}\int_{\norm{y}\leq M }\abs{f(x-y)}dy+\N{1}{f}\sup_{ \norm{y}>M}\abs{g(y)}\\
            \end{split}
        \end{equation*}
        Sea $\varepsilon>0$. Existe $M>0$ tal que:
        \begin{equation*}
            \sup_{ \norm{y}>M}\abs{g(y)}<\frac{\varepsilon}{1+\N{1}{f}+\N{\infty}{g}}
        \end{equation*}
        lo cual sucede, ya que $\lim_{y\rightarrow\infty} g(y)=0$. Ahora, se tiene que:
        \begin{equation*}
            \begin{split}
                \int_{\norm{y}\leq M } \abs{f(x-y)} dy &=\int_{\norm{x-z}\leq M }\abs{f(z)}dz\\
            \end{split}
        \end{equation*}
        Por Lebesgue, existe $R>0$ tal que:
        \begin{equation*}
            \int_{\norm{z}>R }\abs{f(z)}dz<\frac{\varepsilon}{1+\N{1}{f}+\N{\infty}{g}}
        \end{equation*}
        si $\norm{x}>R+M$, entocnes:
        \begin{equation*}
            \begin{split}
                \int_{\norm{y}\leq M } \abs{f(x-y)} dy &\leq\int_{\norm{z}>R }\abs{f(z)}dz\\
                &<\frac{\varepsilon}{1+\N{1}{f}+\N{\infty}{g}}\\
            \end{split}
        \end{equation*}
        Por tanto, si $\norm{x}>R+M$:
        \begin{equation*}
            \begin{split}
                \abs{f*g(x)}&\leq\left[\N{1}{f}+\N{\infty}{g} \right]\cdot \frac{\varepsilon}{1+\N{1}{f}+\N{\infty}{g}}\\
                &<\varepsilon\\
            \end{split}
        \end{equation*}
        lo cual prueba el resultado.
    \end{proof}

    \section{Convolución y diferenciación}

    \begin{propo}
        Sea $\cf{f}{\mathbb{R}^n}{\mathbb{K}}$ es integrable (está en $\mathcal{L}_1$) y $\cf{g}{\mathbb{R}^n}{\mathbb{K}}$ es de clase $C^r$ de tal suerte que $g$ y todas sus derivadas parciales hasta el orden $r$ (incluive) son acotadas, entonces $f*g$ es de clase $C^r$.

        Además, si $D=\partial_{\alpha_1}\cdots\partial_{\alpha_k}$ con $\alpha_1,\alpha_2,...,\alpha_k\in\left\{1,...,n \right\}$ y $k\in\left\{1,...,r \right\}$, se tiene:
        \begin{equation*}
            D(f*g)=f*Dg
        \end{equation*}

    \end{propo}

    \begin{proof}
        Como $f\in\mathcal{L}_1(\mathbb{R}^n,\mathbb{K})$ y $g\in\mathcal{L}_\infty(\mathbb{R}^n,\mathbb{K})$, entonces existen $f*g$ y $f*Dg$ (pues, tanto $g$ como $Dg$ son acotadas) en todo punto de $\mathbb{R}^n$.

        Se afirma que $D(f*g)=f*Dg$. Procederemos por inducción sobre $k$, basta probar que
        \begin{equation*}
            \partial_{\alpha_k}(f*g)=(f*\partial_{\alpha_k})g
        \end{equation*}
        (si se puede para una derivada parcial, se puede continuar con las demás derivadas parciales para obtener el operador $D$). Se tiene que:
        \begin{equation*}
            f*g(x)=\int_{\mathbb{R}^n}f(y)g(x-y)dy
        \end{equation*}
        y
        \begin{equation*}
            (f*\partial_{\alpha_k}g)(x)=\int_{\mathbb{R}^n}f(y)\partial_{\alpha_k}g(x-y)dy
        \end{equation*}
        Si $M=\sup_{z\in\mathbb{R}^n}\abs{\partial_{\alpha_k}g(z)}$, entonces
        \begin{equation*}
            \abs{f(y)\partial_{\alpha_k}g(x-y)}\leq M\abs{f(y)},\quad\forall y\in\mathbb{R}^n
        \end{equation*}
        donde la función de la derecha es integrable e independiente de $x$. Por el teorema de derivación de funciones definidas por integrales, existe $\partial_{\alpha_k}(f*g)$ y su valor es:
        \begin{equation*}
            \partial_{\alpha_k}(f*g)=\int_{\mathbb{R}^n}f(y)\partial_{\alpha_k}g(x-y)dy=(f*\partial_{\alpha_k}g)(x)
        \end{equation*}
        para todo $x\in\mathbb{R}^n$.
    \end{proof}

    \begin{mydef}
        Se dice que una función $\cf{f}{\mathbb{R}^n}{\mathbb{K}}$ es \textbf{localmente integrable}, si $f$ es integrable en todo compacto de $\mathbb{R}^n$. Se denota por $\mathcal{L}_1^{loc}(\mathbb{R}^n,\mathbb{K})$ al espacio vectorial de estas funciones.
    \end{mydef}

    \begin{obs}
        Toda función integrable es localmente integrable, pero no viceversa. En particular, $\mathcal{C}(\mathbb{R}^n,\mathbb{K})\subseteq\mathcal{L}_1^{loc}(\mathbb{R}^n,\mathbb{K})$ y, en particular, todos los polinomios están en $\mathcal{L}_1^{loc}(\mathbb{R}^n,\mathbb{K})$.

        Podemos entonces definir al espacio $\mathcal{L}_p^{loc}(\mathbb{R}^n,\mathbb{K})$ de todas las funciones tales que su módulo a la $p$ están en $\mathcal{L}_1^{loc}(\mathbb{R}^n,\mathbb{K})$. Pero, en particular se tendría que:
        \begin{equation*}
            \mathcal{L}_p^{loc}(\mathbb{R}^n,\mathbb{K})\subseteq\mathcal{L}_1^{loc}(\mathbb{R}^n,\mathbb{K})
        \end{equation*}
        para todo $p\in[1,\infty[$.
        %TODO
    \end{obs}

    \begin{propo}
        Si $f\in\mathcal{L}_1^{loc}(\mathbb{R}^n,\mathbb{K})$ y $g\in\mathcal{C}_c^r(\mathbb{R}^n,\mathbb{K})$, entonces $f*g$ existe en todo punto de $\mathbb{R}^n$, es de clase $C^r$ ($g$ es de clase $C^r$) y para todo $D=\partial_{\alpha_1}\cdots\partial_{\alpha_k}$, con $\alpha_1,\alpha_2,...,\alpha_k\in\left\{1,...,n \right\}$ y $k\in\left\{1,...,r \right\}$, se tiene:
        \begin{equation*}
            D(f*g)=f*D(g)
        \end{equation*}
    \end{propo}

    \begin{proof}
        Sea $K\subseteq\mathbb{R}^n$ el soporte de $g$ (el cual es compacto). Para cada $x\in\mathbb{R}^n$, existe la integral:
        \begin{equation*}
            f*g(x)=\int_{\mathbb{R}^n}f(y)g(x-y)dy
        \end{equation*}
        Esa integral es no cero si $x-y\in K$, es decir si $y\in x-K$. Por ende:
        \begin{equation*}
            f*g(x)=\int_{x-K}f(y)g(x-y)dy
        \end{equation*}
        el conjunto $x-K$ es compacto. Como $f$ es localmente integrable, es integrable en $x-K$ y $g$ es medible acotada, luego está en $\mathcal{L}_\infty^{loc}(x-K,\mathbb{K})$.

        Sea $\norm{\cdot}$ una norma en $\mathbb{R}^n$. Entonces:
        \begin{equation*}
            f*g(x)=\int_{\mathbb{R}^n}\underbrace{f(y)\chi_{ x-K}(y)}_{\in\mathcal{L}_1(\mathbb{R}^n,\mathbb{K})}g(x-y)dy=\int_{\mathbb{R}^n}f_1(y)g(x-y)dy
        \end{equation*}
        no se puede usar directamente el teorema de derivación, ya que $f_1(y)=f(y)\chi_{ x-K}(y)$ depende de $x$. Para ello, sea $R>0$ y
        \begin{equation*}
            B_R'=\left\{x\in\mathbb{R}^n\Big|\norm{x}\leq R \right\}
        \end{equation*}
        Para cada $x\in B_R'$, $x-K\subseteq B_R'+(-K)$ y:
        \begin{equation*}
            \begin{split}
                f*g(x)&=\int_{\mathbb{R}^n}f(y)g(x-y)dy\\
                &=\int_{B_R'+(-K)}f(y)g(x-y)dy\\
                &=\int_{\mathbb{R}^n}\left[f(y)\chi_{B_R'+(-K)}(y) \right]g(x-y)dy\\
                &=\int_{\mathbb{R}^n}f_1(y)g(x-y)dy\\
                &=f_1*g(x)\\
            \end{split}
        \end{equation*}
        para todo $x\in B_R'$. Por la proposición anterior, $f_1*g$ es de clase $C^r$ en $\mathbb{R}^n$, luego $f*g$ es de clase $C^r$ en $B_R'$. Además, para cada $x\in B_R'$,
        \begin{equation*}
            D(f*g)(x)=D(f_1*g)(x)=(f_1*Dg)(x)
        \end{equation*}
        y
        \begin{equation*}
            \begin{split}
                (f_1*g)(x)&=\int_{\mathbb{R}^n}f_1(y)Dg(x-y)dy\\
                &=\int_{ B_R'+(-K)}f(y)Dg(x-y)dy\\
                &=\int_{ x-K}f(y)Dg(x-y)dy\\
                &=f*Dg(x)\\
                \Rightarrow D(f*g)(x)&=f*Dg(x)\\
            \end{split}
        \end{equation*}
        pues, $Dg$ es nula fuera de $K$. Como el $R>0$ fue arbitrario, se sigue que el resultado anterior es válido para todo $x\in\mathbb{R}^n$.
    \end{proof}

    \begin{mydef}
        Sea $p\in[1,\infty[$ y $\cf{f}{\mathbb{R}^n}{\mathbb{K}}$. Se dice que $f\in\mathcal{L}_p^{loc}(\mathbb{R}^n,\mathbb{K})$ si la reestricción de $f$ a cada compacto $C\subseteq\mathbb{R}^n$ pertenece a $\mathcal{L}_p(C,\mathbb{K})$.
    \end{mydef}

    \begin{obs}
        Es claro que si $f\in\mathcal{L}_p^{loc}(\mathbb{R}^n,\mathbb{K})$, entonces $f\in \mathcal{L}_1^{loc}(\mathbb{R}^n,\mathbb{K})$ (pues, para todo compacto $C\subseteq\mathbb{R}^n$, se tiene que $\mathcal{L}_p(C,\mathbb{K})\subseteq\mathcal{L}_1(C,\mathbb{K})$). Y $\mathcal{L}_p(\mathbb{R}^n,\mathbb{K})\subseteq\mathcal{L}_p^{loc}(\mathbb{R}^n,\mathbb{K})$

        Así pues, el último resultado es válido con la hipótesis alternativa de que $f\in\mathcal{L}_p^{loc}(\mathbb{R}^n,\mathbb{K})$, en particular, de que $f\in\mathcal{L}_p^(\mathbb{R}^n,\mathbb{K})$
    \end{obs}

    \section{Sucesiones de Dirac}

    El álgebra de Banach $L_1(\mathbb{R}^n,\mathbb{C})$ no posee elemento uno, es decir, no existe $\delta\in\mathcal{L}_1(\mathbb{R}^n,\mathbb{K})$ tal que
    \begin{equation*}
        f*\delta=f\textup{ c.t.p. en }\mathbb{R}^n\quad\forall f\in\mathcal{L}_1(\mathbb{R}^n,\mathbb{K})
    \end{equation*}
    tampoco existe $\delta\in\mathcal{L}_1(\mathbb{R}^n,\mathbb{K})$ tal que:
    \begin{equation*}
        f*\delta=f\textup{ c.t.p. en }\mathbb{R}^n\quad\forall f\in\mathcal{L}_p(\mathbb{R}^n,\mathbb{K})
    \end{equation*}

    \begin{proof}
        En efecto, suponga que exista tal $\delta>0$. Sea $P\subseteq\mathbb{R}^n$ un rectángulo acotado tal que $\mathring{P}\neq\emptyset$. Se sabe que
        \begin{equation*}
            \delta*\chi_P=\chi_P\textup{ c.t.p. en }\mathbb{R}^n
        \end{equation*}
        por un resultado anterior, $\delta*\chi_P$ es una función continua en $\mathbb{R}^n$ ($\delta\in\mathcal{L}_1$ y $\chi_P\in\mathcal{L}_\infty$). Entonces:
        \begin{equation*}
            \delta*\chi_p=\chi_P=1\textup{ c.t.p. en }\mathbb{R}^n
        \end{equation*}
        como ambas son cintunas, entonces:
        \begin{equation*}
            \delta*\chi_P(x)=\chi_P(x)=1,\quad\forall x\in \mathring{P}
        \end{equation*}
        y
        \begin{equation*}
            \delta*\chi_P(x)=\chi_P(x)=0,\quad\forall x\in\mathbb{R}^n\backslash\overline{P}
        \end{equation*}
        esto contradeciría la continuidad de $\delta*\chi_P$ en $\mathbb{R}^n$.
    \end{proof}

    Las sucesiones de Dirac hacen el papel del elemento uno.

    \renewcommand{\theenumi}{\roman{enumi}}

    \begin{mydef}
        Una sucesión $\left\{\rho_\nu \right\}_{\nu=1}^\infty$ se dice que es una \textbf{sucesión de Dirac} si satisface lo siguiente:
        \begin{enumerate}
            \item $\rho_\nu\geq0$ para todo $\nu\in\mathbb{N}$.
            \item $\int_{\mathbb{R}^n}\rho_\nu=1$, para todo $\nu\in\mathbb{N}$.
            \item Para todo $\delta>0$, $\lim_{\nu\rightarrow\infty}\int_{\norm{x}<\delta}\rho_\nu(x)dx=1$.
        \end{enumerate}
        usar (ii) y (iii), (iii) es equivalente a:
        \begin{enumerate}
            \setcounter{enumi}{3}
            \item Para todo $\delta>0$, $\lim_{\nu\rightarrow\infty}\int_{\norm{x}\geq\delta}\rho_\nu(x)dx=0$.
        \end{enumerate}
        Esta definición es independiente de la norma elegida.
    \end{mydef}

    \begin{exa}
        Considere la sucesión de picos (especificar). Para todo $\nu\in\mathbb{N}$, $\rho_\nu$ es la función cuya gráfica es triangular de base $\left[\frac{1}{\nu},-\frac{1}{\nu} \right]$ sobre el eje $x$ y cuyo vértice está en el punto $(0,\nu)$ sobre el eje $y$ y que es cero fuera del intervalo.

        Entonces, $\left\{\rho_\nu \right\}$ es una sucesión de dirac en $\mathcal{L}_1(\mathbb{R},\mathbb{R})$.
    \end{exa}

    \begin{exa}
        Sea $\cf{\delta}{\mathbb{R}^n}{\mathbb{R}}$ una función no negativa tal que $\int_{\mathbb{R}^n}\rho_\nu=1$. Para cada $\nu\in\mathbb{N}$ se define:
        \begin{equation*}
            \rho_\nu(x)=\nu^n\rho_\nu(\nu x),\quad\forall x\in\mathbb{R}^n
        \end{equation*}
        Entonces, $\left\{\rho_\nu \right\}$ es una sucesión de Dirac en $\mathcal{L}_1(\mathbb{R}^n,\mathbb{K})$.

        Claramente cumple (i). Para (ii), veamos que:
        \begin{equation*}
            \int_{\mathbb{R}^n}\rho_\nu(x)dx=\int_{\mathbb{R}^n}\nu^n\rho(\nu x)dx=\int_{\mathbb{R}^n}\rho(y)dy=1
        \end{equation*}
        haciendo el cambio de variable $x=\frac{y}{\nu}$ de Jacobiano $\frac{1}{\nu^n}$.

        De (iii). Por el mismo cambio de variable:
        \begin{equation*}
            \int_{\norm{x}>\delta }\rho_\nu(x)dx=\nu^n\int_{\norm{x}>\delta}\rho(\nu x)dx=\int_{\norm{y}>\nu\delta}\rho(y)dy\longrightarrow_{\nu\rightarrow\infty}0
        \end{equation*}
        por el Teorema de Lebesgue. Luego, $\left\{\rho_\nu \right\}$ es una sucesión de Dirac.
    \end{exa}

    \subsection{Convolución de sucesiones de Dirac con funciones en $\mathcal{L}_p$, $1\leq p<\infty$}

    \begin{theor}[\textbf{Desigualdad de Jensen}]
        Sean $E\subseteq\mathbb{R}^n$ y $\cf{\rho}{E}{\mathbb{R}}$ tal que $\rho\geq0$, para todo $x\in E$, $\rho$ integrable en $E$ y
        \begin{equation*}
            \int_E\rho=1
        \end{equation*}
        Sea $I\subseteq\mathbb{R}$ un intervalo en $\mathbb{R}$, $\cf{f}{E}{I}$ una función y $\cf{\varphi}{I}{\mathbb{R}}$ uan función convexa. Si $f\cdot\rho$ y $(\varphi\circ f)\rho$ son integrables en $E$, entonces
        \begin{equation*}
            \int_{E}f\cdot\rho\in I
        \end{equation*}
        y
        \begin{equation*}
            \varphi\left(\int_{E}f\cdot\rho \right)\leq\int_E(\varphi\circ f)\rho
        \end{equation*}
    \end{theor}

    \begin{proof}
        
    \end{proof}

    \begin{exa}
        Suponga que $I=[0,\infty[$ en el teorema anterior, luego $f$ debe ser no negativa en $E$.
        \begin{enumerate}
            \item Si $\varphi(t)=t^p$, $t\geq0$ con $p\geq 1$, la desigualdad de Jensen dice que
            \begin{equation*}
                \left(\int_{E}f\cdot\rho \right)^p\leq\int_Ef^p\cdot\rho
            \end{equation*}
            siempre que las integrales existan. La conclusión persiste si $f$ es medible no negativa y $\int_Ef\cdot\rho<\infty$ y $\int_Ef^p\cdot\rho\leq\infty$.
        \end{enumerate}
    \end{exa}

\end{document}