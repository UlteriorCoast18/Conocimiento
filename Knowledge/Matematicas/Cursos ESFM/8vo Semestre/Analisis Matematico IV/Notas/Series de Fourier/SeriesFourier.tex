\documentclass[12pt]{report}
\usepackage[spanish]{babel}
\usepackage[utf8]{inputenc}
\usepackage{amsmath}
\usepackage{amssymb}
\usepackage{amsthm}
\usepackage{graphics}
\usepackage{subfigure}
\usepackage{lipsum}
\usepackage{array}
\usepackage{multicol}
\usepackage{enumerate}
\usepackage[framemethod=TikZ]{mdframed}
\usepackage[a4paper, margin = 1.5cm]{geometry}

%En esta parte se hacen redefiniciones de algunos comandos para que resulte agradable el verlos%

\renewcommand{\theenumii}{\roman{enumii}}

\def\proof{\paragraph{Demostración:\\}}
\def\endproof{\hfill$\blacksquare$}

\def\sol{\paragraph{Solución:\\}}
\def\endsol{\hfill$\square$}

%En esta parte se definen los comandos a usar dentro del documento para enlistar%

\newtheoremstyle{largebreak}
  {}% use the default space above
  {}% use the default space below
  {\normalfont}% body font
  {}% indent (0pt)
  {\bfseries}% header font
  {}% punctuation
  {\newline}% break after header
  {}% header spec

\theoremstyle{largebreak}

\newmdtheoremenv[
    leftmargin=0em,
    rightmargin=0em,
    innertopmargin=-2pt,
    innerbottommargin=8pt,
    hidealllines = true,
    roundcorner = 5pt,
    backgroundcolor = gray!60!red!30
]{exa}{Ejemplo}[section]

\newmdtheoremenv[
    leftmargin=0em,
    rightmargin=0em,
    innertopmargin=-2pt,
    innerbottommargin=8pt,
    hidealllines = true,
    roundcorner = 5pt,
    backgroundcolor = gray!50!blue!30
]{obs}{Observación}[section]

\newmdtheoremenv[
    leftmargin=0em,
    rightmargin=0em,
    innertopmargin=-2pt,
    innerbottommargin=8pt,
    rightline = false,
    leftline = false
]{theor}{Teorema}[section]

\newmdtheoremenv[
    leftmargin=0em,
    rightmargin=0em,
    innertopmargin=-2pt,
    innerbottommargin=8pt,
    rightline = false,
    leftline = false
]{propo}{Proposición}[section]

\newmdtheoremenv[
    leftmargin=0em,
    rightmargin=0em,
    innertopmargin=-2pt,
    innerbottommargin=8pt,
    rightline = false,
    leftline = false
]{cor}{Corolario}[section]

\newmdtheoremenv[
    leftmargin=0em,
    rightmargin=0em,
    innertopmargin=-2pt,
    innerbottommargin=8pt,
    rightline = false,
    leftline = false
]{lema}{Lema}[section]

\newmdtheoremenv[
    leftmargin=0em,
    rightmargin=0em,
    innertopmargin=-2pt,
    innerbottommargin=8pt,
    roundcorner=5pt,
    backgroundcolor = gray!30,
    hidealllines = true
]{mydef}{Definición}[section]

\newmdtheoremenv[
    leftmargin=0em,
    rightmargin=0em,
    innertopmargin=-2pt,
    innerbottommargin=8pt,
    roundcorner=5pt
]{excer}{Ejercicio}[section]

%En esta parte se colocan comandos que definen la forma en la que se van a escribir ciertas funciones%

\newcommand\abs[1]{\ensuremath{\big|#1\big|}}
\newcommand\divides{\ensuremath{\bigm|}}
\newcommand\cf[3]{\ensuremath{#1:#2\rightarrow#3}}
\newcommand\norm[1]{\ensuremath{\|#1\|}}
\newcommand\ora[1]{\ensuremath{\vec{#1}}}
\newcommand\pint[2]{\ensuremath{\left(#1\big| #2\right)}}
\newcommand\conj[1]{\ensuremath{\overline{#1}}}
\newcommand{\N}[2]{\ensuremath{\mathcal{N}_{#1}\left(#2\right)}}

%recuerda usar \clearpage para hacer un salto de página

\begin{document}
    \setlength{\parskip}{5pt} % Añade 5 puntos de espacio entre párrafos
    \setlength{\parindent}{12pt} % Pone la sangría como me gusta
    \title{Notas Análisis Matemático IV}
    \author{Cristo Daniel Alvarado}
    \maketitle

    \tableofcontents %Con este comando se genera el índice general del libro%
    
    \setcounter{chapter}{2}

    \chapter{Series de Fourier}
    
    \section{Series de Fourier de funciones en $\mathcal{L}_1^{2\pi}$}
    
    \begin{mydef}
        Se llama \textbf{serie de Fourier trigonométrica} a una serie de funciones de $\mathbb{R}$ en $\mathbb{C}$ de la forma
        \begin{equation}
            \sum_{k\in\mathbb{Z}}c_k e^{ ikx}
            \label{eq:fourier_1}
        \end{equation}
        donde $c_k\in\mathbb{C}$ para todo $k\in\mathbb{Z}$ son coeficientes constantes. Por definición, las \textbf{sumas parciales} de la serie son:
        \begin{equation*}
            s_m(x)=\sum_{ k=-m}^{m} c_{k}e^{ikx},\forall m\in\mathbb{N}^*
        \end{equation*}
        Se dice que la serie \textbf{converge en un punto $x$ a una suma $f(x)$}, si
        \begin{equation*}
            f(x)=\lim_{ m\rightarrow\infty}s_m(x)=\lim_{ m\rightarrow\infty}\sum_{ k=-m}^m c_k e^{ikx }
        \end{equation*}
        En este caso,
        \begin{equation*}
            f(x)=\sum_{ k\in\mathbb{Z}}c_ke^{ ikx}=\sum_{ k=-\infty}^\infty c_ke^{ ikx}
        \end{equation*}
        Usando la identidad $e^{ ikx}=\cos kx+i\sen kx$, podemos reescribir $s_m$ como
        \begin{equation}
            s_m(x)=c_0+\sum_{ k=1}^m(c_k+c_{ -k})\cos kx+i\sum_{ k=1}^m(c_k-c_{ -k})\sen kx,\quad\forall m\in\mathbb{N}^*
            \label{eq:fourier_2}
        \end{equation}
        definamos
        \begin{equation}
            a_k=c_k+c_{ -k}\quad\textup{y}\quad b_k=c_k-c_{ -k},\quad\forall k\in\mathbb{Z}
            \label{eq:coef_fourier_1}
        \end{equation}
        de la definición es claro que
        \begin{equation*}
            a_{ -k}=a_k\quad\textup{y}\quad b_{-k}=-b_k,\quad\forall k\in\mathbb{Z}
        \end{equation*}
        conociendo los coeficientes $a_k$ y $b_k$ se recobran los $c_k$ mediante las fórmulas
        \begin{equation}
            c_k=\frac{a_k-ib_k}{2},\quad\forall k\in\mathbb{Z}\backslash\left\{0 \right\}
            \label{eq:coef_fourier_2}
        \end{equation}
        y, $c_0=\frac{a_0}{2}$. En términos de los $a_k$ y $b_k$, las sumas \ref{eq:fourier_2} y \ref{eq:fourier_1} pueden ser reescritas como sigue:
        \begin{equation}
            s_m(x)=\frac{a_0}{2}+\sum_{ k=1}^m a_k\cos kx+\sum_{ k=1}^m b_k\sin kx,\quad\forall m\in\mathbb{N}^*
            \label{eq:fourier_3}
        \end{equation}
        y,
        \begin{equation}
            \sum_{k\in\mathbb{Z}}c_k e^{ ikx}=\frac{a_0}{2}+\sum_{ k=1}^\infty a_k\cos kx+\sum_{ k=1}^\infty b_k\sin kx
            \label{eq:fourier_4}
        \end{equation}
        respectivamente.
    \end{mydef}

    \begin{mydef}
        Se dice que la serie trigonométrica es \textbf{real} si $s_m(x)\in\mathbb{R}$ para todo $m\in\mathbb{N}^*$ y para toda $x\in\mathbb{R}$. Se sigue de \ref{eq:fourier_2} que la serie es real si y sólo si $a_k,b_k\in\mathbb{R}$, para todo $k\in\mathbb{N}^*$.
        
        Esta condición es equivalente a que
        \begin{equation*}
            c_{-k}=\overline{c_k},\quad\forall k\in\mathbb{Z}
        \end{equation*}
    \end{mydef}

    Es válido preguntarnos ahora: ¿Qué relación hay entre $f$ y los coeficientes $c_k$?
    
    \begin{propo}
        Considere una serie trigonométrica $\sum_{k\in\mathbb{Z}}c_k e^{ ikx}$. Suponga que esta serie converge uniformemente en $\mathbb{R}$ a alguna función $f$. Entonces, $f\in\mathcal{C}^{2\pi}$ y
        \begin{equation*}
            c_n=\frac{1}{2\pi}\int_{-\pi}^\pi f(x)e^{ -inx}dx,\quad\forall n\in\mathbb{Z}
        \end{equation*}
    \end{propo}

    \begin{proof}
        Se supone que $f(x)=\sum_{k\in\mathbb{Z}}c_k e^{ ikx}$ uniformemente en $\mathbb{R}$. Como el límite uniforme de una sucesión de funciones continuas es continua, se tiene entonces que $f\in\mathcal{C}^{2\pi}$. Para un $n\in\mathbb{Z}$:
        \begin{equation}
            \begin{split}
                f(x)e^{ -inx}=\sum_{k\in\mathbb{Z}}c_k e^{ i(k-n)x}\textup{ uniformemente en }\mathbb{R}
            \end{split}
            \label{eq:p_3_1_1}
        \end{equation}
        pues,
        \begin{equation*}
            \abs{f(x)e^{ -inx}-s_m(x)e^{-inx}}=\abs{f(x)-s_m(x)},\quad\forall m\in\mathbb{N}^*
        \end{equation*}
        Se puede pues integrar término por término \ref{eq:p_3_1_1} en el compacto $[-\pi,\pi]$. Antes veamos que
        \begin{equation*}
            \begin{split}
                \int_{ -\pi}^{\pi}e^{ i(n-k)x}dx=\left\{
                    \begin{array}[pos]{lcr}
                        2\pi & \textup{ si } & n=k\\
                        0 & \textup{ si } & n\neq k\\
                    \end{array}
                \right.
            \end{split}
        \end{equation*}
        por tanto,
        \begin{equation*}
            \begin{split}
                \int_{-\pi}^\pi f(x)e^{ -inx}dx&=\sum_{ k\in\mathbb{Z}}\int_{-\pi}^\pi e^{ i(k-n)x}dx\\
                &= 2\pi c_n\\
                \Rightarrow c_n&=\frac{1}{2\pi}\int_{-\pi}^\pi f(x)e^{ -inx}dx\\
            \end{split}
        \end{equation*}
    \end{proof}
    
    Este resultado sugiere la definición siguiente:

    \begin{mydef}
        Para todo $f\in\mathcal{L}_1^{2\pi}(\mathbb{C})$ se define
        \begin{equation}
            c_k=\frac{1}{2\pi}\int_{ -\pi}^\pi f(x)e^{ -ikx}dx,\quad\forall k\in\mathbb{Z}
            \label{eq:coef_fourier_3}
        \end{equation}
        en particular, $c_0=\frac{1}{2\pi}\int_{ -\pi}^{\pi}f(x)dx$. Los coeficientes $c_k$ se llaman \textbf{los coeficientes de Fourier trigonométricos de $f$} y, la serie
        \begin{equation*}
            \sum_{ k\in\mathbb{Z}}c_k e^{ ikx}
        \end{equation*}
        se llama \textbf{serie de Fourier trigonométrica de $f$}.
    \end{mydef}

    \begin{obs}
        Los correspondientes coeficientes $a_k$ y $b_k$ son los siguientes:
        \begin{equation*}
            a_0=\frac{1}{\pi}\int_{-\pi}^\pi f(x)dx
            \label{eq:coef_fourier_4}
        \end{equation*}
        también,
        \begin{equation*}
            a_k=\frac{1}{\pi}\int_{-\pi}^\pi f(x)\cos kxdx\quad\textup{y}\quad b_k=\frac{1}{\pi}\int_{-\pi}^\pi f(x)\sin kxdx
        \end{equation*}
        para todo $k\in\mathbb{Z}$ (esto se obtiene usando la igualdad entre los $c_k$ y $a_k,b_k$).
    \end{obs}

    \begin{obs}
        Para fines prácticos, conviene tener en cuenta lo siguiente.
        Si $f$ es una función impar en $]-\pi,\pi[$, entonces
        \begin{equation*}
            a_k=0\quad\forall k\in\mathbb{N}^*
        \end{equation*}
        y,
        \begin{equation*}
            b_k=\frac{2}{\pi}\int_{0}^{\pi}f(x)\sen kx,\quad\forall k\in\mathbb{N}
        \end{equation*}
        Si $f$ es una función par en $]-\pi,\pi[$ se invierte el resultado, es decir
        \begin{equation*}
            a_k=\frac{2}{\pi}\int_{0}^{\pi}f(x)\cos kx,\quad\forall k\in\mathbb{N}^*
        \end{equation*}
        y,
        \begin{equation*}
            b_k=0\quad\forall k\in\mathbb{N}
        \end{equation*}
    \end{obs}

    \begin{theor}
        Las aplicaciones $f\mapsto \left\{c_k \right\}_{ k\in\mathbb{Z}}$ y, $f\mapsto\left\{a_0,a_1,b_1,... \right\}$ son aplicaciones lineales inyectivas de $L_1^{2\pi}$ en el espacio de sucesiones complejas y reales, respectivamente. En particular, si $f,g\in\mathcal{L}_1^{2\pi}$ tienen los mismos coeficientes de Fourier trigonométricos, entonces $f=g$ c.t.p. en $\mathbb{R}$.
    \end{theor}

    \begin{proof}
        Por la forma en que se definen los coeficientes de Fourier de una función integrable, es claro que dichas aplicaciones son lineales.

        Resta probar que su kernel es $\left\{0\right\}$. Sea $f\in\mathcal{L}_1^{2\pi}(\mathbb{C})$ tal que
        \begin{equation*}
            c_n=\frac{1}{2\pi}\int_{-\pi}^{\pi}f(x)e^{ inx}dx=0,\quad\forall n\in\mathbb{Z}
        \end{equation*}
        dado que el sistema trigonométrico $\tau_{\mathbb{C}}$ es total en $\mathcal{L}_1^{2\pi}(\mathbb{C})$, necesariamente $f=0$ c.t.p. en $\mathbb{R}$.

        Similarmente se prueba la otra afirmación.
    \end{proof}

    \begin{propo}
        Sean $f,g\in\mathcal{L}_1^{2\pi}(\mathbb{C})$ y $\left\{c_k \right\}_{k\in\mathbb{Z}}$ y $\left\{d_k \right\}_{ k\in\mathbb{Z}}$ los coeficientes de Fourier trigonométricos de $f$ y $g$, respectivamente. Entonces, los coeficientes de Fourier $\left\{\gamma_k \right\}$ de $f*g$ son $\left\{2\pi c_kd_k \right\}_{ k\in\mathbb{Z}}$.
    \end{propo}

    \begin{proof}
        Para todo $k\in\mathbb{Z}$ fijo se tiene lo siguiente:
        \begin{equation*}
            \begin{split}
                \gamma_k&=\frac{1}{2\pi}\int_{-\pi}^\pi f*g(x)e^{ -ikx}dx\\
                &=\frac{1}{2\pi}\int_{-\pi}^\pi e^{ -ikx}dx\int_{-\pi}^\pi f(y)g(x-y)dy\\
            \end{split}
        \end{equation*}
        como la función $(x,y)\mapsto e^{ -ikx}f(y)g(x-y)$ es integrable en $]-\pi,\pi[\times]-\pi,\pi[$ (pues la función es medible y su módulo es el mismo que el de $(x,y)\mapsto f(y)g(x-y)$, la cual es integrable por un teorema de convolución), se puede invertir del orden de integración:
        \begin{equation*}
            \begin{split}
                \gamma_k&=\frac{1}{2\pi}\int_{-\pi}^\pi f(y)dy\int_{-\pi}^\pi g(x-y)e^{ -ikx}dx\\
                &=\frac{1}{2\pi}\int_{-\pi}^\pi f(y)dy\int_{-\pi-y}^{\pi-y} g(z)e^{ -ik(z+y)}dz\\
                &=\frac{1}{2\pi}\int_{-\pi}^\pi f(y)e^{ -iky}dy\int_{-\pi-y}^{\pi-y} g(z)e^{ -ikz}dz\\
                &=\frac{1}{2\pi}\int_{-\pi}^\pi f(y)e^{ -iky}dy\int_{-\pi}^{\pi} g(z)e^{ -ikz}dz\\
                &=c_k\cdot\left(2\pi d_k \right)\\
                &=2\pi c_kd_k\\
            \end{split}
        \end{equation*}
        pues, las funciones son periódicas. Se tieene entonces con lo anterior el resultado para todo $k\in\mathbb{Z}$.
    \end{proof}

    \section{Series de Fourier de funciones en $\mathcal{L}_2^{2\pi}$}

    Recuerde que las funciones
    \begin{equation*}
        \varphi_k(x)=\frac{1}{\sqrt{2\pi}}e^{ ikx},\quad\forall k\in\mathbb{Z}
    \end{equation*}
    constituyen un sistema ortonormal maximal en el espacio Hilbertiano $L_2^{2\pi}(\mathbb{C})$. En el sentido \"Hilbertiano\", los coeficientes de Fourier de algún vector $f\in\mathcal{L}_2^{2\pi}(\mathbb{C})$ con respecto a dicho sistema ortonormal son los siguientes:
    \begin{equation*}
        \begin{split}
            \hat{f}(x)&=\pint{f}{\varphi_k}\\
            &=\frac{1}{\sqrt{2\pi}}\int_{-\pi}^{\pi}f(x)e^{ -ikx}dx\\
            &=\sqrt{2\pi}c_k\\
        \end{split}
    \end{equation*}
    luego, 
    \begin{equation*}
        \begin{split}
            \hat{f}(x)\varphi_k(x)&=\sqrt{2\pi}c_k\frac{1}{\sqrt{2\pi}}e^{ ikx}\\
            &=c_ke^{ ikx},\quad\forall k\in\mathbb{Z}\\
        \end{split}
    \end{equation*}
    La serie de Fourier hilbertiana de $f$ sería:
    \begin{equation*}
        \sum_{ k\in\mathbb{Z}}\hat{f}(x)\varphi_k(x)=\sum_{k\in\mathbb{Z}}c_ke^{ ikx}
    \end{equation*}
    que corresponde a al serie de Fourier trigonométrica de $f$. También, las funciones
    \begin{equation*}
        \frac{1}{\sqrt{2\pi}},\quad \eta_k(x)=\frac{1}{\sqrt{\pi}}\cos kx,\quad\theta_k(x)=\frac{1}{\sqrt{\pi}}\sin kx,\quad \forall k\in\mathbb{Z}
    \end{equation*}
    forman otro sistema O.N. maximal en $L_2^{2\pi}(\mathbb{K})$. Los correspondientes coeficientes de Fourier de $f$ con respecto a este sistema O.N. maximal serían:
    \begin{equation*}
        \left\{
            \begin{array}{rl}
                \pint{f}{\frac{1}{\sqrt{2\pi}}}&=\frac{1}{\sqrt{\pi}}\int_{-\pi}^{\pi}f(x)dx\\
                \pint{f}{\eta_k}&=\frac{1}{\sqrt{\pi}}\int_{-\pi}^{\pi}f(x)\cos kxdx \\
                \pint{f}{\theta_k}&=\frac{1}{\sqrt{\pi}}\int_{-\pi}^{\pi}f(x)\sen kxdx \\
            \end{array}
        \right.
    \end{equation*}
    La serie de Fourier hilbertiana de $f$ será:
    \begin{equation*}
        \begin{split}
            \pint{f}{\frac{1}{\sqrt{2\pi}}}+\sum_{ k=1}^\infty \pint{f}{\eta_k}\eta_k+\sum_{ k=1}^\infty \pint{f}{\theta_k}\theta_k=\frac{a_0}{2}+\sum_{ k=1}^\infty\left[a_k\cos kx+b_k\sin kx \right]
        \end{split}
    \end{equation*}
    Recuerde también que por el teorema de Riesz-Fischer que si $\left\{\vec{u}_\alpha \right\}_{\alpha\in\Omega}$ es un sistema O.N. maximal en un espacio hilbertiano $H$, entonces la aplicación $\vec{x}\mapsto\left\{\hat{x}(\alpha) \right\}$ es una isometría lineal de $H$ en $l_2(\Omega)$. La isometría inversa es:
    \begin{equation*}
        \varphi\mapsto \sum_{\alpha\in\Omega}\varphi(\alpha)\vec{u}_\alpha
    \end{equation*}

    Aplicacndo este resultado al primer caso se tiene que

    \begin{theor}
        Las aplicaciones
        \begin{equation*}
            f\mapsto \left\{\sqrt{2\pi}c_k \right\}_{k\in\mathbb{Z}}\quad\textup{y}\quad f\mapsto\left\{\sqrt{2\pi}\frac{a_0}{2},\sqrt{\pi}a_1,\sqrt{\pi}b_1 \right\}
        \end{equation*}
        son isometrías lineales de $L_2^{2\pi}$ sobre $l_2(\mathbb{Z})$ o $l_2(\mathbb{N})$, respectivamente. Se tienen las identidades siguientes de Parseval:
        \begin{equation*}
            \begin{split}
                \sum_{ k\in\mathbb{Z}}\abs{c_k}^2&=\frac{1}{2\pi}\int_{-\pi}^{\pi}\abs{f(x)}^2dx\\
                \frac{\abs{a_0}^2}{2}+\sum_{ k\in\mathbb{Z}}\left[\abs{a_k}^2+\abs{b_k}^2 \right]&=\frac{1}{\pi}\int_{-\pi}^{\pi}\abs{f(x)}^2dx\\
            \end{split}
        \end{equation*}
        Más generalmente, si $f,g\in\mathcal{L}_2^{2\pi}(\mathbb{K})$ con coeficientes de Fourier trigonométricos $\left\{c_k \right\}_{k\in\mathbb{Z}}$ y $\left\{d_k \right\}_{k\in\mathbb{Z}}$, entonces
        \begin{equation*}
            \sum_{ k\in\mathbb{Z}}c_k\overline{d_k}=\int_{-\pi}^{\pi}f(x)\overline{g(x)}dx
        \end{equation*}
        para los correspondientes coeficientes $\left\{a_k,b_k \right\}$ y  $\left\{\alpha_k,\beta_k \right\}$ se tiene
        \begin{equation*}
            \frac{a_0\overline{\alpha_k}}{2}+\sum_{ k\in\mathbb{Z}}\left[a_k\overline{\alpha_k}+b_k\overline{\beta_k} \right]=\int_{-\pi}^{\pi}f(x)\overline{g(x)}dx
        \end{equation*}
        Además, $f$ es igual al promedio cuadrádtico de su serie de Fourier:
        \begin{equation*}
            \lim_{ m\rightarrow\infty}\N{2}{f-s_m}=0
        \end{equation*}
    \end{theor}
    
    \begin{proof}
        Es inmediata de las observaciones hechas anteriormente y del teorema de Riesz-Fischer, junto con las identidades de Parserval.
    \end{proof}

    \begin{obs}
        Se tiene lo siguiente:
        \begin{enumerate}
            \item La suprayectividad de $f\mapsto\left\{\sqrt{2\pi}c_k \right\}_{ k\in\mathbb{Z}}$ de $L_2^{2\pi}(\mathbb{C})$ sobre $l_2(\mathbb{Z})$ es consecuencia del teorema de Riesz-Fischer, es decir, de la completez de $L_2^{2\pi}$. Dice que dada una sucesión arbitraria $\left\{c_k\right\}_{k\in\mathbb{Z}}$ en $l_2(\mathbb{Z})$ existe una función $f\in\mathcal{L}_2^{2\pi}$ única salvo equivalencias cuyos coeficientes de Fourier son la sucesión dada.
            
            Este resultado fue históricamente un éxito para la integral de Lebesgue.

            \item Carleson demostró en 1966 que para cada $f\in\mathcal{L}_2^{2\pi}$ la serie de Fourier de $f$ converge a $f$ c.t.p. en $\mathbb{R}$. Sin embargo, para funciones en $\mathcal{L}_1^{2\pi}$ ésta no será la misma historia.
        \end{enumerate}
    \end{obs}

    \section{Series de Fourier de funciones de periodo $T>0$}

    Sea $f\in\mathcal{L}_1^T$. Defina
    \begin{equation*}
        g(y)=f\left(\frac{T}{2\pi}y\right),\quad\forall y\in\mathbb{R}
    \end{equation*}
    entonces, $g\in\mathcal{L}_1^{2\pi}$. Por definición, los coeficientes de Fourier de $f$ van a ser los de $g$, estos son
    \begin{equation*}
        c_k=\frac{1}{2\pi}\int_{-\pi}^\pi g(y)e^{ iky}dy
    \end{equation*}
    Por el cambio de variable $y=\frac{2\pi}{T}x$, podemos reescribirlos de la siguiente forma:
    \begin{equation*}
        \begin{split}
            c_k=\frac{1}{T}\int_{ -\frac{T}{2}}^{\frac{T}{2}}f(x)e^{ i\frac{2\pi k}{T}x}dx
        \end{split}
    \end{equation*}
    en particular, $c_0=\frac{1}{T}\int_{ -\frac{T}{2}}^{\frac{T}{2}}f(x)dx$. Los correspondientes $a_k$ y $b_k$ son 
    \begin{equation*}
        \begin{split}
            a_0&=\frac{2}{T}\int_{ -\frac{T}{2}}^{\frac{T}{2}}f(x)dx\\
            a_k&=\frac{1}{T}\int_{ -\frac{T}{2}}^{\frac{T}{2}}f(x)\cos\left(\frac{2\pi k}{T}x \right)dx\\
            b_k&=\frac{1}{T}\int_{ -\frac{T}{2}}^{\frac{T}{2}}f(x)\sin\left(\frac{2\pi k}{T}x \right)dx\\
        \end{split}
    \end{equation*}
    Las series de Fourier trigonométricas correspondientes son
    \begin{equation*}
        \sum_{ k\in\mathbb{Z}}c_ke^{i\frac{2\pi k}{T}x}=\frac{a_0}{2}+\sum_{ k=1}^\infty\left[a_k\cos\left(\frac{2\pi k}{T}x \right)+b_k\sin\left(\frac{2\pi k}{T}x \right)\right]
    \end{equation*}
    Sea ahora $f\in\mathcal{L}_2^T$. Los coeficientes de Fourier de $f$ con respecto al sistema O.N. maximal formado por
    \begin{equation*}
        \varphi_k(x)=\frac{1}{\sqrt{T}}e^{ i\frac{2\pi k}{T}x},\quad\forall k\in\mathbb{Z}
    \end{equation*}
    son
    \begin{equation*}
        \pint{f}{\varphi_k}=\sqrt{T}c_k\quad\forall k\in\mathbb{Z}
    \end{equation*}
    si se usa el sistema O.N. maximal formado por
    \begin{equation*}
        \frac{1}{\sqrt{T}},\quad\eta_k(x)=\sqrt{\frac{2}{T}}\cos\left(\frac{2\pi k}{T}x \right),\quad\theta_k(x)=\sqrt{\frac{2}{T}}\sin\left(\frac{2\pi k}{T}x \right),\quad\forall k\in\mathbb{N}
    \end{equation*}
    se obtienen
    \begin{equation*}
        \begin{split}
            \pint{f}{\frac{1}{\sqrt{T}}}&=\sqrt{T}\frac{a_0}{2}\\
            \pint{f}{\eta_k}&=\sqrt{\frac{T}{2}}a_k\\
            \pint{f}{\theta_k}&=\sqrt{\frac{T}{2}}b_k\\
        \end{split}
    \end{equation*}
    para todo $k\in\mathbb{N}$. Las series de Fourier correspondientes serían
    \begin{equation*}
        \sum_{ k\in\mathbb{Z}}\pint{f}{\varphi_k}\varphi_k=\sum_{ k\in\mathbb{Z}}e^{ i\frac{2\pi k}{T}x}=\frac{a_0}{2}+\sum_{ k=1}^\infty\left[a_k\cos\left(\frac{2\pi k}{T}x \right)+b_k\sin\left(\frac{2\pi k}{T}x \right) \right]
    \end{equation*}
    Se tienen las identidades de Parserval
    \begin{equation*}
        \begin{split}
            \sum_{ k\in\mathbb{Z}}\abs{c_k}^2&=\frac{1}{T}\int_{ -\frac{T}{2}}^{\frac{T}{2}}\abs{f(x)}^2\\
            \frac{\abs{a_0}^2}{2}+\sum_{ k=1}^\infty\left[\abs{a_k}^2+\abs{b_k}^2 \right]&=\frac{T}{2}\int_{ -\frac{T}{2}}^{\frac{T}{2}}\abs{f(x)}dx\\
            \sum_{ k\in\mathbb{Z}}c_k\overline{d_k}&=\frac{1}{T}\int_{ -\frac{T}{2}}^{\frac{T}{2}}f(x)\overline{g(x)}dx\\
            \frac{a_0\overline{\alpha_0}}{2}+\sum_{ k=1}^\infty\left[a_k\overline{\alpha_k}+b_k\overline{\beta_k} \right]&=\frac{2}{T}\int_{ -\frac{T}{2}}^{\frac{T}{2}}f(x)\overline{g(x)}dx
        \end{split}
    \end{equation*}
    por lo cual, en lo sucesivo se trabajará únicamente con funciones de periodo $2\pi$ (la traducción al periodo $T>0$ es un ejercicio).
\end{document}