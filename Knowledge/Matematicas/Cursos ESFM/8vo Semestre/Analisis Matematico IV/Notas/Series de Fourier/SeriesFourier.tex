\documentclass[12pt]{report}
\usepackage[spanish]{babel}
\usepackage[utf8]{inputenc}
\usepackage{amsmath}
\usepackage{amssymb}
\usepackage{amsthm}
\usepackage{graphics}
\usepackage{subfigure}
\usepackage{lipsum}
\usepackage{array}
\usepackage{multicol}
\usepackage{enumerate}
\usepackage[framemethod=TikZ]{mdframed}
\usepackage[a4paper, margin = 1.5cm]{geometry}

%En esta parte se hacen redefiniciones de algunos comandos para que resulte agradable el verlos%

\renewcommand{\theenumii}{\roman{enumii}}

\def\proof{\paragraph{Demostración:\\}}
\def\endproof{\hfill$\blacksquare$}

\def\sol{\paragraph{Solución:\\}}
\def\endsol{\hfill$\square$}

%En esta parte se definen los comandos a usar dentro del documento para enlistar%

\newtheoremstyle{largebreak}
  {}% use the default space above
  {}% use the default space below
  {\normalfont}% body font
  {}% indent (0pt)
  {\bfseries}% header font
  {}% punctuation
  {\newline}% break after header
  {}% header spec

\theoremstyle{largebreak}

\newmdtheoremenv[
    leftmargin=0em,
    rightmargin=0em,
    innertopmargin=-2pt,
    innerbottommargin=8pt,
    hidealllines = true,
    roundcorner = 5pt,
    backgroundcolor = gray!60!red!30
]{exa}{Ejemplo}[section]

\newmdtheoremenv[
    leftmargin=0em,
    rightmargin=0em,
    innertopmargin=-2pt,
    innerbottommargin=8pt,
    hidealllines = true,
    roundcorner = 5pt,
    backgroundcolor = gray!50!blue!30
]{obs}{Observación}[section]

\newmdtheoremenv[
    leftmargin=0em,
    rightmargin=0em,
    innertopmargin=-2pt,
    innerbottommargin=8pt,
    rightline = false,
    leftline = false
]{theor}{Teorema}[section]

\newmdtheoremenv[
    leftmargin=0em,
    rightmargin=0em,
    innertopmargin=-2pt,
    innerbottommargin=8pt,
    rightline = false,
    leftline = false
]{propo}{Proposición}[section]

\newmdtheoremenv[
    leftmargin=0em,
    rightmargin=0em,
    innertopmargin=-2pt,
    innerbottommargin=8pt,
    rightline = false,
    leftline = false
]{cor}{Corolario}[section]

\newmdtheoremenv[
    leftmargin=0em,
    rightmargin=0em,
    innertopmargin=-2pt,
    innerbottommargin=8pt,
    rightline = false,
    leftline = false
]{lema}{Lema}[section]

\newmdtheoremenv[
    leftmargin=0em,
    rightmargin=0em,
    innertopmargin=-2pt,
    innerbottommargin=8pt,
    roundcorner=5pt,
    backgroundcolor = gray!30,
    hidealllines = true
]{mydef}{Definición}[section]

\newmdtheoremenv[
    leftmargin=0em,
    rightmargin=0em,
    innertopmargin=-2pt,
    innerbottommargin=8pt,
    roundcorner=5pt
]{excer}{Ejercicio}[section]

%En esta parte se colocan comandos que definen la forma en la que se van a escribir ciertas funciones%

\newcommand\abs[1]{\ensuremath{\big|#1\big|}}
\newcommand\divides{\ensuremath{\bigm|}}
\newcommand\cf[3]{\ensuremath{#1:#2\rightarrow#3}}

%recuerda usar \clearpage para hacer un salto de página

\begin{document}
    \setlength{\parskip}{5pt} % Añade 5 puntos de espacio entre párrafos
    \setlength{\parindent}{12pt} % Pone la sangría como me gusta
    \title{Notas Análisis Matemático IV}
    \author{Cristo Daniel Alvarado}
    \maketitle

    \tableofcontents %Con este comando se genera el índice general del libro%
    
    \setcounter{chapter}{2}

    \chapter{Series de Fourier}
    
    \section{Series de Fourier de funciones en $\mathcal{L}_1^{2\pi}$}
    
    \begin{mydef}
        Se llama \textbf{serie de Fourier trigonométrica} a una serie de funciones de $\mathbb{R}$ en $\mathbb{C}$ de la forma
        \begin{equation}
            \sum_{k\in\mathbb{Z}}c_k e^{ ikx}
            \label{eq:fourier_1}
        \end{equation}
        donde $c_k\in\mathbb{C}$ para todo $k\in\mathbb{Z}$ son coeficientes constantes. Por definición, las \textbf{sumas parciales} de la serie son:
        \begin{equation*}
            s_m(x)=\sum_{ k=-m}^{m} c_{k}e^{ikx},\forall m\in\mathbb{N}^*
        \end{equation*}
        Se dice que la serie \textbf{converge en un punto $x$ a una suma $f(x)$}, si
        \begin{equation*}
            f(x)=\lim_{ m\rightarrow\infty}s_m(x)=\lim_{ m\rightarrow\infty}\sum_{ k=-m}^m c_k e^{ikx }
        \end{equation*}
        En este caso,
        \begin{equation*}
            f(x)=\sum_{ k\in\mathbb{Z}}c_ke^{ ikx}=\sum_{ k=-\infty}^\infty c_ke^{ ikx}
        \end{equation*}
        Usando la identidad $e^{ ikx}=\cos kx+i\sen kx$, podemos reescribir $s_m$ como
        \begin{equation}
            s_m(x)=c_0+\sum_{ k=1}^m(c_k+c_{ -k})\cos kx+i\sum_{ k=1}^m(c_k-c_{ -k})\sen kx,\quad\forall m\in\mathbb{N}^*
            \label{eq:fourier_2}
        \end{equation}
        definamos
        \begin{equation}
            a_k=c_k+c_{ -k}\quad\textup{y}\quad b_k=c_k-c_{ -k},\quad\forall k\in\mathbb{Z}
            \label{eq:coef_fourier_1}
        \end{equation}
        de la definición es claro que
        \begin{equation*}
            a_{ -k}=a_k\quad\textup{y}\quad b_{-k}=-b_k,\quad\forall k\in\mathbb{Z}
        \end{equation*}
        conociendo los coeficientes $a_k$ y $b_k$ se recobran los $c_k$ mediante las fórmulas
        \begin{equation}
            c_k=\frac{a_k-ib_k}{2},\quad\forall k\in\mathbb{Z}\backslash\left\{0 \right\}
            \label{eq:coef_fourier_2}
        \end{equation}
        y, $c_0=\frac{a_0}{2}$. En términos de los $a_k$ y $b_k$, las sumas \ref{eq:fourier_2} y \ref{eq:fourier_1} pueden ser reescritas como sigue:
        \begin{equation}
            s_m(x)=\frac{a_0}{2}+\sum_{ k=1}^m a_k\cos kx+\sum_{ k=1}^m b_k\sin kx,\quad\forall m\in\mathbb{N}^*
            \label{eq:fourier_3}
        \end{equation}
        y,
        \begin{equation}
            \sum_{k\in\mathbb{Z}}c_k e^{ ikx}=\frac{a_0}{2}+\sum_{ k=1}^\infty a_k\cos kx+\sum_{ k=1}^\infty b_k\sin kx
            \label{eq:fourier_4}
        \end{equation}
        respectivamente.
    \end{mydef}

    \begin{mydef}
        Se dice que la serie trigonométrica es \textbf{real} si $s_m(x)\in\mathbb{R}$ para todo $m\in\mathbb{N}^*$ y para toda $x\in\mathbb{R}$. Se sigue de \ref{eq:fourier_2} que la serie es real si y sólo si $a_k,b_k\in\mathbb{R}$, para todo $k\in\mathbb{N}^*$.
        
        Esta condición es equivalente a que
        \begin{equation*}
            c_{-k}=\overline{c_k},\quad\forall k\in\mathbb{Z}
        \end{equation*}
    \end{mydef}

    Es válido preguntarnos ahora: ¿Qué relación hay entre $f$ y los coeficientes $c_k$?
    
    \begin{propo}
        Considere una serie trigonométrica $\sum_{k\in\mathbb{Z}}c_k e^{ ikx}$. Suponga que esta serie converge uniformemente en $\mathbb{R}$ a alguna función $f$. Entonces, $f\in\mathcal{C}^{2\pi}$ y
        \begin{equation*}
            c_n=\frac{1}{2\pi}\int_{-\pi}^\pi f(x)e^{ -inx}dx,\quad\forall n\in\mathbb{Z}
        \end{equation*}
    \end{propo}

    \begin{proof}
        Se supone que $f(x)=\sum_{k\in\mathbb{Z}}c_k e^{ ikx}$ uniformemente en $\mathbb{R}$. Como el límite uniforme de una sucesión de funciones continuas es continua, se tiene entonces que $f\in\mathcal{C}^{2\pi}$. Para un $n\in\mathbb{Z}$:
        \begin{equation}
            \begin{split}
                f(x)e^{ -inx}=\sum_{k\in\mathbb{Z}}c_k e^{ i(k-n)x}\textup{ uniformemente en }\mathbb{R}
            \end{split}
            \label{eq:p_3_1_1}
        \end{equation}
        pues,
        \begin{equation*}
            \abs{f(x)e^{ -inx}-s_m(x)e^{-inx}}=\abs{f(x)-s_m(x)},\quad\forall m\in\mathbb{N}^*
        \end{equation*}
        Se puede pues integrar término por término \ref{eq:p_3_1_1} en el compacto $[-\pi,\pi]$. Antes veamos que
        \begin{equation*}
            \begin{split}
                \int_{ -\pi}^{\pi}e^{ i(n-k)x}dx=\left\{
                    \begin{array}[pos]{lcr}
                        2\pi & \textup{ si } & n=k\\
                        0 & \textup{ si } & n\neq k\\
                    \end{array}
                \right.
            \end{split}
        \end{equation*}
        por tanto,
        \begin{equation*}
            \begin{split}
                \int_{-\pi}^\pi f(x)e^{ -inx}dx&=\sum_{ k\in\mathbb{Z}}\int_{-\pi}^\pi e^{ i(k-n)x}dx\\
                &= 2\pi c_n\\
                \Rightarrow c_n&=\frac{1}{2\pi}\int_{-\pi}^\pi f(x)e^{ -inx}dx\\
            \end{split}
        \end{equation*}
    \end{proof}
    
    Este resultado sugiere la definición siguiente:

    \begin{mydef}
        Para todo $f\in\mathcal{L}_1^{2\pi}(\mathbb{C})$ se define
        \begin{equation}
            c_k=\frac{1}{2\pi}\int_{ -\pi}^\pi f(x)e^{ -ikx}dx,\quad\forall k\in\mathbb{Z}
            \label{eq:coef_fourier_3}
        \end{equation}
        en particular, $c_0=\frac{1}{2\pi}\int_{ -\pi}^{\pi}f(x)dx$. Los coeficientes $c_k$ se llaman \textbf{los coeficientes de Fourier trigonométricos de $f$} y, la serie
        \begin{equation*}
            \sum_{ k\in\mathbb{Z}}c_k e^{ ikx}
        \end{equation*}
        se llama \textbf{serie de Fourier trigonométrica de $f$}.
    \end{mydef}

    \begin{obs}
        Los correspondientes coeficientes $a_k$ y $b_k$ son los siguientes:
        \begin{equation*}
            a_0=\frac{1}{\pi}\int_{-\pi}^\pi f(x)dx
            \label{eq:coef_fourier_4}
        \end{equation*}
        también,
        \begin{equation*}
            a_k=\frac{1}{\pi}\int_{-\pi}^\pi f(x)\cos kxdx\quad\textup{y}\quad b_k=\frac{1}{\pi}\int_{-\pi}^\pi f(x)\sin kxdx
        \end{equation*}
        para todo $k\in\mathbb{Z}$ (esto se obtiene usando la igualdad entre los $c_k$ y $a_k,b_k$).
    \end{obs}

    \begin{obs}
        Para fines prácticos, conviene tener en cuenta lo siguiente.
        Si $f$ es una función impar en $]-\pi,\pi[$, entonces
        \begin{equation*}
            a_k=0\quad\forall k\in\mathbb{N}^*
        \end{equation*}
        y,
        \begin{equation*}
            b_k=\frac{2}{\pi}\int_{0}^{\pi}f(x)\sen kx,\quad\forall k\in\mathbb{N}
        \end{equation*}
        Si $f$ es una función par en $]-\pi,\pi[$ se invierte el resultado, es decir
        \begin{equation*}
            a_k=\frac{2}{\pi}\int_{0}^{\pi}f(x)\cos kx,\quad\forall k\in\mathbb{N}^*
        \end{equation*}
        y,
        \begin{equation*}
            b_k=0\quad\forall k\in\mathbb{N}
        \end{equation*}
    \end{obs}



\end{document}