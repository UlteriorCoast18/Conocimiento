\documentclass[12pt]{report}
\usepackage[spanish]{babel}
\usepackage[utf8]{inputenc}
\usepackage{amsmath}
\usepackage{amssymb}
\usepackage{amsthm}
\usepackage{graphics}
\usepackage{subfigure}
\usepackage{lipsum}
\usepackage{array}
\usepackage{multicol}
\usepackage{enumerate}
\usepackage[framemethod=TikZ]{mdframed}
\usepackage[a4paper, margin = 1.5cm]{geometry}

%En esta parte se hacen redefiniciones de algunos comandos para que resulte agradable el verlos%

\renewcommand{\theenumii}{\roman{enumii}}

\def\proof{\paragraph{Demostración:\\}}
\def\endproof{\hfill$\blacksquare$}

\def\sol{\paragraph{Solución:\\}}
\def\endsol{\hfill$\square$}

%En esta parte se definen los comandos a usar dentro del documento para enlistar%

\newtheoremstyle{largebreak}
  {}% use the default space above
  {}% use the default space below
  {\normalfont}% body font
  {}% indent (0pt)
  {\bfseries}% header font
  {}% punctuation
  {\newline}% break after header
  {}% header spec

\theoremstyle{largebreak}

\newmdtheoremenv[
    leftmargin=0em,
    rightmargin=0em,
    innertopmargin=-2pt,
    innerbottommargin=8pt,
    hidealllines = true,
    roundcorner = 5pt,
    backgroundcolor = gray!60!red!30
]{exa}{Ejemplo}[section]

\newmdtheoremenv[
    leftmargin=0em,
    rightmargin=0em,
    innertopmargin=-2pt,
    innerbottommargin=8pt,
    hidealllines = true,
    roundcorner = 5pt,
    backgroundcolor = gray!50!blue!30
]{obs}{Observación}[section]

\newmdtheoremenv[
    leftmargin=0em,
    rightmargin=0em,
    innertopmargin=-2pt,
    innerbottommargin=8pt,
    rightline = false,
    leftline = false
]{theor}{Teorema}[section]

\newmdtheoremenv[
    leftmargin=0em,
    rightmargin=0em,
    innertopmargin=-2pt,
    innerbottommargin=8pt,
    rightline = false,
    leftline = false
]{propo}{Proposición}[section]

\newmdtheoremenv[
    leftmargin=0em,
    rightmargin=0em,
    innertopmargin=-2pt,
    innerbottommargin=8pt,
    rightline = false,
    leftline = false
]{cor}{Corolario}[section]

\newmdtheoremenv[
    leftmargin=0em,
    rightmargin=0em,
    innertopmargin=-2pt,
    innerbottommargin=8pt,
    rightline = false,
    leftline = false
]{lema}{Lema}[section]

\newmdtheoremenv[
    leftmargin=0em,
    rightmargin=0em,
    innertopmargin=-2pt,
    innerbottommargin=8pt,
    roundcorner=5pt,
    backgroundcolor = gray!30,
    hidealllines = true
]{mydef}{Definición}[section]

\newmdtheoremenv[
    leftmargin=0em,
    rightmargin=0em,
    innertopmargin=-2pt,
    innerbottommargin=8pt,
    roundcorner=5pt
]{excer}{Ejercicio}[section]

%En esta parte se colocan comandos que definen la forma en la que se van a escribir ciertas funciones%

\renewcommand{\leq}{\ensuremath{\leqslant}}
\renewcommand{\geq}{\ensuremath{\geqslant}}

\newcommand\abs[1]{\ensuremath{\left|#1\right|}}
\newcommand\divides{\ensuremath{\bigm|}}
\newcommand\cf[3]{\ensuremath{#1:#2\rightarrow#3}}
\newcommand\norm[1]{\ensuremath{\|#1\|}}
\newcommand\ora[1]{\ensuremath{\vec{#1}}}
\newcommand\pint[2]{\ensuremath{\left(#1\big| #2\right)}}
\newcommand\conj[1]{\ensuremath{\overline{#1}}}
\newcommand{\N}[2]{\ensuremath{\mathcal{N}_{#1}\left(#2\right)}}

\newcommand{\natint}[1]{\ensuremath{\left[\!\left[#1\right]\!\right]}}

%recuerda usar \clearpage para hacer un salto de página

\begin{document}
    \setlength{\parskip}{5pt} % Añade 5 puntos de espacio entre párrafos
    \setlength{\parindent}{12pt} % Pone la sangría como me gusta
    \title{Notas Análisis Matemático IV}
    \author{Cristo Daniel Alvarado}
    \maketitle

    \tableofcontents %Con este comando se genera el índice general del libro%
    
    \setcounter{chapter}{2}

    \chapter{Series de Fourier}
    
    \section{Series de Fourier de funciones en $\mathcal{L}_1^{2\pi}$}
    
    \begin{mydef}
        Se llama \textbf{serie de Fourier trigonométrica} a una serie de funciones de $\mathbb{R}$ en $\mathbb{C}$ de la forma
        \begin{equation}
            \sum_{k\in\mathbb{Z}}c_k e^{ ikx}
            \label{eq:fourier_1}
        \end{equation}
        donde $c_k\in\mathbb{C}$ para todo $k\in\mathbb{Z}$ son coeficientes constantes. Por definición, las \textbf{sumas parciales} de la serie son:
        \begin{equation*}
            s_m(x)=\sum_{ k=-m}^{m} c_{k}e^{ikx},\forall m\in\mathbb{N}^*
        \end{equation*}
        Se dice que la serie \textbf{converge en un punto $x$ a una suma $f(x)$}, si
        \begin{equation*}
            f(x)=\lim_{ m\rightarrow\infty}s_m(x)=\lim_{ m\rightarrow\infty}\sum_{ k=-m}^m c_k e^{ikx }
        \end{equation*}
        En este caso,
        \begin{equation*}
            f(x)=\sum_{ k\in\mathbb{Z}}c_ke^{ ikx}=\sum_{ k=-\infty}^\infty c_ke^{ ikx}
        \end{equation*}
        Usando la identidad $e^{ ikx}=\cos kx+i\sen kx$, podemos reescribir $s_m$ como
        \begin{equation}
            s_m(x)=c_0+\sum_{ k=1}^m(c_k+c_{ -k})\cos kx+i\sum_{ k=1}^m(c_k-c_{ -k})\sen kx,\quad\forall m\in\mathbb{N}^*
            \label{eq:fourier_2}
        \end{equation}
        definamos
        \begin{equation}
            a_k=c_k+c_{ -k}\quad\textup{y}\quad b_k=c_k-c_{ -k},\quad\forall k\in\mathbb{Z}
            \label{eq:coef_fourier_1}
        \end{equation}
        de la definición es claro que
        \begin{equation*}
            a_{ -k}=a_k\quad\textup{y}\quad b_{-k}=-b_k,\quad\forall k\in\mathbb{Z}
        \end{equation*}
        conociendo los coeficientes $a_k$ y $b_k$ se recobran los $c_k$ mediante las fórmulas
        \begin{equation}
            c_k=\frac{a_k-ib_k}{2},\quad\forall k\in\mathbb{Z}\backslash\left\{0 \right\}
            \label{eq:coef_fourier_2}
        \end{equation}
        y, $c_0=\frac{a_0}{2}$. En términos de los $a_k$ y $b_k$, las sumas \ref{eq:fourier_2} y \ref{eq:fourier_1} pueden ser reescritas como sigue:
        \begin{equation}
            s_m(x)=\frac{a_0}{2}+\sum_{ k=1}^m a_k\cos kx+\sum_{ k=1}^m b_k\sin kx,\quad\forall m\in\mathbb{N}^*
            \label{eq:fourier_3}
        \end{equation}
        y,
        \begin{equation}
            \sum_{k\in\mathbb{Z}}c_k e^{ ikx}=\frac{a_0}{2}+\sum_{ k=1}^\infty a_k\cos kx+\sum_{ k=1}^\infty b_k\sin kx
            \label{eq:fourier_4}
        \end{equation}
        respectivamente.
    \end{mydef}

    \begin{mydef}
        Se dice que la serie trigonométrica es \textbf{real} si $s_m(x)\in\mathbb{R}$ para todo $m\in\mathbb{N}^*$ y para toda $x\in\mathbb{R}$. Se sigue de \ref{eq:fourier_2} que la serie es real si y sólo si $a_k,b_k\in\mathbb{R}$, para todo $k\in\mathbb{N}^*$.
        
        Esta condición es equivalente a que
        \begin{equation*}
            c_{-k}=\overline{c_k},\quad\forall k\in\mathbb{Z}
        \end{equation*}
    \end{mydef}

    Es válido preguntarnos ahora: ¿Qué relación hay entre $f$ y los coeficientes $c_k$?
    
    \begin{propo}
        Considere una serie trigonométrica $\sum_{k\in\mathbb{Z}}c_k e^{ ikx}$. Suponga que esta serie converge uniformemente en $\mathbb{R}$ a alguna función $f$. Entonces, $f\in\mathcal{C}^{2\pi}$ y
        \begin{equation*}
            c_n=\frac{1}{2\pi}\int_{-\pi}^\pi f(x)e^{ -inx}dx,\quad\forall n\in\mathbb{Z}
        \end{equation*}
    \end{propo}

    \begin{proof}
        Se supone que $f(x)=\sum_{k\in\mathbb{Z}}c_k e^{ ikx}$ uniformemente en $\mathbb{R}$. Como el límite uniforme de una sucesión de funciones continuas es continua, se tiene entonces que $f\in\mathcal{C}^{2\pi}$. Para un $n\in\mathbb{Z}$:
        \begin{equation}
            \begin{split}
                f(x)e^{ -inx}=\sum_{k\in\mathbb{Z}}c_k e^{ i(k-n)x}\textup{ uniformemente en }\mathbb{R}
            \end{split}
            \label{eq:p_3_1_1}
        \end{equation}
        pues,
        \begin{equation*}
            \abs{f(x)e^{ -inx}-s_m(x)e^{-inx}}=\abs{f(x)-s_m(x)},\quad\forall m\in\mathbb{N}^*
        \end{equation*}
        Se puede pues integrar término por término \ref{eq:p_3_1_1} en el compacto $[-\pi,\pi]$. Antes veamos que
        \begin{equation*}
            \begin{split}
                \int_{ -\pi}^{\pi}e^{ i(n-k)x}dx=\left\{
                    \begin{array}[pos]{lcr}
                        2\pi & \textup{ si } & n=k\\
                        0 & \textup{ si } & n\neq k\\
                    \end{array}
                \right.
            \end{split}
        \end{equation*}
        por tanto,
        \begin{equation*}
            \begin{split}
                \int_{-\pi}^\pi f(x)e^{ -inx}dx&=\sum_{ k\in\mathbb{Z}}\int_{-\pi}^\pi e^{ i(k-n)x}dx\\
                &= 2\pi c_n\\
                \Rightarrow c_n&=\frac{1}{2\pi}\int_{-\pi}^\pi f(x)e^{ -inx}dx\\
            \end{split}
        \end{equation*}
    \end{proof}
    
    Este resultado sugiere la definición siguiente:

    \begin{mydef}
        Para todo $f\in\mathcal{L}_1^{2\pi}(\mathbb{C})$ se define
        \begin{equation}
            c_k=\frac{1}{2\pi}\int_{ -\pi}^\pi f(x)e^{ -ikx}dx,\quad\forall k\in\mathbb{Z}
            \label{eq:coef_fourier_3}
        \end{equation}
        en particular, $c_0=\frac{1}{2\pi}\int_{ -\pi}^{\pi}f(x)dx$. Los coeficientes $c_k$ se llaman \textbf{los coeficientes de Fourier trigonométricos de $f$} y, la serie
        \begin{equation*}
            \sum_{ k\in\mathbb{Z}}c_k e^{ ikx}
        \end{equation*}
        se llama \textbf{serie de Fourier trigonométrica de $f$}.
    \end{mydef}

    \begin{obs}
        Los correspondientes coeficientes $a_k$ y $b_k$ son los siguientes:
        \begin{equation*}
            a_0=\frac{1}{\pi}\int_{-\pi}^\pi f(x)dx
            \label{eq:coef_fourier_4}
        \end{equation*}
        también,
        \begin{equation*}
            a_k=\frac{1}{\pi}\int_{-\pi}^\pi f(x)\cos kxdx\quad\textup{y}\quad b_k=\frac{1}{\pi}\int_{-\pi}^\pi f(x)\sin kxdx
        \end{equation*}
        para todo $k\in\mathbb{Z}$ (esto se obtiene usando la igualdad entre los $c_k$ y $a_k,b_k$).
    \end{obs}

    \begin{obs}
        Para fines prácticos, conviene tener en cuenta lo siguiente.
        Si $f$ es una función impar en $]-\pi,\pi[$, entonces
        \begin{equation*}
            a_k=0\quad\forall k\in\mathbb{N}^*
        \end{equation*}
        y,
        \begin{equation*}
            b_k=\frac{2}{\pi}\int_{0}^{\pi}f(x)\sen kx,\quad\forall k\in\mathbb{N}
        \end{equation*}
        Si $f$ es una función par en $]-\pi,\pi[$ se invierte el resultado, es decir
        \begin{equation*}
            a_k=\frac{2}{\pi}\int_{0}^{\pi}f(x)\cos kx,\quad\forall k\in\mathbb{N}^*
        \end{equation*}
        y,
        \begin{equation*}
            b_k=0\quad\forall k\in\mathbb{N}
        \end{equation*}
    \end{obs}

    \begin{theor}
        Las aplicaciones $f\mapsto \left\{c_k \right\}_{ k\in\mathbb{Z}}$ y, $f\mapsto\left\{a_0,a_1,b_1,... \right\}$ son aplicaciones lineales inyectivas de $L_1^{2\pi}$ en el espacio de sucesiones complejas y reales, respectivamente. En particular, si $f,g\in\mathcal{L}_1^{2\pi}$ tienen los mismos coeficientes de Fourier trigonométricos, entonces $f=g$ c.t.p. en $\mathbb{R}$.
    \end{theor}

    \begin{proof}
        Por la forma en que se definen los coeficientes de Fourier de una función integrable, es claro que dichas aplicaciones son lineales.

        Resta probar que su kernel es $\left\{0\right\}$. Sea $f\in\mathcal{L}_1^{2\pi}(\mathbb{C})$ tal que
        \begin{equation*}
            c_n=\frac{1}{2\pi}\int_{-\pi}^{\pi}f(x)e^{ inx}dx=0,\quad\forall n\in\mathbb{Z}
        \end{equation*}
        dado que el sistema trigonométrico $\tau_{\mathbb{C}}$ es total en $\mathcal{L}_1^{2\pi}(\mathbb{C})$, necesariamente $f=0$ c.t.p. en $\mathbb{R}$.

        Similarmente se prueba la otra afirmación.
    \end{proof}

    \begin{propo}
        Sean $f,g\in\mathcal{L}_1^{2\pi}(\mathbb{C})$ y $\left\{c_k \right\}_{k\in\mathbb{Z}}$ y $\left\{d_k \right\}_{ k\in\mathbb{Z}}$ los coeficientes de Fourier trigonométricos de $f$ y $g$, respectivamente. Entonces, los coeficientes de Fourier $\left\{\gamma_k \right\}$ de $f*g$ son $\left\{2\pi c_kd_k \right\}_{ k\in\mathbb{Z}}$.
    \end{propo}

    \begin{proof}
        Para todo $k\in\mathbb{Z}$ fijo se tiene lo siguiente:
        \begin{equation*}
            \begin{split}
                \gamma_k&=\frac{1}{2\pi}\int_{-\pi}^\pi f*g(x)e^{ -ikx}dx\\
                &=\frac{1}{2\pi}\int_{-\pi}^\pi e^{ -ikx}dx\int_{-\pi}^\pi f(y)g(x-y)dy\\
            \end{split}
        \end{equation*}
        como la función $(x,y)\mapsto e^{ -ikx}f(y)g(x-y)$ es integrable en $]-\pi,\pi[\times]-\pi,\pi[$ (pues la función es medible y su módulo es el mismo que el de $(x,y)\mapsto f(y)g(x-y)$, la cual es integrable por un teorema de convolución), se puede invertir del orden de integración:
        \begin{equation*}
            \begin{split}
                \gamma_k&=\frac{1}{2\pi}\int_{-\pi}^\pi f(y)dy\int_{-\pi}^\pi g(x-y)e^{ -ikx}dx\\
                &=\frac{1}{2\pi}\int_{-\pi}^\pi f(y)dy\int_{-\pi-y}^{\pi-y} g(z)e^{ -ik(z+y)}dz\\
                &=\frac{1}{2\pi}\int_{-\pi}^\pi f(y)e^{ -iky}dy\int_{-\pi-y}^{\pi-y} g(z)e^{ -ikz}dz\\
                &=\frac{1}{2\pi}\int_{-\pi}^\pi f(y)e^{ -iky}dy\int_{-\pi}^{\pi} g(z)e^{ -ikz}dz\\
                &=c_k\cdot\left(2\pi d_k \right)\\
                &=2\pi c_kd_k\\
            \end{split}
        \end{equation*}
        pues, las funciones son periódicas. Se tieene entonces con lo anterior el resultado para todo $k\in\mathbb{Z}$.
    \end{proof}

    \section{Series de Fourier de funciones en $\mathcal{L}_2^{2\pi}$}

    Recuerde que las funciones
    \begin{equation*}
        \varphi_k(x)=\frac{1}{\sqrt{2\pi}}e^{ ikx},\quad\forall k\in\mathbb{Z}
    \end{equation*}
    constituyen un sistema ortonormal maximal en el espacio Hilbertiano $L_2^{2\pi}(\mathbb{C})$. En el sentido \"Hilbertiano\", los coeficientes de Fourier de algún vector $f\in\mathcal{L}_2^{2\pi}(\mathbb{C})$ con respecto a dicho sistema ortonormal son los siguientes:
    \begin{equation*}
        \begin{split}
            \hat{f}(x)&=\pint{f}{\varphi_k}\\
            &=\frac{1}{\sqrt{2\pi}}\int_{-\pi}^{\pi}f(x)e^{ -ikx}dx\\
            &=\sqrt{2\pi}c_k\\
        \end{split}
    \end{equation*}
    luego, 
    \begin{equation*}
        \begin{split}
            \hat{f}(x)\varphi_k(x)&=\sqrt{2\pi}c_k\frac{1}{\sqrt{2\pi}}e^{ ikx}\\
            &=c_ke^{ ikx},\quad\forall k\in\mathbb{Z}\\
        \end{split}
    \end{equation*}
    La serie de Fourier hilbertiana de $f$ sería:
    \begin{equation*}
        \sum_{ k\in\mathbb{Z}}\hat{f}(x)\varphi_k(x)=\sum_{k\in\mathbb{Z}}c_ke^{ ikx}
    \end{equation*}
    que corresponde a al serie de Fourier trigonométrica de $f$. También, las funciones
    \begin{equation*}
        \frac{1}{\sqrt{2\pi}},\quad \eta_k(x)=\frac{1}{\sqrt{\pi}}\cos kx,\quad\theta_k(x)=\frac{1}{\sqrt{\pi}}\sin kx,\quad \forall k\in\mathbb{Z}
    \end{equation*}
    forman otro sistema O.N. maximal en $L_2^{2\pi}(\mathbb{K})$. Los correspondientes coeficientes de Fourier de $f$ con respecto a este sistema O.N. maximal serían:
    \begin{equation*}
        \left\{
            \begin{array}{rl}
                \pint{f}{\frac{1}{\sqrt{2\pi}}}&=\frac{1}{\sqrt{\pi}}\int_{-\pi}^{\pi}f(x)dx\\
                \pint{f}{\eta_k}&=\frac{1}{\sqrt{\pi}}\int_{-\pi}^{\pi}f(x)\cos kxdx \\
                \pint{f}{\theta_k}&=\frac{1}{\sqrt{\pi}}\int_{-\pi}^{\pi}f(x)\sen kxdx \\
            \end{array}
        \right.
    \end{equation*}
    La serie de Fourier hilbertiana de $f$ será:
    \begin{equation*}
        \begin{split}
            \pint{f}{\frac{1}{\sqrt{2\pi}}}+\sum_{ k=1}^\infty \pint{f}{\eta_k}\eta_k+\sum_{ k=1}^\infty \pint{f}{\theta_k}\theta_k=\frac{a_0}{2}+\sum_{ k=1}^\infty\left[a_k\cos kx+b_k\sin kx \right]
        \end{split}
    \end{equation*}
    Recuerde también que por el teorema de Riesz-Fischer que si $\left\{\vec{u}_\alpha \right\}_{\alpha\in\Omega}$ es un sistema O.N. maximal en un espacio hilbertiano $H$, entonces la aplicación $\vec{x}\mapsto\left\{\hat{x}(\alpha) \right\}$ es una isometría lineal de $H$ en $l_2(\Omega)$. La isometría inversa es:
    \begin{equation*}
        \varphi\mapsto \sum_{\alpha\in\Omega}\varphi(\alpha)\vec{u}_\alpha
    \end{equation*}

    Aplicacndo este resultado al primer caso se tiene que

    \begin{theor}
        Las aplicaciones
        \begin{equation*}
            f\mapsto \left\{\sqrt{2\pi}c_k \right\}_{k\in\mathbb{Z}}\quad\textup{y}\quad f\mapsto\left\{\sqrt{2\pi}\frac{a_0}{2},\sqrt{\pi}a_1,\sqrt{\pi}b_1 \right\}
        \end{equation*}
        son isometrías lineales de $L_2^{2\pi}$ sobre $l_2(\mathbb{Z})$ o $l_2(\mathbb{N})$, respectivamente. Se tienen las identidades siguientes de Parseval:
        \begin{equation*}
            \begin{split}
                \sum_{ k\in\mathbb{Z}}\abs{c_k}^2&=\frac{1}{2\pi}\int_{-\pi}^{\pi}\abs{f(x)}^2dx\\
                \frac{\abs{a_0}^2}{2}+\sum_{ k\in\mathbb{Z}}\left[\abs{a_k}^2+\abs{b_k}^2 \right]&=\frac{1}{\pi}\int_{-\pi}^{\pi}\abs{f(x)}^2dx\\
            \end{split}
        \end{equation*}
        Más generalmente, si $f,g\in\mathcal{L}_2^{2\pi}(\mathbb{K})$ con coeficientes de Fourier trigonométricos $\left\{c_k \right\}_{k\in\mathbb{Z}}$ y $\left\{d_k \right\}_{k\in\mathbb{Z}}$, entonces
        \begin{equation*}
            \sum_{ k\in\mathbb{Z}}c_k\overline{d_k}=\int_{-\pi}^{\pi}f(x)\overline{g(x)}dx
        \end{equation*}
        para los correspondientes coeficientes $\left\{a_k,b_k \right\}$ y  $\left\{\alpha_k,\beta_k \right\}$ se tiene
        \begin{equation*}
            \frac{a_0\overline{\alpha_k}}{2}+\sum_{ k\in\mathbb{Z}}\left[a_k\overline{\alpha_k}+b_k\overline{\beta_k} \right]=\int_{-\pi}^{\pi}f(x)\overline{g(x)}dx
        \end{equation*}
        Además, $f$ es igual al promedio cuadrádtico de su serie de Fourier:
        \begin{equation*}
            \lim_{ m\rightarrow\infty}\N{2}{f-s_m}=0
        \end{equation*}
    \end{theor}
    
    \begin{proof}
        Es inmediata de las observaciones hechas anteriormente y del teorema de Riesz-Fischer, junto con las identidades de Parserval.
    \end{proof}

    \begin{obs}
        Se tiene lo siguiente:
        \begin{enumerate}
            \item La suprayectividad de $f\mapsto\left\{\sqrt{2\pi}c_k \right\}_{ k\in\mathbb{Z}}$ de $L_2^{2\pi}(\mathbb{C})$ sobre $l_2(\mathbb{Z})$ es consecuencia del teorema de Riesz-Fischer, es decir, de la completez de $L_2^{2\pi}$. Dice que dada una sucesión arbitraria $\left\{c_k\right\}_{k\in\mathbb{Z}}$ en $l_2(\mathbb{Z})$ existe una función $f\in\mathcal{L}_2^{2\pi}$ única salvo equivalencias cuyos coeficientes de Fourier son la sucesión dada.
            
            Este resultado fue históricamente un éxito para la integral de Lebesgue.

            \item Carleson demostró en 1966 que para cada $f\in\mathcal{L}_2^{2\pi}$ la serie de Fourier de $f$ converge a $f$ c.t.p. en $\mathbb{R}$. Sin embargo, para funciones en $\mathcal{L}_1^{2\pi}$ ésta no será la misma historia.
        \end{enumerate}
    \end{obs}

    \section{Series de Fourier de funciones de periodo $T>0$}

    Sea $f\in\mathcal{L}_1^T$. Defina
    \begin{equation*}
        g(y)=f\left(\frac{T}{2\pi}y\right),\quad\forall y\in\mathbb{R}
    \end{equation*}
    entonces, $g\in\mathcal{L}_1^{2\pi}$. Por definición, los coeficientes de Fourier de $f$ van a ser los de $g$, estos son
    \begin{equation*}
        c_k=\frac{1}{2\pi}\int_{-\pi}^\pi g(y)e^{ iky}dy
    \end{equation*}
    Por el cambio de variable $y=\frac{2\pi}{T}x$, podemos reescribirlos de la siguiente forma:
    \begin{equation*}
        \begin{split}
            c_k=\frac{1}{T}\int_{ -\frac{T}{2}}^{\frac{T}{2}}f(x)e^{ i\frac{2\pi k}{T}x}dx
        \end{split}
    \end{equation*}
    en particular, $c_0=\frac{1}{T}\int_{ -\frac{T}{2}}^{\frac{T}{2}}f(x)dx$. Los correspondientes $a_k$ y $b_k$ son 
    \begin{equation*}
        \begin{split}
            a_0&=\frac{2}{T}\int_{ -\frac{T}{2}}^{\frac{T}{2}}f(x)dx\\
            a_k&=\frac{1}{T}\int_{ -\frac{T}{2}}^{\frac{T}{2}}f(x)\cos\left(\frac{2\pi k}{T}x \right)dx\\
            b_k&=\frac{1}{T}\int_{ -\frac{T}{2}}^{\frac{T}{2}}f(x)\sin\left(\frac{2\pi k}{T}x \right)dx\\
        \end{split}
    \end{equation*}
    Las series de Fourier trigonométricas correspondientes son
    \begin{equation*}
        \sum_{ k\in\mathbb{Z}}c_ke^{i\frac{2\pi k}{T}x}=\frac{a_0}{2}+\sum_{ k=1}^\infty\left[a_k\cos\left(\frac{2\pi k}{T}x \right)+b_k\sin\left(\frac{2\pi k}{T}x \right)\right]
    \end{equation*}
    Sea ahora $f\in\mathcal{L}_2^T$. Los coeficientes de Fourier de $f$ con respecto al sistema O.N. maximal formado por
    \begin{equation*}
        \varphi_k(x)=\frac{1}{\sqrt{T}}e^{ i\frac{2\pi k}{T}x},\quad\forall k\in\mathbb{Z}
    \end{equation*}
    son
    \begin{equation*}
        \pint{f}{\varphi_k}=\sqrt{T}c_k\quad\forall k\in\mathbb{Z}
    \end{equation*}
    si se usa el sistema O.N. maximal formado por
    \begin{equation*}
        \frac{1}{\sqrt{T}},\quad\eta_k(x)=\sqrt{\frac{2}{T}}\cos\left(\frac{2\pi k}{T}x \right),\quad\theta_k(x)=\sqrt{\frac{2}{T}}\sin\left(\frac{2\pi k}{T}x \right),\quad\forall k\in\mathbb{N}
    \end{equation*}
    se obtienen
    \begin{equation*}
        \begin{split}
            \pint{f}{\frac{1}{\sqrt{T}}}&=\sqrt{T}\frac{a_0}{2}\\
            \pint{f}{\eta_k}&=\sqrt{\frac{T}{2}}a_k\\
            \pint{f}{\theta_k}&=\sqrt{\frac{T}{2}}b_k\\
        \end{split}
    \end{equation*}
    para todo $k\in\mathbb{N}$. Las series de Fourier correspondientes serían
    \begin{equation*}
        \sum_{ k\in\mathbb{Z}}\pint{f}{\varphi_k}\varphi_k=\sum_{ k\in\mathbb{Z}}e^{ i\frac{2\pi k}{T}x}=\frac{a_0}{2}+\sum_{ k=1}^\infty\left[a_k\cos\left(\frac{2\pi k}{T}x \right)+b_k\sin\left(\frac{2\pi k}{T}x \right) \right]
    \end{equation*}
    Se tienen las identidades de Parserval
    \begin{equation*}
        \begin{split}
            \sum_{ k\in\mathbb{Z}}\abs{c_k}^2&=\frac{1}{T}\int_{ -\frac{T}{2}}^{\frac{T}{2}}\abs{f(x)}^2\\
            \frac{\abs{a_0}^2}{2}+\sum_{ k=1}^\infty\left[\abs{a_k}^2+\abs{b_k}^2 \right]&=\frac{T}{2}\int_{ -\frac{T}{2}}^{\frac{T}{2}}\abs{f(x)}dx\\
            \sum_{ k\in\mathbb{Z}}c_k\overline{d_k}&=\frac{1}{T}\int_{ -\frac{T}{2}}^{\frac{T}{2}}f(x)\overline{g(x)}dx\\
            \frac{a_0\overline{\alpha_0}}{2}+\sum_{ k=1}^\infty\left[a_k\overline{\alpha_k}+b_k\overline{\beta_k} \right]&=\frac{2}{T}\int_{ -\frac{T}{2}}^{\frac{T}{2}}f(x)\overline{g(x)}dx
        \end{split}
    \end{equation*}
    por lo cual, en lo sucesivo se trabajará únicamente con funciones de periodo $2\pi$ (la traducción al periodo $T>0$ es un ejercicio).

    \section{Convergencia de series de Fourier de integrales indefinidas}

    \begin{obs}
        Sea $f\in\mathcal{L}_1^{2\pi}(\mathbb{K})$. Considere la integral indefinida $\cf{F}{\mathbb{R}}{\mathbb{K}}$ dada como sigue
        \begin{equation*}
            F(x)=c+\int_0^xf(t)dt,\quad\forall x\in\mathbb{R}
        \end{equation*}
        (en otras palabras $F$ es absolutamente continua y $f$ es su derivada c.t.p. en $\mathbb{R}$). Se sabe que $F$ es continua en $\mathbb{R}$. Una condición necesaria y suficiente para que también $F$ sea periódica es la siguiente:
        \begin{equation*}
            \begin{split}
                F(x-2\pi)-F(x)&=\int_x^{ x+2\pi}f(t)dt\\
                &=\int_{-\pi}^{\pi}f(t)dt\\
                &=0\\
            \end{split}
        \end{equation*}
        por tanto, $F\in\mathcal{C}^{2\pi}$ si y sólo si $\int_{-\pi}^\pi f(t)dt=0$.
    \end{obs}

    \begin{theor}
        Sea $f\in\mathcal{L}_2^{2\pi}(\mathbb{K})$ tal que $\int_{-\pi}^\pi f(t)dt=0$ y sea $\cf{F}{\mathbb{R}}{\mathbb{K}}$ la integral indefinida de $f$ dada por:
        \begin{equation*}
            F(x)=c+\int_{ 0}^xf(t)dt,\quad\forall x\in\mathbb{R}
        \end{equation*}
        donde $c\in\mathbb{K}$. Entonces, $F\in\mathcal{C}^{2\pi}(\mathbb{K})$ y la serie de Fourier de $F$ converge a $F$ uniformemente en $\mathbb{R}$.
    \end{theor}

    \begin{proof}
        Ya se sabe que $F\in\mathcal{C}^{2\pi}(\mathbb{K})$. Sean $\left\{c_k \right\}_{ k\in\mathbb{Z}}$ y $\left\{c_k' \right\}_{ k\in\mathbb{Z}}$ los coeficientes de Fourier de $F$ y $f$, respectivamente. Note que
        \begin{equation*}
            c_0'=\frac{1}{2\pi}\int_{ -\pi}^{\pi}f(t)dt=0
        \end{equation*}
        Integrando por partes se tiene que
        \begin{equation*}
            \begin{split}
                c_k'&=\frac{1}{2\pi}\int_{-\pi}^{\pi}f(x)e^{ -ikx}dx\\
                &=\frac{1}{2\pi}\left[F(x)e^{ -ikx}\Big|_{-\pi}^{\pi}+ik\int_{-\pi}^{\pi}F(x)e^{ -ikx}dx \right]\\
                &=\frac{ik}{2\pi}\int_{-\pi}^{\pi}F(x)e^{ -ikx}dx\\
                &=ikc_k,\quad\forall k\in\mathbb{Z}\backslash\left\{0 \right\} \\
            \end{split}
        \end{equation*}
        Por tanto, en particular se tiene que
        \begin{equation*}
            \abs{c_k}=\frac{\abs{c_k'}}{\abs{k}},\quad\forall k\in\mathbb{Z}\backslash\left\{0 \right\}
        \end{equation*}
        Se tiene
        \begin{equation*}
            \begin{split}
                \abs{c_k}&=\frac{\abs{c_k'}}{\abs{k}}\\
                &\leq\frac{1}{2}\cdot\left[\abs{c_k'}^2+\frac{1}{k^2} \right],\quad\forall k\in\mathbb{Z}\backslash\left\{0 \right\}\\
            \end{split}
        \end{equation*}
        como $f\in\mathcal{L}_2^{2\pi}(\mathbb{K})$, entonces $\left\{\abs{c_k'} \right\}_{ k\in\mathbb{Z}}\in l_2(\mathbb{Z})$ de donde $\sum_{k\in\mathbb{Z}}\abs{c_k´}^2<\infty$. Se sigue de la ecuación anterior que la serie $\sum_{ k\in\mathbb{Z}}c_k$ es absolutamente convergente en $\mathbb{K}$. Ya que
        \begin{equation*}
            \sum_{ k\in\mathbb{Z}}\abs{c_ke^{ ikx}}\leq\sum_{ k\in\mathbb{Z}}\abs{c_k'}
        \end{equation*}
        se sigue del critero $M$ de Weierestrass que la serie de Fourier de $F$ converge absoluta y uniformemente en $\mathbb{R}$. Resta probar que la suma de esta serie es $F$. Sea
        \begin{equation*}
            G(x)=\sum_{ k\in\mathbb{Z}}\textup{ uniformemente en }\mathbb{R}
        \end{equation*}
        Entonces, $G\in\mathcal{C}^{2\pi}(\mathbb{K})$. Además,
        \begin{equation*}
            G(x)e^{ -inx}=\sum_{ k\in\mathbb{Z}}c_ke^{ i(k-n)x}\textup{ uniformemente en }\mathbb{R}
        \end{equation*}
        Se puede pues integrar término por término en $[-\pi,\pi]$. Resulta:
        \begin{equation*}
            \int_{ -\pi}^\pi G(x)e^{ -inx}dx=\sum_{ k\in\mathbb{Z}}c_k\int_{ -\pi}^\pi e^{ i(k-n)x}dx=2\pi c_n
        \end{equation*}
        por tanto, $F$ y $G$ tienen los mismos coeficientes de Fourier. Se sabe entonces que $F=G$ c.t.p. en $\mathbb{R}$ siendo ambas continuas, necesariamente $F=G$ en $\mathbb{R}$.
    \end{proof}

    \begin{cor}
        Si $\cf{F}{\mathbb{R}}{\mathbb{K}}$ es una función de clase $C^1$ periódica de periodo $2\pi$, entonces la serie de Fourier de $F$ converge a $F$ uniformemente en $\mathbb{R}$.
    \end{cor}

    \begin{proof}
        Por el teorema fundamental del cálculo, podemos escribir
        \begin{equation*}
            F(x)=F(0)+\int_{0}^xf(t)dt,\quad\forall x\in\mathbb{R}
        \end{equation*}
        donde $f(x)=F'(x)$ para todo $x\in\mathbb{R}$ es una función continua. Por el teorema anterior, el resultado estará probado si se muestra que $f$ es periódica de periodo $2\pi$.

        Ya que $F(x)=F(x+2\pi)$, entonces del teorema fundamental
        \begin{equation*}
            \begin{split}
                \int_{x}^{ x+2\pi}f(t)dt=0,\quad\forall x\in\mathbb{R}
            \end{split}
        \end{equation*}
        en particular, $\int_{ -\pi}^{\pi}f(t)dt=0$. Para todo $a<b$ se tiene lo siguiente:
        \begin{equation*}
            \begin{split}
                \int_a^bf(x+2\pi)dx&=\int_{ a+2\pi}^{ b+2\pi}f(t)dt\\
                &=\int_{ a+2\pi}^{a}f(t)dt+\int_a^bf(t)dt+\int_b^{ b+2\pi}f(t)dt \\
                &=\int_a^bf(t)dt\\
                \Rightarrow \int_{ a}^b\left[f(x+2\pi)-f(x) \right]dx&=0,\quad\forall a,b\in\mathbb{R}, a<b \\
            \end{split}
        \end{equation*}
        Por el lema de los promedios, se sigue que $f(x+2\pi)-f(x)=0$ para casi toda $x\in\mathbb{R}$. Siendo ambas funciones continuas, se sigue que
        \begin{equation*}
            f(x+2\pi)=f(x),\quad\forall x\in\mathbb{R}
        \end{equation*}
        por tanto $f$ es periódica de periodo $2\pi$.
    \end{proof}

    \begin{obs}
        Como $c_k=\frac{c_k'}{ik}$ para todo $k\in\mathbb{Z}\backslash\left\{0 \right\}$, el término $c_ke^{ ikx}$ de la serie de Fourier de $F$ es una primitiva del térmio $c_k'e^{ ikx}$ de la serie de Fourier de $f$. En este caso $c_0$ juega el papel de constante de integración.
        
        Al considerar los coeficientes $a_k$, $b_k$ y $a_k'$, $b_k'$ resulta lo siguiente:
        \begin{equation*}
            \begin{split}
                a_k-ib_k&=\frac{a_k'-ib_k'}{ik},\quad\forall k\in\mathbb{Z}\backslash\left\{0 \right\} \\
            \end{split}
        \end{equation*}
        cambiando a $k$ por $-k$:
        \begin{equation*}
            a_k+ib_k=\frac{a_k'+ib_k'}{ik},\quad\forall k\in\mathbb{Z}\backslash\left\{0 \right\}
        \end{equation*}
        resulta de estos sistemas de ecuaciones que
        \begin{equation*}
            a_k=-\frac{b_k'}{k}\quad\textup{y}\quad b_k=\frac{a_k'}{k}
        \end{equation*}
        para todo $k\in\mathbb{N}$. Así pues, $a_k\cos kx$ es una primitiva de $b_k'\sen kx$ y que $b_k\sen kx$ es una primitiva de $a_k\cos kx$.
    \end{obs}

    Se verá más adelante que la conclusión del teorema anterior es válida si $f\in\mathcal{L}_1^{2\pi}(\mathbb{K})$.

    \begin{exa}
        Sea $n\in\mathbb{N}^*$. Considere las funciones $\cf{f,g}{\mathbb{R}}{\mathbb{R}}$ las funciones siguiente:
        \begin{equation*}
            f(x)=\cos nx\quad\textup{y}\quad g(x)=\sen nx,\quad\forall x\in\mathbb{R}
        \end{equation*}
        Claramente $f$ y $g$ son funciones de clase $C^1$ en $\mathbb{R}$ periódicas de periodo $2\pi$. Por el último corolario las series de Fourier de $f$ y $g$ convergen a $f$ y $g$ respectivamente, uniformemente en $\mathbb{R}$. Los correspondientes coeficientes de Fourier de $f$ y $g$ serían:
        \begin{equation*}
            \begin{split}
                \frac{a_0}{2}\sum_{ k=1}^\infty\left[a_k\cos kx+b_k\sin kx \right]
            \end{split}
        \end{equation*}
        como $f$ es par, entonces $b_k=0$ para todo $k\in\mathbb{N}$. Además,
        \begin{equation*}
            \begin{split}
                a_k&=\frac{1}{\pi}\int_{-\pi}^{\pi}f(x)\cos kxdx\\
                &=\frac{1}{\pi}\int_{-\pi}^{\pi}\cos nx \cos kxdx\\
                &=\frac{1}{\pi}\pint{\cos nx}{\cos kx}\\
                &=\left\{
                    \begin{array}{lcr}
                       0 & \textup{ si } & k\neq n\\
                       1 & \textup{ si } & k=n\\ 
                    \end{array}
                \right. \\
            \end{split}
        \end{equation*}
        por tanto, $f(x)=\cos nx=\frac{a_0}{2}+\sum_{ k=1}^\infty \cos kx=\cos nx$. Similarmente se prueba que $g(x)=\sen nx$ es su desarrollo en serie de Fourier.
    \end{exa}

    \begin{exa}
        Considere la función $\cf{F}{\mathbb{R}}{\mathbb{R}}$ dada como sigue
        \begin{equation*}
             F(x)=\sen^3 x,\quad\forall x\in\mathbb{R}
        \end{equation*}
        $F$ es de clase $C^1$ y periódica de periodo $2\pi$. Por el corolario, la serie de Fourier de $F$ converge a $F$ uniformemente en $\mathbb{R}$. Como $F$ es impar, $a_k=0$ para todo $k\in\mathbb{N}^*$.

        Se tiene que
        \begin{equation*}
            \begin{split}
                b_1&=\frac{1}{\pi}\int_{ -\pi}^\pi \sen^3 x\sen xdx\\
                &=\frac{1}{\pi}\int_{ -\pi}^\pi \sen^4 xdx\\
                &=\frac{1}{\pi}\int_{ -\pi}^\pi \left(\frac{1-\cos 2x}{2} \right)^2 xdx\\
                &=\frac{1}{4\pi}\int_{ -\pi}^\pi 1-\cos 2x+\cos^2 2x xdx\\
                &=\frac{1}{4\pi}\left[2\pi-\pint{2}{\cos 2x}+\int_{ -\pi}^\pi\cos^2 2xdx \right] \\
                &=\frac{1}{2}+\frac{1}{4\pi}\int_{ -\pi}^\pi\frac{1+\cos 4x}{2}dx\\
                &=\frac{1}{2}+\frac{1}{8\pi}\left[2\pi+\pint{1}{\cos 4x} \right] \\
                &=\frac{3}{4}\\
            \end{split}
        \end{equation*}
        en general
        \begin{equation*}
            \begin{split}
                b_k&=\frac{1}{\pi}\int_{-\pi}^\pi\sen^3x\sen kxdx\\
                &=\frac{1}{\pi}\int_{-\pi}^\pi\sen^2x\sen x \sen kxdx\\
                &=\frac{1}{\pi}\int_{-\pi}^\pi\frac{1-\cos 2x}{2}\sen x\sen x \sen kxdx\\
                &=\frac{1}{2\pi}\pint{\sen x}{\sen kx}-\frac{1}{2\pi}\int_{-\pi}^\pi\cos2x\sen x\sen kxdx\\
                &=-\frac{1}{4\pi}\int_{-\pi}^\pi\left[\cos(k-1)x+\cos(k+1)x \right]\cos 2xdx \\
                &=\frac{1}{2\pi}\pint{\cos (k+1)x}{\cos 2x}-\frac{1}{4}\pint{\frac{\cos (k-1)x}{\sqrt{\pi}}}{\frac{\cos 2x}{\sqrt{\pi}}}\\
                &=\left\{
                    \begin{array}{lcr}
                        0 & \textup{ si } & k=2,4,5,...\\
                        -\frac{1}{4} & \textup{ si } & k=3\\
                    \end{array}
                \right.
            \end{split}
        \end{equation*}
        luego, $F(x)=\frac{3}{4}\sen x-\frac{1}{4}\sen 3x$ para todo $x\in\mathbb{R}$.
    \end{exa}

    \begin{exa}
        ¿Qué pasa con la función $x\mapsto\abs{\sen x}$ y su serie de Fourier?
    \end{exa}

    \section{Teorema fundamental para la convergencia puntual de series de Fourier en $\mathcal{L}_1^{2\pi}$}

    \begin{theor}[\textbf{Teorema de Riemman-Lebesgue}]
        Sean $a,b\in\mathbb{R}$ con $a<b$. Si $f\in\mathcal{L}_1(]a,b[,\mathbb{K})$, entonces
        \begin{equation*}
            \lim_{\lambda\in\mathbb{R},|\lambda|\rightarrow\infty }\int_{a}^bf(x)e^{ i\lambda x}dx=0
        \end{equation*}
    \end{theor}

    \begin{proof}
        Se harán varias cosas:
        \begin{enumerate}
            \item Suponga que $f=\chi_I$ con $I\subseteq ]a,b[$ es un intervalo de extremos $\alpha\leq\beta$. Luego,
            \begin{equation*}
                \begin{split}
                    \int_{a}^b f(x)e^{ i\lambda x}dx&=\int_\alpha^\beta e^{ i\lambda x}dx\\
                    &=\frac{1}{i\lambda}\left(e^{ i\lambda\beta}-e^{ i\lambda\alpha} \right)\\
                    \Rightarrow \abs{\int_{a}^b f(x)e^{ i\lambda x}dx}&\leq\frac{2}{\abs{\lambda}}\\
                \end{split}
            \end{equation*}
            donde el lado de la derecha tiende a cero conforme $\abs{\lambda}\rightarrow\infty$. Por tanto,
            \begin{equation*}
                \lim_{\lambda\in\mathbb{R},|\lambda|\rightarrow\infty }\int_{a}^bf(x)e^{ i\lambda x}dx=0
            \end{equation*}
            por linealidad el resultado es válido si $f$ es una función escalonada en el abierto $]a,b[$.
            \item Suponga que $f\in\mathcal{L}_1^(]a,b[,\mathbb{K})$ y sea $\varepsilon>0$. Se sabe que existe $\varphi\in\mathcal{E}(]a,b[,\mathbb{K})$ tal que
            \begin{equation*}
                \N{1}{f-\varphi}<\frac{\varepsilon}{2}
            \end{equation*}
            También existe $R>0$ tal que
            \begin{equation*}
                \lambda\in\mathbb{R}, \abs{\lambda}>R\Rightarrow\abs{\int_a^b\varphi(x)e^{ i\lambda x}dx}<\frac{\varepsilon}{2}
            \end{equation*}
            entonces, si $\lambda\in\mathbb{R}$ y $\abs{\lambda}>R$, se tiene que
            \begin{equation*}
                \begin{split}
                    \abs{\int_a^b f(x)e^{i\lambda x}dx}&\leq\abs{\int_a^b \left[f(x)-\varphi(x) \right]e^{i\lambda x}dx}+\abs{\int_a^b \varphi(x)e^{i\lambda x}dx}\\
                    &< \int_a^b\abs{f(x)-\varphi(x)}dx+\frac{\varepsilon}{2}\\
                    &<\frac{\varepsilon}{2}+\frac{\varepsilon}{2}\\
                    &=\varepsilon\\
                \end{split}
            \end{equation*}
            lo que termina la prueba.
        \end{enumerate}
    \end{proof}

    \begin{obs}
        Se tiene lo siguiente:
        \begin{enumerate}
            \item Si $\cf{f}{]a,b[}{\mathbb{C}}$ es integrable en un conjunto medible $B\subseteq ]a,b[$ ,entonces
            \begin{equation*}
                \lim_{\lambda\in\mathbb{R},|\lambda|\rightarrow\infty }\int_Bf(x)e^{ i\lambda x}dx=0
            \end{equation*}
            pues, $f\chi_B$ es integrable en $]a,b[$.
            \item Si $f\in\mathcal{L}_1(]a,b[,\mathbb{C})$ al escribir $e^{ i\lambda x}=\cos \lambda x+i\sen \lambda x$ el Teorema de Riemman-Lebesgue implica que
            \begin{equation*}
                \lim_{\lambda\in\mathbb{R},|\lambda|\rightarrow\infty }\int_Bf(x)\cos \lambda x dx=0
            \end{equation*}
            y
            \begin{equation*}
                \lim_{\lambda\in\mathbb{R},|\lambda|\rightarrow\infty }\int_Bf(x)\sin \lambda x dx=0
            \end{equation*}
            \item Recuerde que si $f\in\mathcal{L}_1^{2\pi}(\mathbb{C})$, se definió:
            \begin{equation*}
                c_n=\frac{1}{2\pi}\int_{-\pi}^\pi f(x)e^{ inx}dx,\quad\forall n\in\mathbb{Z}
            \end{equation*}
            Por el Teorema de Riemman-Lebesgue
            \begin{equation*}
                \lim_{ n\rightarrow\infty}c_n=0
            \end{equation*}
            además,
            \begin{equation*}
                \lim_{ k\rightarrow\infty}a_k=\lim_{ k\rightarrow\infty}b_k=0
            \end{equation*}
        \end{enumerate}
    \end{obs}

    Se denota por $c_0(\mathbb{Z})$ al espacio vectorial de todas las sucesiones $\left\{c_n \right\}_{ n\in\mathbb{Z}}$ tal que
    \begin{equation*}
        \lim_{|n| \rightarrow\infty}c_n=0
    \end{equation*}
    $c_0(\mathbb{Z})$ es un subespacio del espacio de Banach $(l_\infty(\mathbb{Z}),\N{\infty}{\cdot})$. Se demuestra que $c_0(\mathbb{Z})$ es un subespacio cerrado de $(l_\infty(\mathbb{Z}),\N{\infty}{\cdot})$, luego $(c_0(\mathbb{Z}),\N{\infty}{\cdot})$ también es de Banach.

    Por cierdo, $(l_\infty(\mathbb{Z}),\N{\infty}{\cdot})\equiv(\mathcal{B}(\mathbb{Z},\mathbb{C}),\N{\infty}{\cdot})$. De hecho, $(c_0(\mathbb{Z}),\N{\infty}{\cdot})$ es un álgebra de Banach con el producto:
    \begin{equation*}
        \left\{c_n \right\}_{ n\in\mathbb{Z}}\cdot\left\{d_n \right\}_{ n\in\mathbb{Z}}=\left\{c_n\cdot d_n \right\}_{ n\in\mathbb{Z}}
    \end{equation*}

    \begin{propo}
        La aplicación $f\mapsto\left\{c_k \right\}_{ k\in\mathbb{Z}}$ es una aplicación lineal continua inyectiva de $L_1^{2\pi}(\mathbb{C})$ en $c_0(\mathbb{Z})$.
    \end{propo}

    \begin{proof}
        Ya se sabe que dicha aplicación es lineal e inyectiva. Veamos que
        \begin{equation*}
            \begin{split}
                \abs{c_k}&=\abs{\frac{1}{2\pi}\int_{ -\pi}^{\pi}f(x)e^{ ikx}dx}\\
                &\leq\frac{1}{2\pi}\int_{ -\pi}^{\pi}\abs{f(x)}dx\\
                &=\frac{1}{2\pi}\N{1}{f},\quad\forall k\in\mathbb{Z}\\
                \Rightarrow \N{\infty}{\left\{c_k \right\}_{ k\in\mathbb{Z}}}&\leq\frac{1}{2\pi}\N{1}{f}\\
            \end{split}
        \end{equation*}
        Por tanto, esta aplicación lineal es continua y de norma menor o igual a $\frac{1}{2\pi}$.
    \end{proof}

    \begin{obs}
        En los ejercicios se verá que dicha aplicación lineal no es suprayectiva (a diferencia del caso en $L_2^{2\pi}$).
    \end{obs}

    \section{Núcleo de Dirichlet y teorema fundamental}

    Para determinar la posible convergencia puntual de una serie de Fourier se debe analizar la sucesión de sumas parciales en un punto $x\in\mathbb{R}$. Recuerde que 
    \begin{equation*}
        s_m(x)=\sum_{ k=-m}^m c_ke^{ ikx},\quad\forall m\in\mathbb{N}^*
    \end{equation*}
    Sustituyendo $c_n=\frac{1}{2\pi}\int_{-\pi}^\pi f(t)e^{ -int}dt$, se obtiene que
    \begin{equation*}
        \begin{split}
            s_m(x)&=\frac{1}{2\pi}\sum_{ k=-m}^m\int_{-\pi}^\pi f(t)e^{ ik(x-t)}dt\\
            &=\frac{1}{2\pi}\int_{-\pi}^\pi f(t)\left[\sum_{ k=-m}^m e^{ ik(x-t)}\right]dt ,\quad\forall m\in\mathbb{N}^*
        \end{split}
    \end{equation*}
    Entonces,
    \begin{equation*}
        s_m(x)=f*D_m(x)\quad\forall m\in\mathbb{N}^*
    \end{equation*}
    donde
    \begin{equation*}
        D_m(x)=\frac{1}{2\pi}\sum_{ k=-m}^m e^{ ikx}
    \end{equation*}
    es el llamado \textbf{Núcleo de Dirichlet}.

    Una expresión alternativa para este núcleo es
    \begin{equation*}
        D_m(x)=\frac{1}{2\pi}\sum_{ k=-m}^me^{ ikx}=\frac{1}{2\pi}\left[1+\sum_{ k=1}^m\left(e^{ ikx}+e^{ -ikx} \right) \right]=\frac{1}{2\pi}\left[2\Re\left(\sum_{ k=1}^m e^{ ikx} \right)-1 \right]
    \end{equation*}
    para todo $m\in\mathbb{N}^*$. Tenemos además que,
    \begin{equation*}
        \begin{split}
            \sum_{ k=1}^m e^{ ikx}&=\frac{1-e^{ i(m+1)x}}{1-e^{ix}}\\
            &=\frac{e^{ -i\frac{x}{2}}-e^{ i(m+\frac{1}{2})x}}{e^{ -i\frac{x}{2}}-e^{i\frac{x}{2}}}\\
            &=\frac{e^{ -i\frac{x}{2}}-e^{ i(m+\frac{1}{2})x}}{-2i\sen\frac{x}{2}}\\
            &=\frac{\cos\frac{x}{2}-\cos\left(m+\frac{1}{2}\right)x}{-2i\sen\frac{x}{2}}+i\frac{-\sen\frac{x}{2}-\sen\left(m+\frac{1}{2}\right)x}{-2i\sen\frac{x}{2}}\\
        \end{split}
    \end{equation*}
    así pues,
    \begin{equation*}
        \begin{split}
            \Re\left(\sum_{ k=1}^m e^{ ikx} \right)&=\frac{1}{2}\left[\frac{\sen\frac{x}{2}+\sen\left(m+\frac{1}{2} \right)x}{\sen\frac{x}{2}} \right]\\
            &=\frac{1}{2}\left[1+\frac{\sen\left(m+\frac{1}{2} \right)x}{\sen\frac{x}{2}} \right]\\
        \end{split}
    \end{equation*}
    sustituyendo en el núcleo de Dirichlet se sigue que
    \begin{equation*}
        D_m(x)=\frac{1}{2\pi}\cdot\frac{\sen\left(m+\frac{1}{2} \right)x}{\sen\frac{x}{2}}
    \end{equation*}
    si $x$ no es múltiplo entero de $2\pi$. En caso contrario obtenemos que
    \begin{equation*}
        D_m(x)=\frac{2m+1}{2\pi}
    \end{equation*}
    Note además que por definición de $D_m(x)$:
    \begin{equation*}
        1=\int_{-\pi}^\pi D_m(x)dx=\frac{1}{2\pi}\int_{-\pi}^{\pi}\frac{\sen\left(m+\frac{1}{2} \right)x}{\sen\frac{x}{2}}dx
    \end{equation*}

    \begin{theor}[\textbf{Teorema fundamental para la convergencia puntual de una serie de Fourier}]
        Sea $f\in\mathcal{L}_1^{2\pi}(\mathbb{C})$ y fijemos $x\in\mathbb{R}$. Para que la serie de Fourier de $f$ converja en $x$ a una suma $s(x)$ (\textit{finita}), es necesario y suficiente que para algún $0<\delta<\pi$ se cumpla alguna de las dos condiciones siguientes:
        \begin{enumerate}
            \item $\lim_{ m\rightarrow\infty}\int_{-\delta}^\delta \frac{f(x+t)-s(x)}{t}\sen\left(m+\frac{1}{2} \right)tdt=0$.
            \item $\lim_{ m\rightarrow\infty}\int_{0}^\delta \frac{f(x+t)+f(x-t)-2s(x)}{t}\sen\left(m+\frac{1}{2} \right)tdt=0$.
        \end{enumerate}
    \end{theor}

    \begin{proof}
        Probaremos que las integrales (1) y (2) son equivalentes (es decir que son la misma integral). En efecto,
        \begin{equation*}
            \begin{split}
                \int_{-\delta}^\delta \frac{f(x+t)-s(x)}{t}\sen\left(m+\frac{1}{2} \right)tdt&=\int_{-\delta}^0\frac{f(x+t)-s(x)}{t}\sen\left(m+\frac{1}{2} \right)tdt\\
                +&\int_0^\delta \frac{f(x+t)-s(x)}{t}\sen\left(m+\frac{1}{2} \right)tdt\\
                &=\int_0^\delta \frac{f(x-u)-s(x)}{-u}\sen\left(m+\frac{1}{2} \right)(-u)du\\
                +&\int_0^\delta \frac{f(x+t)-s(x)}{t}\sen\left(m+\frac{1}{2} \right)tdt\\
                &=\int_0^\delta\frac{f(x+t)+f(x-t)-2s(x)}{t}\sen\left( m+\frac{1}{2}\right)tdt\\
            \end{split}
        \end{equation*}
        Para la demostración, recordemos que
        \begin{equation*}
            s_m(x)=f*D_m(x)=\frac{1}{2\pi}\int_{-\pi}^{\pi}f(x-u)\frac{\sen\left(m+\frac{1}{2} \right)u}{\sen\frac{u}{2}}du=\frac{1}{2\pi}\int_{-\pi}^{\pi}f(x+u)\frac{\sen\left(m+\frac{1}{2} \right)u}{\sen\frac{u}{2}}du
        \end{equation*}
        además,
        \begin{equation*}
            s(x)=s(x)\int_{-\pi}^\pi D_m(u)du=\int_{-\pi}^{\pi}s(x)\frac{\sen\left(m+\frac{1}{2} \right)u}{\sen\frac{u}{2}}du
        \end{equation*}
        por ende,
        \begin{equation*}
            \begin{split}
                s_m(x)-s(x)&=\frac{1}{2\pi}\int_{-\pi}^{\pi}\frac{f(x+t)-s(x)}{\sen\frac{t}{2}}\sen\left(m+\frac{1}{2} \right)tdu
            \end{split}
        \end{equation*}
        Sea $0<\delta<\pi$, sentonces
        \begin{equation*}
            \begin{split}
                s_m(x)-s(x)&=\frac{1}{2\pi}\left[\int_{-\delta}^{\delta}\frac{f(x+t)-s(x)}{\sen\frac{t}{2}}\sen\left(m+\frac{1}{2} \right)tdt+\int_{\delta\leq|t|<\pi}\frac{f(x+t)-s(x)}{\sen\frac{t}{2}}\sen\left(m+\frac{1}{2} \right)tdt\right] \\
            \end{split}
        \end{equation*}
        Como $t\mapsto\frac{f(x+t)-s(x)}{\sen\frac{t}{2}}$ es integrable en $\delta\leq\abs{t}<\pi$, por el teorema de Riemman-Lebesgue la segunda integral tiende a cero conforme $m\rightarrow\infty$. Por lo tanto,
        \begin{equation*}
            \lim_{ m\rightarrow\infty}[s_m(x)-s(x)]=0\iff\lim_{ m\rightarrow\infty}\int_{-\delta}^{\delta}\frac{f(x+t)-s(x)}{2\sen\frac{t}{2}}\sen\left(m+\frac{1}{2} \right)tdt=0
        \end{equation*}
        Veamos que
        \begin{equation*}
            \begin{split}
                \int_{-\delta}^{\delta}\frac{f(x+t)-s(x)}{2\sen\frac{t}{2}}\sen\left(m+\frac{1}{2} \right)tdt&=\int_{-\delta}^{\delta}\frac{f(x+t)-s(x)}{t}\sen\left(m+\frac{1}{2} \right)tdt\\
                +&\int_{-\delta}^{-\delta}\left(f(x+t)-s(x)\right)\left[\frac{1}{2\sen\frac{t}{2}}-\frac{1}{t} \right]\sen\left( m+\frac{1}{2} \right)tdt\\
            \end{split}
        \end{equation*}
        Pero,
        \begin{equation*}
            \begin{split}
                \frac{1}{2\sen\frac{t}{2}}-\frac{1}{t}&=\frac{t-2\sen\frac{t}{2}}{2t\sen\frac{t}{2}}\\
                &\overset{t\rightarrow0}{=}\frac{t-2\left(\frac{t}{2}-\frac{t^3}{48}+O(t^3) \right)}{2t(\frac{t}{2}-O(t))}\\
                &\overset{t\rightarrow0}{=}\frac{\frac{t^3}{48}+O(t^3)}{t^2+2tO(t)}\\
                &\overset{t\rightarrow0}{=}\frac{\frac{t^3}{48}+O(t^3)}{t^2+O(t^2)}\\
                &\overset{t\rightarrow0}{=}\frac{t^3\left(\frac{1}{48}+\frac{O(t^3)}{t^2} \right)}{t^2(1+\frac{O(t^2)}{t^2})}\\
                &\overset{t\rightarrow0}{=}\frac{t\left(\frac{1}{48}+\frac{O(t^3)}{t^2} \right)}{1+\frac{O(t^2)}{t^2}}\\
                &\overset{t\rightarrow0}{=}0\\
            \end{split}
        \end{equation*}
        Si a la función $t\mapsto\frac{1}{2\sen\frac{t}{2}}-\frac{1}{t}$ se le asigna el valor 0 en 0, se hace continua en $[-\delta,\delta]$. Así pues,
        \begin{equation*}
            t\mapsto\left(f(x+t)-s(x) \right)\left[\frac{1}{2\sen\frac{t}{2}}-\frac{1}{t}\right]
        \end{equation*}
        es integrable en $[-\delta,\delta]$. Por Riemman-Lebesgue la segunda integral tiende a 0 cuando $m\rightarrow\infty$. Por tanto se concluye que
        \begin{equation*}
            \lim_{ m\rightarrow\infty}\int_{-\delta}^{\delta}\frac{f(x+t)-s(x)}{2\sen\frac{t}{2}}\sen\left(m+\frac{1}{2} \right)tdt=0\iff\lim_{ m\rightarrow\infty}\int_{-\delta}^{\delta}\frac{f(x+t)-s(x)}{t}\sen\left(m+\frac{1}{2} \right)tdt=0
        \end{equation*}
        por la observación hecha anteriormente, esto es equivalente a
        \begin{equation*}
            \lim_{ m\rightarrow\infty}s_m(x)=s(x)\iff\lim_{ m\rightarrow\infty}\int_{-\delta}^{\delta}\frac{f(x+t)-s(x)}{t}\sen\left(m+\frac{1}{2} \right)tdt=0,\textup{ para algún }0<\delta<\pi
        \end{equation*}
    \end{proof}

    \begin{obs}
        Por el teorema anterior, la convergencia de la serie de Fourier de $f$ en un punto $x$ y la eventual suma de esta serie de Fourier dependen solamente del comportamiento de $f$ en alguna vecindad arbitrariamente pequeña de $x$. A esto se le llama \textbf{el principio de localización de Riemman}. Esto es sorprendente, pues los coeficientes de Fourier de la función $f$ dependen de los valores de $f$ en todo el intervalo $[-\pi,\pi[$.
    \end{obs}

    \begin{theor}[\textbf{Criterio de Dini para la convergencia puntual de una serie de Fourier}]
        Sea $f\in\mathcal{L}_1^{2\pi}(\mathbb{C})$. Para que la serie de Fourier de $f$ converga en un punto $x\in\mathbb{R}$ a una suma $s(x)$ es necesario y suficiente que para algún $0<\delta<\pi$ la función siguiente sea integrable:
        \begin{equation*}
            t\mapsto\frac{f(x+t)+f(x-t)-2s(x)}{t}
        \end{equation*}
        sea integrable en $]0,\delta[$.
    \end{theor}

    \begin{proof}
        $\Rightarrow):$ La condición $\int_{-\delta}^{\delta}\abs{\frac{f(x+t)-s(x)}{t}}dt<\infty$ implica la convergencia puntual de la serie de Fourier. En efecto,
        \begin{equation*}
            \begin{split}
                \int_{-\delta}^0\abs{\frac{f(x+t)-s(x)}{t}}dt+\int_{0}^\delta\abs{\frac{f(x+t)-s(x)}{t}}dt&=\int_{-\delta}^{\delta}\abs{\frac{f(x+t)-s(x)}{t}}dt<\infty\\
                \Rightarrow \int_{0}^\delta\abs{\frac{f(x-u)-s(x)}{u}}du+\int_{0}^\delta\abs{\frac{f(x+t)-s(x)}{t}}dt&=\int_{-\delta}^{\delta}\abs{\frac{f(x+t)-s(x)}{t}}dt<\infty\\
                \Rightarrow \int_{0}^\delta\abs{\frac{f(x+t)+f(x-t)-2s(x)}{t}}dt &\leq\int_{-\delta}^{\delta}\abs{\frac{f(x+t)-s(x)}{t}}dt<\infty\\
            \end{split}
        \end{equation*}

        $\Leftarrow):$ Es inmediato del teorema de Riemman-Lebesgue y del teorema anterior.
    \end{proof}

    \begin{cor}
        Sea $f\in\mathcal{L}_1^{2\pi}(\mathbb{C})$. Si en un punto $x\in\mathbb{R}$ existen la derivda por la derecha $f_{d}'(x)$ y por la izquierda $f_{i}'(x)$, entonces la serie de Fourier de $f$ converge en $x$ a $f(x)$. 
    \end{cor}

    \begin{proof}
        Como existen las derivadas por la derecha e izquierda, para $\varepsilon=1>0$ existe $\delta>0$ tal que 
        \begin{equation*}
            0<t<\delta\Rightarrow\left\{
                \begin{array}{rl}
                    \abs{f(x+t)-s(x)}&\leq t\abs{f_d'(x)+1}\\
                    \abs{f(x-t)-s(x)}&\leq t\abs{f_i'(x)+1}\\
                \end{array}
            \right.
        \end{equation*}
        así pues, $0<t<\delta$ implica que
        \begin{equation*}
            \begin{split}
                \abs{\frac{f(x+t)+f(x-t)-2f(x)}{t}}&\leq\abs{f_d'(x)}+\abs{f_i'(x)}+2
            \end{split}
        \end{equation*}
        por tanto, $t\mapsto \frac{f(x+t)+f(x-t)-2f(x)}{t}$ es integrable en $]0,\delta[$.
    \end{proof}

    \section{Convergencia uniforme de una serie de Fourier}

    \begin{obs}
        Recordemos que un conjunto relativamente compacto en un espacio métrico $(X,d)$ es aquel tal que su cerradura es compacta. Equivalentemente, es totalmente acotado.
    \end{obs}

    \begin{lema}[\textbf{Versión Uniforme del Teorema de Riemman-Lebesgue}]
        Si $\mathcal{F}$ es un conjunto relativamente compacto en $L_1^{2\pi}(\mathbb{C})$, entonces para todo $\varepsilon>0$ existe $N>0$ tal que
        \begin{equation*}
            \lambda\in\mathbb{R},\abs{\lambda}\geq N\Rightarrow\sup_{ f\in\mathcal{F}}\abs{\int_{-\pi}^\pi f(x)e^{ i\lambda x}dx}<\varepsilon
        \end{equation*}
    \end{lema}

    \begin{proof}
        Como $\mathcal{F}$ es relativamente compacto, entonces $\mathcal{F}$ es totalmente acotado (ya que su cerradura por definición es compacta). Luego, dado $\varepsilon>0$ existe una familia finita de elementos de $\mathcal{F}$, digamos $f_1,...,f_r\in\mathcal{F}$ tales que las bolas abiertas $B(f_1,\frac{\varepsilon}{2}),...,B(f_r,\frac{\varepsilon}{2})$ recubren a $\mathcal{F}$. Por Riemman-Lebesgue, existe $N>0$ tal que 
        \begin{equation*}
            \lambda\in\mathbb{R},\abs{\lambda}\geq N\Rightarrow\abs{\int_{-\pi}^\pi f_k(x)e^{i\lambda x}dx}<\frac{\varepsilon}{2},\quad\forall k\in\natint{1,r}
        \end{equation*}
        Sea $f\in\mathcal{F}$ arbitario. Existe $k\in\natint{1,r}$ tal que $f\in B(f_k,\frac{\varepsilon}{2})$, esto es
        \begin{equation*}
            \N{1}{f-f_k}<\frac{\varepsilon}{2}
        \end{equation*}
        Si $\lambda\in\mathbb{R}$ es tal que $\abs{\lambda}>N$, se tiene que
        \begin{equation*}
            \begin{split}
                \abs{\int_{-\pi}^\pi f(x)e^{ i\lambda x}dx}&\leq\abs{\int_{-\pi}^{\pi}\left[f(x)-f_k(x)\right]e^{i\lambda x}dx}+\abs{\int_{-\pi}^\pi f_k(x)e^{ i\lambda x}dx}\\
                &<\N{1}{f-f_k}+\frac{\varepsilon}{2}\\
                &<\varepsilon\\
            \end{split}
        \end{equation*}
        lo que termina la prueba.
    \end{proof}

    \begin{lema}
        Sean $f\in\mathcal{L}_1^{2\pi}(\mathbb{C})$ y $E\subseteq[-\pi,\pi]$. Se supone:
        \begin{enumerate}
            \item $f$ es acotada en $E$.
            \item Para todo $\varepsilon>0$ existe $0<\delta<\pi$ tal que
            \begin{equation*}
                \sup_{ x\in E}\int_{-\delta}^{\delta}\abs{\frac{f(x+t)-f(x)}{t}}dt<\varepsilon
            \end{equation*}
        \end{enumerate}
        Defina para cada $x\in E$,
        \begin{equation*}
            \varphi_x(t)=\frac{f(x+t)-f(x)}{t},\quad\forall t\in[-\pi,\pi]\backslash\left\{0 \right\}
        \end{equation*}
        y extiéndase por periodicidad a todo $\mathbb{R}$. Si $h\in\mathcal{L}_{\infty}^{2\pi}(\mathbb{C})$, entonces la familia de funciones $\left\{h\varphi_x \right\}_{x\in E}$ es relativamente compacta en $L_1^{2\pi}(\mathbb{C})$.
    \end{lema}

    \begin{proof}
        Probaremos que la cerradura de este conjunto es secuencialmente compacto (luego compacto). Sea $\left\{x_n \right\}_{ n=1}^\infty$ una sucesión en $E$. Hay que probar que $\left\{h\varphi_{ x_n} \right\}_{ n=1}^\infty$ contiene una subsucesión convergente en promedio.

        Como $E\subseteq[-\pi,\pi]$, entonces $\left\{x_n \right\}_{ n=1}^\infty$ contiene una subsucesión que converge a algún punto de $[-\pi,\pi]$, digamos $\left\{x_{\alpha(n)} \right\}_{ n=1}^\infty$. Ahora, $\left\{f(x_{\alpha(n)}) \right\}_{ n=1}^\infty$ es una sucesión acotada en $\mathbb{C}$, luego posee una subsucesión $\left\{f(x_{ \beta\circ\alpha(n)}) \right\}_{ n=1}^\infty$ convergente. Para simplificar la notación, se puede suponer que la sucesión $\left\{x_n\right\}_{ n=1}^\infty$ original y la sucesión de valores $\left\{f(x_n) \right\}_{ n=1}^\infty$ son convergentes (en particular, de Cauchy).

        Sea $M>0$ tal que $\abs{h}\leq M$ c.t.p. en $\mathbb{R}$. Sea $\varepsilon<0$ y $0<\delta<\pi$ como en el enunciado. Se afirma que $\left\{h\varphi_{x_n} \right\}$ es de Cauchy en $L_1^{2\pi}$. En efecto, veamos que si $n,m\in\mathbb{N}$:
        \begin{equation*}
            \begin{split}
                \N{1}{h\varphi_{x_n}-h\varphi_{x_m}}&=\int_{-\pi}^\pi\abs{h(t)}\abs{\frac{f(x_n+t)-f(x_n)-f(x_m+t)+f(x_m)}{t}}dt \\
                &\leq M\int_{-\delta }^{\delta}\abs{\frac{f(x_n+t)-f(x_n)-f(x_m+t)+f(x_m)}{t}}dt\\
                +&M\abs{f(x_n)-f(x_m)}\int_{ \delta\leq\abs{t}<\pi}\frac{dt}{\abs{t}}+M\int_{\delta\leq\abs{t}<\pi}\abs{\frac{f(x_n+t)-f(x_n+t)}{t}}dt\\
            \end{split}
        \end{equation*}
        Por la hipótesis, la primera integral a la derecha es $<2M\varepsilon$. Como $\left\{f(x_n) \right\}$ es de Cauchy, existe $n_0\in\mathbb{N}$ tal que
        \begin{equation*}
            n,m\geq n_0\Rightarrow M\abs{f(x_n)-f(x_m)}\int_{\delta\leq\abs{t}<\pi}\frac{dt}{\abs{t}}<M\varepsilon
        \end{equation*}
        Y, la tercera integral a la derecha es menor o igual a
        \begin{equation*}
            \frac{M}{\delta}\int_{-\pi}^{\pi}\abs{f(x_n+t)-f(x_m+t)}dt
        \end{equation*}
        (mayorando a $t\mapsto\frac{1}{\abs{t}}$). Ahora, ya que la función $y\mapsto f_y$ es uniformemente continua de $\mathbb{R}$ en $L_1^{2\pi}(\mathbb{C})$, existe $\eta>0$ tal que
        \begin{equation*}
            \abs{x_n-x_m}<\eta\Rightarrow\frac{M}{\delta}\int_{-\pi}^{\pi}\abs{f(x_n+t)-f(x_m+t)}dt<\delta\cdot\varepsilon
        \end{equation*}
        Por ser la sucesión $\left\{x_n \right\}$ de Cauchy, existe $N\geq n_0$ tal que
        \begin{equation*}
            n,m\geq N\Rightarrow\abs{x_n-x_m}<\eta\Rightarrow\int_{-\pi}^{\pi}\abs{f(x_n+t)-f(x_m+t)}dt<M\varepsilon
        \end{equation*}
        Por tanto, $n,m\geq N$ implica que
        \begin{equation*}
            \N{1}{h\varphi_{x_n}-h\varphi_{x_m}}\leq 4M\varepsilon
        \end{equation*}
        así, $\left\{h\varphi_{ x_n} \right\}$ es de Cauchy en $L_1^{2\pi}(\mathbb{C})$, luego convergente.
    \end{proof}

    \begin{theor}[\textbf{Criterio de Dini para la convergencia uniforme de una serie de Fourier}]
        Sean $f\in\mathcal{L}_1^{2\pi}(\mathbb{C})$ y $E\subseteq[-\pi,\pi]$. Se supone que $f$ es acotada en $E$ y que para todo $\varepsilon>0$ existe $0<\delta<\pi$ tal que
        \begin{equation*}
            \sup_{ x\in E}\int_{-\delta}^{\delta}\abs{\frac{f(x+t)-f(x)}{t}}dt<\varepsilon
        \end{equation*}
        Entonces, la serie de Fourier de $f$ converge a $f$ uniformemente en $E$.
    \end{theor}

    \begin{proof}
        Sea $x\in E$. Se tiene que
        \begin{equation*}
            \begin{split}
                \abs{s_m(x)-f(x)}&=\frac{1}{\pi}\int_{-\pi}^{\pi}\frac{f(x+t)-f(x)}{2\sen\frac{t}{2}}\sen\left(m+\frac{1}{2}\right)tdt\\
            \end{split}
        \end{equation*}
        Hay que probar que
        \begin{equation*}
            \lim_{ m\rightarrow\infty}\int_{-\pi}^\pi\frac{f(x+t)-f(x)}{2\sen\frac{t}{2}}\sen\left(m+\frac{1}{2}\right)tdt=0\textup{ uniformemente con respecto a }x\in E
        \end{equation*}
        Se tiene
        \begin{equation*}
            \begin{split}
                \int_{-\pi}^{\pi}\frac{f(x+t)-f(x)}{2\sen\frac{t}{2}}\sen\left(m+\frac{1}{2}\right)tdt&=\int_{-\pi}^{\pi}\frac{f(x+t)-f(x)}{t}\sen\left(m+\frac{1}{2}\right)tdt\\+&\int_{-\pi}^{\pi}\frac{f(x+t)-f(x)}{t}\left[\frac{t}{2\sen\frac{t}{2}}-1 \right] \sen\left(m+\frac{1}{2}\right)tdt\\
            \end{split}
        \end{equation*}
        para $t\in[-\pi,\pi[$ se define
        \begin{equation*}
            h(t)=\left\{
                \begin{array}{lcr}
                    \frac{t}{2\sen\frac{t}{2}}-1 & \textup{ si } & t\neq 0\\
                    0 & \textup{ si } & t=0\\
                \end{array}
            \right.
        \end{equation*}
        se verifica rápidamente que $h$ es continua en $[-\pi,\pi[$, luego acotada. Además, para cada $x\in E$ se define
        \begin{equation*}
            \varphi_x(t)=\frac{f(x+t)-f(x)}{t},\quad\forall t\in[-\pi,\pi[
        \end{equation*}
        Por el último lema, $\left\{\varphi_x \right\}_{x\in E}$ y $\left\{h\varphi_x \right\}_{x\in E}$ son relativamente compactas en $L_1^{2\pi}(\mathbb{C})$. Por la versión uniforme del teorema de Riemman-Lebesgue, entonces las dos integrales de arriba tienden a cero cuando $m\rightarrow\infty$ uniformemente con respecto a $x\in E$.
    \end{proof}

    \begin{obs}
        Veamos como se aplica el primer lema de la sección en la demostración de este teorema. La primera familia es relativamente compacta tomando $h=1$ y la segunda tomando a la $h$ dada, siendo ambas acotadas (por ser continuas en un compacto y ser extendidas por periodicidad a todo $\mathbb{R}$).
    \end{obs}

    \begin{obs}
        Usando la convergencia de series no se puede reconstruir en general una función $f\in\mathcal{L}_1^{2\pi}$ conociendo su serie de Fourier, de hecho, Kolmogorov demostró que existen funciones en $\mathcal{L}_1^{2\pi}$ cuyas series de Fourier divergen en todo punto. También existen funciones en $\mathcal{C}^{2\pi}$ cuyas series de Fourier divergen en algunos puntos. 
        
        La situación se arregla considerando un modo de convergencia distinto: la convergencia en el sentido de Cesáro.
    \end{obs}

    \section{Convergencia en sentido de Cesáro de series de Fourier en $\mathcal{L}_1^{2\pi}$}

    \begin{lema}
        Si $\left\{a_n \right\}_{ n=1}^\infty$ es una sucesión en un espacio normado $(E,\norm{\cdot})$ converge a un límite $l\in E$ y
        \begin{equation*}
            b_n=\frac{a_1+...+a_n}{n},\quad\forall n\in\mathbb{N}
        \end{equation*}
        entonces,
        \begin{equation*}
            \lim_{ n\rightarrow\infty}b_n=l
        \end{equation*}
    \end{lema}

    \begin{proof}
        Veamos que si $n\in\mathbb{N}$:
        \begin{equation*}
            \begin{split}
                \norm{b_n-l}&=\frac{1}{n}\norm{\sum_{ k=1}^na_k-l }\\
                &\leq\frac{1}{n}\sum_{ k=1}^n\norm{a_k-l}\\
            \end{split}
        \end{equation*}
        dado $\varepsilon>0$ existe $n_0\in\mathbb{N}$ tal que
        \begin{equation*}
            n\geq n_0\Rightarrow\norm{a_n-l}<\frac{\varepsilon}{2}
        \end{equation*}
        Entonces,
        \begin{equation*}
            \begin{split}
                n> n_0\Rightarrow\norm{b_n-l}&\leq\frac{1}{n}\left(\sum_{ k=1}^{ n_0}\norm{a_k-l} \right)+\frac{1}{n}\sum_{ k=n_0+1}^{ n}\norm{a_k-l}\\
                &\leq\frac{1}{n}\underbrace{\left(\sum_{ k=1}^{ n_0}\norm{a_k-l} \right)}_{cte.}+\frac{n-n_0}{n}\cdot\frac{\varepsilon}{2}\\
                &<\frac{1}{n}\left(\sum_{ k=1}^{ n_0}\norm{a_k-l} \right)+\frac{\varepsilon}{2}\\
            \end{split}
        \end{equation*}
        Además, existe $n_1\in\mathbb{N}$ con $n_1>n_0$ tal que
        \begin{equation*}
            n\geq n_1\Rightarrow \frac{1}{n}\left(\sum_{ k=1}^{ n_0}\norm{a_k-l} \right)<\frac{\varepsilon}{2}
        \end{equation*}
        luego, $n\geq n_1$ implica que $\norm{b_n-l}<\varepsilon$.
    \end{proof}

    \begin{mydef}
        Sea $\sum_{ n=1}^\infty u_n$ una serie en un espacio normado. Defina
        \begin{equation*}
            s_n=\sum_{ k=1}^n u_k\quad\textup{y}\quad\sigma_n=\frac{1}{n}\sum_{ k=1}^n u_k
        \end{equation*}
        Si $\lim_{ n\rightarrow\infty}\sigma_n=s$, se dice que la sucesión \textbf{converge en el sentido de Cesáro}.
    \end{mydef}

    Por el lema, si la serie converge en el sentido usual a $s$, entonces la serie converge en el sentido de Cesáro. La recíproca no es cierta en general (basta con observar lo que sucede con $\left\{(-1)^n \right\}_{ n=1}^\infty$).
    
    En casos favorables la recíproca es cierta (más adelante se verá uno de esos casos).

    Sea $\left\{s_m \right\}_{ m=1}^\infty$ la sucesión de sumas parciales de una serie de Fourier, se define
    \begin{equation*}
        \sigma_n(x)=\frac{1}{n}\left[s_0(x)+...+s_{n-1}(x) \right],\quad\forall n\in\mathbb{N}^*
    \end{equation*}
    con lo que decir que la serie de Fourier de una función $f$ converge en un punto $x\in\mathbb{R}$ en el sentido de Cesáro a una suma $s(x)$ es decir que
    \begin{equation*}
        \lim_{ n\rightarrow\infty}\sigma_n(x)=s(x)
    \end{equation*}

    \subsection{Núcleo de Fejér y el Teorema de Fejér}

    Es posible calcular explícitamente usando la fórmula
    \begin{equation*}
        s_m(x)=\frac{a_0}{2}+\sum_{ k=1}^{m}\left[a_k\cos kx+b_k\sen kx \right]
    \end{equation*}
    Resulta
    \begin{equation*}
        \begin{split}
            \sigma_m(x)&=\frac{1}{n}\left[n\frac{a_0}{2}+\sum_{m=1}^{ n-1}\sum_{ k=1}^{m}\left[a_k\cos kx+b_k\sen kx \right] \right]\\
            &=\frac{1}{n}\left[n\frac{a_0}{2}+(n-1)+(n-1)(a_1\cos x+b_1\sen x)\right.\\
            +&\cdots+(n-k)(a_k\cos kx+b_k\sen kx)\\
            +&\left. \cdots+(a_{ n-1}\cos (n-1)x+b_{ n-1}\sen(n-1)x)\right] \\
        \end{split}
    \end{equation*}
    o sea,
    \begin{equation*}
        \sigma_n(x)=\frac{a_0}{2}+\sum_{ k=1}^{n-1}\left(1-\frac{k}{n} \right)\left[a_k\cos kx+b_k\sen kx \right]
    \end{equation*}
    Alternativamente, $\sigma_n(x)$ se puede calcular como sigue:
    \begin{equation*}
        \sigma_n(x)=\frac{1}{n}\left[\sum_{ k=0}^{ n-1}s_k(x) \right]=\frac{1}{n}\sum_{ m=0}^{ n-1}f*D_m(x)
    \end{equation*}
    donde $D_m$ es el núcleo de Dirichlet. Entonces,
    \begin{equation*}
        \sigma_n(x)=f*k_n(x)
    \end{equation*}
    donde
    \begin{equation*}
        k_n=\frac{1}{n}\sum_{ m=0}^{ n-1}D_m
    \end{equation*}
    el cual es llamado el \textbf{núcleo de Fejér}.

    Se tiene
    \begin{equation*}
        \begin{split}
            k_n(x)&=\frac{1}{n}\sum_{ m=0}^{ n-1}D_m(x)\\
            &=\frac{1}{n}\sum_{ m=0}^{ n-1}\frac{\sen\left(m+\frac{1}{2} \right)x}{\sen\frac{x}{2}}\\
            &=\frac{1}{n\sen\frac{x}{2}}\sum_{ m=0}^{ n-1}\sen\left(m+\frac{1}{2} \right)x\\
        \end{split}
    \end{equation*}
    donde
    \begin{equation*}
        \begin{split}
            \sum_{ m=0}^{ n-1}e^{ imx}&=\frac{1-e^{ inx}}{1-e^{ ix}}\\
            &=\frac{1-e^{ inx}}{e^{ i\frac{x}{2}}(e^{ -i\frac{x}{2}}-e^{ i\frac{x}{2}})}\\
            \Rightarrow \sum_{ m=0}^{ n-1}e^{ imx}&=i\frac{1-e^{ inx}}{2\sen\frac{x}{2}}\\
        \end{split}
    \end{equation*}
    igualando los coeficientes de $i$:
    \begin{equation*}
        \begin{split}
            \sum_{ m=0}^{ n-1}\sen\left(m+\frac{1}{2} \right)x&=\frac{1-\cos nx}{2\sen\frac{x}{2}}\\
            &=\frac{2\sen^2 n\frac{x}{2}}{2\sen\frac{x}{2}}\\
            &=\frac{\sen^2 n\frac{x}{2}}{\sen\frac{x}{2}}\\
            \therefore k_n(x)&=\frac{1}{2\pi n}\cdot\frac{\sen^2 n\frac{x}{2}}{\sen^2\frac{x}{2}}\\
        \end{split}
    \end{equation*}
    si $x$ no es múltiplo entero de $2\pi$. Para tales $x$, se tiene que $k_n(x)=\frac{n}{2\pi}$.

    \begin{propo}
        $\left\{k_n \right\}_{ n=1}^\infty$ es una sucesión de Dirac fuerte en $\mathcal{L}_1^{2\pi}$.
    \end{propo}

    \begin{proof}
        Claramente $k_n\geq 0$ para todo $n\in\mathbb{N}^*$. Además,
        \begin{equation*}
            k_n=\frac{1}{n}\left(D_0+\cdots+D_{ n-1} \right)
        \end{equation*}
        y
        \begin{equation*}
            \int_{ -\pi}^{\pi}D_m=1\Rightarrow \int_{ -\pi}^{ \pi}k_n=1
        \end{equation*}
        Si $0<\delta<\pi$, entonces
        \begin{equation*}
            \sup_{ \delta\leq x<\pi}k_n(x)\leq\frac{1}{2\pi n}\cdot\frac{1}{\sen^2\frac{\delta}{2}}\underset{n\rightarrow\infty}{\rightarrow}0
        \end{equation*}
    \end{proof}

    \begin{theor}[\textbf{Teorema de Féjer}]
        Sea $f\in\mathcal{L}_1^{2\pi}(\mathbb{C})$.
        \begin{enumerate}
            \item Si $1\leq p<\infty$ y $f\in\mathcal{L}_p^{2\pi}(\mathbb{C})$ entonces, la serie de Fourier de $f$ converge en el sentido de Cesáro a $f$ en $p$-promedio.
            \item Si en un punto $x\in\mathbb{R}$ existen $f(x^+)$ y $f(x^-)$ (siendo ambos finitos) entonces la serie de Fourier de $f$ converge en el sentido de Cesáro en el punto $x$ a
            \begin{equation*}
                \frac{f(x^+)+f(x^-)}{2}
            \end{equation*}
            en particular, si $f$ es continua en ese punto la serie de Fourier de $f$ converge en el punto $x$ a $f(x)$ en el sentido de Cesáro.
            \item Si $f$ es continua en $J\subseteq\mathbb{R}$ abierto, entonces la serie de Fourier de $f$ converge en el sentido de Cesáro a $f$ uniformemente en todo compacto $C\subseteq J$.
            \item Si $f$ es continua periódica, entonces la serie de Fourier de $f$ converge en el sentido de Cesáro a $f$ uniformemente en $\mathbb{R}$.
        \end{enumerate}
    \end{theor}

    \begin{proof}
        Es inmediata de la definición del núcleo de Fejér $k_n$ y del hecho que $\left\{k_n \right\}_{ n=1}^\infty$ es una sucesión de Dirac fuerte.
    \end{proof}

    \section{Convergencia c.t.p. en el sentido de Cesáro de series de Fourier en $\mathcal{L}_1^{2\pi}$}

    \begin{mydef}
        Sea $f\in\mathcal{L}_1^{loc}(\mathbb{R},\mathbb{C})$ localmente integrable en $\mathbb{R}$. Se dice que un punto $x\in\mathbb{R}$ es un \textbf{punto de Lebesgue} de $f$ si
        \begin{equation*}
            \lim_{ h\rightarrow 0}\frac{1}{h}\int_0^h\abs{f(x+t)-f(x)}dt=0
        \end{equation*}
        el conjunto de puntos de Lebesgue de una función $f$ se llama \textbf{conjunto de Lebesgue de $f$}.
    \end{mydef}

    \begin{exa}
        Si $f$ es continua en $x\in\mathbb{R}$, entonces $x$ es un punto de Lebesgue.
    \end{exa}

    Más adelante se demostrará que si $f$ es localmente integrable en $\mathbb{R}$, entonces el complemento del conjunto de Lebesgue de $f$ es despreciable. Osea que casi todo punto de $\mathbb{R}$ es de Lebesgue. Por el momento no se probará este resultado.

    \begin{obs}
        Se tiene siempre lo siguiente:
        \begin{equation*}
            \frac{2}{\pi}x \leq \sin x\leq x,\quad\forall x\in\left[0,\frac{\pi}{2}\right]
        \end{equation*}
        por la concávidad de $x\mapsto \sin x$.
    \end{obs}

    \begin{theor}[\textbf{Teorema de Féjer-Lebesgue}]
        Si $f\in\mathcal{L}_1^{2\pi}(\mathbb{C})$ entonces en todo punto de Lebesgue $x$ de $f$ (es decir, c.t.p. en $\mathbb{R}$) la serie de Fourier de $f$ converge en el sentido de Cesáro a $f(x)$.
    \end{theor}

    \begin{proof}
        Sea $f\in\mathcal{L}_1^{2\pi}(\mathbb{C})$ y $x\in\mathbb{R}$ un punto de Lebesgue de $f$. Por el cambio de variable $t=-u$ se tiene que
        \begin{equation*}
            \frac{1}{h}\int_0^h\abs{f(x-t)-f(x)}dt=\frac{1}{-h}\int_0^{-h}\abs{f(x+u)-f(x)}du
        \end{equation*}
        Por tanto, también se cumple
        \begin{equation*}
            \lim_{ -h\rightarrow0}\frac{1}{-h}\int_0^{-h}\abs{f(x+t)-f(x)}dt=\lim_{ h\rightarrow0}\frac{1}{h}\int_0^h\abs{f(x-t)-f(x)}dt=0
        \end{equation*}
        (recuerde que la integral es orientada). Usando la paridad de $k_n$ y el cambio de variable $t=-u$ obtenemos
        \begin{equation*}
            \begin{split}
                \sigma_n(x)&=f*k_n(x)\\
                &=\int_{-\pi}^{\pi}f(x-t)k_n(t)dt\\
                &=\int_{-\pi}^{0}f(x-t)k_n(t)dt+\int_{0}^{\pi}f(x-t)k_n(t)dt\\
                &=\int_{0}^{\pi}\left[f(x+t)+f(x-t)\right]k_n(t)dt\\
            \end{split}
        \end{equation*}
        también, por la paridad de $k_n$: $\int_{0}^{\pi}k_n(t)dt=\frac{1}{2}$, luego
        \begin{equation*}
            \begin{split}
                f(x)&=\int_0^{\pi}2f(x)k_n(t)dt\\
                \Rightarrow \sigma_n(x)-f(x)&=\int_0^{\pi}g_x(t)k_n(t)dt\\
                \Rightarrow \abs{\sigma_n(x)-f(x)}&\leq\int_0^h\abs{g_x(t)}k_n(t)dt\\
            \end{split}
        \end{equation*}
        donde
        \begin{equation*}
            g_x(t)=f(x+t)+f(x-t)-2f(x),\quad\forall t\in\mathbb{R}
        \end{equation*}
        
        Observemos ahora que
        \begin{equation*}
            \begin{split}
                \abs{\int_0^{h}\abs{g_x(t)}dt}&\leq\abs{\int_0^h\abs{f(x+t)-f(x)}dt}+\abs{\int_0^h\abs{f(x-t)-f(x)}dt}
            \end{split}
        \end{equation*}
        para todo $h\in\mathbb{R}$. Por ser $x$ un punto de Lebesgue de $f$ dado $\varepsilon>0$ existe $0<\delta<\pi$ tal que
        \begin{equation*}
            0<h<\delta\Rightarrow\int_0^{h}\abs{g_x(t)}dt<h\varepsilon
        \end{equation*}
        Escriba $G_x(h)=\int_0^h \abs{g_x(t)}dt$, para todo $h\in\mathbb{R}$. Se tiene que
        \begin{equation*}
            0<h<\delta\Rightarrow G_x(h)<h\varepsilon
        \end{equation*}
        Sea $n\in\mathbb{N}$ tal que $\frac{1}{n}<\delta$. Entonces
        \begin{equation*}
            \begin{split}
                \abs{\sigma_n(x)-f(x)}&\leq\int_0^{\frac{1}{n}}\abs{g_x(t)}k_n(t)dt+\int_{\frac{1}{n}}^\delta\abs{g_x(t)}k_n(t)dt+\int_\delta^{\pi}\abs{g_x(t)}k_n(t)dt\\
            \end{split}
        \end{equation*}
        Recordemos que
        \begin{equation*}
            0\leq k_n(t)=\frac{1}{2\pi n}\frac{\sin^2 n\frac{t}{2}}{\sin^2\frac{t}{2}}\leq\frac{1}{2\pi n}\frac{n^2\left( \frac{t}{2}\right)^2}{\left(\frac{2}{\pi} \right)^2\left(\frac{t}{2} \right)^2}=\frac{n\pi}{8},\quad\forall t\in\mathbb{R}
        \end{equation*}
        Luego
        \begin{equation*}
            \int_0^{\frac{1}{n}}\abs{g_x(t)}k_n(t)dt\leq\frac{n\pi}{8}\int_0^{\frac{1}{n}}\abs{g_x(t)}dt\leq\frac{n\pi}{8}\left(\frac{1}{n}\cdot\varepsilon \right)
        \end{equation*}
        Ahora, si $\frac{1}{n}<t<\delta$:
        \begin{equation*}
            \begin{split}
                k_n(t)&=\frac{1}{2\pi n}\frac{\sin^2\left(n\frac{t}{2} \right)}{\sin^2\frac{t}{2}}\\
                &\leq\frac{1}{2\pi n}\frac{1}{\left(\frac{2}{\pi} \right)^2 \left(\frac{t}{2} \right)^2}\\
                &=\frac{\pi}{2n}\frac{1}{t^2}\\
            \end{split}
        \end{equation*}
        Integrando por partes,
        \begin{equation*}
            \begin{split}
                \int_{\frac{1}{n}}^\delta\abs{g_x(t)}k_n(t)dt&\leq\frac{\pi}{2n}\int_{\frac{1}{n}}^\delta\abs{g_x(t)}\frac{dt}{t^2}\\
                &=\frac{\pi}{2n}\left[\frac{G_x(t)}{t^2}\Big|_{\frac{1}{n}}^{\delta}+2\int_{\frac{1}{n}}^\delta \frac{G_x(t)}{t^3}dt\right]\\
                &=\frac{\pi}{2n}\left[\frac{G_x(\delta)}{\delta^2}-n^2G_x\left(\frac{1}{n}\right)+2\int_{\frac{1}{n}}^\delta \frac{G_x(t)}{t^3}dt\right]\\
                &\leq\frac{\pi}{2n}\left[\frac{G_x(\delta)}{\delta^2}+2\int_{\frac{1}{n}}^\delta \frac{G_x(t)}{t^3}dt\right]\\
                &\leq\frac{\pi}{2n}\left[\frac{\delta\varepsilon}{\delta^2}+2\int_{\frac{1}{n}}^\delta \frac{t\varepsilon}{t^3}dt\right]\\
                &=\frac{\pi}{2n}\left[\frac{\varepsilon}{\delta}+2\varepsilon\int_{\frac{1}{n}}^\delta \frac{dt}{t^2}\right]\\
                &=\frac{\pi}{2n}\left[\frac{\varepsilon}{\delta}+2\varepsilon\left(n-\frac{1}{\delta} \right) \right]\\
                &\leq\frac{\pi}{2n}\left[n\varepsilon+2n\varepsilon\right]\\
                &=\frac{2\pi}{2}\varepsilon\\
            \end{split}
        \end{equation*}
        Ahora, para $\delta\leq t<\pi$ se tiene
        \begin{equation*}
            \begin{split}
                k_n(t)=\frac{1}{2\pi n}\frac{\sin^2\left(n\frac{t}{2} \right)}{\sin^2\frac{t}{2}}\leq\frac{1}{2\pi n}\frac{1}{\sin^2\frac{\delta}{2}}
            \end{split}
        \end{equation*}
        Entonces,
        \begin{equation*}
            \int_\delta^{\pi}\abs{g_x(t)}k_n(t)dt\leq\frac{1}{2\pi n}\frac{1}{\sin^2\left(\frac{\delta}{2} \right)}\int_\delta^{\pi}\abs{g_x(t)}dt\underset{ n\rightarrow\infty}{\rightarrow}0
        \end{equation*}
        pues las dos cantidades de la derecha no dependen de $n$ y más aún, son constantes (al menos la segunda está acotada pues la función $f$ es integrable en $[0,\pi]$). Existe pues $N>\frac{1}{\delta}$ tal que
        \begin{equation*}
            n\geq N\Rightarrow\frac{1}{2\pi n}\frac{1}{\sin^2\left(\frac{\delta}{2} \right)}\int_\delta^{\pi}\abs{g_x(t)}dt<\varepsilon
        \end{equation*}
        Por tanto,
        \begin{equation*}
            n\geq N\Rightarrow\abs{\sigma_n(x)-f(x)}<\frac{\pi}{8}\varepsilon+\frac{3\pi}{2}\varepsilon+\varepsilon=\left(\frac{13}{8}\pi+1 \right)\varepsilon
        \end{equation*}
        lo que prueba el resultado.
    \end{proof}

    \begin{cor}
        Sea $f\in\mathcal{L}_1^{2\pi}(\mathbb{C})$. Si la serie de Fourier de $f$ converge en el sentido usual c.t.p. en $\mathbb{R}$ a una función $g$, entonces $f=g$ c.t.p. en $\mathbb{R}$.
    \end{cor}

    \begin{proof}
        Por el teorema anterior, la serie de Fourier de $f$ converge en el sentido de Cesáro a $f$ c.t.p. en $\mathbb{R}$. Como dicha serie converge en el sentido usual a $g$ c.t.p. en $\mathbb{R}$, entonces también lo hace en el sentido de Cesáro. Por tanto, $f=g$ c.t.p. en $\mathbb{R}$.
    \end{proof}
    
    \begin{theor}[\textbf{Teorema de Hardy-Landau}]
        Sea $\sum_{ n=1}^\infty u_n$ una serie en un espacio normado $(E,\norm{\cdot})$ que converge en el sentido de Cesáro. Si existe $A>0$ tal que
        \begin{equation*}
            \norm{u_n}\leq\frac{A}{n},\quad\forall n\in\mathbb{N}
        \end{equation*}
        entonces, la serie converge en el sentido usual.
    \end{theor}

    \begin{proof}
        Sea $s_m=\sum_{ k=1}^m u_k$ y $\sigma_n=\frac{1}{n}\sum_{ m=1}^n s_m$ (es decir que $\sum_{ m=1}^n s_m=n\sigma_n$), para todo $n,m\in\mathbb{N}$. Por hipótesis existe $\sigma$ tal que
        \begin{equation*}
            \lim_{ n\rightarrow\infty}\sigma_n=\sigma
        \end{equation*}
        Sea $\varepsilon>0$ tal que $0<\varepsilon<1$. Entonces, existe $N\in\mathbb{N}$ tal que
        \begin{equation*}
            n\geq N\Rightarrow\norm{\sigma-\sigma_n}<\varepsilon
        \end{equation*}
        Sean $m,n\in\mathbb{N}$ tales que $m>n\geq N$. Por definición se tiene el siguiente sistema de ecuaciones:
        \begin{equation*}
            \left\{\begin{array}{rcl}
                s_m&=&u_1+\cdots+u_n+u_{ n+1}+\cdots+u_m\\
                s_{ m-1}&=&u_1+\cdots+u_n+u_{ n+1}+\cdots+u_{ m-1}\\
                \vdots\:\:\:\: & \vdots & \quad\quad\quad\quad\quad\quad\: \vdots \\
                s_{ n+2}&=&u_1+\cdots+u_n+u_{ n+1}+u_{ n+2}\\
                s_{ n+1}&=&u_1+\cdots+u_n+u_{ n+1}\\
            \end{array}\right.
        \end{equation*}
        Sumando todas las ecuaciones y agregando términos adicionales (de tal forma que la diagonal incompleta del sistema se complete) obtenemos que
        \begin{equation*}
            \begin{split}
                \Rightarrow (m-n)s_m&=s_m+s_{ m-1}+\cdots+s_{ n+1}+(m-n-1)u_m+(m-n-2)u_{ m-1}+\cdots+2u_{ n+3}+u_{ n+2}\\
            \end{split}
        \end{equation*}
        por tanto,
        \begin{equation*}
            \begin{split}
                (m-n)(s_m-\sigma)&=m(\sigma_m-\sigma)-n(\sigma_n-\sigma)+\\
                &(m-n-1)u_m+(m-n-2)u_{ m-1}+\cdots+2u_{ n+3}+u_{ n+2}\\
            \end{split}
        \end{equation*}
        Por la hipótesis, tomando norma y mayorando por desigualdad triangular, se sigue que
        \begin{equation*}
            \begin{split}
                (m-n)\norm{s_m-\sigma}&\leq (m+n)\varepsilon+(m-n-1)\frac{A}{m}+(m-n-2)\frac{A}{m-1}+\cdots+2\frac{A}{n+3}+\frac{A}{n+2}\\
                &\leq (m+n)\varepsilon+\frac{A}{n}\left[1+2+\cdots+(m-n-1) \right]\\
                &=(m-n)\varepsilon+2n\varepsilon+\frac{A}{n}\frac{(m-n-1)(m-n)}{2}\\
                &\leq(m-n)\varepsilon+2n\varepsilon+\frac{A}{n}\frac{(m-n)^2}{2}\\
                \Rightarrow \norm{s_m-\sigma}&\leq\varepsilon+\frac{2n\varepsilon}{m-n}+\frac{A}{2n}(m-n)\\
                &=\varepsilon+\frac{2\varepsilon}{\left(\frac{m}{n}-1\right)}+\frac{A}{2}\left(\frac{m}{n}-1\right)\\
            \end{split}
        \end{equation*}
        Suponga por el momento que para toda $m\in\mathbb{N}$ suficientemente grande se puede encontrar $n\in\mathbb{N}$ tal que
        \begin{equation*}
            m>n\geq N
        \end{equation*}
        y
        \begin{equation*}
            \sqrt{\varepsilon}\leq\frac{m}{n}-1\leq2\sqrt{\varepsilon}
        \end{equation*}
        Se seguiría de la ecuación anterior que
        \begin{equation*}
            \begin{split}
                \norm{s_m-\sigma}&<\varepsilon+2\sqrt{\varepsilon}+A\sqrt{\varepsilon}\\
                \Rightarrow \norm{s_m-\sigma}&<(3+a)\sqrt{\varepsilon}\\
            \end{split}
        \end{equation*}
        con lo que el teorema estaría probado. La condición enunciada anteriormente es equivalente a decir que
        \begin{equation*}
            \frac{m}{1+2\sqrt{\varepsilon}}\leq n<\frac{m}{1+\sqrt{\varepsilon}}
        \end{equation*}
        para que exista $n\in\mathbb{N}$ que cumpla lo anterior es suficiente que
        \begin{equation*}
            \begin{split}
                \frac{m}{1+\sqrt{\varepsilon}}-\frac{m}{1+2\sqrt{\varepsilon}}&\geq 1\\
                \iff \frac{\sqrt{\varepsilon}m}{(1+\sqrt{\varepsilon})(1+2\sqrt{\varepsilon})}&\geq 1\\
                \Leftarrow \frac{\sqrt{\varepsilon}m}{6}&\geq 1\\
            \end{split}
        \end{equation*}
        Así pues, si se toma $m>\frac{6}{\sqrt{\varepsilon}}$ y se cumple que podemos tomar $n\in\mathbb{N}$ con $m>n$ tal que
        \begin{equation*}
            \sqrt{\varepsilon}\leq\frac{m}{n}-1\leq2\sqrt{\varepsilon}
        \end{equation*}
        Entonces la condición estará probada. ¿Qué más hay que pedir a $m$ para que se cumpla que $m>n\geq N$? Para que también se cumpla la otra condidición basta con pedir que 
        \begin{equation*}
            \begin{split}
                \frac{m}{1+2\sqrt{\varepsilon}}&\geq N\\
                \Rightarrow m&\geq(1+2\sqrt{\varepsilon})N\\
            \end{split}
        \end{equation*}
        así pues, si $m\geq\max\left\{\frac{6}{\sqrt{\varepsilon}},(1+2\sqrt{\varepsilon})N \right\}$ con lo cual podemos elegir $n\geq N$ con $m>n\geq N$ que cumple lo deseado.
    \end{proof}

    \begin{cor}
        Sea $\left\{u_n \right\}_{ n=1}^\infty$ una sucesión en el espacio normado $\mathcal{B}(S,E)$ de funciones acotadas de $S$ en un espacio normado $E$, provisto de la norma uniforme. Si la serie $\sum_{ n=1}^\infty u_n$ converge en el sentido de Cesáro uniformemente en $S$ y existe $A>0$ tal que
        \begin{equation*}
            \norm{u_n(x)}\leq\frac{A}{n},\quad\forall x\in S \textup{ y }\forall n\in\mathbb{N}
        \end{equation*}
        entonces, la serie converge en el sentido usual uniformemente en $S$.
    \end{cor}

    \begin{proof}
        Es inmediato del teorema anterior.
    \end{proof}

    \section{Teorema de Jordan}

    Recuerde que una función $\cf{f}{[a,b]}{\mathbb{K}}$ es de variación acotada si el conjunto de sumas
    \begin{equation*}
        S_\Delta(f)=\sum_{ k=1}^n\abs{f(x_k)-f(x_{ k-1})}
    \end{equation*}
    donde $\Delta=\left\{a=x_0<x_1<\cdots<x_n=b \right\}$ es una subdivisión finita de $[a,b]$, es acotado en $\mathbb{R}$. En tal caso la variación de $f$ en $[a,b]$ se define como
    \begin{equation*}
        V_f([a,b])=\sup_{\Delta}S_\Delta(f)
    \end{equation*}

    Se sabe que toda función de variación acotada es acotada en $[a,b]$. En general, una función de variación acotada no necesariamente es continua en $[a,b]$ ni viceversa. Sin embargo, toda función monótona o de clase $C^1$ en $[a,b]$ es de variación acotada en $[a,b]$. También las monóntonas por trozos.

    También se sabe que para una función $\cf{f}{[a,b]}{\mathbb{C}}$ es de variación acotada si y sólo si $\Re f$ e $\Im f$ lo son en $[a,b]$. Además, el conjunto de funciones de variación acotada es un álgebra con identidad sobre el campo $\mathbb{K}$. Finalmente, $\cf{f}{[a,b]}{\mathbb{R}}$ es de variación acotada en $[a,b]$ si y sólo si $f$ se puede escribir como diferencia de dos funciones monótonas.

    \begin{obs}
        Si $\cf{f}{\mathbb{R}}{\mathbb{K}}$ es una función periódica de periodo $2\pi$ y si la reestricción de $f$ a algún intervalo compacto de longitud $2\pi$ es de variación acotada, entonces $f\in\mathcal{L}_{\infty}^{2\pi}$.
    \end{obs}

    \begin{lema}
        Sea $\cf{f}{\mathbb{R}}{\mathbb{K}}$ periódica de periodo $2\pi$ y de variación acotada en $[-\pi,\pi]$. Entonces, existe una constante $A>0$ tal que los coeficientes de Fourier $\left\{c_k \right\}_{ k\in\mathbb{Z}}$ de $f$ satisfacen lo siguiente:
        \begin{equation*}
            \abs{c_k}\leq\frac{A}{\abs{k}},\quad\forall k\in\mathbb{Z}\backslash\left\{0\right\}
        \end{equation*} 
    \end{lema}

    \begin{proof}
        Considerando por separado las partes reale imaginaria de la función se puede suponer que $f$ es real y más aún, que es monótona (pues al ser $f$ de variación acotada es la diferencia de funciones monótonas).

        Se tiene
        \begin{equation*}
            \begin{split}
                \abs{c_k}&=\frac{1}{2\pi}\abs{\int_{-\pi}^\pi f(t)e^{ -ikt}dt}\\
                &\leq\frac{1}{2\pi}\left[2\abs{f(\pi)}+\abs{f(-\pi)} \right]\max_{ -\pi\leq x\leq\pi}\abs{\int_{ -\pi}^x e^{ -ikt}dt}\\
            \end{split}
        \end{equation*}
        donde la primera desigualdad es por el segundo teorema del valor medio. Además,
        \begin{equation*}
            \abs{\int_{ -\pi}^x e^{ -ikt}dt}=\abs{\frac{e^{ ik\pi}-e^{- ikx}}{ik}}\leq\frac{2}{\abs{k}}
        \end{equation*}
        para todo $-\pi\leq x\leq\pi$. Por tanto
        \begin{equation*}
            \abs{c_k}\leq\frac{1}{\pi}[2\abs{f(\pi)}+\abs{f(-\pi)}]\frac{1}{\abs{k}},\quad\forall k\in\mathbb{Z}\backslash\left\{0\right\}
        \end{equation*}
        Tomando $A=\frac{1}{\pi}[2\abs{f(\pi)}+\abs{f(-\pi)}]>0$ se sigue el resultado (esto si la función no es cero c.t.p. en $\mathbb{R}$, en tal caso basta con tomar $A=1$).
    \end{proof}

    \begin{theor}[\textbf{Teorema de Jordan}]
        Sea $\cf{f}{\mathbb{R}}{\mathbb{K}}$ periódica de periodo $2\pi$ y de variación acotada en $[-\pi,\pi]$. Entonces,
        \begin{enumerate}
            \item Para todo $1\leq p<\infty$, la serie de Fourier de $f$ converge en el sentido usual a $f$ en $p$-promedio, o sea
            \begin{equation*}
                \lim_{ m\rightarrow\infty}\N{p}{f-s_m}=0
            \end{equation*}
            \item Si en un punto $x\in\mathbb{R}$ existen los límites laterales $f(x^+)$ y $f(x^-)$, entonces la serie de Fourier de $f$ converge en el sentido usual en $x$ a $\frac{f(x^+)+f(x^-)}{2}$. En particular, si $f$ es continua en $x$ se sigue que la serie de Fourier de $f$ converge en $x$ a $f(x)$ en el sentido usual (hablando de convergencia puntual).
            \item Si $f$ es continua en un abierto $J\subseteq\mathbb{R}$, entonces la serie de Fourier de $f$ converge en el sentido usual a $f$ uniformemente en todo compacto contenido en $J$.
            \item Si $f$ es continua en $\mathbb{R}$, entonces la serie de Fourier de $f$ converge en el sentido usual a $f$ uniformemente en $\mathbb{R}$.
        \end{enumerate}
    \end{theor}

    \begin{proof}
        Sean $\left\{c_k \right\}_{ k\in\mathbb{Z}}$ los coeficientes de Fourier de $f$. Por el lema anterior se tiene que existe $A>0$ tal que
        \begin{equation*}
            \abs{c_k}\leq\frac{A}{\abs{k}},\quad\forall k\in\mathbb{Z}\backslash\left\{0\right\}
        \end{equation*}
        Sea $m\geq2$. Se tiene 
        \begin{equation*}
            \begin{split}
                s_{ m-1}-s_{ m-2}&=c_{ m-1}e^{ i(m-1)x}+c_{ -(m-1)}e^{ -i(m-1)x}
            \end{split}
        \end{equation*}
        de donde,
        \begin{equation*}
            \begin{split}
                \abs{s_{ m-1}-s_{ m-2}}&\leq\abs{c_{ m-1}}+\abs{c_{ -(m-1)}}\\
                &\leq\frac{2A}{m-1}\\
                &=\frac{m}{m-1}\cdot\frac{2A}{m}\\
                &\leq\frac{4A}{m},\quad\forall m\geq2\\
            \end{split}
        \end{equation*}
        de donde
        \begin{equation*}
            \begin{split}
                \N{p}{s_{ m-1}-s_{ m-2}}&\leq\frac{4A}{m}\N{p}{1}\\
                &\leq\frac{4(2\pi)^{1/p}A}{m},\quad\forall m\geq 2\\
            \end{split}
        \end{equation*}
        El teorema de Jordan se sigue ahora del teorema de Fejér y de Hardy-Landau. Para obtener 1) se considera la serie de Fourier de $f$ como una serie en el espacio de Banach $L_p^{2\pi}(\mathbb{C})$.

        Para obtener 2), se considera la serie de Fourier de $f$ en el punto $x$ como una serie en el espacio de Banach $\mathbb{C}$. Para 3) se fija un compacto $C\subseteq J$ y se considera la serie de Fourier de $f$ como una serie en el espacio de Banach $\mathcal{BC}(K,\mathbb{C})$ provisto de la norma uniforme.

        De forma similar a 3), en 4) se considera a la serie de Fourier de $f$ como una serie en $\mathcal{ BC}(\mathbb{R},\mathbb{C})$, provisto de la norma uniforme.
    \end{proof}

    \begin{cor}
        Sea $f\in\mathcal{L}_1^{2\pi}$ tal que $\int_{-\pi}^\pi f=0$ y sea $\cf{F}{\mathbb{R}}{\mathbb{K}}$ la integral indefinida
        \begin{equation*}
            F(x)=c+\int_0^x f(t)dt,\quad\forall x\in\mathbb{R}
        \end{equation*}
        donde $c\in\mathbb{R}$. Entonces, $F\in\mathcal{C}^{2\pi}(\mathbb{K})$ y la serie de Fourier de $F$ converge a $F$ uniformemente en $\mathbb{R}$. 
    \end{cor}

    \begin{proof}
        Ya se sabe en esas condiciones que $F\in\mathcal{C}^{2\pi}(\mathbb{K})$. El teorema se seguirá del teorema de Jordan si se prueba que $F$ es de variación acotada en $[-\pi,\pi]$.

        Sea $x_0=-\pi<x_1<\cdots<x_k=\pi$. Se tiene
        \begin{equation*}
            \begin{split}
                \sum_{ k=1}^n\abs{F(x_k)-F(x_{ k-1})}&=\sum_{ k=1}^n\abs{\int_{ x_{ k-1}}^{x_k}f}\\
                &\leq\sum_{ k=1}^n\int_{ x_{ k-1}}^{x_k}\abs{f}\\
                &=\int_{-\pi}^\pi\abs{f}<\infty \\
            \end{split}
        \end{equation*}
        por tanto, $f$ es de variación acotada en $[-\pi,\pi]$.
    \end{proof}

    \begin{exa}
        Sea $\cf{f}{\mathbb{R}}{\mathbb{R}}$ periódica de periodo $2\pi$ tal que
        \begin{equation*}
            f(x)=\pi-\abs{x},\quad\forall x\in[-\pi,\pi[
        \end{equation*}
        Por el teorema de Jordan se sigue que la serie de Fourier de $f$ converge a $f$ uniformemente en $\mathbb{R}$ (pues la función $f$ es continua en $\mathbb{R}$). Calculemos los coeficientes de Fourier de $f$:
        \begin{equation*}
            \begin{split}
                a_n&=\frac{1}{\pi}\int_{ -\pi}^\pi f(x)\cos nxdx\\
                &=\frac{1}{\pi}\int_{ -\pi}^\pi (\pi-\abs{x})\cos nxdx\\
                &=\frac{1}{\pi}\left[\int_{ -\pi}^0 (\pi-\abs{x})\cos nxdx+\int_{0}^\pi (\pi-\abs{x})\cos nxdx\right]\\
                &=\frac{1}{\pi}\left[\int_{ -\pi}^0 (\pi+x)\cos nxdx+\int_{0}^\pi (\pi-x)\cos nxdx\right]\\
                &=\frac{2}{\pi}\int_0^{\pi}(\pi-x)\cos nxdx\\
                &=2\int_0^{\pi}\cos nxdx-\frac{2}{\pi}\int_{0}^\pi x\cos nxdx \\
            \end{split}
        \end{equation*}
        si $n=0$:
        \begin{equation*}
            \begin{split}
                a_0&=2\int_0^{\pi}
            \end{split}
        \end{equation*}
        y para $n\geq1$:
        \begin{equation*}
            a_n=2\int_0^{\pi}\cos nxdx-\frac{2}{\pi}\int_{0}^\pi x\cos nxdx
        \end{equation*}
        donde, haciendo el cambio de variable $u=nx$
        \begin{equation*}
            \begin{split}
                \int_0^{\pi}\cos nxdx&=\int_0^{ n\pi}\frac{\cos udu}{n}\\
                &=\frac{1}{n}\sin u\Big|_0^{n\pi}\\
                &=\frac{1}{n}[\sin n\pi-0]\\
                &=0\\
            \end{split}
        \end{equation*}
        y,
        \begin{equation*}
            \begin{split}
                \int_{0}^\pi x\cos nxdx&=\int_{0}^{n\pi}\frac{u}{n}\cos u\frac{du}{n}\\
                &=\frac{1}{n^2}\int_{0}^{n\pi}u\cos udu\\
                &=\frac{1}{n^2}[u\sin u\Big|_0^{n\pi}-\int_0^{n\pi}\sin udu]\\
                &=\frac{1}{n^2}[n_\pi\sin n\pi-0+\cos u\Big|_0^{n\pi}]\\
                &=\frac{1}{n^2}[n\pi\sin n\pi-0+\cos u\Big|_0^{n\pi}]\\
                &=\frac{1}{n^2}\cos u\Big|_0^{n\pi}\\
                &=\frac{1}{n^2}[(-1)^{n}-1] \\
            \end{split}
        \end{equation*}
        por tanto,
        \begin{equation*}
            a_n=-\frac{2}{\pi}\cdot\frac{1}{n^2}[(-1)^{ n}-1]=\frac{2}{\pi n^2}[(-1)^{ n+1}+1]
        \end{equation*}
        y, como $f$ es par, entonces $b_n=0$. Por tanto, la serie de Fourier de $f$ es:
        \begin{equation*}
            \frac{\pi}{2}+\sum_{ k=1}^\infty\frac{4}{\pi(2k-1)^2}\cos(2k-1)x
        \end{equation*}
        Por el criterio de Dini:
        \begin{equation*}
            \frac{\pi}{2}+\sum_{ k=1}^\infty\frac{4}{\pi(2k-1)^2}\cos(2k-1)x=\left\{
                \begin{array}{lcr}
                    f(x) & \textup{ si } & x\in\mathbb{R}\\
                    \pi-\abs{x} & \textup{ si } & -\pi\leq x\leq\pi\\
                \end{array}
            \right.
        \end{equation*}
        Si $x=0$:
        \begin{equation*}
            \frac{\pi}{2}+\sum_{ k=1}^\infty\frac{4}{\pi(2k-1)^2}=\frac{\pi}{2}
        \end{equation*}
        luego,
        \begin{equation*}
            \sum_{ k=1}^\infty\frac{1}{(2k-1)^2}=\frac{\pi^2}{8}
        \end{equation*}
        Si $x=\pi$ se puede hacer lo mismo. Además,
        \begin{equation*}
            \begin{split}
                \alpha&=\sum_{ k=1}^\infty\frac{1}{(2k-1)^2}+\sum_{ k=1}^\infty\frac{1}{(2k)^2}\\
                &=\sum_{ k=1}^\infty\frac{1}{(2k-1)^2}+\frac{1}{4}\alpha\\
                \Rightarrow \frac{3}{4}\alpha&=\sum_{ k=1}^\infty\frac{1}{(2k-1)^2}\\
                \Rightarrow \alpha&=\frac{4}{3}\cdot\frac{\pi^2}{8}\\
                &=\frac{\pi^2}{6}\\
            \end{split}
        \end{equation*}

        Además, por Jordan la serie de Fourier de $f$ converge a $f$ uniformemente en $\mathbb{R}$. Más aún, $f\in\mathcal{L}_2^{2\pi}$, luego por Parseval:
        \begin{equation*}
            \frac{\abs{a_0}^2}{2}+\sum_{ k=1}^\infty\left[\abs{a_k}^2+\abs{b_k}^2 \right]=\frac{1}{\pi}\int_{-\pi}^{\pi}\abs{f}
        \end{equation*}
        por tanto,
        \begin{equation*}
            \frac{\pi^2}{2}+\frac{16}{\pi^2}\sum_{ k=1}^\infty\frac{1}{(2k-1)^4}=\frac{1}{\pi}\int_{-\pi}^{\pi}\abs{\pi-\abs{x}}dx=\frac{2\pi^2}{3}
        \end{equation*}
        con lo que se llega a que
        \begin{equation*}
            \sum_{ k=1}^\infty\frac{1}{(2k-1)^4}=\frac{\pi^4}{96}
        \end{equation*}
        y,
        \begin{equation*}
            \sum_{ k=1}^\infty\frac{1}{k^4}=\frac{\pi^4}{90}
        \end{equation*}
    \end{exa}

    \section{Puntos de Lebesgue}    

    Se probará que si $f\in\mathcal{L}_1(\mathbb{R}^n,\mathbb{K})$, entonces casi todo punto de $\mathbb{R}^n$ es un punto de Lebesgue de $f$. Será necesario probar algunos resultados preliminares.

    \begin{mydef}
        Sea $f\in\mathcal{L}_1(\mathbb{R}^n,\mathbb{K})$. Para cada $x\in\mathbb{R}^n$ se define
        \begin{equation*}
            \mathcal{M}f(x)=\sup_{r>0}\frac{1}{m(B(x,r))}\int_{B(x,r)}\abs{f}\leq\infty
        \end{equation*}
        donde $B(x,r)$ es la bola con centro $x$ y de radio $r$ con respecto a cualquier norma de $\mathbb{R}^n$ (pues todas las normas son equivalentes).
    \end{mydef}

    \begin{propo}
        Si $f\in\mathcal{L}_1(\mathbb{R})^n,\mathbb{K}$, entonces para todo $\lambda>0$ el conjunto
        \begin{equation*}
            E_\lambda=\left\{x\in\mathbb{R}^n\Big|\mathcal{M}f(x)>\lambda \right\}
        \end{equation*}
        es abierto. 
    \end{propo}

    \begin{proof}
        Sea $\lambda>0$. Suponga que el conjunto $E_\lambda$ es no vacío. Como $\mathcal{M}f(x)>\lambda$ para todo $x\in\lambda$. Sea $x\in E_\lambda$, entonces existe $r>0$ tal que
        \begin{equation*}
            c=\frac{1}{m(B(x,r))}\int_{ B(x,r)}\abs{f}>\lambda
        \end{equation*}
        Sea $\delta>0$. Si $y\in B(x,\delta)$, entonces $B(x,r)\subseteq B(y,r+\delta)$. Entonces
        \begin{equation*}
            \begin{split}
                \lambda&<c\\
                &=\frac{1}{m(B(x,r))}\int_{ B(x,r)}\abs{f}\\
                &\leq\frac{m(B(y,r+\delta))}{m(B(x,r))}\cdot\frac{1}{m(B(y,r+\delta))}\int_{ B(y,r+\delta)}\abs{f}\\
                &=\left(\frac{r+\delta}{r} \right)^n\frac{1}{m(B(y,r+\delta))}\int_{ B(y,r+\delta)}\abs{f}\\
            \end{split}
        \end{equation*}
        Ahora, como
        \begin{equation*}
            \lim_{\delta\rightarrow0^+}\left(\frac{r+\delta}{r}\right)^n=1
        \end{equation*}
        y
        \begin{equation*}
            \frac{c}{\lambda}>1
        \end{equation*}
        entonces existe $\delta_0>0$ tal que
        \begin{equation*}
            \left(\frac{r+\delta_0}{r}\right)^n<\frac{c}{\lambda}
        \end{equation*}
        Por tanto,
        \begin{equation*}
            \begin{split}
                c\leq\left(\frac{r+\delta_0}{r} \right)^n\frac{1}{m(B(y,r+\delta_0))}\int_{ B(y,r+\delta_0)}\abs{f}\\
                &<\frac{c}{\lambda}\mathcal{M}f(y)\\
                \Rightarrow \lambda<\mathcal{M}f(y)\\
            \end{split}
        \end{equation*}
        por tanto, $y\in B(x,\delta_0)$ implica que $\mathcal{M}f(y)>\lambda$, es decir que $B(x,\delta_0)\subseteq E_\lambda$. Se sigue entonces que $E_\lambda$ es abierto.
    \end{proof}

    \begin{lema}[\textbf{Lema de recubrimiento}]
        Para cada colección finita $\left\{B(x_1,r_1),...,B(x_N,r_N) \right\}$, de bolas abiertas en $\mathbb{R}^n$, existe un conjunto $S\subseteq\left\{1,N\right\}$ tal que la correspondiente subfamilia
        \begin{equation*}
            \left\{B(x_i,r_i)\right\}_{ i\in S}
        \end{equation*}
        es disjunta y
        \begin{equation*}
            \bigcup_{ i=1}^N B(x_i,r_i)\subseteq\bigcup_{ i\in S}B(x_r,3r_i)
        \end{equation*}
    \end{lema}

    \begin{proof}
        Podemos suponer que
        \begin{equation*}
            r_1\geq r_2\geq ...\geq r_N
        \end{equation*}
        Se construirá $S=\left\{i_1,...,i_m \right\}\subseteq\left\{1,...,N\right\}$ inductivamente. Se define $i_1=1$. Sea
        \begin{equation*}
            i_2=\min\left\{j\in\natint{i_1+1,N}\Big|B(x_j,r_j)\cap B(x_2, r_1)=\emptyset \right\}
        \end{equation*}
        si el conjunto es no vacío y si el conjunto es vacío se toma $S=\left\{i_1 \right\}$.

        Suponga elegidos $i_1,...,i_k\in\natint{1,N}$ con $i_1<...<i_k<N$. Sea
        \begin{equation*}
            i_{ k+1}=\left\{i\in\natint{i_k+1,N}\Big|B(x_j,r_j)\cap B(x_i,r_i)=\emptyset,\forall i\in\natint{1,k} \right\}
        \end{equation*}
        Si el conjunto es no vacío, en caso contrario se pone $S=\left\{i_1,...,i_k\right\}$.

        Este proceso eventualmente termina después de un número finito de pasos. Suponga así elegido el conjunto $S\subseteq\natint{1,N}$, siendo
        \begin{equation*}
            S=\left\{i_1,...,i_m\right\}
        \end{equation*}
        Por construcción, la familia $\left\{B(x_j,r_j) \right\}_{ j\in S}$ es disjunta. Fije $j\in\natint{1,N}$, entonces existe $k\in\natint{1,m}$ tal que $i_k\leq j<i_{ k+1}$ o bien $k=m$ e $i_m<j$. En ambos casos existe $l\in\natint{1,k}$ tal que
        \begin{equation*}
            B(x_j,r_j)\cap B(x_{ i_l},r_{ i_l})\neq\emptyset
        \end{equation*}
        (por la forma en que se eligieron los elementos de $S$) note que
        \begin{equation*}
            r_j\leq r_{ i_k}\leq r_{ i_l}
        \end{equation*}
        Sea $z\in B(x_j,r_j)$ y sea $w\in B(x_j,r_j)\cap B(x_{ i_l},r_{ i_l})$. Entonces,
        \begin{equation*}
            \begin{split}
                \norm{z-x_{ i_l}}&\leq\norm{z-x_j}+\norm{x_j-w}+\norm{w-x_{ i_l}}\\
                &\leq r_j+r_j+r_{ i_l}\\
                &\leq 3r_{ i_l}\\
            \end{split}
        \end{equation*}
        Por ende, $z\in B(x_{ i_l},r_{ i_l})$. Luego,
        \begin{equation*}
            \bigcup_{ i=1}^N B(x_i,r_i)\subseteq\bigcup_{ i\in S}B(x_r,3r_i)
        \end{equation*}
    \end{proof}

    \begin{propo}
        Si $f\in\mathcal{L}_1(\mathbb{R}^n,\mathbb{K})$, entonces para todo $\lambda>0$ sea
        \begin{equation*}
            E_\lambda=\left\{x\in\mathbb{R}^n\Big|\mathcal{M}f(x)>\lambda \right\}
        \end{equation*}
        Entonces,
        \begin{equation*}
            m(E_\lambda)\leq\frac{3^n}{\lambda}\N{1}{f}
        \end{equation*}
    \end{propo}

    \begin{proof}
        Sea $\lambda>0$. Es claro que $E_\lambda$ es medible (por ser abierto). Sea $K\subseteq E_\lambda$ un subconjunto compacto arbitrario de $E_\lambda$.

        Para cada $x\in K$ existe $r_x>0$ tal que
        \begin{equation*}
            \frac{1}{m(B(x,r_x))}\int_{B(x,r_x)}\abs{f}>\lambda
        \end{equation*}
        (por ser $E_\lambda$ abierto). Ya que $K\subseteq \bigcup_{ x\in K}B(x,r_x)$ y $K$ es compacto, existen $x_1,...,x_N\in K$ tales que
        \begin{equation*}
            K\subseteq \bigcup_{ i=1}^N B(x_i,r_{ x_i})
        \end{equation*}
        Por el Lema del recubrimiento existe una subfamilia disjunta $\left\{B(x_{ i_1},r_{ x_{i_1}}),...,B(x_{ i_m},r_{ x_{i_m}})\right\}$ tales que
        \begin{equation*}
            \bigcup_{ i=1}^N B(x_i,r_{ x_i})\subseteq\bigcup_{ j=1}^{ m}B(x_{ i_j},3r_{x_{ i_j}} )
        \end{equation*}
        Entonces,
        \begin{equation*}
            \begin{split}
                m(K)&\leq\sum_{ j=1}^m B(x_{ i_j},3r_{x_{ i_j}})\\
                &\leq3^n\sum_{ j=1}^m B(x_{ i_j},r_{x_{ i_j}})\\
                &<3^n\sum_{ j=1}^m \frac{1}{\lambda}\int_{ B(x_{ i_j},r_{x_{ i_j}})}\abs{f}\\
                &\leq\frac{3^n}{\lambda}\int_{\bigcup_{ j=1}^{ m}B(x_{ i_j},r_{x_{ i_j}})}\abs{f}\\
                &\leq\frac{3^n}{\lambda}\int_{\mathbb{R}^n}\abs{f}\\
                &=\frac{3^n}{\lambda}\N{1}{f}\\
                \Rightarrow m(K)&<\frac{3^n}{\lambda}\N{1}{f}\\
            \end{split}
        \end{equation*}
        Ahora, como $E_\lambda$ es abierto, podemos escribirlo como:
        \begin{equation*}
            E_\lambda=\bigcup_{ \nu=1}^\infty C_{\nu}
        \end{equation*}
        donde los $C_\nu$ son cubos disjuntos tales que $\overline{C_\nu}\subseteq E_\lambda$ para todo $\nu\in\mathbb{N}$. Si
        \begin{equation*}
            Q_k=\bigcup_{ \nu=1}^k\overline{C_\nu}
        \end{equation*}
        entonces $\left\{Q_k \right\}_{ k=1}^\infty$ es una sucesión creciente de compactos cuya unión es $E_\lambda$, entonces por el Teorema de continuidad para la medida de Lebesgue:
        \begin{equation*}
            m(E_\lambda)=\lim_{ k\rightarrow\infty}m(Q_k)\leq\frac{3^n}{\lambda}\N{1}{f}
        \end{equation*}
        como se quería demostrar.
    \end{proof}

    \begin{mydef}
        Sea $\cf{f}{\mathbb{R}^n}{\mathbb{K}}$ función localmente integrable. Se dice que $x\in\mathbb{R}^n$ \textbf{es un punto de Lebesgue de $f$}, si
        \begin{equation*}
            \lim_{r\rightarrow0^+} \frac{1}{m(B(x,r))}\int_{B(x,r)}\abs{f(y)-f(x)}dy=0
        \end{equation*}
    \end{mydef}

    \begin{obs}
        Claramente si $f$ es continua en un punto $x\in\mathbb{R}^n$, entonces $x$ es un punto de Lebesgue de $f$.
    \end{obs}

    \begin{obs}
        La definición de punto de Lebesgue para una función $\cf{f}{I\subseteq\mathbb{R}}{\mathbb{K}}$ coincide con la nueva definición (siendo $I$ un intervalo abierto).
    \end{obs}

    \begin{proof}
        En efecto, sea $I\subseteq\mathbb{R}$ un intervalo abierto y $\cf{f}{I\subseteq\mathbb{R}}{\mathbb{K}}$ localmente integrable en $I$. Fije $x\in I$ y sea $h_0>0$ tal que
        $]x-h_0,x+h_0[\subseteq I$. Se tiene
        \begin{equation*}
            \begin{split}
                \frac{1}{m(B(x,h))}\int_{B(x,h)}\abs{f(t)-f(x)}\:dt&=\frac{1}{2h}\int_{ x-h}^{ x+h}\abs{f(t)-f(x)}\:dt\\
                &=\frac{1}{2}\left[\frac{1}{h}\int_{ x-h}^{x}\abs{f(t)-f(x)}\:dt+\frac{1}{h}\int_{ x}^{ x+h}\abs{f(t)-f(x)}\:dt \right]
            \end{split}
        \end{equation*}
        para $0<h<h_0$. Se tiene que
        \begin{equation*}
            \lim_{h\rightarrow0^+}\frac{1}{m(B(x,h))}\int_{B(x,h)}\abs{f(t)-f(x)}\:dt=0
        \end{equation*}
        si y sólo si
        \begin{equation*}
            \lim_{h\rightarrow0^+}\frac{1}{h}\int_{ x-h}^{x}\abs{f(t)-f(x)}\:dt=0=\lim_{h\rightarrow0^+}\frac{1}{h}\int_{ x}^{ x+h}\abs{f(t)-f(x)}\:dt
        \end{equation*}
        (pues recuerde que las integrales no son orientadas y son positivas ambos lados) si y sólo si
        \begin{equation*}
            \lim_{h\rightarrow0}\frac{1}{h}\int_{ x}^{ x+h}\abs{f(t)-f(x)}\:dt=0
        \end{equation*}
        por el teorema de cambio de variable tomando $t=u+x$ se tiene que esto ocurre si y sólo si
        \begin{equation*}
            \lim_{h\rightarrow0}\frac{1}{h}\int_{ x}^{ x+h}\abs{f(x+u)-f(x)}\:du=0
        \end{equation*}
        lo cual prueba la equivalencia.
    \end{proof}

    \begin{mydef}
        Sean $(X,d)$ un espacio métrico y $\cf{f}{X}{\overline{\mathbb{R}}}$ una función y $a\in X$. Se definen los \textbf{límites superior e inferior de $f$ en el punto $a$} como:
        \begin{equation*}
            \limsup_{ x\rightarrow a}f(x)=\inf_{r>0}\left(\sup_{ x\in B(a,r)}f(x) \right)=\lim_{ r\rightarrow0^+}\left(\sup_{ x\in B(a,r)}f(x) \right)
        \end{equation*}
        y
        \begin{equation*}
            \liminf_{ x\rightarrow a}f(x)=\sup_{r>0}\left(\inf_{ x\in B(a,r)}f(x) \right)=\lim_{ r\rightarrow0^+}\left(\inf_{ x\in B(a,r)}f(x) \right)
        \end{equation*}
    \end{mydef}

    \begin{obs}
        Los límites de la definición anterior son límites que siempre existen y son los ordinarios en $\overline{\mathbb{R}}$, pues las funciones
        \begin{equation*}
            r\mapsto\sup_{x\in B(a,r)}f(x)\quad\textup{y}\quad r\mapsto\inf_{x\in B(a,r)}f(x)
        \end{equation*}
        son funciones monótonas.
    \end{obs}

    \begin{propo}
        Sea $(X,d)$ espacio métrico y $\cf{f,g}{X}{\overline{\mathbb{R}}}$ funciones. Entonces,
        \begin{enumerate}
            \item $\liminf_{x\rightarrow a}f(x)\leq\limsup_{x\rightarrow a}f(x)$.
            \item $\liminf_{x\rightarrow a}f(x)=\limsup_{x\rightarrow a}f(x)$ si y sólo si existe $\lim_{x\rightarrow a}f(x)$ y
            \begin{equation*}
                \liminf_{x\rightarrow a}f(x)=\lim_{x\rightarrow a}f(x)\limsup_{x\rightarrow a}f(x)
            \end{equation*}
            \item Si $f,g$ son no negativas, entonces
            \begin{equation*}
                \limsup_{x\rightarrow a}(f+g)(x)\leq \limsup_{x\rightarrow a}f(x)+\limsup_{x\rightarrow a}g(x)
            \end{equation*}
            y,
            \begin{equation*}
                \liminf_{x\rightarrow a}f(x)+\liminf_{x\rightarrow a}g(x)\leq \liminf_{x\rightarrow a}(f+g)(x)
            \end{equation*}
            \item Si $(Y,\rho)$ es otro espacio métrico, $\cf{f}{X}{Y}$ y, $a\in X$ y $b\in Y$, entonces
            \begin{equation*}
                \lim_{x\rightarrow a}f(x)=b\iff \limsup_{x\rightarrow a}\rho(f(x),b)=0
            \end{equation*}
            en particular, si $f\geq0$ (siendo $Y=\overline{\mathbb{R}}$), entonces
            \begin{equation*}
                \lim_{x\rightarrow a}f(x)=0\iff \limsup_{x\rightarrow a}f(x)=0
            \end{equation*}
            esta propiedad se utiliza para determinar si puede o no ser $b$ el límite de $f$ en $a$.
        \end{enumerate}
    \end{propo}

    \begin{proof}
        Ejercicio.
    \end{proof}

    \begin{theor}
        Si $f\in\mathcal{L}_1(\mathbb{R}^n,\mathbb{K})$ entonces, casi todo punto de $\mathbb{R}^n$ es un punto de Lebesgue.
    \end{theor}

    \begin{proof}
        $\forall x\in\mathbb{R}^n$ y para todo $r>0$, escriba
        \begin{equation*}
            T_rf(x)=\frac{1}{m(B(x,r))}\int_{ B(x,r)}\abs{f(y)-f(x)}dy
        \end{equation*}
        y, sea
        \begin{equation*}
            Tf(x)=\limsup_{ r\rightarrow 0^+}T_rf(x)
        \end{equation*}
        Es claro por la observación de la proposición anterior que $x$ es un punto de Lebesgue si y sólo si $Tf(x)=0$. Así pues, basta probar que $Tf=0$ c.t.p. en $\mathbb{R}^n$.

        Sea $\varepsilon>0$, existe una función $g\in\mathcal{C}_c(\mathbb{R}^n,\mathbb{K})$ que aproxima en promedio a la $f$, esto es
        \begin{equation*}
            \N{1}{f-g}<\varepsilon
        \end{equation*}
        si $h=f-g$, entonces $f=g+h$ donde $\N{1}{h}<\varepsilon$. Se tiene lo siguiente: $\forall x\in\mathbb{R}^n$ y para todo $r>0$
        \begin{equation*}
            \begin{split}
                T_rf(x)&=T_r(g+h)(x)\\
                &=\frac{1}{m(B(x,r))}\int_{B(x,r)}\abs{(g+h)(y)-(g+h)(x)}dy\\
                &\leq\frac{1}{m(B(x,r))}\abs{g(y)-g(x)}dy+\frac{1}{m(B(x,r))}\abs{h(y)-h(x)}dy\\
                &=T_rg(x)+T_rh(x)\\
            \end{split}
        \end{equation*}
        tomando límite superior cuando $r\rightarrow0^+$ obtenemos que
        \begin{equation*}
            Tf(x)\leq Tg(x)-Th(x)=Th(x)
        \end{equation*}
        donde la última igualdad se cumple por ser $g$ continua en $\mathbb{R}^n$ (por esto se tiene que $Tg(x)=0$). Ahora, para todo $x\in\mathbb{R}^n$ y para todo $r>0$:
        \begin{equation*}
            \begin{split}
                T_rh(x)&=\frac{1}{m(B(x,r))}\int_{B(x,r)}\abs{h(y)-h(x)}dy\\
                &\leq\frac{1}{m(B(x,r))}\int_{B(x,r)}\abs{h(y)}dy+\abs{h(x)}\\
                &\leq\mathcal{M}h(x)+\abs{h(x)}\\
            \end{split}
        \end{equation*}
        donde
        \begin{equation*}
            \mathcal{M}h(x)=\sup_{ r>0}\left(\frac{1}{m(B(x,r))}\int_{B(x,r)}\abs{h}\right)
        \end{equation*}
        Al tomar límite superior cuando $r\rightarrow0^+$, resulta
        \begin{equation*}
            Th(x)\leq\mathcal{M}h(x)+\abs{h(x)}
        \end{equation*}
        Luego,
        \begin{equation*}
            Tf(x)\leq\mathcal{M}h(x)+\abs{h(x)}
        \end{equation*}
        Siendo $Tf\geq0$, para probar que $Tf=0$ c.t.p. en $\mathbb{R}^n$, basta probar que el conjunto
        \begin{equation*}
            \left\{x\in\mathbb{R}^n\Big|Tf(x)>\frac{1}{k} \right\}
        \end{equation*}
        es despreciable para todo $k\in\mathbb{N}$. Por una desigualdad anterior:
        \begin{equation*}
            \begin{split}
                \left\{x\in\mathbb{R}^n\Big|Tf(x)>\frac{1}{k} \right\}\subseteq\underbrace{\left\{x\in\mathbb{R}^n\Big|\mathcal{M}h(x)>\frac{1}{2k} \right\}}_{ =A_k} \cup\underbrace{\left\{x\in\mathbb{R}^n\Big|\abs{h(x)}>\frac{1}{2k}\right\}}_{=B_k}\\
            \end{split}
        \end{equation*}
        es decir que
        \begin{equation*}
            C_k=A_k\cup B_k
        \end{equation*}
        Se sabe de antemano que
        \begin{equation*}
            m(A_k)\leq 3^n2k\N{1}{h}<3^n2k\varepsilon
        \end{equation*}
        Además,
        \begin{equation*}
            \abs{h}\geq\frac{1}{2k}\chi_B\Rightarrow m(B)\leq 2k\N{1}{h}<2k\varepsilon
        \end{equation*}
        Por tanto, para cada $k\in\mathbb{N}$ fijo,
        \begin{equation*}
            m(C_k)\leq (3^n2k+2k)\varepsilon
        \end{equation*}
        Como esta desigualdad es cierta para todo $\varepsilon>0$, necesariamente
        \begin{equation*}
            m(C_k)=0,\quad\forall k\in\mathbb{N}
        \end{equation*}
        Así pues,
        \begin{equation*}
            \begin{split}
                m^*\left(\left\{x\in\mathbb{R}^n\Big|Tf(x)>0 \right\} \right)&=m^*\left(\bigcup_{ k=1}^\infty C_k \right)\\
                &\leq \sum_{ k=1}^\infty m^*(C_k)\\
                &=0\\
            \end{split}
        \end{equation*}
        por tanto, $Tf=0$ c.t.p. en $\mathbb{R}^n$.
    \end{proof}

    \begin{cor}
        Si $\cf{f}{\mathbb{R}^n}{\mathbb{K}}$ es localmente integrable en $\mathbb{R}^n$, entonces casi todo punto de $\mathbb{R}^n$ es punto de Lebesgue.
    \end{cor}

    \begin{proof}
        Sea $L$ el conjunto de puntos de Lebesgue de $f$. Como $f$ es localmente integrable, entonces
        \begin{equation*}
            f_k=f\chi_{B(0,k)}
        \end{equation*}
        es integrable para todo $k\in\mathbb{N}$. Observe que el conjunto de puntos de Lebesgue de $f_k$ en $B(0,k)$ es $L\cap B(0,k)$. Por el teorema anterior
        \begin{equation*}
            m(L^c\cap B(0,k))=0,\quad\forall k\in\mathbb{N}
        \end{equation*}
        luego,
        \begin{equation*}
            m(L^c)\leq\sum_{ k=1}^\infty m(L^c\cap B(0,k))=0
        \end{equation*}
        luego, casi todo punto de $f$ es de Lebesgue.
    \end{proof}

    \begin{cor}
        Sea $\Omega\subseteq\mathbb{R}^n$ un conjunto abierto y $\cf{f}{\Omega}{\mathbb{K}}$. Entonces, casi todo punto de $\Omega$ es un punto de Lebesgue de $f$.
    \end{cor}

    \begin{proof}
        Sea $L$ el conjunto de puntos de Lebesgue de $f$ en $\Omega$. Para cada $x\in\Omega$ escoja un radio $r_x>0$ tal que $B'(x,r_x)\subseteq\Omega$. Por hipótesis $\cf{\tilde{f}\chi_{ B'(x,r_x)}}{\mathbb{R}^n}{\mathbb{K}}$ es integrable en $\mathbb{R}^n$. Observe que el conjunto de puntos de Lebesgue de $\tilde{f}\chi_{ B'(x,r_x)}$ dentro de $B(x,r_x)$ coincide con $L\cap B(x,r_x)$. Por el teorema
        \begin{equation*}
            m(B(x,r_x)\backslash L)=0,\forall x\in\Omega
        \end{equation*}
        Como $\left\{B(x,r_x) \right\}_{x\in\Omega}$ es una cubierta abierta de $\Omega$, existe una subcubierta numerable $\left\{B(x_n,r_{ x_n})\right\}_{n=1}^\infty$ de $\Omega$ (por Lindelöf), de donde
        \begin{equation*}
            m(\Omega\backslash L)=m\left(\bigcup_{ n=1}^\infty B(x_n,r_{ x_n})\backslash L\right)\leq\sum_{ n=1}^\infty m(B(x_n,r_{x_n})\backslash L)=0
        \end{equation*}
        es decir que casi todo punto de $f$ es de Lebesgue.
    \end{proof}
    
    \begin{theor}[\textbf{Primer Teorema Fundamental del Cálculo} (otro Teorema de Lebesgue).]
        Sea $\cf{f}{I\subseteq\mathbb{R}}{\mathbb{K}}$ función localmente integrable en $I$, siendo éste un intervalo abierto. Fije $a\in I$, sea $\cf{F}{I}{\mathbb{K}}$ la función:
        \begin{equation*}
            F(t)=\int_a^t f,\quad\forall t\in I
        \end{equation*}
        Entonces $F'(x)=f(x)$ para casi toda $x\in I$.
    \end{theor}

    \begin{proof}
        Basta probar que $F'(x)=f(x)$ en todo punto de Lebesgue $x\in I$ (ya que casi todos los puntos de $f$ son de Lebesgue). Sea $x\in I$ un punto de Lebesgue de $f$. Se tiene, $\forall h\neq0$ tal que $[x-\abs{h},x+\abs{h}]\subseteq I$,
        \begin{equation*}
            \begin{split}
                \abs{\frac{F(x+h)-F(x)}{h}-f(x)}&=\abs{\frac{1}{h}\int_x^{x+h}f(t)\:dt-f(x)}\\
                &=\abs{\frac{1}{h}\int_x^{x+h}\left[f(t)-f(x)\right]\:dt}\\
                &\leq\abs{\frac{1}{h}\int_x^{ x+h}\abs{f(t)-f(x)}\:dt}\\
                &=\abs{\frac{1}{h}\int_0^h\abs{f(x+u)-f(x)}\:du}\\
            \end{split}
        \end{equation*}
        haciendo el cambio de variable $t=x+u$. Como $x$ es un punto de Lebesgue se tiene que
        \begin{equation*}
            \lim_{ h\rightarrow0}\frac{1}{h}\int_0^h\abs{f(x+h)-f(x)}\:du=0
        \end{equation*}
        por tanto,
        \begin{equation*}
            \lim_{ h\rightarrow0}\abs{\frac{1}{h}\frac{F(x+h)-F(x)}{h}-f(x)}=0
        \end{equation*}
        Así, $F'(x)=f(x)$.
    \end{proof}

\end{document}