\documentclass[12pt]{report}
\usepackage[spanish]{babel}
\usepackage[utf8]{inputenc}
\usepackage{amsmath}
\usepackage{amssymb}
\usepackage{amsthm}
\usepackage{graphics}
\usepackage{subfigure}
\usepackage{lipsum}
\usepackage{array}
\usepackage{multicol}
\usepackage{enumerate}
\usepackage[framemethod=TikZ]{mdframed}
\usepackage[a4paper, margin = 1.5cm]{geometry}

%En esta parte se hacen redefiniciones de algunos comandos para que resulte agradable el verlos%

\renewcommand{\theenumii}{\roman{enumii}}

\def\proof{\paragraph{Demostración:\\}}
\def\endproof{\hfill$\blacksquare$}

\def\sol{\paragraph{Solución:\\}}
\def\endsol{\hfill$\square$}

%En esta parte se definen los comandos a usar dentro del documento para enlistar%

\newtheoremstyle{largebreak}
  {}% use the default space above
  {}% use the default space below
  {\normalfont}% body font
  {}% indent (0pt)
  {\bfseries}% header font
  {}% punctuation
  {\newline}% break after header
  {}% header spec

\theoremstyle{largebreak}

\newmdtheoremenv[
    leftmargin=0em,
    rightmargin=0em,
    innertopmargin=-2pt,
    innerbottommargin=8pt,
    hidealllines = true,
    roundcorner = 5pt,
    backgroundcolor = gray!60!red!30
]{exa}{Ejemplo}[section]

\newmdtheoremenv[
    leftmargin=0em,
    rightmargin=0em,
    innertopmargin=-2pt,
    innerbottommargin=8pt,
    hidealllines = true,
    roundcorner = 5pt,
    backgroundcolor = gray!50!blue!30
]{obs}{Observación}[section]

\newmdtheoremenv[
    leftmargin=0em,
    rightmargin=0em,
    innertopmargin=-2pt,
    innerbottommargin=8pt,
    rightline = false,
    leftline = false
]{theor}{Teorema}[section]

\newmdtheoremenv[
    leftmargin=0em,
    rightmargin=0em,
    innertopmargin=-2pt,
    innerbottommargin=8pt,
    rightline = false,
    leftline = false
]{propo}{Proposición}[section]

\newmdtheoremenv[
    leftmargin=0em,
    rightmargin=0em,
    innertopmargin=-2pt,
    innerbottommargin=8pt,
    rightline = false,
    leftline = false
]{cor}{Corolario}[section]

\newmdtheoremenv[
    leftmargin=0em,
    rightmargin=0em,
    innertopmargin=-2pt,
    innerbottommargin=8pt,
    rightline = false,
    leftline = false
]{lema}{Lema}[section]

\newmdtheoremenv[
    leftmargin=0em,
    rightmargin=0em,
    innertopmargin=-2pt,
    innerbottommargin=8pt,
    roundcorner=5pt,
    backgroundcolor = gray!30,
    hidealllines = true
]{mydef}{Definición}[section]

\newmdtheoremenv[
    leftmargin=0em,
    rightmargin=0em,
    innertopmargin=-2pt,
    innerbottommargin=8pt,
    roundcorner=5pt
]{excer}{Ejercicio}[section]

%En esta parte se colocan comandos que definen la forma en la que se van a escribir ciertas funciones%

\renewcommand{\leq}{\ensuremath{\leqslant}}
\renewcommand{\geq}{\ensuremath{\geqslant}}

\newcommand\abs[1]{\ensuremath{\left|#1\right|}}
\newcommand\divides{\ensuremath{\bigm|}}
\newcommand\cf[3]{\ensuremath{#1:#2\rightarrow#3}}
\newcommand\norm[1]{\ensuremath{\|#1\|}}
\newcommand\ora[1]{\ensuremath{\vec{#1}}}
\newcommand\pint[2]{\ensuremath{\langle#1| #2\rangle}}
\newcommand\conj[1]{\ensuremath{\overline{#1}}}
\newcommand{\N}[2]{\ensuremath{\mathcal{N}_{#1}\left(#2\right)}}
\newcommand{\natint}[1]{\ensuremath{\left[\!\left[#1\right]\!\right]}}
\newcommand{\fou}[1]{\ensuremath{\mathcal{F}#1}}

\begin{document}
    \setlength{\parskip}{5pt} % Añade 5 puntos de espacio entre párrafos
    \setlength{\parindent}{12pt} % Pone la sangría como me gusta
    \title{Notas de Análisis Matemático IV}
    \author{Cristo Daniel Alvarado}
    \maketitle

    \tableofcontents %Con este comando se genera el índice general del libro%

    %\setcounter{chapter}{3} %En esta parte lo que se hace es cambiar la enumeración del capítulo%
    
    \chapter{Transformación de Fourier}
    
    La transformada de Fourier de una función de $\mathbb{R}^n$ en $\mathbb{C}$ generaliza en cierta forma la noción de coeficientes de Fourier de funciones periódicas
    
    \section{Conceptos Fundamentales}

    \begin{mydef}
        Se define el \textbf{producto escalar usual en $\mathbb{R}^n$} como
        \begin{equation*}
            \pint{x}{y}=\sum_{ k=1}^n x_ky_x,\quad\forall x,y\in\mathbb{R}^n
        \end{equation*}
        en ocasiones también denotado como $(x|y)=\pint{x}{y}$.
    \end{mydef}

    \begin{mydef}
        Sea $f\in\mathcal{L}_1(\mathbb{R}^n,\mathbb{C})$. Se definen $\cf{\mathcal{F}f,\mathcal{F}^*f}{\mathbb{R}^n}{\mathbb{C}} $ como
        \begin{equation*}
            \mathcal{F}f(x)=\int_{ \mathbb{R}^n}e^{ -i\pint{x}{y}}f(y)\:dy\quad\textup{y}\quad\mathcal{F}^*f(x)=\int_{ \mathbb{R}^n}e^{i\pint{x}{y}}f(y)\:dy
        \end{equation*}
        para todo $x\in\mathbb{R}^n$. Las funciones $\mathcal{F}f$ y $\mathcal{F}^*f$ se llaman las \textbf{transformaciones de Fourier de $f$}. Las aplicaciones $\mathcal{F}$ y $\mathcal{F}^*$ de $\mathcal{L}_1(\mathbb{R}^n,\mathbb{C})$ en el conjunto de funciones de $\mathbb{R}^n$ en $\mathbb{C}$ se llaman las \textbf{transformaciones de Fourier}.
    \end{mydef}

    \renewcommand{\theenumi}{\roman{enumi}}

    \begin{obs}
        Se tiene lo siguiente:
        \begin{enumerate}
            \item Los operadores $\mathcal{F}$ y $\mathcal{F}^*$ son lineales de $\mathcal{L}_1(\mathbb{R}^n,\mathbb{C})$ en el espacio de funciones de $\mathbb{R}^n$ en $\mathbb{C}$.
            \item Las funciones $\fou{f}(x)$ y $\fou^*{f}(x)$ están definidas para todo $x\in\mathbb{R}^n$ si y sólo si $f\in\mathcal{L}_1(\mathbb{R}^n,\mathbb{C})$.
            \item En caso de existir, se tiene que $\fou{f}(x)=\fou^*{f}(-x)$.
        \end{enumerate}
    \end{obs}

    \begin{proof}
        De (i): Es claro que si $f\in\mathcal{L}_1(\mathbb{R}^n,\mathbb{C})$ entonces $\fou{f}(x)$ y $\fou^*{f}(x)$ están definidas para todo $x\in\mathbb{R}^n$. Para la recíproca, en particular están definidas para $x=\vec{0}$, es decir que
        \begin{equation*}
            \fou{f}\left(\vec{0}\right)=\int_{\mathbb{R}^n}e^{ -i\pint{\vec{0}}{y}}f(y)\:dy=\int_{\mathbb{R}^n}e^{0}f(y)\:dy=\int_{\mathbb{R}^n}f(y)\:dy<\infty
        \end{equation*}
        luego $f\in\mathcal{L}_1(\mathbb{R}^n,\mathbb{C})$.

        De (ii): Es inmediata.
    \end{proof}

    \begin{mydef}
        Sea $f\in\mathcal{L}_1([0,\infty[,\mathbb{C})$. Se definen
        \begin{equation*}
            \mathcal{F}_cf(x)=\int_0^\infty f(y)\cos xy\:dy\quad\textup{y}\quad\mathcal{F}_sf(x)=\int_0^\infty f(y)\sin xy\:dy
        \end{equation*}
        para todo $x\in\mathbb{R}$. Las funciones $\mathcal{F}_cf$ y  $\mathcal{F}_sf$ se llaman \textbf{las trasnformadas coseno y seno de Fourier de $f$}.
    \end{mydef}

    \begin{mydef}
        Sea $\cf{f}{[0,\infty[}{\mathbb{C}}$ una función. Se definen las funciones $f^P$ y $f^I$ de $\mathbb{R}$ en $\mathbb{C}$ como
        \begin{equation*}
            f^P(x)=\left\{ 
                \begin{array}{lcr}
                    f(x) & \textup{ si } & x\geq0\\
                    f(-x) & \textup{ si } & x<0\\
                \end{array}
            \right.
        \end{equation*}
        y,
        \begin{equation*}
            f^I(x)=\left\{ 
                \begin{array}{lcr}
                    f(x) & \textup{ si } & x\geq0\\
                    -f(-x) & \textup{ si } & x<0\\
                \end{array}
            \right.
        \end{equation*}
    \end{mydef}

    \begin{propo}
        Sea $f\in\mathcal{L}_1([0,\infty[,\mathbb{C})$. Se tiene
        \begin{equation*}
            \fou{f^P}(x)=2\mathcal{F}_cf(x)\quad\textup{y}\quad\fou{f^I}(x)=-2i\mathcal{F}_2f(x)
        \end{equation*}
        para todo $x\in\mathbb{R}$.
    \end{propo}

    \begin{proof}
        Sea $x\in\mathbb{R}$, entonces
        \begin{equation*}
            \begin{split}
                \fou{f^P}(x)&=\int_{\mathbb{R}} f^P(y) e^{ -i\pint{x}{y}} \:dy\\
                &=\int_{-\infty}^\infty f^P(y)e^{-ixy}\:dy\\
                &=\int_{-\infty}^0f^P(y)e^{-ixy}\:dy+\int_0^{\infty}f^P(y)e^{-ixy}\:dy \\
                &=\int_{-\infty}^0f(-y)e^{-ixy}\:dy+\int_0^{\infty}f(y)e^{-ixy}\:dy\\
                &=\int_{0}^\infty f(y)e^{ixy}\:dy+\int_0^{\infty}f(y)e^{-ixy}\:dy\\
                &=\int_{0}^\infty f(y)\left[e^{ixy}+\conj{e^{ixy}}\right] \:dy\\
                &=\int_{0}^\infty f(y)\left[2\Re(e^{ ixy}) \right] \:dy\\
                &=\int_{0}^\infty 2f(y)\cos xy \:dy\\
                &=2\int_{0}^\infty f(y)\cos xy \:dy\\
                &=2\mathcal{F}_cf(x)\\
            \end{split}
        \end{equation*}
        y,
        \begin{equation*}
            \begin{split}
                \fou{f^I}(x)&=\int_{\mathbb{R}}f^I(y)e^{ -ixy}\:dy\\
                &=\int_{-\infty}^0f^I(y)e^{ -ixy}\:dy+\int_{0}^\infty f^I(y)e^{ -ixy}\:dy\\
                &=\int_{-\infty}^0(-f(-y))e^{ -ixy}\:dy+\int_{0}^\infty f(y)e^{ -ixy}\:dy\\
                &=-\int_{0}^\infty f(y)e^{ixy}\:dy+\int_{0}^\infty f(y)e^{ -ixy}\:dy\\
                &=\int_{0}^\infty f(y)\left[-e^{ixy}+e^{-ixy}\right] \:dy\\
                &=\int_{0}^\infty f(y)\left[-\cos xy-i\sin xy+\cos(-xy)+i\sin(-xy)\right] \:dy\\
                &=\int_{0}^\infty f(y)\left[-2i\sin xy\right]\:dy\\
                &=-2i\int_{0}^\infty f(y)\sin xy\:dy\\
                &=-2i\mathcal{F}_sf(x)\\
            \end{split}
        \end{equation*}
        lo que prueba el resultado.
    \end{proof}

    \begin{cor}
        Sea $f\in\mathcal{L}_1(\mathbb{R},\mathbb{C})$.
        \begin{enumerate}
            \item Si $f$ es par, entonces $\fou{f}(x)=2\int_0^{\infty}f(y)\cos xy\:dy$ para todo $x\in\mathbb{R}$.
            \item Si $f$ es impar, entonces $\fou{f}(x)=-2i\int_0^{\infty}f(y)\sin xy\:dy$ para todo $x\in\mathbb{R}$.
        \end{enumerate}
    \end{cor}

    \begin{exa}
        Se tiene lo siguiente:
        \begin{enumerate}
            \item Sea $\cf{f}{\mathbb{R}}{\mathbb{R}}$ la función $f=\chi_I$ donde $I$ es un intervalo con extremos $a<b$ en $\mathbb{R}$. Entonces,
            \begin{equation*}
                \begin{split}
                    \fou{f}(x)&=\int_{-\infty}^\infty \chi_I(y)e^{ -ixy}\:dy\\
                    &=\int_{a}^b e^{ -ixy}\:dy\\
                    &=\left\{ 
                        \begin{array}{lcr}
                            \frac{e^{ -ixb}-e^{ -ixa}}{-ix} & \textup{ si } & x\neq0\\
                            b-a & \textup{si} & x=0\\
                        \end{array}
                    \right.\\
                \end{split}
            \end{equation*}
            En particular, si $a>0$ se tiene que
            \begin{equation*}
                \fou{\chi_[-a,a]}(x)=\left\{ 
                    \begin{array}{lcr}
                        \frac{2\sin ax}{x} & \textup{ si } & x\neq0\\
                        2a & \textup{si} & x=0\\
                    \end{array}
                \right.
            \end{equation*}
            Como $\fou{\chi_[-a,a]}$ no es integrable en $\mathbb{R}$ se concluye que, en general, la transformada de Fourier de una función integrable no necesariamente es integrable.
            \item Sea $\cf{f}{\mathbb{R}}{\mathbb{R}}$ la función
            \begin{equation*}
                f(x)=e^{-k\abs{x}},\quad\forall x\in\mathbb{R}
            \end{equation*}
            donde $k>0$. Como $f$ es integrable, entonces
            \begin{equation*}
                \begin{split}
                    \fou{f}(x)&=\int_{-\infty}^\infty e^{-k\abs{y}}e^{-ixy}\:dy\\
                    &=\int_{-\infty}^0 e^{ky}e^{-ixy}\:dy+\int_{0}^\infty e^{-ky}e^{-ixy}\:dy\\
                    &=\int_{0}^\infty e^{-ky}e^{ixy}\:dy+\int_{0}^\infty e^{-ky}e^{-ixy}\:dy\\
                    &=\int_{0}^\infty e^{-ky}e^{(-k+ix)y}\:dy+\int_{0}^\infty e^{-ky}e^{(-k-ix)y}\:dy\\
                    &=\frac{-1}{-k+ix}+\frac{-1}{-k-ix}\\
                    &=\frac{k+ix+k-ix}{k^2+x^2}\\
                    &=\frac{2k}{k^2+x^2}\\
                \end{split}
            \end{equation*}
        \end{enumerate}
    \end{exa}

    \begin{exa}
        Sea $\cf{f}{\mathbb{R}}{\mathbb{R}}$ la función
        \begin{equation*}
            f(x)=e^{ -kx^2},\quad\forall x\in\mathbb{R}
        \end{equation*}
        donde $k>0$. Como $f$ es par se tiene que
        \begin{equation*}
            \fou{f}(x)=2\int_0^{\infty}e^{ -ky^2}\cos xy\:dy
        \end{equation*}
        Sea $g(x)=\int_0^\infty e^{ -ky^2}\cos xy\:dy$ para todo $x\in\mathbb{R}$. Se afirma que
        \begin{equation*}
            g'(x)=-\int_0^{\infty}ye^{ -ky^2}\sin xy\:dy,\quad\forall x\in\mathbb{R}
        \end{equation*}
        En efecto, observemos que
        \begin{equation*}
            \begin{split}
                \abs{ye^{-ky^2}\sin xy}\leq ye^{-ky^2},\quad\forall y\geq0
            \end{split}
        \end{equation*}
        donde la función de la derecha es integrable e independiente de $x$ (se nota fácilmente que una de sus antiderivadas es $y\mapsto-\frac{1}{2k}e^{ -ky^2}$, por el T.F.C. II evaluando en $0$ e $\infty$ se obtiene que la función original es integrable en $[0,\infty[$). Por el Teorema de derivación se sigue que
        \begin{equation*}
            g'(x)=-\int_0^{\infty}ye^{ -ky^2}\sin xy\:dy,\quad\forall x\in\mathbb{R}
        \end{equation*}
        Veamos ahora que
        \begin{equation*}
            \begin{split}
                g'(x)&=\int_0^{\infty}ye^{ -ky^2}\sin xy\:dy\\
                &=-\left[-\frac{1}{2k}e^{ -ky^2}\sin xy\Big|_0^{\infty}+\frac{1}{2k}\int_0^{\infty}xe^{ -ky^2}\cos xy\:dy \right]\\
                &=-\left[0-0+\frac{x}{2k}\int_0^{\infty}e^{ -ky^2}\cos xy\:dy \right]\\
                &=-\frac{x}{2k}\int_0^{\infty}e^{ -ky^2}\cos xy\:dy\\
                &=-\frac{x}{2k}g(x),\quad\forall x\in\mathbb{R} \\
            \end{split}
        \end{equation*}
        Luego, 
        \begin{equation*}
            \begin{split}
                g'(x)+\frac{x}{2k}g(x)&=0,\quad\forall x\in\mathbb{R}\\
                \Rightarrow e^{\frac{x^2}{4k}}\left(g'(x)+\frac{x}{2k}g(x)\right)&=0,\quad\forall x\in\mathbb{R}\\
                \Rightarrow \frac{d}{dx}\left(e^{\frac{x^2}{4k}}g(x) \right)(x_0)&=0,\quad\forall x_0\in\mathbb{R} \\
                \Rightarrow e^{\frac{x^2}{4k}}g(x)&=c,\quad\forall x\in\mathbb{R}\\
            \end{split}
        \end{equation*}
        En particular,
        \begin{equation*}
            \begin{split}
                c&=g(0)\\
                &=\int_0^\infty e^{ -ky^2}\:dy\\
                &=\frac{1}{\sqrt{k}}\int_0^\infty e^{-u^2}\:du\\
                &=\frac{1}{2}\cdot\sqrt{\frac{\pi}{k}}
            \end{split}
        \end{equation*}
        Por ende,
        \begin{equation*}
            g(x)=\frac{1}{2}\sqrt{\frac{\pi}{k}}e^{ -\frac{x^2}{4k}},\quad\forall x\in\mathbb{R}
        \end{equation*}
        De donde se sigue que
        \begin{equation*}
            \fou{f}(x)=\sqrt{\frac{\pi}{k}}e^{ -\frac{x^2}{4k}}, \quad\forall x\in\mathbb{R}
        \end{equation*}
        En particular, si $k=\frac{1}{2}$ entonces $f(x)=e^{-\frac{x^2}{2}}$ para todo $x\in\mathbb{R}$ y,
        \begin{equation*}
            \fou{f}(x)=\sqrt{2\pi}e^{ -\frac{x^2}{2}}=\sqrt{2\pi}f(x)
        \end{equation*}
        es decir que $f$ es un vector propio del operador transformada de Fourier.
    \end{exa}

    \begin{propo}
        Sea $f\in\mathcal{L}_1(\mathbb{R}^n,\mathbb{C})$.
        \begin{enumerate}
            \item Si $g(x)=e^{i\pint{a}{y}}f(x)$ para todo $x\in\mathbb{R}^n$, entonces
            \begin{equation*}
                \fou{g}(x)=\fou{f}(x-a),\quad\forall x\in\mathbb{R}^n
            \end{equation*}
            \item Si $g(x)=f(x-a)$ para todo $x\in\mathbb{R}^n$, entonces
            \begin{equation*}
                \fou{g}(x)=e^{ -i\pint{x}{a}}\fou{f}(x),\quad\forall x\in\mathbb{R}^n
            \end{equation*}
            \item Si $g(x)=\conj{f(-x)}$ para todo $x\in\mathbb{R}^n$, entonces
            \begin{equation*}
                \fou{g}(x)=\conj{\fou{f}(x)},\quad\forall x\in\mathbb{R}^n
            \end{equation*}
            \item Sea $\lambda\in\mathbb{R}\backslash\left\{0\right\}$. Si $g(x)=f(\frac{x}{\lambda})$ para todo $x\in\mathbb{R}^n$, entonces
            \begin{equation*}
                \fou{g}(x)=\abs{\lambda}^n\fou{f}(\lambda x),\quad\forall x\in\mathbb{R}^n
            \end{equation*}
        \end{enumerate}
    \end{propo}

    \begin{proof}
        De (i): Veamos que
        \begin{equation*}
            \begin{split}
                \fou{g}(x)&=\int_{\mathbb{R}^n}e^{ -i\pint{x}{y}}g(y)\:dy\\
                &=\int_{\mathbb{R}^n}e^{ -i\pint{x}{y}}e^{ i\pint{a}{y}}f(y)\:dy\\
                &=\int_{\mathbb{R}^n}e^{ -i\pint{x-a}{y}}f(y)\:dy\\
                &=\fou{f}(x-a)
            \end{split}
        \end{equation*}
        para todo $x\in\mathbb{R}^n$.

        De (ii): Veamos que
        \begin{equation*}
            \begin{split}
                \fou{g}(x)&=\int_{\mathbb{R}^n}e^{ -i\pint{x}{y}}g(y)\:dy\\
                &=\int_{\mathbb{R}^n}e^{ -i\pint{x}{y}}f(y-a)\:dy\\
                &=\int_{\mathbb{R}^n}e^{ -i\pint{x}{u+a}}f(u)\:du\\
                &=e^{ -i\pint{x}{a}}\fou{f}(x) \\
            \end{split}
        \end{equation*}
        para todo $x\in\mathbb{R}^n$.

        De (iii): Veamos que
        \begin{equation*}
            \begin{split}
                \fou{g}(x)&=\int_{\mathbb{R}^n}e^{-i\pint{x}{y}}\conj{f(-y)}\:dy\\
                &=\int_{\mathbb{R}^n}e^{i\pint{x}{y}}\conj{f(y)}\:dy\\
                &=\conj{\int_{\mathbb{R}^n}e^{-i\pint{x}{y}}f(y)\:dy}\\
                &=\conj{\fou{f}(x)}\\
            \end{split}
        \end{equation*}
        para todo $x\in\mathbb{R}^n$.
        
        De (iv): Veamos que
        \begin{equation*}
            \begin{split}
                \fou{g}(x)&=\int_{\mathbb{R}^n}e^{-i\pint{x}{y}}g(y)\:dy\\
                &=\int_{\mathbb{R}^n}e^{-i\pint{x}{y}}f\left(\frac{y}{\lambda}\right) \:dy,\textup{ haciendo el cambio de variable }u=\frac{y}{\lambda} \\
                &=\int_{\mathbb{R}^n}e^{-i\pint{x}{\lambda u}}f(u) \:\abs{\lambda}^n du\\
                &=\abs{\lambda}^n\int_{\mathbb{R}^n}e^{-i\pint{\lambda x}{ u}}f(u) \:du\\
                &=\abs{\lambda}^n\fou{f}(\lambda x)\\
            \end{split}
        \end{equation*}
        para todo $x\in\mathbb{R}^n$.
    \end{proof}

    \begin{theor}
        Si $f\in\mathcal{L}_1(\mathbb{R}^n,\mathbb{C})$, entonces,
        \begin{equation*}
            \abs{\fou{f}(x)}\leq\N{1}{f},\quad\forall x\in\mathbb{R}^n
        \end{equation*}
        Así pues, $\cf{\fou{f}}{\mathbb{R}^n}{\mathbb{C}}$ es una función acotada. Si $\mathcal{B}(\mathbb{R}^n,\mathbb{C})$ denota al espacio de funciones acotadas de $\mathbb{R}^n$ en $\mathbb{C}$ provisto de la norma uniforme, entonces $\fou{\cdot}$ es una aplicación lineal continua de $L_1(\mathbb{R}^n,\mathbb{C})$ en $\mathcal{B}(\mathbb{R}^n,\mathbb{C})$ tal que $\norm{\fou{\cdot}}=1$.
    \end{theor}

    \begin{proof}
        Para todo $f\in\mathcal{L}_1(\mathbb{R}^n,\mathbb{C})$, se tiene que
        \begin{equation*}
            \begin{split}
                \abs{\fou{f}(x)}&=\abs{\int_{\mathbb{R}^n}e^{ -i\pint{x}{y}}f(y)\:dy}\\
                &\leq\int_{\mathbb{R}^n}\abs{f(y)}\:dy\\
                &=\N{1}{f},\quad\forall x\in\mathbb{R}^n \\
            \end{split}
        \end{equation*}
        Notemos también que $\norm{\fou{\cdot}}\leq1$.

        Para probar la otra desiguladad se busca una función $P\in\mathcal{L}_1(\mathbb{R}^n,\mathbb{C})$ tal que $\N{\infty}{\fou{P}}=\N{1}{P}>0$. Por ejemplo, la función $\cf{P}{\mathbb{R}^n}{\mathbb{C}}$ dada por:
        \begin{equation*}
            P(x)=e^{ -\sum_{ k=1}^n\abs{x_k}},\quad\forall x\in\mathbb{R}^n
        \end{equation*}
        satisface
        \begin{equation*}
            \begin{split}
                \fou{P}(x)&=\int_{\mathbb{R}^n}e^{ -i\pint{x}{y}}P(y)\:dy\\
                &=\int_{\mathbb{R}^n}e^{ -i\pint{x}{y}}P(y)\:dy\\
                &=\int_{\mathbb{R}^n}e^{ -\abs{y_1}-ix_1y_1}\cdots e^{ -\abs{y_n}-ix_ny_n}\:dy_1\cdots dy_n\\
                &=\left(\int_{-\infty}^{\infty}e^{ -\abs{y_1}-ix_1y_1}\:dy_1\right)\cdots\left(\int_{-\infty}^{\infty}e^{ -\abs{y_n}-ix_ny_n}\:dy_n\right)\\
            \end{split}
        \end{equation*}
        Se sabe por ejemplos anteriores que la transformada de $t\mapsto e^{-\abs{t}}$ es $\frac{2}{1+t^2}$, para todo $t\in\mathbb{R}$, así pues,
        \begin{equation*}
            \fou{P}(x)=\frac{2^n}{(1+x_2^2)\cdots(1+x_n^2)},\quad\forall x\in\mathbb{R}^n
        \end{equation*}
        de donde,
        \begin{equation*}
            \N{\infty}{\fou{P}}=2^n
        \end{equation*}
        Por otra parte,
        \begin{equation*}
            \begin{split}
                \N{1}{P}&=\int_{\mathbb{R}^n}e^{ -\sum_{ k=1}^n\abs{x_k}}\:dx_1\cdots dx_n\\
                &=\left[\int_{-\infty}^\infty e^{-\abs{t}}\:dt \right]^n\\
                &=2^n\left[\int_0^{\infty}e^{-\abs{t}}\:dt \right]\\
                &=2^n\\
            \end{split}
        \end{equation*}
        Por tanto,
        \begin{equation*}
            \begin{split}
                \N{1}{P}&=\N{\infty}{\fou{P}}\\
                &\leq\norm{\fou{\cdot}}\N{1}{P}\\
                \Rightarrow 1&\leq\norm{\fou{\cdot}}\\
            \end{split}
        \end{equation*}
        por tanto, de lo anterior se deduce que $\norm{\fou{\cdot}}=1$. 
    \end{proof}

    \begin{propo}
        Si $f\in\mathcal{L}_1(\mathbb{R}^n,\mathbb{C})$, entonces $\fou{f}$ es uniformemente continua en $\mathbb{R}^n$.
    \end{propo}

    \begin{proof}
        Basta probar que si $\left\{x_\nu\right\}_{ \nu=1}^\infty$ y $\left\{y_\nu \right\}_{ \nu=1}^\infty$ son dos sucesiones en $\mathbb{R}^n$ tales que $\lim_{ \nu\rightarrow\infty}\norm{x_\nu-y_\nu}=0$, entonces
        \begin{equation*}
            \lim_{\nu\rightarrow\infty}\abs{\fou{f}(x_\nu)-\fou{f}(y_\nu)}=0
        \end{equation*}

        Considere entonces dos sucesiones que cumplan lo anterior. Se tiene
        \begin{equation*}
            \begin{split}
                \abs{\fou{f}(x_\nu)-\fou{f}(y_\nu)}&=\abs{\int_{\mathbb{R}^n}\left(e^{ -i\pint{x_\nu}{z}}-e^{ -i\pint{y_\nu}{z}} \right)f(z)\:dz}\\
                &=\abs{\int_{\mathbb{R}^n}e^{ -i\pint{x_\nu}{z}}\left(1-e^{ -i\pint{y_\nu-x_\nu}{z}} \right)f(z)\:dz}\\
                &\leq\int_{\mathbb{R}^n}\abs{\left(1-e^{ -i\pint{y_\nu-x_\nu}{z}} \right)}\abs{f(z)}\:dz\\
            \end{split}
        \end{equation*}
        donde
        \begin{equation*}
            \lim_{\nu\rightarrow\infty}\abs{1-e^{-i\pint{y_\nu-x\nu}{z}}}\abs{f(z)}=0,\quad\forall z\in\mathbb{R}^n
        \end{equation*}
        y, además
        \begin{equation*}
            \abs{1-e^{-i\pint{y_\nu-x\nu}{z}}}\abs{f(z)}\leq2\abs{f(z)},\quad\forall z\in\mathbb{R}^n
        \end{equation*}
        donde la función de la derecha es integrable e independiente de $\nu$. Por Lebesgue se sigue que
        \begin{equation*}
            \lim_{\nu\rightarrow\infty}\abs{\fou{f}(x_\nu)-\fou{f}(y_\nu)}=0
        \end{equation*}
        así, $\cf{\fou{f}}{\mathbb{R}^n}{\mathbb{C}}$ es una función uniformemente continua.
    \end{proof}

    \begin{obs}
        $\fou{f}$ es una función uniformemente continua y acotada en $\mathbb{R}^n$ si $f\in\mathcal{L}_1(\mathbb{R}^n,\mathbb{C})$.
    \end{obs}

    \begin{theor}[\textbf{Teorema de Riemman-Lebesgue}]
        Si $f\in\mathcal{L}_1(\mathbb{R}^n,\mathbb{C})$, entonces
        \begin{equation*}
            \lim_{\abs{x}\rightarrow\infty}\fou{f}(x)=0
        \end{equation*}
    \end{theor}

    \begin{proof}
        Se probará por casos:
        \begin{enumerate}
            \item Sea $P=I_1\times\cdot\times I_n \subseteq\mathbb{R}^n$ un rectángulo acotado en $\mathbb{R}^n$ donde $I_k$ es un intervalo de extremos $a_k\leq b_k$ para todo $k\in\natint{1,n}$. Se considera el caso en que $f=\chi_P$. En particular, notemos que
            \begin{equation*}
                \begin{split}
                    f(x)=\chi_{ I_1}(x_1)\cdots\chi_{ I_n}(x_n),\quad\forall x\in\mathbb{R}^n
                \end{split}
            \end{equation*}
            Entonces,
            \begin{equation*}
                \begin{split}
                    \fou{f}(x)=\int_{\mathbb{R}^n}e^{ -i\pint{x}{z}}f(z)\:dz\\
                    &=\int_{\mathbb{R}^n}e^{ -ix_1z_1}\chi_{I_1}(z_1)\cdots e^{ -ix_nz_n}\chi_{I_n}(z_n)\:dz_1\cdots dz_n\\
                    &=\left(\int_{-\infty}^\infty e^{ -ix_1z_1}\chi_{ I_1}(z_1)\:dz_1 \right)\cdots\left(\int_{-\infty}^\infty e^{ -ix_nz_n}\chi_{ I_n}(z_n)\:dz_n\right)\\
                \end{split}
            \end{equation*}
            luego,
            \begin{equation*}
                \fou{f}(x)=\varphi_1(x_1)\cdots\varphi_n(x_n),\quad\forall x\in\mathbb{R}^n
            \end{equation*}
            donde
            \begin{equation*}
                \varphi_k(x_k)=\left\{
                    \begin{array}{lcr}
                        \frac{e^{ -ix_kb_k}-e^{ -ix_ka_k}}{-ik}, & \textup{ si } & x_k\neq0\\
                        b_k-a_k & \textup{ si } & x_k=0\\
                    \end{array}
                \right.
            \end{equation*}
            para $k\in\natint{1,n}$. Es claro que $\lim_{ x_k\rightarrow\infty}\varphi_k(x_k)=0$ para todo $k\in\natint{1,n}$. Por otra parte,
            \begin{equation*}
                \begin{split}
                    \abs{\varphi_k(x_k)}&\leq\abs{\fou{\chi_{I_k}}(x_k)}\\
                    &\leq\N{1}{\chi_{I_k}}\\
                    &=b_k-a_k\\
                \end{split}
            \end{equation*}
            para $k\in\natint{1,n}$. Sea
            \begin{equation*}
                c=\max_{ 1\leq k\leq n}\left\{b_k-a_k \right\}
            \end{equation*}
            Dado $\varepsilon>0$ existe $R>0$ tal que para todo $k\in\natint{1,n}$ es tiene
            \begin{equation*}
                \abs{x_k}>R\Rightarrow\abs{\varphi_k(x_k)}<\varepsilon
            \end{equation*}
            Si se toma la norma cúbica $\norm{\cdot}$ de $\mathbb{R}^n$, al suponer que $\norm{x}>R$ se tendrá que $\abs{x_k}>R$ para algún $k\in\natint{1,n}$, luego
            \begin{equation*}
                \norm{x}>R\Rightarrow\abs{\fou{\chi_P}(x)}=\abs{\varphi_1(x_1)\cdots\varphi_n(x_n)}\leq c^{ n-1}\varepsilon
            \end{equation*}
            Así pues, el Teorema es cierto para $f=\chi_P$. Claramente por linealidad de la transformación de Fourier el Teorema sigue siendo cierto si $f$ es una función escalonada en $\mathbb{R}^n$.

            \item Sea $f\in\mathcal{L}_1(\mathbb{R}^n,\mathbb{C})$ y tomemos $\varepsilon>0$. Por la densidad de $\mathcal{E}(\mathbb{R}^n,\mathbb{C})$ en $\mathcal{L}_1(\mathbb{R}^n,\mathbb{C})$, existe $\varphi\in\mathcal{E}(\mathbb{R}^n,\mathbb{C})$ tal que
            \begin{equation*}
                \N{1}{f-\varphi}<\frac{\varepsilon}{2}
            \end{equation*}
            Entonces,
            \begin{equation*}
                \begin{split}
                    \abs{\fou{f}(x)}&=\abs{\fou{f}(x)-\fou{\varphi}(x)}+\abs{\fou{\varphi}(x)}\\
                    &=\abs{\fou{(f-\varphi)}(x)}+\abs{\fou{\varphi}(x)}\\
                    &\leq\N{1}{f-\varphi}+\abs{\fou{\varphi}(x)},\quad\forall x\in\mathbb{R}^n \\
                    &<\frac{\varepsilon}{2}+\abs{\fou{\varphi}(x)},\quad\forall x\in\mathbb{R}^n \\
                \end{split}
            \end{equation*}
            Por tanto, de (i) existe $R>0$ tal que
            \begin{equation*}
                \norm{x}>R\Rightarrow\abs{\fou{\varphi}(x)}<\frac{\varepsilon}{2}
            \end{equation*}
            de donde se sigue que
            \begin{equation*}
                \norm{x}>R\Rightarrow\abs{\fou{f}(x)}<\frac{\varepsilon}{2}+\abs{\fou{\varphi}(x)}<\varepsilon
            \end{equation*}
            lo que prueba el resultado.
        \end{enumerate}
    \end{proof}

    \begin{theor}
        Si $f,g\in\mathcal{L}_1(\mathbb{R}^n,\mathbb{C})$, entonces $\fou{(f*g)}=(\fou{f})(\fou{g})$.
    \end{theor}

    \begin{proof}
        Sean $f,g\in\mathcal{L}_1(\mathbb{R}^n,\mathbb{C})$. Se tiene que
        \begin{equation*}
            \begin{split}
                \fou{(f*g)}(x)&=\int_{\mathbb{R}^n}e^{ -i\pint{x}{y}}f*g(y)\:dy\\
                &=\int_{\mathbb{R}^n}e^{ -i\pint{x}{y}}\:dy\int_{\mathbb{R}^n}f(z)g(y-z)\:dz\\
            \end{split}
        \end{equation*}
        ya se sabe que $(y,z)\mapsto f(z)g(y-z)e^{ -i\pint{x}{y}}$ es integrable en $\mathbb{R}^n\times\mathbb{R}^n$ para todo $x\in\mathbb{R}^n$. Por Fubini:
        \begin{equation*}
            \begin{split}
                \fou{(f*g)}(x)&=\int_{\mathbb{R}^n}f(z)\:dz\int_{\mathbb{R}^n}e^{ -i\pint{x}{y}}g(y-z)\:dy,\textup{ haciendo el cambio de variable }y=u+z\\
                &=\int_{\mathbb{R}^n}f(z)\:dz\int_{\mathbb{R}^n}e^{-i\pint{x}{u+z}}g(u)\:du\\
                &=\left(\int_{\mathbb{R}^n}e^{ -i\pint{x}{z}}f(z)\:dz \right)\left(\int_{\mathbb{R}^n}e^{ -i\pint{x}{y}}g(y)\:dy \right)\\
                &=(\fou{f}(x))(\fou{g}(x))\\
            \end{split}
        \end{equation*}
        lo que prueba el resultado.
    \end{proof}

    \begin{theor}
        Sea $\mathcal{C}_0(\mathbb{R}^n,\mathbb{C})$ el álgebra de Banach de las funciones de $\mathbb{R}^n$ en $\mathbb{C}$ continuas y nulas en el infinito provisto de la norma uniforme. Entonces la aplicación $\cf{\fou{\cdot}}{L_1(\mathbb{R}^n,\mathbb{C})}{\mathcal{C}_0(\mathbb{R}^n,\mathbb{C})}$ es un homomorfismo continuo entre ambas álgebras de Banach.

        La norma de $\fou{\cdot}$ considerada como aplicación lineal es $\norm{\fou{\cdot}}=1$.
    \end{theor}

    \begin{proof}
        Es un resumen de las propiedades anteriores.
    \end{proof}

    \begin{obs}
        Más adelante se verá que $\fou{\cdot}$ es inyectiva pero no es suprayectiva.
    \end{obs}

    \begin{propo}
        Sea $\cf{f}{\mathbb{R}^n}{\mathbb{C}}$ y $r\in\mathbb{N}$. Se supone que $x\mapsto x_1^{m_1}\cdots x_m^{m_n}f(x)$ e sintegrable en $R^n$ para toda colección $m_1,...,m_n\in\mathbb{N}$ tales que $m_1+\cdots+m_n\leq r$. Entonces, $\fou{f}$ es de clase $C^r$ en $\mathbb{R}^n$. Si $k\in\natint{1,k}$ y $\alpha_1,...,\alpha_k\in\natint{1,n}$ se tiene que
        \begin{equation*}
            \partial_{\alpha_1}\cdots\partial_{\alpha_k}\fou{f}=\fou{g}
        \end{equation*}
        donde $g(x)=(-ix_{\alpha_1})(-ix_{\alpha_2})\cdots(-ix_{\alpha_k})f(x)$.
    \end{propo}

    \begin{proof}
        Se tiene
        \begin{equation*}
            \fou{f}(x)=\int_{\mathbb{R}^n}e^{ -i(x_1y_1+\cdots+x_ny_n)}f(y)\:dy
        \end{equation*}
        Al aplicar el operador $\partial_{\alpha_1}\cdots\partial_{\alpha_k}$ a $x\mapsto e^{ -i(x_1y_1+\cdots+x_ny_n)}f(y)$ obtenemos
        \begin{equation*}
            (-iy_{\alpha_i})\cdots(-iy_{\alpha_k})f(y)
        \end{equation*}
        Esta función en valor absoulto es menor o igual a
        \begin{equation*}
            \abs{y_{\alpha_1}\cdots y_{\alpha_k}f(y)}
        \end{equation*}
        la cual por hipótesis es integrable en $\mathbb{R}^n$ e independiente de $x$. Por el Teorema de derivación parcial de funciones definidas por integrales, se tiene que
        \begin{equation*}
            \partial_{\alpha_1}\cdots\partial_{\alpha_k}\fou{f}=\fou{g}
        \end{equation*}
        y, además $\fou{f}$ es de clase $C^r$ en $\mathbb{R}^n$.
    \end{proof}

    \begin{obs}
        Si $f\in\mathcal{L}_1(\mathbb{R}^n,\mathbb{K})$ y existe $\lim_{x\rightarrow\infty}f(x)$, necesariamente
        \begin{equation*}
            \lim_{x\rightarrow\infty}f(x)=0
        \end{equation*}
    \end{obs}

    \begin{propo}
        Sea $\cf{f}{\mathbb{R}}{\mathbb{C}}$ función de clase $C^r$ en $\mathbb{R}^n$. Se supone que $f$ y todas sus derivadas parciales hasta el orden $r$ (inclusive) son integrables. Si $k\in\natint{1,r}$ y $\alpha_1,...,\alpha_k\in\natint{1,n}$, entonces
        \begin{equation*}
            \fou{(\partial_{\alpha_1}\cdots\partial_{\alpha_k}f)}(x)=(ix_{\alpha_1})\cdots(ix_{\alpha_k})\fou{f}(x)
        \end{equation*}
    \end{propo}

    \begin{proof}
        Basta probar que
        \begin{equation*}
            \fou{(\partial_jf)}(x)=(ix_j)\fou{f}(x)
        \end{equation*}
        con $j\in\natint{1,n}$ pues el resto se sigue por inducción. Se tiene que
        \begin{equation*}
            \begin{split}
                \fou{(\partial_jf)}(x)-(ix_j)\fou{f}(x)&=\int_{\mathbb{R}^n}e^{ -i\pint{x}{y}}[\partial_jf(y)-ix_jf(y)]\:dy\\
                &=\int_{\mathbb{R}^n}\frac{\partial}{\partial y_j} \left[e^{ -i\pint{x}{y}}f(y) \right]\:dy\\
            \end{split}
        \end{equation*}
        de donde, por el Teorema de Fubini
        \begin{equation*}
            \begin{split}
                \fou{(\partial_jf)}(x)-(ix_j)\fou{f}(x)&=\int_{\mathbb{R}^{ n-1}}\:dy_1\cdots dy_{ j-1}dy_{ j+1}\cdots dy_n\int_{-\infty}^\infty\frac{\partial}{\partial y_j}\left[e^{-i\pint{x}{y}}f(y)\right] \:dy_j\\
            \end{split}
        \end{equation*}
        El Teorema de Fubini asegura que existe un conjunto despreciable $Z_1\subseteq\mathbb{R}^{ n-1}$ tal que para todo $y'=(y_1,...,y_{ j-1},y_{ j+1},...,y_n)\in\mathbb{R}^{ n-1}\backslash Z_1$ existe la integral
        \begin{equation*}
            \int_{-\infty}^\infty\frac{\partial}{\partial y_j}\left[e^{-i\pint{x}{y}}f(y)\right] \:dy_j
        \end{equation*}
        o sea que $y_j\mapsto\frac{\partial}{\partial y_j}\left[e^{-i\pint{x}{y}}f(y)\right]$ es integrable en $\mathbb{R}$. Por el 2° T.F.C. para intervalos abiertos
        \begin{equation*}
            \begin{split}
                \int_{-\infty}^\infty\frac{\partial}{\partial y_j}\left[e^{-i\pint{x}{y}}f(y)\right] \:dy_j&=\lim_{ y_j\rightarrow\infty}e^{ -i\pint{x}{y}}f(y)-\lim_{ y_j\rightarrow-\infty}e^{ -i\pint{x}{y}}f(y)\\
            \end{split}
        \end{equation*}
        puesto que la función $y\mapsto e^{ -i\pint{x}{y}}f(y)$ es integrable en $\mathbb{R}^n$, por el Teorema de Fubini existe un conjunto $Z_2\subseteq\mathbb{R}^{ n-1}$ tal que para todo $y'=(y_1,...,y_{ j-1},y_{ j+1},...,y_n)\in\mathbb{R}^{ n-1}\backslash Z_2$, la función $y_j\mapsto e^{ -i\pint{x}{y}}f(y)$ es integable en $\mathbb{R}$. Sea $Z=Z_1\cup Z_2\subseteq\mathbb{R}^{ n-1}$. Por la última observación, los límites a la derecha de la ecuación anterior deben ser $0$ para todo $y'=(y_1,...,y_{ j-1},y_{ j+1},...,y_n)\in\mathbb{R}^{n-1}\backslash Z$. Por tanto,
        \begin{equation*}
            \int_{-\infty}^\infty\frac{\partial}{\partial y_j}\left[e^{-i\pint{x}{y}}f(y)\right] \:dy_j =0
        \end{equation*}
        para todo $y'=(y_1,...,y_{ j-1},y_{ j+1},...,y_n)\in\mathbb{R}^{n-1}\backslash Z$. Se sigue entonces que
        \begin{equation*}
            \fou{(\partial_jf)}(x)-(ix_j)\fou{f}(x)=0
        \end{equation*}
        lo que prueba el resultado.
    \end{proof}

    \section{Teoremas de Transferencia e Inversión}

    \begin{theor}[\textbf{Teorema de Transferencia}]
        Sean $f,g\in\mathcal{L}_1(\mathbb{R}^n,\mathbb{K})$. Entonces
        \begin{equation*}
            \int_{\mathbb{R}^n}f\cdot\fou{g}=\int_{\mathbb{R}^n}\fou{f}\cdot g
        \end{equation*}
        y,
        \begin{equation*}
            \int_{\mathbb{R}^n}f\cdot\fou^*{g}=\int_{\mathbb{R}^n}\fou^*{f}\cdot g
        \end{equation*}
    \end{theor}

    \begin{proof}
        Como $\fou{f}$ y $\fou{g}$ son continuas acotadas en $\mathbb{R}^n$ y $f,g\in\mathcal{L}_1(\mathbb{R}^n,\mathbb{K})$, ambas integrales existen (pues en particular $\fou{f},\fou{g}\in\mathcal{L}_{\infty}(\mathbb{R}^n,\mathbb{K})$). Se tiene
        \begin{equation*}
            \begin{split}
                \int_{\mathbb{R}^n}\fou{f}(x)\cdot g(x)\:dx&=\int_{\mathbb{R}^n}g(x)\:dx\int_{\mathbb{R}^n}e^{ -i\pint{x}{y}}f(y)\:dy\\
            \end{split}
        \end{equation*}
        ya se sabe que $(x,y)\mapsto e^{ -i\pint{x}{y}}g(x)f(y)$ es integrable en $\mathbb{R}^n\times\mathbb{R}^n$ (pues en módulo es igual al módulo del producto tensorial de $g$ y $f$, siendo éste integrable). Por Fubini podemos invertir el orden de integración, lo que resulta:
        \begin{equation*}
            \begin{split}
                \int_{\mathbb{R}^n}\fou{f}(x)\cdot g(x)\:dx&=\int_{\mathbb{R}^n}f(y)\:dy\int_{\mathbb{R}^n}e^{ -i\pint{x}{y}}g(x)\:dx\\
                &=\int_{\mathbb{R}^n}f(y)\:dy\int_{\mathbb{R}^n}e^{ -i\pint{y}{x}}g(x)\:dx\\
                &=\int_{\mathbb{R}^n}f(y)\cdot\fou{g}(y)\:dy\\
                \Rightarrow \int_{\mathbb{R}^n}f\cdot\fou{g}&=\int_{\mathbb{R}^n}\fou{f}\cdot g\\
            \end{split}
        \end{equation*}
        para la $\fou^*{\cdot}$ el procedimiento es análogo.
    \end{proof}

    \begin{lema}[\textbf{Efecto de la transformación de Fourier sobre sucesiones de Dirac}]
        Sea $\left\{\rho_\nu \right\}_{\nu=1}^\infty$ una sucesión de Dirac en $\mathcal{L}_1(\mathbb{R}^n,\mathbb{C})$. Defina
        \begin{equation*}
            h_\nu=\fou{\rho_\nu},\quad\forall\nu\in\mathbb{N}
        \end{equation*}
        Entonces,
        \begin{enumerate}
            \item $\abs{h_\nu(x)}\leq1$ para todo $x\in\mathbb{R}^n$.
            \item $\lim_{\nu\rightarrow\infty}h_\nu(x)=1$, para todo $x\in\mathbb{R}^n$.
        \end{enumerate}
    \end{lema}

    \begin{proof}
        De (i): Se tiene
        \begin{equation*}
            \begin{split}
                \abs{h_\nu(x)}&=\abs{\fou{\rho_\nu}(x)}\\
                &\leq\N{1}{\rho_\nu}\\
                &=\int_{\mathbb{R}^n}\rho_\nu\\
                &=1\\
            \end{split}
        \end{equation*}
        para todo $x\in\mathbb{R}^n$.

        De (ii): Sea $f\in\mathcal{L}_1(\mathbb{R}^n,\mathbb{K})$. Entonces $\left\{f*\rho_\nu \right\}_{ \nu=1}^\infty$ es una sucesión en $\mathcal{L}_1(\mathbb{R}^n,\mathbb{K})$ que converge en promedio a $f$. Como la transformación de Fourier es un homomorfismo continuo del álgebra de Banach $L_1(\mathbb{R}^n,\mathbb{K})$ en $\mathcal{C}_0(\mathbb{R}^n,\mathbb{K})$, se debe tener que
        \begin{equation*}
            \lim_{\nu\rightarrow\infty}\fou{(f*\rho_\nu)}=\fou{f}\textup{ uniformemente en }\mathbb{R}^n
        \end{equation*}
        pero,
        \begin{equation*}
            \fou{(f*\rho_\nu)}=\fou{f}\cdot\fou{\rho_\nu}=h_\nu\fou{f}
        \end{equation*}
        es decir,
        \begin{equation*}
            \begin{split}
                \lim_{\nu\rightarrow\infty}h_\nu\fou{f}&=\fou{f}\textup{ uniformemente en }\mathbb{R}^n\\
                \Rightarrow \fou{f}\lim_{\nu\rightarrow\infty}h_\nu&=\fou{f}\textup{ uniformemente en }\mathbb{R}^n\\
            \end{split}
        \end{equation*}
        Fijando $f$ de tal suerte que $\fou{f}(x)\neq0$ para todo $x\in\mathbb{R}^n$ se concluye que
        \begin{equation*}
            \lim_{\nu\rightarrow\infty}h_\nu(x)=1
        \end{equation*}
        (por ejemplo, tome $f(x)=e^{-\frac{x^2}{2}}$).
    \end{proof}

    \begin{obs}
        Si $f\in\mathcal{L}_1(\mathbb{R}^n,\mathbb{C})$, entonces no necesariamente su transformada de Fourier es integrable. Por ejemplo
        \begin{equation*}
             \fou{\chi_{[-1,1]}}=\left\{
                \begin{array}{lcr}
                    \frac{2\sin x}{x} & \textup{ si } & x\neq0\\
                    2 & \textup{ si } & x=0\\
                \end{array}
             \right.
        \end{equation*}
        es continua, nula en el infinito pero no es integrable en $\mathbb{R}$.
    \end{obs}

    \renewcommand{\theenumi}{\arabic{enumi}}

    \begin{theor}[\textbf{Teorema de Inversión de Fourier}]
        Si $f\in\mathcal{L}_1(\mathbb{R}^n,\mathbb{C})$ es tal que $\fou{f}\in\mathcal{L}_1(\mathbb{R}^n,\mathbb{C})$, entonces
        \begin{equation*}
            \fou^*{(\fou{f})}=\fou{(\fou^*{f})}=(2\pi)^n f\textup{ c.t.p. en }\mathbb{R}^n
        \end{equation*}
        Si además $f$ es continua en $\mathbb{R}^n$, la fórmula es válida en todo punto de $\mathbb{R}^n$.
    \end{theor}

    \begin{proof}
        Se probará por casos:
        \begin{enumerate}
            \item Suponga por el momento hallada una sucesión de Dirac $\left\{\rho_\nu\right\}_{\nu=1}^\infty$ en $\mathcal{L}_1(\mathbb{R}^n,\mathbb{C})$ que cumpla las condiciones:
            \begin{enumerate}
                \item $\fou{\rho_\nu}\in\mathcal{L}_1(\mathbb{R}^n,\mathbb{C})$, luego también $\fou^*{\rho_\nu}\in\mathcal{L}_1(\mathbb{R}^n,\mathbb{C})$.
                \item $\fou^*{\fou{\rho_\nu}}=\fou{(\fou^*{\rho_\nu})}=(2\pi)^n\rho_\nu$ c.t.p. en $\mathbb{R}^n$.
            \end{enumerate}
            Sea $h_\nu=\fou{\rho_\nu}$ para todo $\nu\in\mathbb{N}$. Por (ii), $\rho_\nu$ es una función acotada, luego $f*\rho_\nu$ existe en todo punto de $\mathbb{R}^n$. Se tiene
            \begin{equation*}
                \begin{split}
                    f*\rho_\nu(x)&=\int_{\mathbb{R}^n}\rho_\nu(y)f(x-y)\:dy\\
                    &=\frac{1}{(2\pi)^n}\int_{\mathbb{R}^n}\fou^*{h_\nu}(y)f(x-y)\:dy\\
                \end{split}
            \end{equation*}
            será necesario aplicar $\fou^*{\cdot}$ a la función de $y$ tal que $y\mapsto f(x-y)$. Sea $s(y)=f(-y)$, para todo $y\in\mathbb{R}^n$. Entonces,
            \begin{equation*}
                \begin{split}
                    \fou{s}(y)&=\int_{\mathbb{R}^n}e^{ -i\pint{u}{y}}s(u)\:du\\
                    &=\int_{\mathbb{R}^n}e^{i\pint{u}{y}}f(u)\:du\\
                    &=\int_{\mathbb{R}^n}e^{-i\pint{u}{-y}}f(u)\:du\\
                    &=\fou{f}(-y)\\
                \end{split}
            \end{equation*}
            sea ahora $r(y)=f(x-y)=f(-(y-x))=s(y-x)$. Por (ii) de las propiedades de la transformación de Fourier:
            \begin{equation*}
                \fou{r}(y)=e^{ -i\pint{x}{y}}\fou{s}(y)=e^{ -i\pint{x}{y}}\fou{f}(y),\quad\forall y\in\mathbb{R}^n
            \end{equation*}
            Así pues,
            \begin{equation*}
                \fou^*{r}(y)=\fou{r}(-y)=e^{-i\pint{x}{-y}}\fou{f}(y)=e^{ i\pint{x}{y}}\fou{f}(y),\quad\forall y\in\mathbb{R}^n
            \end{equation*}
            Por el Teorema de transferencia (aplicado a $\fou^*{}$) se sigue que:
            \begin{equation*}
                \begin{split}
                    f*\rho_\nu(x)&=\frac{1}{(2\pi)^n}\int_{\mathbb{R}^n}\fou^*{h_\nu}(y)f(x-y)\:dy\\
                    &=\frac{1}{(2\pi)^n}\int_{\mathbb{R}^n}\fou^*{h_\nu}(y)r(y)\:dy\\
                    &=\frac{1}{(2\pi)^n}\int_{\mathbb{R}^n}h_\nu\fou^*{r}(y)\:dy\\
                    &=\frac{1}{(2\pi)^n}\int_{\mathbb{R}^n}h_\nu e^{i\pint{x}{y}}\fou{f}(y)\:dy\\
                \end{split}
            \end{equation*}
            Todo lo anterior es válido bajo la sola hipótesis de que $f\in\mathcal{L}_1(\mathbb{R}^n,\mathbb{C})$. Suponga también que $\fou{f}\in\mathcal{L}_1(\mathbb{R}^n,\mathbb{C})$. Entonces,
            \begin{equation*}
                \lim_{ \nu\rightarrow\infty}h_\nu(y)e^{ i\pint{x}{y}}\fou{f}(y)=e^{ i\pint{x}{y}}\fou{f}(y)
            \end{equation*}
            y
            \begin{equation*}
                \abs{h_\nu(y)e^{ i\pint{x}{y}}\fou{f}(y)}\leq\fou{f}(y),\quad\forall y\in\mathbb{R}^n
            \end{equation*}
            donde la función de la derecha es integrable e independiente de $\nu\in\mathbb{N}$. Por Lebesgue se sigue pues que
            \begin{equation*}
                \lim_{ \nu\rightarrow\infty}\frac{1}{(2\pi)^n}\int_{\mathbb{R}^n}h_\nu(y)e^{ i\pint{x}{y}}\fou{f}(y)\:dy=\frac{1}{(2\pi)^n}\int_{\mathbb{R}^n}e^{ i\pint{x}{y}}\fou{f}(y)\:dy,\quad\forall x\in\mathbb{R}^n
            \end{equation*}
            Lo que realmente estamos diciendo es que
            \begin{equation*}
                \lim_{ \nu\rightarrow\infty}f*\rho_\nu(x)=\frac{1}{(2\pi)^n}\int_{\mathbb{R}^n}e^{ i\pint{x}{y}}\fou{f}(y)\:dy,\quad\forall x\in\mathbb{R}^n
            \end{equation*}
            puntualmente en $\mathbb{R}^n$. Pero $\left\{f*\rho_\nu \right\}_{\nu=1}^\infty$ converge en promedio a $f$, entonces debe tenerse que
            \begin{equation*}
                f(x)=\frac{1}{(2\pi)^n}\int_{\mathbb{R}^n}e^{ i\pint{x}{y}}\fou{f}(y)\:dy
            \end{equation*}
            para casi todo $x\in\mathbb{R}^n$.
            \item Queda por construir una sucesión de Dirac tal que cumpla (i) y (ii). La función $x\mapsto e^{-\sum_{ k=1}^n x_k^2}$ es no negativa y
            \begin{equation*}
                \int_{\mathbb{R}^n}e^{ -\sum_{ k=1}^n x_k^2}\:dx_1\cdots dx_n=\left(\int_{-\infty}^\infty e^{ -t^2}\:dt\right)^n=\pi^{ n/2}
            \end{equation*}
            defina
            \begin{equation*}
                \rho(x)=\frac{1}{\pi^{ n/2}}e^{-\sum_{ k=1}^n x_k^2},\quad\forall x\in\mathbb{R}^n
            \end{equation*}
            Esta función satisface que
            \begin{equation*}
                \int_{\mathbb{R}^n}\rho=1
            \end{equation*}
            Se sabe que la sucesión $\left\{\rho_\nu\right\}_{\nu=1}^\infty$ dada por:
            \begin{equation*}
                \rho_\nu(x)=\nu^n \rho(\nu x)=\frac{\nu^n}{\pi^{ n/2}}e^{-\sum_{ k=1}^n \nu^2 x_k^2},\quad\forall x\in\mathbb{R}^n
            \end{equation*}
            es una sucesión de Dirac en $\mathcal{L}_1(\mathbb{R}^n,\mathbb{C})$. Recuerde que si $a>0$, la transformada de Fourier de $t\mapsto e^{ -at^2}$ es
            \begin{equation*}
                t\mapsto\sqrt{\frac{\pi}{a}}e^{-\frac{t^2}{4a}}
            \end{equation*}
            para todo $t\in\mathbb{R}$. Entonces,
            \begin{equation*}
                \begin{split}
                    \fou{\rho_\nu}(x)&=\frac{\nu^n}{\pi^{ n/2}}e^{ -\sum_{ k=1}^n \frac{x_k^2}{4\nu^2}}\\
                    &=e^{ -\sum_{ k=1}^n \frac{x_k^2}{4\nu^2}},\quad\forall x\in\mathbb{R}^n \\
                \end{split}
            \end{equation*}
            En particular, $\fou{\rho_\nu}$ es integrable en $\mathbb{R}^n$. Además, por la parte (iv) de las propieaddes de la trasnformación de Fourier
            \begin{equation*}
                \begin{split}
                    \fou^*{(\fou{\rho_\nu})}(x)&=\fou{(\fou^*{\rho_\nu})}(x)\\
                    &=\fou{(\fou{\rho_\nu})}(-x)\\
                    &=\fou{(\fou{\rho_\nu})}(x)\\
                    &=\fou{\left[e^{ -\sum_{ k=1}^n \frac{x_k^2}{4\nu^2}} \right]}\\
                    &=\left[\frac{\pi}{\frac{1}{4\nu^2}}\right]^{ n/2}e^{-\sum_{ k=1}^n \frac{x_k^2}{4\left(\frac{1}{4\nu^2}\right)}}\\
                    &=(2\pi)^n\cdot\frac{\nu^n}{\pi^{n/2}}e^{ -\sum_{ k=1}^n \nu^2 x_k^2}\\
                    &=(2\pi)^n\rho_\nu(x),\quad\forall x\in\mathbb{R}^n\\
                \end{split}
            \end{equation*}
            lo que demuestra la existencia de tal sucesión de Dirac.
        \end{enumerate}
    \end{proof}

    \begin{propo}
        La transformación de Fourier $\fou{\cdot}$ es un homomorfismo inyectivo del álgebra de Banach $L_1(\mathbb{R}^n,\mathbb{C})$ en el álgebra de Banach $\mathcal{C}_0(\mathbb{R}^n,\mathbb{C})$.
    \end{propo}

    \begin{proof}
        Basta probar que el kernel de $\fou{\cdot}$ se reduce a $0$. En efecto, sea $f\in\mathcal{L}_1(\mathbb{R}^n,\mathbb{C})$ tal que $\fou{f}=0$ en $\mathbb{R}^n$, luego $\fou{f}\in\mathcal{L}_1(\mathbb{R}^n,\mathbb{C})$. Así se puede aplicar el Teorema anterior, que resulta en que
        \begin{equation*}
            0=\fou^*{0}=\fou^*{(\fou{f})}=(2\pi)^nf\textup{ c.t.p. en }\mathbb{R}^n
        \end{equation*}
        por tanto, $f=0$ c.t.p. en $\mathbb{R}^n$.
    \end{proof}

    \begin{propo}
        Sean $f,g\in\mathcal{L}_1(\mathbb{R}^n,\mathbb{C})$. Se supone que alguna de $\fou{f}$ y/o $\fou{g}$ es integrable en $\mathbb{R}^n$. Entonces se cumple la \textbf{Identidad de Parseval}.
        \begin{equation*}
            \int_{\mathbb{R}^n}\fou{f}\conj{\fou{g}}=(2\pi)^n\int_{\mathbb{R}^n}f\conj{g}
        \end{equation*}
    \end{propo}

    \begin{proof}
        Suponga que $\fou{f}$ es integrable en $\mathbb{R}^n$. Siendo $\fou{g}$ medible acotada, el primer lado tiene sentido. Se tiene que
        \begin{equation*}
            \begin{split}
                \conj{\fou{g}(x)}&=\conj{\int_{\mathbb{R}^n}e^{ -i\pint{x}{y}}g(y)\:dy}\\
                &=\int_{\mathbb{R}^n}e^{i\pint{x}{y}}\conj{g(y)}\:dy\\
                &=\fou^*{\conj{g}}(x)\\
            \end{split}
        \end{equation*}
        Así pues
        \begin{equation*}
            \begin{split}
                \int_{\mathbb{R}^n}\fou{f}\conj{\fou{g}}&=\int_{\mathbb{R}^n}\fou{f}\fou^*{\conj{g}}\\
                &=\int_{\mathbb{R}^n}\fou^*{\fou{f}}\cdot\conj{g}\\
                &=(2\pi)^n\int_{\mathbb{R}^n}f\conj{g}\\
            \end{split}
        \end{equation*}
    \end{proof}

    \section{Fórmula de inversión en $\mathbb{R}$}

    \begin{obs}
        Se afirma que
        \begin{equation*}
            \int_{0}^{\rightarrow\infty}\frac{\sin ax}{x}\:dx=\frac{\pi}{2}
        \end{equation*}
        y de la paridad de $x\mapsto\frac{\sin ax}{x}$ se concluye que
        \begin{equation*}
            \int_{ \rightarrow-\infty}^{\rightarrow\infty}\frac{\sin ax}{x}\:dx=\pi,\quad\forall a>0.
        \end{equation*}
    \end{obs}

    \begin{proof}
        En efecto, veamos que para $a>0$:
        \begin{equation*}
            \begin{split}
                \int_0^R\frac{\sin ax}{x}\:dx&=\int_0^{aR}\frac{\sin y}{\frac{y}{a}}\:\frac{dy}{a}\\
                &=\int_0^{aR} \frac{\sin y}{y}\:dy\\
            \end{split}
        \end{equation*}
        de donde se sigue el resultado. Si $a<0$, se tiene que
        \begin{equation*}
            \begin{split}
                \int_0^R\frac{\sin ax}{x}\:dx&=-\int_0^R\frac{\sin (-a)x}{x}\:dx\longrightarrow -\frac{\pi}{2}\\
            \end{split}
        \end{equation*}
        Se concluye que
        \begin{equation*}
            \int_0^{ \rightarrow\infty}\frac{\sin ax}{x}\:dx=\left\{
                \begin{array}{lcr}
                    \frac{\pi}{2} & \textup{ si } & a>0\\
                    0 & \textup{ si } & a=0\\
                    -\frac{\pi}{2} & \textup{ si } & a<0\\
                \end{array}
            \right.
        \end{equation*}
    \end{proof}

    \begin{theor}[\textbf{Teorema de inversión en $\mathbb{R}$}]
        Sean $f\in\mathcal{L}_1(\mathbb{R},\mathbb{C})$. Se supone que $f$ cumple la condición de Dini en cierto punto $x\in\mathbb{R}$, es decir, existe $\delta>0$ tal que
        \begin{equation*}
            \int_{-\delta}^\delta\abs{\frac{f(x+t)-f(x)}{t}}\:dt<\infty
        \end{equation*}
        Entonces,
        \begin{equation*}
            \begin{split}
                f(x)&=\lim_{ R\rightarrow\infty}\frac{1}{2\pi}\int_{ -R}^{ R}e^{ixy}\fou{f}(y)\:dy\\
                &=\lim_{ R\rightarrow\infty}\frac{1}{2\pi}\int_{ -R}^{ R}e^{ixy}\:dy\int_{-\infty}^\infty e^{ -iyz}f(z)\:dz \\
            \end{split}
        \end{equation*}
        y más aún:
        \begin{equation*}
            f(x)=\frac{1}{2\pi}\int_{ \rightarrow-\infty}^{\rightarrow\infty}\:dy\int_{ -\infty}^{ -\infty}f(t)\cos(y(t-x))\:dt
        \end{equation*}
    \end{theor}
    
    \begin{proof}
        Para $R>0$ se tiene lo siguiente:
        \begin{equation*}
            \begin{split}
                \int_{ -R}^{ R}e^{ixy}\:dy\int_{-\infty}^\infty e^{ -iyt}f(t)\:dt&=\int_{ -R}^{ R}\:dy\int_{-\infty}^\infty e^{ixy}e^{ -iyt}f(t)\:dt\\
                &=\int_{ -R}^{ R}\:dy\int_{-\infty}^\infty e^{ -iy(t-x)}f(t)\:dt\\
                &=\int_{ -R}^{ R}\:dy\int_{-\infty}^\infty f(t)\cos(y(t-x))\:dt+i\int_{ -R}^{ R}\:dy\int_{-\infty}^\infty f(t)\sin(y(t-x))\:dt\\
            \end{split}
        \end{equation*}
        como $y\mapsto\int_{-\infty}^\infty f(t)\sin(y(t-x))\:dt$ es impar, entonces:
        \begin{equation*}
            \int_{ -R}^{ R}\:dy\int_{-\infty}^\infty f(t)\sin(y(t-x))\:dt=0
        \end{equation*}
        y, como $y\mapsto\int_{-\infty}^\infty f(t)\cos(y(t-x))\:dt$ es par,
        \begin{equation*}
            \int_{ -R}^{ R}\:dy\int_{-\infty}^\infty f(t)\cos(y(t-x))\:dt=2\int_{0}^{ R}\:dy\int_{-\infty}^\infty f(t)\cos(y(t-x))\:dt
        \end{equation*}
        Si se prueba que
        \begin{equation*}
            f(x)=\lim_{ R\rightarrow\infty}\frac{1}{\pi}\int_{0}^{ R}\:dy\int_{-\infty}^\infty f(t)\cos(y(t-x))\:dt
        \end{equation*}
        se habrá probado también que
        \begin{equation*}
            f(x)=\frac{1}{2\pi}\int_{ \rightarrow-\infty}^{\rightarrow\infty}\:dy\int_{ -\infty}^{ -\infty}f(t)\cos(y(t-x))\:dt
        \end{equation*}
        (que es más de lo que se pide probar). Sea
        \begin{equation*}
            J(R)=\frac{1}{\pi}\int_0^R\:dy\int_{-\infty}^\infty f(t)\cos(y(t-x))\:dt,\quad\forall R>0
        \end{equation*}
        Veamos que
        \begin{equation*}
            \lim_{ R\rightarrow\infty}J(R)=f(x)
        \end{equation*}
        En efecto, 
    \end{proof}

    \begin{obs}
        Recuerde que la condición de Dini se cumple, por ejemplo, si $f$ tiene derivada por la derecha y la izquierda en ese punto.
    \end{obs}

\end{document}