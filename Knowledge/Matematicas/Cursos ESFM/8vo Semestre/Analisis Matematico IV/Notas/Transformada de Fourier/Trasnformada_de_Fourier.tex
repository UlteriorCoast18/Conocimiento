\documentclass[12pt]{report}
\usepackage[spanish]{babel}
\usepackage[utf8]{inputenc}
\usepackage{amsmath}
\usepackage{amssymb}
\usepackage{amsthm}
\usepackage{graphics}
\usepackage{subfigure}
\usepackage{lipsum}
\usepackage{array}
\usepackage{multicol}
\usepackage{enumerate}
\usepackage[framemethod=TikZ]{mdframed}
\usepackage[a4paper, margin = 1.5cm]{geometry}

%En esta parte se hacen redefiniciones de algunos comandos para que resulte agradable el verlos%

\renewcommand{\theenumii}{\roman{enumii}}

\def\proof{\paragraph{Demostración:\\}}
\def\endproof{\hfill$\blacksquare$}

\def\sol{\paragraph{Solución:\\}}
\def\endsol{\hfill$\square$}

%En esta parte se definen los comandos a usar dentro del documento para enlistar%

\newtheoremstyle{largebreak}
  {}% use the default space above
  {}% use the default space below
  {\normalfont}% body font
  {}% indent (0pt)
  {\bfseries}% header font
  {}% punctuation
  {\newline}% break after header
  {}% header spec

\theoremstyle{largebreak}

\newmdtheoremenv[
    leftmargin=0em,
    rightmargin=0em,
    innertopmargin=-2pt,
    innerbottommargin=8pt,
    hidealllines = true,
    roundcorner = 5pt,
    backgroundcolor = gray!60!red!30
]{exa}{Ejemplo}[section]

\newmdtheoremenv[
    leftmargin=0em,
    rightmargin=0em,
    innertopmargin=-2pt,
    innerbottommargin=8pt,
    hidealllines = true,
    roundcorner = 5pt,
    backgroundcolor = gray!50!blue!30
]{obs}{Observación}[section]

\newmdtheoremenv[
    leftmargin=0em,
    rightmargin=0em,
    innertopmargin=-2pt,
    innerbottommargin=8pt,
    rightline = false,
    leftline = false
]{theor}{Teorema}[section]

\newmdtheoremenv[
    leftmargin=0em,
    rightmargin=0em,
    innertopmargin=-2pt,
    innerbottommargin=8pt,
    rightline = false,
    leftline = false
]{propo}{Proposición}[section]

\newmdtheoremenv[
    leftmargin=0em,
    rightmargin=0em,
    innertopmargin=-2pt,
    innerbottommargin=8pt,
    rightline = false,
    leftline = false
]{cor}{Corolario}[section]

\newmdtheoremenv[
    leftmargin=0em,
    rightmargin=0em,
    innertopmargin=-2pt,
    innerbottommargin=8pt,
    rightline = false,
    leftline = false
]{lema}{Lema}[section]

\newmdtheoremenv[
    leftmargin=0em,
    rightmargin=0em,
    innertopmargin=-2pt,
    innerbottommargin=8pt,
    roundcorner=5pt,
    backgroundcolor = gray!30,
    hidealllines = true
]{mydef}{Definición}[section]

\newmdtheoremenv[
    leftmargin=0em,
    rightmargin=0em,
    innertopmargin=-2pt,
    innerbottommargin=8pt,
    roundcorner=5pt
]{excer}{Ejercicio}[section]

%En esta parte se colocan comandos que definen la forma en la que se van a escribir ciertas funciones%

\renewcommand{\leq}{\ensuremath{\leqslant}}
\renewcommand{\geq}{\ensuremath{\geqslant}}

\newcommand\abs[1]{\ensuremath{\left|#1\right|}}
\newcommand\divides{\ensuremath{\bigm|}}
\newcommand\cf[3]{\ensuremath{#1:#2\rightarrow#3}}
\newcommand\norm[1]{\ensuremath{\|#1\|}}
\newcommand\ora[1]{\ensuremath{\vec{#1}}}
\newcommand\pint[2]{\ensuremath{\left(#1\big| #2\right)}}
\newcommand\conj[1]{\ensuremath{\overline{#1}}}
\newcommand{\N}[2]{\ensuremath{\mathcal{N}_{#1}\left(#2\right)}}
\newcommand{\natint}[1]{\ensuremath{\left[\!\left[#1\right]\!\right]}}
\newcommand{\fou}[1]{\ensuremath{\mathcal{F}#1}}

\begin{document}
    \setlength{\parskip}{5pt} % Añade 5 puntos de espacio entre párrafos
    \setlength{\parindent}{12pt} % Pone la sangría como me gusta
    \title{Notas de Análisis Matemático IV}
    \author{Cristo Daniel Alvarado}
    \maketitle

    \tableofcontents %Con este comando se genera el índice general del libro%

    %\setcounter{chapter}{3} %En esta parte lo que se hace es cambiar la enumeración del capítulo%
    
    \chapter{Transformación de Fourier}
    
    La transformada de Fourier de una función de $\mathbb{R}^n$ en $\mathbb{C}$ generaliza en cierta forma la noción de coeficietes de Fourier de funciones periódicas
    
    \section{Conceptos Fundamentales}

    \begin{mydef}
        Sea $f\in\mathcal{L}_1(\mathbb{R}^n,\mathbb{C})$. Se definen $\cf{\mathcal{F}f,\mathcal{F}^*f}{\mathbb{R}^n}{\mathbb{C}} $ como
        \begin{equation*}
            \mathcal{F}f(x)=\int_{ \mathbb{R}^n}e^{ -i\pint{x}{y}}f(y)\:dy\quad\textup{y}\quad\mathcal{F}^*f(x)=\int_{ \mathbb{R}^n}e^{i\pint{x}{y}}f(y)\:dy
        \end{equation*}
        para todo $x\in\mathbb{R}^n$. Las funciones $\mathcal{F}f$ y $\mathcal{F}^*f$ se llaman las \textbf{transformaciones de Fourier de $f$}. Las aplicaciones $\mathcal{F}$ y $\mathcal{F}^*$ de $\mathcal{L}_1(\mathbb{R}^n,\mathbb{C})$ en el conjunto de funciones de $\mathbb{R}^n$ en $\mathbb{C}$ se llaman las \textbf{transformaciones de Fourier}.
    \end{mydef}

    \renewcommand{\theenumi}{\roman{enumi}}

    \begin{obs}
        Se tiene lo siguiente:
        \begin{enumerate}
            \item Los operadores $\mathcal{F}$ y $\mathcal{F}^*$ son lineales de $\mathcal{L}_1(\mathbb{R}^n,\mathbb{C})$ en el espacio de funciones de $\mathbb{R}^n$ en $\mathbb{C}$.
            \item Las funciones $\fou{f}(x)$ y $\fou^*{f}(x)$ están definidas para todo $x\in\mathbb{R}^n$ si y sólo si $f\in\mathcal{L}_1(\mathbb{R}^n,\mathbb{C})$.
            \item En caso de existir, se tiene que $\fou{f}(x)=\fou^*{f}(-x)$.
        \end{enumerate}
    \end{obs}

    \begin{proof}
        De (i): Es claro que si $f\in\mathcal{L}_1(\mathbb{R}^n,\mathbb{C})$ entonces $\fou{f}(x)$ y $\fou^*{f}(x)$ están definidas para todo $x\in\mathbb{R}^n$. Para la recíproca, en particular están definidas para $x=\vec{0}$, es decir que
        \begin{equation*}
            \fou{f}\left(\vec{0}\right)=\int_{\mathbb{R}^n}e^{ -i\pint{\vec{0}}{y}}f(y)\:dy=\int_{\mathbb{R}^n}e^{0}f(y)\:dy=\int_{\mathbb{R}^n}f(y)\:dy<\infty
        \end{equation*}
        luego $f\in\mathcal{L}_1(\mathbb{R}^n,\mathbb{C})$.

        De (ii): Es inmediata.
    \end{proof}

    \begin{mydef}
        Sea $f\in\mathcal{L}_1([0,\infty[,\mathbb{C})$. Se definen
        \begin{equation*}
            \mathcal{F}_cf(x)=\int_0^\infty f(y)\cos xy\:dy\quad\textup{y}\quad\mathcal{F}_sf(x)=\int_0^\infty f(y)\sin xy\:dy
        \end{equation*}
        para todo $x\in\mathbb{R}$. Las funciones $\mathcal{F}_cf$ y  $\mathcal{F}_sf$ se llaman \textbf{las trasnformadas coseno y seno de Fourier de $f$}.
    \end{mydef}

    \begin{mydef}
        Sea $\cf{f}{[0,\infty[}{\mathbb{C}}$ una función. Se definen las funciones $f^P$ y $f^I$ de $\mathbb{R}$ en $\mathbb{C}$ como
        \begin{equation*}
            f^P(x)=\left\{ 
                \begin{array}{lcr}
                    f(x) & \textup{ si } & x\geq0\\
                    f(-x) & \textup{ si } & x<0\\
                \end{array}
            \right.
        \end{equation*}
        y,
        \begin{equation*}
            f^I(x)=\left\{ 
                \begin{array}{lcr}
                    f(x) & \textup{ si } & x\geq0\\
                    -f(-x) & \textup{ si } & x<0\\
                \end{array}
            \right.
        \end{equation*}
    \end{mydef}

    \begin{propo}
        Sea $f\in\mathcal{L}_1([0,\infty[,\mathbb{C})$. Se tiene
        \begin{equation*}
            \fou{f^P}(x)=2\mathcal{F}_cf(x)\quad\textup{y}\quad\fou{f^I}(x)=-2i\mathcal{F}_2f(x)
        \end{equation*}
        para todo $x\in\mathbb{R}$.
    \end{propo}

    \begin{proof}
        Sea $x\in\mathbb{R}$, entonces
        \begin{equation*}
            \begin{split}
                \fou{f^P}(x)&=\int_{\mathbb{R}} f^P(y) e^{ -i\pint{x}{y}} \:dy\\
                &=\int_{-\infty}^\infty f^P(y)e^{-ixy}\:dy\\
                &=\int_{-\infty}^0f^P(y)e^{-ixy}\:dy+\int_0^{\infty}f^P(y)e^{-ixy}\:dy \\
                &=\int_{-\infty}^0f(-y)e^{-ixy}\:dy+\int_0^{\infty}f(y)e^{-ixy}\:dy\\
                &=\int_{0}^\infty f(y)e^{ixy}\:dy+\int_0^{\infty}f(y)e^{-ixy}\:dy\\
                &=\int_{0}^\infty f(y)\left[e^{ixy}+\conj{e^{ixy}}\right] \:dy\\
                &=\int_{0}^\infty f(y)\left[2\Re(e^{ ixy}) \right] \:dy\\
                &=\int_{0}^\infty 2f(y)\cos xy \:dy\\
                &=2\int_{0}^\infty f(y)\cos xy \:dy\\
                &=2\mathcal{F}_cf(x)\\
            \end{split}
        \end{equation*}
        y,
        \begin{equation*}
            \begin{split}
                \fou{f^I}(x)&=\int_{\mathbb{R}}f^I(y)e^{ -ixy}\:dy\\
                &=\int_{-\infty}^0f^I(y)e^{ -ixy}\:dy+\int_{0}^\infty f^I(y)e^{ -ixy}\:dy\\
                &=\int_{-\infty}^0(-f(-y))e^{ -ixy}\:dy+\int_{0}^\infty f(y)e^{ -ixy}\:dy\\
                &=-\int_{0}^\infty f(y)e^{ixy}\:dy+\int_{0}^\infty f(y)e^{ -ixy}\:dy\\
                &=\int_{0}^\infty f(y)\left[-e^{ixy}+e^{-ixy}\right] \:dy\\
                &=\int_{0}^\infty f(y)\left[-\cos xy-i\sin xy+\cos(-xy)+i\sin(-xy)\right] \:dy\\
                &=\int_{0}^\infty f(y)\left[-2i\sin xy\right]\:dy\\
                &=-2i\int_{0}^\infty f(y)\sin xy\:dy\\
                &=-2i\mathcal{F}_sf(x)\\
            \end{split}
        \end{equation*}
        lo que prueba el resultado.
    \end{proof}

    \begin{cor}
        Sea $f\in\mathcal{L}_1(\mathbb{R},\mathbb{C})$.
        \begin{enumerate}
            \item Si $f$ es par, entonces $\fou{f}(x)=2\int_0^{\infty}f(y)\cos xy\:dy$ para todo $x\in\mathbb{R}$.
            \item Si $f$ es impar, entonces $\fou{f}(x)=-2i\int_0^{\infty}f(y)\sin xy\:dy$ para todo $x\in\mathbb{R}$.
        \end{enumerate}
    \end{cor}

    \begin{exa}
        Se tiene lo siguiente:
        \begin{enumerate}
            \item Sea $\cf{f}{\mathbb{R}}{\mathbb{R}}$ la función $f=\chi_I$ donde $I$ es un intervalo con extremos $a<b$ en $\mathbb{R}$. Entonces,
            \begin{equation*}
                \begin{split}
                    \fou{f}(x)&=\int_{-\infty}^\infty \chi_I(y)e^{ -ixy}\:dy\\
                    &=\int_{a}^b e^{ -ixy}\:dy\\
                    &=\left\{ 
                        \begin{array}{lcr}
                            \frac{e^{ -ixb}-e^{ -ixa}}{-ix} & \textup{ si } & x\neq0\\
                            b-a & \textup{si} & x=0\\
                        \end{array}
                    \right.\\
                \end{split}
            \end{equation*}
            En particular, si $a>0$ se tiene que
            \begin{equation*}
                \fou{\chi_[-a,a]}(x)=\left\{ 
                    \begin{array}{lcr}
                        \frac{2\sin ax}{x} & \textup{ si } & x\neq0\\
                        2a & \textup{si} & x=0\\
                    \end{array}
                \right.
            \end{equation*}
            Como $\fou{\chi_[-a,a]}$ no es integrable en $\mathbb{R}$ se concluye que, en general, la transformada de Fourier de una función integrable no necesariamente es integrable.
            \item Sea $\cf{f}{\mathbb{R}}{\mathbb{R}}$ la función
            \begin{equation*}
                f(x)=e^{-k\abs{x}},\quad\forall x\in\mathbb{R}
            \end{equation*}
            donde $k>0$. Como $f$ es integrable, entonces
            \begin{equation*}
                \begin{split}
                    \fou{f}(x)&=\int_{-\infty}^\infty e^{-k\abs{y}}e^{-ixy}\:dy\\
                    &=\int_{-\infty}^0 e^{ky}e^{-ixy}\:dy+\int_{0}^\infty e^{-ky}e^{-ixy}\:dy\\
                    &=\int_{0}^\infty e^{-ky}e^{ixy}\:dy+\int_{0}^\infty e^{-ky}e^{-ixy}\:dy\\
                    &=\int_{0}^\infty e^{-ky}e^{(-k+ix)y}\:dy+\int_{0}^\infty e^{-ky}e^{(-k-ix)y}\:dy\\
                    &=\frac{-1}{-k+ix}+\frac{-1}{-k-ix}\\
                    &=\frac{k+ix+k-ix}{k^2+x^2}\\
                    &=\frac{2k}{k^2+x^2}\\
                \end{split}
            \end{equation*}
        \end{enumerate}
    \end{exa}

    \begin{exa}
        Sea $\cf{f}{\mathbb{R}}{\mathbb{R}}$ la función
        \begin{equation*}
            f(x)=e^{ -kx^2},\quad\forall x\in\mathbb{R}
        \end{equation*}
        donde $k>0$. Como $f$ es par se tiene que
        \begin{equation*}
            \fou{f}(x)=2\int_0^{\infty}e^{ -ky^2}\cos xy\:dy
        \end{equation*}
        Sea $g(x)=\int_0^\infty e^{ -ky^2}\cos xy\:dy$ para todo $x\in\mathbb{R}$. Se afirma que
        \begin{equation*}
            g'(x)=-\int_0^{\infty}ye^{ -ky^2}\sin xy\:dy,\quad\forall x\in\mathbb{R}
        \end{equation*}
        En efecto, observemos que
        \begin{equation*}
            \begin{split}
                \abs{ye^{-ky^2}\sin xy}\leq ye^{-ky^2},\quad\forall y\geq0
            \end{split}
        \end{equation*}
        donde la función de la derecha es integrable e independiente de $x$ (se nota fácilmente que una de sus antiderivadas es $y\mapsto-\frac{1}{2k}e^{ -ky^2}$, por el T.F.C. II evaluando en $0$ e $\infty$ se obtiene que la función original es integrable en $[0,\infty[$). Por el Teorema de derivación se sigue que
        \begin{equation*}
            g'(x)=-\int_0^{\infty}ye^{ -ky^2}\sin xy\:dy,\quad\forall x\in\mathbb{R}
        \end{equation*}
        Veamos ahora que
        \begin{equation*}
            \begin{split}
                g'(x)&=\int_0^{\infty}ye^{ -ky^2}\sin xy\:dy\\
                &=-\left[-\frac{1}{2k}e^{ -ky^2}\sin xy\Big|_0^{\infty}+\frac{1}{2k}\int_0^{\infty}xe^{ -ky^2}\cos xy\:dy \right]\\
                &=-\left[0-0+\frac{x}{2k}\int_0^{\infty}e^{ -ky^2}\cos xy\:dy \right]\\
                &=-\frac{x}{2k}\int_0^{\infty}e^{ -ky^2}\cos xy\:dy\\
                &=-\frac{x}{2k}g(x),\quad\forall x\in\mathbb{R} \\
            \end{split}
        \end{equation*}
        Luego, 
        \begin{equation*}
            \begin{split}
                g'(x)+\frac{x}{2k}g(x)&=0,\quad\forall x\in\mathbb{R}\\
                \Rightarrow e^{\frac{x^2}{4k}}\left(g'(x)+\frac{x}{2k}g(x)\right)&=0,\quad\forall x\in\mathbb{R}\\
                \Rightarrow \frac{d}{dx}\left(e^{\frac{x^2}{4k}}g(x) \right)(x_0)&=0,\quad\forall x_0\in\mathbb{R} \\
                \Rightarrow e^{\frac{x^2}{4k}}g(x)&=c,\quad\forall x\in\mathbb{R}\\
            \end{split}
        \end{equation*}
        En particular,
        \begin{equation*}
            \begin{split}
                c&=g(0)\\
                &=\int_0^\infty e^{ -ky^2}\:dy\\
                &=\frac{1}{\sqrt{k}}\int_0^\infty e^{-u^2}\:du\\
                &=\frac{1}{2}\cdot\sqrt{\frac{\pi}{k}}
            \end{split}
        \end{equation*}
        Por ende,
        \begin{equation*}
            g(x)=\frac{1}{2}\sqrt{\frac{\pi}{k}}e^{ -\frac{x^2}{4k}},\quad\forall x\in\mathbb{R}
        \end{equation*}
        De donde se sigue que
        \begin{equation*}
            \fou{f}(x)=\sqrt{\frac{\pi}{k}}e^{ -\frac{x^2}{4k}}, \quad\forall x\in\mathbb{R}
        \end{equation*}
        En particular, si $k=\frac{1}{2}$ entonces $f(x)=e^{-\frac{x^2}{2}}$ para todo $x\in\mathbb{R}$ y,
        \begin{equation*}
            \fou{f}(x)=\sqrt{2\pi}e^{ -\frac{x^2}{2}}=\sqrt{2\pi}f(x)
        \end{equation*}
        es decir que $f$ es un vector propio del operador transformada de Fourier.
    \end{exa}

    \begin{propo}
        Sea $f\in\mathcal{L}_1(\mathbb{R}^n,\mathbb{C})$.
        \begin{enumerate}
            \item Si $g(x)=e^{i\pint{a}{y}}f(x)$ para todo $x\in\mathbb{R}^n$, entonces
            \begin{equation*}
                \fou{g}(x)=\fou{f}(x-a),\quad\forall x\in\mathbb{R}^n
            \end{equation*}
            \item Si $g(x)=f(x-a)$ para todo $x\in\mathbb{R}^n$, entonces
            \begin{equation*}
                \fou{g}(x)=e^{ -i\pint{x}{a}}\fou{f}(x),\quad\forall x\in\mathbb{R}^n
            \end{equation*}
            \item Si $g(x)=\conj{f(-x)}$ para todo $x\in\mathbb{R}^n$, entonces
            \begin{equation*}
                \fou{g}(x)=\conj{\fou{f}(x)},\quad\forall x\in\mathbb{R}^n
            \end{equation*}
            \item Sea $\lambda\in\mathbb{R}\backslash\left\{0\right\}$. Si $g(x)=f(\frac{x}{\lambda})$ para todo $x\in\mathbb{R}^n$, entonces
            \begin{equation*}
                \fou{g}(x)=\abs{\lambda}^n\fou{f}(\lambda x),\quad\forall x\in\mathbb{R}^n
            \end{equation*}
        \end{enumerate}
    \end{propo}

    \begin{proof}
        De (i): Veamos que
        \begin{equation*}
            \begin{split}
                \fou{g}(x)&=\int_{\mathbb{R}^n}e^{ -i\pint{x}{y}}g(y)\:dy\\
                &=\int_{\mathbb{R}^n}e^{ -i\pint{x}{y}}e^{ i\pint{a}{y}}f(y)\:dy\\
                &=\int_{\mathbb{R}^n}e^{ -i\pint{x-a}{y}}f(y)\:dy\\
                &=\fou{f}(x-a)
            \end{split}
        \end{equation*}
        para todo $x\in\mathbb{R}^n$.

        De (ii): Veamos que
        \begin{equation*}
            \begin{split}
                \fou{g}(x)&=\int_{\mathbb{R}^n}e^{ -i\pint{x}{y}}g(y)\:dy\\
                &=\int_{\mathbb{R}^n}e^{ -i\pint{x}{y}}f(y-a)\:dy\\
                &=\int_{\mathbb{R}^n}e^{ -i\pint{x}{u+a}}f(u)\:du\\
                &=e^{ -i\pint{x}{a}}\fou{f}(x) \\
            \end{split}
        \end{equation*}
        para todo $x\in\mathbb{R}^n$.

        De (iii): Veamos que
        \begin{equation*}
            \begin{split}
                \fou{g}(x)&=\int_{\mathbb{R}^n}e^{-i\pint{x}{y}}\conj{f(-y)}\:dy\\
                &=\int_{\mathbb{R}^n}e^{i\pint{x}{y}}\conj{f(y)}\:dy\\
                &=\conj{\int_{\mathbb{R}^n}e^{-i\pint{x}{y}}f(y)\:dy}\\
                &=\conj{\fou{f}(x)}\\
            \end{split}
        \end{equation*}
        para todo $x\in\mathbb{R}^n$.
        
        De (iv): Veamos que
        \begin{equation*}
            \begin{split}
                \fou{g}(x)&=\int_{\mathbb{R}^n}e^{-i\pint{x}{y}}g(y)\:dy\\
                &=\int_{\mathbb{R}^n}e^{-i\pint{x}{y}}f\left(\frac{y}{\lambda}\right) \:dy,\textup{ haciendo el cambio de variable }u=\frac{y}{\lambda} \\
                &=\int_{\mathbb{R}^n}e^{-i\pint{x}{\lambda u}}f(u) \:\abs{\lambda}^n du\\
                &=\abs{\lambda}^n\int_{\mathbb{R}^n}e^{-i\pint{\lambda x}{ u}}f(u) \:du\\
                &=\abs{\lambda}^n\fou{f}(\lambda x)\\
            \end{split}
        \end{equation*}
        para todo $x\in\mathbb{R}^n$.
    \end{proof}
\end{document}