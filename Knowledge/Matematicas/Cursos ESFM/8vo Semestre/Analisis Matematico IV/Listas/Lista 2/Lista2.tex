\documentclass[12pt]{report}
\usepackage[spanish]{babel}
\usepackage[utf8]{inputenc}
\usepackage{amsmath}
\usepackage{amssymb}
\usepackage{amsthm}
\usepackage{graphics}
\usepackage{subfigure}
\usepackage{lipsum}
\usepackage{array}
\usepackage{multicol}
\usepackage{enumerate}
\usepackage[framemethod=TikZ]{mdframed}
\usepackage[a4paper, margin = 1.5cm]{geometry}

%En esta parte se hacen redefiniciones de algunos comandos para que resulte agradable el verlos%

\renewcommand{\theenumii}{\roman{enumii}}

\def\proof{\paragraph{Demostración:\\}}
\def\endproof{\hfill$\blacksquare$}

\def\sol{\paragraph{Solución:\\}}
\def\endsol{\hfill$\square$}

%En esta parte se definen los comandos a usar dentro del documento para enlistar%

\newtheoremstyle{largebreak}
  {}% use the default space above
  {}% use the default space below
  {\normalfont}% body font
  {}% indent (0pt)
  {\bfseries}% header font
  {}% punctuation
  {\newline}% break after header
  {}% header spec

\theoremstyle{largebreak}

\newmdtheoremenv[
    leftmargin=0em,
    rightmargin=0em,
    innertopmargin=-2pt,
    innerbottommargin=8pt,
    hidealllines = true,
    roundcorner = 5pt,
    backgroundcolor = gray!60!red!30
]{exa}{Ejemplo}[section]

\newmdtheoremenv[
    leftmargin=0em,
    rightmargin=0em,
    innertopmargin=-2pt,
    innerbottommargin=8pt,
    hidealllines = true,
    roundcorner = 5pt,
    backgroundcolor = gray!50!blue!30
]{obs}{Observación}[section]

\newmdtheoremenv[
    leftmargin=0em,
    rightmargin=0em,
    innertopmargin=-2pt,
    innerbottommargin=8pt,
    rightline = false,
    leftline = false
]{theor}{Teorema}[section]

\newmdtheoremenv[
    leftmargin=0em,
    rightmargin=0em,
    innertopmargin=-2pt,
    innerbottommargin=8pt,
    rightline = false,
    leftline = false
]{propo}{Proposición}[section]

\newmdtheoremenv[
    leftmargin=0em,
    rightmargin=0em,
    innertopmargin=-2pt,
    innerbottommargin=8pt,
    rightline = false,
    leftline = false
]{cor}{Corolario}[section]

\newmdtheoremenv[
    leftmargin=0em,
    rightmargin=0em,
    innertopmargin=-2pt,
    innerbottommargin=8pt,
    rightline = false,
    leftline = false
]{lema}{Lema}[section]

\newmdtheoremenv[
    leftmargin=0em,
    rightmargin=0em,
    innertopmargin=-2pt,
    innerbottommargin=8pt,
    roundcorner=5pt,
    backgroundcolor = gray!30,
    hidealllines = true
]{mydef}{Definición}[section]

\newmdtheoremenv[
    leftmargin=0em,
    rightmargin=0em,
    innertopmargin=-2pt,
    innerbottommargin=8pt,
    roundcorner=5pt
]{excer}{Ejercicio}[section]

%En esta parte se colocan comandos que definen la forma en la que se van a escribir ciertas funciones%

\newcommand\abs[1]{\ensuremath{\big|#1\big|}}
\newcommand\divides{\ensuremath{\bigm|}}
\newcommand\cf[3]{\ensuremath{#1:#2\rightarrow#3}}
\newcommand{\Vol}[1]{\ensuremath{\textup{Vol}\left(#1\right)}}
\newcommand{\N}[2]{\ensuremath{\mathcal{N}_{#1}\left(#2\right)}}

%recuerda usar \clearpage para hacer un salto de página

\begin{document}
    \setlength{\parskip}{5pt} % Añade 5 puntos de espacio entre párrafos
    \setlength{\parindent}{12pt} % Pone la sangría como me gusta
    \title{Lista 2 de Ejercicios 
    
    Análisis Matemático IV
    }
    \author{Cristo Daniel Alvarado}
    \maketitle

    \tableofcontents %Con este comando se genera el índice general del libro%

    %\setcounter{chapter}{3} %En esta parte lo que se hace es cambiar la enumeración del capítulo%
    
    \chapter{Ejercicios Convolución}
    
    \setcounter{section}{1}

    \renewcommand{\theenumi}{\roman{enumi}}

    \begin{excer}
        Sean $\cf{f,g}{\mathbb{R}}{\mathbb{K}}$ funciones nulas en $]-\infty,0[$. Si existe $f*g(x)$, demuestre que:
        \begin{equation*}
            f*g(x)=\left\{
                \begin{array}{lcr}
                    \int_{0}^\infty f(y)g(x-y)dy & \textup{ si } & x\geq0\\
                    0 & \textup{ si }&x<0\\
                \end{array}
            \right.
        \end{equation*}
        En los casos siguientes $f$ y $g$ son nulas en $]-\infty.0[$ y sus valores en $[0,\infty[$ se indican abajo. Calcule $f*g$.
        \begin{enumerate}
            \item $f(x)=e^{-x}$ y $g(x)=\left\{
                    \begin{array}{lcr}
                        x & \textup{ si } & 0\leq x\leq 1 \\
                        0 & \textup{ si }&x>1\\
                    \end{array}
                \right.$
            \item $f(x)=g(x)=e^{-x}$.
            \item $f(x)=\left\{
                \begin{array}{lcr}
                    1 & \textup{ si } & 0\leq x\leq 1 \\
                    0 & \textup{ si }&x>1\\
                \end{array}
            \right.$ y $g(x)=\left\{
                \begin{array}{lcr}
                    x & \textup{ si } & 0\leq x\leq 1 \\
                    0 & \textup{ si }&x>1\\
                \end{array}
            \right.$
            \item $f(x)=g(x)=\left\{
                \begin{array}{lcr}
                    x & \textup{ si } & 0\leq x\leq 1 \\
                    0 & \textup{ si }&x>1\\
                \end{array}
            \right.$
        \end{enumerate}
    \end{excer}

    \begin{sol}
        Para la demostracíón, el caso $x\geq0$ es inmediato de la definición de convolución y del hecho de que $f$ es nula en $]-\infty,0[$. Suponga que existe $f*g(x)$ con $x<0$. Entonces:
        \begin{equation*}
            \begin{split}
                f*g(x)&=\int_{-\infty}^\infty f(y)g(x-y)dy\\
                &=\int_{0}^\infty f(y)g(x-y)dy\\
            \end{split}
        \end{equation*}
        sea $y\in[0,\infty[$, es decir que $0\leq y<\infty$, por lo cual $-\infty<-y\leq0$. Sumando $x$ a ambos lados se sigue que:
        \begin{equation*}
            -\infty<x-y\leq x<0\Rightarrow x-y\in ]-\infty,0[
        \end{equation*}
        por tanto, $g(x-y)=0$, para todo $y\in[0,\infty[$. Por tanto, $f*g(x)=0$.

        De (i): Veamos que:
        \begin{equation*}
            \begin{split}
                f*g(x)&=\left\{
                    \begin{array}{lcr}
                       \int_0^\infty e^{-y}g(x-y)dy & \textup{ si } & x\geq0\\
                       0 & \textup{ si } & x<0\\ 
                    \end{array}
                \right.\\
            \end{split}
        \end{equation*}
        Sea $x\geq0$. Analicemos varios casos:
        \begin{itemize}
            \item $0\leq x\leq 1$, en este caso $0\leq x-y\leq 1$ si y sólo si  $y\leq x$ y $x-1\leq y$ (pero, $x-1\leq 0$, por lo cual $0\leq y$), por ende:
            \begin{equation*}
                \begin{split}
                    f*g(x)&=\int_0^x e^{-y}g(x-y)dy\\
                    &=\int_0^x e^{-y}(x-y)dy\\
                    &=x\int_0^x e^{-y}dy-\int_0^x ye^{-y}dy\\
                    &=x\left[-e^{-y} \right]_0^x-\left[-e^{-y}(y+1) \right]_0^x\\
                    &=x-xe^{-x}+\left[e^{-y}(y+1) \right]_0^x\\
                    &=x-xe^{-x}+(x+1)e^{-x}-1\\
                    &=(x-1)+e^{-x}\\
                \end{split}
            \end{equation*}
            \item $1<x$, en este caso $0\leq x-y\leq 1$ si y sólo si $y\leq x$ y $x-1\leq y$ (donde $0<x-1$ por como se eligió el $x$). Por ende:
            \begin{equation*}
                \begin{split}
                    f*g(x)&=\int_{x-1}^x e^{-y}g(x-y)dy\\
                    &=\int_{x-1}^x e^{-y}(x-y)dy\\
                    &=x\int_{x-1}^x e^{-y}dy-\int_{x-1}^xye^{-y}dy\\
                    &=x\left[-e^{-y} \right]_{x-1}^x+\left[(y+1)e^{-y}\right]_{x-1}^x \\
                    &=xe^{1-x}-xe^{-x}+(x+1)e^{-x}-(x-1+1)e^{1-x}\\
                    &=xe^{1-x}-xe^{-x}+xe^{-x}+e^{-x}-xe^{1-x}\\
                    &=e^{-x}\\
                \end{split}
            \end{equation*}
        \end{itemize}
        Por tanto:
        \begin{equation*}
            f*g(x)=\left\{
                \begin{array}{lcr}
                    e^{-x} & \textup{ si } & 1<x \\
                    (x-1)+e^{-x} & \textup{ si } & 0\leq x \leq1 \\
                    0 & \textup{ si } & x<0\\ 
                \end{array}
            \right.
        \end{equation*}

        De (ii): Veamos que:
        \begin{equation*}
            f*g(x)=\left\{
                \begin{array}{lcr}
                    \int_0^\infty e^{-y}g(x-y) & \textup{ si } & 0\leq x \\
                    0 & \textup{ si } & x<0\\ 
                \end{array}
            \right.
        \end{equation*}
        analicemos a $g(x-y)$. Si $x\geq 0$ entonces, $x-y\geq 0$ si y sólo si $x\geq y$. Por tanto, para $x\geq 0$:
        \begin{equation*}
            \begin{split}
                \int_0^\infty e^{-y}g(x-y)&=\int_0^x e^{-y}e^{y-x}dy\\
                &=\int_0^x e^{-x}dy\\
                &=xe^{-x}\\
            \end{split}
        \end{equation*}
        de esta forma:
        \begin{equation*}
            f*g(x)=\left\{
                \begin{array}{lcr}
                    xe^{-x} & \textup{ si } & 0\leq x \\
                    0 & \textup{ si } & x<0\\ 
                \end{array}
            \right.
        \end{equation*}

        De (iii): Veamos que:
        \begin{equation*}
            f*g(x)=\left\{
                \begin{array}{lcr}
                    \int_0^1g(x-y)dy & \textup{ si } & 0\leq x \\
                    0 & \textup{ si } & x<0\\ 
                \end{array}
            \right.
        \end{equation*}
    \end{sol}
    
    \begin{excer}
        Haga lo siguiente:
        \begin{enumerate}
            \item Para toda $m\in\mathbb{N}$ se define $\cf{e_m}{\mathbb{R}}{\mathbb{R}}$ como:
            \begin{equation*}
                e_m(x)=\left\{
                    \begin{array}{lcr}
                        \frac{x^{m-1}}{(m-1)!} & \textup{ si } & x\geq 0 \\
                        0 & \textup{ si }&x<0\\
                    \end{array}
                \right.
            \end{equation*}
            \textbf{Pruebe} que
            \begin{equation*}
                e_p*e_q=e_{p+q}
            \end{equation*}
            \item Sea $\cf{f}{\mathbb{R}}{\mathbb{K}}$ integrable en todo intervalo acotado tal que $f(x)=0$ para todo $x\leq a$. \textbf{Muestre} que
            \begin{equation*}
                e_m*f(x)=\left\{
                    \begin{array}{lcr}
                        \int_{a}^x\frac{(x-y)^{m-1}}{(m-1)!}f(y)dy & \textup{ si } & x\geq 0 \\
                        0 & \textup{ si }&x<0\\
                    \end{array}
                \right.
            \end{equation*}
            \item \textbf{Deduzca} que para $x\geq a$ se cumple la siguiente \textbf{fórmula de Cauchy para la $n$-ésima integral indefinida}
            \begin{equation*}
                \int_{a}^xdx_{ m-1}\int_a^{ x_{m-1}}dx_{m-2}\cdots\int_{a}^{x_{2}} dx_{1}\int_a^{ x_1}f(x_0)dx_0=\int_a^x\frac{(x-y)^{m-1}}{(m-1)!}f(y)dy
            \end{equation*}
        \end{enumerate}
    \end{excer}

    \begin{proof}
        De (1): Sean $p,q\in\mathbb{N}$. Entonces:
        \begin{equation*}
            e_p(x)=\left\{
                    \begin{array}{lcr}
                        \frac{x^{p-1}}{(p-1)!} & \textup{ si } & x\geq 0 \\
                        0 & \textup{ si }&x<0\\
                    \end{array}
                \right.
            \quad\textup{y}\quad
            e_q(x)=\left\{
                    \begin{array}{lcr}
                        \frac{x^{q-1}}{(q-1)!} & \textup{ si } & x\geq 0 \\
                        0 & \textup{ si }&x<0\\
                    \end{array}
                \right.
        \end{equation*}
        Por lo tanto,
        \begin{equation*}
            \begin{split}
                e_p*e_q(y)&=\int_{-\infty}^\infty e_p(x)\cdot e_q(y-x)dx\\
                &=\int_{0}^\infty\frac{x^{p-1}}{(p-1)!}\cdot e_q(y-x)dx\\ 
            \end{split}
        \end{equation*}
        analicemos dos casos:
        \begin{itemize}
            \item $y<0$: Entonces, para todo $x\geq 0$, se sigue que $-x\leq 0$, luego $y-x<0$. Por ende, $e(y-x)=0$. Luego:
            \begin{equation*}
                e_p*e_q(y)=0=e_{p+q}(y)
            \end{equation*}
            \item $y\geq 0$: Entonces, $y-x\geq 0$ si y sólo si $x\in[0,y]$. Por tanto, la integral se vuelve en:
            \begin{equation*}
                \begin{split}
                    e_p*e_q(y)&=\int_{0}^y\frac{x^{p-1}}{(p-1)!}\cdot e_q(y-x)dx\\
                    &=\int_{0}^y\frac{x^{p-1}}{(p-1)!}\cdot\frac{(y-x)^{q-1}}{(q-1)!}dx\\
                    &=\frac{1}{(p-1)!(q-1)!}\cdot\int_{0}^y x^{p-1}(y-x)^{q-1}dx\\
                \end{split}
            \end{equation*}
            donde:
            \begin{equation*}
                \begin{split}
                    \int_{0}^y x^{p-1}(y-x)^{q-1}dx&=\int_0^yx^{p-1}\sum_{k=0}^{q-1}\binom{q-1}{k}(-1)^{q-1}x^k(-y)^{q-1-k}dx\\
                    &=(-1)^{q-1}\int_0^y\sum_{k=0}^{q-1}\binom{q-1}{k}x^{p+k-1}(-y)^{q-1-k}dx\\
                    &=(-1)^{q-1}\sum_{k=0}^{q-1}\binom{q-1}{k}(-y)^{q-1-k}\int_0^yx^{p+k-1}dx\\
                    &=(-1)^{q-1}\sum_{k=0}^{q-1}\binom{q-1}{k}(-y)^{q-1-k}\left[\frac{x^{p+k}}{p+k} \right]_0^y \\
                    &=(-1)^{q-1}\sum_{k=0}^{q-1}\binom{q-1}{k}(-y)^{q-1-k}\frac{y^{p+k}}{p+k} \\
                    &=(-1)^{2q-2}\sum_{k=0}^{q-1}\binom{q-1}{k}\frac{(-1)^ky^{p+q-1}}{p+k} \\
                    &=y^{p+q-1}\sum_{k=0}^{q-1}\binom{q-1}{k}\frac{(-1)^k}{p+k} \\
                \end{split}
            \end{equation*}
            veamos que:
            \begin{equation*}
                \begin{split}
                    \frac{1}{(p-1)!(q-1)!}\sum_{k=0}^{q-1}\binom{q-1}{k}\frac{(-1)^k}{p+k}
                    &=\frac{1}{(p-1)!(q-1)!}\sum_{k=0}^{q-1}\frac{(q-1)!}{k!(q-1-k)!}\cdot\frac{(-1)^k}{p+k}\\
                    &=\frac{1}{(p-1)!(q-1)!}\sum_{k=0}^{q-1}\frac{(-1)^k(q-1)!}{k!(q-1-k)!(p+k)} \\
                    &=\frac{1}{(p-1)!}\sum_{k=0}^{q-1}\frac{(-1)^k}{k!(q-1-k)!(p+k)} \\
                \end{split}
            \end{equation*}
        \end{itemize}
    \end{proof}

    \begin{excer}
        La integral fraccional de orden $\alpha>0$ sobre un intervalo $[a,x]$ de una función medible $f$ se define como:
        \begin{equation*}
            I_a^\alpha[f](x)=\frac{1}{\Gamma(\alpha)}\int_a^x\frac{f(t)}{(x-t)^{1-\alpha}}dt
        \end{equation*}
        para toda $x\geq a$ tal que la integral exista.
        \begin{enumerate}
            \item Fije $a,b\in\mathbb{R}$, $a<b$. Para cada $\alpha>0$ se define
            \begin{equation*}
                g_\alpha(x)=\frac{1}{\Gamma(\alpha)}\frac{1}{x^{1-\alpha}}\chi_{]0,b-a[}(x),\quad\forall x\in\mathbb{R}
            \end{equation*}
            \textbf{Pruebe} que si $f\in\mathcal{C}([a,b],\mathbb{K})$, entonces existe la convolución $\widetilde{f}*g_\alpha$. \textbf{Calcule} $\widetilde{f}*g_\alpha$.

            \item \textbf{Calcule} $I_0^{1/2}[t](x)$ y $I_0^{1/2}[I_0^{1/2}[t]](x)$. ¿\textbf{Conclusión}? Justifique.
        \end{enumerate}
    \end{excer}

    \begin{proof}
        
    \end{proof}

    \begin{excer}
        Para todo $p>0$ se define:
        \begin{equation*}
            f_p(t)=\left\{\begin{array}{lcr}
                t^{p-1}e^{-t} & \textup{ si }& t>0\\
                0 & \textup{ si } & t\leq 0\\
            \end{array}
            \right.
        \end{equation*}
        Calculando de dos modos distintos la integral $\int_{-\infty}^\infty f_p*f_q$ con $p,q>0$, \textbf{pruebe} la fórmula
        \begin{equation*}
            B(p,q)=\frac{\Gamma(p)\Gamma(q)}{\Gamma(p+q)},
        \end{equation*}
        donde $B(p,q)$ es la función beta y $\Gamma(q)$ es la función gama.
    \end{excer}
    
    \begin{proof}
        
    \end{proof}

    \begin{excer}
        Sea $\cf{f}{\mathbb{R}}{\mathbb{K}}$ una función localmente integrable en $\mathbb{R}$. Defina para todo $h>0$, la función
        \begin{equation*}
            J_hf=f*\left(\frac{1}{h}\chi_{]-h,0[}\right)
        \end{equation*}
        \begin{enumerate}
            \item \textbf{Muestre} que, $\forall x\in\mathbb{R}$,
            \begin{equation*}
                J_hf(x)=\frac{1}{h}\int_0^hf(x+y)dy
            \end{equation*}
            y que $J_hf$ es continua en $\mathbb{R}$.
            \item Si $f$ es integrable en $\mathbb{R}$, \textbf{pruebe} que también lo es $J_hf$ y que
            \begin{equation*}
                \int_\mathbb{R}J_hf=\int_{\mathbb{R}}f
            \end{equation*}
            \item Si $f$ es de clase $C^r$ en $\mathbb{R}$, \textbf{muestre} que también lo es $J_hf$ y que $\left(J_hf\right)^{(k)}=J_hf^{(k)}$ para $k=1,...,r$.
        \end{enumerate}
    \end{excer}

    \begin{sol}
        
    \end{sol}

    \begin{excer}
        Sea $\cf{f}{\mathbb{R}^n}{\mathbb{K}}$ una función localmente integrable en $\mathbb{R}^n$. Sea $B=\left\{x\in\mathbb{R}^n\Big|\|x\|\leq R \right\}$. Defina:
        \begin{equation*}
            \mathcal{M}_Rf=f*\frac{\chi_B}{\Vol{B}}
        \end{equation*}
        \begin{enumerate}
            \item \textbf{Muestre} que, para todo $x\in\mathbb{R}^n$:
            \begin{equation*}
                \mathcal{M}_Rf(x)=\frac{1}{\Vol{B}}\int_{\|x-y\|\leq R}f(y)dy
            \end{equation*}
            y que $\mathcal{M}_Rf$ es continua en $\mathbb{R}^n$.
            \item Si $f$ es integrable en $\mathbb{R}^n$, \textbf{pruebe} que también lo es $\mathcal{M}_Rf$ y que:
            \begin{equation*}
                \int_{\mathbb{R}^n}\mathcal{M}_Rf=\int_{\mathbb{R}^n}f
            \end{equation*}
            \item Si $f$ es de clase $C^r$ en $\mathbb{R}^n$, \textbf{muestre} que también lo es $\mathcal{M}_Rf$ y que $D(\mathcal{M}_Rf)=\mathcal{M}_R(Df)$ para todo opeardor $D=\partial_{\alpha_1}\cdots\partial_{\alpha_k}$, con $k\in\left\{1,...,r \right\}$.
        \end{enumerate}
    \end{excer}
    
    \begin{sol}
        
    \end{sol}

    \begin{excer}
        Haga lo siguiente:
        \begin{enumerate}
            \item Sean $f$ y $g$ dos funciones en $\mathcal{L}_1(\mathbb{R}^n,\mathbb{K})$. Sea $\lambda\in\mathbb{K}$ tal que $\N{1}{f}<1/\abs{\lambda}$. \textbf{Demuestre} que la ecuación
            \begin{equation*}
                x=\lambda x*f+g
            \end{equation*}
            admite una solución $x\in\mathcal{L}_1(\mathbb{R}^n,\mathbb{K})$ salvo equivalencias. \textbf{Muestre} que la solución puede ser representada en forma de una serie
            \begin{equation*}
                x=\sum_{\nu=0}^\infty \lambda^\nu g*\underbrace{f*\cdots*f}_{\nu\textup{-veces}}
            \end{equation*}
            que es convergente en el espacio de Banach $L_1(\mathbb{R}^n,\mathbb{K})$.

            \item Al suponer $f\in\mathcal{L}_1(\mathbb{R}^n,\mathbb{K})$ y $g\in\mathcal{L}_p(\mathbb{R}^n,\mathbb{K})$, estudie la misma ecuación con la incógnita $x$ en $\mathcal{L}_p(\mathbb{R}^n,\mathbb{K})$.
        \end{enumerate}
    \end{excer}

    \begin{proof}
        
    \end{proof}

    \begin{excer}
        Haga lo siguiente:
        \begin{enumerate}
            \item Sea $\cf{g}{\mathbb{R}^n}{\mathbb{K}}$ una función medible. \textbf{Muestre} que existe una función medible acotada $\cf{\alpha}{\mathbb{R}^n}{\mathbb{K}}$ tal que $\abs{g}=\alpha g$ en todo punto de $\mathbb{R}^n$.
            
            \textit{Sugerencia.} Intente con la función $\frac{g+\chi_S}{\abs{g+\chi_S}}$ donde $S=\left\{x\in\mathbb{R}^n\Big|g(x)=0 \right\}$.

            \item Sean $1<p<\infty$ y $g\in\mathcal{L}_{p^*}(\mathbb{R}^n,\mathbb{K})$. Defina $\cf{\phi_g}{\mathcal{L}_p(\mathbb{R}^n,\mathbb{K})}{\mathbb{K}}$ como:
            \begin{equation*}
                \phi_g(f)=\int_{\mathbb{R}^n}fg,\quad\forall f\in\mathcal{L}_p(\mathbb{R}^n,\mathbb{K})
            \end{equation*}
            \textbf{Pruebe} que $\phi_g$ es una aplicación lineal continua sobre $L_p(\mathbb{R}^n,\mathbb{K})$ y que $\|\phi_g\|=\N{p^*}{g}$.

            Así pues, la aplicación $g\mapsto\phi_g$ es una isometría de $\mathcal{L}_{p^*}(\mathbb{R}^n,\mathbb{K})$ en $\mathcal{L}_{p^*}(\mathbb{R}^n,\mathbb{K})$ (dicha isometría también es suprayectiva, pero este hecho más profundo no se pide probar aquí).

            \textit{Sugerencia.} Para probar la desigualdad $\N{p^*}{g}\leq\|\phi_g\|$ considere la función $f=\alpha\abs{g}^{p^*-1}$, donde $\alpha$ es la función del inciso (i).

            \item Sea $\left\{\rho_\nu \right\}_{\nu=1}^\infty$ una sucesión de Dirac en $\mathcal{L}_1(\mathbb{R}^n,\mathbb{K})$. Se quiere demostrar, sin usar la desigualdad de Jensen, que si $1\leq p<\infty$ y $f\in\mathcal{L}_p(\mathbb{R}^n,\mathbb{K})$, entonces
            \begin{equation*}
                \lim_{\nu\rightarrow\infty}\N{p}{f-\rho_\nu*f}=0
            \end{equation*}
            Defina $g_\nu=f-\rho_\nu*f$ y considere la aplicación lineal $\phi_{g_\nu}\in L_{p}(\mathbb{R}^n,\mathbb{K})^*$, donde
            \begin{equation*}
                \phi_{g_\nu}(h)=\int_{\mathbb{R}^n}hg_\nu,\quad\forall h\in \mathcal{L}_p(\mathbb{R}^n,\mathbb{K})
            \end{equation*}
            \textbf{Establezca} la desigualdad
            \begin{equation*}
                \abs{\phi_{g_\nu}}\leq\N{p^*}{h}\int_{\mathbb{R}^n}\rho_\nu(y)\N{p}{f_{-y}-f}dy
            \end{equation*}
            Sea $\varepsilon>0$. \textbf{Demuestre} que para $\nu$ suficientemente grande,
            \begin{equation*}
                \abs{\phi_{g_\nu}}\leq\N{p^*}{h}\varepsilon
            \end{equation*}
            Utilizando el inciso (ii) termine la demostración.
        \end{enumerate}
    \end{excer}

    \begin{proof}
        
    \end{proof}

    \begin{excer}
        Demuestre que el sistema de potencias enteras $\left\{x^\nu\Big|\nu\in\mathbb{N}^* \right\}$ es total en $L_p([a,b],\mathbb{C})$ para $p\in[1,\infty[$.

        \textit{Sugerencia.} Basta demostrarlo para $L_1([-\pi,\pi],\mathbb{C})$. El sistema trigonométrico es total en este espacio. Desarrolle $e^{ik\pi}$ en serie de potencias de Maclaurin.
    \end{excer}

    \begin{proof}
        
    \end{proof}

    \begin{excer}
        Demuestre que el sistema de potencias enteras $\left\{x^\nu\Big|\nu\in\mathbb{N}^* \right\}$ es completo en $L_p([a,b],\mathbb{C})$ para $p\in[1,\infty[$.
    \end{excer}

    \begin{proof}
        
    \end{proof}

    \begin{excer}
        Sean $E\subseteq\mathbb{R}^n$ un conjunto medible con medida finita y $1<p<\infty$. \textbf{Muestre} que si una familia de funciones $\left\{\varphi_i\Big|i\in I \right\}$ es completa en $L_p(E,\mathbb{K})$, entonces dicha familia es total en $L_{p^*}(E,\mathbb{K})$.

        \textit{Sugerencia.} Sea $f\in\mathcal{L}_{p^*}(\mathbb{R}^n,\mathbb{K})$. Se supone que $\int_Ef\varphi_i=0$ para toda $i\in I$. Sea $\alpha$ una función medible acotada tal que $\abs{f}=\alpha f$. Por hipótesis existe una sucesión de funciones $\left\{\psi_\nu \right\}_{\nu=1}^\infty$ en $\mathcal{L}(\left\{\varphi_i\Big|i\in I \right\})$ tal que $\lim_{\nu\rightarrow\infty}\N{p}{\alpha-\psi_\nu}=0$.
    \end{excer}

    \begin{proof}
        
    \end{proof}

\end{document}