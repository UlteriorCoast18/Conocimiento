\documentclass[12pt]{report}
\usepackage[spanish]{babel}
\usepackage[utf8]{inputenc}
\usepackage{amsmath}
\usepackage{amssymb}
\usepackage{amsthm}
\usepackage{graphics}
\usepackage{subfigure}
\usepackage{lipsum}
\usepackage{array}
\usepackage{multicol}
\usepackage{enumerate}
\usepackage[framemethod=TikZ]{mdframed}
\usepackage[a4paper, margin = 1.5cm]{geometry}

%En esta parte se hacen redefiniciones de algunos comandos para que resulte agradable el verlos%

\renewcommand{\theenumii}{\roman{enumii}}

\def\proof{\paragraph{Demostración:\\}}
\def\endproof{\hfill$\blacksquare$}

\def\sol{\paragraph{Solución:\\}}
\def\endsol{\hfill$\square$}

%En esta parte se definen los comandos a usar dentro del documento para enlistar%

\newtheoremstyle{largebreak}
  {}% use the default space above
  {}% use the default space below
  {\normalfont}% body font
  {}% indent (0pt)
  {\bfseries}% header font
  {}% punctuation
  {\newline}% break after header
  {}% header spec

\theoremstyle{largebreak}

\newmdtheoremenv[
    leftmargin=0em,
    rightmargin=0em,
    innertopmargin=-2pt,
    innerbottommargin=8pt,
    hidealllines = true,
    roundcorner = 5pt,
    backgroundcolor = gray!60!red!30
]{exa}{Ejemplo}[section]

\newmdtheoremenv[
    leftmargin=0em,
    rightmargin=0em,
    innertopmargin=-2pt,
    innerbottommargin=8pt,
    hidealllines = true,
    roundcorner = 5pt,
    backgroundcolor = gray!50!blue!30
]{obs}{Observación}[section]

\newmdtheoremenv[
    leftmargin=0em,
    rightmargin=0em,
    innertopmargin=-2pt,
    innerbottommargin=8pt,
    rightline = false,
    leftline = false
]{theor}{Teorema}[section]

\newmdtheoremenv[
    leftmargin=0em,
    rightmargin=0em,
    innertopmargin=-2pt,
    innerbottommargin=8pt,
    rightline = false,
    leftline = false
]{propo}{Proposición}[section]

\newmdtheoremenv[
    leftmargin=0em,
    rightmargin=0em,
    innertopmargin=-2pt,
    innerbottommargin=8pt,
    rightline = false,
    leftline = false
]{cor}{Corolario}[section]

\newmdtheoremenv[
    leftmargin=0em,
    rightmargin=0em,
    innertopmargin=-2pt,
    innerbottommargin=8pt,
    rightline = false,
    leftline = false
]{lema}{Lema}[section]

\newmdtheoremenv[
    leftmargin=0em,
    rightmargin=0em,
    innertopmargin=-2pt,
    innerbottommargin=8pt,
    roundcorner=5pt,
    backgroundcolor = gray!30,
    hidealllines = true
]{mydef}{Definición}[section]

\newmdtheoremenv[
    leftmargin=0em,
    rightmargin=0em,
    innertopmargin=-2pt,
    innerbottommargin=8pt,
    roundcorner=5pt
]{excer}{Ejercicio}[section]

%En esta parte se colocan comandos que definen la forma en la que se van a escribir ciertas funciones%

\newcommand\abs[1]{\ensuremath{\biglvert#1\bigrvert}}
\newcommand\divides{\ensuremath{\bigm|}}
\newcommand\cf[3]{\ensuremath{#1:#2\rightarrow#3}}
\newcommand{\Vol}[1]{\ensuremath{\textup{Vol}\left(#1\right)}}

%recuerda usar \clearpage para hacer un salto de página

\begin{document}
    \setlength{\parskip}{5pt} % Añade 5 puntos de espacio entre párrafos
    \setlength{\parindent}{12pt} % Pone la sangría como me gusta
    \title{Lista 2 de Ejercicios 
    
    Análisis Matemático IV
    }
    \author{Cristo Daniel Alvarado}
    \maketitle

    \tableofcontents %Con este comando se genera el índice general del libro%

    %\setcounter{chapter}{3} %En esta parte lo que se hace es cambiar la enumeración del capítulo%
    
    \chapter{Ejercicios Convolución}
    
    \setcounter{section}{1}

    \renewcommand{\theenumi}{\roman{enumi}}

    \begin{excer}
        Sean $\cf{f,g}{\mathbb{R}}{\mathbb{K}}$ funciones nulas en $]-\infty,0[$. Si existe $f*g(x)$, demuestre que:
        \begin{equation*}
            f*g(x)=\left\{
                \begin{array}{lcr}
                    \int_{0}^\infty f(y)g(x-y)dy & \textup{ si } & x\geq0\\
                    0 & \textup{ si }&x<0\\
                \end{array}
            \right.
        \end{equation*}
        En los casos siguientes $f$ y $g$ son nulas en $]-\infty.0[$ y sus valores en $[0,\infty[$ se indican abajo. Calcule $f*g$.
        \begin{enumerate}
            \item $f(x)=e^{-x}$ y $g(x)=\left\{
                    \begin{array}{lcr}
                        x & \textup{ si } & 0\leq x\leq 1 \\
                        0 & \textup{ si }&x>1\\
                    \end{array}
                \right.$
            \item $f(x)=g(x)=e^{-x}$.
            \item $f(x)=\left\{
                \begin{array}{lcr}
                    1 & \textup{ si } & 0\leq x\leq 1 \\
                    0 & \textup{ si }&x>1\\
                \end{array}
            \right.$ y $g(x)=\left\{
                \begin{array}{lcr}
                    x & \textup{ si } & 0\leq x\leq 1 \\
                    0 & \textup{ si }&x>1\\
                \end{array}
            \right.$
            \item $f(x)=g(x)=\left\{
                \begin{array}{lcr}
                    x & \textup{ si } & 0\leq x\leq 1 \\
                    0 & \textup{ si }&x>1\\
                \end{array}
            \right.$
        \end{enumerate}
    \end{excer}

    \begin{sol}
        
    \end{sol}
    
    \begin{excer}
        Haga lo siguiente:
        \begin{enumerate}
            \item Para toda $m\in\mathbb{N}$ se define $\cf{e_m}{\mathbb{R}}{\mathbb{R}}$ como:
            \begin{equation*}
                e_m(x)=\left\{
                    \begin{array}{lcr}
                        \frac{x^{m-1}}{(m-1)!} & \textup{ si } & x\geq 0 \\
                        0 & \textup{ si }&x<0\\
                    \end{array}
                \right.
            \end{equation*}
            \textbf{Pruebe} que
            \begin{equation*}
                e_p*e_q=e_{p+q}
            \end{equation*}
            \item Sea $\cf{f}{\mathbb{R}}{\mathbb{K}}$ integrable en todo intervalo acotado tal que $f(x)=0$ para todo $x\leq a$. \textbf{Muestre} que
            \begin{equation*}
                e_m*f(x)=\left\{
                    \begin{array}{lcr}
                        \int_{a}^x\frac{(x-y)^{m-1}}{(m-1)!}f(y)dy & \textup{ si } & x\geq 0 \\
                        0 & \textup{ si }&x<0\\
                    \end{array}
                \right.
            \end{equation*}
            \item \textbf{Deduzca} que para $x\geq a$ se cumple la siguiente \textbf{fórmula de Cauchy para la $n$-ésima integral indefinida}
            \begin{equation*}
                \int_{a}^xdx_{ m-1}\int_a^{ x_{m-1}}dx_{m-2}\cdots\int_{a}^{x_{2}} dx_{1}\int_a^{ x_1}f(x_0)dx_0=\int_a^x\frac{(x-y)^{m-1}}{(m-1)!}f(y)dy
            \end{equation*}
        \end{enumerate}
    \end{excer}

    \begin{proof}
        
    \end{proof}

    \begin{excer}
        La integral fraccional de orden $\alpha>0$ sobre un intervalo $[a,x]$ de una función medible $f$ se define como:
        \begin{equation*}
            I_a^\alpha[f](x)=\frac{1}{\Gamma(\alpha)}\int_a^x\frac{f(t)}{(x-t)^{1-\alpha}}dt
        \end{equation*}
        para toda $x\geq a$ tal que la integral exista.
        \begin{enumerate}
            \item Fije $a,b\in\mathbb{R}$, $a<b$. Para cada $\alpha>0$ se define
            \begin{equation*}
                g_\alpha(x)=\frac{1}{\Gamma(\alpha)}\frac{1}{x^{1-\alpha}}\chi_{]0,b-a[}(x),\quad\forall x\in\mathbb{R}
            \end{equation*}
            \textbf{Pruebe} que si $f\in\mathcal{C}([a,b],\mathbb{K})$, entonces existe la convolución $\widetilde{f}*g_\alpha$. \textbf{Calcule} $\widetilde{f}*g_\alpha$.

            \item \textbf{Calcule} $I_0^{1/2}[t](x)$ y $I_0^{1/2}[I_0^{1/2}[t]](x)$. ¿\textbf{Conclusión}? Justifique.
        \end{enumerate}
    \end{excer}

    \begin{proof}
        
    \end{proof}

    \begin{excer}
        Para todo $p>0$ se define:
        \begin{equation*}
            f_p(t)=\left\{\begin{array}{lcr}
                t^{p-1}e^{-t} & \textup{ si }& t>0\\
                0 & \textup{ si } & t\leq 0\\
            \end{array}
            \right.
        \end{equation*}
        Calculando de dos modos distintos la integral $\int_{-\infty}^\infty f_p*f_q$ con $p,q>0$, \textbf{pruebe} la fórmula
        \begin{equation*}
            B(p,q)=\frac{\Gamma(p)\Gamma(q)}{\Gamma(p+q)},
        \end{equation*}
        donde $B(p,q)$ es la función beta y $\Gamma(q)$ es la función gama.
    \end{excer}
    
    \begin{proof}
        
    \end{proof}

    \begin{excer}
        Sea $\cf{f}{\mathbb{R}}{\mathbb{K}}$ una función localmente integrable en $\mathbb{R}$. Defina para todo $h>0$, la función
        \begin{equation*}
            J_hf=f*\left(\frac{1}{h}\chi_{]-h,0[}\right)
        \end{equation*}
        \begin{enumerate}
            \item \textbf{Muestre} que, $\forall x\in\mathbb{R}$,
            \begin{equation*}
                J_hf(x)=\frac{1}{h}\int_0^hf(x+y)dy
            \end{equation*}
            y que $J_hf$ es continua en $\mathbb{R}$.
            \item Si $f$ es integrable en $\mathbb{R}$, \textbf{pruebe} que también lo es $J_hf$ y que
            \begin{equation*}
                \int_\mathbb{R}J_hf=\int_{\mathbb{R}}f
            \end{equation*}
            \item Si $f$ es de clase $C^r$ en $\mathbb{R}$, \textbf{muestre} que también lo es $J_hf$ y que $\left(J_hf\right)^{(k)}=J_hf^{(k)}$ para $k=1,...,r$.
        \end{enumerate}
    \end{excer}

    \begin{sol}
        
    \end{sol}

    \begin{excer}
        Sea $\cf{f}{\mathbb{R}^n}{\mathbb{K}}$ una función localmente integrable en $\mathbb{R}^n$. Sea $B=\left\{x\in\mathbb{R}^n\Big|\|x\|\leq R \right\}$. Defina:
        \begin{equation*}
            \mathcal{M}_Rf=f*\frac{\chi_B}{\Vol{B}}
        \end{equation*}
        \begin{enumerate}
            \item \textbf{Muestre} que, para todo $x\in\mathbb{R}^n$:
            \begin{equation*}
                \mathcal{M}_Rf(x)=\frac{1}{\Vol{B}}\int_{\|x-y\|\leq R}f(y)dy
            \end{equation*}
            y que $\mathcal{M}_Rf$ es continua en $\mathbb{R}^n$.
            \item Si $f$ es integrable en $\mathbb{R}^n$, \textbf{pruebe} que también lo es $\mathcal{M}_Rf$ y que:
            \begin{equation*}
                \int_{\mathbb{R}^n}\mathcal{M}_Rf=\int_{\mathbb{R}^n}f
            \end{equation*}
            \item Si $f$ es de clase $C^r$ en $\mathbb{R}^n$, \textbf{muestre} que también lo es $\mathcal{M}_Rf$ y que $D(\mathcal{M}_Rf)=\mathcal{M}_R(Df)$ para todo opeardor $D=\partial_{\alpha_1}\cdots\partial_{\alpha_k}$, con $k\in\left\{1,...,r \right\}$.
        \end{enumerate}
    \end{excer}

\end{document}