\documentclass[12pt]{report}
\usepackage[spanish]{babel}
\usepackage[utf8]{inputenc}
\usepackage{amsmath}
\usepackage{amssymb}
\usepackage{amsthm}
\usepackage{graphics}
\usepackage{subfigure}
\usepackage{lipsum}
\usepackage{array}
\usepackage{multicol}
\usepackage{enumerate}
\usepackage[framemethod=TikZ]{mdframed}
\usepackage[a4paper, margin = 1.5cm]{geometry}

%En esta parte se hacen redefiniciones de algunos comandos para que resulte agradable el verlos%

\renewcommand{\theenumii}{\roman{enumii}}

\def\proof{\paragraph{Demostración:\\}}
\def\endproof{\hfill$\blacksquare$}

\def\sol{\paragraph{Solución:\\}}
\def\endsol{\hfill$\square$}

%En esta parte se definen los comandos a usar dentro del documento para enlistar%

\newtheoremstyle{largebreak}
  {}% use the default space above
  {}% use the default space below
  {\normalfont}% body font
  {}% indent (0pt)
  {\bfseries}% header font
  {}% punctuation
  {\newline}% break after header
  {}% header spec

\theoremstyle{largebreak}

\newmdtheoremenv[
    leftmargin=0em,
    rightmargin=0em,
    innertopmargin=-2pt,
    innerbottommargin=8pt,
    hidealllines = true,
    roundcorner = 5pt,
    backgroundcolor = gray!60!red!30
]{exa}{Ejemplo}[section]

\newmdtheoremenv[
    leftmargin=0em,
    rightmargin=0em,
    innertopmargin=-2pt,
    innerbottommargin=8pt,
    hidealllines = true,
    roundcorner = 5pt,
    backgroundcolor = gray!50!blue!30
]{obs}{Observación}[section]

\newmdtheoremenv[
    leftmargin=0em,
    rightmargin=0em,
    innertopmargin=-2pt,
    innerbottommargin=8pt,
    rightline = false,
    leftline = false
]{theor}{Teorema}[section]

\newmdtheoremenv[
    leftmargin=0em,
    rightmargin=0em,
    innertopmargin=-2pt,
    innerbottommargin=8pt,
    rightline = false,
    leftline = false
]{propo}{Proposición}[section]

\newmdtheoremenv[
    leftmargin=0em,
    rightmargin=0em,
    innertopmargin=-2pt,
    innerbottommargin=8pt,
    rightline = false,
    leftline = false
]{cor}{Corolario}[section]

\newmdtheoremenv[
    leftmargin=0em,
    rightmargin=0em,
    innertopmargin=-2pt,
    innerbottommargin=8pt,
    rightline = false,
    leftline = false
]{lema}{Lema}[section]

\newmdtheoremenv[
    leftmargin=0em,
    rightmargin=0em,
    innertopmargin=-2pt,
    innerbottommargin=8pt,
    roundcorner=5pt,
    backgroundcolor = gray!30,
    hidealllines = true
]{mydef}{Definición}[section]

\newmdtheoremenv[
    leftmargin=0em,
    rightmargin=0em,
    innertopmargin=-2pt,
    innerbottommargin=8pt,
    roundcorner=5pt
]{excer}{Ejercicio}[section]

%En esta parte se colocan comandos que definen la forma en la que se van a escribir ciertas funciones%

\renewcommand{\leq}{\ensuremath{\leqslant}}
\renewcommand{\geq}{\ensuremath{\geqslant}}

\newcommand\abs[1]{\ensuremath{\left|#1\right|}}
\newcommand\divides{\ensuremath{\bigm|}}
\newcommand\cf[3]{\ensuremath{#1:#2\rightarrow#3}}
\newcommand\norm[1]{\ensuremath{\|#1\|}}
\newcommand\ora[1]{\ensuremath{\vec{#1}}}
\newcommand\pint[2]{\ensuremath{\langle#1| #2\rangle}}
\newcommand\conj[1]{\ensuremath{\overline{#1}}}
\newcommand{\N}[2]{\ensuremath{\mathcal{N}_{#1}\left(#2\right)}}
\newcommand{\natint}[1]{\ensuremath{\left[\!\left[#1\right]\!\right]}}
\newcommand{\fou}[1]{\ensuremath{\mathcal{F}#1}}
\newcommand{\diag}[1]{\ensuremath{\left(#1\right)}}
\newcommand{\contradiction}{\ensuremath{\#_c}}
\newcommand{\Sgn}[1]{\ensuremath{\textup{Sgn}\left(#1\right)}}

%recuerda usar \clearpage para hacer un salto de página

\begin{document}
    \setlength{\parskip}{5pt} % Añade 5 puntos de espacio entre párrafos
    \setlength{\parindent}{12pt} % Pone la sangría como me gusta
    \title{Lista Ejercicios Análisis Matemático IV}
    \author{Cristo Daniel Alvarado}
    \maketitle

    \tableofcontents %Con este comando se genera el índice general del libro%

    %\setcounter{chapter}{3} %En est a parte lo que se hace es cambiar la enumeración del capítulo%
    
    \chapter{Lista 4}
    
    \renewcommand{\theenumi}{\textbf{\roman{enumi}}}

    \begin{excer}
        Haga lo siguiente:
        \begin{enumerate}
            \item Sea $f\in\mathcal{L}_1(\mathbb{R}^n,\mathbb{C})$. Defina $\cf{P}{\mathbb{R}^n}{\mathbb{R}}$ como:
            \begin{equation*}
                P(x_1,...,x_n)=e^{ -\sum_{ k=1}^n \abs{x_k}},\quad\forall x\in\mathbb{R}^n
            \end{equation*}
            Fije $\nu\in\mathbb{N}$, \textbf{demuestre} la fórmula:
            \begin{equation*}
                \int_{\mathbb{R}^n}\fou{f}(x)P\left(\frac{x}{\nu}\right)\:dx=(2\nu)^n\int_{\mathbb{R}^n}\frac{f(x_1,...,x_n)}{(1+\nu^2x_1^2)\cdots(1+\nu^2x_n^2)}\:dx_1\cdots dx_n
            \end{equation*}
            \item \textbf{Deduzca} que si $f\in\mathcal{L}_1(\mathbb{R}^n,\mathbb{C})\cap\mathcal{L}_{\infty}(\mathbb{R}^n,\mathbb{C})$ y $\fou{f}\geq0$, entonces $\fou{f}\in\mathcal{L}_1(\mathbb{R}^n,\mathbb{C})$.
            
            \textit{Sugerencia.} Aplique el teorema de Beppo-Levi.
        \end{enumerate}
    \end{excer}

    \begin{proof}
        De (i): Defina $g(x)=P\left(\frac{x}{\nu}\right)$, para todo $x\in\mathbb{R}^n$. Veamos que $g\in\mathcal{L}_1(\mathbb{R}^n,\mathbb{C})$. En efecto, se tiene que
        \begin{equation*}
            \begin{split}
                \int_{\mathbb{R}^n}g\left(x\right)\:dx&=\int_{\mathbb{R}^n}P\left(\frac{x}{\nu}\right)\:dx\\
                &=\int_{\mathbb{R}^n}P\left(\frac{x_1}{\nu},...,\frac{x_1}{\nu}\right)\:dx_1\cdots dx_n \\
                &=\int_{\mathbb{R}^n}e^{ -\sum_{ k=1}^n\abs{\frac{x_k}{\nu}}}\:dx_1\cdots dx_n\\
                &=\int_{\mathbb{R}^n}e^{ -\frac{1}{\nu}\sum_{ k=1}^n\abs{x_k}}\:dx_1\cdots dx_n\\
                &=\int_{\mathbb{R}^n}e^{-\frac{\abs{x_1}}{\nu}}\cdot...\cdot e^{-\frac{\abs{x_n}}{\nu}}\:dx_1\cdots dx_n\\
                &=\underbrace{\left(\int_{\mathbb{R}}e^{-\frac{\abs{x_1}}{\nu}}\:dx_1\right)\cdots\left(\int_{\mathbb{R}}e^{-\frac{\abs{x_n}}{\nu}}\:dx_n\right)}_{ n\textup{-veces}}\\
                &=\left(\int_{\mathbb{R}}e^{-\frac{\abs{t}}{\nu}}\:dt\right)^n\\
                &<\infty
            \end{split}
        \end{equation*}
        Usando Fubini para funciones medibles no negativas. Por tanto, por el Teorema de transferencia se sigue que
        \begin{equation*}
            \begin{split}
                \int_{\mathbb{R}^n}\fou{f}(x)P\left(\frac{x}{\nu}\right)\:dx&=\int_{\mathbb{R}^n}\fou{f}(x)g(x)\:dx\\
                &=\int_{\mathbb{R}^n}f(x)\fou{g}(x)\:dx
            \end{split}
        \end{equation*}
        Calculemos $\fou{g}(x)$. Como $g(x)=P\left(\frac{x}{\nu}\right)$ para todo $x\in\mathbb{R}^n$, entonces
        \begin{equation*}
            \begin{split}
                \fou{g}(x)=\nu^n\fou{P}(\nu x)
            \end{split}
        \end{equation*}
        (por una proposición), donde
        \begin{equation*}
            \begin{split}
                \fou{P}(x)&=\int_{\mathbb{R}^n}e^{-i\pint{x}{y}}P(y)\:dy\\
                &=\int_{\mathbb{R}^n}e^{-i\sum_{ k=1}^n x_ky_k}e^{ -\sum_{ k=1}^n \abs{y_k}}\:dy\\
                &=\int_{\mathbb{R}^n}e^{-\sum_{ k=1}^n(\abs{y_k}+ix_ky_k)}\:dy\\
                &=\int_{\mathbb{R}^n}e^{-\abs{y_1}-ix_1y_1}\cdot...\cdot e^{-\abs{y_n}-ix_ny_n}\:dy_1\cdots dy_n\\
                &=\underbrace{\left(\int_{-\infty}^{\infty}e^{-\abs{y_1}-ix_1y_1}\:dy_1 \right)\cdots\left(\int_{-\infty}^{\infty}e^{-\abs{y_n}-ix_ny_n}\:dy_n \right)}_{ n\textup{-veces}}\\
                &=\underbrace{\left(\int_{-\infty}^{\infty}e^{-\abs{t}-ix_1t}\:dt \right)\cdots\left(\int_{-\infty}^{\infty}e^{-\abs{t}-ix_nt}\:dt \right)}_{ n\textup{-veces}}\\
                &=\fou{h}(x_1)\cdots\fou{h}(x_n)\\
            \end{split}
        \end{equation*}
        donde $\cf{h}{\mathbb{R}}{\mathbb{R}}$ es la función tal que $t\mapsto e^{-\abs{t}}$ y, se sabe que
        \begin{equation*}
            \fou{h}(x)=\frac{2}{1+x^2}
        \end{equation*}
        Por tanto,
        \begin{equation*}
            \begin{split}
                \fou{P}(x)&=\frac{2^n}{(1+x_1^2)\cdots(1+x_n^2)}\\
                \Rightarrow \fou{P}(\nu x)&=\frac{2^n}{(1+\nu^2x_1^2)\cdots(1+\nu^2x_n^2)}\\
            \end{split}
        \end{equation*}
        Se sigue que
        \begin{equation*}
            \begin{split}
                \int_{\mathbb{R}^n}\fou{f}(x)P\left(\frac{x}{\nu}\right)\:dx&=\nu^n\int_{\mathbb{R}^n}f(x)\fou{P}(\nu x)\:dx\\
                &=\nu^n\int_{\mathbb{R}^n}\frac{2^nf(x_1,...,x_n)}{(1+\nu^2x_1^2)\cdots(1+\nu^2x_n^2)}\:dx_1\cdots dx_n\\
                &=(2\nu)^n\int_{\mathbb{R}^n}\frac{f(x_1,...,x_n)}{(1+\nu^2x_1^2)\cdots(1+\nu^2x_n^2)}\:dx_1\cdots dx_n\\
            \end{split}
        \end{equation*}

        De (ii): Para cada $\nu\in\mathbb{N}$ defina la función $\cf{g_\nu}{\mathbb{R}^n}{\mathbb{C}}$ como sigue:
        \begin{equation*}
            g_\nu(x)=\fou{f}(x)P\left(\frac{x}{\nu}\right),\quad\forall x\in\mathbb{R}^n
        \end{equation*}
        Esta es una sucesión creciente de funciones en $\mathcal{L}_1(\mathbb{R}^n,\mathbb{C})$, pues si $\nu\in\mathbb{N}$:
        \begin{equation*}
            \begin{split}
                \frac{1}{\nu+1}\sum_{ k=1}^n\abs{x_k}&\leq\frac{1}{\nu}\sum_{ k=1}^n\abs{x_k},\quad\forall x\in\mathbb{R}^n\\
                \Rightarrow -\frac{1}{\nu}\sum_{ k=1}^n\abs{x_k}&\leq-\frac{1}{\nu+1}\sum_{ k=1}^n\abs{x_k},\quad\forall x\in\mathbb{R}^n\\
                \Rightarrow e^{\frac{1}{\nu}\sum_{ k=1}^n\abs{x_k}}&\leq e^{-\frac{1}{\nu+1}\sum_{ k=1}^n\abs{x_k}},\quad\forall x\in\mathbb{R}^n\\
                \Rightarrow P\left(\frac{x}{\nu}\right)&\leq P\left(\frac{x}{\nu+1}\right),\quad\forall x\in\mathbb{R}^n \\
                \Rightarrow  \fou{f}(x)P\left(\frac{x}{\nu}\right)&\leq \fou{f}(x)P\left(\frac{x}{\nu+1}\right),\quad\forall x\in\mathbb{R}^n \\
                \Rightarrow  g_\nu&\leq g_{\nu+1} \\
            \end{split}
        \end{equation*}
        pues, $\fou{f}\geq0$.
        Además, como $f\in\mathcal{L}_{\infty}(\mathbb{R}^n,\mathbb{C})$, entonces
        \begin{equation*}
            f\leq\N{\infty}{f}\textup{ c.t.p. en }\mathbb{R}^n
        \end{equation*}
        luego,
        \begin{equation*}
            \begin{split}
                \int_{\mathbb{R}^n}g_\nu(x)\:dx&=\int_{\mathbb{R}^n}\fou{f}(x)P\left(\frac{x}{\nu}\right)\:dx\\
                &=(2\nu)^n\int_{\mathbb{R}^n}\frac{f(x_1,...,x_n)}{(1+\nu^2x_1^2)\cdots(1+\nu^2x_n^2)}\:dx_1\cdots dx_n\\
            \end{split}
        \end{equation*}
        Hagamos el cambio de variable $(y_1,...,y_n)=(\frac{x_1}{\nu},...,\frac{x_n}{\nu})$, se tiene que
        \begin{equation*}
            \begin{split}
                (2\nu)^n\int_{\mathbb{R}^n}\frac{f(x_1,...,x_n)}{(1+\nu^2x_1^2)\cdots(1+\nu^2x_n^2)}\:dx_1\cdots dx_n&=(2\nu)^n\int_{\mathbb{R}^n}\frac{f(\nu y_1,...,\nu y_n)}{(1+y_1^2)\cdots(1+y_n^2)}\: \frac{dy_1\cdots dy_n}{\nu^n}\\
                &=2^n\int_{\mathbb{R}^n}\frac{f(\nu y_1,...,\nu y_n)}{(1+y_1^2)\cdots(1+y_n^2)}\:dy_1\cdots dy_n\\
                &=2^n\int_{\mathbb{R}^n}\frac{\N{\infty}{f}}{(1+y_1^2)\cdots(1+y_n^2)}\:dy_1\cdots dy_n\\
                &=2^n\N{\infty}{f}\int_{\mathbb{R}^n}\frac{dy_1\cdots dy_n}{(1+y_1^2)\cdots(1+y_n^2)}\\
                \Rightarrow \abs{\int_{\mathbb{R}^n}g_\nu(x)}&=2^n\N{\infty}{f}\int_{\mathbb{R}^n}\frac{dy_1\cdots dy_n}{(1+y_1^2)\cdots(1+y_n^2)}\\
            \end{split}
        \end{equation*}
        pues $\int_{\mathbb{R}^n}\frac{dy_1\cdots dy_n}{(1+y_1^2)\cdots(1+y_n^2)}<\infty$ y $\int_{\mathbb{R}^n}g_\nu(x)\geq0$. Por tanto, por Beppo-Levi se sigue que existe una función $g\in\mathcal{L}_1(\mathbb{R}^n,\mathbb{C})$ tal que
        \begin{equation*}
            \lim_{\nu\rightarrow\infty}g_\nu=g\textup{ c.t.p. en }\mathbb{R}^n
        \end{equation*}
        Pero, también se tiene que
        \begin{equation*}
            \begin{split}
                \lim_{\nu\rightarrow\infty}g_\nu(x)&=\lim_{\nu\rightarrow\infty}\fou{f}(x)P\left(\frac{x}{\nu}\right)\\
                &=\fou{f}(x)\lim_{\nu\rightarrow\infty}P\left(\frac{x}{\nu}\right)\\
                &=\fou{f}(x)P\left(0,...,0\right)\\
                &=\fou{f}(x),\quad\forall x\in\mathbb{R}^n \\
            \end{split}
        \end{equation*}
        Entonces $\fou{f}=g$ c.t.p. en $\mathbb{R}^n$, luego $\fou{f}\in\mathcal{L}_1(\mathbb{R}^n,\mathbb{C})$. Más aún,
        \begin{equation*}
            \begin{split}
                \int_{\mathbb{R}^n}\fou{f}(x)\:dx&=\lim_{\nu\rightarrow\infty}2^n\N{\infty}{f}\int_{\mathbb{R}^n}\frac{dy_1\cdots dy_n}{(1+y_1^2)\cdots(1+y_n^2)}\\
                &=2^n\N{\infty}{f}\int_{\mathbb{R}^n}\frac{dy_1\cdots dy_n}{(1+y_1^2)\cdots(1+y_n^2)}\\
            \end{split}
        \end{equation*}
    \end{proof}

    \begin{excer}[\textbf{Problema 2 Lista 6 Análisis Matemático II}]
        \textbf{Pruebe} que si $f\in\mathcal{L}_1(\mathbb{R}^n,\mathbb{C})$, entonces
        \begin{equation*}
            \abs{\int_{\mathbb{R}^n}f}=\int_{\mathbb{R}^n}\abs{f}
        \end{equation*}
        si y sólo si existe $\alpha\in\mathbb{R}$ fijo tal que $f=e^{i\alpha}\abs{f}$ c.t.p. en $\mathbb{R}^n$.

        \textit{Sugerencia.} Suponiendo que $\abs{\int_{\mathbb{R}^n}f}=\int_{\mathbb{R}^n}\abs{f}$, existe $\alpha\in\mathbb{R}$ tal que $\int_{\mathbb{R}^n}f=e^{i\alpha}\int_{\mathbb{R}^n}\abs{f}$. Escriba
        \begin{equation*}
            e^{ -i\alpha}f=g+ih
        \end{equation*}
        donde $g$ y $h$ son funciones reales.
    \end{excer}

    \begin{proof}
        
    \end{proof}

    \begin{excer}
        Sea $f\in\mathcal{L}_1(\mathbb{R}^n,\mathbb{C})$. Se supone que $f(x)>0$, para todo $x\in\mathbb{R}^n$. \textbf{Pruebe} que si $x\neq0$, entonces
        \begin{equation*}
            \fou{f}(0)>\abs{\fou{f}(x)}
        \end{equation*}
        \textit{Sugerencia.} Una vez que ha demostrado $\abs{\fou{f}(x)}\leq\fou{f}(0)$, para todo $x\in\mathbb{R}^n$, Para demostrar la desigualdad estricta para $x\neq0$ proceda por reducción al absurdo y use el Problema 2 de la Lista 6 de Análisis Matemático II.
    \end{excer}

    \begin{proof}
        Notemos que como $f(x)>0$ para todo $x\in\mathbb{R}$, se tiene en particular que $\cf{f}{\mathbb{R}^n}{\mathbb{R}}$.

        Sea $x\in\mathbb{R}^n$. Primero probaremos que
        \begin{equation*}
            \fou{f}(0)\geq\abs{\fou{f}(x)}
        \end{equation*}
        es decir:
        \begin{equation*}
            \begin{split}
                \int_{\mathbb{R}^n}e^{ -i\pint{0}{y}}f(y)\:dy&\geq\abs{\int_{\mathbb{R}^n}e^{ -i\pint{x}{y}}f(y)\:dy}\\
                \iff\int_{\mathbb{R}^n}f(y)\:dy&\geq\abs{\int_{\mathbb{R}^n}e^{ -i\pint{x}{y}}f(y)\:dy}\\
                \iff\int_{\mathbb{R}^n}\abs{f(y)}\:dy&\geq\abs{\int_{\mathbb{R}^n}e^{ -i\pint{x}{y}}f(y)\:dy}\\
            \end{split}
        \end{equation*}
        pues $f(x)>0$ para todo $x\in\mathbb{R}^n$. Veamos que
        \begin{equation}
            \begin{split}
                \abs{\int_{\mathbb{R}^n}e^{ -i\pint{x}{y}}f(y)\:dy}&\leq\int_{\mathbb{R}^n}\abs{e^{ -i\pint{x}{y}}f(y)}\:dy\\
                &=\int_{\mathbb{R}^n}\abs{e^{ -i\pint{x}{y}}}\abs{f(y)}\:dy\\
                &=\int_{\mathbb{R}^n}\abs{f(y)}\:dy\\
            \end{split}
        \end{equation}
        lo que prueba el resultado. Para la desigualdad estricta suponga que existe $x\in\mathbb{R}^n$ no cero tal que
        \begin{equation*}
            \fou{f}(x)=\fou{f}(0)
        \end{equation*}
        esto es
        \begin{equation*}
            \begin{split}
                \abs{\int_{\mathbb{R}^n}e^{ -i\pint{x}{y}}f(y)\:dy}&=\int_{\mathbb{R}^n}f(y)\:dy\\
                &=\int_{\mathbb{R}^n}\abs{f(y)}\:dy\\
                &=\int_{\mathbb{R}^n}\abs{e^{ -i\pint{x}{y}}f(y)}\:dy\\
            \end{split}
        \end{equation*}
        Por el ejercicio anterior existe $\alpha\in\mathbb{R}$ fijo tal que
        \begin{equation*}
            e^{ -i\pint{x}{y}}f(y)=e^{ i\alpha}\abs{f(y)}=e^{ i\alpha}f(y)
        \end{equation*}
        para casi todo $y\in\mathbb{R}^n$. En particular, tenemos que
        \begin{equation*}
            f(y)\left(e^{-i\pint{x}{y}}-e^{i\alpha}\right)=0
        \end{equation*}
        para casi todo $y\in\mathbb{R}^n$. Como $f(y)>0$ para todo $y\in\mathbb{R}^n$, entonces
        \begin{equation*}
            e^{-i\pint{x}{y}}-e^{i\alpha}=0
        \end{equation*}
        nuevamente para casi todo $y\in\mathbb{R}^n$. Como las dos funciones involucradas son continuas y coinciden c.t.p. en $\mathbb{R}^n$, debe tenerse pues que
        \begin{equation*}
            e^{-i\pint{x}{y}}=e^{i\alpha},\quad\forall y\in\mathbb{R}^n
        \end{equation*}
        lo cual ocurre si y sólo si
        \begin{equation*}
            e^{i(\pint{x}{y}+\alpha)}=1,\quad\forall y\in\mathbb{R}^n
        \end{equation*}
        es decir que
        \begin{equation*}
            \pint{x}{y}+\alpha=0,\quad\forall y\in\mathbb{R}^n
        \end{equation*}
        como $x\neq0$ en particular se tiene que
        \begin{equation*}
            \pint{x}{x}=-\alpha
        \end{equation*}
        y, además (tomando $y=2x$):
        \begin{equation*}
            2\pint{x}{x}=-\alpha
        \end{equation*}
        pero esto sólo puede suceder si $\pint{x}{x}=0$, es decir que $x=0$\contradiction. Luego entonces
        \begin{equation*}
            \abs{\fou{f}(x)}<\fou{f}(0)
        \end{equation*}
    \end{proof}

    \begin{excer}
        Haga lo siguiente:
        \begin{enumerate}
            \item Sean $a>0$ y $\lambda\in\mathbb{R}$. \textbf{Pruebe} que la función $x\mapsto (\cos \lambda x)/(x^2+a^2)$ es integrable en $[0,\infty[$. \textbf{Muestre} que si $\lambda\neq0$, la función $x\mapsto (x\sin \lambda x)/(x^2+a^2)$ no es integrable en $[0,\infty[$, pero existe la integral impropia
            \begin{equation*}
                \int_0^{\rightarrow\infty}\frac{x\sin\lambda x}{x^2+a^2}\:dx
            \end{equation*}
            \textit{Sugerencia.} Muestre que
            \begin{equation*}
                \abs{\frac{x\sin\lambda x}{x^2+a^2}}\underset{x\rightarrow\infty}{\sim}\abs{\frac{\sin\lambda x}{x}}
            \end{equation*}
            Para probar la existencia de la integral impropia use los criterios de Abel.
            \item Recuerde que la función $x\mapsto (2a)/(x^2+a^2)$ es la transformada de Fourier de la función $x\mapsto e^{-a\abs{x}}$. Usando el teorema de inversión de Fourier, \textbf{demuestre} que
            \begin{equation*}
                \int_0^{\infty}\frac{\cos\lambda x}{x^2+a^2}\:dx=\frac{\pi}{2a}e^{ -a\abs{\lambda}}
            \end{equation*}
            \item Usando el inciso (ii), calcule la integral impropia
            \begin{equation*}
                \int_0^{\rightarrow\infty}\frac{x\sin\lambda x}{x^2+a^2}\:dx
            \end{equation*}
            \textit{Sugerencia.} Para $\lambda\neq0$ defina
            \begin{equation*}
                \Phi(\lambda)=\int_0^{\rightarrow\infty}\frac{\cos\lambda x}{x^2+a^2}\:dx
            \end{equation*}
            \textbf{Calcule} $\Phi'(\lambda)$ primero suponiendo $\lambda>\lambda_0$, donde $\lambda_0>0$ es arbitrario fijo, de forma análoga para $\lambda<0$ y finalmente para $\lambda=0$.
        \end{enumerate}
    \end{excer}

    \begin{proof}
        De (i): Para todo $x\in\mathbb{R}^n$ defina
        \begin{equation*}
            f(x)=\frac{\cos\lambda x}{x^2+a^2}
        \end{equation*}
        Afirmamos que $f$ es integrable en $[0,\infty[$. Para ello, veamos que
        \begin{equation*}
            \begin{split}
                \int_{0}^{\infty}\abs{f}&=\int_{0}^{\infty}\frac{\abs{\cos\lambda x}}{x^2+a^2}\:dx\\
                &\leq\int_{0}^\infty\frac{dx}{x^2+a^2}\\
            \end{split}
        \end{equation*}
        donde la función de la derecha es integrable en tal invervalo. Por tanto, $f\in\mathcal{L}_1(\mathbb{R},\mathbb{R})$.

        Sea ahora $\lambda\in\mathbb{R}$ tal que $\lambda\neq0$. Afirmamos que la función
        \begin{equation*}
            g(x)=\frac{x\sin\lambda x}{x^2+a^2}
        \end{equation*}
        para todo $x\in\mathbb{R}$ no es integrable en $[0,\infty[$. En efecto, veamos que
        \begin{equation*}
            \begin{split}
                \lim_{ x\rightarrow\infty}\frac{\abs{\frac{x\sin\lambda x}{x^2+a^2}}}{\abs{\frac{\sin\lambda x}{x}}}&=\lim_{ x\rightarrow\infty}\abs{\frac{x^2}{x^2+a^2}}\\
                &=\lim_{ x\rightarrow\infty}\abs{\frac{1}{1+\frac{a^2}{x^2}}}\\
                &=\frac{1}{1+0}\\
                &=1\\
            \end{split}
        \end{equation*}
        por tanto,
        \begin{equation*}
            \abs{\frac{x\sin\lambda x}{x^2+a^2}}\underset{x\rightarrow\infty}{\sim}\abs{\frac{\sin\lambda x}{x}}
        \end{equation*}
        Luego entonces, por la proposición 8.53 Análisis Matemático II:
        \begin{equation*}
            \abs{\frac{\sin\lambda x}{x}}\underset{x\rightarrow\infty}{=}O\left(\abs{\frac{x\sin\lambda x}{x^2+a^2}}\right)
        \end{equation*}
        donde la función $x\mapsto\abs{\frac{\sin\lambda x}{x}}$ no es integrable en $[0,\infty[$, luego tampoco puede serlo $g$ (siendo que ambas funciones son integrables en todo subconjunto acotado de $\mathbb{R}$).

        Veamos que si existe la integral impropia
        \begin{equation*}
            \begin{split}
                \int_{ 0}^{\rightarrow\infty}\frac{x\sin\lambda x}{x^2+a^2}
            \end{split}
        \end{equation*}
        En efecto, defina para cada $x\in[0,\infty[$ las funciones:
        \begin{equation*}
            G(x)=\int_{ 0}^x \sin\lambda x\:dx\quad\textup{y}\quad f(x)=\frac{x}{x^2+a^2}
        \end{equation*}
        se tienen dos cosas:
        \begin{itemize}
            \item $\lim_{x\rightarrow\infty}f(x)=0$ (es claro de la definición de $f$).
            \item $G$ es una función acotada en $[0,\infty[$, pues para cada $x\in[0,\infty[$:
            \begin{equation*}
                \begin{split}
                    \abs{G(x)}&=\abs{\int_{ 0}^x \sin\lambda x\:dx},\textup{ haciendo el cambio de variable }u=\lambda x\\
                    &=\abs{\frac{1}{\lambda}\cdot \int_{ 0}^{\lambda x}\sin u\:du}\\
                    &=\abs{\frac{1}{\lambda}\cdot\left[-\cos u\Big|_{ 0}^{\lambda x} \right]}\\
                    &=\abs{\frac{1}{\lambda}\cdot\left[-\cos \lambda x+\cos 0\right]}\\
                    &\leq\frac{2}{\abs{\lambda}}
                \end{split}
            \end{equation*}
        \end{itemize}
        Por tanto, del primer criterio de Abel se sigue que la integral impropia
        \begin{equation*}
            \int_{ 0}^{\rightarrow\infty}\frac{x\sin\lambda x}{x^2+a^2}
        \end{equation*}
        es convergente.

        De (ii): Sea $\cf{h}{\mathbb{R}}{\mathbb{R}}$ la función dada por
        \begin{equation*}
            \begin{split}
                h(x)=e^{-a\abs{x}},\quad\forall x\in\mathbb{R}
            \end{split}
        \end{equation*}
        se sabe por un ejercicio de las notas que
        \begin{equation*}
            \fou{h}(x)=\frac{2a}{x^2+a^2}
        \end{equation*}
        la función $h$ cumple la condición de Dini en todo punto $\lambda\in\mathbb{R}\neq0$ (también lo hace en cero, pero no es relevante), por lo cual se tiene que
        \begin{equation*}
            h(\lambda)=\lim_{ R\rightarrow\infty}\frac{1}{2\pi}\int_{ -R}^R e^{ i\lambda x}\fou{h}(x)\:dx
        \end{equation*}
        es decir,
        \begin{equation*}
            \begin{split}
                e^{ -a\abs{\lambda}}&=\lim_{ R\rightarrow\infty}\frac{1}{2\pi}\int_{ -R}^R e^{ i\lambda x}\fou{h}(x)\:dx\\
                &=\frac{2a}{2\pi}\lim_{ R\rightarrow\infty}\left[\int_{ -R}^R\frac{\cos\lambda x}{x^2+a^2}\:dx+i\int_{ -R}^R\frac{\sin\lambda x}{x^2+a^2}\:dx \right] \\
                &=\frac{a}{\pi}\lim_{ R\rightarrow\infty}\left[\int_{ -R}^0\frac{\cos\lambda x}{x^2+a^2}\:dx+\int_{0}^R\frac{\cos\lambda x}{x^2+a^2}\:dx+i\int_{ -R}^0\frac{\sin\lambda x}{x^2+a^2}\:dx+i\int_{0}^R\frac{\sin\lambda x}{x^2+a^2}\:dx \right]\\
                &=\frac{a}{\pi}\lim_{ R\rightarrow\infty}\left[\int_{0}^R\frac{\cos\lambda x}{x^2+a^2}\:dx+\int_{0}^R\frac{\cos\lambda x}{x^2+a^2}\:dx-i\int_{0}^R\frac{\sin\lambda x}{x^2+a^2}\:dx+i\int_{0}^R\frac{\sin\lambda x}{x^2+a^2}\:dx\right]\\
                &=\frac{2a}{\pi}\lim_{ R\rightarrow\infty}\int_{0}^R\frac{\cos\lambda x}{x^2+a^2}\:dx\\
                &=\frac{2a}{\pi}\int_{0}^{\rightarrow\infty}\frac{\cos\lambda x}{x^2+a^2}\:dx\\
            \end{split}
        \end{equation*}
        pero, de (i) se sabe que $x\mapsto\frac{\cos\lambda x}{x^2+a^2}$ es integrable en $[0,\infty[$, por tanto coincide su valor con el de la integral impropia, así:
        \begin{equation*}
            \begin{split}
                e^{ -a\abs{\lambda}}&=\frac{2a}{\pi}\int_{0}^{\infty}\frac{\cos\lambda x}{x^2+a^2}\:dx\\
                \Rightarrow \int_{0}^{\infty}\frac{\cos\lambda x}{x^2+a^2}\:dx&=\frac{\pi}{2a}e^{ -a\abs{\lambda}}\\
            \end{split}
        \end{equation*}

        De (iii): Por la parte anterior, para todo $\lambda\in\mathbb{R}\backslash\left\{0\right\}$ tiene que 
        \begin{equation*}
            \Phi(\lambda)=\int_{0}^{\infty}\frac{\cos\lambda x}{x^2+a^2}\:dx=\frac{\pi}{2a}e^{ -a\abs{\lambda}}=\frac{\pi}{2a}e^{-a\abs{\lambda}}
        \end{equation*}
        Si todo funciona bien, por el Teorema de derivación para funciones definidas por integrales impropias, se tendría para $\lambda>0$:
        \begin{equation*}
            \begin{split}
                -\int_0^{\rightarrow\infty}\frac{x\sin\lambda x}{x^2+a^2}\:dx&=\Phi'(\lambda)\\
                &=-\frac{\pi}{2}e^{-a\lambda}\\
                &=-\frac{\pi}{2}e^{-a\abs{\lambda}}\\
                \Rightarrow \int_0^{\rightarrow\infty}\frac{x\sin\lambda x}{x^2+a^2}\:dx&=\frac{\pi}{2}e^{-a\abs{\lambda}}\\
            \end{split}
        \end{equation*}
        y, para $\lambda<0$:
        \begin{equation*}
            \begin{split}
                -\int_0^{\rightarrow\infty}\frac{x\sin\lambda x}{x^2+a^2}\:dx&=\Phi'(\lambda)\\
                &=\frac{d}{d\lambda}\left(\frac{\pi}{2a}e^{-a(-\lambda)}\right) \\
                &=\frac{\pi}{2}e^{a\lambda}\\
                &=\frac{\pi}{2}e^{-a\abs{\lambda}}\\
                \Rightarrow \int_0^{\rightarrow\infty}\frac{x\sin\lambda x}{x^2+a^2}\:dx&=-\frac{\pi}{2}e^{-a\abs{\lambda}}\\
            \end{split}
        \end{equation*}
        es decir que
        \begin{equation*}
            \int_0^{\rightarrow\infty}\frac{x\sin\lambda x}{x^2+a^2}\:dx=\Sgn{\lambda}\cdot\frac{\pi}{2}e^{ -a\abs{\lambda}}
        \end{equation*}
        %TODO Usar el teorema de derivación de funciones definidas por integrales impropias
    \end{proof}

    \begin{excer}
        Sea $H$ una matriz simétrica real $n\times n$ positiva definida, es decir, la forma cuadrática $\pint{x}{Hx}$ sobre $\mathbb{R}^n$ es positiva definida. Sea $\cf{f}{\mathbb{R}^n}{\mathbb{R}}$ la función
        \begin{equation*}
            f(x)=e^{ -\pint{Hx}{x}},\quad\forall x\in\mathbb{R}^n
        \end{equation*}
        \textbf{Demuestre} que $f$ es integrable y que
        \begin{equation*}
            \fou{f}(x)=\frac{\pi^{ n/2}}{\left(\det H\right)^{ 1/2}}e^{ -\frac{1}{4}\pint{H^{-1}x}{x}},\quad\forall x\in\mathbb{R}^n
        \end{equation*}
        
        \textit{Sugerencia.} $f$ es medible. Para ver que es integrable, pruebe que $\pint{Hx}{x}\geq m\norm{x}^2$, donde
        \begin{equation*}
            m=\min_{ x\in S} \left\{\pint{Hx}{x} \right\}>0
        \end{equation*}
        con $S=\left\{x\in\mathbb{R}^n\Big|\norm{x}=1 \right\}$. Se sabe de álgebra que existe una matriz ortogonal $U$ tal que $U^{-1}HU=\textup{Diag}\left(\lambda_1,...,\lambda_n\right)$, donde $\lambda_1,...,\lambda_n$ son números estrictamente positivos. En la integral $\fou{f}(x)=\int_{\mathbb{R}^n}e^{ -i\pint{x}{y}}e^{ -\pint{Hy}{y}}\:dy$ haga el cambio de variable $y=Uz$ siendo tal que $\abs{\det U}=1$, $\pint{Ur}{Us}=\pint{r}{s}$ (y lo análogo para $U^{-1}$) y observe que $\diag{1/\lambda_1,...,1/\lambda_n}=U^{-1}H^{-1}U$.
    \end{excer}

    \begin{proof}
        Veamos que $f$ es medible (más aún, es continua). En efecto, considere la matriz $H$ dada por
        \begin{equation*}
            H=\left[\begin{array}{ccccc}
                h_{1,1} & h_{1,2} & ... & h_{1,n-1} & h_{1,n}\\
                h_{2,1} & h_{2,2} & ... & h_{2,n-1} & h_{2,n}\\
                \vdots & \vdots & \ddots & \vdots & \vdots \\
                h_{n,1} & h_{n,2} & ... & h_{n,n-1} & h_{n,n}\\
            \end{array}\right]
        \end{equation*}
        Se tiene entonces que para cada $x=(x_1,...,x_n)\in\mathbb{R}^n$:
        \begin{equation*}
            \begin{split}
                \pint{x}{Hx}&=\left[
                    \begin{array}{cccc}
                        x_1 & x_2 & ... & x_n\\
                    \end{array}
                \right]\cdot\left[\begin{array}{c}
                    y_1\\
                    y_2\\
                    \vdots\\
                    y_n\\
                \end{array}\right]\\
                &=\sum_{ k=1}^n x_ky_k\\
            \end{split}
        \end{equation*}
        donde
        \begin{equation*}
            \left[\begin{array}{c}
                y_1\\
                y_2\\
                \vdots\\
                y_n\\
            \end{array}\right]=\left[\begin{array}{ccccc}
                h_{1,1} & h_{1,2} & ... & h_{1,n-1} & h_{1,n}\\
                h_{2,1} & h_{2,2} & ... & h_{2,n-1} & h_{2,n}\\
                \vdots & \vdots & \ddots & \vdots & \vdots \\
                h_{n,1} & h_{n,2} & ... & h_{n,n-1} & h_{n,n}\\
            \end{array}\right]\cdot\left[\begin{array}{c}
                x_1\\
                x_2\\
                \vdots\\
                x_n\\
            \end{array}\right]
        \end{equation*}
        es decir que
        \begin{equation*}
            y_k=\sum_{i=1}^n h_{k,i}x_i
        \end{equation*}
        Por tanto,
        \begin{equation*}
            \begin{split}
                \pint{x}{Hx}&=\sum_{ k=1}^n x_k\sum_{ i=1}^n h_{k,i}x_i\\
                &=\sum_{ k=1}^n\sum_{ i=1}^n x_kx_ih_{k,i}\\
            \end{split}
        \end{equation*}
        siendo la aplicación $x\mapsto \pint{x}{Hx}$ una aplicación polinomial de $n$-variables, luego es continua. Así, la composición con $t\mapsto e^{-t}$ es continua, es decir que la función $f$ es continua en $\mathbb{R}^n$, luego medible.

        Ahora, sea ahora
        \begin{equation*}
            m=\inf_{ x\in S} \left\{\pint{Hx}{x} \right\}
        \end{equation*}
        donde $S=\left\{x\in\mathbb{R}^n\Big|\norm{x}=1 \right\}$. Afirmamos que $m>0$. En el caso que $m=0$, como la función $x\mapsto\pint{x}{Hx}$ es continua de $\mathbb{R}^n$ en $\mathbb{R}$, se tendría que alcanzaría su máximo y mínimo, luego existiría $x_0\in S$ tal que
        \begin{equation*}
            \pint{x_0}{Hx_0}=0
        \end{equation*}
        lo cual contradeciría el hecho de que $H$ es positiva definida. Por tanto, $m>0$. Más aún,
        \begin{equation*}
            m=\inf_{ x\in S} \left\{\pint{Hx}{x} \right\}=\min_{ x\in S} \left\{\pint{Hx}{x} \right\}
        \end{equation*}
        
        Sea $x\in\mathbb{R}^n$ no cero, se tiene que
        \begin{equation*}
            \begin{split}
                \pint{Hx}{x}
                &=\pint{H\left(\norm{x}\cdot\frac{x}{\norm{x}}\right)}{\norm{x}\cdot\frac{x}{\norm{x}}}\\
                &=\norm{x}^2\pint{H\left(\frac{x}{\norm{x}}\right)}{\frac{x}{\norm{x}}}\\
                &\geq m\norm{x}^2\\
                \Rightarrow -\pint{Hx}{x}&\leq -m\norm{x}^2\\
            \end{split}
        \end{equation*}
        Considere la función $x\mapsto e^{ -m\norm{x}^2}$ de $\mathbb{R}^n$ en $\mathbb{R}$. Se tiene que
        \begin{equation*}
            0\leq f(x)\leq e^{ -m\norm{x}^2},\quad\forall x\in\mathbb{R}^n
        \end{equation*}
        siendo $x\mapsto e^{ -m\norm{x}^2}$ intergable en $\mathbb{R}^n$, luego $f$ lo es en $\mathbb{R}^n$, luego la transformada de Fourier $\fou{f}(\cdot)$ está definida en todo $\mathbb{R}^n$.

        Sea $x\in\mathbb{R}^n$, entonces
        \begin{equation*}
            \begin{split}
                \fou{f}(x)&=\int_{\mathbb{R}^n}e^{ -i\pint{x}{y}}f(y)\:dy\\
                &=\int_{\mathbb{R}^n}e^{ -i\pint{x}{y}}\cdot e^{ -\pint{Hy}{y}}\:dy\\
            \end{split}
        \end{equation*}
        Se sabe por un resultado de álgebra lineal que existen una matriz $D$ $n\times n$ diagonal con entradas en la diagonal positivas, y una matriz ortogonal $U$ (también $n\times n$) con $\abs{\det U}=1$ tal que
        \begin{equation*}
            U^{-1}HU=D\Rightarrow HU=UD
        \end{equation*}
        Hagáse el cambio de variable $y=Uz$, se tiene en la integral anterior que
        \begin{equation*}
            \begin{split}
                \fou{f}(x)&=\int_{\mathbb{R}^n}e^{ -i\pint{x}{y}}\cdot e^{ -\pint{Hy}{y}}\:dy\\
                &=\int_{\mathbb{R}^n}e^{ -i\pint{x}{Uz}}\cdot e^{ -\pint{HUz}{Uz}}\:dz\\
                &=\int_{\mathbb{R}^n}e^{ -i\pint{x}{Uz}}\cdot e^{ -\pint{UDz}{Uz}}\:dz\\
                &=\int_{\mathbb{R}^n}e^{ -i\pint{x}{Uz}}\cdot e^{-\pint{Dz}{z}}\:dz\\
            \end{split}
        \end{equation*}
        %TODO
    \end{proof}

    \begin{excer}
        Recuerde que si $f=\chi_{[-a,a]}$, entonces
        \begin{equation*}
            \fou{f}(x)=\frac{2\sin ax}{x},\quad\forall x\neq0
        \end{equation*}
        \textbf{Deduzca} la fórmula
        \begin{equation*}
            \int_{ -\infty}^{\infty}\left(\frac{\sin ax}{x} \right)^2\:dx=\pi a
        \end{equation*}
    \end{excer}

    \begin{proof}
        %TODO Usar proposición 4.5.2

        Se sabe que la función $x\mapsto\left(\frac{\sin ax}{x}\right)^2$ es integrable en $\mathbb{R}$. Observemos ahora que
        \begin{equation*}
            \begin{split}
                \int_{-\infty}^{\infty}\left(\frac{\sin ax}{x}\right)^2\:dx&=\frac{1}{4}\cdot\int_{-\infty}^{\infty}\fou{f}(x)\cdot\fou{f}(x)\:dx\\
                &=\frac{1}{4}\cdot\int_{-\infty}^{\infty}\fou{f}(x)\:dx\int_{-\infty}^\infty e^{ -ixy}f(y)\:dy\\
                &=\frac{1}{4}\cdot\int_{-\infty}^{\infty}\fou{f}(x)\:dx\int_{-\infty}^\infty e^{ -ixy}\chi_{[-a,a]}(y)\:dy\\
                &=\frac{1}{2}\cdot\int_{-\infty}^{\infty}\frac{\sin ax}{x}\:dx\int_{-a}^a e^{ -ixy}\:dy\\
            \end{split}
        \end{equation*}
        considere la función $(x,y)\mapsto e^{-ixy}\cdot\frac{\sin ax}{x}$. Esta función es integrable en $\mathbb{R}\times[-a,a]$. En efecto, basta con ver que la función
        \begin{equation*}
            (x,y)\mapsto \frac{e^{-ixy}}{x}
        \end{equation*}
        lo es en $[0,\infty[\times[-a,a]$.%TODO

        Luego por Fubini se sigue que
        \begin{equation*}
            \begin{split}
                \int_{-\infty}^{\infty}\frac{\sin ax}{x}\:dx\int_{-a}^a e^{ -ixy}\:dy&=\int_{-a}^{a}\frac{\sin ax}{x}\:dy\int_{-\infty}^\infty e^{ -ixy}\:dx\\
                &=\int_{-a}^{a}\:dy\int_{-\infty}^\infty e^{ -ixy}\cdot\frac{\sin ax}{x}\:dx\\
                &=\int_{-a}^{a}\:dy \lim_{ R\rightarrow\infty}\int_{-R}^R e^{ -ixy}\cdot\frac{\sin ax}{x}\:dx\\
                &=\int_{-a}^{a}\:dx \lim_{ R\rightarrow\infty}\int_{-R}^R e^{ -ixy}\cdot\frac{\sin ay}{y}\:dy\\
                \Rightarrow 2\int_{-\infty}^{\infty}\left(\frac{\sin ax}{x}\right)^2\:dx&=\int_{-a}^{a}\:dx \lim_{ R\rightarrow\infty}\int_{-R}^R e^{ -ixy}\cdot\frac{\sin ay}{y}\:dy\\
            \end{split}
        \end{equation*}
        pues, la función $x\mapsto \int_{-\infty}^\infty e^{ -ixy}\cdot\frac{\sin ay}{y}\:dy$ definida c.t.p. en $[-a,a]$ coincide con su integral impropia. 

        Sea $x\in\mathbb{R}$. Se tiene que $f$ posee derivadas laterales derecha e izquierda en $x$, por lo cual del teorema de inversión de Fourier en $\mathbb{R}$ se sigue que
        \begin{equation*}
            \begin{split}
                f(x)&=\frac{1}{2\pi}\lim_{ R\rightarrow\infty}\int_{-R}^R e^{ ixy}\fou{f}(y)\:dy\\
                \Rightarrow\chi_{[-a,a]}(-x)&=\frac{1}{\pi}\lim_{ R\rightarrow\infty}\int_{-R}^R e^{-ixy}\cdot\frac{\sin ay}{y} \:dy\\
                \Rightarrow\pi\cdot\chi_{[-a,a]}(-x)&=\lim_{ R\rightarrow\infty}\int_{-R}^R e^{-ixy}\cdot\frac{\sin ay}{y} \:dy\\
            \end{split}
        \end{equation*}
        Por tanto, tenemos que
        \begin{equation*}
            \begin{split}
                \Rightarrow \pi\int_{-a}^a\chi_{[-a,a]}(-x)\:dx&=\int_{-a}^a\:dx \lim_{ R\rightarrow\infty}\int_{-R}^R e^{-ixy}\cdot\frac{\sin ay}{y} \:dy\\
                \Rightarrow \pi\int_{-a}^a\:dx&=2\int_{-\infty}^{\infty}\left(\frac{\sin ax}{x}\right)^2\:dx\\
                \Rightarrow 2\pi a&=2\int_{-\infty}^{\infty}\left(\frac{\sin ax}{x}\right)^2\:dx\\
                \Rightarrow \int_{-\infty}^{\infty}\left(\frac{\sin ax}{x}\right)^2\:dx&=\pi a\\
            \end{split}
        \end{equation*}
        lo que prueba el resultado.
    \end{proof}

    \begin{excer}
        Haga lo siguiente:
        \begin{enumerate}
            \item Sea $f(x)=\left(1-\frac{\abs{x}}{a}\right)\chi_{[-a,a]}(x)$, para todo $x\in\mathbb{R}$. \textbf{Pruebe } que
            \begin{equation*}
                \fou{f}(x)=a\left(\frac{\sin\frac{ax}{2}}{\frac{ax}{2}}\right)^2
            \end{equation*}
            \item Usando $\fou_2{f}$ \textbf{muestre} la fórmula
            \begin{equation*}
                \int_{-\infty }^{\infty}\left(\frac{\sin ax}{x} \right)^4\:dx=\frac{2}{3}\pi a^3
            \end{equation*}
            \item \textbf{Calcule} la integral
            \begin{equation*}
                \int_{-\infty}^{\infty}\left(\frac{\sin ax}{x}\right)^3\:dx
            \end{equation*}
            \textit{Sugerencia.} Escriba $f(x)=\left(1-\frac{\abs{x}}{a} \right)\chi_[-a,a](x)$ y $g(x)=\chi_{[-a,a]}(x)$, para todo $x\in\mathbb{R}$. Aplique la identidad de Parseval
            \begin{equation*}
                \int_\mathbb{R}\fou_2{f}\fou_2{g}=\pint{\fou_2{f}}{\fou_2{g}}=\pint{f}{g}=\int_{\mathbb{R}}fg
            \end{equation*}
            para deducir el resultado.
        \end{enumerate}
    \end{excer}

    \begin{sol}
        De (i): Sea $x\in\mathbb{R}\backslash\left\{0\right\}$, tenemos que
        \begin{equation*}
            \begin{split}
                \fou{f}(x)&=\int_{-\infty}^\infty e^{ -ixy}f(y)\:dy\\
                &=\int_{-\infty}^\infty e^{ -ixy}\left(1-\frac{\abs{y}}{a} \right)\chi_{[-a,a]}(y)\:dy\\
                &=\int_{-a}^a e^{ -ixy}\left(1-\frac{\abs{y}}{a}\right)\:dy\\
                &=\int_{-a}^0 e^{ -ixy}\left(1-\frac{\abs{y}}{a}\right)\:dy+\int_{0}^a e^{ -ixy}\left(1-\frac{\abs{y}}{a}\right)\:dy\\
                &=\int_{-a}^0 e^{ -ixy}\left(1+\frac{y}{a}\right)\:dy+\int_{0}^a e^{ -ixy}\left(1-\frac{y}{a}\right)\:dy\\
                &=-\int_{a}^0 e^{ ixu}\left(1-\frac{u}{a}\right)\:du+\int_{0}^a e^{ -ixy}\left(1-\frac{y}{a}\right)\:dy\\
                &=\int_{0}^a e^{ ixy}\left(1-\frac{y}{a}\right)\:dy+\int_{0}^a e^{ -ixy}\left(1-\frac{y}{a}\right)\:dy\\
                &=\int_{0}^a\left(e^{ ixy}+e^{-ixy}\right)\cdot\left(1-\frac{y}{a}\right)\:dy\\
                &=2\int_{0}^a\left(1-\frac{y}{a}\right)\cos xy\:dy\\
                &=2\left[\int_{0}^a\cos xy\:dy-\frac{1}{a}\int_0^a y\cos xy\:dy\right]\\
            \end{split}
        \end{equation*}
        donde
        \begin{equation*}
            \begin{split}
                \int_{0}^a\cos xy\:dy&=\frac{1}{x}\int_0^{ ax}\cos u\:du\\
                &=\frac{1}{x}\sin u\Big|_0^{ax}\\
                &=\frac{\sin ax}{x}\\
            \end{split}
        \end{equation*}
        y,
        \begin{equation*}
            \begin{split}
                \int_0^a y\cos xy\:dy&=\int_0^{ax}\frac{u}{x}\cos u\: \frac{du}{x}\\
                &=\frac{1}{x^2}\int_0^{ax}u\cos u\:du\\
                &=\frac{1}{x^2}\left[ax\sin ax+\cos ax-1\right]
            \end{split}
        \end{equation*}
        Por tanto,
        \begin{equation*}
            \begin{split}
                \fou{f}(x)&=2\left[\int_{0}^a\cos xy\:dy-\frac{1}{a}\int_0^a y\cos xy\:dy\right]\\
                &=2\left[\frac{\sin ax}{x}-\frac{ax\sin ax}{ax^2}-\frac{\cos ax}{ax^2}+\frac{1}{ax^2}\right]\\
                &=2\left[\frac{\sin ax}{x}-\frac{\sin ax}{x}+\frac{1-\cos ax}{ax^2}\right]\\
                &=\frac{1-\cos ax}{\frac{ax^2}{2}}\\
                &=a\cdot\frac{2\sin^2\left(\frac{ax}{2}\right)}{\frac{a^2x^2}{2}}\\
                &=a\cdot\frac{\sin^2\frac{ax}{2}}{\frac{a^2x^2}{4}}\\
                &=a\cdot\left(\frac{\sin\frac{ax}{2}}{\frac{ax}{2}}\right)^2\\
            \end{split}
        \end{equation*}
        
        De (ii): 

    \end{sol}

    \begin{excer}
        Sea $n\geq2$ y $\cf{r}{\mathbb{R}^n}{\mathbb{R}}$ la función $x\mapsto r(x)=\norm{x}=\sqrt{x_1^2+...+x_n^2}$. Sea $\cf{f}{[0,\infty[}{\mathbb{C}}$ una función tal que $f\circ r$ es integrable en $\mathbb{R}^n$.
        \begin{enumerate}
            \item \textbf{Pruebe} que la transformada de Fourier $\fou{(f\circ r)}$ es una función radial.
            
            \textit{Sugerencia.} Si $U$ es una matriz ortogonal $n\times n$, se tiene que $\fou{(f\circ r)}(Ux)=\fou{(f\circ r)}(x)$. Dados $x,y\in\mathbb{R}^n$ tales que $\norm{x}=\norm{y}$, siempre existe una matriz ortogonal $U$ tal que $Ux=y$.

            \item \textbf{Muestre} que se cumple la \textbf{fórmula de Bochner}
            \begin{equation*}
                \fou{(f\circ r)}(x)=2(n-1)\omega_{ n-1}\int_0^\infty v_n(u(\norm{x}))f(u)u^{ n-1}\:du,\quad\forall x\in\mathbb{R}^n
            \end{equation*}
            donde $\omega_{n-1}$ es el volumen de la bola euclideana de radio uno en $\mathbb{R}^{ n-1}$ y $v_n$ se define por la fórmula
            \begin{equation*}
                v_n(t)=\int_0^{\frac{\pi}{2}}\cos(t\cos\theta)\sin^{ n-2}\theta\:d\theta
            \end{equation*}
            \textit{Sugerencia.} Según el inciso (i),
            \begin{equation*}
                \fou{(f\circ r)}(x)=\fou{(f\circ r)}(\norm{x},0,...,0)=\int_{\mathbb{R}^n}f(\norm{y})e^{ -i\norm{x}y_1}\:dy_1\cdots dy_n
            \end{equation*}
            Transforme esta integral por el Teorema de Fubini y exprese la integral con respecto a $y_2,...,y_n$ como una integral simple. La doble integral resultante se transforma a coordenadas polares.
        \end{enumerate}
    \end{excer}

    \begin{sol}
        De (i): Ya se sabe que $f\circ r$ es una función radial (de la definición es claro este hecho). Como $f\circ r$ es integrable en $\mathbb{R}^n$, entonces la transformada de Fourier de $f\circ r$ está definida para todo $x\in\mathbb{R}^n$.
        
        Sean $x,y\in\mathbb{R}^n$ tales que $\norm{x}=\norm{y}$. Para probar que $\fou{(f\circ r)}$ es una función radial, basta con ver que
        \begin{equation*}
            \fou{(f\circ r)}(x)=\fou{(f\circ r)}(y)
        \end{equation*}
        En efecto, como $\norm{x}=\norm{y}$ por álgebra lineal se sabe que existe una matriz ortogonal $n\times n$ con determinante $1$ tal que $z=Uw$. Por lo cual, por el teorema de cambio de variable se sigue que:
        \begin{equation*}
            \begin{split}
                \fou{(f\circ r)}(x)&=\int_{\mathbb{R}^n}e^{-i\pint{x}{z}}f\circ r(z)\:dz\\
                &=\int_{\mathbb{R}^n}e^{ -i\pint{x}{Uw}}f\circ r(Uw)\:dw\\
            \end{split}
        \end{equation*}
    \end{sol}

    \begin{excer}
        Haga lo siguiente:
        \begin{enumerate}
            \item Sea $\cf{h}{[0,\infty[}{0\mathbb{C}}$ una función integrable en $[0,\infty[$. Sea $a>0$, \textbf{demuestre} que existe la integral impropia
            \begin{equation*}
                \int_{a}^{ \rightarrow\infty}\frac{dx}{x}\int_0^{\infty}h(y)\sin xy\:dy
            \end{equation*}
            \textit{Sugerencia.} Justifique la inversión del orden de las integraciones.
            \item Sea $\cf{f}{\mathbb{R}}{\mathbb{R}}$ la función
            \begin{equation*}
                f(x)=\left\{
                    \begin{array}{lcr}
                        \frac{1}{e}x & \textup{ si } & \abs{x}<e\\
                        \frac{\Sgn{x}}{\log\abs{x}} & \textup{ si } & \abs{x}\geq e\\
                    \end{array}
                \right.,\quad\forall x\in\mathbb{R}
            \end{equation*}
            Muestre que, para $a>0$ no existe la integral impropia
            \begin{equation*}
                \int_a^{\rightarrow \infty}\frac{f(x)}{x}\:dx
            \end{equation*}
            De este hecho y del inciso (i) \textbf{deduzca} que no existe $g\in\mathcal{L}_1(\mathbb{R},\mathbb{C})$ tal que $f=\fou{g}$. Así pues, la transformación de Fourier no es una aplicación suprayectiva de $L_1(\mathbb{R},\mathbb{C})$ en $\mathcal{C}_0(\mathbb{R},\mathbb{C})$.
        \end{enumerate}
    \end{excer}

    \begin{sol}
        
    \end{sol}

    \begin{excer}
        Haga lo siguiente:
        \begin{enumerate}
            \item Sea $\cf{f}{\mathbb{R}}{\mathbb{C}}$ una función integrable en $\mathbb{R}$. Se supone que existe una función $\cf{\varphi}{\mathbb{R}}{\mathbb{C}}$ localmente integrable en $\mathbb{R}$ tal que
            \begin{equation*}
                f(x)=f(0)+\int_0^{x}\varphi,\quad\forall x\in\mathbb{R}\quad\textup{y}\quad\abs{\varphi(x)}\underset{\abs{x}\rightarrow\infty}{=}O\left(\frac{1}{\abs{x}^m}\right)
            \end{equation*}
        \end{enumerate}
        donde $m>2$. \textbf{Pruebe} que
        \begin{equation*}
            \abs{f(x)}=\underset{\abs{x}\rightarrow\infty}{=}O\left(\frac{1}{x^{ m-1}}\right)
        \end{equation*}
        y que, para todo $x\in\mathbb{R}$, existen las sumas
        \begin{equation*}
            \Phi(x)=\sum_{k\in\mathbb{Z}}\varphi(x+k)\quad\textup{y}\quad F(x)=\sum_{ k\in\mathbb{Z}}f(x+k)
        \end{equation*}
        siendo la convergencia absoluta y uniforme en $[-1,1]$, luego en $\mathbb{R}$. \textbf{Muestre} finalmente que
        \begin{equation*}
            F(x)=F(0)+\int_0^x\Phi,\quad\forall x\in\mathbb{R}
        \end{equation*}
        \item $F$ es una función periódica de periodo uno. \textbf{Demuestre} que los coeficientes de Fourier de $F$ respecto al sistema O.N. $\left(e^{2\pi int} \right)_{n\in\mathbb{Z}}$ son
        \begin{equation*}
            \int_0^1 F(x)e^{ -2\pi inx}\:dx=\fou{f}(2\pi n),\quad\forall n\in\mathbb{Z}
        \end{equation*}
        \textbf{Deduzca la fórmula Sumatoria de Poisson}
        \begin{equation*}
            \sum_{ k\in\mathbb{Z}}f(x+k)=\sum_{ k\in\mathbb{Z}}\fou{f}(x+k),\quad\forall x\in\mathbb{R}
        \end{equation*}
        \item Aplicando la fórmula sumatoria de Poisson a la función $x\mapsto e^{-\alpha\abs{x}}$ para $\alpha>0$, obtenga el desarrollo
        \begin{equation*}
            \coth x = \frac{1}{x}+2x\sum_{ n=1}^{\infty}\frac{1}{x^2+n^2\pi^2},\quad\forall x\geq0
        \end{equation*}
        Se define la \textbf{función theta} por
        \begin{equation*}
            \Theta(x)=\sum_{ n=-\infty}^{\infty}e^{ -\pi n^2x},\quad\forall x>0
        \end{equation*}
        Aplicando la fórmula sumatoria de Poisson a la función $x\mapsto e^{-\alpha x^2}$ para $\alpha>0$, \textbf{pruebe} la identidad
        \begin{equation*}
            \Theta(x)=\sqrt{\frac{1}{x}}\Theta\left(\frac{1}{x}\right),\quad\forall x>0
        \end{equation*}
    \end{excer}

    \begin{sol}
        
    \end{sol}

    \begin{excer}
        Haga lo siguiente:
        \begin{enumerate}
            \item Sea $f\in\mathcal{L}_1(\mathbb{R},\mathbb{C})$ tal que $f(x)=0$ para todo $x<0$. Para todo $z\in\mathbb{C}$ tal que $\Im z\leq 0$ se define
            \begin{equation*}
                \fou{f}(x)=\int_0^{\infty}e^{ -izx}f(x)\:dx
            \end{equation*}
            \textbf{Pruebe} que esta definción tiene sentido, que $\fou{f}$ es continua en el semiplano cerrado $\left\{z\in\mathbb{C}\Big|\Im z\leq0 \right\}$ y holomorfa en el semiplano abierto $\left\{z\in\mathbb{C}\Big|\Im z<0 \right\}$.

            \textit{Sugerencia.} El teorema de derivación de funciones defindas por integrales continúa siendo válido al sustituir el intervalo $I$ por un abierto de $\mathbb{C}$.

            \item Sean $f,g\in\mathcal{L}_1(\mathbb{R},\mathbb{C})$ tales que $f(x)=g(x)=0$, $\forall x<0$. \textbf{Muestre} que para todo $z\in\mathbb{C}$ tal que $\Im z\leq 0$ se tiene
            \begin{equation*}
                \fou{(f*g)}(z)=\fou{f}(z)\fou{g}(z)
            \end{equation*}
            \item Sean $f,g$ como en el inciso (ii). Se supone además que $\fou{(f*g)}=0$ c.t.p. en $\mathbb{R}$. \textbf{Demuestre} que $f=0$ c.t.p. en $\mathbb{R}$ o bien $g=0$ c.t.p. en $\mathbb{R}$.
            
            \textit{Sugerencia.} Deduzca de (i) y (ii) que $\fou{f}=0$ o bien $\fou{g}=0$.
        \end{enumerate}
    \end{excer}

    \begin{sol}
        
    \end{sol}

    \begin{excer}
        Haga lo siguiente:
        \begin{enumerate}
            \item Sea $f\in\mathcal{L}_2(\mathbb{R},\mathbb{C})$. Se define para todo $z\in\mathbb{C}$,
            \begin{equation*}
                F(z)=\int_{-\infty}^{\infty}e^{ -izx-\frac{x^2}{2}}\conj{f(x)}\:dx
            \end{equation*}
            \textbf{Pruebe} que $F$ es holomorfa en $\mathbb{C}$ y que, para todo $n\in\mathbb{N}$ y para todo $z\in\mathbb{C}$,
            \begin{equation*}
                F^{(n)}(z)=(-i)^n\int_{-\infty}^\infty x^ne^{ -izx-\frac{x^2}{2}}\conj{f(x)}\:dx
            \end{equation*}
            \textit{Sugerencia.} La misma que la del Problema 11.
            \item Se supone que $f$ es ortogonal a todas las funciones de Hermite. Muestre que $F=0$ y \textbf{deduzca} que $f=0$ c.t.p. en $\mathbb{R}^n$.
            
            Así pues, el sistema de funciones de Hermite normalizadas es un sistema ortonormal maximal en $L_2(\mathbb{R},\mathbb{C})$.

            \textit{Sugerencia.} Observe que $F^{(n)}(0)=0$, para todo $n\in\mathbb{N}$. La condición $F=0$ implica que la transformada de Fourier de la función integrable $x\mapsto e^{ -\frac{x^2}{2}}\conj{f(x)}$ es cero.
        \end{enumerate}
    \end{excer}

    \begin{sol}
        
    \end{sol}

    \begin{excer}
        Haga lo siguiente:
        \begin{enumerate}
            \item \textbf{Demuestre} la fórmula.
            \begin{equation*}
                D_x^n\int_{-\infty}^\infty e^{ -y^2}e^{\frac{1}{2}(y-ix)^2}\:dy=(-1)^n\int_{-\infty}^\infty e^{ -y^2}D_x^n e^{\frac{1}{2}(y-ix)^2}\:dy
            \end{equation*}
            \item Se consideran las funciones de Hermite
            \begin{equation*}
                \varphi(x)=(-1)^n e^{\frac{x^2}{2}}D^n e^{ -x^2}
            \end{equation*}
            \textbf{Pruebe} que $\fou_2{\varphi_n}=(-1)^n\varphi_n$. Así pues, las funciones de Hermite son vectores propios para el operador $\fou_2{\cdot}$.

            \textit{Sugerencia.} Tranforme $\fou_2{\varphi_n}$ por la ``fórmula de integración por partes de orden $n$".
            \begin{equation*}
                \int_a^b f^(n)g=\left[\sum_{ k=0}^{ n-1}(-1)^kf^{ (n-k-1)}g^{(k)} \right]+(-1)^n\int_a^b fg^(n)
            \end{equation*}
            (al suponer $f^(n)$ y $g^(n)$ continuas en $[a,b]$). Después, use la fórmula del inciso (i).
        \end{enumerate}
    \end{excer}

    \begin{sol}
        
    \end{sol}

\end{document}