\documentclass[12pt]{report}
\usepackage[spanish]{babel}
\usepackage[utf8]{inputenc}
\usepackage{amsmath}
\usepackage{amssymb}
\usepackage{amsthm}
\usepackage{graphics}
\usepackage{subfigure}
\usepackage{lipsum}
\usepackage{array}
\usepackage{multicol}
\usepackage{enumerate}
\usepackage[framemethod=TikZ]{mdframed}
\usepackage[a4paper, margin = 1.5cm]{geometry}

%En esta parte se hacen redefiniciones de algunos comandos para que resulte agradable el verlos%

\renewcommand{\theenumii}{\roman{enumii}}

\def\proof{\paragraph{Demostración:\\}}
\def\endproof{\hfill$\blacksquare$}

\def\sol{\paragraph{Solución:\\}}
\def\endsol{\hfill$\square$}

%En esta parte se definen los comandos a usar dentro del documento para enlistar%

\newtheoremstyle{largebreak}
  {}% use the default space above
  {}% use the default space below
  {\normalfont}% body font
  {}% indent (0pt)
  {\bfseries}% header font
  {}% punctuation
  {\newline}% break after header
  {}% header spec

\theoremstyle{largebreak}

\newmdtheoremenv[
    leftmargin=0em,
    rightmargin=0em,
    innertopmargin=-2pt,
    innerbottommargin=8pt,
    hidealllines = true,
    roundcorner = 5pt,
    backgroundcolor = gray!60!red!30
]{exa}{Ejemplo}[section]

\newmdtheoremenv[
    leftmargin=0em,
    rightmargin=0em,
    innertopmargin=-2pt,
    innerbottommargin=8pt,
    hidealllines = true,
    roundcorner = 5pt,
    backgroundcolor = gray!50!blue!30
]{obs}{Observación}[section]

\newmdtheoremenv[
    leftmargin=0em,
    rightmargin=0em,
    innertopmargin=-2pt,
    innerbottommargin=8pt,
    rightline = false,
    leftline = false
]{theor}{Teorema}[section]

\newmdtheoremenv[
    leftmargin=0em,
    rightmargin=0em,
    innertopmargin=-2pt,
    innerbottommargin=8pt,
    rightline = false,
    leftline = false
]{propo}{Proposición}[section]

\newmdtheoremenv[
    leftmargin=0em,
    rightmargin=0em,
    innertopmargin=-2pt,
    innerbottommargin=8pt,
    rightline = false,
    leftline = false
]{cor}{Corolario}[section]

\newmdtheoremenv[
    leftmargin=0em,
    rightmargin=0em,
    innertopmargin=-2pt,
    innerbottommargin=8pt,
    rightline = false,
    leftline = false
]{lema}{Lema}[section]

\newmdtheoremenv[
    leftmargin=0em,
    rightmargin=0em,
    innertopmargin=-2pt,
    innerbottommargin=8pt,
    roundcorner=5pt,
    backgroundcolor = gray!30,
    hidealllines = true
]{mydef}{Definición}[section]

\newmdtheoremenv[
    leftmargin=0em,
    rightmargin=0em,
    innertopmargin=-2pt,
    innerbottommargin=8pt,
    roundcorner=5pt
]{excer}{Ejercicio}[section]

%En esta parte se colocan comandos que definen la forma en la que se van a escribir ciertas funciones%

\renewcommand{\leq}{\ensuremath{\leqslant}}
\renewcommand{\geq}{\ensuremath{\geqslant}}

\newcommand\abs[1]{\ensuremath{\left|#1\right|}}
\newcommand\divides{\ensuremath{\bigm|}}
\newcommand\cf[3]{\ensuremath{#1:#2\rightarrow#3}}
\newcommand\norm[1]{\ensuremath{\|#1\|}}
\newcommand\ora[1]{\ensuremath{\vec{#1}}}
\newcommand\pint[2]{\ensuremath{\langle#1| #2\rangle}}
\newcommand\conj[1]{\ensuremath{\overline{#1}}}
\newcommand{\N}[2]{\ensuremath{\mathcal{N}_{#1}\left(#2\right)}}
\newcommand{\natint}[1]{\ensuremath{\left[\!\left[#1\right]\!\right]}}
\newcommand{\fou}[1]{\ensuremath{\mathcal{F}#1}}
\newcommand{\diag}[1]{\ensuremath{\left(#1\right)}}

%recuerda usar \clearpage para hacer un salto de página

\begin{document}
    \setlength{\parskip}{5pt} % Añade 5 puntos de espacio entre párrafos
    \setlength{\parindent}{12pt} % Pone la sangría como me gusta
    \title{Lista Ejercicios Análisis Matemático IV}
    \author{Cristo Daniel Alvarado}
    \maketitle

    \tableofcontents %Con este comando se genera el índice general del libro%

    %\setcounter{chapter}{3} %En est a parte lo que se hace es cambiar la enumeración del capítulo%
    
    \chapter{Lista 4}
    
    \renewcommand{\theenumi}{\roman{enumi}}

    \begin{excer}
        Haga lo siguiente:
        \begin{enumerate}
            \item Sea $f\in\mathcal{L}_1(\mathbb{R}^n,\mathbb{C})$. Defina $\cf{P}{\mathbb{R}^n}{\mathbb{R}}$ como:
            \begin{equation*}
                P(x_1,...,x_n)=e^{ -\sum_{ k=1}^n \abs{x_k}},\quad\forall x\in\mathbb{R}^n
            \end{equation*}
            Fije $\nu\in\mathbb{N}$, \textbf{demuestre} la fórmula:
            \begin{equation*}
                \int_{\mathbb{R}^n}\fou{f}(x)P\left(\frac{x}{\nu}\right)\:dx=(2\nu)^n\int_{\mathbb{R}^n}\frac{f(x_1,...,x_n)}{(x+\nu^2x_1^2)\cdots(x+\nu^2x_n^2)}\:dx_1\cdots dx_n
            \end{equation*}
            \item \textbf{Deduzca} que si $f\in\mathcal{L}_1(\mathbb{R}^n,\mathbb{C})\cap\mathcal{L}_{\infty}(\mathbb{R}^n,\mathbb{C})$ y $\fou{f}\geq0$, entonces $\fou{f}\in\mathcal{L}_1(\mathbb{R}^n,\mathbb{C})$.
            
            \textit{Sugerencia.} Aplique el teorema de Beppo-Levi.
        \end{enumerate}
    \end{excer}

    \begin{proof}
        
    \end{proof}

    \begin{excer}
        Sea $f\in\mathcal{L}_1(\mathbb{R}^n,\mathbb{C})$. Se supone que $f(x)>0$, para todo $x\in\mathbb{R}^n$. \textbf{Pruebe} que si $x\neq0$, entonces
        \begin{equation*}
            \fou{f}(0)>\abs{\fou{f}(x)}
        \end{equation*}
        \textit{Sugerencia.} Una vez que ha demostrado $\abs{\fou{f}(x)}\leq\fou{f}(0)$, para todo $x\in\mathbb{R}^n$, Para demostrar la desigualdad estricta para $x\neq0$ proceda por reducción al absurdo y use el Problema 2 de la Lista 6 de Análisis Matemático II.
    \end{excer}

    \begin{proof}
        
    \end{proof}
    
    \begin{excer}
        Haga lo siguiente:
        \begin{enumerate}
            \item Sean $a>0$ y $\lambda\in\mathbb{R}$. \textbf{Pruebe} que la función $x\mapsto (\cos \lambda x)/(x^2+a^2)$ es integrable en $[0,\infty[$. \textbf{Muestre} que si $\lambda\neq0$, la función $x\mapsto (x\sin \lambda x)/(x^2+a^2)$ no es integrable en $[0,\infty[$, pero existe la integral impropia
            \begin{equation*}
                \int_0^{\rightarrow\infty}\frac{x\sin\lambda x}{x^2+a^2}\:dx
            \end{equation*}
            \textit{Sugerencia.} Muestre que
            \begin{equation*}
                \abs{\frac{x\sin\lambda x}{x^2+a^2}}\underset{x\rightarrow\infty}{\sim}\abs{\frac{\sin\lambda x}{x}}
            \end{equation*}
            Para probar la existencia de la integral impropia use los criterios de Abel.
            \item Recuerde que la función $x\mapsto (2a)/(x^2+a^2)$ es la transformada de Fourier de la función $x\mapsto e^{-a\abs{x}}$. Usando el teorema de inversión de Fourier, \textbf{demuestre} que
            \begin{equation*}
                \int_0^{\infty}\frac{\cos\lambda x}{x^2+a^2}\:dx=\frac{\pi}{2a}e^{ -a\abs{\lambda}}
            \end{equation*}
            \item Usando el inciso (ii), calcule la integral impropia
            \begin{equation*}
                \int_0^{\rightarrow\infty}\frac{x\sin\lambda x}{x^2+a^2}\:dx
            \end{equation*}
            \textit{Sugerencia.} Para $\lambda\neq0$ defina
            \begin{equation*}
                \Phi(\lambda)=\int_0^{\rightarrow\infty}\frac{\cos\lambda x}{x^2+a^2}\:dx
            \end{equation*}
            \textbf{Calcule} $\Phi'(\lambda)$ primero suponiendo $\lambda>\lambda_0$, donde $\lambda_0>0$ es arbitrario fijo, de forma análoga para $\lambda<0$ y finalmente para $\lambda=0$.
        \end{enumerate}
    \end{excer}

    \begin{proof}
        
    \end{proof}

    \begin{excer}
        Sea $H$ una matriz simétrica real $n\times n$ positiva definida, es decir, la forma cuadrática $\pint{x}{Hx}$ sobre $\mathbb{R}^n$ es positiva definida. Sea $\cf{f}{\mathbb{R}^n}{\mathbb{R}}$ la función
        \begin{equation*}
            f(x)=e^{ -\pint{Hx}{x}},\quad\forall x\in\mathbb{R}^n
        \end{equation*}
        \textbf{Demuestre} que $f$ es integrable y que
        \begin{equation*}
            \fou{f}(x)=\frac{\pi^{ n/2}}{\left(\det H\right)^{ 1/2}}e^{ -\frac{1}{4}\pint{H^{-1}x}{x}},\quad\forall x\in\mathbb{R}^n
        \end{equation*}
        
        \textit{Sugerencia.} $f$ es medible. Para ver que es integrable, pruebe que $\pint{Hx}{x}\geq m\norm{x}^2$, donde
        \begin{equation*}
            m=\min_{ x\in S} \left\{\pint{Hx}{x} \right\}>0
        \end{equation*}
        con $S=\left\{x\in\mathbb{R}^n\Big|\norm{x}=1 \right\}$. Se sabe de álgebra que existe una matriz ortogonal $U$ tal que $U^{-1}HU=\diag{\lambda_1,...,\lambda_n}$, donde $\lambda_1,...,\lambda_n$ son números estrictamente positivos. En la integral $\fou{f}(x)=\int_{\mathbb{R}^n}e^{ -i\pint{x}{y}}e^{ -\pint{Hx}{x}}\:dy$ haga el cambio de variable $y=Uz$ siendo tal que $\abs{\det U}=0$, $\pint{Ur}{Us}=\pint{r}{s}$ (y lo análogo para $U^{-1}$) y observe que $\diag{1/\lambda_1,...,1/\lambda_n}=U^{-1}H^{-1}U$.
    \end{excer}

    \begin{proof}
        
    \end{proof}

    \begin{excer}
        Recuerde que si $f=\chi_{[-a,a]}$, entonces
        \begin{equation*}
            \fou{f}(x)=\sqrt{\frac{\pi}{2}}\frac{\sin ax}{x},\quad\forall x\neq0
        \end{equation*}
        \textbf{Deduzca} la fórmula
        \begin{equation*}
            \int_{ -\infty}^{\infty}\left(\frac{\sin ax}{x} \right)^2\:dx=\pi a
        \end{equation*}
    \end{excer}

    \begin{proof}
        
    \end{proof}

    \begin{excer}
        Haga lo siguiente:
        \begin{enumerate}
            \item 
        \end{enumerate}
    \end{excer}

\end{document}