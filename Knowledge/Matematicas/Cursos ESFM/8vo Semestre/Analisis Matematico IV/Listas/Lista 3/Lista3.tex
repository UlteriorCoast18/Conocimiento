\documentclass[12pt]{report}
\usepackage[spanish]{babel}
\usepackage[utf8]{inputenc}
\usepackage{hyperref}
\usepackage{amsmath}
\usepackage{amssymb}
\usepackage{amsthm}
\usepackage{graphics}
\usepackage{subfigure}
\usepackage{lipsum}
\usepackage{array}
\usepackage{multicol}
\usepackage{enumerate}
\usepackage[framemethod=TikZ]{mdframed}
\usepackage[a4paper, margin = 1.5cm]{geometry}

%En esta parte se hacen redefiniciones de algunos comandos para que resulte agradable el verlos%

\renewcommand{\theenumii}{\roman{enumii}}

\def\proof{\paragraph{Demostración:\\}}
\def\endproof{\hfill$\blacksquare$}

\def\sol{\paragraph{Solución:\\}}
\def\endsol{\hfill$\square$}

%En esta parte se definen los comandos a usar dentro del documento para enlistar%

\newtheoremstyle{largebreak}
  {}% use the default space above
  {}% use the default space below
  {\normalfont}% body font
  {}% indent (0pt)
  {\bfseries}% header font
  {}% punctuation
  {\newline}% break after header
  {}% header spec

\theoremstyle{largebreak}

\newmdtheoremenv[
    leftmargin=0em,
    rightmargin=0em,
    innertopmargin=-2pt,
    innerbottommargin=8pt,
    hidealllines = true,
    roundcorner = 5pt,
    backgroundcolor = gray!60!red!30
]{exa}{Ejemplo}[section]

\newmdtheoremenv[
    leftmargin=0em,
    rightmargin=0em,
    innertopmargin=-2pt,
    innerbottommargin=8pt,
    hidealllines = true,
    roundcorner = 5pt,
    backgroundcolor = gray!50!blue!30
]{obs}{Observación}[section]

\newmdtheoremenv[
    leftmargin=0em,
    rightmargin=0em,
    innertopmargin=-2pt,
    innerbottommargin=8pt,
    rightline = false,
    leftline = false
]{theor}{Teorema}[section]

\newmdtheoremenv[
    leftmargin=0em,
    rightmargin=0em,
    innertopmargin=-2pt,
    innerbottommargin=8pt,
    rightline = false,
    leftline = false
]{propo}{Proposición}[section]

\newmdtheoremenv[
    leftmargin=0em,
    rightmargin=0em,
    innertopmargin=-2pt,
    innerbottommargin=8pt,
    rightline = false,
    leftline = false
]{cor}{Corolario}[section]

\newmdtheoremenv[
    leftmargin=0em,
    rightmargin=0em,
    innertopmargin=-2pt,
    innerbottommargin=8pt,
    rightline = false,
    leftline = false
]{lema}{Lema}[section]

\newmdtheoremenv[
    leftmargin=0em,
    rightmargin=0em,
    innertopmargin=-2pt,
    innerbottommargin=8pt,
    roundcorner=5pt,
    backgroundcolor = gray!30,
    hidealllines = true
]{mydef}{Definición}[section]

\newmdtheoremenv[
    leftmargin=0em,
    rightmargin=0em,
    innertopmargin=-2pt,
    innerbottommargin=8pt,
    roundcorner=5pt
]{excer}{Ejercicio}[section]

%En esta parte se colocan comandos que definen la forma en la que se van a escribir ciertas funciones%

\renewcommand{\leq}{\ensuremath{\leqslant}}
\renewcommand{\geq}{\ensuremath{\geqslant}}

\newcommand\abs[1]{\ensuremath{\left|#1\right|}}
\newcommand\divides{\ensuremath{\bigm|}}
\newcommand\cf[3]{\ensuremath{#1:#2\rightarrow#3}}
\newcommand\norm[1]{\ensuremath{\|#1\|}}
\newcommand\ora[1]{\ensuremath{\vec{#1}}}
\newcommand\pint[2]{\ensuremath{\left(#1\big| #2\right)}}
\newcommand\conj[1]{\ensuremath{\overline{#1}}}
\newcommand{\N}[2]{\ensuremath{\mathcal{N}_{#1}\left(#2\right)}}

\newcommand{\natint}[1]{\ensuremath{\left[\!\left[#1\right]\!\right]}}

%recuerda usar \clearpage para hacer un salto de página

\begin{document}
    \setlength{\parskip}{5pt} % Añade 5 puntos de espacio entre párrafos
    \setlength{\parindent}{12pt} % Pone la sangría como me gusta
    \title{Lista 3 de Ejercicios Análisis Matemático IV}
    \author{Cristo Daniel Alvarado}
    \maketitle

    \setcounter{chapter}{2} %En esta parte lo que se hace es cambiar la enumeración del capítulo%
    
    \chapter{Ejercicios}
    
    \setcounter{section}{1}

    \begin{excer}
        \textbf{Pruebe} que, para todo $x\in]0,2\pi[$,
        \begin{equation*}
            \frac{\pi-x}{2}=\sum_{ n=1}^\infty\frac{\sin nx}{n}
        \end{equation*}
        Usando la identidad de Parseval, \textbf{demuestre} que
        \begin{equation*}
            \sum_{ n=1}^\infty\frac{1}{n^2}=\frac{\pi^2}{6}.
        \end{equation*}
    \end{excer}

    \begin{proof}
        Sea $\cf{f}{\mathbb{R}}{\mathbb{R}}$ tal que
        \begin{equation*}
            f(x)=\frac{\pi-x}{2},\quad\forall x\in[0,2\pi[
        \end{equation*}
        y extiéndase por periodicidad a todo $\mathbb{R}$. Es claro que $f\in\mathcal{L}_1^{2\pi}(\mathbb{R})$, sea ahora $x\in]0,2\pi[$. Por el teorema fundamental para la convergencia puntual de una serie de Fourier hay que encontrar un $0<\delta<\pi$ tal que
        \begin{equation*}
            \lim_{ m\rightarrow\infty}\int_{0}^\delta\frac{f(x+t)+f(x-t)-2f(x)}{t}\sin\left(m+\frac{1}{2}\right)dt
        \end{equation*}
        tomemos $\delta=\min\left\{x, 2\pi-x \right\}>0$. Se tienen dos casos:
        \begin{enumerate}
            \item $\delta=x$, entonces
            \begin{equation*}
                \begin{split}
                    \int_{0}^\delta\frac{f(x+t)+f(x-t)-2f(x)}{t}&\sin\left(m+\frac{1}{2}\right)dt\\
                    &=\int_{0}^\delta \frac{1}{t}\left[\frac{\pi-x-t}{2}+\frac{\pi-x+t}{2}-\frac{2\left(\pi-x \right)}{2}\right]\sin\left(m+\frac{1}{2}\right)dt\\
                    &=\int_{0}^\delta \frac{1}{t}\left[\pi-x-\pi+x \right]\sin\left(m+\frac{1}{2}\right)dt\\
                    &=\int_{0}^\delta 0\:dt\\
                    &=0\\
                \end{split}
            \end{equation*}
            por tanto, el límite cuando $m\rightarrow\infty$ resulta que da cero.
            \item $\delta=2\pi x$. El caso es análogo al anterior.
        \end{enumerate}
        por ambos incisos se concluye que
        \begin{equation*}
            \lim_{ m\rightarrow\infty}\int_{0}^\delta\frac{f(x+t)+f(x-t)-2f(x)}{t}\sin\left(m+\frac{1}{2}\right)dt=0
        \end{equation*}
        por tanto, la serie de Fourier de $f$ converge a $f$ puntualmente en $x$. Computemos ahora los coeficientes de la serie de Fourier de $f$. Si $n\geq0$:
        \begin{equation*}
            \begin{split}
                a_n&=\frac{1}{\pi}\int_{-\pi}^{\pi}f(x)\cos nxdx\\
                &=\frac{1}{\pi}\int_{0}^{2\pi}f(x)\cos nxdx\\
                &=\frac{1}{\pi}\int_{0}^{2\pi}\frac{\pi-x}{2}\cos nxdx\\
                &=\frac{1}{2}\int_{0}^{2\pi}\cos nxdx-\frac{1}{2\pi}\int_0^{2\pi} x\cos nxdx\textup{ haciendo }u=nx \\
                &=\frac{1}{2}\int_{0}^{2n\pi}\cos u\frac{du}{n}-\frac{1}{2\pi}\int_0^{2n\pi} \frac{u}{n}\cos u\frac{du}{n} \\
                &=\frac{1}{2n}\sin u\Big|_0^{ 2n\pi}-\frac{1}{2\pi n^2}\int_0^{2n\pi} u\cos udu \\
                &=\frac{1}{2n}\left[\sin 2\pi n-\sin 0 \right] -\frac{1}{2\pi n^2}\left(u\sin u\Big|_{0}^{2n\pi}+\int_0^{2n\pi}\sin udu\right) \\
                &=-\frac{1}{2\pi n^2}\left(u\sin u\Big|_{0}^{2n\pi}+\int_0^{2n\pi}\sin udu\right) \\
                &=-\frac{1}{2\pi n^2}\left(\left[2n\pi\sin 2n\pi-0 \right]-\cos u \Big|_0^{2n\pi}\right) \\
                &-\frac{1}{2\pi n^2}\left(0-0-1+1 \right)\\
                &=0\\
            \end{split}
        \end{equation*}
        para todo $n\geq0$. Si $n\in\mathbb{N}$:
        \begin{equation*}
            \begin{split}
                b_n&=\frac{1}{\pi}\int_{-\pi}^{\pi}f(x)\sin nxdx\\
                &=\frac{1}{\pi}\int_{0}^{2\pi}f(x)\sin nxdx\\
                &=\frac{1}{\pi}\int_{0}^{2\pi}\frac{\pi-x}{2}\sin nxdx\\
                &=\frac{1}{2}\int_{0}^{2\pi}\sin nxdx-\frac{1}{2\pi}\int_0^{2\pi} x\sin nxdx\textup{ haciendo }u=nx \\
                &=\frac{1}{2}\int_{0}^{2n\pi}\sin u\frac{du}{n}-\frac{1}{2\pi}\int_0^{2n\pi} \frac{u}{n}\sin u\frac{du}{n} \\
                &=\frac{1}{2n}(-\cos u)\Big|_{0}^{2n\pi}-\frac{1}{2\pi n^2}\int_0^{2n\pi} u\sin udu \\
                &=\frac{1}{2n}(-\cos 2n\pi+1)-\frac{1}{2\pi n^2}\left(-u\cos u\Big|_{0}^{2n\pi}+\int_{0}^{2n\pi}\cos udu \right) \\
                &=\frac{1}{2n}(-1+1)-\frac{1}{2\pi n^2}\left(-2n\pi\cos 2n\pi+0+\sin u\Big|_{0}^{2n\pi} \right) \\
                &=-\frac{1}{2\pi n^2}\left(-2n\pi\cos 2n\pi+\sin 2n\pi-\sin 0 \right) \\
                &=-\frac{1}{2\pi n^2}\left(-2n\pi+\sin 2n\pi-\sin 0 \right) \\
                &=-\frac{1}{2\pi n^2}\left(-2n\pi \right) \\
                &=\frac{1}{n}\\
            \end{split}
        \end{equation*}
        Por tanto, la serie de Fourier de $f$ en $x\in]0,2\pi[$ está dada por:
        \begin{equation*}
            \frac{a_0}{2}+\sum_{ k=1}^\infty\left[a_k\cos kx+b_k\sin kx \right]=\sum_{ n=1}^\infty b_n\sin nx=\sum_{ n=1}^\infty\frac{\sin nx}{n}
        \end{equation*}
        Por el criterio de Dini se sigue que
        \begin{equation*}
            \frac{\pi-x}{2}=\sum_{ n=1}^\infty\frac{\sin nx}{n},\quad\forall x\in]0,2\pi[
        \end{equation*}

        Ahora, como $x\mapsto\frac{\pi-x}{2}$ es una función en $\mathcal{L}_2^{2\pi}$, por Parseval se tiene que
        \begin{equation*}
            \begin{split}
                \frac{\abs{a_0}^2}{2}+\sum_{n=1}^\infty\left[\abs{a_n}^2+\abs{b_n}^2 \right]&=\frac{1}{\pi}\int_{-\pi}^{\pi}\abs{\frac{\pi-x}{2}}^2\:dx\\
                \Rightarrow \sum_{ n=1}^\infty\frac{1}{n^2}&=\frac{1}{4\pi}\int_0^{2\pi}\abs{\pi-x}^2\:dx\\
                &=\frac{1}{2\pi}\int_0^{\pi}\abs{\pi-x}^2\:dx\\
                &=\frac{1}{2\pi}\int_0^{\pi}(\pi-x)^2\:dx\textup{ haciendo el cambio de variable }u=\pi-x \\
                &=\frac{1}{2\pi}\int_{\pi}^{0}-u^2\:du\\
                &=\frac{1}{2\pi}\cdot\frac{-u^3}{3}\Big|_{\pi}^{0}\\
                &=\frac{1}{2\pi}\cdot[-\frac{0}{3}+\frac{\pi^3}{3} ]\\
                &=\frac{1}{2\pi}\cdot\frac{\pi^3}{3}\\
                &=\frac{\pi^2}{6}\\
                \therefore \sum_{ n=1}^\infty\frac{1}{n^2}&=\frac{\pi^2}{6}\\
            \end{split}
        \end{equation*}
        Como se quería demostrar.
    \end{proof}

    \begin{excer}
        Sea $f\in\mathcal{L}_2^{2\pi}(\mathbb{R})$ y sean $a_n,b_n$ los coeficientes de Fourier de $f$. \textbf{Pruebe} que
        \begin{equation*}
            \frac{1}{\pi}\int_0^{2\pi}xf(x)dx=\pi a_0-2\sum_{ n=1}^\infty \frac{b_n}{n}.
        \end{equation*}
    \end{excer}

    \begin{proof}
        Considere la función $\cf{g}{\mathbb{R}}{\mathbb{R}}$ dada por:
        \begin{equation*}
            g(x)=x,\quad\forall x\in[0,2\pi[
        \end{equation*}
        y extiéndase por periodicidad a todo $\mathbb{R}$. Es claro que $g\in\mathcal{L}_2^{2\pi}(\mathbb{R})$. Si $\left\{c_k=\frac{a_k-ib_k}{2}\right\}_{ k\in\mathbb{Z}}$ y $\left\{d_k=\frac{\alpha_k-i\beta_k}{2}\right\}_{ k\in\mathbb{Z}}$ son los coeficientes de Fourier de $f$ y $g$, respectivamente, al estar ambas funciones en $\mathcal{L}_2^{2\pi}(\mathbb{R})$ se tiene por las identidades de Parserval que
        \begin{equation*}
            \frac{1}{\pi}\int_{-\pi}^{\pi}f(x)\overline{g(x)}=\frac{a_0\overline{\alpha_0}}{2}+\sum_{n=1}^\infty\left[a_n\overline{\alpha_n}+b_n\overline{\beta_n}\right]
        \end{equation*}
        en particular,
        \begin{equation*}
            \begin{split}
                \Rightarrow\frac{1}{\pi}\int_{0}^{\pi}f(x)\overline{g(x)}\:dx&=\frac{a_0\overline{\alpha_0}}{2}+\sum_{n=1}^\infty\left[a_n\overline{\alpha_n}+b_n\overline{\beta_n}\right]\\
                \Rightarrow \frac{1}{\pi}\int_{0}^{2\pi}xf(x)\:dx&=\frac{a_0\overline{\alpha_0}}{2}+\sum_{n=1}^\infty\left[a_n\overline{\alpha_k}+b_n\overline{\beta_n}\right]\\
            \end{split}
        \end{equation*}
        Calculemos los coeficientes de Fourier de $g$. Veamos que
        \begin{equation*}
            \begin{split}
                \alpha_0&=\frac{1}{\pi}\int_{0}^{2\pi}g(x)\:dx\\
                &=\frac{1}{\pi}\int_{0}^{2\pi}x\:dx\\
                &=\frac{1}{2\pi} x^{2}\Big|_{0}^{2\pi}\\
                &=\frac{1}{2\pi} x^{2}\Big|_{0}^{2\pi}\\
                &=2\pi\\
            \end{split}
        \end{equation*}
        y, para $k\in\mathbb{N}$:
        \begin{equation*}
            \begin{split}
                \alpha_k&=\frac{1}{\pi}\int_{0}^{2\pi}x\cos kx\:dx\\
                &=\frac{1}{\pi}\int_{0}^{2k\pi}\frac{u}{k}\cos u\: \frac{du}{k}\\
                &=\frac{1}{k^2\pi}\int_{0}^{2k\pi}u\cos u\:du\\
                &=\frac{1}{k^2\pi}\left[u\sin u+\cos u\Big|_{0}^{2k\pi}\right]\\
                &=\frac{1}{k^2\pi}\left[2k\pi\sin 2k\pi+\cos 2k\pi-\cos0\right]\\
                &=\frac{1}{k^2\pi}\left[2k\pi\sin 2k\pi+\cos 2k\pi-\cos0\right]\\
                &=\frac{1}{k^2\pi}\left[0+1-1\right]\\
                &=0\\
            \end{split}
        \end{equation*}
        y,
        \begin{equation*}
            \begin{split}
                \beta_k&=\frac{1}{\pi}\int_{0}^{2\pi} x\sin kx\:dx\\
                &=\frac{1}{\pi}\int_{0}^{2k\pi} \frac{u}{k}\sin kx\:\frac{du}{k}\\
                &=\frac{1}{k^2\pi}\int_{0}^{2k\pi} u\sin kx\:du\\
                &=\frac{1}{k^2\pi}\left[\sin u-u\cos u\Big|_{0}^{2k\pi}\right]\\
                &=\frac{1}{k^2\pi}\left[\sin 2k\pi-2k\pi\cos 2k\pi-\sin 0+0\cos 0\right]\\
                &=\frac{1}{k^2\pi}\left[0-2k\pi-0+0\right]\\
                &=-\frac{2}{k}\\
            \end{split}
        \end{equation*}
        haciendo el cambio de variable $u=kx$. Por tanto,
        \begin{equation*}
            \begin{split}
                \frac{1}{\pi}\int_{0}^{2\pi}xf(x)\:dx&=\frac{a_0\overline{\alpha_0}}{2}+\sum_{n=1}^\infty\left[a_n\overline{\alpha_n}+b_n\overline{\beta_n}\right]\\
                &=\frac{2\pi a_0}{2}+\sum_{n=1}^\infty\left[a_n\cdot 0+b_n\cdot\left(\frac{-2}{n}\right)\right]\\
                &=\pi a_0-2\sum_{ n=1}^\infty\frac{b_n}{n}\\
                \Rightarrow \frac{1}{\pi}\int_{0}^{2\pi}xf(x)\:dx&=\pi a_0-2\sum_{ n=1}^\infty\frac{b_n}{n}\\
            \end{split}
        \end{equation*}
        como se quería demostrar.
    \end{proof}

    \begin{excer}
        Sea $\cf{f}{\mathbb{R}}{\mathbb{R}}$ periódica de periodo $2\pi$ definida como
        \begin{equation*}
            f(x)=\left\{
                \begin{array}{lcr}
                    \pi^2 & \textup{ si } & -\pi\leq x<0,\\
                    (x-\pi)^2 & \textup{ si } & 0\leq x <\pi.
                \end{array}
            \right.
        \end{equation*}
        calcule los coeficientes de Fourier $a_n$, con $n=0,1,2,...$ de $f$ y \textbf{pruebe} las fórmulas
        \begin{equation*}
            \sum_{ n=1}^\infty\frac{1}{n^2}=\frac{\pi^2}{6}\quad\textup{y}\quad\sum_{ n=1}^\infty\frac{(-1)^{ n-1}}{n^2}=\frac{\pi^2}{12}.
        \end{equation*}
    \end{excer}

    \begin{sol}
        Primero determinemos los coeficientes de Fourier de $f$.
        \begin{equation*}
            \begin{split}
                a_0&=\frac{1}{\pi}\int_{ -\pi}^\pi f(x)\:dx\\
                &=\frac{1}{\pi}\left[\int_{ -\pi}^0 f(x)\:dx+\int_{0}^\pi f(x)\:dx\right] \\
                &=\frac{1}{\pi}\left[\int_{ -\pi}^0 \pi^2\:dx+\int_{0}^\pi (x-\pi)^2\:dx\right] \\
                &=\frac{1}{\pi}\left[\pi^3+\int_{-\pi}^0 u^2\:du\right] \\
                &=\frac{1}{\pi}\left[\pi^3+\int_{-\pi}^0 u^2\:du\right] \\
                &=\frac{1}{\pi}\left[\pi^3+ \frac{u^3}{3}\Big|_{-\pi}^0 \right] \\
                &=\frac{1}{\pi}\left[\pi^3+\frac{\pi^3}{3}\right] \\
                &=\frac{4\pi^2}{3}
            \end{split}
        \end{equation*}
        ahora, para $n\in\mathbb{N}$:
        \begin{equation*}
            \begin{split}
                a_n&=\frac{1}{\pi}\int_{ -\pi}^{\pi}f(x)\cos nx\:dx\\
                &=\frac{1}{\pi}\left[\int_{ -\pi}^{0}f(x)\cos nx\:dx+\int_{0}^{\pi}f(x)\cos nx\:dx\right] \\
                &=\frac{1}{\pi}\left[\int_{ -\pi}^{0}\pi^2\cos nx\:dx+\int_{0}^{\pi}(x-\pi)^2\cos nx\:dx\right]\\
                &=\frac{1}{\pi}\left[\int_{ -\pi}^{0}\pi^2\cos nx\:dx+\int_{0}^{\pi}(x^2-2x\pi+\pi^2)\cos nx\:dx\right]\\
                &=\frac{1}{\pi}\left[\int_{ -\pi}^{\pi}\pi^2\cos nx\:dx+\int_{0}^{\pi}x^2\cos nx\:dx-\int_{0}^{\pi}2x\pi\cos nx\:dx\right]\\
                &=\frac{1}{\pi}\left[\int_{ -\pi}^{\pi}\pi^2\cos nx\:dx+\int_{0}^{\pi}x^2\cos nx\:dx-\int_{0}^{\pi}2x\pi\cos nx\:dx\right]\\
                &=\frac{1}{\pi}\left[\frac{\pi^2}{n}\int_{ -n\pi}^{n\pi}\cos u\:du+\frac{1}{n}\int_{0}^{n\pi}\left(\frac{u}{n}\right)^2 \cos u\:du-\frac{1}{n}\int_{0}^{n\pi}2\left(\frac{u}{n}\right)\pi\cos u\:du\right]\\
                &=\frac{1}{n\pi}\left[\pi^2\int_{ -n\pi}^{n\pi}\cos u\:du+\frac{1}{n^2}\int_{0}^{n\pi}u^2 \cos u\:du-\frac{2\pi}{n}\int_{0}^{n\pi}u\cos u\:du\right]\\
            \end{split}
        \end{equation*}
        donde
        \begin{equation*}
            \pi^2\int_{ -n\pi}^{n\pi}\cos u\:du=2\sin\left(n\pi\right)
        \end{equation*}
        con
        \begin{equation*}
            \int_{0}^{n\pi}u^2 \cos u\:du=\left(\pi^2n^2-2 \right)\sin\left(n\pi\right)+2n\pi\cos (n\pi)
        \end{equation*}
        y
        \begin{equation*}
            \int_{0}^{n\pi}u\cos u\:du=n\pi\sin (n\pi)+\cos(n\pi)-1
        \end{equation*}
        por tanto,
        \begin{equation*}
            \begin{split}
                a_n&=\frac{1}{n\pi}\left[\pi^2\int_{ -n\pi}^{n\pi}\cos u\:du+\frac{1}{n^2}\int_{0}^{n\pi}u^2 \cos u\:du-\frac{2\pi}{n}\int_{0}^{n\pi}u\cos u\:du\right]\\
                &=\frac{1}{n\pi}\left[\pi^2\cdot2\sin\left(n\pi\right)+\frac{1}{n^2}\cdot\left[\left(\pi^2n^2-2 \right)\sin\left(n\pi\right)+2n\pi\cos (n\pi)\right]-\frac{2\pi}{n}\cdot\left[n\pi\sin(n\pi)+\cos (n\pi)-1\right]\right]\\
                &=\frac{1}{n\pi}\left[2\pi^2\sin\left(n\pi\right)+\pi^2\sin (n\pi)-\frac{2}{n^2}\sin\left(n\pi\right)+\frac{2\pi}{n}\cos (n\pi)-2\pi^2\sin (n\pi)-\frac{2\pi}{n}\cos (n\pi)+\frac{2\pi}{n}\right]\\
                &=\frac{1}{n\pi}\left[\pi^2\sin (n\pi)-\frac{2}{n^2}\sin\left(n\pi\right)+\frac{2\pi}{n}\right]\\
                &=\frac{1}{n\pi}\left[\frac{n^3\pi^2\sin (n\pi)+2n^2\pi-2n\sin(n\pi)}{n^3}\right]\\
                &=\frac{1}{n\pi}\left[\frac{0+2n^2\pi-0}{n^3}\right]\\
                &=\frac{1}{n\pi}\left[\frac{2n^2\pi}{n^3}\right]\\
                &=\frac{2}{n^2}
            \end{split}
        \end{equation*}
        pues, $\sin(n\pi)=0$ para todo $n\in\mathbb{N}$. Notemos que la función $f$ es monotóna decreciente, en particular es de variación acotada. Luego, por el teorema de Jordan al ser $f$ continua en $]-\pi,\pi]$ se sigue que la serie de Fourier de $f$ en $x$ converge a $f(x)$ para todo $x\in]-\pi,\pi]$. Esto es:
        \begin{equation*}
            f(x)=\frac{a_0}{2}+\sum_{ k=1}^\infty\left[a_k\cos kx+b_k\sin kx \right]
        \end{equation*}
        en particular, en $x=0$:
        \begin{equation*}
            \begin{split}
                f(0)&=\frac{a_0}{2}+\sum_{ k=1}^\infty a_k\\
                &=\frac{2\pi^2}{3}+\sum_{ n=1}^\infty\frac{2}{n^2}\\
                \Rightarrow \sum_{ n=1}^\infty\frac{1}{n^2}&=\frac{1}{2}\left[\pi^2-\frac{2\pi^2}{3}\right] \\
                &=\frac{1}{2}\cdot\frac{\pi^2}{3}\\
                &=\frac{\pi^2}{6}\\
            \end{split}
        \end{equation*}
        
        Para la segunda parte, notemos que
        \begin{equation*}
            \begin{split}
                \sum_{ n=1}^\infty\frac{1}{n^2}&=\sum_{ n=1}^\infty\left[\frac{1}{(2n-1)^2}+\frac{1}{(2n)^2} \right]\\
                &=\sum_{ n=1}^\infty\frac{1}{(2n-1)^2}+\frac{1}{4}\sum_{ n=1}^\infty\frac{1}{n^2}\\
                \Rightarrow \sum_{ n=1}^\infty\frac{1}{(2n-1)^2}&=\frac{3}{4}\sum_{ n=1}^\infty\frac{1}{n^2}\\
                &=\frac{3}{4}\cdot\frac{\pi^2}{6}\\
                &=\frac{\pi^2}{8}\\
            \end{split}
        \end{equation*}
        Por ende,
        \begin{equation*}
            \begin{split}
                \sum_{ n=1}^\infty\frac{(-1)^{ n-1}}{n^2}&=\sum_{ n=1}^\infty\frac{(-1)^{ 2n-1-1}}{(2n-1)^2}+\sum_{ n=1}^\infty\frac{(-1)^{ 2n-1}}{(2n)^2}\\
                &=\sum_{ n=1}^\infty\frac{(-1)^{ 2n-2}}{(2n-1)^2}+\frac{1}{4}\sum_{ n=1}^\infty\frac{-1}{n^2}\\
                &=\sum_{ n=1}^\infty\frac{1}{(2n-1)^2}-\frac{1}{4}\sum_{ n=1}^\infty\frac{1}{n^2}\\
                &=\frac{\pi^2}{8}-\frac{1}{4}\cdot\frac{\pi^2}{6}\\
                &=\frac{\pi^2}{8}-\frac{\pi^2}{24}\\
                &=\frac{\pi^2\left[3-1\right]}{24}\\
                &=\frac{\pi^2}{12}\\
            \end{split}
        \end{equation*}
    \end{sol}

    \begin{excer}
        \textbf{Pruebe} que
        \begin{equation*}
            \frac{1}{3}x(\pi-x)(\pi-2x)=\sum_{ n=1}^\infty\frac{\sin 2nx}{n^3},\quad 0\leq x\leq \pi.
        \end{equation*}
        \textbf{Deduzca} el valor de
        \begin{equation*}
            \sum_{ n=1}^\infty\frac{(-1)^{ n-1}}{(2n-1)^3}.
        \end{equation*}
    \end{excer}

    \begin{proof}
        Considere la función $\cf{g}{\mathbb{R}}{\mathbb{R}}$ dada por:
        \begin{equation*}
            g(x)=\frac{1}{3}x(\pi-x)(\pi-2x),\quad\forall x\in [0,\pi[
        \end{equation*}
        y extiéndase por periodicidad a todo $\mathbb{R}^+$ y hágase
        \begin{equation*}
            g(x)=-g(-x),\quad\forall x\in\mathbb{R}^-
        \end{equation*}. Afirmamos que $g$ es continua en $\mathbb{R}$. En efecto, ya se tiene que $g$ es continua en $]0,\pi[$, por lo cual para ver que es continua en $\mathbb{R}$ basta con ver que $g(0)=g(\pi)$ (en particular se tiene que $g$ es $\pi$-periódica, luego también es $2\pi$-periódica). Veamos que
        \begin{equation*}
            g(0)=0=g(\pi)
        \end{equation*}
        luego, $g$ es continua en $\mathbb{R}$. Además, $g$ es $C^1$ (e casi todo $\mathbb{R}$), luego de variación acotada en $[-\pi,\pi]$. Por el Teorema de Jordan, la serie de Fourier de $g$ converge a $g$ uniformemente en $\mathbb{R}$, en particular lo hace puntualmente en el intervalo $[0,\pi]$. Calculemos los coeficientes de Fourier de $g$.

        Como $g$ se definió de tal forma que fuese impar, se sigue que
        \begin{equation*}
            \begin{split}
                a_k=0,\quad\forall k\in\mathbb{N}
            \end{split}
        \end{equation*}
        y, para $k\in\mathbb{N}$:
        \begin{equation*}
            \begin{split}
                b_k&=\frac{2}{\pi}\int_{0}^{\pi}g(x)\sin kx\:dx \\
                &=\frac{2}{\pi}\int_{0}^{\pi}\frac{1}{3}x(\pi-x)(\pi-2x) \sin kx\:dx\\
                &=\frac{2}{3\pi}\int_{0}^{\pi}(\pi^2x-3\pi x^2+2x^3)\sin kx\:dx\\
                &=\frac{2}{3\pi}\left[\pi^2\int_{0}^{\pi}x\sin kx\:dx-3\pi\int_{0}^{\pi}x^2\sin kx\:dx+2\int_{0}^{\pi}x^3\sin kx\:dx \right]\\
                &=\frac{2}{3\pi}\left[\pi^2\left[\frac{\sin k\pi-k\pi\cos k\pi}{k^2} \right] -3\pi\left[\frac{(2-k^2\pi^2)\cos k\pi+2k\pi\sin k\pi-2}{k^3}\right]\right.\\
                &\left.+2\left[\frac{3(k^2\pi^2-2)\sin k\pi+k\pi(6-k^2\pi^2)\cos k\pi}{k^4}\right]\right]\\
                &=\frac{2}{3\pi}\left[\pi^2\left[\frac{-k\pi(-1)^k}{k^2} \right] -3\pi\left[\frac{(2-k^2\pi^2)(-1)^k-2}{k^3}\right]+2\left[\frac{k\pi(6-k^2\pi^2)(-1)^k}{k^4}\right]\right]\\
                &=\frac{2}{3\pi}\left[\frac{\pi^3(-1)^{k+1}}{k}+\frac{3\pi(2-k^2\pi^2)(-1)^{k+1}+6\pi}{k^3}+\frac{2k\pi(6-k^2\pi^2)(-1)^k}{k^4}\right]\\
                &=\frac{2}{3\pi}\left[\frac{\pi^3k^3(-1)^{k+1}}{k^4}+\frac{3k\pi(2-k^2\pi^2)(-1)^{k+1}+6k\pi}{k^4}+\frac{2k\pi(6-k^2\pi^2)(-1)^k}{k^4}\right]\\
                &=\frac{2}{3k^4\pi}\left[\pi^3k^3(-1)^{k+1}+3k\pi(2-k^2\pi^2)(-1)^{k+1}+6k\pi+2k\pi(6-k^2\pi^2)(-1)^k\right]\\
                &=\frac{2}{3k^4\pi}\left[\pi^3k^3(-1)^{k+1}+6k\pi(-1)^{k+1}+3k^3\pi^3(-1)^{k}+6k\pi+12k\pi(-1)^k+2k^3\pi^3(-1)^{ k+1}\right]\\
                &=\frac{2}{3k^4\pi}\left[3k^3\pi^3(-1)^{k+1}+3k^3\pi^3(-1)^{k}+6k\pi+6k\pi(-1)^k\right]\\
                &=\frac{2}{3k^4\pi}\left[6k\pi+6k\pi(-1)^k\right]\\
                &=\frac{4}{k^3}\left[1+(-1)^k\right]\\
            \end{split}
        \end{equation*}
        si $k=2m-1$ con $m\in\mathbb{N}$:
        \begin{equation*}
            \begin{split}
                b_{ 2m-1}&=\frac{4}{(2m-1)^3}\left[1+(-1)^{2m-1}\right]\\
                &=\frac{4}{(2m-1)^3}\left[1-1\right]\\
                &=0\\
            \end{split}
        \end{equation*}
        y,si $k=2m$ con $m\in\mathbb{N}$:
        \begin{equation*}
            \begin{split}
                b_{2m}&=\frac{4}{(2m)^3}\left[1+(-1)^{2m}\right]\\
                &=\frac{8}{8m^3}\\
                &=\frac{1}{m^3}\\
            \end{split}
        \end{equation*}
        Por ende, se tiene que
        \begin{equation*}
            \begin{split}
                \frac{1}{3}x(\pi-x)(\pi-2x)&=\sum_{n=1}^\infty b_{ 2n}\sin 2nx\\
                &=\sum_{n=1}^\infty\frac{\sin 2nx}{n^3},\quad\forall 0\leq x\leq\pi \\
            \end{split}
        \end{equation*}
        Para la otra parte, recuerde que
        \begin{equation*}
            \sin\frac{n\pi}{2}=\left\{
                \begin{array}{lcr}
                    0 & \textup{ si } & n=2m\textup{ para algún }m\in\mathbb{N}\\
                    (-1)^{m-1} & \textup{ si } & n=2m-1\textup{ para algún }m\in\mathbb{N} \\
                \end{array}
            \right.
        \end{equation*}
        Tomemos $x=\frac{\pi}{4}$, se tiene que
        \begin{equation*}
            \begin{split}
                \frac{1}{3}\cdot\frac{\pi}{4}\cdot(\pi-\frac{\pi}{4})(\pi-\frac{\pi}{2})&=\sum_{ n=1}^{\infty}\frac{\sin\frac{n\pi}{2}}{n^3}\\
                \Rightarrow \frac{1}{3}\cdot\frac{\pi}{4}\cdot\frac{3\pi}{4}\cdot\frac{\pi}{2}&=\sum_{ n=1}^{\infty }\frac{(-1)^{ n-1}}{(2n-1)^3}\\
                \Rightarrow \sum_{ n=1}^{\infty }\frac{(-1)^{ n+1}}{(2n-1)^3}&=\frac{\pi^3}{32}\\
            \end{split}
        \end{equation*}
    \end{proof}

    \renewcommand{\theenumi}{\textbf{\roman{enumi}}}
    
    \begin{excer}
        Haga lo siguiente:
        \begin{enumerate}
            \item \textbf{Pruebe} que
            
            \begin{equation*}
                \int_0^\pi\log\sin\frac{x}{2}dx=-\pi\log2.
            \end{equation*}
            \textit{Sugerencia}. Haga el cambio de variables $x=2t$ y escriba $\sin t =2\sin\frac{t}{2}\cos\frac{t}{2}$.
            \item \textbf{Muestre} que
            \begin{equation*}
                -\log\Big|2\sin\frac{x}{2}\Big|=\sum_{ n=1}^\infty\frac{\cos nx}{n},\quad\textup{ si }x\neq 2k\pi,k\in\mathbb{Z}.
            \end{equation*}
            \textit{Sugerencia.} Use el inciso (i) para probar que $a_0=0$. A fin de calcular $a_n$ para $n\in\mathbb{N}$, escriba $a_n=\frac{2}{\pi}\int_0^\pi\log\cos\frac{x}{2}dx$, efectúe una integración por partes y transforme el nuevo integrando de suerte que aparezca el núcleo de Dirichlet.
            \item \textbf{Deduzca} de (ii) la fórmula
            \begin{equation*}
                \log2=\sum_{ n=1}^\infty\frac{(-1)^{ n-1}}{n}.
            \end{equation*}
            \item \textbf{Desarrolle} en serie de Fourier la función
            \begin{equation*}
                x\mapsto\log\Big|2\cos\frac{x}{2}\Big|
            \end{equation*}
        \end{enumerate}
    \end{excer}

    \begin{sol}
        De (i): (justificar porqué esa función es integrable). Veamos que
        \begin{equation*}
            \begin{split}
                \int_0^\pi\log\sin\frac{x}{2}\:dx&=2\int_0^{\frac{\pi}{2}}\log\sin t\:dt\\
                &=2\int_0^{\frac{\pi}{2}}\log\left(2\sin
                \frac{t}{2}\cos\frac{t}{2}\right) \:dt\\
                &=2\int_0^{\frac{\pi}{2}}\left[\log2+\log\sin\frac{t}{2}+\log\cos\frac{t}{2} \right]  \:dt\\
                &=2\int_0^{\frac{\pi}{2}}\left[\log2+\log\sin\frac{t}{2}+\log\cos\frac{t}{2} \right]  \:dt\\
                &=2\int_0^{\frac{\pi}{2}}\log2\:dt+2\int_0^{\frac{\pi}{2}}\log\sin\frac{t}{2}\:dt+2\int_0^{\frac{\pi}{2}}\log\cos\frac{t}{2}\:dt\\
                &=\pi\log2+2\int_0^{\frac{\pi}{2}}\log\sin\frac{t}{2}\:dt+2\int_0^{\frac{\pi}{2}}\log\cos\frac{t}{2}\:dt\\
            \end{split}
        \end{equation*}
        donde,
        \begin{equation*}
            \begin{split}
                \int_0^{\frac{\pi}{2}}\log\cos\frac{t}{2}\:dt&=\int_0^{\frac{\pi}{2}}\log\sin\left(\frac{\pi}{2}-\frac{t}{2}\right) \:dt\\
                &=\int_{\pi}^{\frac{\pi}{2}}-\log\sin\left(\frac{u}{2}\right) \:dt\\
                &=\int_{\frac{\pi}{2}}^{\pi}\log\sin\left(\frac{u}{2}\right) \:dt\\
            \end{split}
        \end{equation*}
        haciendo el cambio de variable $u=\pi-t$. Por ende,
        \begin{equation*}
            \begin{split}
                \int_0^\pi\log\sin\frac{x}{2}\:dx&=\pi\log2+2\int_0^{\frac{\pi}{2}}\log\sin\frac{t}{2}\:dt+2\int_{\frac{\pi}{2}}^{\pi}\log\sin\frac{u}{2}\:du\\
                &=\pi\log2+2\int_0^{\pi}\log\sin\frac{x}{2}\:dx\\
                \Rightarrow \int_0^\pi\log\sin\frac{x}{2}\:dx&=-\pi\log2\\
            \end{split}
        \end{equation*}
        
        De (ii): Por lo anterior, $f\in\mathcal{L}_1^{2\pi}$ donde $f(x)=-\log\abs{2\sin\frac{x}{2}}$ para todo $x\in\mathbb{R}\backslash2\pi\mathbb{Z}$. Ahora, como $f$ es par, se tiene que
        \begin{equation*}
            b_n=0,\quad\forall n\in\mathbb{N}
        \end{equation*}
        Ahora, veamos que
        \begin{equation*}
            \begin{split}
                a_0&=\frac{2}{\pi}\int_0^\pi f(x)\:dx\\
                &=-\frac{2}{\pi}\int_0^\pi -\log\abs{2\sin\frac{x}{2}}\:dx\\
                &=\frac{2}{\pi}\int_0^\pi \left[\log2+\log\sin\frac{x}{2}\right]\:dx\\
                &=\frac{2}{\pi}\left[\int_0^\pi\log2+\int_0^\pi\log\sin\frac{x}{2}\right]\:dx\\
                &=\frac{2}{\pi}\left[\pi\log2-\pi\log2\right]\\
                &=0\\
            \end{split}
        \end{equation*}
        Ahora, si $n\in\mathbb{N}$:
        \begin{equation*}
            \begin{split}
                a_n&=-\frac{2}{\pi}\int_0^{\pi}\log\abs{2\sin\frac{x}{2}}\cos nx\:dx\\
                &=-\frac{2}{\pi}\int_0^{\pi}\log\left(2\sin\frac{x}{2}\right)\cos nx\:dx\\
                &=\frac{2}{\pi}\left[-\log\left(2\sin\frac{x}{2}\right)\frac{1}{n}\sin nx\Big|_0^{\pi}+\frac{1}{2n}\int_0^{\pi}\frac{\cos\frac{x}{2}\sin nx}{\sin\frac{x}{2}}\:dx \right]\\
                &=\frac{1}{\pi n}\int_0^{\pi}\frac{\cos\frac{x}{2}\sin nx}{\sin\frac{x}{2}}\:dx
            \end{split}
        \end{equation*}
        pero,
        \begin{equation*}
            \sin A\cos B=\frac{1}{2}\left[\sin(A+B)+\sin(A-B) \right]
        \end{equation*}
        Por ende,
        \begin{equation*}
            \begin{split}
                a_n&=\frac{1}{2\pi n}\int_0^{\pi}\frac{\sin\left(n+\frac{1}{2}\right)x+\sin\left(n-\frac{1}{2}\right)x}{\sin\frac{x}{2}}\:dx\\
                &=\frac{1}{n}\int_0^{\frac{\pi}{2}}\left[D_n(x)+D_{ n-1}(x)\right]\:dx\\
                &=\frac{1}{n}\\
            \end{split}
        \end{equation*}

        De (iii): Veamos la convergencia (usar el teorema de Carleson y más cosas), de donde se deduce el hecho sorprendente que
        \begin{equation*}
            \int_0^{\pi}\left(\log\abs{2\cos\frac{x}{2}} \right)^2\:dx=\frac{\pi^2}{6}
        \end{equation*}
    \end{sol}

    \begin{excer}
        Sea $f\in\mathcal{L}_1^{2\pi}(\mathbb{R})$ y sea $x\in\mathbb{R}$. Se supone que para algúun $\alpha>0$ se cumple
        \begin{equation*}
            f(x+t)-f(x)=O(\abs{t^\alpha}),\quad\textup{cuando}\quad t\rightarrow 0
        \end{equation*}
        \textbf{Demuestre} que la serie de Fourier de $f$ en $x$ converge a $f(x)$.
    \end{excer}

    \begin{proof}
        Como
        \begin{equation*}
            f(x+t)-f(x)=O(\abs{t^\alpha}),\quad\textup{cuando}\quad t\rightarrow 0
        \end{equation*}
        entonces existe $A>0$ y $\delta>0$ tales que
        \begin{equation*}
            \begin{split}
                \abs{t}<\delta&\Rightarrow \abs{f(x+t)-f(x)}\leq A\abs{t}^\alpha\\
                &\Rightarrow -A\abs{t}^\alpha\leq f(x+t)-f(x)\leq A\abs{t}^\alpha\\
            \end{split}
        \end{equation*}
        Para ver que la serie de Fourier de $f$ en $x$ converge a $f(x)$, usaremos el Teorema Fundamental para la convergencia puntual de una serie de Fourier. Ahora, si $m\in\mathbb{N}$
        \begin{equation*}
            \begin{split}
                -A\frac{\abs{t}^\alpha}{t}\sin\left(m+\frac{1}{2}\right)t\:dt \leq \frac{f(x+t)-f(x)}{t}\sin\left(m+\frac{1}{2}\right)t\:dt\leq A\frac{\abs{t}^\alpha}{t}\sin\left(m+\frac{1}{2}\right)t\:dt
            \end{split}
        \end{equation*}
        Para todo $0<\abs{t}<\min\left\{\delta,\pi\right\}$. Considere ahora la función $t\mapsto\frac{\abs{t}^\alpha}{t}$ (llamémosla g) definida c.t.p. en $\mathbb{R}$. Esta función es la diferencia de dos funciones monótonas en $[-\pi,\pi]$, luego de variación acotada. Además es integrable. En efecto,
        \begin{equation*}
            \abs{g(t)}=\abs{t}^{\alpha-1}
        \end{equation*}
        donde $t\mapsto \abs{t}^{\alpha-1}$ es integrable en $[-\pi,\pi]$ pues $\alpha-1>-1$. Luego, por el Teorema de Jordan la serie de Fourier de $g$ en $x$ converge a $g(x)$. Así, por el Teorema Fundamnetal para la convergencia puntual de una serie de Fourier, se tiene que existe $0<\gamma<\pi$ tal que
        \begin{equation*}
            \lim_{ m\rightarrow\infty}\int_{-\gamma'}^{\gamma'}\frac{\abs{t}^\alpha}{t}\sin\left(m+\frac{1}{2}\right)t\:dt=0
        \end{equation*}
        Afirmamos que si $0<\zeta\leq\gamma'$, entonces
        \begin{equation*}
            \lim_{ m\rightarrow\infty}\int_{-\zeta}^{\zeta}\frac{\abs{t}^\alpha}{t}\sin\left(m+\frac{1}{2}\right)t\:dt=0
        \end{equation*}
        En efecto, sea $0<\zeta\leq\gamma'$, se tiene para $m\in\mathbb{N}$:
        \begin{equation*}
            \begin{split}
                \int_{-\zeta}^{\zeta}\frac{\abs{t}^\alpha}{t}\sin\left(m+\frac{1}{2}\right)t\:dt&=\int_{-\zeta}^{0}\frac{\abs{t}^\alpha}{t}\sin\left(m+\frac{1}{2}\right)t\:dt+\int_{0}^{\zeta}\frac{\abs{t}^\alpha}{t}\sin\left(m+\frac{1}{2}\right)t\:dt\\
                &=-\int_{\zeta}^{0}\frac{\abs{-t}^\alpha}{-t}\sin\left(m+\frac{1}{2}\right)(-t)\:dt+\int_{0}^{\zeta}\frac{\abs{t}^\alpha}{t}\sin\left(m+\frac{1}{2}\right)t\:dt\\
                &=\int_{0}^{\zeta}\frac{\abs{t}^\alpha}{t}\sin\left(m+\frac{1}{2}\right)t\:dt+\int_{0}^{\zeta}\frac{\abs{t}^\alpha}{t}\sin\left(m+\frac{1}{2}\right)t\:dt\\
                &=2\int_{0}^{\zeta}\frac{\abs{t}^\alpha}{t}\sin\left(m+\frac{1}{2}\right)t\:dt\\
            \end{split}
        \end{equation*}
        %TODO
        Tomemos $\delta'=\min\left\{\delta,\gamma\right\}$. Se tiene que
        \begin{equation*}
            -A\frac{\abs{t}^\alpha}{t}\sin\left(m+\frac{1}{2}\right)t\:dt \leq \frac{f(x+t)-f(x)}{t}\sin\left(m+\frac{1}{2}\right)t\:dt\leq A\frac{\abs{t}^\alpha}{t}\sin\left(m+\frac{1}{2}\right)t\:dt
        \end{equation*}
        para todo $\abs{t}<\delta'$
    \end{proof}

    \begin{excer}
        Por el problema \textbf{3.1.1} se sabe que
        \begin{equation*}
            \frac{\pi-x}{2}=\sum_{ n=1}^\infty\frac{\sin nx}{n},\quad\forall x\in]0,2\pi[
        \end{equation*}
        \begin{enumerate}
            \item Póngase
            \begin{equation*}
                s_n(x)=\sum_{k=1}^n \frac{\sin kx}{k}.
            \end{equation*}
            \textbf{Muestre} que
            \begin{equation*}
                \frac{x}{2}+s_n(x)=\pi\int_0^x D_n(t)dt,
            \end{equation*}
            donde $D_n$ es el núcleo de Dirichlet.
            \item Si $x\in]0,2\pi[$, \textbf{pruebe} que
            \begin{equation*}
                \lim_{ n\rightarrow\infty}\left[\pi\int_0^x D_n(t)dt- \int_0^x\frac{\sin nt}{t}dt \right]=0.
            \end{equation*}
            \item \textbf{Deduzca} una nueva demostración de la fórmula
            \begin{equation*}
                \int_0^{\rightarrow\infty}\frac{\sin t}{t}dt=\frac{\pi}{2}.
            \end{equation*}
        \end{enumerate}
    \end{excer}

    \begin{proof}
        De (i): Recordemos que el núcleo de Dirichlet está dado por:
        \begin{equation*}
            D_n(x)=\frac{1}{2\pi}\sum_{ k=-m}^m e^{ ikx}
        \end{equation*}
        para todo $n\in\mathbb{N}$. Por ende,
        \begin{equation*}
            \begin{split}
                \int_0^x D_n(t)\:dt&=\int_0^x \frac{1}{2\pi}\sum_{ k=-m}^m e^{ ikt}\:dt\\
                &=\frac{1}{2\pi}\int_0^x \left[1+\sum_{ k=-m,k\neq0}^m e^{ ikt}\right] \:dt\\
                &=\frac{1}{2\pi}\left[t\Big|_0^x+\sum_{ k=-m,k\neq0}^m \frac{e^{ ikt}}{ik}\Big|_0^{x} \right]\\
                &=\frac{1}{2\pi}\left[x+\sum_{ k=-m,k\neq0}^m \frac{e^{ ikx}-1}{ik} \right]\\
                &=\frac{1}{2\pi}\left[x+\sum_{ k=-m,k\neq0}^m \frac{e^{ ikx}}{ik}-\underbrace{\sum_{ k=-m,k\neq0}^m \frac{1}{ik}}_{=0} \right]\\
                &=\frac{1}{2\pi}\left[x+\sum_{ k=-m,k\neq0}^m \frac{\cos ikx+\sin ikx}{ik}\right]\\
                &=\frac{1}{2\pi}\left[x+\sum_{ k=-m}^{-1} \frac{\cos ikx+\sin ikx}{ik}+\sum_{ k=1}^m \frac{\cos ikx+i\sin ikx}{ik}\right]\\
                &=\frac{1}{2\pi}\left[x+\sum_{ k=1}^m \frac{\cos (-ikx)+i\sin (-ikx)}{-ik}+\sum_{ k=1}^m \frac{\cos ikx+i\sin ikx}{ik}\right]\\
                &=\frac{1}{2\pi}\left[x+\sum_{ k=1}^m \frac{-\cos ikx+i\sin ikx}{ik}+\sum_{ k=1}^m \frac{\cos ikx+i\sin ikx}{ik}\right]\\
                &=\frac{1}{2\pi}\left[x+2\sum_{ k=1}^m \frac{\sin ikx}{k}\right]\\
                &=\frac{x}{2\pi}+\frac{1}{\pi}\sum_{ k=1}^m \frac{\sin ikx}{k}\\
            \end{split}
        \end{equation*}
        luego,
        \begin{equation*}
            \pi\int_0^x D(t)\:dt=\frac{x}{2}+s_n(x)
        \end{equation*}

        De (ii): Recordemos que podemos escribir al Núcleo de Dirichlet como:
        \begin{equation*}
            D_m(t)=\frac{1}{2\pi}\cdot\frac{\sin\left(m+\frac{1}{2} \right)t}{\sin\frac{t}{2}},\quad\forall m\in\mathbb{N}
        \end{equation*}
        Por tanto, si $m\in\mathbb{N}$ y $x\in]0,2\pi[$:
        \begin{equation*}
            \begin{split}
                \abs{\pi\int_0^x D_m(t)\:dt-\int_{0}^{x}\frac{\sin mt}{t}\:dt}&=\abs{\frac{1}{2}\int_{0}^{x} \frac{\sin\left(m+\frac{1}{2} \right)t}{\sin\frac{t}{2}}\:dt-\int_{0}^{x}\frac{\sin mt}{t}\:dt}\\
                &=\abs{\frac{1}{2}\int_{0}^{x} \frac{\sin mt\cos\frac{t}{2}+\sin\frac{t}{2}\cos mt}{\sin\frac{t}{2}}\:dt-\int_{0}^{x}\frac{\sin mt}{t}\:dt}\\
                &=\abs{\int_{0}^{x}\left[\frac{\sin mt}{2\tan\frac{t}{2}}+\frac{1}{2}\cos mt\right]\:dt-\int_{0}^{x}\frac{\sin mt}{t}\:dt}\\
                &=\abs{\frac{1}{2}\int_{0}^{x}\cos mt\:dt+\int_{0}^{x}\left[\frac{\sin mt}{2\tan\frac{t}{2}}-\frac{\sin mt}{t}\right]\:dt}\\
                &=\abs{\frac{1}{2m}\int_{0}^{mx}\cos u\:du+\int_{0}^{x}\left[\frac{1}{2\tan\frac{t}{2}}-\frac{1}{t}\right]\sin mt\:dt}\\
                &\leq\frac{1}{2m}\abs{\sin u\Big|_{0}^{mx}}+\abs{\int_{0}^{x}\left[\frac{1}{2\tan\frac{t}{2}}-\frac{1}{t}\right]\sin mt\:dt}\\
                &\leq\frac{1}{2m}\abs{\sin mx}+\abs{\int_{0}^{x}\left[\frac{1}{2\tan\frac{t}{2}}-\frac{1}{t}\right]\sin mt\:dt}\\
                &\leq\frac{1}{2m}+\abs{\int_{0}^{x}\left[\frac{1}{2\tan\frac{t}{2}}-\frac{1}{t}\right]\sin mt\:dt}\\
            \end{split}
        \end{equation*}
        Pero, la función
        \begin{equation*}
            t\mapsto \frac{1}{2\tan\frac{t}{2}}-\frac{1}{t}
        \end{equation*}
        es integrable en $]0,x[$. En efecto, como es continua en $[0,x]$ haciendo que valga $0$ en $t=0$, ya que
        \begin{equation*}
            \lim_{t\rightarrow 0}\left[\frac{1}{2\tan\frac{t}{2}}-\frac{1}{t}\right]=0
        \end{equation*}
        %TODO
        hace que la función sea continua, luego al ser continua en un compacto es integrable. Así, por el teorema de Riemman-Lebesgue se sigue que
        \begin{equation*}
            \lim_{ m\rightarrow\infty}\int_{0}^{x}\left[\frac{1}{2\tan\frac{t}{2}}-\frac{1}{t}\right]\sin mt\:dt=0
        \end{equation*}
        Por tanto, para $\varepsilon>0$ existe $N\in\mathbb{N}$ tal que si $m\geq N$:
        \begin{equation*}
            \frac{1}{2m}\leq\frac{\varepsilon}{2}\quad\textup{y}\quad\abs{\int_{0}^{x}\left[\frac{1}{2\tan\frac{t}{2}}-\frac{1}{t}\right]\sin mt\:dt}<\frac{\varepsilon}{2}
        \end{equation*}
        luego,
        \begin{equation*}
            \begin{split}
                m\geq N\Rightarrow\abs{\pi\int_0^x D_m(t)\:dt-\int_{0}^{x}\frac{\sin mt}{t}\:dt}&\leq\frac{1}{2m}+\abs{\int_{0}^{x}\left[\frac{1}{2\tan\frac{t}{2}}-\frac{1}{t}\right]\sin mt\:dt}\\
                &<\frac{\varepsilon}{2}+\frac{\varepsilon}{2}\\
                &=\varepsilon\\
            \end{split}
        \end{equation*}
        así,
        \begin{equation*}
            \lim_{ m\rightarrow\infty}\abs{\pi\int_0^x D_m(t)\:dt-\int_{0}^{x}\frac{\sin mt}{t}\:dt}=0
        \end{equation*}

        De (iii): Como dado $x\in]0,2\pi[$ se tiene que
        \begin{equation*}
            \frac{\pi-x}{2}=\sum_{ n=1}^\infty\frac{\sin nx}{n}
        \end{equation*}
        y,
        \begin{equation*}
            \lim_{ m\rightarrow\infty}\abs{\pi\int_0^x D_m(t)\:dt-\int_{0}^{x}\frac{\sin mt}{t}\:dt}=0
        \end{equation*}
        entonces para $x\in ]0,2\pi[$ y $\varepsilon>0$ existe $N\in\mathbb{N}$ tal que $m\geq N$ implica que
        \begin{equation*}
            \abs{\frac{\pi-x}{2}-\sum_{ n=1}^m\frac{\sin nx}{n}}<\frac{\varepsilon}{2}\quad\textup{y}\quad\abs{\pi\int_0^x D_m(t)\:dt-\int_{0}^{x}\frac{\sin mt}{t}\:dt}<\frac{\varepsilon}{2}
        \end{equation*}
        luego, si $m\geq N$ se tiene que
        \begin{equation*}
            \begin{split}
                \abs{\pi\int_0^x D_m(t)\:dt-\int_{0}^{x}\frac{\sin mt}{t}\:dt}&=\abs{\frac{x}{2}+s_n(x)-\int_{0}^{x}\frac{\sin mt}{t}\:dt}\\
                &=\abs{\frac{x}{2}+\sum_{k=1}^m \frac{\sin kx}{k}-\int_{0}^{x}\frac{\sin mt}{t}\:dt}\\
                &=\abs{\frac{-\pi}{2}+\frac{x}{2}+\sum_{k=1}^m \frac{\sin kx}{k}+\frac{\pi}{2}-\int_{0}^{x}\frac{\sin mt}{t}\:dt}\\
                &=\abs{-\left(\frac{\pi-x}{2}-\sum_{k=1}^m \frac{\sin kx}{k}\right)+\left(\frac{\pi}{2}-\int_{0}^{x}\frac{\sin mt}{t}\:dt\right)}\\
                &\geq\abs{\int_{0}^{x}\frac{\sin mt}{t}\:dt-\frac{\pi}{2}}-\abs{\frac{\pi-x}{2}-\sum_{k=1}^m \frac{\sin kx}{k}}\\
            \end{split}
        \end{equation*}
        Por tanto,
        \begin{equation*}
            \begin{split}
                \abs{\int_{0}^{x}\frac{\sin mt}{t}\:dt-\frac{\pi}{2}}&\leq\abs{\pi\int_0^x D_m(t)\:dt-\int_{0}^{x}\frac{\sin mt}{t}\:dt}+\abs{\frac{\pi-x}{2}-\sum_{k=1}^m \frac{\sin kx}{k}}\\
                &<\frac{\varepsilon}{2}+\frac{\varepsilon}{2}\\
                &=\varepsilon\\
            \end{split}
        \end{equation*}
        por ende,
        \begin{equation*}
            \lim_{m\rightarrow\infty}\int_{0}^{x}\frac{\sin mt}{t}\:dt=\frac{\pi}{2}
        \end{equation*}
        Pero, por el T.C.V. se tiene que para todo $m\in\mathbb{N}$:
        \begin{equation*}
            \begin{split}
                \int_{0}^{x}\frac{\sin mt}{t}\:dt&\overset{u=mt}{=} \int_{0}^{mx}\frac{\sin u}{\frac{u}{m}}\: \frac{du}{m}\\
                \Rightarrow \int_{0}^{x}\frac{\sin mt}{t}\:dt&\overset{u=mt}{=} \int_{0}^{mx}\frac{\sin u}{u}\: du\\
            \end{split}
        \end{equation*}
        por ende,
        \begin{equation*}
            \lim_{m\rightarrow\infty}\int_{0}^{mx}\frac{\sin u}{u}\: du=\frac{\pi}{2}
        \end{equation*}
        Para todo $x\in]0,2\pi[$. Veamos ahora que
        \begin{equation*}
            \int_{0}^{\rightarrow\infty}\frac{\sin t}{t}\:dt=\frac{\pi}{2}
        \end{equation*}
        Ya se sabe que la integral impropia converge, por tanto, considerando la sucesión $\left\{\frac{\pi m}{2}\right\}_{ m=1}^\infty$ se tiene que:
        \begin{equation*}
            \begin{split}
                \int_{0}^{\rightarrow\infty}\frac{\sin t}{t}\:dt&=\lim_{m\rightarrow\infty}\int_{0}^{\frac{\pi m}{2}} \frac{\sin u}{u}\: du=\frac{\pi}{2}\\
                &=\frac{\pi}{2}\\
            \end{split}
        \end{equation*}
    \end{proof}

    \begin{excer}
        Sea $f\in\mathcal{L}_1^{2\pi}(\mathbb{C})$ y sean $\left\{c_k \right\}_{ k\in\mathbb{Z}}$ los coeficientes de Fourier de $f$. \textbf{Demuestre} que
        \begin{equation*}
            \int_0^x f=c+c_0x+\sum_{ k\in\mathbb{Z}\backslash\left\{0\right\}}\frac{c_ke^{ ikx}}{ik},\quad\forall x\in\mathbb{R}.
        \end{equation*}
        donde $c$ es una constante, la convergencia siendo uniforme en $\mathbb{R}$.

        \textit{Sugerencia.} Considere la función $F(x)=\int_0^x (f-c_0)$.

        \textbf{Deduzca} que los coeficientes de Fourier $b_n$ de cualquier función $f\in\mathcal{L}_1^{2\pi}(\mathbb{C})$ satisfacen la condición de que la serie
        \begin{equation*}
            \sum_{ n=1}^\infty\frac{b_n}{n}
        \end{equation*}
        es convergente. \textbf{Concluya} que la aplicación $f\mapsto\left\{c_n \right\}_{ n\in\mathbb{Z}}$ no es una aplicación suprayectiva de $\mathcal{L}_1^{2\pi}(\mathbb{C})$ en $c_0(\mathbb{Z})$.
    \end{excer}

    \begin{proof}
        Como $f\in\mathcal{L}_1^{2\pi}$, entonces la función
        \begin{equation*}
            F(x)=\int_{0}^{x}\left(f(t)-c_0 \right)\:dt
        \end{equation*}
        es una función absolutamente continua y de periodicidad $2\pi$, pues
        \begin{equation*}
            \int_{-\pi}^{\pi}(f(t)-c_0)\:dt=0
        \end{equation*}
        En efecto, veamos que
        \begin{equation*}
            \begin{split}
                \int_{-\pi}^{\pi}(f(t)-c_0)\:dt&=\int_{-\pi}^{\pi}f(t)\:dt+2\pi c_0\\
                &=\int_{-\pi}^{\pi}f(t)\:dt+2\pi\cdot\frac{1}{2\pi}\int_{-\pi}^{\pi}f(t)\:dt\\
                &=0\\
            \end{split}
        \end{equation*}
        En particular, es de variación acotada y continua (por ser absolutamente continua), luego por el Teorema de Jordan la serie de Fourier de $F$ en $x$ converge puntualmente a $F$ en $x$ para todo $x\in\mathbb{R}$, esto es
        \begin{equation*}
            \begin{split}
                F(x)&=\sum_{ k\in\mathbb{Z}}c_k'e^{ ikx},\quad\forall x\in\mathbb{R}\\
                \Rightarrow \int_{0}^{x}(f(t)-c_0)&=\sum_{ k\in\mathbb{Z}}c_k'e^{ ikx},\quad\forall x\in\mathbb{R}\\
                \Rightarrow \int_{0}^{\pi}f(t)\:dt&=c_0x+\sum_{ k\in\mathbb{Z}}c_k'e^{ ikx},\quad\forall x\in\mathbb{R}\\
            \end{split}
        \end{equation*}
        siendo $\left\{c_k'\right\}_{ k\in\mathbb{Z}}$ los coeficientes de Fourier de $F$. Calculemos estos coeficientes, para ello, calculemos los de $f-c_0$ y recordemos que la aplicación que a cada función en $L_1^{2\pi}(\mathbb{R},\mathbb{C})$ envía a sus coeficientes de Fourier es una aplicación lineal inyectiva, en particular por ser lineal basta con calcular los coeficientes de $f$ y $c_0$ por separado y, los coeficientes de $f-c_0$ serán la diferencia de cada uno de estos coeficientes.

        Sean entonces $\left\{c_k \right\}_{ k\in\mathbb{Z}}$ y $\left\{d_k \right\}_{ k\in\mathbb{Z}}$ los coeficientes de Fourier de $f$ y $c_0$ respectivamente, se tiene entonces que
        \begin{equation*}
            c_k'=\frac{c_k-d_k}{ik},\quad\forall k\in\mathbb{Z}\backslash\left\{0\right\}
        \end{equation*}
        tomando $c_0=c'\in\mathbb{C}$ una constante. Siendo:
        \begin{equation*}
            \begin{split}
                d_k&=\int_{-\pi}^\pi c_0e^{ -ikx}\:dx\\
                &=c_0\cdot\frac{e^{ -ikx}}{-ik}\Big|_{-\pi}^\pi\\
                &=c_0\cdot\frac{e^{ -ik\pi}-e^{ik\pi}}{-ik}\\
                &=c_0\cdot\frac{\cos(-k\pi)+i\sin(-k\pi)-\cos(k\pi)-i\sin(k\pi)}{-ik}\\
                &=c_0\cdot\frac{\cos(k\pi)-i\sin(k\pi)-\cos(k\pi)-i\sin(k\pi)}{-ik}\\
                &=c_0\cdot\frac{2i\sin(k\pi)}{ik}\\
                &=0,\quad\forall k\in\mathbb{Z} \\
            \end{split}
        \end{equation*}
        Por tanto:
        \begin{equation*}
            c_k'=\frac{c_k}{ik},\quad\forall k\in\mathbb{Z}\backslash\left\{0\right\}
        \end{equation*}
        De esta forma, tenemos que:
        \begin{equation*}
            \begin{split}
                F(x)&=\sum_{ k\in\mathbb{Z}}c_k'e^{ ikx}\\
                &=c+\sum_{k\in\mathbb{Z}\backslash\left\{0\right\}}c_k'e^{ ikx}\\
                &=c+\sum_{k\in\mathbb{Z}\backslash\left\{0\right\}}\frac{c_ke^{ ikx}}{ik},\quad\forall x\in\mathbb{R} \\
                \Rightarrow \int_{0}^{x}(f(t)-c_0)\:dt&=c+\sum_{k\in\mathbb{Z}\backslash\left\{0\right\}}\frac{c_ke^{ ikx}}{ik},\quad\forall x\in\mathbb{R} \\
                \Rightarrow \int_{0}^{x}f(t)\:dt&=c+c_0x+\sum_{k\in\mathbb{Z}\backslash\left\{0\right\}}\frac{c_ke^{ ikx}}{ik},\quad\forall x\in\mathbb{R} \\
            \end{split}
        \end{equation*}
        En particular para $x=0$ se tiene que
        \begin{equation*}
            \begin{split}
                0&=c+\sum_{ k\in\mathbb{Z}\backslash\left\{0\right\}}\frac{c_k}{ik}\\
                \Rightarrow -c&=\sum_{ k\in\mathbb{Z}\backslash\left\{0\right\}}\frac{c_k}{ik}\\
                &=\sum_{k\in1}^\infty\frac{c_k}{ik}+\sum_{k\in1}^\infty\frac{c_{-k}}{-ik}\\
                &=\sum_{k\in1}^\infty\frac{c_k}{ik}+\sum_{k\in1}^\infty\frac{-c_{-k}}{ik}\\
                &=\frac{1}{i}\cdot\sum_{k\in1}^\infty\frac{c_k-c_{ -k}}{k}\\
                &=\frac{1}{i}\cdot\sum_{k\in1}^\infty\frac{b_k}{k}\\
            \end{split}
        \end{equation*}
        pues, la serie $\sum_{ k\in\mathbb{Z}\backslash\left\{0\right\}}\frac{c_k}{ik}$ es convergente. Por tanto, la serie
        \begin{equation*}
            \sum_{ n=1}^\infty\frac{b_n}{n}
        \end{equation*}
        es convergente y converge a $-ic$.

        Para la última parte, considere la sucesión $\left\{c_k\right\}_{k}\in\mathbb{Z}$ tal que:
        \begin{equation*}
            c_k=\frac{a_k-ib_k}{2},\quad\forall k\in\mathbb{Z}\backslash\left\{0\right\}
        \end{equation*}
        tomando
        \begin{equation*}
            a_k=0,\quad\forall k\in\mathbb{Z}
        \end{equation*}
        y,
        \begin{equation*}
            b_k=\frac{1}{\ln k},\quad\forall k\in\mathbb{N},k\geq2
        \end{equation*}
        y haciendo:
        \begin{equation*}
            b_{-k}=-b_k,\quad\forall k\in\mathbb{N},k\geq2
        \end{equation*}
        Afirmamos que no puede existir ninguna función $f\in\mathcal{L}_1^{2\pi}(\mathbb{C})$ tal que tenga como coeficientes de Fourier los escritos anteriormente, ya que en tal caso, se tendría que la serie:
        \begin{equation*}
            \sum_{ k=2}^\infty\frac{1}{k\ln k}
        \end{equation*}
        sería convergente. Pero, como la integral (haciendo $u=\ln x\Rightarrow du=\frac{dx}{x}$):
        \begin{equation*}
            \begin{split}
                \int_1^{ \infty}\frac{dx}{x\ln x}&=\int_{0}^{\infty}\frac{du}{u}\\
                &=\infty\\
            \end{split}
        \end{equation*}
        tal serie no puede converger, luego estos coeficientes de Fourier no pueden pertenecer a ninguna función en $\mathcal{L}_1^{2\pi}(\mathbb{C})$.
    \end{proof}

    \begin{excer}
        Haga lo siguiente:
        \begin{enumerate}
            \item Sea $\alpha$ un número real no entero. \textbf{Pruebe} que
            \begin{equation*}
                \pi\cos\alpha x=2\alpha\sin\pi\alpha\left(\frac{1}{2\alpha^2}+\sum_{ n=1}^\infty(-1)^n\frac{\cos nx}{\alpha^2-n^2} \right),\quad\forall x\in[-\pi,\pi].
            \end{equation*}
            De ahí obtenga las fórmulas clásicas
            \begin{equation*}
                \frac{\pi\alpha}{\sin\pi\alpha}=1+2\alpha^2\sum_{ n=1}^\infty\frac{(-1)^n}{\alpha^2-n^2}\quad\textup{y}\quad\pi\alpha\cot\pi\alpha=1+2\alpha^2\sum_{ n=1}^\infty\frac{1}{\alpha^2-n^2}.
            \end{equation*}
            \item Sea $x\in]0,1[$. \textbf{Pruebe} que la serie
            \begin{equation*}
                \sum_{ n=1}^\infty\frac{2\alpha}{n^2-\alpha^2}
            \end{equation*}
            se puede integrar término por término en el intervalo $[0,x]$. De la última fórmula del inciso (i) \textbf{deduzca} la fórmula
            \begin{equation*}
                \sin\pi x=\pi x\prod_{ n=1}^\infty\left(1-\frac{x^2}{n^2} \right),\quad\forall x\in]-1,1[.
            \end{equation*}
        \end{enumerate}
    \end{excer}

    \begin{proof}
        De (i): Considere la función $\cf{f}{\mathbb{R}}{\mathbb{R}}$ dada como sigue:
        \begin{equation*}
            x\mapsto \cos\alpha x,\quad\forall x\in[-\pi,\pi]
        \end{equation*}
        y extiéndase por periodicidad a todo $\mathbb{R}$. Es claro que esta función es continua, pues
        \begin{equation*}
            f(-\pi)=\cos(-\alpha\pi)=\cos\alpha\pi=f(\pi)
        \end{equation*}
        (al hacerse la extensión se tiene que es continua en $\pi$, como es continua en $]-\pi,\pi[$, se sigue que lo es en todo $\mathbb{R}$). Esta función es clase $C^1$ en $]-\pi,\pi[$, luego es de variación acotada en $[-\pi,\pi]$. Por el Teorema de Jordan la serie de Fourier de $f$ en $x$ converge puntualmente a $f(x)$ para todo $x\in\mathbb{R}$ (en particular, en $[-\pi,\pi]$). Notemos que $f$ es par, por lo cual se tiene que
        \begin{equation*}
            b_k=0,\quad\forall k\in\mathbb{N}
        \end{equation*}
        Y,
        \begin{equation*}
            a_k=\frac{2}{\pi}\int_{0}^{\pi}f(x)\cos kx\:dx,\quad\forall k\in\mathbb{N}^+
        \end{equation*}
        calculemos estos coeficientes.
        \begin{equation*}
            \begin{split}
                a_0&=\frac{2}{\pi}\int_{0}^{\pi}f(x)\:dx\\
                &=\frac{2}{\pi}\int_{0}^{\pi}\cos\alpha x\:dx,\textup{  sea }u=\alpha x \\
                &=\frac{2}{\pi}\int_{0}^{\alpha\pi}\cos u\: \frac{du}{\alpha}\\
                &=\frac{2}{\pi\alpha}\int_{0}^{\pi\alpha}\cos u\:du\\
                &=\frac{2}{\pi\alpha}\sin u\Big|_{0}^{\alpha\pi}\\
                &=\frac{2}{\pi\alpha}\sin u\Big|_{0}^{\pi\alpha}\\
                &=\frac{2}{\pi\alpha}\sin\pi\alpha\\
            \end{split}
        \end{equation*}
        Recordemos antes que
        \begin{equation*}
            \cos A\cos B=\frac{\cos (A-B)+\cos (A+B)}{2}
        \end{equation*}
        y,
        \begin{equation*}
            \sin A+\sin B=2\sin\left(\frac{A+B}{2}\right)\cos\left(\frac{A-B}{2}\right)
        \end{equation*}
        Ahora, para $k\in\mathbb{N}$:
        \begin{equation*}
            \begin{split}
                a_k&=\frac{2}{\pi}\int_{0}^{\pi}f(x)\cos kx\:dx\\
                &=\frac{2}{\pi}\int_{0}^{\pi}\cos\alpha x\cos kx\:dx\\
                &=\frac{2}{\pi}\int_{0}^{\pi}\frac{\cos(\alpha-k)x+\cos(\alpha+k)x}{2}\:dx\\
                &=\frac{1}{\pi}\left[\int_{0}^{\pi}\cos(\alpha-k)x\:dx+\int_{0}^{\pi}\cos(\alpha+k)x\:dx\right]\\
                &=\frac{1}{\pi}\left[\frac{1}{\alpha-k}\int_{0}^{(\alpha-k)\pi}\cos u\:du+\frac{1}{\alpha+k}\int_{0}^{(\alpha+k)\pi}\cos v\:dv\right]\\
                &=\frac{1}{\pi}\left[\frac{1}{\alpha-k}\sin u\Big|_{0}^{(\alpha-k)\pi}+\frac{1}{\alpha+k}\sin v\Big|_{0}^{(\alpha+k)\pi} \right]\\
                &=\frac{1}{\pi}\left[\frac{1}{\alpha-k}\sin u(\alpha-k)\pi+\frac{1}{\alpha+k}\sin (\alpha+k)\pi \right]\\
                &=\frac{1}{\pi}\left[\frac{\alpha\sin(\alpha-k)\pi+k\sin(\alpha-k)\pi}{\alpha^2-k^2}+\frac{\alpha\sin(\alpha+k)\pi-k\sin(\alpha+k)\pi}{\alpha^2-k^2}\right]\\
                &=\frac{1}{\pi}\left[\frac{\alpha\sin(\alpha-k)\pi+\alpha\sin(\alpha+k)\pi+k\sin(\alpha-k)\pi+k\sin(-\alpha-k)\pi}{\alpha^2-k^2}\right]\\
                &=\frac{1}{\pi}\left[\frac{2\alpha\sin(\frac{\alpha-k+\alpha+k}{2})\pi\cos(\frac{\alpha-k-\alpha-k}{2})\pi+2k\sin(\frac{\alpha-k-\alpha-k}{2})\pi\cos(\frac{\alpha-k+\alpha+k}{2})\pi}{\alpha^2-k^2}\right]\\
                &=\frac{1}{\pi}\left[\frac{2\alpha\sin\alpha\pi\cos(-k\pi)+2k\sin(-k\pi)\cos\alpha\pi}{\alpha^2-k^2}\right]\\
                &=\frac{1}{\pi}\left[\frac{2\alpha\sin\alpha\pi\cos k\pi-2k\sin k\pi\cos \alpha\pi}{\alpha^2-k^2}\right],\textup{ pero }\sin k\pi=0\textup{ y }\cos k\pi=(-1)^k \\
                &=\frac{1}{\pi}\left[\frac{2(-1)^k\alpha\sin\alpha\pi}{\alpha^2-k^2}\right]\\
                &=\frac{2\alpha\sin\alpha\pi}{\pi}\cdot\frac{(-1)^k}{\alpha^2-k^2}\\
            \end{split}
        \end{equation*}
        Por tanto, se tiene que
        \begin{equation*}
            \begin{split}
                \cos\alpha x&=\frac{a_0}{2}+\sum_{n=1}^{\infty}a_n\cos nx\\
                &=\frac{\sin \pi\alpha}{\pi\alpha}+\sum_{n=1}^{\infty} \frac{2\alpha\sin\alpha\pi}{\pi}\cdot\frac{(-1)^n}{\alpha^2-n^2}\cos nx\\
                &=\frac{2\alpha\sin\pi\alpha}{\pi}\left(\frac{1}{2\alpha^2}+\sum_{n=1}^{\infty}(-1)^n\frac{\cos nx}{\alpha^2-n^2}\right),\quad\forall x\in[-\pi,\pi]\\
                \Rightarrow \pi\cos \alpha x&=2\alpha\sin\pi\alpha\left(\frac{1}{2\alpha^2}+\sum_{n=1}^{\infty}(-1)^n\frac{\cos nx}{\alpha^2-n^2}\right),\quad\forall x\in[-\pi,\pi] \\
            \end{split}
        \end{equation*}
        
        En particular, cuando $x=0$ y $x=\pi$ se tiene que
        \begin{equation*}
            \begin{split}
                \pi\cos \alpha 0&=2\alpha\sin\pi\alpha\left(\frac{1}{2\alpha^2}+\sum_{n=1}^{\infty}(-1)^n\frac{\cos 0}{\alpha^2-n^2}\right)\\
                \Rightarrow \pi\alpha&=2\alpha^2\sin\pi\alpha\left(\frac{1}{2\alpha^2}+\sum_{n=1}^{\infty}\frac{(-1)^n}{\alpha^2-n^2}\right)\\
                \Rightarrow \frac{\pi\alpha}{\sin\pi\alpha}&=1+2\alpha^2\sum_{n=1}^{\infty}\frac{(-1)^n}{\alpha^2-n^2}\\
            \end{split}
        \end{equation*}
        y,
        \begin{equation*}
            \begin{split}
                \pi\cos \alpha\pi&=2\alpha\sin\pi\alpha\left(\frac{1}{2\alpha^2}+\sum_{n=1}^{\infty}(-1)^n\frac{\cos n\pi}{\alpha^2-n^2}\right)\\
                \Rightarrow \pi\cot\alpha\pi&=2\alpha\left(\frac{1}{2\alpha^2}+\sum_{n=1}^{\infty}\frac{(-1)^n\cos n\pi}{\alpha^2-n^2}\right)\\
                \Rightarrow \alpha\pi\cot\alpha\pi&=2\alpha^2\left(\frac{1}{2\alpha^2}+\sum_{n=1}^{\infty}\frac{1}{\alpha^2-n^2}\right)\\
                &=1+2\alpha^2\sum_{n=1}^{\infty}\frac{1}{\alpha^2-n^2}\\
                \therefore\alpha\pi\cot\alpha\pi&=1+2\alpha^2\sum_{n=1}^{\infty}\frac{1}{\alpha^2-n^2}\\
            \end{split}
        \end{equation*}

        De (ii): Veamos que se puede integrar respecto a $\alpha$. En efecto, para cada $\nu\in\mathbb{N}$ defina:
        \begin{equation*}
            s_\nu(\alpha)=\sum_{n=1}^\nu\frac{2\alpha}{n^2-\alpha^2}
        \end{equation*}
        Considere así la sucesión de funciones $\left\{s_\nu\right\}_{\nu=1}^\infty$. Como cada función es integrable en $]0,x[$ (pues el $0<x<1$) y,
        \begin{equation*}
            \begin{split}
                \abs{s_\nu(\alpha)}&=\abs{\sum_{n=1}^\nu\frac{2\alpha}{n^2-\alpha^2}}\\
                &=\sum_{n=1}^\nu\frac{2\alpha}{n^2-\alpha^2}\\
                &=\frac{2\alpha}{1-\alpha^2}+\sum_{n=2}^\nu\frac{2\alpha}{n^2-\alpha^2}\\
                &\leq\frac{2\alpha}{1-\alpha^2}+\sum_{n=2}^\nu\frac{2\alpha}{n^2-1^2}\\
                &\leq\frac{2\alpha}{1-\alpha^2}+2\alpha\sum_{n=2}^\infty\frac{1}{n^2-1},\quad\forall\nu\in\mathbb{N} \\
            \end{split}
        \end{equation*}
        donde la función $\cf{g}{[0,1[}{\mathbb{R}}$ tal que $\alpha\mapsto \frac{2\alpha}{1-\alpha^2}+2\alpha\sum_{n=2}^\infty\frac{1}{n^2-1}$ es continua en el compacto $[0,x]$ pues
        \begin{equation*}
            \sum_{n=2}^\infty\frac{1}{n^2-1}=\frac{3}{4}<\infty
        \end{equation*}
        luego integrable en $[0,x]$. Por tanto, del Teorema de Lebesgue se sigue que:
        \begin{equation*}
            \begin{split}
                \lim_{\nu\rightarrow\infty}\int_{0}^x s_\nu(\alpha)\:d\alpha&=\int_0^x\lim_{\nu\rightarrow\infty}s_\nu(\alpha)\:d\alpha\\
                \Rightarrow \lim_{\nu\rightarrow\infty}\int_{0}^x \sum_{n=1}^\nu\frac{2\alpha}{n^2-\alpha^2}\:d\alpha&=\int_0^x\sum_{ n=1}^\infty\frac{2\alpha}{n^2-\alpha^2}\:d\alpha\\
            \end{split}
        \end{equation*}
        siendo
        \begin{equation*}
            \begin{split}
                \int_{0}^x \sum_{n=1}^\nu\frac{2\alpha}{n^2-\alpha^2}\:d\alpha&=\sum_{n=1}^\nu\int_{0}^x\frac{2\alpha}{n^2-\alpha^2}\:d\alpha,\textup{ tomemos }u=n^2-\alpha^2\Rightarrow du=-2\alpha\:d\alpha \\
                &=\sum_{n=1}^\nu\int_{n^2}^{n^2-x^2}\frac{2\alpha}{u}\: \frac{du}{-2\alpha}\\
                &=\sum_{n=1}^\nu\int_{n^2-x^2}^{n^2}\frac{du}{u}\\
                &=\sum_{n=1}^\nu\ln\abs{u}\Big|_{n^2-x^2}^{n^2}\\
                &=\sum_{n=1}^\nu\left[\ln\abs{n^2}-\ln\abs{n^2-x^2}\right]\\
                &=\sum_{n=1}^\nu\left[\ln\abs{\frac{n^2}{n^2-x^2}}\right]\\
                &=\sum_{n=1}^\nu\ln\frac{n^2}{n^2-x^2}\\
                &=\sum_{n=1}^\nu\ln\left(\frac{n^2}{n^2-x^2}\right)\\
                &=\ln\left(\prod_{ n=1}^\nu\frac{n^2}{n^2-x^2}\right)\\
            \end{split}
        \end{equation*}
        Por tanto, como la función $x\mapsto\ln x$ es continua en $]0,\infty[$, se sigue que:
        \begin{equation*}
            \ln\left(\prod_{ n=1}^\infty\frac{n^2}{n^2-x^2}\right)=-\int_0^x\sum_{ n=1}^\infty\frac{2\alpha}{n^2-\alpha^2}\:d\alpha
        \end{equation*}
        Ahora, si integramos en $[0,x]$ ambos lados de la última igualdad de (i), se tiene que:
        \begin{equation*}
            \begin{split}
                \alpha\pi\cot\alpha\pi&=1+2\alpha^2\sum_{n=1}^{\infty}\frac{1}{\alpha^2-n^2}\\
                \Rightarrow \pi\cot\pi\alpha&=\frac{1}{\alpha}+\sum_{n=1}^{\infty}\frac{2\alpha}{\alpha^2-n^2}\\
                \Rightarrow \pi\cot\pi\alpha-\frac{1}{\alpha}&=\sum_{n=1}^{\infty}\frac{2\alpha}{\alpha^2-n^2}\\
            \end{split}
        \end{equation*}
        Ahora, veamos que
        \begin{equation*}
            \begin{split}
                \frac{d}{d\alpha}\left(\ln\left(\frac{\sin \pi\alpha}{\alpha}\right)\right)&=\frac{1}{\frac{\sin \pi\alpha}{\alpha}}\cdot\frac{d}{d\alpha}\left(\frac{\sin \pi\alpha}{\alpha}\right)\\
                &=\frac{\alpha}{\sin \pi\alpha}\cdot\frac{d}{d\alpha}\left(\alpha^{-1}\sin \pi\alpha\right)\\
                &=\frac{\alpha}{\sin \pi\alpha}\cdot\left(-\alpha^{-2}\sin\pi\alpha+\pi\alpha^{-1}\cos\pi\alpha\right)\\
                &=\frac{1}{\alpha\sin \pi\alpha}\cdot\left(-\sin\pi\alpha+\pi\alpha\cos\pi\alpha\right)\\
                &=-\frac{1}{\alpha}+\pi\cot\pi\alpha\\
                &=\pi\cot\pi\alpha-\frac{1}{\alpha}\\
            \end{split}
        \end{equation*}
        Luego, por el primer Teorema Fundamental del Cálculo para intervalos abiertos, se sigue que:
        \begin{equation*}
            \begin{split}
                \int_0^x\left(\pi\cot\pi\alpha-\frac{1}{\alpha}\right)\:d\alpha&=\ln\left(\frac{\sin \pi x}{x}\right)-\lim_{\alpha\rightarrow0^+}\ln\left(\frac{\sin \pi\alpha}{\alpha}\right)\\
                &=\ln\left(\frac{\sin \pi x}{x}\right)-\lim_{\alpha\rightarrow0^+}\ln\left(\pi\frac{\sin \pi\alpha}{\pi\alpha}\right)\\
                &=\ln\left(\frac{\sin \pi x}{x}\right)-\ln\left(\pi\lim_{\alpha\rightarrow0^+}\frac{\sin \pi\alpha}{\pi\alpha}\right)\\
                &=\ln\left(\frac{\sin \pi x}{x}\right)-\ln\left(\pi\cdot1\right)\\
                &=\ln\left(\frac{\sin \pi x}{x}\right)-\ln(\pi)\\
                &=\ln\left(\frac{\sin \pi x}{\pi x}\right)
            \end{split}
        \end{equation*}
        Entonces,
        \begin{equation*}
            \begin{split}
                \int_0^x\left(\pi\cot\pi\alpha-\frac{1}{\alpha}\right)\:d\alpha&=\int_0^x\sum_{ n=1}^\infty\frac{2\alpha}{n^2-\alpha^2}\:d\alpha\\
                &=-\int_0^x\sum_{ n=1}^\infty\frac{2\alpha}{\alpha^2-n^2}\:d\alpha\\
                \Rightarrow \ln\left(\frac{\sin \pi x}{\pi x}\right)&=-\ln\left(\prod_{ n=1}^\infty\frac{n^2}{n^2-x^2}\right)\\
                &=\ln\left(\prod_{ n=1}^\infty\frac{n^2-x^2}{n^2}\right)\\
                &=\ln\left(\prod_{ n=1}^\infty\left(1-\frac{x^2}{n^2}\right) \right)\\
                \Rightarrow \frac{\sin \pi x}{\pi x}&=\prod_{ n=1}^\infty\left(1-\frac{x^2}{n^2}\right)\\
                \Rightarrow \sin\pi x&=\pi x\prod_{ n=1}^\infty\left(1-\frac{x^2}{n^2}\right)\\
            \end{split}
        \end{equation*}
        para todo $0<x<1$. Sea ahora $-1<x<0$. Como la función seno es impar, entonces:
        \begin{equation*}
            \begin{split}
                \sin(\pi x)&=-\sin(-\pi x)\\
                &=\pi (-x)\prod_{ n=1}^\infty\left(1-\frac{(-x)^2}{n^2}\right)\\
                &=-\left[\pi (-x)\prod_{ n=1}^\infty\left(1-\frac{(-x)^2}{n^2}\right)\right] \\
                &=\pi x\prod_{ n=1}^\infty\left(1-\frac{x^2}{n^2}\right)\\
            \end{split}
        \end{equation*}
        y, si $x=0$ se tiene de forma inmediata que:
        \begin{equation*}
            \sin(\pi x)=0=0\cdot\prod_{ n=1}^\infty\left(1\right)=\pi x\prod_{ n=1}^\infty\left(1-\frac{x^2}{n^2}\right)
        \end{equation*}
        Por tanto, para todo $x\in]-1,1[$ se cumple:
        \begin{equation*}
            \sin\pi x=\pi x\prod_{ n=1}^\infty\left(1-\frac{x^2}{n^2}\right)
        \end{equation*}
    \end{proof}

    \begin{excer}
        Se supone que la serie de Fourier de una función $f\in\mathcal{L}_1^{2\pi}(\mathbb{K})$ converge en el sentido de Cesáro uniformemente en $\mathbb{R}$. \textbf{Pruebe} que $f$ es equivalente a una función continua de $\mathbb{R}$ en $\mathbb{K}$.
    \end{excer}

    \begin{proof}
        Ya se sabe por Fejér-Lebesgue que la serie de Fourier $f$ converge puntualmente c.t.p. en $\mathbb{R}$ en el sentido de Cesáro a $f$.

        Ahora, como la serie de Fourier de $f$ converge en el sentido de Cesáro uniformemente a una función $\cf{g}{\mathbb{R}}{\mathbb{K}}$, al ser $g$ el límite uniforme de funciones continuas en $\mathbb{R}$ (por ser combinaciones lineales de $\sin$ y $\cos$), entonces $g$ es continua en $\mathbb{R}$. En particular, la serie de Fourier de $f$ converge puntualmente en el sentido de Cesáro a $g$ en $\mathbb{R}$.

        Por tanto, de lo anterior se sigue que $f=g$ c.t.p. en $\mathbb{R}$, es decir que $f$ es equivalente a una función continua de $\mathbb{R}$ en $\mathbb{K}$.

        (Nota: \href{https://mathcs.org/analysis/reals/funseq/proofs/uconvcont.html#:~:text=If%20a%20sequence%20of%20functions,is%20also%20continuous%20on%20D.&text=as%20long%20as%20%7Cx0,is%20continuous%20at%20x0}{convergencia uniforme de funciones continuas.})
    \end{proof}

    \begin{excer}
        Sea $f\in\mathcal{C}^{2\pi}(\mathbb{R})$ la función
        \begin{equation*}
            f(x)=\pi-\abs{2x},\quad-\pi\leq x\leq\pi
        \end{equation*}
        Demuestre que la serie de Fourier de $f$ converge a $f$ uniformemente en $\mathbb{R}$ aplicando primero el Teorema 3.5 y después el Teorema de Jordan. \textbf{Calcule}
        \begin{equation*}
            \sum_{ k=1}^\infty\frac{1}{(2k-1)^2}\quad\textup{y}\quad\sum_{ k=1}^\infty\frac{1}{(2k-1)^4}.
        \end{equation*}
    \end{excer}

    \begin{sol}
        Se tiene que probar el resultado usando el Teorema 3.4.1 (de mis notas). Para ello, debemos encontrar una función $\cf{g}{\mathbb{R}}{\mathbb{R}}$ tal que $g\in\mathcal{L}_2^{2\pi}(\mathbb{R})$ que satisfaga:
        \begin{equation*}
            \int_{-\pi}^{\pi}g(t)\:dt=0
        \end{equation*}
        y que
        \begin{equation*}
            f(x)=c+\int_{0}^{x}g(t)\:dt
        \end{equation*}
        para todo $x\in\mathbb{R}$ siendo $c\in\mathbb{R}$. De esta forma se sigue de manera inmediata que la serie de Fourier de $f$ converge a $f$ uniformemente en $\mathbb{R}$.

        Tomemos $c=\pi$ y,
        \begin{equation*}
            g(t)=\left\{
                \begin{array}{lcr}
                     2 & \textup{ si } & -\pi\leq t<0\\
                     -2 & \textup{ si } & 0\leq t<\pi\\
                \end{array}
            \right.,\quad\forall t\in[-\pi,\pi[
        \end{equation*}
        y extiéndase por periodicidad a todo $\mathbb{R}$. Afirmamos que
        \begin{equation*}
            \int_{0}^{x}g(t)\:dt = -\abs{2x},\quad\forall x\in[-\pi,\pi]
        \end{equation*}
        En efecto, sea $x\in[-\pi,\pi]$, se tienen dos casos:
        \begin{itemize}
            \item $x<0$, se tiene que 
            \begin{equation*}
                \begin{split}
                    \int_{0}^{x}g(t)\:dt&=-\int_{x}^{0}2\:dt\\
                    &=-2 t\Big|_{x}^0\\
                    &=2x\\
                    &=-(-2x)\\
                    &=-\abs{2x}\\
                \end{split}
            \end{equation*}
            \item $x\geq 0$, se tiene que
            \begin{equation*}
                \begin{split}
                    \int_{0}^{x}g(t)\:dt&=-\int_0^x -2\:dt\\
                    &=-2\int_0^x\:dt\\
                    &=-2t\Big|_{0}^x\\
                    &=-2x\\
                    &=-\abs{2x}\\
                \end{split}
            \end{equation*}
        \end{itemize}
        luego, como $g\in\mathcal{L}_2^{2\pi}(\mathbb{R})$ se sigue que la serie de Fourier de $f$ converge a $f$ uniformement en $\mathbb{R}$, pues
        \begin{equation*}
            f(x)=\pi+\int_{0}^{x}g(t)\:dt
        \end{equation*}
        para todo $x\in\mathbb{R}$.

        Ahora, aplicando el teorema de Jordan basta con ver que $f$ es $2\pi$ periódica, de variación acotada en $[-\pi,\pi]$ y continua en $\mathbb{R}$. En efecto, ya se tiene que $f$ es $2\pi$ periódica, veamos que
        \begin{itemize}
            \item \textbf{$f$ es de variación acotada}: Sea $\Delta=\left\{-\pi=x_0<x_1<...<x_n=\pi \right\}$ una partición del intervalo $[-\pi,\pi]$, entonces
            \begin{equation*}
                \begin{split}
                    S_\Delta(f)&=\sum_{ k=1}^n\abs{f(x_k)-f(x_{k-1})}\\
                    &=\sum_{ k=1}^n\abs{\pi-\abs{2x_k}-\pi+\abs{2x_{ k-1}}}\\
                    &=2\sum_{ k=1}^n\abs{\abs{x_k}-\abs{x_{ k-1}}}\\
                \end{split}
            \end{equation*}
            existe $m\in\mathbb{N}\cup\left\{0\right\}$ tal que
            \begin{equation*}
                x_{m}\leq0<x_{ m+1}
            \end{equation*}
            se divide pues la suma como:
            \begin{equation*}
                \begin{split}
                    S_\Delta(f)&=2\sum_{ k=1}^n\abs{\abs{x_k}-\abs{x_{ k-1}}}\\
                    &=2\sum_{ k=1}^m\abs{\abs{x_k}-\abs{x_{ k-1}}}+2\sum_{ k=m+1}^n\abs{\abs{x_k}-\abs{x_{ k-1}}}\\
                \end{split}
            \end{equation*}
            si $k\in\natint{1,m}$, entonces
            \begin{equation*}
                x_{ k-1}<x_k\leq0\Rightarrow \abs{x_k}<\abs{x_{ k-1}}
            \end{equation*}
            y, si $k\in\natint{m+1,n}$,
            \begin{equation*}
                0<x_{ k-1}<x_{k}\Rightarrow \abs{x_{ k-1}}<\abs{x_k}
            \end{equation*}
            por lo cual
            \begin{equation*}
                \begin{split}
                    S_\Delta(f)&=2\sum_{ k=1}^m\abs{\abs{x_k}-\abs{x_{ k-1}}}+2\sum_{ k=m+1}^n\abs{\abs{x_k}-\abs{x_{ k-1}}}\\
                    &=2\sum_{ k=1}^m(\abs{x_{k-1}}-\abs{x_k})+2\sum_{ k=m+1}^n(\abs{x_k}-\abs{x_{ k-1}})\\
                    &=2\sum_{ k=1}^m(\abs{x_{k-1}}-\abs{x_k})+2\sum_{ k=m+1}^n(\abs{x_k}-\abs{x_{ k-1}})\\
                    &=2(\abs{x_0}-\abs{x_m})+2(\abs{x_n}-\abs{x_m})\\
                    &=4(\abs{x_0}+\abs{x_n})\\
                    &=4\pi\\
                \end{split}
            \end{equation*}
            por tanto,
            \begin{equation*}
                V_f([-\pi,\pi])=4\pi
            \end{equation*}
            así, $f$ es de variación acotada en $[-\pi,\pi]$.
            \item \textbf{$f$ es continua en $\mathbb{R}$}. Ya se tiene que $f$ es continua en $]-\pi,\pi[$, para ver que es continua en $\mathbb{R}$ basta con ver que es continua en $\pi$, para ello, se debe verificar que
            \begin{equation*}
                \lim_{x\rightarrow \pi^-}f(x)=f(\pi)=f(-\pi)=\lim_{x\rightarrow-\pi^+}f(x)
            \end{equation*}
            Los dos límites ya se tienen pues la función $\pi-\abs{2x}$ es continua en $\mathbb{R}$. Por lo cual, solo basta con ver que
            \begin{equation*}
                f(\pi)=\pi-\abs{2\pi}=-\pi=\pi-\abs{2(-\pi)}=f(-\pi)
            \end{equation*}
            luego, $f$ es continua en $\mathbb{R}$.
        \end{itemize}
        Por tanto, usando el Teorema de Jordan se sigue que la serie de Fourier de $f$ converge a $f$ uniformemente en $\mathbb{R}$.

        Ahora, calculemos los coeficientes de la serie de Fourier de $f$, notemos antes que $f$ es par, por ende $b_k=0$ para todo $k\in\mathbb{N}$. Sea $k\in\mathbb{N}$, se tiene que
        \begin{equation*}
            \begin{split}
                a_0&=\frac{2}{\pi}\int_{0}^{\pi}f(t)\:dt\\
                &=\frac{2}{\pi}\int_{0}^{\pi}(\pi-\abs{2t})\:dt\\
                &=\frac{2}{\pi}\int_{0}^{\pi}(\pi-2t)\:dt\\
                &=\frac{2}{\pi}\left[\pi t-t^2\Big|_0^\pi \right] \\
                &=\frac{2}{\pi}\left[\pi^2-\pi^2\right] \\
                &=0\\
            \end{split}
        \end{equation*}
        y,
        \begin{equation*}
            \begin{split}
                a_k&=\frac{2}{\pi}\int_{0}^{\pi}f(t)\cos kt\:dt\\
                &=\frac{2}{\pi}\int_{0}^{\pi}(\pi-\abs{2t})\cos kt\:dt\\
                &=\frac{2}{\pi}\int_{0}^{\pi}(\pi-2t)\cos kt\:dt\\
                &=\frac{2}{\pi}\left[\int_{0}^{\pi}\pi\cos kt\:dt-\int_{0}^{\pi}2t\cos kt\:dt\right]\\
                &=\frac{2}{\pi}\left[\pi\int_{0}^{\pi}\cos kt\:dt-2\int_{0}^{\pi}t\cos kt\:dt\right],\textup{ haciendo el cambio de variable }u=kt \\
                &=\frac{2}{\pi}\left[\pi\int_{0}^{k\pi}\cos u\:\frac{du}{k}-2\int_{0}^{k\pi}\frac{u}{k}\cos u\:\frac{du}{k} \right]\\
                &=\frac{2}{\pi}\left[\frac{\pi}{k}\int_{0}^{k\pi}\cos u\:du-\frac{2}{k^2}\int_{0}^{k\pi}u\cos u\:du\right]\\
                &=\frac{2}{\pi}\left[\frac{\pi}{k}\sin u\Big|_{0}^{k\pi} -\frac{2}{k^2}\left[u\sin u+\cos u\Big|_{0}^{k\pi}\right] \right]\\
                &=\frac{2}{\pi}\left[\frac{\pi}{k}\left[\sin k\pi-0 \right] -\frac{2}{k^2}\left[k\pi\sin k\pi+\cos k\pi-0-\cos0\right] \right]\\
                &=\frac{2}{\pi}\left[\frac{2}{k^2}\left[1-(-1)^k\right]\right]\\
                &=\frac{4}{\pi k^2}\left[1-(-1)^k \right]\\
            \end{split}
        \end{equation*}
        si $k=2m-1$ con $m\in\mathbb{N}$, entonces
        \begin{equation*}
            \begin{split}
                a_k&=\frac{4}{\pi k^2}\left[1-(-1)^k\right]\\
                &=\frac{4}{\pi k^2}\left[1-(-1)^{2m-1}\right]\\
                &=\frac{4}{\pi k^2}\left[1+1 \right]\\
                &=\frac{8}{\pi k^2}\\
            \end{split}
        \end{equation*}
        y, si $k=2m$ con $m\in\mathbb{N}$ se tiene que $a_k=0$. Por tanto, se tiene que
        \begin{equation*}
            \begin{split}
                \pi-\abs{2x}&=\frac{a_0}{2}+\sum_{ k=1}^\infty a_k\cos kx\\
                &=\sum_{ k=1}^\infty \frac{8}{\pi (2k-1)^2}\cos (2k-1)x\\
                &=\frac{8}{\pi}\sum_{ k=1}^\infty\frac{\cos (2k-1)x}{(2k-1)^2},\quad\forall x\in[-\pi,\pi] \\
            \end{split}
        \end{equation*}
        en particular, lo anterior se cumple para $x=0$, es decir que
        \begin{equation*}
            \begin{split}
                \pi&=\frac{8}{\pi}\sum_{ k=1}^\infty\frac{1}{(2k-1)^2}\\
                \Rightarrow \sum_{ k=1}^\infty\frac{1}{(2k-1)^2}&=\frac{\pi^2}{8}\\
            \end{split}
        \end{equation*}
        Ahora, se sabe que en particular $f\in\mathcal{L}_2^{2\pi}(\mathbb{R})$. Por las identidades de Parserval se tiene que:
        \begin{equation*}
            \begin{split}
                \sum_{ k=1}^\infty\abs{a_{2k-1}}^2&=\frac{1}{\pi}\int_{-\pi}^\pi\abs{f(x)}^2\:dx\\
                \Rightarrow \sum_{ k=1}^\infty\frac{64}{\pi^2(2k-1)^4}&=\frac{1}{\pi}\int_{-\pi}^\pi\abs{\pi-\abs{2x}}^2\:dx\\
                &=\frac{1}{\pi}\left[\int_{-\pi}^{-\frac{\pi}{2}}\abs{\pi-\abs{2x}}^2\:dx +\int_{-\frac{\pi}{2}}^{\frac{\pi}{2}}\abs{\pi-\abs{2x}}^2\:dx+\int_{\frac{\pi}{2}}^{\pi}\abs{\pi-\abs{2x}}^2\:dx \right]\\
                &=\frac{1}{\pi}\left[\int_{-\pi}^{-\frac{\pi}{2}}(\abs{2x}-\pi)^2\:dx +\int_{-\frac{\pi}{2}}^{\frac{\pi}{2}}(\pi-\abs{2x})^2\:dx+\int_{\frac{\pi}{2}}^{\pi}(\abs{2x}-\pi)^2\:dx \right]\\
                &=\frac{1}{\pi}\left[\int_{-\pi}^{-\frac{\pi}{2}}(-2x-\pi)^2\:dx+\int_{-\frac{\pi}{2}}^{0}(\pi+2x)^2\:dx+\int_{0}^{\frac{\pi}{2}}(\pi-2x)^2\:dx+\int_{\frac{\pi}{2}}^{\pi}(2x-\pi)^2\:dx \right]\\
                &=\frac{1}{\pi}\left[\int_{-\pi}^{0}(\pi+2x)^2\:dx+\int_{0}^{\pi}(\pi-2x)^2\:dx \right]\\
                &=\frac{1}{\pi}\left[\int_{\pi}^{0}(\pi-2x)^2\:(-dx)+\int_{0}^{\pi}(\pi-2x)^2\:dx \right]\\
                &=\frac{1}{\pi}\left[\int_{0}^{\pi}(\pi-2x)^2\:dx+\int_{0}^{\pi}(\pi-2x)^2\:dx \right]\\
                &=\frac{2}{\pi}\int_{0}^{\pi}(\pi-2x)^2\:dx,\textup{ sea }u=\pi-2x \\
                &=\frac{2}{\pi}\int_{\pi}^{-\pi}u^2\:\frac{-du}{2}\\
                &=\frac{1}{\pi}\frac{u^3}{3}\Big|_{\pi}^{-\pi}\\
                &=\frac{1}{\pi}\frac{2\pi^3}{3}\\
                &=\frac{2\pi^2}{3}\\
                \Rightarrow \sum_{ k=1}^\infty\frac{64}{\pi^2(2k-1)^4}&=\frac{2\pi^2}{3}\\
                \Rightarrow \sum_{ k=1}^\infty\frac{1}{(2k-1)^4}&=\frac{\pi^4}{96}\\
            \end{split}
        \end{equation*}
    \end{sol}

    \begin{excer}
        Sea $f\in\mathcal{L}_1^{2\pi}(\mathbb{R})$ la función
        \begin{equation*}
            f(x)=\left\{ 
                \begin{array}{lcr}
                    0 & \textup{ si } & -\pi\leq x<0,\\
                    x^2 & \textup{ si } & 0\leq x<\pi.\\
                \end{array}
            \right.
        \end{equation*}
        \textbf{Calcule} la serie de Fourier de $f$. Usando el teorema fundamental para la convergencia puntual de una serie de Fourier, \textbf{muestre} que la serie de Fourier converge a alguna suma $s(x)$ para todo $x\in[-\pi,\pi]$. \textbf{Calcule} $s(x)$ para todo $x\in[-\pi,\pi]$.
    \end{excer}

    \begin{sol}
        Calculemos los coeficientes de Fourier de $f$:
        \begin{itemize}
            \item Veamos que
            \begin{equation*}
                \begin{split}
                    a_0&=\frac{1}{\pi}\int_{-\pi}^\pi f(x)\:dx\\
                    &=\frac{1}{\pi}\left[\int_{-\pi}^0 f(x)\:dx+\int_{0}^\pi f(x)\:dx \right]\\
                    &=\frac{1}{\pi}\left[\int_{-\pi}^0 0\:dx+\int_{0}^\pi x^2\:dx\right]\\
                    &=\frac{1}{\pi}\left[\frac{x^3}{3}\Big|_{0}^\pi \right]\\
                    &=\frac{1}{\pi}\cdot\frac{\pi^3}{3}\\
                    &=\frac{\pi^2}{3}\\
                \end{split}
            \end{equation*}

            \item Sea $k\in\mathbb{N}$, entonces:
            \begin{equation*}
                \begin{split}
                    a_k&=\frac{1}{\pi}\int_{-\pi}^{\pi}f(x)\cos kx\:dx\\
                    &=\frac{1}{\pi}\left[\int_{-\pi}^{0}f(x)\cos kx\:dx+\int_{0}^{\pi}f(x)\cos kx\:dx\right] \\
                    &=\frac{1}{\pi}\left[\int_{-\pi}^{0}0\cdot\cos kx\:dx+\int_{0}^{\pi}x^2\cos kx\:dx\right]\\
                    &=\frac{1}{\pi}\int_{0}^{\pi}x^2\cos kx\:dx,\textup{ sea }u=kx \\
                    &=\frac{1}{\pi}\int_{0}^{k\pi}\left(\frac{u}{k}\right)^2\cos u\:\frac{du}{k}\\
                    &=\frac{1}{k^3\pi}\int_{0}^{k\pi}u^2\cos u\:du\\
                    &=\frac{1}{k^3\pi}\left[(u^2-2)\sin u+2u\cos u\Big|_{0}^{k\pi} \right]\\
                    &=\frac{1}{k^3\pi}\left[((k\pi)^2-2)\sin k\pi+2k\pi\cos k\pi-(0^2-2)\sin 0-2\cdot0\cdot\cos0\right]\\
                    &=\frac{1}{k^3\pi}\cdot2k\pi(-1)^{k}\\
                    &=\frac{2(-1)^{k}}{k^2}\\
                \end{split}
            \end{equation*}
            y,
            \begin{equation*}
                \begin{split}
                    b_k&=\frac{1}{\pi}\int_{-\pi}^{\pi}f(x)\sin kx\:dx\\
                    &=\frac{1}{\pi}\left[\int_{-\pi}^{0}f(x)\sin kx\:dx+\int_{0}^{\pi}f(x)\sin kx\:dx\right] \\
                    &=\frac{1}{\pi}\left[\int_{-\pi}^{0}0\cdot\sin kx\:dx+\int_{0}^{\pi}x^2\sin kx\:dx\right]\\
                    &=\frac{1}{\pi}\int_{0}^{\pi}x^2\sin kx\:dx,\textup{ sea }u=kx \\
                    &=\frac{1}{\pi}\int_{0}^{k\pi}\left(\frac{u}{k}\right)^2\sin u\:\frac{du}{k}\\
                    &=\frac{1}{k^3\pi}\int_{0}^{k\pi}u^2\sin u\:du\\
                    &=\frac{1}{k^3\pi}\left[2u\sin u-(u^2-2)\cos u\Big|_{0}^{k\pi}\right]\\
                    &=\frac{1}{k^3\pi}\left[2k\pi\sin k\pi-((k\pi)^2-2)\cos k\pi-2\cdot0\cdot\sin 0+(0^2-2)\cos 0\right]\\
                    &=\frac{1}{k^3\pi}\left[-((k\pi)^2-2)(-1)^k-2\right]\\
                    &=\frac{1}{k^3\pi}\left[((k\pi)^2-2)(-1)^{k+1}-2\right]\\
                    &=\frac{1}{k^3\pi}\left[(k\pi)^2(-1)^{k+1}+2(-1)^k-2\right]\\
                    &=\frac{(k\pi)^2(-1)^{k+1}+2(-1)^k-2}{k^3\pi}\\
                \end{split}
            \end{equation*}
            Se tienen dos casos, si $k=2m-1$ con $m\in\mathbb{N}$, entonces:
            \begin{equation*}
                \begin{split}
                    a_{2m-1}&=\frac{1}{k^3\pi}\left[(k\pi)^2(-1)^{2m-1+1}+2(-1)^{2m-1}-2\right]\\
                    &=\frac{1}{k^3\pi}\left[(k\pi)^2(-1)^{2m}-2-2\right]\\
                    &=\frac{1}{k^3\pi}\left[(k\pi)^2-4\right]\\
                    &=\frac{(k\pi)^2-4}{k^3\pi}\\
                \end{split}
            \end{equation*}
            y, si $k=2m$ con $m\in\mathbb{N}$:
            \begin{equation*}
                \begin{split}
                    a_k&=\frac{1}{k^3\pi}\left[(k\pi)^2(-1)^{2m+1}+2(-1)^{2m}-2\right]\\
                    &=\frac{1}{k^3\pi}\left[-(k\pi)^2+2-2\right]\\
                    &=-\frac{(k\pi)^2}{k^3\pi}\\
                    &=\frac{-\pi}{k}\\
                \end{split}
            \end{equation*}
        \end{itemize}
        Por tanto, la serie de Fourier de $f$ es:
        \begin{equation*}
            \begin{split}
                s_f(x)&=\frac{a_0}{2}+\sum_{ k=1}^\infty\left(a_k\cos kx+b_k\sin kx\right)\\
                &=\frac{\pi^2}{6}+\sum_{ k=1}^\infty\left(\frac{2(-1)^k}{k^2}\cos kx+\frac{(k\pi)^2(-1)^{k+1}+2(-1)^k-2}{k^3\pi}\sin kx\right),\quad\forall x\in[-\pi,\pi[ \\
            \end{split}
        \end{equation*}
        Ahora, sea $x\in[-\pi,\pi]$. Para que la serie de Fourier de $f$ converja en $x$ a una suma $s(x)$, es necesario y suficiente que para algún $0<\delta<\pi$ se cumpla:
        \begin{equation*}
            \lim_{m\rightarrow\infty}\int_{0}^\delta\frac{f(x+t)+f(x-t)-2s(x)}{t}\sin\left(m+\frac{1}{2}\right)t=0
        \end{equation*}
        afirmamos que la serie de Fourier converge puntualmente a $f$ en $]-\pi,\pi[$ y en $x=\pi$ o $x=-\pi$ lo hace a $\frac{\pi^2}{2}$, esto es:
        \begin{equation*}
            s(x)=\left\{
                \begin{array}{lcr}
                    f(x) & \textup{ si } & x\in]-\pi,\pi[\\
                    \frac{\pi^2}{2} & \textup{ si } & x=\pi\textup{ ó }x=-\pi\\
                \end{array}
            \right.
        \end{equation*}
        \begin{itemize}
            \item Si $x\in]-\pi,\pi[$, se tienen dos casos:
            \begin{itemize}
                \item $x\in]-\pi,0[$, tomemos $\delta=\min\left\{-x,\pi+x\right\}>0$, entonces:
                \begin{equation*}
                    \begin{split}
                        \lim_{m\rightarrow\infty}\int_{0}^\delta\frac{f(x+t)+f(x-t)-2s(x)}{t}\sin\left(m+\frac{1}{2}\right)t&=\lim_{m\rightarrow\infty}\int_{0}^{\delta}\frac{0+0-2\cdot0}{t}\sin\left(m+\frac{1}{2}\right)t\\
                        &=\lim_{m\rightarrow\infty}\int_{0}^{\delta}0\cdot\sin\left(m+\frac{1}{2}\right)t\\
                        &=\lim_{m\rightarrow\infty}0\\
                        &=0\\
                    \end{split}
                \end{equation*}
                \item $x\in]0,\pi[$, tomemos $\delta=\min\left\{x,\pi-x\right\}>0$, entonces:
                \begin{equation*}
                    \begin{split}
                        \int_{0}^\delta\frac{f(x+t)+f(x-t)-2s(x)}{t}\sin\left(m+\frac{1}{2}\right)t&=\int_{0}^{\delta}\frac{(x+t)^2+(x-t)^2-2x^2}{t}\sin\left(m+\frac{1}{2}\right)t\\
                        &=\int_{0}^{\delta}\frac{2x^2+2t^2-2x^2}{t}\sin\left(m+\frac{1}{2}\right)t\\
                        &=\int_{0}^{\delta}\frac{2t^2}{t}\sin\left(m+\frac{1}{2}\right)t\\
                        &=2\int_{0}^{\delta}t\sin\left(m+\frac{1}{2}\right)t\\
                    \end{split}
                \end{equation*}
                Por Riemman-Lebsgue, como la función $t\mapsto t$ está en $\mathcal{L}_1(]0,\pi[,\mathbb{R})$, entonces
                \begin{equation*}
                    \lim_{ m\rightarrow\infty}\int_{0}^{\delta}t\sin\left(m+\frac{1}{2}\right)t=0
                \end{equation*}
                por tanto,
                \begin{equation*}
                    \lim_{m\rightarrow\infty}\int_{0}^\delta\frac{f(x+t)+f(x-t)-2s(x)}{t}\sin\left(m+\frac{1}{2}\right)t=0
                \end{equation*}
                \item $x=0$
            \end{itemize}
            (no es necesario hacer lo anterior ya que con solo afirmar que $s$ tiene derivadas por la derecha e izquierda en $x$ para todo $x\in]-\pi,\pi[$ se sigue la convergencia puntual de la serie de Fourier a $s(x)=f(x)$).
        \end{itemize}
    \end{sol}

    \begin{excer}
        Haga lo mismo que en el problema \textbf{3.12} con $f\in\mathcal{L}_1^{2\pi}(\mathbb{R})$ dada por
        \begin{equation*}
            f(x)=\left\{ 
                \begin{array}{lcr}
                    0 & \textup{ si } & -\pi\leq x<0,\\
                    x & \textup{ si } & 0\leq x<\pi.\\
                \end{array}
            \right.
        \end{equation*}
    \end{excer}

    \begin{sol}
        
    \end{sol}

\end{document}