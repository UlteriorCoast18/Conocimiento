\documentclass[12pt]{report}
\usepackage[spanish]{babel}
\usepackage[utf8]{inputenc}
\usepackage{amsmath}
\usepackage{amssymb}
\usepackage{amsthm}
\usepackage{graphics}
\usepackage{subfigure}
\usepackage{lipsum}
\usepackage{array}
\usepackage{multicol}
\usepackage{enumerate}
\usepackage[framemethod=TikZ]{mdframed}
\usepackage[a4paper, margin = 1.5cm]{geometry}

%En esta parte se hacen redefiniciones de algunos comandos para que resulte agradable el verlos%

\renewcommand{\theenumii}{\roman{enumii}}

\def\proof{\paragraph{Demostración:\\}}
\def\endproof{\hfill$\blacksquare$}

\def\sol{\paragraph{Solución:\\}}
\def\endsol{\hfill$\square$}

%En esta parte se definen los comandos a usar dentro del documento para enlistar%

\newtheoremstyle{largebreak}
  {}% use the default space above
  {}% use the default space below
  {\normalfont}% body font
  {}% indent (0pt)
  {\bfseries}% header font
  {}% punctuation
  {\newline}% break after header
  {}% header spec

\theoremstyle{largebreak}

\newmdtheoremenv[
    leftmargin=0em,
    rightmargin=0em,
    innertopmargin=-2pt,
    innerbottommargin=8pt,
    hidealllines = true,
    roundcorner = 5pt,
    backgroundcolor = gray!60!red!30
]{exa}{Ejemplo}[section]

\newmdtheoremenv[
    leftmargin=0em,
    rightmargin=0em,
    innertopmargin=-2pt,
    innerbottommargin=8pt,
    hidealllines = true,
    roundcorner = 5pt,
    backgroundcolor = gray!50!blue!30
]{obs}{Observación}[section]

\newmdtheoremenv[
    leftmargin=0em,
    rightmargin=0em,
    innertopmargin=-2pt,
    innerbottommargin=8pt,
    rightline = false,
    leftline = false
]{theor}{Teorema}[section]

\newmdtheoremenv[
    leftmargin=0em,
    rightmargin=0em,
    innertopmargin=-2pt,
    innerbottommargin=8pt,
    rightline = false,
    leftline = false
]{propo}{Proposición}[section]

\newmdtheoremenv[
    leftmargin=0em,
    rightmargin=0em,
    innertopmargin=-2pt,
    innerbottommargin=8pt,
    rightline = false,
    leftline = false
]{cor}{Corolario}[section]

\newmdtheoremenv[
    leftmargin=0em,
    rightmargin=0em,
    innertopmargin=-2pt,
    innerbottommargin=8pt,
    rightline = false,
    leftline = false
]{lema}{Lema}[section]

\newmdtheoremenv[
    leftmargin=0em,
    rightmargin=0em,
    innertopmargin=-2pt,
    innerbottommargin=8pt,
    roundcorner=5pt,
    backgroundcolor = gray!30,
    hidealllines = true
]{mydef}{Definición}[section]

\newmdtheoremenv[
    leftmargin=0em,
    rightmargin=0em,
    innertopmargin=-2pt,
    innerbottommargin=8pt,
    roundcorner=5pt
]{excer}{Ejercicio}[section]

%En esta parte se colocan comandos que definen la forma en la que se van a escribir ciertas funciones%

\newcommand\abs[1]{\ensuremath{\left|#1\right|}}
\newcommand\divides{\ensuremath{\bigm|}}
\newcommand\cf[3]{\ensuremath{#1:#2\rightarrow#3}}

%recuerda usar \clearpage para hacer un salto de página

\begin{document}
    \setlength{\parskip}{5pt} % Añade 5 puntos de espacio entre párrafos
    \setlength{\parindent}{12pt} % Pone la sangría como me gusta
    \title{Lista 3 de Ejercicios Análisis Matemático IV}
    \author{Cristo Daniel Alvarado}
    \maketitle

    \tableofcontents %Con este comando se genera el índice general del libro%

    \setcounter{chapter}{2} %En esta parte lo que se hace es cambiar la enumeración del capítulo%
    
    \chapter{Ejercicios}
    
    \setcounter{section}{1}

    \begin{excer}
        \textbf{Pruebe} que, para todo $x\in]0,2\pi[$,
        \begin{equation*}
            \frac{\pi-x}{2}=\sum_{ n=1}^\infty\frac{\sen nx}{n}
        \end{equation*}
        Usando la identidad de Parseval, \textbf{demuestre} que
        \begin{equation*}
            \sum_{ n=1}^\infty\frac{1}{n^2}=\frac{\pi^2}{6}.
        \end{equation*}
    \end{excer}

    \begin{proof}
        Sea $\cf{f}{\mathbb{R}}{\mathbb{R}}$ tal que
        \begin{equation*}
            f(x)=\frac{\pi-x}{2},\quad\forall x\in[0,2\pi[
        \end{equation*}
        y extiéndase por periodicidad a todo $\mathbb{R}$. Es claro que $f\in\mathcal{L}_1^{2\pi}(\mathbb{R})$, sea ahora $x\in]0,2\pi[$. Por el teorema fundamental para la convergencia puntual de una serie de Fourier hay que encontrar un $0<\delta<\pi$ tal que
        \begin{equation*}
            \lim_{ m\rightarrow\infty}\int_{0}^\delta\frac{f(x+t)+f(x-t)-2f(x)}{t}\sin\left(m+\frac{1}{2}\right)dt
        \end{equation*}
        tomemos $\delta=\min\left\{x, 2\pi-x \right\}>0$. Se tienen dos casos:
        \begin{enumerate}
            \item $\delta=x$, entonces
            \begin{equation*}
                \begin{split}
                    \int_{0}^\delta\frac{f(x+t)+f(x-t)-2f(x)}{t}&\sin\left(m+\frac{1}{2}\right)dt\\
                    &=\int_{0}^\delta \frac{1}{t}\left[\frac{\pi-x-t}{2}+\frac{\pi-x+t}{2}-\frac{2\left(\pi-x \right)}{2}\right]\sin\left(m+\frac{1}{2}\right)dt\\
                    &=\int_{0}^\delta \frac{1}{t}\left[\pi-x-\pi+x \right]\sin\left(m+\frac{1}{2}\right)dt\\
                    &=\int_{0}^\delta 0\:dt\\
                    &=0\\
                \end{split}
            \end{equation*}
            por tanto, el límite cuando $m\rightarrow\infty$ resulta que da cero.
            \item $\delta=2\pi x$. El caso es análogo al anterior.
        \end{enumerate}
        por ambos incisos se concluye que
        \begin{equation*}
            \lim_{ m\rightarrow\infty}\int_{0}^\delta\frac{f(x+t)+f(x-t)-2f(x)}{t}\sin\left(m+\frac{1}{2}\right)dt=0
        \end{equation*}
        por tanto, la serie de Fourier de $f$ converge a $f$ puntualmente en $x$. Computemos ahora los coeficientes de la serie de Fourier de $f$. Si $n\geq0$:
        \begin{equation*}
            \begin{split}
                a_n&=\frac{1}{\pi}\int_{-\pi}^{\pi}f(x)\cos nxdx\\
                &=\frac{1}{\pi}\int_{0}^{2\pi}f(x)\cos nxdx\\
                &=\frac{1}{\pi}\int_{0}^{2\pi}\frac{\pi-x}{2}\cos nxdx\\
                &=\frac{1}{2}\int_{0}^{2\pi}\cos nxdx-\frac{1}{2\pi}\int_0^{2\pi} x\cos nxdx\textup{ haciendo }u=nx \\
                &=\frac{1}{2}\int_{0}^{2n\pi}\cos u\frac{du}{n}-\frac{1}{2\pi}\int_0^{2n\pi} \frac{u}{n}\cos u\frac{du}{n} \\
                &=\frac{1}{2n}\sin u\Big|_0^{ 2n\pi}-\frac{1}{2\pi n^2}\int_0^{2n\pi} u\cos udu \\
                &=\frac{1}{2n}\left[\sin 2\pi n-\sin 0 \right] -\frac{1}{2\pi n^2}\left(u\sin u\Big|_{0}^{2n\pi}+\int_0^{2n\pi}\sin udu\right) \\
                &=-\frac{1}{2\pi n^2}\left(u\sin u\Big|_{0}^{2n\pi}+\int_0^{2n\pi}\sin udu\right) \\
                &=-\frac{1}{2\pi n^2}\left(\left[2n\pi\sin 2n\pi-0 \right]-\cos u \Big|_0^{2n\pi}\right) \\
                &-\frac{1}{2\pi n^2}\left(0-0-1+1 \right)\\
                &=0\\
            \end{split}
        \end{equation*}
        para todo $n\geq0$. Si $n\in\mathbb{N}$:
        \begin{equation*}
            \begin{split}
                b_n&=\frac{1}{\pi}\int_{-\pi}^{\pi}f(x)\sin nxdx\\
                &=\frac{1}{\pi}\int_{0}^{2\pi}f(x)\sin nxdx\\
                &=\frac{1}{\pi}\int_{0}^{2\pi}\frac{\pi-x}{2}\sin nxdx\\
                &=\frac{1}{2}\int_{0}^{2\pi}\sin nxdx-\frac{1}{2\pi}\int_0^{2\pi} x\sin nxdx\textup{ haciendo }u=nx \\
                &=\frac{1}{2}\int_{0}^{2n\pi}\sin u\frac{du}{n}-\frac{1}{2\pi}\int_0^{2n\pi} \frac{u}{n}\sin u\frac{du}{n} \\
                &=\frac{1}{2n}(-\cos u)\Big|_{0}^{2n\pi}-\frac{1}{2\pi n^2}\int_0^{2n\pi} u\sin udu \\
                &=\frac{1}{2n}(-\cos 2n\pi+1)-\frac{1}{2\pi n^2}\left(-u\cos u\Big|_{0}^{2n\pi}+\int_{0}^{2n\pi}\cos udu \right) \\
                &=\frac{1}{2n}(-1+1)-\frac{1}{2\pi n^2}\left(-2n\pi\cos 2n\pi+0+\sin u\Big|_{0}^{2n\pi} \right) \\
                &=-\frac{1}{2\pi n^2}\left(-2n\pi\cos 2n\pi+\sin 2n\pi-\sin 0 \right) \\
                &=-\frac{1}{2\pi n^2}\left(-2n\pi+\sin 2n\pi-\sin 0 \right) \\
                &=-\frac{1}{2\pi n^2}\left(-2n\pi \right) \\
                &=\frac{1}{n}\\
            \end{split}
        \end{equation*}
        Por tanto, la serie de Fourier de $f$ en $x\in]0,2\pi[$ está dada por:
        \begin{equation*}
            \frac{a_0}{2}+\sum_{ k=1}^\infty\left[a_k\cos kx+b_k\sin kx \right]=\sum_{ n=1}^\infty b_n\sin nx=\sum_{ n=1}^\infty\frac{\sin nx}{n}
        \end{equation*}
        Por el criterio de Dini se sigue que
        \begin{equation*}
            \frac{\pi-x}{2}=\sum_{ n=1}^\infty\frac{\sin nx}{n},\quad\forall x\in]0,2\pi[
        \end{equation*}

        Ahora, como $x\mapsto\frac{\pi-x}{2}$ es una función en $\mathcal{L}_2^{2\pi}$, por Parseval se tiene que
        \begin{equation*}
            \begin{split}
                \frac{\abs{a_0}^2}{2}+\sum_{n=1}^\infty\left[\abs{a_n}^2+\abs{b_n}^2 \right]&=\frac{1}{\pi}\int_{-\pi}^{\pi}\abs{\frac{\pi-x}{2}}^2\:dx\\
                \Rightarrow \sum_{ n=1}^\infty\frac{1}{n^2}&=\frac{1}{4\pi}\int_0^{2\pi}\abs{\pi-x}^2\:dx\\
                &=\frac{1}{2\pi}\int_0^{\pi}\abs{\pi-x}^2\:dx\\
                &=\frac{1}{2\pi}\int_0^{\pi}(\pi-x)^2\:dx\textup{ haciendo el cambio de variable }u=\pi-x \\
                &=\frac{1}{2\pi}\int_{\pi}^{0}-u^2\:du\\
                &=\frac{1}{2\pi}\cdot\frac{-u^3}{3}\Big|_{\pi}^{0}\\
                &=\frac{1}{2\pi}\cdot[-\frac{0}{3}+\frac{\pi^3}{3} ]\\
                &=\frac{1}{2\pi}\cdot\frac{\pi^3}{3}\\
                &=\frac{\pi^2}{6}\\
                \therefore \sum_{ n=1}^\infty\frac{1}{n^2}&=\frac{\pi^2}{6}\\
            \end{split}
        \end{equation*}
        Como se quería demostrar.
    \end{proof}

    \begin{excer}
        Sea $f\in\mathcal{L}_2^{2\pi}(\mathbb{R})$ y sean $a_n,b_n$ los coeficientes de Fourier de $f$. \textbf{Pruebe} que
        \begin{equation*}
            \frac{1}{\pi}\int_0^{2\pi}xf(x)dx=\pi a_0-2\sum_{ n=1}^\infty \frac{b_n}{n}.
        \end{equation*}
    \end{excer}

    \begin{proof}
        
    \end{proof}

    \begin{excer}
        Sea $\cf{f}{\mathbb{R}}{\mathbb{R}}$ periódica de periodo $2\pi$ definida como
        \begin{equation*}
            f(x)=\left\{
                \begin{array}{lcr}
                    \pi^2 & \textup{ si } & -\pi\leq x<0,\\
                    (x-\pi)^2 & \textup{ si } & 0\leq x <\pi.
                \end{array}
            \right.
        \end{equation*}
        calcule los coeficientes de Fourier $a_n$, con $n=0,1,2,...$ de $f$ y \textbf{pruebe} las fórmulas
        \begin{equation*}
            \sum_{ n=1}^\infty\frac{1}{n^2}=\frac{\pi^2}{6}\quad\textup{y}\quad\sum_{ n=1}^\infty\frac{(-1)^{ n-1}}{n^2}=\frac{\pi^2}{12}.
        \end{equation*}
    \end{excer}

    \begin{proof}
        
    \end{proof}

    \begin{excer}
        \textbf{Pruebe} que
        \begin{equation*}
            \frac{1}{3}x(\pi-x)(\pi-2x)=\sum_{ n=1}^\infty\frac{\sen 2nx}{n^3},\quad 0\leq x\leq \pi.
        \end{equation*}
        \textbf{Deduzca} el valor de
        \begin{equation*}
            \sum_{ n=1}^\infty\frac{(-1)^{ n-1}}{(2n-1)^3}.
        \end{equation*}
    \end{excer}

    \begin{proof}
        
    \end{proof}

    \renewcommand{\theenumi}{\textbf{\roman{enumi}}}
    
    \begin{excer}
        Haga lo siguiente:
        \begin{enumerate}
            \item \textbf{Pruebe} que
            
            \begin{equation*}
                \int_0^\pi\log\sen\frac{x}{2}dx=-\pi\log2.
            \end{equation*}
            \textit{Sugerencia}. Haga el cambio de variables $x=2t$ y escriba $\sen t =2\sen\frac{t}{2}\cos\frac{t}{2}$.
            \item \textbf{Muestre} que
            \begin{equation*}
                -\log\Big|2\sen\frac{x}{2}\Big|=\sum_{ n=1}^\infty\frac{\cos nx}{n},\quad\textup{ si }x\neq 2k\pi,k\in\mathbb{Z}.
            \end{equation*}
            \textit{Sugerencia.} Use el inciso (i) para probar que $a_0=0$. A fin de calcular $a_n$ para $n\in\mathbb{N}$, escriba $a_n=\frac{2}{\pi}\int_0^\pi\log\cos\frac{x}{2}dx$, efectúe una integración por partes y transforme el nuevo integrando de suerte que aparezca el núcleo de Dirichlet.
            \item \textbf{Deduzca} de (ii) la fórmula
            \begin{equation*}
                \log2=\sum_{ n=1}^\infty\frac{(-1)^{ n-1}}{n}.
            \end{equation*}
            \item \textbf{Desarrolle} en serie de Fourier la función
            \begin{equation*}
                x\mapsto\log\Big|2\cos\frac{x}{2}\Big|
            \end{equation*}
        \end{enumerate}
    \end{excer}

    \begin{sol}
        
    \end{sol}

    \begin{excer}
        Sea $f\in\mathcal{L}_1^{2\pi}(\mathbb{R})$ y sea $x\in\mathbb{R}$. Se supone que para algúun $\alpha>0$ se cumple
        \begin{equation*}
            f(x+t)-f(x)=O(\abs{t^\alpha}),\quad\textup{cuando}t\rightarrow 0
        \end{equation*}
        \textbf{Demuestre} que la serie de Fourier de $f$ en $x$ converge a $f(x)$.
    \end{excer}

    \begin{proof}
        
    \end{proof}

    \begin{excer}
        Por el problema \textbf{3.1.1} se sabe que
        \begin{equation*}
            \frac{\pi-x}{2}=\sum_{ n=1}^\infty\frac{\sen nx}{n}
        \end{equation*}
        \begin{enumerate}
            \item Póngase
            \begin{equation*}
                s_n(x)=\sum_{  k=1}^\infty\frac{\sen kx}{k}.
            \end{equation*}
            \textbf{Muestre} que
            \begin{equation*}
                \frac{x}{2}+s_n(x)=\pi\int_0^\pi D_n(t)dt,
            \end{equation*}
            donde $D_n$ es el núcleo de Dirichlet.
            \item Si $x\in]0,2\pi[$, \textbf{pruebe} que
            \begin{equation*}
                \lim_{ n\rightarrow\infty}\left[\pi\int_0^x D_n(t)dt- \int_0^x\frac{\sen nt}{t}dt \right]=0.
            \end{equation*}
            \item \textbf{Deduzca} una nueva demostración de la fórmula
            \begin{equation*}
                \int_0^{\rightarrow\infty}\frac{\sen t}{t}dt=\frac{\pi}{2}.
            \end{equation*}
        \end{enumerate}
    \end{excer}

    \begin{proof}
        
    \end{proof}

    \begin{excer}
        Sea $f\in\mathcal{L}_1^{2\pi}(\mathbb{C})$ y sean $\left\{c_k \right\}_{ k\in\mathbb{Z}}$ los coeficientes de Fourier de $f$. \textbf{Demuestre} que
        \begin{equation*}
            \int_0^x f=c+c_0x+\sum_{ k\in\mathbb{Z}\backslash\left\{0\right\}}\frac{c_ke^{ ikx}}{ik},\quad\forall x\in\mathbb{R}.
        \end{equation*}
        donde $c$ es una constante, la convergencia siendo uniforme en $\mathbb{R}$.

        \textit{Sugerencia.} Considere la función $F(x)=\int_0^x (f-c_0)$.

        \textbf{Deduzca} que los coeficientes de Fourier $b_n$ de cualquier función $f\in\mathcal{L}_1^{2\pi}(\mathbb{C})$ satisfacen la condición de que la serie
        \begin{equation*}
            \sum_{ n=1}^\infty\frac{b_n}{n}
        \end{equation*}
        es convergente. \textbf{Concluya} que la aplicación $f\mapsto\left\{c_n \right\}_{ n\in\mathbb{Z}}$ no es una aplicación suprayectiva de $\mathcal{L}_1^{2\pi}(\mathbb{C})$ en $c_0(\mathbb{Z})$.
    \end{excer}

    \begin{proof}
        
    \end{proof}

    \begin{excer}
        Haga lo siguiente:
        \begin{enumerate}
            \item Sea $\alpha$ un número real no entero. \textbf{Pruebe} que
            \begin{equation*}
                \pi\cos\alpha x=2\alpha\sen\pi\alpha\left(\frac{1}{2\alpha^2}+\sum_{ n=1}^\infty(-1)^n\frac{\cos nx}{\alpha^2-n^2} \right),\quad\forall x\in[-\pi,\pi].
            \end{equation*}
            De ahí obtenga las fórmulas clásicas
            \begin{equation*}
                \frac{\pi\alpha}{\sen\pi\alpha}=1+2\alpha^2\sum_{ n=1}^\infty\frac{(-1)^n}{\alpha^2-n^2}\quad\textup{y}\quad\pi\alpha\cot\pi\alpha=1+2\alpha^2\sum_{ n=1}^\infty\frac{1}{\alpha^2-n^2}.
            \end{equation*}
            \item Sea $x\in]0,1[$. \textbf{Pruebe} que la serie
            \begin{equation*}
                \sum_{ n=1}^\infty\frac{2\alpha}{n^2-\alpha^2}
            \end{equation*}
            se puede integrar término por término en el intervalo $[0,x]$. De la última fórmula del inciso (i) \textbf{deduzca} la fórmula
            \begin{equation*}
                \sen\pi x=\pi x\prod_{ n=1}^\infty\left(1-\frac{x^2}{n^2} \right),\quad\forall x\in]-1,1[.s
            \end{equation*}
        \end{enumerate}
    \end{excer}

    \begin{proof}
        
    \end{proof}

    \begin{excer}
        Se supone que la serie de Fourier de una función $f\in\mathcal{L}_1^{2\pi}(\mathbb{K})$ converge en el sentido de Cesáro uniformemente en $\mathbb{R}$. \textbf{Pruebe} que $f$ es equivalente a una función continua de $\mathbb{R}$ en $\mathbb{K}$.
    \end{excer}

    \begin{proof}
        
    \end{proof}

    \begin{excer}
        Sea $f\in\mathcal{C}^{2\pi}(\mathbb{R})$ la función
        \begin{equation*}
            f(x)=\pi-\abs{2x},\quad-\pi\leq x\leq\pi
        \end{equation*}
        Aplique el teorema 3.9 para mostrar que la serie de Fourier de $f$ converge a $f$ uniformemente en $\mathbb{R}$. \textbf{Calcule}
        \begin{equation*}
            \sum_{ k=1}^\infty\frac{1}{(2k-1)^2}\quad\textup{y}\quad\sum_{ k=1}^\infty\frac{1}{(2k-1)^4}.
        \end{equation*}
    \end{excer}

    \begin{sol}
        
    \end{sol}

    \begin{excer}
        Sea $f\in\mathcal{L}_1^{2\pi}(\mathbb{R})$ la función
        \begin{equation*}
            f(x)=\left\{ 
                \begin{array}{lcr}
                    0 & \textup{ si } & -\pi\leq x<0,\\
                    x^2 & \textup{ si } & 0\leq x<\pi.\\
                \end{array}
            \right\}
        \end{equation*}
        \textbf{Calcule} la serie de Fourier de $f$. Usando el teorema fundamental para la convergencia de una serie de Fourier, \textbf{muestre} que la serie de Fourier de $f$ converge a alguna suma $s(x)$ para todo $x\in[-\pi,\pi]$. \textbf{Calcule} $s(x)$ para todo $x\in[-\pi,\pi]$.
    \end{excer}

    \begin{sol}
        
    \end{sol}

    \begin{excer}
        Haga lo mismo que en el problema \textbf{3.12} con $f\in\mathcal{L}_1^{2\pi}(\mathbb{R})$ dada por
        \begin{equation*}
            f(x)=\left\{ 
                \begin{array}{lcr}
                    0 & \textup{ si } & -\pi\leq x<0,\\
                    x & \textup{ si } & 0\leq x<\pi.\\
                \end{array}
            \right.
        \end{equation*}
    \end{excer}

    \begin{sol}
        
    \end{sol}

\end{document}