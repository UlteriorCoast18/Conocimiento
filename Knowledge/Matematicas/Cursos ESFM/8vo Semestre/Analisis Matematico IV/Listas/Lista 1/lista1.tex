\documentclass[12pt]{report}
\usepackage[spanish]{babel}
\usepackage[utf8]{inputenc}
\usepackage{amsmath}
\usepackage{amssymb}
\usepackage{amsthm}
\usepackage{graphics}
\usepackage{subfigure}
\usepackage{lipsum}
\usepackage{array}
\usepackage{multicol}
\usepackage{enumerate}
\usepackage[framemethod=TikZ]{mdframed}
\usepackage[a4paper, margin = 1.5cm]{geometry}

%En esta parte se hacen redefiniciones de algunos comandos para que resulte agradable el verlos%

\renewcommand{\theenumii}{\roman{enumii}}

\def\proof{\paragraph{Demostración:\\}}
\def\endproof{\hfill$\blacksquare$}

\def\sol{\paragraph{Solución:\\}}
\def\endsol{\hfill$\square$}

%En esta parte se definen los comandos a usar dentro del documento para enlistar%

\newtheoremstyle{largebreak}
  {}% use the default space above
  {}% use the default space below
  {\normalfont}% body font
  {}% indent (0pt)
  {\bfseries}% header font
  {}% punctuation
  {\newline}% break after header
  {}% header spec

\theoremstyle{largebreak}

\newmdtheoremenv[
    leftmargin=0em,
    rightmargin=0em,
    innertopmargin=0pt,
    innerbottommargin=5pt,
    hidealllines = true,
    roundcorner = 5pt,
    backgroundcolor = gray!60!red!30
]{exa}{Ejemplo}[section]

\newmdtheoremenv[
    leftmargin=0em,
    rightmargin=0em,
    innertopmargin=0pt,
    innerbottommargin=5pt,
    hidealllines = true,
    roundcorner = 5pt,
    backgroundcolor = gray!50!blue!30
]{obs}{Observación}[section]

\newmdtheoremenv[
    leftmargin=0em,
    rightmargin=0em,
    innertopmargin=0pt,
    innerbottommargin=5pt,
    rightline = false,
    leftline = false
]{theor}{Teorema}[section]

\newmdtheoremenv[
    leftmargin=0em,
    rightmargin=0em,
    innertopmargin=0pt,
    innerbottommargin=5pt,
    rightline = false,
    leftline = false
]{propo}{Proposición}[section]

\newmdtheoremenv[
    leftmargin=0em,
    rightmargin=0em,
    innertopmargin=0pt,
    innerbottommargin=5pt,
    rightline = false,
    leftline = false
]{cor}{Corolario}[section]

\newmdtheoremenv[
    leftmargin=0em,
    rightmargin=0em,
    innertopmargin=0pt,
    innerbottommargin=5pt,
    rightline = false,
    leftline = false
]{lema}{Lema}[section]

\newmdtheoremenv[
    leftmargin=0em,
    rightmargin=0em,
    innertopmargin=0pt,
    innerbottommargin=5pt,
    roundcorner=5pt,
    backgroundcolor = gray!30,
    hidealllines = true
]{mydef}{Definición}[section]

\newmdtheoremenv[
    leftmargin=0em,
    rightmargin=0em,
    innertopmargin=0pt,
    innerbottommargin=5pt,
    roundcorner=5pt
]{excer}{Ejercicio}[section]

%En esta parte se colocan comandos que definen la forma en la que se van a escribir ciertas funciones%

\newcommand\abs[1]{\ensuremath{\big|#1\big|}}
\newcommand\divides{\ensuremath{\bigm|}}
\newcommand\cf[3]{\ensuremath{#1:#2\rightarrow#3}}
\newcommand{\contradiction}{\ensuremath{\#_c}}
\newcommand\adj[1]{\ensuremath{\widetilde{#1}}}
\newcommand\pint[2]{\ensuremath{\left(#1\big|#2\right)}}
\newcommand\conj[1]{\ensuremath{\overline{#1}}}
\newcommand\norm[1]{\ensuremath{\|#1\|}}

%recuerda usar \clearpage para hacer un salto de página

\begin{document}
    \title{Ejercicios Análisis Matemático IV}
    \author{Cristo Daniel Alvarado}
    \maketitle

    \tableofcontents %Con este comando se genera el índice general del libro%

    \chapter{Espacios Hilbertianos}

    \section{Ejercicios}

    \renewcommand{\theenumi}{\roman{enumi}}

    \begin{excer}
        Pruebe lo siguiente:
        \begin{enumerate}
            \item Sean $H,H'$ espacios hilbertianos y sea $T$ una aplicación lineal continua de $H$ en $H'$. \textbf{Demuestre} que existe una única aplicación lineal $\cf{\adj{T}}{H'}{H}$ tal que
            \begin{equation*}
                \pint{\vec{x}}{\adj{T}\vec{x'}}=\pint{T\vec{x}}{\vec{x'}},\quad\forall\vec{x}\in H\textup{ y }\forall\vec{x'}\in H'
            \end{equation*}
            \textbf{Pruebe} también que $\adj{T}$ es continua, $\adj{\adj{T}}=T$ y $\norm{\adj{T}}=\norm{T}$. El operador $\adj{T}$ se llama la \textbf{adjunta de $T$}.
            \item \textbf{Demuestre} las reglas:
            \begin{equation*}
                \adj{T_1+T_2}=\adj{T_1}+\adj{T_2}\quad\textup{y}\quad\adj{\alpha T}=\conj{\alpha}\adj{T}
            \end{equation*}
            \item Sea $H''$ un tercer espacio hilbertiano. Sean $T$ una aplicación lineal continua de $H$ en $H'$ y $U$ una aplicación lineal continua de $H'$ en $H''$. \textbf{Pruebe} que:
            \begin{equation*}
                \adj{U\circ T}=\adj{T}\circ\adj{U}
            \end{equation*}
        \end{enumerate}
    \end{excer}

    \begin{proof}
        De (i): Se probarán dos cosas:
        \begin{itemize}
            \item \textbf{Unicidad}. Suponga que existen $\cf{S,W}{H'}{H}$ tales que:
            \begin{equation*}
                \pint{\vec{x}}{S\vec{x'}}=\pint{T\vec{x}}{\vec{x'}}\quad\textup{y}\quad\pint{\vec{x}}{W\vec{x'}}=\pint{T\vec{x}}{\vec{x'}},\quad\forall\vec{x}\textup{ y }\vec{x'}\in H'
            \end{equation*}
            entonces, se tiene que para $\vec{x'}\in H'$ fijo:
            \begin{equation}
                \begin{split}
                    \pint{\vec{x}}{S\vec{x'}}=&\pint{\vec{x}}{W\vec{x'}}\\
                    \Rightarrow \pint{\vec{x}}{S\vec{x'}}-\pint{\vec{x}}{W\vec{x'}}=&0\\
                    \Rightarrow \pint{\vec{x}}{S\vec{x'}-W\vec{x'}}=&0\forall\vec{x}\in H \\
                \end{split}
            \end{equation}
            Por tanto, $S\vec{x'}=W\vec{x'}$. Como el $\vec{x'}\in H'$ fue arbitrario, se sigue que $S=W$.

            \item \textbf{Existencia}. Para cada $\vec{x'}\in H'$, sea $\cf{L_{\vec{x'}}}{H}{\mathbb{K}}$ definida como sigue:
            \begin{equation*}
                L_{\vec{x'}}(\vec{x})=\pint{T\vec{x}}{\vec{x'}}
            \end{equation*}
            Afirmamos que $L_{\vec{x'}}$ es lineal continuo. En efecto, si $\vec{x},\vec{y}\in H$ y $\alpha\in\mathbb{K}$, tenemos que:
            \begin{equation*}
                \begin{split}
                    L_{\vec{x'}}(\vec{x}+\alpha\vec{y})=&\pint{T(\vec{x}+\alpha\vec{y})}{\vec{x'}} \\
                    =&\pint{T\vec{x}}{\vec{x'}}+\alpha\pint{T\vec{y}}{\vec{x'}} \\
                    =& L_{\vec{x'}}(\vec{x})+\alpha L_{\vec{x'}}(\vec{y})\\
                \end{split}
            \end{equation*}
            luego es lineal, y es continuo ya que
            \begin{equation*}
                \begin{split}
                    \abs{L_{\vec{x}}(\vec{x'})}=&\abs{\pint{T\vec{x}}{\vec{x'}}}\\
                    \leq&\norm{T\vec{x}}\norm{\vec{x'}} \\
                    \leq&(\norm{T}\norm{\vec{x'}})\norm{\vec{x}} \\
                \end{split}
            \end{equation*}
            donde la primera desigualdad es por Cauchy-Schwarts, y la segunda es por el hecho de que $T$ es un funcional lineal continuo. Por tanto: $\norm{L_{\vec{x'}}}\leq\norm{T}\norm{\vec{x'}}$. Luego, $L_{\vec{x'}}$ es lineal continuo, i.e. $L_{\vec{x'}}\in H^*$.

            Por el teorema de Riesz, como la aplicación $\cf{G}{H}{H^*}$ es suprayectiva, para $\vec{x'}\in H'$ existe $\adj{T}\vec{x'}\in H$ tal que $L_{\vec{x'}}=G_{\adj{T}\vec{x'}}$, es decir que:
            \begin{equation*}
                L_{\vec{x'}}\vec{x}=\pint{T\vec{x}}{\vec{x'}}=\pint{\vec{x}}{\adj{T}\vec{x'}}=G_{\adj{T}\vec{x'}}\vec{x},\quad\forall\vec{x}\in H
            \end{equation*}
            Afirmamos que la aplicación $\cf{\adj{T}}{H'}{H}$ está bien definida y es lineal. En efecto, si $\adj{T}\vec{x_1'},\adj{T}\vec{x_2'}\in H$ son tales que $L_{\vec{x'}}=G_{\adj{T}\vec{x_1'}}$ y $L_{\vec{x}}=G_{\adj{T}\vec{x_1'}}$, entonces:
            \begin{equation*}
                \pint{T\vec{x}}{\vec{x'}}=\pint{\vec{x}}{\adj{T}\vec{x_1'}}\quad\textup{y}\quad\pint{T\vec{x}}{\vec{x'}}=\pint{\vec{x}}{\adj{T}\vec{x_2'}},\quad\forall\vec{x}\in H
            \end{equation*}
            entonces:
            \begin{equation*}
                \begin{split}
                    \pint{\vec{x}}{\adj{T}\vec{x_1'}}=&\pint{\vec{x}}{\adj{T}\vec{x_2'}},\quad\forall\vec{x}\in H\\
                    \Rightarrow \pint{\vec{x}}{\adj{T}\vec{x_1'}-\adj{T}\vec{x_2'}}=&0\quad\forall\vec{x}\in H\\
                    \Rightarrow\adj{T}\vec{x_1'}-\adj{T}\vec{x_2'}=&\vec{0}\\
                    \Rightarrow\adj{T}\vec{x_1'}=&\adj{T}\vec{x_2'}\\ 
                \end{split}
            \end{equation*}
            por tanto, $\cf{\adj{T}}{H'}{H}$ está bien definida. Comprobemos ahora la linealidad, sean $\vec{x'},\vec{y'}\in H'$, entonces:
            \begin{equation*}
                \begin{split}
                    &\pint{T\vec{x}}{\vec{x'}}=\pint{\vec{x}}{\adj{T}\vec{x'}},\pint{T\vec{x}}{\vec{y'}}=\pint{\vec{x}}{\adj{T}\vec{y'}}\textup{ y }\pint{T\vec{x}}{\vec{x'}+\vec{y'}}=\pint{\vec{x}}{\adj{T}(\vec{x'}+\vec{y'})}\quad\forall\vec{x}\in H\\
                    \Rightarrow& \pint{T\vec{x}}{\vec{x'}}+\pint{T\vec{x}}{\vec{y'}}=\pint{\vec{x}}{\adj{T}\vec{x'}}+\pint{\vec{x}}{\adj{T}\vec{y'}}\textup{ y }\pint{T\vec{x}}{\vec{x'}+\vec{y'}}=\pint{\vec{x}}{\adj{T}(\vec{x'}+\vec{y'})}\quad\forall\vec{x}\in H\\
                    \Rightarrow& \pint{T\vec{x}}{\vec{x'}+\vec{y'}}=\pint{\vec{x}}{\adj{T}\vec{x'}+\adj{T}\vec{y'}}\textup{ y }\pint{T\vec{x}}{\vec{x'}+\vec{y'}}=\pint{\vec{x}}{\adj{T}(\vec{x'}+\vec{y'})}\quad\forall\vec{x}\in H\\
                    \Rightarrow& \pint{\vec{x}}{\adj{T}\vec{x'}+\adj{T}\vec{y'}}=\pint{\vec{x}}{\adj{T}(\vec{x'}+\vec{y'})}\quad\forall\vec{x}\in H\\
                    \Rightarrow& \pint{\vec{x}}{(\adj{T}\vec{x'}+\adj{T}\vec{y'})-\adj{T}(\vec{x'}+\vec{y'})}=0\quad\forall\vec{x}\in H\\
                    \Rightarrow& (\adj{T}\vec{x'}+\adj{T}\vec{y'})-\adj{T}(\vec{x'}+\vec{y'})=0\\
                    \Rightarrow& \adj{T}\vec{x'}+\adj{T}\vec{y'}=\adj{T}(\vec{x'}+\vec{y'})\\
                \end{split}
            \end{equation*}
            y, si $\alpha\in\mathbb{K}$, tenemos que:
            \begin{equation*}
                \begin{split}
                    &\pint{T\vec{x}}{\alpha\vec{x'}}=\pint{\vec{x}}{\adj{T}(\alpha\vec{x'})}\textup{ y }\pint{T\vec{x}}{\vec{x'}}=\pint{\vec{x}}{\adj{T}\vec{x'}}\quad\forall\vec{x}\in H\\
                    \Rightarrow&\conj{\alpha} \pint{T\vec{x}}{\vec{x'}}=\pint{\vec{x}}{\adj{T}(\alpha\vec{x'})}\textup{ y }\conj{\alpha}\pint{T\vec{x}}{\vec{x'}}=\conj{\alpha}\pint{\vec{x}}{\adj{T}\vec{x'}}\quad\forall\vec{x}\in H\\
                    \Rightarrow&\conj{\alpha} \pint{T\vec{x}}{\vec{x'}}=\pint{\vec{x}}{\adj{T}(\alpha\vec{x'})}\textup{ y }\conj{\alpha}\pint{T\vec{x}}{\vec{x'}}=\pint{\vec{x}}{\alpha\adj{T}\vec{x'}}\quad\forall\vec{x}\in H\\
                    \Rightarrow&\pint{\vec{x}}{\adj{T}(\alpha\vec{x'})}=\pint{\vec{x}}{\alpha\adj{T}\vec{x'}}\quad\forall\vec{x}\in H\\
                    \Rightarrow&\pint{\vec{x}}{\adj{T}(\alpha\vec{x'})}-\pint{\vec{x}}{\alpha\adj{T}\vec{x'}}=0\quad\forall\vec{x}\in H\\
                    \Rightarrow&\pint{\vec{x}}{\adj{T}(\alpha\vec{x'})-\alpha\adj{T}\vec{x'}}=0\quad\forall\vec{x}\in H\\
                    \Rightarrow&\adj{T}(\alpha\vec{x'})-\alpha\adj{T}\vec{x'}=\vec{0}\\
                    \Rightarrow&\adj{T}(\alpha\vec{x'})=\alpha\adj{T}\vec{x'}\\
                \end{split}
            \end{equation*}
            por tanto $\adj{T}$ es lineal. Además, se cumple para todos $\vec{x}\in H$ y $\vec{x'}\in H'$ que:
            \begin{equation*}
                \pint{T\vec{x}}{\vec{x'}}=\pint{\vec{x}}{\adj{T} \vec{x'}}
            \end{equation*}
        \end{itemize}

        Veamos ahora que es continua, en efecto, por Cauchy-Schwartz se tiene que para todo $\vec{x'}\in H'\backslash\left\{\vec{0} \right\}$:
        \begin{equation*}
            \begin{split}
                \norm{\adj{T}\vec{x'}}^2&=\pint{\adj{T}\vec{x'}}{\adj{T}\vec{x'}}\\
                &=\pint{T(\adj{T}\vec{x'})}{\vec{x'}}\\
                &\leq\abs{\pint{T(\adj{T}\vec{x'})}{\vec{x'}}}\\
                &\leq\norm{T(\adj{T}\vec{x'})}\norm{\vec{x'}} \\
                &\leq\norm{T}\norm{\adj{T}\vec{x'}}\norm{\vec{x'}} \\
            \end{split}
        \end{equation*}
        si $\vec{x'}\in\ker \adj{T}$ es claro que
        \begin{equation*}
            0=\norm{\adj{T}\vec{x'}}\leq\norm{T}\norm{\vec{x'}}
        \end{equation*}
        y, en caso de que no esté, por la ecuación anterior se sigue que:
        \begin{equation*}
            \Rightarrow \norm{\adj{T} \vec{x'}}\leq\norm{T}\norm{\vec{x'}}
        \end{equation*}
        En cuyo caso se sigue que $\adj{T}$ es continua y tal que $\norm{\adj{T}}\leq\norm{T}$. Para ver la igualdad se intercambian los papeles de $T$ y $\adj{T}$ en las desigualdades anteriores, con lo que se obtiene que $\norm{T}\leq\norm{\adj{T}}$.

        Y, para ver que $\adj{\adj{T}}$, notemos que para todo $\vec{x}\in H$ y $\vec{x'}\in H'$
        \begin{equation*}
            \pint{\adj{\adj{T}}\vec{x}}{\vec{x'}}=\pint{\vec{x}}{\adj{T}\vec{x'}}=\pint{T\vec{x}}{\vec{x'}}
        \end{equation*}
        por ende
        \begin{equation*}
            \pint{\adj{\adj{T}}\vec{x}}{\vec{x'}}=\pint{T\vec{x}}{\vec{x'}}
        \end{equation*}
        pero, por unicidad de la adjunta debe suceder que $\adj{\adj{T}}=T$.

        De (ii): Probaremos las dos igualdades.
        \begin{enumerate}
            \item $\adj{T_1+T_2}=\adj{T_1}+\adj{T_2}$. Tenemos que:
            \begin{equation*}
                \pint{\vec{x}}{\adj{T_1}\vec{x'}}=\pint{T_1\vec{x}}{\vec{x'}}\quad\textup{y}\quad\pint{\vec{x}}{\adj{T_2}\vec{x'}}=\pint{T_2\vec{x}}{\vec{x'}},\quad\forall\vec{x}\in H,\vec{x'}\in H'
            \end{equation*}
            por tanto
            \begin{equation*}
                \begin{split}
                    \pint{\vec{x}}{\adj{T_1}\vec{x'}}+\pint{\vec{x}}{\adj{T_2}\vec{x'}}=&\pint{T_1\vec{x}}{\vec{x'}}+\pint{T_2\vec{x}}{\vec{x'}}\quad\forall\vec{x}\in H,\vec{x'}\in H'\\
                    \Rightarrow\pint{\vec{x}}{\adj{T_1}\vec{x'}+\adj{T_2}\vec{x'}}=&\pint{T_1\vec{x}+T_2\vec{x}}{\vec{x'}}\quad\forall\vec{x}\in H,\vec{x'}\in H'\\
                    \Rightarrow \pint{\vec{x}}{(\adj{T_1}+\adj{T_2})\vec{x'}}=&\pint{(T_1+T_2)\vec{x}}{\vec{x'}}\quad\forall\vec{x}\in H,\vec{x'}\in H'\\
                \end{split}
            \end{equation*}
            de la unicidad de la adjunta, se sigue que $\adj{T_1+T_2}=\adj{T_1}+\adj{T_2}$.
            \item $\adj{\alpha T}=\conj{\alpha}\adj{T}$. Es similar al caso anterior.
        \end{enumerate}
        De los dos incisos anteriores se sigue el resultado.

        De (iii): Se tiene que:
        \begin{equation*}
            \pint{\vec{x}}{\adj{T}\vec{x'}}=\pint{T\vec{x}}{\vec{x'}}\quad\textup{y}\quad\pint{\vec{x'}}{\adj{U}\vec{x''}}=\pint{U\vec{x'}}{\vec{x''}},\quad\forall\vec{x}\in H,\vec{x'}\in H',\vec{x''}\in H''
        \end{equation*}
        debemos probar que:
        \begin{equation*}
            \pint{\vec{x}}{(\adj{T}\circ\adj{U})\vec{x''}}=\pint{(U\circ T)\vec{x}}{\vec{x''}}\quad\forall\vec{x}\in H,\vec{x''}\in H''
        \end{equation*}
        para usar la unicidad y de forma inmediata dedudcir el resultado.
        Sean $\vec{x}\in H$ y $\vec{x''}\in H''$. Como $\adj{U}\vec{x''}T\vec{x} \in H'$, tenemos que:
        \begin{equation*}
            \pint{\vec{x}}{\adj{T}(\adj{U}\vec{x''})}=\pint{T\vec{x}}{\adj{U}\vec{x''}}\quad\textup{y}\quad\pint{T\vec{x}}{\adj{U}\vec{x''}}=\pint{U(T\vec{x})}{\vec{x''}}
        \end{equation*}
        por tanto:
        \begin{equation*}
            \begin{split}
                \pint{\vec{x}}{\adj{T}(\adj{U}\vec{x''})}=&\pint{U(T\vec{x})}{\vec{x''}}\\
                \Rightarrow \pint{\vec{x}}{(\adj{T}\circ \adj{U})\vec{x''}}=&\pint{(U\circ T)\vec{x}}{\vec{x''}}\\
            \end{split}
        \end{equation*}
        lo cual prueba el resultado al ser los vectores arbitrarios.

    \end{proof}

    \begin{excer}
        Sea $H$ un espacio hilbertiano complejo. A toda aplicación lineal continua $T$ de $H$ en $H$ se le asocia la aplicación $\cf{Q_T}{H}{\mathbb{C}}$ (llamada \textbf{forma hermitiana}) definida por:
        \begin{equation*}
            Q_T(\vec{x})=\pint{T\vec{x}}{\vec{x}},\quad\forall\vec{x}\in H
        \end{equation*}
        Haga lo siguiente:
        \begin{enumerate}
            \item \textbf{Establezca} la fórmula:
            \begin{equation*}
                \pint{T\vec{x}}{\vec{y}}=\frac{1}{4}\left[Q_T(\vec{x}+\vec{y})-Q_T(\vec{x}-\vec{y})+iQ_T(\vec{x}+i\vec{y})-iQ_T(\vec{x}-i\vec{y})\right]
            \end{equation*}
            \item \textbf{Muestre} que
            \begin{equation*}
                Q_{\adj{T}}(\vec{x})=\conj{Q_T(\vec{x})},\quad\forall\vec{x}\in H
            \end{equation*}
            y que $Q_T(\vec{x})$ es real, $\forall\vec{x}\in H$, si y sólo si $T$ es autoadjunto (es decir, que $T=\adj{T}$).
        \end{enumerate}
    \end{excer}

    \begin{sol}
        Establezcamos ambos incisos:
        
        De (i): Sean $\vec{x},\vec{y}\in H$. Tenemos que:
        \begin{equation*}
            \begin{split}
                Q_T(\vec{x}+\vec{y})=&\pint{T(\vec{x}+\vec{y})}{\vec{x}+\vec{y}}\\
                =&\pint{T(\vec{x})+T(\vec{y})}{\vec{x}+\vec{y}}\\
                =&\pint{T(\vec{x})+T(\vec{y})}{\vec{x}}+\pint{T(\vec{x})+T(\vec{y})}{\vec{y}}\\
                =&\pint{T(\vec{x})}{\vec{x}}+\pint{T(\vec{y})}{\vec{x}}+\pint{T(\vec{x})}{\vec{y}}+\pint{T(\vec{y})}{\vec{y}}\\
                =&Q_T(\vec{x})+\pint{T(\vec{y})}{\vec{x}}+\pint{T(\vec{x})}{\vec{y}}+Q_T(\vec{y})\\
            \end{split}
        \end{equation*}
        por lo cual,
        \begin{equation*}
            \begin{split}
                Q_T(\vec{x}-\vec{y})=&Q_T(\vec{x})+\pint{T(-\vec{y})}{\vec{x}}+\pint{T(\vec{x})}{-\vec{y}}+Q_T(-\vec{y})\\
                =&Q_T(\vec{x})-\pint{T(\vec{y})}{\vec{x}}-\pint{T(\vec{x})}{\vec{y}}+Q_T(\vec{y})\\
            \end{split}
        \end{equation*}
        Luego:
        \begin{equation*}
            \begin{split}
                Q_T(\vec{x}+\vec{y})-Q_T(\vec{x}-\vec{y})&=2\left(\pint{T(\vec{y})}{\vec{x}}+\pint{T(\vec{x})}{\vec{y}}\right)\\
            \end{split}
        \end{equation*}
        y, por ende:
        \begin{equation*}
            \begin{split}
                iQ_T(\vec{x}+i\vec{y})-iQ_T(\vec{x}-i\vec{y})&=2i\left(\pint{T(i\vec{y})}{\vec{x}}+\pint{T(\vec{x})}{i\vec{y}}\right)\\
                &=2i\left(i\pint{T(\vec{y})}{\vec{x}}-i\pint{T(\vec{x})}{\vec{y}}\right)\\
                &=2\left(-\pint{T(\vec{y})}{\vec{x}}+\pint{T(\vec{x})}{\vec{y}}\right)\\
            \end{split}
        \end{equation*}
        Finalmente, se sigue que
        \begin{equation*}
            \begin{split}
                \frac{1}{4}\left[Q_T(\vec{x}+\vec{y})-Q_T(\vec{x}-\vec{y})+iQ_T(\vec{x}+i\vec{y})-iQ_T(\vec{x}-i\vec{y})\right]=&\frac{1}{4}\left[4\pint{T(\vec{x})}{\vec{y}}\right] \\
                =&\pint{T(\vec{x})}{\vec{y}}\\
            \end{split}
        \end{equation*}
        lo cual establece la fórmula.

        De (ii): Sea $\vec{x}\in H$, entonces:
        \begin{equation*}
            \begin{split}
                Q_{\adj{T}}(\vec{x})&=\pint{\adj{T}\vec{x}}{\vec{x}}\\
                &=\pint{\adj{T}\vec{x}}{\vec{x}}\\
                &=\conj{\pint{\vec{x}}{\adj{T}\vec{x}}}\\
                &=\conj{\pint{T\vec{x}}{\vec{x}}}\\
                &=\conj{Q_T(\vec{x})}\\
            \end{split}
        \end{equation*}

        Para la otra parte, veamos que:
        \begin{equation*}
            \begin{split}
                Q_T(\vec{x})\in\mathbb{R},\forall\vec{x}\in H\iff&Q_T(\vec{x})=\conj{Q_T(\vec{x})},\forall\vec{x}\in H\\
                \iff&Q_T(\vec{x})=Q_{\adj{T}}(\vec{x}),\forall\vec{x}\in H\\
                \iff&\pint{T\vec{x}}{\vec{x}}=\pint{\adj{T}\vec{x}}{\vec{x}},\forall\vec{x}\in H\\
                \iff&\pint{T\vec{x}}{\vec{x}}-\pint{\adj{T}\vec{x}}{\vec{x}}=0,\forall\vec{x}\in H\\
                \iff&\pint{T\vec{x}-\adj{T}\vec{x}}{\vec{x}}=0,\forall\vec{x}\in H\\
                \iff&\pint{\left[T-\adj{T}\right]\vec{x}}{\vec{x}}=0,\forall\vec{x}\in H\\
            \end{split}
        \end{equation*}
        Veamos que $\pint{\left[T-\adj{T}\right]\vec{x}}{\vec{x}}=0,\forall\vec{x}\in H$ si y sólo si $T=\adj{T}$.

        $\Rightarrow)$ Suponga que $\pint{\left[T-\adj{T}\right]\vec{x}}{\vec{x}}=0,\forall\vec{x}\in H$. Esto es inmediato, pues se tiene que: $Q_{T-\adj{T}}(\vec{x})=0$, para todo $\vec{x}\in H$, luego
        \begin{equation*}
            \pint{\left[T-\adj{T}\right]\vec{x}}{\vec{y}}=0,\quad\forall\vec{x},\vec{y}\in H
        \end{equation*}
        en particular para $\vec{x}$ fijo, $\pint{\left[T-\adj{T}\right]\vec{x}}{\vec{y}}=0$ para todo $\vec{y}\in H$, luego $\left[T-\adj{T}\right]\vec{x}=\vec{0}$. Como fue arbitrario se sigue entonces que $T=\adj{T}$.

        $\Leftarrow)$ Suponga que $T=\adj{T}$. De forma inmediata se sigue que $\pint{\left[T-\adj{T}\right]\vec{x}}{\vec{x}}=0,\forall\vec{x}\in H$.

    \end{sol}

    \begin{excer}
        Sea $A$ un endomorfismo lineal continuo de un espacio prehilbertiano $H$. Defina $\cf{Q_A}{H}{\mathbb{K}}$ como:
        \begin{equation*}
            Q_A(\vec{x})=\pint{A\vec{x}}{\vec{x}},\quad\forall\vec{x}\in H
        \end{equation*}
        Sea
        \begin{equation*}
            \alpha=\sup\left\{\frac{\abs{Q_A(\vec{x})}}{\norm{\vec{x}}^2}\big| \vec{x}\in H,\vec{x}\neq\vec{0} \right\}
        \end{equation*}
        \begin{enumerate}
            \item \textbf{Pruebe} que $\alpha\leq\norm{A}$.
            \item Al suponer $A$ autoadjunto, \textbf{demuestre} la igualdad opuesta. Luego, si $A$ es autoadjunto se tiene que
            \begin{equation*}
                \norm{A}=\sup\left\{\frac{\abs{Q_A(\vec{x})}}{\norm{\vec{x}}^2}\big| \vec{x}\in H,\vec{x}\neq\vec{0} \right\}
            \end{equation*}
        \end{enumerate}
        \textit{Indicación}. Compruebe que $\forall\vec{x}\in H$ y $\forall\lambda>0$,
        \begin{equation*}
            \pint{A\vec{x}}{A\vec{x}}=\frac{1}{4}\left(Q_A(\lambda\vec{x}+\lambda^{-1}A\vec{x})-Q_A(\lambda\vec{x}-\lambda^{-1}A\vec{x})\right)
        \end{equation*}
        de ahí obtenga que $\norm{A\vec{x}}^2\leq\frac{\alpha}{2}\left(\lambda^2\norm{\vec{x}}^2+\frac{1}{\lambda^2}\norm{A\vec{x}}^2 \right)$ y elija $\lambda$ convenientemente.
    \end{excer}

    \begin{proof}
        Demostremos cada inciso.

        De (i): Basta con ver que $\norm{A}$ es cota superior del conjunto al que se le quiere sacar el supremo. Para ello, notemos que al ser $A$ lineal continuo, se tiene que:
        \begin{equation*}
            \begin{split}
                \abs{\pint{A\vec{x}}{\vec{x}}} \leq \norm{A\vec{x}}\norm{\vec{x}}\leq\norm{A}\norm{\vec{x}}^2
            \end{split}
        \end{equation*}
        para todo $\vec{x}\in H$. En particular, para $\vec{x}\neq\vec{0}$ se tiene que:
        \begin{equation*}
            \begin{split}
                \frac{\abs{\pint{A\vec{x}}{\vec{x}}}}{\norm{\vec{x}}^2}\leq&\norm{A}\\
                \Rightarrow \frac{\abs{Q_A(\vec{x})}}{\norm{\vec{x}}^2}\leq&\norm{A}\\
            \end{split}
        \end{equation*}
        luego, $\norm{A}$ es cota superior del conjunto. Por tanto $\alpha\leq\norm{A}$.

        De (ii): Suponga que $A$ es autoadjunto. Sean $\vec{x}\in H$ y $\lambda>0$. Entonces:
        \begin{equation*}
            \begin{split}
                Q_A(\lambda\vec{x}+\lambda^{-1}A\vec{x})=&\pint{A(\lambda\vec{x}+\lambda^{-1}A\vec{x})}{\lambda\vec{x}+\lambda^{-1}A\vec{x}} \\
                =&\pint{\lambda A\vec{x}+\lambda^{-1}(A\circ A)\vec{x}}{\lambda\vec{x}+\lambda^{-1}A\vec{x}} \\
                =&\pint{\lambda A\vec{x}+\lambda^{-1}(A\circ A)\vec{x}}{\lambda\vec{x}}+\pint{\lambda A\vec{x}+\lambda^{-1}(A\circ A)\vec{x}}{\lambda^{-1}A\vec{x}}\\
                =&\pint{\lambda A\vec{x}}{\lambda\vec{x}}+\pint{\lambda^{-1}(A\circ A)\vec{x}}{\lambda\vec{x}}+\pint{\lambda A\vec{x}}{\lambda^{-1}A\vec{x}}+\pint{\lambda^{-1}(A\circ A)\vec{x}}{\lambda^{-1}A\vec{x}}\\
            \end{split}
        \end{equation*}
        y
        \begin{equation*}
            \begin{split}
                Q_A(\lambda\vec{x}-\lambda^{-1}A\vec{x})=&\pint{A(\lambda\vec{x}-\lambda^{-1}A\vec{x})}{\lambda\vec{x}-\lambda^{-1}A\vec{x}} \\
                =&\pint{\lambda A\vec{x}-\lambda^{-1}(A\circ A)\vec{x}}{\lambda\vec{x}-\lambda^{-1}A\vec{x}} \\
                =&\pint{\lambda A\vec{x}-\lambda^{-1}(A\circ A)\vec{x}}{\lambda\vec{x}}-\pint{\lambda A\vec{x}-\lambda^{-1}(A\circ A)\vec{x}}{\lambda^{-1}A\vec{x}}\\
                =&\pint{\lambda A\vec{x}}{\lambda\vec{x}}-\pint{\lambda^{-1}(A\circ A)\vec{x}}{\lambda\vec{x}}-\pint{\lambda A\vec{x}}{\lambda^{-1}A\vec{x}}+\pint{\lambda^{-1}(A\circ A)\vec{x}}{\lambda^{-1}A\vec{x}}\\
            \end{split}
        \end{equation*}
        por tanto:
        \begin{equation*}
            \begin{split}
                Q_A(\lambda\vec{x}+\lambda^{-1}A\vec{x})-Q_A(\lambda\vec{x}-\lambda^{-1}A\vec{x})=&2(\pint{\lambda^{-1}(A\circ A)\vec{x}}{\lambda\vec{x}}+\pint{\lambda A\vec{x}}{\lambda^{-1}A\vec{x}})\\
                =&2(\pint{(A\circ A)\vec{x}}{\vec{x}}+\pint{A\vec{x}}{A\vec{x}})\\
                =&2(\pint{A\vec{x}}{A\vec{x}}+\pint{A\vec{x}}{A\vec{x}})\\
                =&4\pint{A\vec{x}}{A\vec{x}}\\
            \end{split}
        \end{equation*}
        pues, $A$ es autoadjunto. Luego:
        \begin{equation*}
            \pint{A\vec{x}}{A\vec{x}}=\frac{1}{4}(Q_A(\lambda\vec{x}+\lambda^{-1}A\vec{x})-Q_A(\lambda\vec{x}-\lambda^{-1}A\vec{x}))
        \end{equation*}
        por tanto:
        \begin{equation*}
            \begin{split}
                \norm{A\vec{x}}^2=&\frac{1}{4}\abs{Q_A(\lambda\vec{x}+\lambda^{-1}A\vec{x})-Q_A(\lambda\vec{x}-\lambda^{-1}A\vec{x})}\\
                \leq&\frac{1}{4}(\abs{Q_A(\lambda\vec{x}+\lambda^{-1}A\vec{x})}+\abs{Q_A(\lambda\vec{x}-\lambda^{-1}A\vec{x})})\\
                =&\frac{1}{4}(\frac{\norm{\lambda\vec{x}+\lambda^{-1}A\vec{x}}^2}{\norm{\lambda\vec{x}+\lambda^{-1}A\vec{x}}^2} \abs{Q_A(\lambda\vec{x}+\lambda^{-1}A\vec{x})}+\frac{\norm{\lambda\vec{x}-\lambda^{-1}A\vec{x}}^2}{\norm{\lambda\vec{x}-\lambda^{-1}A\vec{x}}^2}\abs{Q_A(\lambda\vec{x}-\lambda^{-1}A\vec{x})})\\
                \leq&\frac{1}{4}(\alpha\norm{\lambda\vec{x}+\lambda^{-1}A\vec{x}}^2+\alpha\norm{\lambda\vec{x}-\lambda^{-1}A\vec{x}}^2)\\
                =&\frac{\alpha}{4}(\pint{\lambda\vec{x}+\lambda^{-1}A\vec{x}}{\lambda\vec{x}+\lambda^{-1}A\vec{x}}+\pint{\lambda\vec{x}-\lambda^{-1}A\vec{x}}{\lambda\vec{x}-\lambda^{-1}A\vec{x}})\\
                =&\frac{\alpha}{4}(\pint{\lambda\vec{x}+\lambda^{-1}A\vec{x}}{\lambda\vec{x}}+\pint{\lambda\vec{x}+\lambda^{-1}A\vec{x}}{\lambda^{-1}A\vec{x}}+\pint{\lambda\vec{x}-\lambda^{-1}A\vec{x}}{\lambda\vec{x}}-\pint{\lambda\vec{x}-\lambda^{-1}A\vec{x}}{\lambda^{-1}A\vec{x}})\\
                =&\frac{\alpha}{4}(\lambda^2\pint{\vec{x}}{\vec{x}}+\pint{A\vec{x}}{\vec{x}}+\pint{\vec{x}}{A\vec{x}}+\lambda^{-2}\pint{A\vec{x}}{A\vec{x}}+\lambda^2\pint{\vec{x}}{\vec{x}}-\pint{A\vec{x}}{\vec{x}}-\pint{\vec{x}}{A\vec{x}}+\lambda^{-2}\pint{A\vec{x}}{A\vec{x}})\\
                =&\frac{\alpha}{2}(\lambda^2\pint{\vec{x}}{\vec{x}}+\lambda^{-2}\pint{A\vec{x}}{A\vec{x}})\\
                =&\frac{\alpha}{2}(\lambda^2\norm{\vec{x}}^2+\lambda^{-2}\norm{A\vec{x}}^2)\\
            \end{split}
        \end{equation*}
        por Cauchy-Schwartz y usando la definición de $\alpha$. Por tanto, si consideramos que $\alpha>0$:
        \begin{equation*}
            \begin{split}
                \norm{A\vec{x}}^2&\leq\frac{\alpha}{2}(\lambda^2\norm{\vec{x}}^2+\lambda^{-2}\norm{A\vec{x}}^2)\\
                \Rightarrow \left(1-\frac{\alpha}{2\lambda^2}\right)\norm{A\vec{x}}^2&\leq\frac{\alpha\lambda^2}{2}\norm{\vec{x}}^2\\
                \Rightarrow \frac{2\lambda^2-\alpha}{2\lambda^2}\norm{A\vec{x}}^2&\leq\frac{\alpha\lambda^2}{2}\norm{\vec{x}}^2\\
                \Rightarrow \norm{A\vec{x}}^2&\leq\frac{2\alpha\lambda^4}{2(2\lambda^2-\alpha)}\norm{\vec{x}}^2\\
                \Rightarrow \norm{A\vec{x}}^2&\leq\frac{\alpha\lambda^4}{2\lambda^2-\alpha}\norm{\vec{x}}^2\\
            \end{split}
        \end{equation*}
        tomemos $\lambda>0$ tal que:
        \begin{equation*}
            \begin{split}
                \alpha^2=\frac{\alpha\lambda^4}{2\lambda^2-\alpha}&\iff\alpha=\frac{\lambda^4}{2\lambda^2-\alpha}\\
                &\iff\alpha(2\lambda^2-\alpha)=\lambda^4\\
                &\iff0=\lambda^4-2\alpha\lambda^2+\alpha^2\\
                &\iff0=\left(\lambda^2-\alpha\right)^2 \\
                &\iff0=\left(\lambda^2-\alpha\right)^2 \\
                &\iff0=\lambda^2-\alpha \\
                &\iff\sqrt{\alpha}=\lambda \\
            \end{split}
        \end{equation*}
        de esta forma:
        \begin{equation*}
            \begin{split}
                \norm{A\vec{x}}^2&\leq\alpha^2\norm{\vec{x}}^2 \\
                \norm{A\vec{x}}&\leq\alpha\norm{\vec{x}}\\
            \end{split}
        \end{equation*}
        es decir que $\norm{A}\leq\alpha$ y por ende $\alpha=\norm{A}$, esto si $\alpha>0$. Si $\alpha=0$, entonces:
        \begin{equation*}
            \begin{split}
                \frac{\abs{Q_A(\vec{x})}}{\norm{\vec{x}}^2}=&0\quad\forall\vec{x}\in H\\
                \Rightarrow \abs{Q_A(\vec{x})}=&0\quad\forall\vec{x}\in H\\
                \Rightarrow \pint{A\vec{x}}{\vec{x}}=&0\quad\forall\vec{x}\in H\\
            \end{split}
        \end{equation*}
        pero, por (i) de 1.4 se sigue que $A=0$, pues $\pint{A\vec{x}}{\vec{y}}=0$ para todo $\vec{x},\vec{y}\in H\backslash\left\{\vec{0} \right\}$. En este caso $\alpha=0=\norm{A}$. En cualquier caso, se concluye que $\alpha=\norm{A}$.
    \end{proof}

    \begin{excer}
        \textbf{Muestre} que todo endomorfismo continuo $T$ de un espacio hilbertiano $H$ se expresa únicamente en la forma:
        \begin{equation*}
            T=A+iB
        \end{equation*}
        donde $A$ y $B$ son endomorfismos autoadjuntos de $H$.
    \end{excer}

    \begin{proof}
        Tomemos $A=\frac{1}{2}(T+\adj{T})$ y $B=\frac{1}{2i}(T-\adj{T})$, siendo $\cf{\adj{T}}{H}{H}$ la adjunta de $T$. Es claro que $T=A+iB$ y, que tanto $A$ como $B$ son adjuntos, pues:
        \begin{equation*}
            \begin{split}
                \adj{A}=&\adj{\frac{1}{2}(T+\adj{T})} \\
                =&\frac{1}{2}\adj{(T+\adj{T})} \\
                =&\frac{1}{2}(\adj{T}+\adj{\adj{T}}) \\
                =&\frac{1}{2}(T+\adj{T}) \\
                =&A\\
            \end{split}
        \end{equation*}
        y
        \begin{equation*}
            \begin{split}
                \adj{B}=&\adj{\frac{1}{2i}(T-\adj{T})} \\
                =&\adj{\frac{-i}{2}(T-\adj{T})} \\
                =&\conj{-\frac{i}{2}}\adj{(T-\adj{T})} \\
                =&\frac{i}{2}(\adj{T}-\adj{\adj{T}}) \\
                =&-\frac{1}{2i}(\adj{T}-T) \\
                =&\frac{1}{2i}(T-\adj{T}) \\
                =&B \\
            \end{split}
        \end{equation*}
        además, son endomorfismos. Para ello, basta ver que $T+\adj{T}$ y $T-\adj{T}$ lo son. En efecto, si $\vec{y}\in H$, entonces 
    \end{proof}

    \begin{excer}
        Sea $H$ un espacio prehilbertiano. \textbf{Construya} un espacio hilbertiano $\hat{H}$ y una inyección lineal $\cf{j}{H}{\hat{H}}$ tal que
        \begin{equation*}
            \pint{j\vec{x}}{j\vec{y}}=\pint{\vec{x}}{\vec{y}},\quad\forall\vec{x},\vec{y}\in H.
        \end{equation*}
        y que $j(H)$ sea denso en $\hat{H}$. El espacio hilbertiano $\hat{H}$ se llama la \textbf{completación} del espacio prehilibertiano $H$. \textbf{Formule y demuestre} un teorema de unicidad de esta completación.
    \end{excer}

    \begin{proof}
        Sea
        \begin{equation*}
            \hat{H}=\left\{\left\{\vec{x_n} \right\}_{n=1}^\infty\Big|\left\{\vec{x_n} \right\}_{n=1}^\infty\textup{ es sucesión de Cauchy en }H \right\}
        \end{equation*}
        se definen sobre $\hat{H}'$ dos operaciones, para todo $\hat{x'}=\left\{\vec{x_n} \right\}_{n=1}^\infty,\hat{y'}=\left\{\vec{y_n} \right\}_{n=1}^\infty\in \hat{H}'$ y $\alpha\in K$:
        \begin{equation*}
            \hat{x'}+\hat{y'}=\left\{\vec{x_n}+\vec{y_n} \right\}_{n=1 }^\infty\quad\textup{y}\quad\alpha\hat{x'}=\left\{\alpha\vec{x_n} \right\}_{n=1 }^\infty
        \end{equation*}
        Con estas operaciones $\hat{H}'$ es un espacio vectorial sobre $\mathbb{K}$. Definimos una relación $\sim$ en $\hat{H}'$ dada como sigue:
        \begin{equation*}
            \hat{x'}\sim\hat{y'}\iff\lim_{n\rightarrow\infty }\norm{\vec{x_n}-\vec{y_n}}=0
        \end{equation*}
        donde $\norm{\cdot}$ es la norma inducida por el producto interno sobre $H$. Tomemos $\hat{0'}=\left\{\vec{0} \right\}_{ n=1}^\infty\in\hat{H}'$, y sea:
        \begin{equation*}
            \hat{K}=\left\{\hat{x'}\in\hat{H}'\Big|\hat{x'}\sim\hat{0'} \right\}
        \end{equation*}
        Afirmamos que $\hat{K}$ es subespacio vectorial de $\hat{H}$. En efecto, sean $\hat{x'}=\left\{\vec{x_n} \right\}_{n=1}^\infty,\hat{y'}=\left\{\vec{y_n} \right\}_{n=1}^\infty\in \hat{K}'$ y $\alpha\in\mathbb{K}$, entonces:
        \begin{equation*}
            \begin{split}
                0\leq&\lim_{n\rightarrow\infty}\norm{\vec{x_n}+\alpha\vec{y_n}-\vec{0}}\\
                =&\lim_{n\rightarrow\infty}\norm{\vec{x_n}+\alpha\vec{y_n}}\\
                \leq&\lim_{n\rightarrow\infty}\norm{\vec{x_n}}+\lim_{n\rightarrow\infty}\norm{\alpha\vec{y_n}}\\
                =&\lim_{n\rightarrow\infty}\norm{\vec{x_n}}+\lim_{n\rightarrow\infty}\abs{\alpha}\cdot\norm{\vec{y_n}}\\
                =&\lim_{n\rightarrow\infty}\norm{\vec{x_n}-\vec{0}}+\abs{\alpha}\lim_{n\rightarrow\infty}\norm{\vec{y_n}-\vec{0}}\\
                =&0+\abs{\alpha}\cdot0\\
                =&0\\
            \end{split}
        \end{equation*}
        por tanto, $\hat{x'}+\alpha\hat{y'}\in\hat{K}'$. Así $\hat{K}'$ es espacio vectorial. Tomemos
        \begin{equation*}
            \hat{H}=\hat{H}'/\hat{K}'
        \end{equation*}
        el espacio vectorial cociente, cuyos elementos los denotaremos por $\hat{x}=[\hat{x'}]=\hat{x'}+\hat{K}'$. Definimos un producto escalar en $\hat{H}$ como sigue; para cada $\hat{x},\hat{y}\in H$:
        \begin{equation*}
            \pint{\hat{x}}{\hat{y}}=\lim_{n\rightarrow\infty }\pint{\vec{x_n}}{\vec{y_n}}
        \end{equation*}

    \end{proof}

    \begin{excer}
        Si $E$ es un espacio vectorial complejo, la adición de elementos de $E$ y la multiplicación de elementos de $E$ por números reales, hacen de $E$ un espacio vectorial real que se designa por $E_\mathbb{R}$.
        \begin{enumerate}
            \item Sea $H$ un espacio prehilbertiano complejo. Se designa por $\pint{\vec{x}}{\vec{y}}$ un producto escalar en $H$. \textbf{Muestre} que la aplicación:
            \begin{equation*}
                (\vec{x},\vec{y})\mapsto\pint{\vec{x}}{\vec{y}}_\mathbb{R}=\Re\pint{\vec{x}}{\vec{y}}
            \end{equation*}
            hace de $H_\mathbb{R}$ un espacio prehilbertiano real para el que se cumple:
            \begin{equation*}
                \pint{i\vec{x}}{i\vec{y}}_\mathbb{R}=\pint{\vec{x}}{\vec{y}}_\mathbb{R}
            \end{equation*}
            \textbf{Pruebe} la relación:
            \begin{equation}
                \pint{\vec{x}}{\vec{y}}=\pint{\vec{x}}{\vec{y}}_\mathbb{R}+i\pint{i\vec{x}}{i\vec{y}}_\mathbb{R}
            \end{equation}
            \item Sea $H$ un espacio vectorial complejo. Se supone que $H_\mathbb{R}$ está provisto de un producto escalar $(\vec{x},\vec{y})\mapsto\pint{\vec{x}}{\vec{y}}_\mathbb{R}$ que hace de $H_\mathbb{R}$ un espacio prehilbertiano real. Se supone también que $\pint{i\vec{x}}{i\vec{y}}_\mathbb{R}=\pint{\vec{x}}{\vec{y}}_\mathbb{R}$, para todo $\vec{x},\vec{y}\in H$. Se define en $H$ un producto $\pint{\vec{x}}{\vec{y}}$ por la fórmula (1.1). \textbf{Demuestre} que $(\vec{x},\vec{y})\mapsto\pint{\vec{x}}{\vec{y}}$ es un producto escalar complejo que hace de $H$ un espacio prehilbertiano complejo. 
        \end{enumerate}
    \end{excer}

    \begin{proof}
        
    \end{proof}

    \begin{excer}
        Haga lo sugiente:
        \begin{enumerate}
            \item \textbf{Muestre} que en todo espacio prehilbertiano real se cumple
            \begin{equation*}
                \pint{\vec{x}}{\vec{y}}=\frac{1}{4}\left(\norm{\vec{x}+\vec{y}}^2-\norm{\vec{x}-\vec{y}}^2 \right)
            \end{equation*}
            y en todo espacio prehilbertiano complejo se cumple
            \begin{equation*}
                \pint{\vec{x}}{\vec{y}}=\frac{1}{4}\left(\norm{\vec{x}+\vec{y}}^2-\norm{\vec{x}-\vec{y}}^2+i\norm{\vec{x}+i\vec{y}}^2-i\norm{\vec{x}-i\vec{y}}^2 \right)
            \end{equation*}
            \item Sea $E$ un espacio vectorial normado real en el que se verifica la identidad del paralelogramo:
            \begin{equation*}
                \norm{\vec{x}+\vec{y}}^2+\norm{\vec{x}-\vec{y}}^2=2(\norm{\vec{x}}^2+\norm{\vec{y}}^2)
            \end{equation*}
            \textbf{Pruebe} que se puede definir de manera única un producto escalar $\pint{ }{ }$ sobre $E$ que hace de $E$ un espacio prehilbertiano real para el cual $\norm{\vec{x}}^2=\pint{\vec{x}}{\vec{x}}$, $\forall\vec{x}\in E$.

            \textit{Indicación.} Defina $\pint{\vec{x}}{\vec{y}}$ por la primera fórmula del inciso (i). Usando la fórmula del paralelogramo compruebe que $\pint{\vec{x}}{2\vec{y}}=2\pint{\vec{x}}{\vec{y}}$. Transforme $\pint{\vec{x_1}}{\vec{y_1}}+\pint{\vec{x_2}}{\vec{y_2}}$ por la identidad del paralelogramo y deduzca la fórmula $\pint{\vec{x_1}}{\vec{y}}+\pint{\vec{x_2}}{\vec{y}}=\pint{\vec{x_1}+\vec{x_2}}{\vec{y}}$.

            \item Misma pregunta que en (ii) en el caso de ser $E$ espacio vectorial complejo. 
            
            \textit{Indicación.} Use (ii) y el problema 1.6.
        \end{enumerate}
    \end{excer}

    \begin{sol}
        
    \end{sol}

    \begin{excer}
        Para todo $s\in\mathbb{R}$ sea $\cf{u_s}{\mathbb{R}}{\mathbb{C}}$ la función definida por:
        \begin{equation*}
            u_s(x)=e^{isx},\quad\forall x\in\mathbb{R}.
        \end{equation*}
        Sea $X$ el espacio vectorial complejo compuesto de todas las combinaciones lineales finitas de estas funciones $u_s$, $\forall f,g\in X$ se define:
        \begin{equation*}
            \pint{f}{g}=\lim_{R\rightarrow\infty}\frac{1}{2R}\int_{-R}^Rf\conj{g}.
        \end{equation*}
        \textbf{Pruebe} que esta definición tiene sentido y que la aplicación $(f,g)\mapsto\pint{f}{g}$ es un producto escalar que hace de $X$ un espacio prehilbertiano.

        Sea $H$ el espacio prehilbertiano, completación del espacio prehilbertiano $X$ (ver problema 1.5). \textbf{Muestre} que $H$ es un espacio hilbertiano no separable y que la familia $\left(u_s\right)_{s\in\mathbb{R}}$ es un sistema ortonormal maximal en $H$.
    \end{excer}

    \begin{proof}
        
    \end{proof}

    \begin{excer}
        Sea $H$ un espacio hilbertiano de dimensión infinita. \textbf{Demuestre} que existe una aplicación continua inyectiva $\gamma$ de $[0,1]$ en $H$ (un \textbf{camino simple} en $H$) tal que si $0\leq a\leq b\leq c\leq d\leq 1$, los vectores $\gamma(b)-\gamma(a)$ y $\gamma(d)-\gamma(c)$ son ortogonales.

        \textit{Indicación.} Tome $H=L_2([0,1],\mathbb{K})$ y considere funciones características de ciertos subconjuntos de $[0,1]$.
    \end{excer}

    \begin{proof}
        
    \end{proof}

    \begin{excer}
        Sea $\left\{\vec{x_\nu} \right\}_{\nu=1}^\infty$ una sucesión de elementos de un espacio hilbertiano $H$. La sucesión $\left\{\vec{x_\nu} \right\}_{\nu=1}^\infty$ se llama \textbf{martingala} (en el sentido amplio) si, $\forall\nu\in\mathbb{N}$, $\vec{x_\nu}$ es el vector de $\mathcal{L}(\vec{x_1},...,\vec{x_\nu})$ menos alejado de $\vec{x_{\nu+1}}$.
        \begin{enumerate}
            \item Sea $\left\{\vec{x_\nu} \right\}_{\nu=1}^\infty$ una martingala. Se definen:
            \begin{equation*}
                \vec{y_1}=\vec{x_1}\quad\textup{e}\quad\vec{y_\nu}=\vec{x_\nu}-\vec{x_{\nu-1}},\quad\forall\nu\geq2.
            \end{equation*}
            \textbf{Muestre} que los vectores $\vec{y_\nu}$ son ortogonales a pares y que $\left\{\norm{\vec{x_\nu}} \right\}_{\nu=1}^\infty$ es una sucesión creciente de números no negativos.
            \item Sea $\left\{\vec{y_\nu} \right\}_{\nu=1}^\infty$ una sucesión de vectores en $H$ ortogonales a pares. Se define
            \begin{equation*}
                \vec{x_\nu}=\sum{k=1}^\nu\vec{y_k},\quad\forall\nu\in\mathbb{N}.
            \end{equation*}
            \textbf{Pruebe} que $\left\{\vec{y_\nu} \right\}_{\nu=1}^\infty$ es una martingala.
        \end{enumerate}
    \end{excer}

    \begin{proof}
        
    \end{proof}

    \begin{excer}
        Sea $\cf{f}{\mathbb{R}^n}{\mathbb{K}}$ una función medible, integrable en todo subconjunto de $\mathbb{R}^n$ de medida finita. Si
        \begin{equation*}
            \int f=0,\quad\forall\textup{ rectángulo acotado }P,
        \end{equation*}
        \textbf{demuestre} que $f=0$ c.t.p. en $\mathbb{R}^n$.
        
        \textit{Indicación.} Redúzcase a un corolario del lema de los promedios.
    \end{excer}

    \begin{proof}
        
    \end{proof}

    \begin{excer}[\textbf{Funciones de Hermite}]
        Por inducción se ve inmediatamente que
        \begin{equation*}
            D^ne^{-x^2}=(-1)^nH_n(x)e^{-x^2}, \quad n=0,1,2,...
        \end{equation*}
        donde $(-1)H_n$ es un polinomio de grado $n$. Estos polinomios $(-1)H_n$ se llaman \textbf{polinomios de Hermite}. Se definen las \textbf{funciones de Hermite} $\varphi_n$ por:
        \begin{equation*}
            \varphi_n(x)=H_n(x)e^{-\frac{x^2}{2}},\quad n=0,1,2,...
        \end{equation*}
        equivalentemente,
        \begin{equation*}
            \varphi_n(x)=(-1)^ne^{-\frac{x^2}{2}}D^ne^{-x^2},\quad n=0,1,2,...
        \end{equation*}
        \begin{enumerate}
            \item \textbf{Demuestre} que las funciones de Hermite satisfacen la relación:
            \begin{equation*}
                \varphi_n''(x)=(x^2-2n-1)\varphi_n(x),\quad\forall x\in\mathbb{R}.
            \end{equation*}
            \textit{Indicación.} Exprese a $\varphi_n''(x)$ mediante $D^ne^{-x^2}$, $D^{n+1}e^{-x^2}$ y $D^{n+2}e^{-x^2}$ y calcule $D^{n+2}e^{-x^2}=D^{n+1}(-2xe^{-x^2})$ por la fórmula de Leibniz para la derivada $n+1$-enésima de un producto de factores.

            \item \textbf{Muestre} que las funciones de Hermite consistituyen un sistema ortogonal en el espacio hilbertiano $L_2(\mathbb{R},\mathbb{K})$
            
            \textit{Indicación.} Del inciso (i) se sigue que $\varphi_n''\varphi_m-\varphi_m''\varphi_n=2(m-n)\varphi_n\varphi_m$.

            \item \textbf{Pruebe} la relación
            \begin{equation*}
                H_n'(x)=2nH_{n-1}(x).
            \end{equation*}
            \textit{Indicación.} Exprese $H_n'(x)$ mediante $D^ne^{-x^2}$ y $D^{n+1}e^{-x^2}$. Calcule $D^{n+1}e^{-x^2}=D^n(-2xe^{-x^2})$ por la fórmula de Leibniz.

            \item \textbf{Demuestre} que
            \begin{equation*}
                \int_{-\infty}^\infty\varphi_n^2(x)dx=2n\int_{-\infty}^\infty\varphi_{n-1}^2(x)dx
            \end{equation*}
            y deduzca que
            \begin{equation*}
                \int_{-\infty}^\infty\varphi_n^2(x)dx=\pi^{1/2}2^nn!.
            \end{equation*}
            Luego el sistema de funciones:
            \begin{equation*}
                \Psi_n=\frac{1}{\pi^{1/2}2^nn!}\varphi_n
            \end{equation*}
            es un sistema ortonormal en $L_2(\mathbb{R},\mathbb{K})$ (En un ejercicio posterior se probará que dicho sistema ortonormal es, de hecho, maximal).

            \textit{Indicación.} Integre por partes
            \begin{equation*}
                \int_{-\infty}^\infty\varphi_n^2(x)dx=(-1)^n\int_{-\infty}^\infty H_n(x)D^ne^{-x^2}dx
            \end{equation*}
            y use (iii).
        \end{enumerate}
    \end{excer}

\end{document} 