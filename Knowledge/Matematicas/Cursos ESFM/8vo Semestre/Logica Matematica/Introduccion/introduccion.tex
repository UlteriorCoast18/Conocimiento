\documentclass[12pt]{report}
\usepackage[spanish]{babel}
\usepackage[utf8]{inputenc}
\usepackage{amsmath}
\usepackage{amssymb}
\usepackage{amsthm}
\usepackage{graphics}
\usepackage{subfigure}
\usepackage{lipsum}
\usepackage{array}
\usepackage{multicol}
\usepackage{enumerate}
\usepackage[framemethod=TikZ]{mdframed}
\usepackage[a4paper, margin = 1.5cm]{geometry}

%En esta parte se hacen redefiniciones de algunos comandos para que resulte agradable el verlos%

\renewcommand{\theenumii}{\roman{enumii}}

\def\proof{\paragraph{Demostración:\\}}
\def\endproof{\hfill$\blacksquare$}

\def\sol{\paragraph{Solución:\\}}
\def\endsol{\hfill$\square$}

%En esta parte se definen los comandos a usar dentro del documento para enlistar%

\newtheoremstyle{largebreak}
  {}% use the default space above
  {}% use the default space below
  {\normalfont}% body font
  {}% indent (0pt)
  {\bfseries}% header font
  {}% punctuation
  {\newline}% break after header
  {}% header spec

\theoremstyle{largebreak}

\newmdtheoremenv[
    leftmargin=0em,
    rightmargin=0em,
    innertopmargin=-2pt,
    innerbottommargin=8pt,
    hidealllines = true,
    roundcorner = 5pt,
    backgroundcolor = gray!60!red!30
]{exa}{Ejemplo}[section]

\newmdtheoremenv[
    leftmargin=0em,
    rightmargin=0em,
    innertopmargin=-2pt,
    innerbottommargin=8pt,
    hidealllines = true,
    roundcorner = 5pt,
    backgroundcolor = gray!50!blue!30
]{obs}{Observación}[section]

\newmdtheoremenv[
    leftmargin=0em,
    rightmargin=0em,
    innertopmargin=-2pt,
    innerbottommargin=8pt,
    rightline = false,
    leftline = false
]{theor}{Teorema}[section]

\newmdtheoremenv[
    leftmargin=0em,
    rightmargin=0em,
    innertopmargin=-2pt,
    innerbottommargin=8pt,
    rightline = false,
    leftline = false
]{propo}{Proposición}[section]

\newmdtheoremenv[
    leftmargin=0em,
    rightmargin=0em,
    innertopmargin=-2pt,
    innerbottommargin=8pt,
    rightline = false,
    leftline = false
]{cor}{Corolario}[section]

\newmdtheoremenv[
    leftmargin=0em,
    rightmargin=0em,
    innertopmargin=-2pt,
    innerbottommargin=8pt,
    rightline = false,
    leftline = false
]{lema}{Lema}[section]

\newmdtheoremenv[
    leftmargin=0em,
    rightmargin=0em,
    innertopmargin=-2pt,
    innerbottommargin=8pt,
    roundcorner=5pt,
    backgroundcolor = gray!30,
    hidealllines = true
]{mydef}{Definición}[section]

\newmdtheoremenv[
    leftmargin=0em,
    rightmargin=0em,
    innertopmargin=-2pt,
    innerbottommargin=8pt,
    roundcorner=5pt
]{excer}{Ejercicio}[section]

%En esta parte se colocan comandos que definen la forma en la que se van a escribir ciertas funciones%

\newcommand\abs[1]{\ensuremath{\biglvert#1\bigrvert}}
\newcommand\divides{\ensuremath{\bigm|}}
\newcommand\cf[3]{\ensuremath{#1:#2\rightarrow#3}}

%recuerda usar \clearpage para hacer un salto de página

\begin{document}
    \title{Curso de Lógica Matemática}
    \author{Cristo Daniel Alvarado}
    \maketitle

    \tableofcontents %Con este comando se genera el índice general del libro%

    \setcounter{chapter}{-1} %En esta parte lo que se hace es cambiar la enumeración del capítulo%
    
    \chapter{Introducción}
    
    \section{Temario}
    

    Los siguientes temas se verán a lo largo del curso:

    \begin{enumerate}
        \item Lógica (Teoría de Modelos).
        \begin{enumerate}
            \item Lógica proposicional.
            \item Lógica de primer orden.
        \end{enumerate}
        \item Teoría de la Computabilidad.
        \item Teoría de Conjuntos.
    \end{enumerate}

    Y la bibliografía para el curso es la siguiente:

    \begin{itemize}
        \item Enderton, 'Introducción matemática a la lógica'.
        \item  Enderton, 'Teoría de la computabilidad'.
        \item Copi, 'Lógica Simbólica' o 'Computability Theory'.
        \item Rebeca Weber 'Computability Theory'.
    \end{itemize}

    \section{Conectivas Lógicas}

    La disyunción ($\land$), conjunción ($\lor$), negación ($\neg$), implicación ($\Rightarrow$) y si y sólo si ($\iff$) son las conectivas lógicas usadas usualmente. 

    (Se habló un poco de una cosa llamada forma normal disyuntiva).
    
    A $\left\{\land, \lor, \neg \right\}$ se le conoce como un conjunto completo de conectivas lógicas. Nos podemos quedar simplemente con conjuntos completos de disyuntivas con solo dos elementos, a saber: $\left\{\land, \neg \right\}$ y $\left\{\lor, \neg \right\}$, ya que $P\lor Q$ es $\neg(\neg P\land \neg Q)$. (de forma similar a lo otro $P\land Q$ es $\neg(\neg P\lor \neg Q)$).

    También $\left\{\Rightarrow, \neg \right\}$ es otro conjunto completo de conectivas lógicas, ya que $P\land Q$ es $\neg(P\Rightarrow\neq Q)$.

    Y, $\left\{|\right\}$ es un conjunto completo, donde $|$ es llamado la \textbf{barra de Scheffel}, que tiene la siguiente tabla de verdad.

    \begin{center}
        \begin{tabular}{c c | c}
            \hline
            $P$ & $Q$ & $P|Q$ \\
            \hline
            V & V & F \\
            V & F & V \\
            F & V & V \\
            F & F & V \\
        \end{tabular}
    \end{center}
    con este, se tiene un conjunto completo de conectivas lógicas.

    Como muchas veces se usan conectivas de este tipo:
    \begin{equation*}
        (P\Rightarrow \neg Q)\Rightarrow((P\Rightarrow R)\land\neg(Q\Rightarrow S)\land T)
    \end{equation*}
    al ser muy largas, a veces es más conveniente escribirlas en forma Polaca. De esta forma, lo anterior quedaría de la siguiente manera:
    \begin{equation*}
        \Rightarrow\Rightarrow P\neg Q\land\land PR\neg\Rightarrow Q S T 
    \end{equation*}
    \newpage

    \begin{proof}
        Entorno de Prueba
    \end{proof}

    \begin{sol}
        Entorno de Solución
    \end{sol}

    \begin{theor}[Nombre]
        Teorema
    \end{theor}

    \begin{propo}[Nombre]
        Proposición
    \end{propo}

    \begin{cor}[Nombre]
        Corolario
    \end{cor}

    \begin{lema}[Nombre]
        Lema
    \end{lema}

    \begin{mydef}[Nombre]
        Definición
    \end{mydef}

    \begin{obs}[Nombre]
        Observación
    \end{obs}

    \begin{exa}[Nombre]
        Ejemplo
    \end{exa}

    \begin{excer}[Nombre]
        Ejercicio
    \end{excer}

\end{document}