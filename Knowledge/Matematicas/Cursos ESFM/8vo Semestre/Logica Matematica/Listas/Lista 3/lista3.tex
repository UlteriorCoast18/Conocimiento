\documentclass[12pt]{report}
\usepackage[spanish]{babel}
\usepackage[utf8]{inputenc}
\usepackage{amsmath}
\usepackage{amssymb}
\usepackage{amsthm}
\usepackage{graphics}
\usepackage{subfigure}
\usepackage{lipsum}
\usepackage{array}
\usepackage{multicol}
\usepackage{enumerate}
\usepackage[framemethod=TikZ]{mdframed}
\usepackage[a4paper, margin = 1.5cm]{geometry}
\usepackage{tikz}
\usepackage{pgffor}
\usepackage{ifthen}
\usepackage{enumitem}
\usepackage{listings}
\usepackage{hyperref}
\usepackage{xcolor}
\usepackage{mathdots}

%Gestión de Hipervínculos

\hypersetup{
    colorlinks=true,
    linkcolor=black,
    filecolor=magenta,      
    urlcolor=cyan
}

%Gestión de Código de Programación

\definecolor{listing-background}{HTML}{F7F7F7}
\definecolor{listing-rule}{HTML}{B3B2B3}
\definecolor{listing-numbers}{HTML}{B3B2B3}
\definecolor{listing-text-color}{HTML}{000000}
\definecolor{listing-keyword}{HTML}{435489}
\definecolor{listing-keyword-2}{HTML}{1284CA} % additional keywords
\definecolor{listing-keyword-3}{HTML}{9137CB} % additional keywords
\definecolor{listing-identifier}{HTML}{435489}
\definecolor{listing-string}{HTML}{00999A}
\definecolor{listing-comment}{HTML}{8E8E8E}

\lstdefinestyle{myStyle}{
    language         = java,
    alsolanguage     = scala,
    numbers          = left,
    xleftmargin      = 2.7em,
    framexleftmargin = 2.5em,
    backgroundcolor  = \color{gray!15},
    basicstyle       = \color{listing-text-color}\linespread{1.0}\ttfamily,
    breaklines       = true,
    frameshape       = {RYR}{Y}{Y}{RYR},
    rulecolor        = \color{black},
    tabsize          = 2,
    numberstyle      = \color{listing-numbers}\linespread{1.0}\small\ttfamily,
    aboveskip        = 1.0em,
    belowskip        = 0.1em,
    abovecaptionskip = 0em,
    belowcaptionskip = 1.0em,
    keywordstyle     = {\color{listing-keyword}\bfseries},
    keywordstyle     = {[2]\color{listing-keyword-2}\bfseries},
    keywordstyle     = {[3]\color{listing-keyword-3}\bfseries\itshape},
    sensitive        = true,
    identifierstyle  = \color{listing-identifier},
    commentstyle     = \color{listing-comment},
    stringstyle      = \color{listing-string},
    showstringspaces = false,
}

\lstset{style = myStyle}

%Gestión de Marca de Agua

\usetikzlibrary{shapes.multipart}

\newcounter{it}
\newcommand*\watermarktext[1]{\begin{tabular}{c}
    \setcounter{it}{1}%
    \whiledo{\theit<100}{%
    \foreach \col in {0,...,15}{#1\ \ } \\ \\ \\
    \stepcounter{it}%
    }
    \end{tabular}
    }

\AddToHook{shipout/foreground}{
    \begin{tikzpicture}[remember picture,overlay, every text node part/.style={align=center}]
        \node[rectangle,black,rotate=30,scale=2,opacity=0.04] at (current page.center) {\watermarktext{Cristo Daniel Alvarado ESFM\quad}};
  \end{tikzpicture}
}
%En esta parte se hacen redefiniciones de algunos comandos para que resulte agradable el verlos%

\def\proof{\paragraph{Demostración:\\}}
\def\endproof{\hfill$\blacksquare$}

\def\sol{\paragraph{Solución:\\}}
\def\endsol{\hfill$\square$}

%En esta parte se definen los comandos a usar dentro del documento para enlistar%

\newtheoremstyle{largebreak}
  {}% use the default space above
  {}% use the default space below
  {\normalfont}% body font
  {}% indent (0pt)
  {\bfseries}% header font
  {}% punctuation
  {\newline}% break after header
  {}% header spec

\theoremstyle{largebreak}

\newmdtheoremenv[
    leftmargin=0em,
    rightmargin=0em,
    innertopmargin=0pt,
    innerbottommargin=5pt,
    hidealllines = true,
    roundcorner = 5pt,
    backgroundcolor = gray!60!red!30
]{exa}{Ejemplo}[section]

\newmdtheoremenv[
    leftmargin=0em,
    rightmargin=0em,
    innertopmargin=0pt,
    innerbottommargin=5pt,
    hidealllines = true,
    roundcorner = 5pt,
    backgroundcolor = gray!50!blue!30
]{obs}{Observación}[section]

\newmdtheoremenv[
    leftmargin=0em,
    rightmargin=0em,
    innertopmargin=0pt,
    innerbottommargin=5pt,
    rightline = false,
    leftline = false
]{theor}{Teorema}[section]

\newmdtheoremenv[
    leftmargin=0em,
    rightmargin=0em,
    innertopmargin=0pt,
    innerbottommargin=5pt,
    rightline = false,
    leftline = false
]{propo}{Proposición}[section]

\newmdtheoremenv[
    leftmargin=0em,
    rightmargin=0em,
    innertopmargin=0pt,
    innerbottommargin=5pt,
    rightline = false,
    leftline = false
]{cor}{Corolario}[section]

\newmdtheoremenv[
    leftmargin=0em,
    rightmargin=0em,
    innertopmargin=0pt,
    innerbottommargin=5pt,
    rightline = false,
    leftline = false
]{lema}{Lema}[section]

\newmdtheoremenv[
    leftmargin=0em,
    rightmargin=0em,
    innertopmargin=0pt,
    innerbottommargin=5pt,
    roundcorner=5pt,
    backgroundcolor = gray!30,
    hidealllines = true
]{mydef}{Definición}[section]

\newmdtheoremenv[
    leftmargin=0em,
    rightmargin=0em,
    innertopmargin=0pt,
    innerbottommargin=5pt,
    roundcorner=5pt
]{excer}{Ejercicio}[section]

%En esta parte se colocan comandos que definen la forma en la que se van a escribir ciertas funciones%

\newcommand\abs[1]{\ensuremath{\left|#1\right|}}
\newcommand\divides{\ensuremath{\bigm|}}
\newcommand\cf[3]{\ensuremath{#1:#2\rightarrow#3}}
\newcommand\contradiction{\ensuremath{\#_c}}
\newcommand\natint[1]{\ensuremath{\left[\big|#1\big|\right]}}

\begin{document}
    \setlength{\parskip}{5pt} % Añade 5 puntos de espacio entre párrafos
    \setlength{\parindent}{12pt} % Pone la sangría como me gusta
    \title{Lista 3 de Problemas y Ejercicios
    
    Lógica Matemática}
    \author{Cristo Daniel Alvarado}
    \maketitle

    %\setcounter{chapter}{3} %En esta parte lo que se hace es cambiar la enumeración del capítulo%

    \newpage

    \setcounter{chapter}{3}

    \section{Ejercicios}
   
    \begin{excer}
        Demuestre que todo subconjunto cofinito de $\mathbb{N}$ (es decir, cuyo complemento es finito) es computable.
    \end{excer}

    \begin{proof}
        Sea $X\subseteq\mathbb{N}$ un conjunto cofinito, entonces su complemento $\mathbb{N}\setminus X$ es finito. Sea $N\in\mathbb{N}$, se tiene que la función característica $\chi_{\left\{N \right\}}$ es computable, pues tiene como algoritmo:
        \begin{lstlisting}
int chi_N(int n){
    if(n == N) return 1;
    else return 0;
}
        \end{lstlisting}
        por lo que el conjunto $\left\{N \right\}$ es computable, en particular el conjunto:
        \begin{equation*}
            \mathbb{N}\setminus X=\left\{n\in\mathbb{N}\Big|n\notin X \right\}
        \end{equation*}
        es computable (por ser finito) ya que es unión finita de conjuntos numerables, luego su complemento el cual es $X$ es computable.
    \end{proof}

    \begin{excer}
        Suponga que $X\subseteq\mathbb{N}$ es computable, y sea $\cf{f}{\mathbb{N}}{\mathbb{N}}$ una función total computable. Demuestre que:
        \begin{equation*}
            f^{-1}[X]=\left\{n\in\mathbb{N}\Big|f(n)\in X \right\}
        \end{equation*}
        es un conjunto computable.
    \end{excer}

    \begin{proof}
        Considere el siguiente algoritmo de la función característica de $f^{-1}[X]$:

        \begin{lstlisting}
int f_1_[X](int n){
    if(chi_X(f(n))) return 1;
    else return 0;
}
        \end{lstlisting}
        como $f$ es total computable, entonces $f(n)$ existe para todo $n$, luego al ser $X$ un conjunto computable, en una cantidad finita de tiempo se obtiene si $\chi_X$ evaluada en $f(n)$ es cero o uno, en cuyo caso se retorna cero o uno en el algoritmo definido anteriormente, el cual siempre retorna algo.
    \end{proof}

    \begin{excer}
        Defina la función $\cf{f}{\mathbb{N}}{\mathbb{N}}$ mediante:
        \begin{equation*}
            f(n)=\left\{
                \begin{array}{rl}
                    1 & \textup{ si la expansión decimal de $\pi$ contiene una sucesion de al menos $n$ digitos}\\ 
                     & \textup{consecutivos iguales a 7.}\\
                    0 & \textup{ en otro caso.}\\
                \end{array}
            \right.
        \end{equation*}
        Demuestre (sin usar ningún hecho especial sobre $\pi$) que la función $f$ es total computable.
    \end{excer}

    \begin{proof}
        Es inmediato del siguiente algoritmo:
        \begin{lstlisting}
int f(int n){
    recorrer digito por digito la expansion decimal de pi hasta encontrar un 7{
        int cont = 1;
        cont cuenta el numero de 7s despues del primer 7;
        if(n <= cont) return 1;
    }
}
        \end{lstlisting}
        que $f$ es computable. Si $f$ no fuese total computable (es decir, que $f$ no sea la función constante uno), entonces existiría al menos un $N\in\mathbb{N}$ tal que $f(N)$ no está bien definido, por la forma en que definimos el algoritmo de $f$, se tendría que $N+1,...$ tampoco estarían bien definidos. Sea $n_0$ el mínimo entero no negativo tal que $f(n_0)$ no está bien definido (es decir que el algoritmo anterior sigue funcionando). Construímos el algoritmo:
        \begin{lstlisting}
int f_2(int n){
    if(n < n_0) return f(n);
    else return 0;
}
        \end{lstlisting}
        esta es el algoritmo de la función $f$, mismo que es total.
    \end{proof}

    \begin{obs}
        ¿Puedo elegir tal $n_0$ en la demostración anterior?
    \end{obs}

    \begin{excer}
        Suponga que $\cf{g}{\mathbb{N}}{\mathbb{N}}$ es una función no-creciente, es decir que $g(n+1)\leq g(n)$ para todo $n\in\mathbb{N}$. Pruebe que $g$ debe ser total computable.
    \end{excer}

    \begin{proof}
        Primero veamos que $g$ es computable.
    \end{proof}

    \begin{excer}
        Demuestre que la función $\cf{f}{\mathbb{N}^3}{\mathbb{N}}$ dada por:
        \begin{equation*}
            f(x,y,z)=\left\{
                \begin{array}{lcr}
                    y & \textup{ si } & x=0.\\
                    z & \textup{ si } & x\neq0.\\
                \end{array}
            \right.
        \end{equation*}
        es total computable.
    \end{excer}

    \begin{proof}
        
    \end{proof}

    \begin{excer}
        Considere una retícula de calles que conste de $n$ calles que van de este a oeste, atravesadas por $m$ calles que van de norte a sur, de tal suerte que se genere un mapa rectangular con $mn$ intersecciones. Si un peatón se propone caminar (utilizando dichas calles) para llegar desde la esquina noreste hasta la suroeste, caminando únicamente hacia el este o hacia el sur, y cambiando de dirección únicamente en las esquinas, denote por $r(n,m)$ a la cantidad de posibles rutas que nuestro peatón puede tomar. Demuestre que la función $\cf{r}{\mathbb{N}^2}{\mathbb{N}}$ es total computable.
    \end{excer}

    \begin{proof}
        
    \end{proof}

    \begin{excer}
        Proporcione un ejemplo de una función no-total, $\cf{g}{\mathbb{N}^2}{\mathbb{N}}$, tal que la función $\cf{h}{\mathbb{N}}{\mathbb{N}}$ obtenida por medio de una búsqueda no acotada:
        \begin{equation*}
            h(x)=(\mu y)(g(x,y)=0)
        \end{equation*}
        sí es total.
    \end{excer}

    \begin{sol}
        
    \end{sol}

    \begin{mydef}
        El operador $\mu$ significa \textbf{el mínimo tal que}, en caso de que exista (y la función queda sin definir en caso de que no). El acto de invocar a $\mu$ se conoce como búsqueda no acotada.
    \end{mydef}

    \begin{excer}
        Demuestre que si $X\subseteq\mathbb{N}$ es el conjunto de números de Gödel de máquinas de Turing (es decir, $n\in X$ si y sólo si $\varphi(n,\cdot)$ es una máquina de Turing válida, en donde $\varphi$ es la máquina de Turing universal), entonces la función característica $\chi_X$ es total computable.
    \end{excer}

    \begin{proof}
        
    \end{proof}

    \begin{excer}
        Haga lo siguiente:
        \begin{enumerate}[label = \textit{(\alph*)}]
            \item Construya una función $h$ que \textit{eventualmente domine} a todas las funciones computables, es decir, que para toda función computable $f$ exista un $N\in\mathbb{N}$ tal que $f(n)\leq h(n)$ para todo $n\geq N$.
            
            \textit{Sugerencia}. Hay por lo menos tres maneras naturales de definir a $h$.
            \item ¿Es posible lograr en el inciso anterior que la función $h$ sea total computable?
        \end{enumerate}
    \end{excer}

    \begin{sol}
        
    \end{sol}

    \begin{excer}
        Dado un algoritmo $\mathcal{A}$ y un $t\in\mathbb{N}\cup\left\{0\right\}$, la \textit{foto instantánea de $\mathcal{A}$ en el tiempo $t$} es una compliación de toda la información que se encuentra en el algoritmo en el instante de tiempo $t$.

        Demuestre que, si un algoritmo eventualmente se detiene, entonces todas las fotos instantáneas previas a la foto instantánea terminal (es decir, aquella que corresponde al tiempo en el cual el algoritmo se detiene) deben de ser distintas dos a dos.
    \end{excer}

    \begin{proof}
        
    \end{proof}

    \begin{exa}
        En el ejercicio anterior, un ejemplo sería si concebimos a nuestro algoritmo como una máquina de Turing, entonces la foto instantánea en el tiempo $t$ es un conjunto que contiene al $t$-ésimo estado visitado, así como el contenido de la cinta (visto como una sucesión finita de símbolos del alfabeto correspondiente junto con el símbolo \textit{en blanco}) justo en el momento en que se visita ese $t$-ésimo estado, así como la posición exacta del cabezal lector/escritor en ese momento.

        Por otra parte, si concebimos a nuestro algoritmo como un programa en algún lenguaje de programación, entonces la foto instantánea en el tiempo $t$ consta de la información acerca de los valores que tienen las variables en el momento de correr la $t$-ésima instrucción, así como el reglón del programa que se está corriendo en ese momento.
    \end{exa}

    \begin{excer}
        Recuerde que un conjunto no vacío $X\subseteq\mathbb{N}$ es \textbf{computablemente enumerable} si y sólo si $X$ es el rango de alguna función total computable. Demuestre ahora que un conjunto no vacío $X\subseteq\mathbb{N}$ es computable si y sólo si $X=\textup{ran}(f)$ para alguna función $\cf{f}{\mathbb{N}}{X}$ total computable que es \textit{no decreciente}.
    \end{excer}

    \begin{proof}
        
    \end{proof}

    \begin{excer}
        Sea $A\subseteq\mathbb{N}$ un conjunto infinito, computablemente enumerable.
        \begin{enumerate}[label = \textit{(\alph*)}]
            \item Demuestre que existe una función total computable $\cf{g}{\mathbb{N}}{\mathbb{N}}$ que es estrictamente creciente tal que $\textup{ran}(g)\subseteq A$.
            \item Concluya que todo conjunto computablemente enumerable infinito contiene un subconjunto computable infinito.
        \end{enumerate}
    \end{excer}

    \begin{proof}
        
    \end{proof}

    \begin{excer}
        Decimos que un conjunto $X\subseteq\mathbb{N}^k$ es $\Sigma_1^0$ si existe algún conjunto computable $A\subseteq\mathbb{N}^m$ tal que:
        \begin{equation*}
            X=\left\{(x_1,...,x_k)\Big|\exists a_1,...,a_{ m-k}\textup{ tales que }(x_1,...,x_k,a_1,...,a_{m-k})\in A\right\}
        \end{equation*}
        Por otra parte, decimos que un conjunto $Y\subseteq\mathbb{N}^k$ es $\Pi_1^0$ si existe algún conjunto computable $B\subseteq\mathbb{N}^m$ tal que:
        \begin{equation*}
            Y=\left\{(y_1,...,y_k)\Big|\forall b_1,...,b_{ m-k}\textup{ tales que }(y_1,...,y_k,b_1,...,b_{m-k})\in B\right\}
        \end{equation*}
        Demuestre que todo conjunto que es al mismo tiempo $\Sigma_1^0$ y $\Pi_1^0$ es computable.
    \end{excer}

    \begin{proof}
        
    \end{proof}

    \begin{excer}
        Dado un conjunto $X\subseteq\mathbb{N}^k$, demuestre que las siguientes condiciones son equivalentes:
        \begin{enumerate}[label = \textit{(\alph*)}]
            \item $X$ es computablement enumerable.
            \item O bien $X=\emptyset$, o bien $X$ es el rango de alguna función parcial computable.
            \item Existe una sucesión computable de conjuntos finitos $Y_s\subseteq\mathbb{N}^k$ (lo cual realmente significa: existe un conjunto computable $Y\subseteq\mathbb{N}^{ k+1}$ tal que, para cada $s\in\mathbb{N}$
            \begin{equation*}
                Y_s=\left\{(x_1,...,x_k)\Big|(x_1,...,x_k,s)\in Y \right\}
            \end{equation*}
            ) que satisfacen:
            \begin{equation*}
                Y_s\subseteq Y_{ s+1},\quad\forall s\in\mathbb{N}
            \end{equation*}
            de tal suerte que $X=\bigcup_{ s\in\mathbb{N}}Y_s$
        \end{enumerate}
    \end{excer}

    \begin{proof}
        
    \end{proof}

    \begin{excer}
        Demuestre que todo conjunto infinito computablemente enumerable es el rango de alguna función total computable $\cf{f}{\mathbb{N}}{\mathbb{N}}$ que es inyectiva.
    \end{excer}

    \begin{proof}
        
    \end{proof}
        
    \begin{excer}
        Demuestre que si $X\subseteq\mathbb{N}$ es computable y $Y\subseteq\mathbb{N}$ es computable enumerable, entonces $Y$ es computable si y sólo si $X\setminus Y$ es computable.
    \end{excer}

    \begin{proof}
        
    \end{proof}

    \begin{excer}
        Demuestre la \textit{propiedad de reducción}: para cualesquiera dos conjuntos computablemente enumerables $X$ y $Y$ existen conjuntos computablemente enumerables $A\subseteq X$ y $B\subseteq Y$ tales que $A\cap B=\emptyset$ y $A\cup B=X\cup Y$. 
    \end{excer}

    \begin{proof}
        
    \end{proof}

    \begin{excer}
        Demuestre que todo conjunto infinito $X\subseteq\mathbb{N}$ contiene un subconjunto que no es computable.
    \end{excer}
    
    \begin{proof}
        
    \end{proof}

    \begin{excer}
        Dados dos conjuntos $X,Y\subseteq\mathbb{N}$, definimos la \textbf{yunta} de $X$ y $Y$ como:
        \begin{equation*}
            X\oplus Y=\left\{2n\Big|n\in X \right\}\cup\left\{2n+1\Big|n\in Y \right\}
        \end{equation*}
        \begin{enumerate}[label = \textit{(\alph*)}]
            \item Demuestre que, si $X$ y $Y$ son ambos computables, entonces $X\oplus Y$ también lo es.
            \item Demuestre que, si $X$ y $Y$ son ambos computablemente enumerables, entonces $X\oplus Y$ también lo es.
            \item Demuestre que, si $X\oplus Y$ es computable, entonces tanto $X$ como $Y$ también son computables.
            \item Demuestre que, si $X\oplus Y$ es computablemente enumerable, entonces tanto $X$ como $Y$ también son computablemente enumerables.
        \end{enumerate}
    \end{excer}

    \begin{proof}
        
    \end{proof}

    \begin{obs}
        En términos de grados de Turing, la yunta es importante porque representa al supremo de los grados de Turing de los conjuntos correspondientes.
    \end{obs}

    \begin{excer}
        Considere una codificación de los algoritmos de enumeración por medio de números naturales, y denote al conjunto impreso por el $n$-ésimo algoritmo como $W_n$ (en otras palabras, $W_n$ representa al $n$-ésimo conjunto computablemente enumerable, de acuerdo con alguna enumeración efectiva).
        \begin{enumerate}[label = \textit{(\alph*)}]
            \item Demuestre que, si $\cf{f}{\mathbb{N}}{\mathbb{N}}$ es una función total computable, entonces
            \begin{equation*}
                \bigcup_{ n=1}^\infty W_{f(n)}
            \end{equation*}
            es computablemente enumerable (en otras palabras, las uniones computables de conjuntos computablement enumerables son computablemente enumerables).
            \item ¿Es posible afirmar algo acerca de la computabilidad/computable enumerabilidad del conjunto $\bigcap_{ n=1}^\infty W_{f(n)}$?
            \item Demuestre que existe una función parcial computable $\cf{f}{;\mathbb{N}}{\mathbb{N}}$ tal que $f(n)$ está definida siempre que $W_n\neq\emptyset$, en cuyo caso $f(n)\in W_n$ (en otras palabras, se cumple el \textbf{axioma de elección computable}).
            \item Dada una relación binaria $R\subseteq\mathbb{N}^2$ computablemente enumerable, exhiba una función parcial computable $\cf{f}{;\mathbb{N}}{\mathbb{N}}$ tal que $f(n)$ esté definida siempre que $(n,m)\in R$ para algun $m\in\mathbb{N}$, en cuyo caso, además debe cumplirse que $(n,f(n))\in R$.
        \end{enumerate}
    \end{excer}

    \begin{proof}
        
    \end{proof}

    \begin{excer}
        Definamos los siguientes subconjuntos de $\mathbb{N}$:
        \begin{equation*}
            \begin{split}
                X&=\left\{n\in\mathbb{N}\Big|\varphi(n,n)=0 \right\},\\
                Y&=\left\{n\in\mathbb{N}\Big|\varphi(n,n)=1 \right\},\\
            \end{split}
        \end{equation*}
        (donde $\varphi$ reprsenta la máquiana de Turing universal).
        \begin{enumerate}[label = \textit{(\alph*)}]
            \item Demuestre que $X$ y $Y$ son conjuntos computablemente enumerables, amén de que $X\cap Y=\emptyset$.
            \item Demuestre que $X$ y $Y$ son \textbf{computablemente inseprables}: en otras palabras, no existe ningún conjunto computable $Z$ tal que $X\subseteq Z$ y $Y\cap Z=\emptyset$. 
            
            \textit{Sugerencia.} Diagonalización, es decir, si existiera tal $Z$, su función característica sería $\varphi(d,\cdot)$ para algún $d\in\mathbb{N}$; luego, ¿qué podemos decir de $\varphi(d,d)$?
        \end{enumerate}
    \end{excer}

    \begin{proof}
        
    \end{proof}

\end{document}
