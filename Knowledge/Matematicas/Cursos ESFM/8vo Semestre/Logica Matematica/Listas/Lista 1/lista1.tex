\documentclass[12pt]{report}
\usepackage[spanish]{babel}
\usepackage[utf8]{inputenc}
\usepackage{amsmath}
\usepackage{amssymb}
\usepackage{amsthm}
\usepackage{graphics}
\usepackage{subfigure}
\usepackage{lipsum}
\usepackage{array}
\usepackage{multicol}
\usepackage{enumerate}
\usepackage[framemethod=TikZ]{mdframed}
\usepackage[a4paper, margin = 1.5cm]{geometry}
\usepackage{tikz}
\usepackage{pgffor}
\usepackage{ifthen}

\usetikzlibrary{shapes.multipart}

\newcounter{it}
\newcommand*\watermarktext[1]{\begin{tabular}{c}
    \setcounter{it}{1}%
    \whiledo{\theit<100}{%
    \foreach \col in {0,...,15}{#1\ \ } \\ \\ \\
    \stepcounter{it}%
    }
    \end{tabular}
    }

\AddToHook{shipout/foreground}{
    \begin{tikzpicture}[remember picture,overlay, every text node part/.style={align=center}]
        \node[rectangle,black,rotate=30,scale=2,opacity=0.08] at (current page.center) {\watermarktext{Cristo Daniel Alvarado ESFM\quad}}; 
  \end{tikzpicture}
}
%En esta parte se hacen redefiniciones de algunos comandos para que resulte agradable el verlos%

\renewcommand{\theenumii}{\roman{enumii}}

\def\proof{\paragraph{Demostración:\\}}
\def\endproof{\hfill$\blacksquare$}

\def\sol{\paragraph{Solución:\\}}
\def\endsol{\hfill$\square$}

%En esta parte se definen los comandos a usar dentro del documento para enlistar%

\newtheoremstyle{largebreak}
  {}% use the default space above
  {}% use the default space below
  {\normalfont}% body font
  {}% indent (0pt)
  {\bfseries}% header font
  {}% punctuation
  {\newline}% break after header
  {}% header spec

\theoremstyle{largebreak}

\newmdtheoremenv[
    leftmargin=0em,
    rightmargin=0em,
    innertopmargin=-2pt,
    innerbottommargin=8pt,
    hidealllines = true,
    roundcorner = 5pt,
    backgroundcolor = gray!60!red!30
]{exa}{Ejemplo}[section]

\newmdtheoremenv[
    leftmargin=0em,
    rightmargin=0em,
    innertopmargin=-2pt,
    innerbottommargin=8pt,
    hidealllines = true,
    roundcorner = 5pt,
    backgroundcolor = gray!50!blue!30
]{obs}{Observación}[section]

\newmdtheoremenv[
    leftmargin=0em,
    rightmargin=0em,
    innertopmargin=-2pt,
    innerbottommargin=8pt,
    rightline = false,
    leftline = false
]{theor}{Teorema}[section]

\newmdtheoremenv[
    leftmargin=0em,
    rightmargin=0em,
    innertopmargin=-2pt,
    innerbottommargin=8pt,
    rightline = false,
    leftline = false
]{propo}{Proposición}[section]

\newmdtheoremenv[
    leftmargin=0em,
    rightmargin=0em,
    innertopmargin=-2pt,
    innerbottommargin=8pt,
    rightline = false,
    leftline = false
]{cor}{Corolario}[section]

\newmdtheoremenv[
    leftmargin=0em,
    rightmargin=0em,
    innertopmargin=-2pt,
    innerbottommargin=8pt,
    rightline = false,
    leftline = false
]{lema}{Lema}[section]

\newmdtheoremenv[
    leftmargin=0em,
    rightmargin=0em,
    innertopmargin=-2pt,
    innerbottommargin=8pt,
    roundcorner=5pt,
    backgroundcolor = gray!30,
    hidealllines = true
]{mydef}{Definición}[section]

\newmdtheoremenv[
    leftmargin=0em,
    rightmargin=0em,
    innertopmargin=-2pt,
    innerbottommargin=8pt,
    roundcorner=5pt
]{excer}{Ejercicio}[section]

%En esta parte se colocan comandos que definen la forma en la que se van a escribir ciertas funciones%

\newcommand\abs[1]{\ensuremath{\left|#1\right|}}
\newcommand\divides{\ensuremath{\bigm|}}
\newcommand\cf[3]{\ensuremath{#1:#2\rightarrow#3}}
\newcommand\natint[1]{\ensuremath{\left[\!\left[ #1\right]\!\right]}}
\newcommand{\afa}{\:
    \begin{tikzpicture}
        \draw [line width = 0.17 mm, black] (0,0) -- (-0.115,0.29);
        \draw [line width = 0.17 mm, black] (0,0) -- (0.115,0.29);
        \draw [line width = 0.17 mm, black] (-0.12,0) arc (190:-10:0.12cm);
    \end{tikzpicture}
    \:
}
\newcommand{\pstable}[1]{\arabic{#1})\stepcounter{#1}}
\newcounter{tablec}

%Este símvolo es para casi todo salvo una cantidad finita

%recuerda usar \clearpage para hacer un salto de página

\begin{document}
    \setlength{\parskip}{5pt} % Añade 5 puntos de espacio entre párrafos
    \setlength{\parindent}{12pt} % Pone la sangría como me gusta
    \title{Lista 1 de Ejercicios Lógica Matemática: Lógica Proposicional}
    \author{Cristo Daniel Alvarado}
    \maketitle

    \setcounter{chapter}{1}

    \section{Ejercicios}

    \begin{mydef}
        Una \textbf{conectiva booleana $n$-aria} es una función $\cf{B}{\left\{T,F \right\}^n}{\left\{T,F \right\}}$.
    \end{mydef}

    \begin{obs}
        La idea de la función anterior es que se codifique una tabla de verdad.
    \end{obs}

    \begin{excer}
        Considere la conectiva booleana dada por:
        \begin{equation*}
            \begin{array}{lr}
                B(T,T,T)=F, & B(F,T,T)=F,\\
                B(T,T,F)=F, & B(F,T,F)=T,\\
                B(T,F,T)=F, & B(F,F,T)=T,\\
                B(T,F,F)=T, & B(F,F,F)=T,\\
            \end{array}
        \end{equation*}
        escriba una fórmula bien formada, utilizando el conjunto de conectivas $\left\{\neg,\land,\lor \right\}$ que realice esta función booleana.
    \end{excer}

    \begin{sol}
        Sea $\cf{B}{\left\{T,F \right\}^3}{\left\{T,F \right\}}$ dada por:
        \begin{equation*}
            \begin{split}
                B(p_1,p_2,p_3)&=(p_1\land\neg p_2\land\neg p_3)\lor(\neg p_1\land p_2\land\neg p_3)\lor(\neg p_1\land\neg p_2\land p_3)\lor(\neg p_1\land\neg p_2\land\neg p_3)
            \end{split}
        \end{equation*}
        se verifica rápidamente que ésta función $B$ satiface lo deseado.
    \end{sol}

    \begin{excer}
        Muestre que el conjunto de conectivas $\left\{\perp,\Rightarrow \right\}$ es completo (donde $\perp$ es la conectiva $0$-aria con valor constante $F$).
    \end{excer}

    \begin{proof}
        Basta con ver que si $\varphi$ y $\psi$ son fórmulas, entonces $\neg\varphi$ y $\varphi\Rightarrow\psi$ se pueden expresar con conectivas $\left\{\perp,\Rightarrow \right\}$.

        En efecto, ya se tiene la implicación. Veamos que:
        \begin{equation*}
            \neg\varphi \equiv \varphi\Rightarrow\perp
        \end{equation*}
        para un modelo $m$ se tiene que:
        \begin{center}
            \begin{tabular}{ c | c | c | c}
                $\varphi$ & $\perp$ & $\varphi\Rightarrow\perp$ & $\neg\varphi$\\
                \hline
                $T$ & $F$ & $F$ & $F$ \\
                $F$ & $F$ & $T$ & $T$ \\
            \end{tabular}
        \end{center}
        es decir, que en cualquier caso $\overline{m}(\neg\varphi)=\overline{m}(\perp\Rightarrow\varphi)$. Se sigue entonces la equivalencia. Como $\left\{\neg,\Rightarrow \right\}$ es un conjunto completo de conectivas, también lo debe ser pues $\left\{\perp,\Rightarrow \right\}$.
    \end{proof}

    \renewcommand{\theenumi}{\alph{enumi})}

    \begin{excer}
        Reescriba las siguientes fórmulas en notación polaca a notación usual:
        \begin{enumerate}
            \item $\neg\neg\Rightarrow\lor\land p_3p_8\neg p_{10}\neg\lor p_1p_5$.
            \item $\land\neg\Rightarrow p_3\lor p_4p_1\iff\lor\neg p_{10}\iff p_{ 15}p_{18}q$.
            \item $\land\Rightarrow p_3\land p_2p_1\neg\lor\land p_4p_5\neg p_{10}$.
        \end{enumerate}
    \end{excer}

    \begin{sol}
        Veamos que
        \begin{enumerate}
            \item $\neg\neg\Rightarrow\lor\land p_3p_8\neg p_{10}\neg\lor p_1p_5\equiv\neg\neg(((p_3\land p_8)\lor\neg p_{10})\Rightarrow\neg( p_1\lor p_5))$.
            \item $\land\neg\Rightarrow p_3\lor p_4p_1\iff\lor\neg p_{10}\iff p_{ 15}p_{18}q\equiv(\neg(p_3\Rightarrow(p_4\lor p_1)))\land((\neg p_{10} \lor (p_{15}\iff p_{18}))\iff q)$.
            \item $\land\Rightarrow p_3\land p_2p_1\neg\lor\land p_4p_5\neg p_{10}\equiv(p_3\Rightarrow(p_2\lor p_1))\land\neg((p_4\land p_5)\lor\neg p_10)$.
        \end{enumerate}
    \end{sol}

    \begin{excer}
        Demuestre que toda fórmula bien formada (en el formato de clase, es decir, en notación polaca) en la que no aparezca el símbolo $\neg$ debe tener longitud impar.
    \end{excer}

    \begin{proof}
        Procederemos por inducción del número de implicaciones $\Rightarrow$, digamos $n$, en la cadena de la fórmula $\varphi$.
        \begin{itemize}
            \item Si $n=0$, entonces $\varphi\equiv p_1$, siendo $p_1$ una variable. Luego la longitud de $\varphi$ es $1$ que es impar.
            \item Si $n=1$, entonces $\varphi\equiv\Rightarrow p_1p_2$, siendo $p_1$ y $p_2$ variables. Luego la longitud de $\varphi$ es $3$ que es impar.
            \item Suponga que existe un $n\in\mathbb{N}$ tal que para todo $k\in\natint{0,n}$ se cumple que toda FBF que no contenga a $\neg$ y con una cantidad de implicaciones $k$ tiene longitud impar.
            
            Sea $\varphi$ una fórmula bien formada que no contenga $\neg$ y que tiene $n+1$ implicaciones, es decir que es de la forma:
            \begin{equation*}
                \varphi \equiv\Rightarrow\psi_1\psi_2
            \end{equation*}
            donde $\psi_1,\psi_2$ son FBF. Como $\varphi$ tiene $n+1$ implicaciones, entonces debe suceder que $\psi_1$ y $\psi_2$ contengan entre $0$ y $n$ implicaciones. Por hipótesis de inducción, tanto $\psi_1$ como $\psi_2$ tienen longitud impar, luego $\varphi$ tiene longitud la suma de estos dos impares (que es un par) más $1$ (la primera implicación). Por tanto, $\varphi$ tiene longitud impar.
        \end{itemize}
        Por inducción se sigue el resultado. 
    \end{proof}

    \begin{excer}
        Sea $\varphi$ una fórmula bien formada. Sea $c$ la cantidad de veces que aparece el símbolo $\Rightarrow$ en la fórmula $\varphi$, y sea $s$ la cantidad de veces que aparecen variables en la fórmula $\varphi$ (en donde, si alguna variable aparece varias veces, se cuentan cada una de sus apariciones por separado). Demuestre que
        \begin{equation*}
            s=c+1
        \end{equation*}
    \end{excer}

    \begin{proof}
        Procederemos por inducción sobre $c$.
        \begin{itemize}
            \item Para $c=0$, se tiene que $\varphi$ solo está conformada por variables y por aplicaciones sucesivas de la operación unaria $\neg$, por lo que solamente puede tener una variable. Así que $s=1$. Se sigue entonces que:
            \begin{equation*}
                s=c+1
            \end{equation*}
            \item Para $c=1$, se tiene que $\varphi$ es de la forma $\Rightarrow\psi\chi$, donde $\psi$ y $\chi$ son subfórmulas bien formadas de $\varphi$ que no contienen implicaciones, luego por la parte anterior $\psi$ y $\chi$ contienen una variable, es decir que $\varphi$ contiene dos variables. Luego $s=2$. Así que:
            \begin{equation*}
                s=c+1
            \end{equation*}
            \item Suponga que existe $n\in\mathbb{N}$ tal que para todo $k\leq n$ con $k\in\mathbb{N}\cup\left\{0\right\}$ se tiene que toda fórmula bien formada en la que aparecen $k$ veces el símbolo $\Rightarrow$, se tiene que
            \begin{equation*}
                s=k+1
            \end{equation*}
            siendo $s$ el número de variables de la fórmula.

            Suponga que $\varphi$ es una fórmula bien formada en la que el símbolo $\Rightarrow$ aparece $n+1$ veces, esto es que $c=n+1$. Entonces, $\varphi$ es de la forma:
            \begin{equation*}
                \Rightarrow\psi\chi
            \end{equation*}
            donde $\psi$ y $\chi$ son subfórmulas bien formadas de $\varphi$. Sean $c_1$ y $c_2$ el número de veces que aparece el símbolo $\Rightarrow$ en $\psi$ y $\chi$, respectivamente. Se tiene que $0\leq c_1,c_2\leq n$, luego por hipótesis indictiva se sigue que
            \begin{equation*}
                s_i=c_i+1,\quad\forall i=1,2
            \end{equation*}
            donde $s_1$ y $s_2$ es el número de variables que aparecen en $\psi$ y $\chi$, respectivamente. Así pues, el número de variables que aparecen en $\varphi$ es:
            \begin{equation*}
                s=s_1+s_2
            \end{equation*}
            y, el número de veces que aparece el símbolo $\Rightarrow$ es la suma de el número de veces que aparece en $\psi$ y $\chi$ más uno. Por lo cual:
            \begin{equation*}
                c=c_1+c_2+1
            \end{equation*}
            Así pues:
            \begin{equation*}
                \begin{split}
                    s&=s_1+s_2\\
                    &=c_1+1+c_2+1\\
                    &=(c_1+c_2+1)+1\\
                    &=c+1
                \end{split}
            \end{equation*}
        \end{itemize}
        Aplicando inducción se sigue el resultado.
    \end{proof}

    \begin{excer}
        Sea $\varphi$ una fórmula bien formada, y suponga que todos los símbolos de la variable que aparecen en $\varphi$ se encuentran entre $p_1,...,p_n$. Supóngase que $m,m'$ son dos modelos que satisfacen $m(p_i)=m'(p_i)$ para todo $i\in\natint{1,n}$. Demuestre que
        \begin{equation*}
            \overline{m}(\varphi)=\overline{m'}(\varphi)
        \end{equation*}
    \end{excer}

    \begin{proof}
        Procederemos por inducción sobre $\varphi$.
        \begin{itemize}
            \item Si $\varphi$ es una variable, digamos $p_i$ (con $i\in\natint{1,n}$), se tiene que:
            \begin{equation*}
                \begin{split}
                    \overline{m}(\varphi)&=m(p_i)\\
                    &=m'(p_i)\\
                    &=\overline{m'}(\varphi)\\
                \end{split}
            \end{equation*}
            \item Se verán dos casos:
            \begin{itemize}
                \item $\varphi$ es de la forma $\neg\psi$ siendo $\psi$ una fórmula bien formada. Se tiene que las variables de $\psi$ son las mismas que las variables de $\varphi$. Suponga que $\overline{m}(\psi)=\overline{m'}(\psi)$, entonces:
                \begin{equation*}
                    \begin{split}
                        \overline{m}(\varphi)=V & \textup{ si y sólo si }\overline{m}(\neg\psi)=V\\
                        &\textup{ si y sólo si }\overline{m}(\psi)=F\\
                        &\textup{ si y sólo si }\overline{m'}(\psi)=F\\
                        &\textup{ si y sólo si }\overline{m'}(\neg\psi)=V\\
                        &\textup{ si y sólo si }\overline{m'}(\varphi)=V\\
                    \end{split}
                \end{equation*}
                de forma análoga se deduce que $\overline{m}(\varphi)=F$ si y sólo si $\overline{m'}(\varphi)=F$. Así que:
                \begin{equation*}
                    \overline{m}(\varphi)=\overline{m'}(\varphi)
                \end{equation*}
                \item $\varphi$ es de la forma $\Rightarrow\psi\chi$ siendo $\psi$ y $\chi$ subfórmulas bien formadas de $\varphi$. Se tiene en el inciso anterior que $\psi$ y $\chi$ son tienen algunas de las variables $p_1,...,p_n$. Supongamos que $\overline{m}(\psi)=\overline{m'}(\psi)$ y $\overline{m}(\chi)=\overline{m'}(\chi)$. Entonces:
                \begin{equation*}
                    \begin{split}
                        \overline{m}(\varphi)=F&\textup{ si y sólo si }\overline{m}(\Rightarrow\psi\chi)=F\\
                        &\textup{ si y sólo si }\overline{m}(\psi)=F\textup{ y }\overline{m}(\chi)=V\\
                        &\textup{ si y sólo si }\overline{m'}(\psi)=F\textup{ y }\overline{m'}(\chi)=V\\
                        &\textup{ si y sólo si }\overline{m'}(\Rightarrow\psi\chi)=F\\
                        &\textup{ si y sólo si }\overline{m'}(\varphi)=F\\
                    \end{split}
                \end{equation*}
                se sigue entonces que $\overline{m}(\varphi)=\overline{m'}(\varphi)$.
            \end{itemize}
        \end{itemize}
        Por induccioń, se sigue que
        \begin{equation*}
            \overline{m}(\varphi)=\overline{m'}(\varphi)
        \end{equation*}
        %DUDA
    \end{proof}

    \begin{excer}
        Demuestre o refute, para un conjunto de fórmulas $\Sigma$, y $\varphi,\psi$ dos fórmulas:
        \begin{enumerate}
            \item Si o bien $\Sigma\vDash\varphi$, o bien $\Sigma\vDash\psi$, entonces $\Sigma\vDash\varphi\land\psi$.
            \item Si $\Sigma\vDash\varphi\land\psi$ entonces o bien $\Sigma\vDash\varphi$, o bien $\Sigma\vDash\psi$.
        \end{enumerate}
    \end{excer}
    
    \begin{sol}
        De (a): Suponga que $\Sigma=\left\{p_1 \right\}$. Entonces, $\Sigma\vDash p_1$. No puede suceder que $\Sigma\vDash\neg p_1$. En efecto, si $m$ es un modelo tal que $m\vDash\Sigma$, entonces:
        \begin{equation*}
            m(p_1)=V
        \end{equation*}
        por tanto:
        \begin{equation*}
            m(\neg p_1)=F
        \end{equation*}
        luego, $m\nvDash\neg p_1$. Por tanto, $\Sigma\nvDash\neg p_1$. Afirmamos que $\Sigma\nvDash p_1\land\neg p_1$. En efecto, si $m$ es un modelo que satisface $\Sigma$, entonces:
        \begin{equation*}
            \begin{split}
                \overline{m}(p_1\land\neg p_1)&=\overline{m}(\neg(p_1\Rightarrow \neg\neg p_1))\\
                &=\overline{m}(\neg(p_1\Rightarrow p_1))\\
            \end{split}
        \end{equation*}
        como $\overline{m}(p_1)=m(p_1)=V$, entonces $\overline{m}(p_1\Rightarrow p_1)=V$. Así, $\overline{m}(p_1\land\neg p_1)=F$. Así que $\Sigma\nvDash p_1\land\neg p_1$.

        De (b): Probaremos que $\Sigma\vDash\varphi\land\psi$ implica que $\Sigma\vDash\varphi$ y $\Sigma\vDash\psi$. En efecto, por el Teorema de Completud se tiene que
        \begin{equation*}
            \Sigma\vdash\chi\textup{ si y sólo si }\Sigma\vDash\chi
        \end{equation*}
        para toda fórmula $\chi$. Por tanto, $\Sigma\vdash\varphi\land\psi$. Entonces, $\Sigma\vdash\varphi$ y $\Sigma\vdash\psi$ (usando conjunción). Luego, $\Sigma\vDash\varphi$ y $\Sigma\vDash\psi$.
        %DUDA
    \end{sol}

    \begin{excer}[\textbf{Sustitución}]
        Suponga que tenemos una lista de fórmulas bien formadas $\varphi_1,...,\varphi_n,...$. Quisiéramos definir formalmente la opearción que dada una fórmula bien formada $\psi$, reemplaza cada aparición del símbolo de la variable $p_i$ con la fórmula $\varphi_i$, de modo que se obtiene una nueva fórmula bien formada $\psi^*$. Por ejemplo, si $\psi$ es $p_4\Rightarrow p_{ 32}$, entonces $\psi^*$ es $\varphi_4\Rightarrow\varphi_{ 2}$.
        \begin{enumerate}
            \item ¿Cómo definiría formalmente la operación $\psi\mapsto\psi^*$ por recursión?
            \item Sea $m$ cualquier modelo, y defina $m'$ como el modelo dado por $m'(p_i)=\overline{m}(\varphi_i)$ para todo $i\in\mathbb{N}$. Demuestre que $\overline{m'}(\psi)=\overline{m}(\psi^*)$, para cada fórmula bien formada $\psi$.
            \item Concluya que si $\psi$ es una tautología, entonces $\psi^*$ también lo es.
        \end{enumerate}
    \end{excer}

    \begin{proof}
        De (a): Sea $\cf{f}{\textup{FBF}}{\textup{FBF}}$ dada como sigue:
        \begin{itemize}
            \item Si $\psi$ es una variable, digamos $p_i$ con $i\in\mathbb{N}$, entonces $f(\psi)=\varphi_i$.
            \item Para las conectivas:
            \begin{itemize}
                \item Si $\psi$ es de la forma $\neg\chi$, entonces $f(\psi)=\neg f(\chi)$.
                \item Si $\psi$ es de la forma $\Rightarrow\chi\xi$, entonces $f(\psi)=\Rightarrow f(\chi)f(\xi)$.
            \end{itemize}
        \end{itemize}
        De tal forma, se define recursivamente el valor de $\psi$, ya que se va descomponiendo en sus subfórmulas en las cuales, cada variable $p_i$ es sustutuída por $\varphi_i$.

        De (b): %TODO
    \end{proof}

    \begin{excer}
        Sea $\Sigma$ un conjunto de fórmulas bien formadas. Definimos la operación $\mathcal{C}(\Sigma)$ mediante
        \begin{equation*}
            \mathcal{C}(\Sigma)=\Sigma\cup\left\{\varphi\Big|\neg\varphi\in\Sigma \right\}\cup\left\{\varphi\Big|\varphi\land\psi\in\Sigma\textup{ o }\psi\land\varphi\in\Sigma\textup{ para alguna FBF }\psi \right\}
        \end{equation*}
        Definimos también recursivamente, para cada conjunto de fórmulas bien formadas $\Sigma$ los conjuntos $\mathcal{C}^n(\Sigma)$ como sigue:
        \begin{equation*}
            \begin{split}
                \mathcal{C}^0(\Sigma)&=\Sigma\\
                \mathcal{C}^{n+1}(\Sigma)&=\mathcal{C}(\mathcal{C}^n(\Sigma)),\quad\forall n\in\mathbb{N}\cup\left\{0\right\}\\
            \end{split}
        \end{equation*}
        y más aún, se define
        \begin{equation*}
            \mathcal{C}^\infty(\Sigma)=\bigcup_{ n\in\mathbb{N}}\mathcal{C}^n(\Sigma)
        \end{equation*}
        Haga lo siguiente:
        \begin{enumerate}
            \item Considere $\Sigma=\left\{p_1\land\neg p_2,\neg(p_3\land(p_4\land p_5)) \right\}$. Calcule $\mathcal{C}(\Sigma)$ y $\mathcal{C}(\mathcal{C}(\Sigma))$.
            \item Si $\Sigma$ es como en el inciso $(a)$, ¿a qué es igual $\mathcal{C}^\infty(\Sigma)$?
            \item Ahora, sea
            \begin{equation*}
                \Sigma=\left\{p_n\land\cdots p_n\Big|n\in\mathbb{N} \right\}
            \end{equation*}
            ¿A qué es igual $\mathcal{C}^\infty(\Sigma)$?
            \item ¿Se te puede ocurrir de alguna manera intuitiva (verbal, corta) de describir a qué es igual $\mathcal{C}^\infty(\Sigma)$?
        \end{enumerate}
    \end{excer}
    
    \begin{sol}
        De (a): %TODO
    \end{sol}

    \begin{excer}
        Demuestre que existe una demostración formal de los siguientes argumentos (en su defecto, complete las demostraciones):
    \end{excer}

    \begin{sol}
        \begin{enumerate}
            \item
            \begin{center}
                \setcounter{tablec}{1}
                \begin{tabular}{l r l c l r}
                    & \pstable{tablec} & $A$ & $\Rightarrow$ & $(B\land\neg C)$ & Premisa \\
                    & \pstable{tablec} & $(B\lor C)$ & $\Rightarrow$ & $D$ & Premisa \\
                    & \pstable{tablec} & $A$ &  &  & Premisa \\
                    & \pstable{tablec} & $B$ & $\land$ & $\neg C$ & 1,3 M.P. \\
                    & \pstable{tablec} & $B$ &  &  & 4 Simp. \\
                    & \pstable{tablec} & $B$ & $\lor$ & $C$ & 5 Ad. \\
                    & \pstable{tablec} & $D$ &  &  & 2,6 M.P. \\
                    \hline
                    & & & $\therefore$ & $D$ & \\
                \end{tabular}
            \end{center}
            \item
            \begin{center}
                \setcounter{tablec}{1}
                \begin{tabular}{l r l c l r}
                    & \pstable{tablec} & $A$ & $\Rightarrow$ & $B$ & Premisa \\
                    & \pstable{tablec} & $A$ & $\lor$ & $(B\lor\neg C)$ & Premisa \\
                    & \pstable{tablec} & $\neg B$ &  &  & Premisa \\
                    & \pstable{tablec} & $\neg A$ &  &  & 1,2 M.T. \\
                    & \pstable{tablec} & $B$ & $\lor$ & $\neg C$ & 3,2 S.D. \\
                    & \pstable{tablec} & $\neg C$ &  &  & 5,3 S.D. \\
                    & \pstable{tablec} & $\neg C$ & $\land$ & $\neg B$ & 5,3 Conj. \\
                    \hline
                    & & & $\therefore$ & $\neg C\land\neg B$ & \\
                \end{tabular}
            \end{center}
            \item
            \begin{center}
                \setcounter{tablec}{1}
                \begin{tabular}{l r l c l r}
                    & \pstable{tablec} & $A$ & $\Rightarrow$ & $B$ & Premisa \\
                    & \pstable{tablec} & $B$ & $\Rightarrow$ & $C$ & Premisa \\
                    & \pstable{tablec} & $(A\Rightarrow C)$ & $\Rightarrow$ & $(B\Rightarrow D)$ & Premisa \\
                    & \pstable{tablec} & $(A\Rightarrow D)$ & $\Rightarrow$ & $E$ & Premisa \\
                    $|\longrightarrow$& \pstable{tablec} & $A$ &  &  & Sup. \\
                    $|$& \pstable{tablec} & $B$ &  &  & 1,5 M.P. \\
                    $|$& \pstable{tablec} & $C$ &  &  & 2,6 M.P. \\
                    \hline
                    & \pstable{tablec} & $A$ & $\Rightarrow$ & $C$ & 5-7 M.D. \\
                    & \pstable{tablec} & $B$ & $\Rightarrow$ & $D$ & 3,8 M.P. \\
                    $|\longrightarrow$& \pstable{tablec} & $A$ &  &  & Sup. \\
                    $|$& \pstable{tablec} & $B$ &  &  & 1,10 M.P. \\
                    $|$& \pstable{tablec} & $D$ &  &  & 9,11 M.P. \\
                    \hline
                    & \pstable{tablec} & $A$ & $\Rightarrow$ & $D$ & 10-12 M.D. \\
                    & \pstable{tablec} & $E$ &  &  & 4,14 M.P. \\
                    \hline
                    & & & $\therefore$ & $E$ & \\
                \end{tabular}
            \end{center}
            \item
            \begin{center}
                \setcounter{tablec}{1}
                \begin{tabular}{l r l c l r}
                    & \pstable{tablec} & $A$ & $\Rightarrow$ & $(B\land C)$ & Premisa \\
                    & \pstable{tablec} & $\neg A$ & $\Rightarrow$ & $((D\Rightarrow E)\land(F\Rightarrow H))$ & Premisa \\
                    & \pstable{tablec} & $(B\land C)$ & $\lor$ & $((\neg A\Rightarrow D)\land(\neg A\Rightarrow F))$ & Premisa \\
                    & \pstable{tablec} & $\neg(B\land C)$ & $\land$ & $\neg(H\land D)$ & Premisa \\
                    & \pstable{tablec} & $\neg(B$ & $\land$ & $C)$ & 4 Simp. \\
                    & \pstable{tablec} & $\neg A$ &  &  & 1,5 M.T. \\
                    & \pstable{tablec} & $(D\Rightarrow E)$ & $\land$ & $(F\Rightarrow H)$ & 2,6 M.P.\\
                    & \pstable{tablec} & $D$ & $\Rightarrow$ & $E$ & 7 Simp.\\
                    & \pstable{tablec} & $F$ & $\Rightarrow$ & $H$ & 7 Conm. y Simp. \\
                    & \pstable{tablec} & $(\neg A\Rightarrow D)$ & $\land$ & $(\neg A\Rightarrow F)$ & 3,5 S.D. \\
                    & \pstable{tablec} & $\neg A$ & $\Rightarrow$ & $D$ & 10 Simp. \\
                    & \pstable{tablec} & $\neg A$ & $\Rightarrow$ & $F$ & 10 Conm. y Simp. \\
                    & \pstable{tablec} & $D$ &  &  & 11,6 M.P. \\
                    & \pstable{tablec} & $F$ &  &  & 12,6 M.P. \\
                    & \pstable{tablec} & $E$ &  &  & 8,13 M.P. \\
                    & \pstable{tablec} & $H$ &  &  & 9,14 M.P. \\
                    & \pstable{tablec} & $E$ & $\land$ & $H$ & 15,16 Conj. \\
                    \hline
                    & & & $\therefore$ & $E\land H$ & \\
                \end{tabular}
            \end{center}
            \item
            \begin{center}
                \setcounter{tablec}{1}
                \begin{tabular}{l r l c l r}
                    & \pstable{tablec} & $(A\Rightarrow B)$ & $\land$ & $(C\Rightarrow D)$ & Premisa \\
                    & \pstable{tablec} & $(B\Rightarrow E)$ & $\land$ & $(D\Rightarrow F)$ & Premisa \\
                    & \pstable{tablec} & $(\neg A\Rightarrow E)$ & $\land$ & $(\neg B\Rightarrow D)$ & Premisa \\
                    & \pstable{tablec} & $\neg E$ &  &  & Premisa \\
                    & \pstable{tablec} & $A$ & $\Rightarrow$ & $B$ & 1 Simp. \\
                    & \pstable{tablec} & $C$ & $\Rightarrow$ & $D$ & 1 Conm. y Simp. \\
                    & \pstable{tablec} & $B$ & $\Rightarrow$ & $E$ & 2 Simp. \\
                    & \pstable{tablec} & $D$ & $\Rightarrow$ & $F$ & 2 Conm. y Simp. \\
                    & \pstable{tablec} & $\neg A$ & $\Rightarrow$ & $E$ & 3 Simp. \\
                    & \pstable{tablec} & $\neg B$ & $\Rightarrow$ & $D$ & 3 Conm. y Simp. \\
                    & \pstable{tablec} & $\neg B$ &  &  & 7,4 M.T. \\
                    & \pstable{tablec} & $\neg B$ & $\lor$ & $\neg C$ & 11 Ad. \\
                    & \pstable{tablec} & $\neg C$ & $\lor$ & $\neg B$ & 12 Conm. \\
                    \hline
                    & & & $\therefore$ & $\neg C\lor\neg B$ & \\
                \end{tabular}
            \end{center}
            \item
            \begin{center}
                \setcounter{tablec}{1}
                \begin{tabular}{l r l c l r}
                    & \pstable{tablec} & $A$ & $\Rightarrow$ & $(B\Rightarrow C)$ & Premisa \\
                    $|\longrightarrow$ & \pstable{tablec} & $B$ &  &  & Sup. \\
                    $||\longrightarrow$ & \pstable{tablec} & $A$ &  &  & Sup. \\
                    $||$ & \pstable{tablec} & $B$ & $\Rightarrow$ & $C$ & 1,3 M.P. \\
                    \hline
                    $|$ & \pstable{tablec} & $A$ & $\Rightarrow$ & $C$ & 3-4 M.D. \\
                    \hline
                     & \pstable{tablec} & $B$ & $\Rightarrow$ & $(A\Rightarrow C)$ & 2-5 M.D. \\
                    \hline
                    & & & $\therefore$ & $B\Rightarrow(A\Rightarrow C)$ & \\
                \end{tabular}
            \end{center}
            %TODO
            \item
            \begin{center}
                \setcounter{tablec}{1}
                \begin{tabular}{l r l c l r}
                    & \pstable{tablec} & $A$ & $\Rightarrow$ & $(B\land C)$ & Premisa \\
                    \hline
                    & & & $\therefore$ & $A\Rightarrow B$ & \\
                \end{tabular}
            \end{center}
            \item
            \begin{center}
                \setcounter{tablec}{1}
                \begin{tabular}{l r l c l r}
                    & \pstable{tablec} & $A$ & $\Rightarrow$ & $(B\land C)$ & Premisa \\
                    & \pstable{tablec} & $C$ & $\Rightarrow$ & $(D\land E)$ & Premisa \\
                    \hline
                    & & & $\therefore$ & $A\Rightarrow(B\land D)$ & \\
                \end{tabular}
            \end{center}
            \item
            \begin{center}
                \setcounter{tablec}{1}
                \begin{tabular}{l r l c l r}
                    & \pstable{tablec} & $A$ & $\Rightarrow$ & $B$ & Premisa \\
                    & \pstable{tablec} & $C$ & $\Rightarrow$ & $B$ & Premisa \\
                    \hline
                    & & & $\therefore$ & $(A\lor C)\Rightarrow B$ & \\
                \end{tabular}
            \end{center}
            \item
            \begin{center}
                \setcounter{tablec}{1}
                \begin{tabular}{l r l c l r}
                    & \pstable{tablec} & $((A\lor B)\Rightarrow C)$ & $\land$ & $(\neg D\Rightarrow(B\land\neg C))$ & Premisa \\
                    \hline
                    & & & $\therefore$ & $A\Rightarrow D$ & \\
                \end{tabular}
            \end{center}
            \item
            \begin{center}
                \setcounter{tablec}{1}
                \begin{tabular}{l r l c l r}
                    & \pstable{tablec} & $(A\lor B)$ & $\Rightarrow$ & $C$ & Premisa \\
                    & \pstable{tablec} & $D$ & $\Rightarrow$ & $(E\land F)$ & Premisa \\
                    \hline
                    & & & $\therefore$ & $(A\Rightarrow C)\land(D\Rightarrow F)$ & \\
                \end{tabular}
            \end{center}
            \item
            \begin{center}
                \setcounter{tablec}{1}
                \begin{tabular}{l r l c l r}
                    & \pstable{tablec} & $(A\Rightarrow B)$ & $\land$ & $(C\Rightarrow D)$ & Premisa \\
                    & \pstable{tablec} & $(B\lor D)$ & $\Rightarrow$ & $((E\Rightarrow(E\lor F))\Rightarrow A\land C)$ & Premisa \\
                    \hline
                    & & & $\therefore$ & $A\iff C$ & \\
                \end{tabular}
            \end{center}
            \item
            \begin{center}
                \setcounter{tablec}{1}
                \begin{tabular}{l r l c l r}
                    & \pstable{tablec} & $A$ & $\lor$ & $(B\Rightarrow C)$ & Premisa \\
                    & \pstable{tablec} & $(B\Rightarrow (B\land C))$ & $\Rightarrow$ & $(D\lor E)$ & Premisa \\
                    & \pstable{tablec} & $(D\Rightarrow A)$ & $\land$ & $(E\Rightarrow F)$ & Premisa \\
                    \hline
                    & & & $\therefore$ & $A\lor F$ & \\
                \end{tabular}
            \end{center}
            \item
            \begin{center}
                \setcounter{tablec}{1}
                \begin{tabular}{l r l c l r}
                    & \pstable{tablec} & $(A\Rightarrow(\neg B\land\neg C))$ & $\land$ & $(D\Rightarrow\neg(B\lor C))$ & Premisa \\
                    & \pstable{tablec} & $(\neg E\Rightarrow A)$ & $\land$ & $(\neg F\Rightarrow D)$ & Premisa \\
                    & \pstable{tablec} & $(E\Rightarrow B)$ & $\land$ & $(F\Rightarrow C)$ & Premisa \\
                    \hline
                    & & & $\therefore$ & $B\iff C$ & \\
                \end{tabular}
            \end{center}
            \setcounter{enumi}{15}
            \item
            \begin{center}
                \setcounter{tablec}{1}
                \begin{tabular}{l r l c l r}
                    & \pstable{tablec} & $(A\lor B)$ & $\Rightarrow$ & $(C\Rightarrow D)$ & Premisa \\
                    & \pstable{tablec} & $(C\Rightarrow(C\land D))$ & $\Rightarrow$ & $E$ & Premisa \\
                    & \pstable{tablec} & $E$ & $\Rightarrow$ & $((\neg F\lor\neg\neg F)\Rightarrow(A\land F))$ & Premisa \\
                    \hline
                    & & & $\therefore$ & $A\iff E$ & \\
                \end{tabular}
            \end{center}
            \item
            \begin{center}
                \setcounter{tablec}{1}
                \begin{tabular}{l r l c l r}
                    \hline
                    & & & $\therefore$ & $(A\Rightarrow(B\Rightarrow C))\Rightarrow((A\Rightarrow B)\Rightarrow(B\Rightarrow C))$ & \\
                \end{tabular}
            \end{center}
            \item
            \begin{center}
                \setcounter{tablec}{1}
                \begin{tabular}{l r l c l r}
                    \hline
                    & & & $\therefore$ & $(A\Rightarrow B)\Rightarrow ((A\land C)\Rightarrow(B\land C))$ & \\
                \end{tabular}
            \end{center}
            \item
            \begin{center}
                \setcounter{tablec}{1}
                \begin{tabular}{l r l c l r}
                    \hline
                    & & & $\therefore$ & $((A\Rightarrow B)\Rightarrow A)\Rightarrow A$ & \\
                \end{tabular}
            \end{center}
            \item
            \begin{center}
                \setcounter{tablec}{1}
                \begin{tabular}{l r l c l r}
                    & \pstable{tablec} & $A$ & $\lor$ & $(B\land C)$ & Premisa \\
                    & \pstable{tablec} & $A$ & $\Rightarrow$ & $C$ & Premisa \\
                    \hline
                    & & & $\therefore$ & $C$ & \\
                \end{tabular}
            \end{center}
            \item
            \begin{center}
                \setcounter{tablec}{1}
                \begin{tabular}{l r l c l r}
                    & \pstable{tablec} & $(A\lor B)$ & $\Rightarrow$ & $(C\Rightarrow D)$ & Premisa \\
                    & \pstable{tablec} & $(\neg D\lor E)$ & $\Rightarrow$ & $(A\land C)$ & Premisa \\
                    \hline
                    & & & $\therefore$ & $D$ & \\
                \end{tabular}
            \end{center}
            \item
            \begin{center}
                \setcounter{tablec}{1}
                \begin{tabular}{l r l c l r}
                    & \pstable{tablec} & $(A\lor B)$ & $\Rightarrow$ & $(C\land D)$ & Premisa \\
                    & \pstable{tablec} & $(C\lor E)$ & $\Rightarrow$ & $(\neg F\land H)$ & Premisa \\
                    & \pstable{tablec} & $(F\lor G)$ & $\Rightarrow$ & $(A\land I)$ & Premisa \\
                    \hline
                    & & & $\therefore$ & $\neg F$ & \\
                \end{tabular}
            \end{center}
            \item
            \begin{center}
                \setcounter{tablec}{1}
                \begin{tabular}{l r l c l r}
                    \hline
                    & & & $\therefore$ & $(A\Rightarrow B)\lor (B\Rightarrow C)$ & \\
                \end{tabular}
            \end{center}
            \item
            \begin{center}
                \setcounter{tablec}{1}
                \begin{tabular}{l r l c l r}
                    \hline
                    & & & $\therefore$ & $A\Rightarrow ((A\Rightarrow B)\Rightarrow B)$ & \\
                \end{tabular}
            \end{center}
            \item
            \begin{center}
                \begin{tabular}{l r l c l r}
                    \hline
                    & & & $\therefore$ & $(A\Rightarrow B)\Rightarrow((A\Rightarrow (B\Rightarrow C))\Rightarrow(A\Rightarrow C))$ & \\
                \end{tabular}
            \end{center}
            \item
            \begin{center}
                \setcounter{tablec}{1}
                \begin{tabular}{l r l c l r}
                    \hline
                    & & & $\therefore$ & $(A\land B)\Rightarrow B$ & \\
                \end{tabular}
            \end{center}
            \item
            \begin{center}
                \setcounter{tablec}{1}
                \begin{tabular}{l r l c l r}
                    \hline
                    & & & $\therefore$ & $A\Rightarrow(B\Rightarrow A\land B)$ & \\
                \end{tabular}
            \end{center}
        \end{enumerate}
    \end{sol}

\end{document}