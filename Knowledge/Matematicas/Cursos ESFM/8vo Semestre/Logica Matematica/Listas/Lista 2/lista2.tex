\documentclass[12pt]{report}
\usepackage[spanish]{babel}
\usepackage[utf8]{inputenc}
\usepackage{amsmath,bbm}
\usepackage{amssymb}
\usepackage{amsthm}
\usepackage{graphics}
\usepackage{subfigure}
\usepackage{lipsum}
\usepackage{array}
\usepackage{multicol}
\usepackage{enumerate}
\usepackage[framemethod=TikZ]{mdframed}
\usepackage[a4paper, margin = 1.5cm]{geometry}
\usepackage{enumitem}

%En esta parte se hacen redefiniciones de algunos comandos para que resulte agradable el verlos%

\renewcommand{\theenumii}{\roman{enumii}}

\def\proof{\paragraph{Demostración:\\}}
\def\endproof{\hfill$\blacksquare$}

\def\sol{\paragraph{Solución:\\}}
\def\endsol{\hfill$\square$}

%En esta parte se definen los comandos a usar dentro del documento para enlistar%

\newtheoremstyle{largebreak}
  {}% use the default space above
  {}% use the default space below
  {\normalfont}% body font
  {}% indent (0pt)
  {\bfseries}% header font
  {}% punctuation
  {\newline}% break after header
  {}% header spec

\theoremstyle{largebreak}

\newmdtheoremenv[
    leftmargin=0em,
    rightmargin=0em,
    innertopmargin=-2pt,
    innerbottommargin=8pt,
    hidealllines = true,
    roundcorner = 5pt,
    backgroundcolor = gray!60!red!30
]{exa}{Ejemplo}[section]

\newmdtheoremenv[
    leftmargin=0em,
    rightmargin=0em,
    innertopmargin=-2pt,
    innerbottommargin=8pt,
    hidealllines = true,
    roundcorner = 5pt,
    backgroundcolor = gray!50!blue!30
]{obs}{Observación}[section]

\newmdtheoremenv[
    leftmargin=0em,
    rightmargin=0em,
    innertopmargin=-2pt,
    innerbottommargin=8pt,
    rightline = false,
    leftline = false
]{theor}{Teorema}[section]

\newmdtheoremenv[
    leftmargin=0em,
    rightmargin=0em,
    innertopmargin=-2pt,
    innerbottommargin=8pt,
    rightline = false,
    leftline = false
]{propo}{Proposición}[section]

\newmdtheoremenv[
    leftmargin=0em,
    rightmargin=0em,
    innertopmargin=-2pt,
    innerbottommargin=8pt,
    rightline = false,
    leftline = false
]{cor}{Corolario}[section]

\newmdtheoremenv[
    leftmargin=0em,
    rightmargin=0em,
    innertopmargin=-2pt,
    innerbottommargin=8pt,
    rightline = false,
    leftline = false
]{lema}{Lema}[section]

\newmdtheoremenv[
    leftmargin=0em,
    rightmargin=0em,
    innertopmargin=-2pt,
    innerbottommargin=8pt,
    roundcorner=5pt,
    backgroundcolor = gray!30,
    hidealllines = true
]{mydef}{Definición}[section]

\newmdtheoremenv[
    leftmargin=0em,
    rightmargin=0em,
    innertopmargin=-2pt,
    innerbottommargin=8pt,
    roundcorner=5pt
]{excer}{Ejercicio}[section]

%En esta parte se colocan comandos que definen la forma en la que se van a escribir ciertas funciones%

\newcommand\abs[1]{\ensuremath{\left|#1\right|}}
\newcommand\divides{\ensuremath{\bigm|}}
\newcommand\cf[3]{\ensuremath{#1:#2\rightarrow#3}}
\newcommand\natint[1]{\ensuremath{\left[\!\left[ #1\right]\!\right]}}
\newcommand{\afa}{\:
    \begin{tikzpicture}
        \draw [line width = 0.17 mm, black] (0,0) -- (-0.115,0.29);
        \draw [line width = 0.17 mm, black] (0,0) -- (0.115,0.29);
        \draw [line width = 0.17 mm, black] (-0.12,0) arc (190:-10:0.12cm);
    \end{tikzpicture}
    \:
}
\newcommand{\bbm}[1]{\mathbbm{#1}}
\newcommand{\pstable}[1]{\arabic{#1})\stepcounter{#1}}
\newcounter{tablec}
\newcommand{\free}{\textup{Free}}
%Este símvolo es para casi todo salvo una cantidad finita

%recuerda usar \clearpage para hacer un salto de página

\begin{document}
    \setlength{\parskip}{5pt} % Añade 5 puntos de espacio entre párrafos
    \setlength{\parindent}{12pt} % Pone la sangría como me gusta
    \title{Lista 2 de Problemas y Ejercicios
    
    Lógica Matemática}
    \author{Cristo Daniel Alvarado}
    \maketitle

    %\setcounter{chapter}{3} %En esta parte lo que se hace es cambiar la enumeración del capítulo%
    
    \setcounter{chapter}{1}

    \chapter{Lista 2}
    
    \setcounter{section}{1}

    \begin{excer}
        Traduzca cada una de las siguientes fórmulas de primer orden a español ordinario, utilizando los símbolos de predicado unario $A$, $B$ e $I$ para representar \textit{es un autor}, \textit{es un libro} y \textit{es interesante}, respectivamente; por otra parte, el símbolo de predicado binario $C$ debe ser interpretado como \textit{es más caro que}. Finalmente, el símbolo de función unario $P$ representa a la función que recibe como entrada un libro y, arroja como salida a su autor.
        \begin{enumerate}[label=($\alph*$)]
            \item $(\forall x)(B(x)\Rightarrow (\exists y)(A(y)\land y=P(x)))$.
            \item $(\forall x)(\forall y)((B(x)\land B(y)\land I(x)\land\neg I(y))\Rightarrow C(x,y))$.
            \item $(\forall x)(B(x)\Rightarrow((\exists y)(B(y)\land C(x,y))\Rightarrow I(x)))$.
        \end{enumerate}
    \end{excer}

    \begin{sol}
        %TODO
    \end{sol}

    \begin{excer}
        Dada cada uno de los siguientes enunciados en español, identifique el universo de discurso apropiado, así como los símbolos adecuados de constante, función y relación (específicando la aridad de cada uno de ellos) para poder escribir simbólicamente una traducción a la lógica de primer orden:
        \begin{enumerate}[label=($\alph*$)]
            \item Todo número primo debe ser impar.
            \item Si hay al menos una manzana, entonces hay al menos una manzana podrida.
            \item Todo número complejo $z$ tal que $z=\overline{z}$ pertenece al conjunto $\mathbb{R}$.
        \end{enumerate}
    \end{excer}

    \begin{sol}
        %TODO
    \end{sol}

    \begin{excer}
        Considere el llamado \textbf{lenguaje de la aritmética de Peano}, el cual consta de un símbolo de constante $1$, un símbolo de relación binaria $<$, un símbolo de función unaria $S$ y dos símbolos de función binaria, $+$ y $\cdot$. Determine cuáles de las siguiente sucesiones de símbolos denotan términos y/o fórmulas bien formadas (o bien, de manera más precisa, cuáles de las siguientes podrían convertirse en términos y/o fórmulas bien formadas módulo algunas abreviaturas).
        \begin{enumerate}[label=($\alph*$)]
            \item $v_1<(v_2+S(v_3))$.
            \item $S(v_5+(S(S(1))+v_27))$.
            \item $S(v_9+S((\forall 1)(1+S(1)=v_{57})))$.
            \item $(\forall v_{48})(S(v_{48}+1)\cdot S(S(S(1)))=v_{23})$.
        \end{enumerate}
    \end{excer}

    \begin{sol}
        %TODO
    \end{sol}

    \begin{excer}
        Sea $\mathcal{L}$ un lenguaje de primer orden con cuatro símbolos de relación unaria $P,Q,S,T$, así cómo símbolos de relación binaria $B,C,D$ y símbolos de constante $c,d$. Construya demostraciones formales de validez para cada uno de los siguientes argumentos:
        \begin{enumerate}[label=($\alph*$)]
            \item \begin{center}
                    \setcounter{tablec}{1}
                    \begin{tabular}{l r l c l r}
                        & \pstable{tablec} & $(\exists x)(\forall y)(Px$ & $\iff$ & $Qy)$ & Premisa \\
                        \hline
                        & & & $\therefore$ & $(\forall y)(\exists x)(Px\iff Qy)$ & \\
                    \end{tabular}
                \end{center}
            \item \begin{center}
                \setcounter{tablec}{1}
                \begin{tabular}{l r l c l r}
                    & \pstable{tablec} & $(\forall x)(\exists y)(Px$ & $\land$ & $Qy)$ & Premisa \\
                    \hline
                    & & & $\therefore$ & $(\exists y)(\forall x)(Px\land Qy)$ & \\
                \end{tabular}
            \end{center}
            \item \begin{center}
                \setcounter{tablec}{1}
                \begin{tabular}{l r l c l r}
                    \hline
                    & & & $\therefore$ & $(\forall x)(Px\Rightarrow Qx)\Rightarrow ((\forall x)(Px)\Rightarrow (\forall x)(Qx))$ & \\
                \end{tabular}
            \end{center}
            \item \begin{center}
                \setcounter{tablec}{1}
                \begin{tabular}{l r l c l r}
                    \hline
                    & & & $(\forall x)(Px\Rightarrow\varphi)\iff((\exists x)(Px)\Rightarrow\varphi)$ & \\
                \end{tabular}
            \end{center}
            \item \begin{center}
                \setcounter{tablec}{1}
                \begin{tabular}{l r l c l r}
                    \hline
                    & & & $\therefore$ & $(\exists x)(Px\land\varphi)\iff((\exists x)(Px)\land\varphi)$ & \\
                \end{tabular}
            \end{center}
            \item \begin{center}
                \setcounter{tablec}{1}
                \begin{tabular}{l r l c l r}
                    \hline
                    & & & $\therefore$ & $(\forall x)(\exists y)(Px\Rightarrow Qy)\Rightarrow((\forall x)Px\Rightarrow(\exists y)Qy)$ & \\
                \end{tabular}
            \end{center}
            \item \begin{center}
                \setcounter{tablec}{1}
                \begin{tabular}{l r l c l r}
                    \hline
                    & & & $\therefore$ & $(\exists x)(Px\Rightarrow\varphi)\iff((\forall x)(Px)\Rightarrow\varphi)$ & \\
                \end{tabular}
            \end{center}
            \item \begin{center}
                \setcounter{tablec}{1}
                \begin{tabular}{l r l c l r}
                    \hline
                    & & & $\therefore$ & $(\exists x)(Px\lor\varphi)\Rightarrow((\forall x)(Px)\lor\varphi)$ & \\
                \end{tabular}
            \end{center}
            \item \begin{center}
                \setcounter{tablec}{1}
                \begin{tabular}{l r l c l r}
                    \hline
                    & & & $\therefore$ & $((\exists x)(Px)\lor(\exists x)(Qx))\iff(\exists x)(Px\lor Qx)$ & \\
                \end{tabular}
            \end{center}
            \item \begin{center}
                \setcounter{tablec}{1}
                \begin{tabular}{l r l c l r}
                    \hline
                    & & & $\therefore$ & $(\forall x)(Px\lor\varphi)\iff((\forall x)(Px)\lor\varphi)$ & \\
                \end{tabular}
            \end{center}
            \item \begin{center}
                \setcounter{tablec}{1}
                \begin{tabular}{l r l c l r}
                    \hline
                    & & & $\therefore$ & $(\forall x)(\exists y)(Px\land Qy)\iff(\exists y)(\forall x)(Px\land Qx)$ & \\
                \end{tabular}
            \end{center}
            \item \begin{center}
                \setcounter{tablec}{1}
                \begin{tabular}{l r l c l r}
                    \hline
                    & & & $\therefore$ & $(\forall x)(\exists y)(Px\land Qy)\iff(\exists y)(\forall x)(Px\land Qy)$ & \\
                \end{tabular}
            \end{center}
            \item \begin{center}
                \setcounter{tablec}{1}
                \begin{tabular}{l r l c l r}
                    \hline
                    & & & $\therefore$ & $(\exists x)(Px\Rightarrow\varphi)\iff((\forall x)(Px)\Rightarrow\varphi)$ & \\
                \end{tabular}
            \end{center}
        \end{enumerate}
        %TODO
    \end{excer}

    \begin{excer}
        Sea $\mathcal{L}$ el lenguaje de primer orden cuyo único símbolo lógico es una relación unaria $P$. Demuestre o refute las siguientes afirmaciones:
        \begin{enumerate}[label=$(\alph*)$]
            \item $(\forall v_1)(Pv_1)\vDash P(v_2)$.
            \item $Pv_2\vDash(\forall v_2)(Pv_1)$.
            \item $(\forall v_1)(Pv_1)\vDash(\exists v_1)(Pv_1)$.
            \item $\emptyset\vDash(\exists v_5)(Pv_5)\Rightarrow(\forall v_6)(Pv_6)$.
            \item $\vDash(\exists v_5)(Pv_5\Rightarrow(\forall v_6)(Pv_6))$.
        \end{enumerate}
    \end{excer}

    \begin{sol}
        %TODO
    \end{sol}

    \begin{excer}
        Sea $\mathcal{L}$ el mismo lenguaje del problema anterior.
        \begin{enumerate}[label=$(\alph*)$]
            \item Caracterice a los modelos que satisfacen el enunciado $(\forall x)(\forall y)(x=y)$.
            \item Caracterice a los modelos que satisfacen el enunciado $(\exists x)(\exists y)(\neg(x=y)\land(\forall z)(z=x\lor z=y))$.
            \item Caracterice a los modelos que satisfacen el enunciado $(\forall x)(\neg Px)$.
        \end{enumerate}
    \end{excer}

    \begin{sol}
        %TODO
    \end{sol}

    \begin{excer}
        Sea $\varphi$ una fórmula (de algún lenguaje de primer orden, $\mathcal{L}$), y sea $\left\{v_{ i_1},...,v_{ i_k} \right\}=\free\left(\varphi\right)$ (supongamos que para todo esté bien definido y que $i_1<...<i_k$). Definimos la \textbf{cerradura universal} de $\varphi$ como la oración $(\forall v_{ i_k})\cdots(\forall v_{ i_1})(\varphi)$.
        \begin{enumerate}[label=$(\alph*)$]
            \item Escriba la definición formal de la cerradura universal de una fórmula.
            
            \textit{Sugerencia.} Inducción sobre el número de variables libres.
            \item Demuestre que, para cualquier conjunto de enunciados $\Sigma$, se cumple que $\Sigma\vDash\varphi$ si y sólo si $\Sigma\vDash\psi$, donde $\psi$ es la cerradura universal de $\varphi$.
        \end{enumerate}
    \end{excer}

    \begin{proof}
        %TODO
    \end{proof}

    \begin{excer}
        Considere el lenguaje de primer orden $\mathcal{L}$ cuyo único símbolo no lógico es uno de relación binaria, $P$. Demuestre que, de entre los siguientes enunciados, no hay dos de ellos que contengan al otro como consecuencia lógica.
        \begin{enumerate}[label=$(\alph*)$]
            \item $(\forall x)(\forall y)(\forall z)(P(x,y)\Rightarrow P(x,z))$.
            \item $(\forall x)(\forall y)(P(x,y)\Rightarrow (P(y,x)\Rightarrow x=y))$.
            \item $(\forall x)(\exists y)(P(x,y))\Rightarrow(\exists y)(\forall x)(P(x,y))$.
        \end{enumerate} 
    \end{excer}

    \begin{proof}
        %TODO
    \end{proof}

    \begin{excer}
        Sea $\mathcal{L}$ el lenguaje de primer orden cuyos símbolos no lógicos son un símbolo de función unaria $F$, y un símbolo de relación biaria $P$. Demuestre que $\emptyset\vDash x=y\Rightarrow(P(z,F(x))\Rightarrow P(z,F(y)))$.
    \end{excer}

    \begin{proof}
        %TODO
    \end{proof}

\end{document}