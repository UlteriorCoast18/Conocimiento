\documentclass[12pt]{report}
\usepackage[spanish]{babel}
\usepackage[utf8]{inputenc}
\usepackage{amsmath}
\usepackage{amssymb}
\usepackage{amsthm}
\usepackage{graphics}
\usepackage{subfigure}
\usepackage{lipsum}
\usepackage{array}
\usepackage{multicol}
\usepackage{enumerate}
\usepackage[framemethod=TikZ]{mdframed}
\usepackage[a4paper, margin = 1.5cm]{geometry}
\usepackage{tikz}
\usepackage{pgffor}
\usepackage{ifthen}
\usepackage{enumitem}
\usepackage{hyperref}

\usepackage{listings}

%Gestión de Hipervínculos

\hypersetup{
    colorlinks=true,
    linkcolor=black,
    filecolor=magenta,      
    urlcolor=cyan
}

%Gestión de Código de Programación

\definecolor{listing-background}{HTML}{F7F7F7}
\definecolor{listing-rule}{HTML}{B3B2B3}
\definecolor{listing-numbers}{HTML}{B3B2B3}
\definecolor{listing-text-color}{HTML}{000000}
\definecolor{listing-keyword}{HTML}{435489}
\definecolor{listing-keyword-2}{HTML}{1284CA} % additional keywords
\definecolor{listing-keyword-3}{HTML}{9137CB} % additional keywords
\definecolor{listing-identifier}{HTML}{435489}
\definecolor{listing-string}{HTML}{00999A}
\definecolor{listing-comment}{HTML}{8E8E8E}

\lstdefinestyle{myStyle}{
    language         = C++,
    alsolanguage     = scala,
    numbers          = left,
    xleftmargin      = 2.7em,
    framexleftmargin = 2.5em,
    backgroundcolor  = \color{gray!15},
    basicstyle       = \color{listing-text-color}\linespread{1.0}\ttfamily,
    breaklines       = true,
    frameshape       = {RYR}{Y}{Y}{RYR},
    rulecolor        = \color{black},
    tabsize          = 2,
    numberstyle      = \color{listing-numbers}\linespread{1.0}\small\ttfamily,
    aboveskip        = 1.0em,
    belowskip        = 0.1em,
    abovecaptionskip = 0em,
    belowcaptionskip = 1.0em,
    keywordstyle     = {\color{listing-keyword}\bfseries},
    keywordstyle     = {[2]\color{listing-keyword-2}\bfseries},
    keywordstyle     = {[3]\color{listing-keyword-3}\bfseries\itshape},
    sensitive        = true,
    identifierstyle  = \color{listing-identifier},
    commentstyle     = \color{listing-comment},
    stringstyle      = \color{listing-string},
    showstringspaces = false,
    label            = lst:bar,
    captionpos       = b,
}

\lstset{style = myStyle}

%Estilo del capítulo y sección

\makeatletter
\def\thickhrulefill{\leavevmode \leaders \hrule height 1ex \hfill \kern \z@}
\def\@makechapterhead#1{%
  {\parindent \z@ \raggedright
    \reset@font
    \hrule
    \vspace*{10\p@}%
    \par
    \center \LARGE \scshape \@chapapp{} \huge \thechapter
    \vspace*{10\p@}%
    \par\nobreak
    \vspace*{10\p@}%
    \par
    \vspace*{1\p@}%
    \hrule
    %\vskip 40\p@
    \vspace*{60\p@}
    \Huge #1\par\nobreak
    \vskip 50\p@
  }}

\def\section#1{%
  \par\bigskip\bigskip
  \hrule\par\nobreak\noindent
  \refstepcounter{section}%
  \addcontentsline{toc}{chapter}{#1}%
  \reset@font
  { \large \scshape
    \strut\S \thesection \quad
    #1}% 
    \hrule   
  \par
  \medskip
}

%Gestión marca de agua

\usetikzlibrary{shapes.multipart}

\newcounter{it}
\newcommand*\watermarktext[1]{\begin{tabular}{c}
    \setcounter{it}{1}%
    \whiledo{\theit<100}{%
    \foreach \col in {0,...,15}{#1\ \ } \\ \\ \\
    \stepcounter{it}%
    }
    \end{tabular}
    }

\AddToHook{shipout/foreground}{
    \begin{tikzpicture}[remember picture,overlay, every text node part/.style={align=center}]
        \node[rectangle,black,rotate=30,scale=2,opacity=0.04] at (current page.center) {\watermarktext{Cristo Daniel Alvarado ESFM\quad}};
  \end{tikzpicture}
}

%En esta parte se hacen redefiniciones de algunos comandos para que resulte agradable el verlos%

\def\proof{\paragraph{Demostración:\\}}
\def\endproof{\hfill$\blacksquare$}

\def\sol{\paragraph{Solución:\\}}
\def\endsol{\hfill$\square$}

%En esta parte se definen los comandos a usar dentro del documento para enlistar%

\newtheoremstyle{largebreak}
  {}% use the default space above
  {}% use the default space below
  {\normalfont}% body font
  {}% indent (0pt)
  {\bfseries}% header font
  {}% punctuation
  {\newline}% break after header
  {}% header spec

\theoremstyle{largebreak}

\newmdtheoremenv[
    leftmargin=0em,
    rightmargin=0em,
    innertopmargin=0pt,
    innerbottommargin=5pt,
    hidealllines = true,
    roundcorner = 5pt,
    backgroundcolor = gray!60!red!30
]{exa}{Ejemplo}[section]

\newmdtheoremenv[
    leftmargin=0em,
    rightmargin=0em,
    innertopmargin=0pt,
    innerbottommargin=5pt,
    hidealllines = true,
    roundcorner = 5pt,
    backgroundcolor = gray!50!blue!30
]{obs}{Observación}[section]

\newmdtheoremenv[
    leftmargin=0em,
    rightmargin=0em,
    innertopmargin=0pt,
    innerbottommargin=5pt,
    rightline = false,
    leftline = false
]{theor}{Teorema}[section]

\newmdtheoremenv[
    leftmargin=0em,
    rightmargin=0em,
    innertopmargin=0pt,
    innerbottommargin=5pt,
    rightline = false,
    leftline = false
]{propo}{Proposición}[section]

\newmdtheoremenv[
    leftmargin=0em,
    rightmargin=0em,
    innertopmargin=0pt,
    innerbottommargin=5pt,
    rightline = false,
    leftline = false
]{cor}{Corolario}[section]

\newmdtheoremenv[
    leftmargin=0em,
    rightmargin=0em,
    innertopmargin=0pt,
    innerbottommargin=5pt,
    rightline = false,
    leftline = false
]{lema}{Lema}[section]

\newmdtheoremenv[
    leftmargin=0em,
    rightmargin=0em,
    innertopmargin=0pt,
    innerbottommargin=5pt,
    roundcorner=5pt,
    backgroundcolor = gray!30,
    hidealllines = true
]{mydef}{Definición}[section]

\newmdtheoremenv[
    leftmargin=0em,
    rightmargin=0em,
    innertopmargin=0pt,
    innerbottommargin=5pt,
    roundcorner=5pt
]{excer}{Ejercicio}[section]

%En esta parte se colocan comandos que definen la forma en la que se van a escribir ciertas funciones%

\newcommand\abs[1]{\ensuremath{\left|#1\right|}}
\newcommand\divides{\ensuremath{\bigm|}}
\newcommand\cf[3]{\ensuremath{#1:#2\rightarrow#3}}
\newcommand\contradiction{\ensuremath{\#_c}}
\newcommand\natint[1]{\ensuremath{\left[\big|#1\big|\right]}}

\begin{document}
    \setlength{\parskip}{5pt} % Añade 5 puntos de espacio entre párrafos
    \setlength{\parindent}{12pt} % Pone la sangría como me gusta
    \title{Lista 4 de Problemas y Ejercicios
    
    Curso de Lógica Matemática}
    \author{Cristo Daniel Alvarado}
    \maketitle

    \setcounter{chapter}{1}

    \setcounter{section}{3}

    \section{Lista 4}

    \begin{excer}
        Utilizando la definición de par ordenado de Kuratowski (a saber, $(a,b)=\left\{\left\{a\right\},\left\{a,b\right\} \right\}$), demuestre que:
        \begin{equation*}
            (a,b)=(c,d)\textup{ si y sólo si }a=c\textup{ y }c=d
        \end{equation*}
    \end{excer}

    \begin{proof}
        $\Rightarrow):$ Suponga que $(a,b)=(c,d)$, entonces:
        \begin{equation*}
            \left\{\left\{a\right\},\left\{a,b\right\} \right\}=\left\{\left\{c\right\},\left\{c,d\right\} \right\}
        \end{equation*}
        Analicemos por casos:
        \begin{itemize}
            \item $a=b$, en cuyo caso el conjunto de la izquierda se convierte en:
            \begin{equation*}
                \left\{\left\{a\right\} \right\}=\left\{\left\{c\right\},\left\{c,d\right\} \right\}
            \end{equation*}
            por lo que $\left\{c\right\}=\left\{a\right\}$ y $\left\{c,d\right\}=\left\{a\right\}$. De la primera igualdad se sigue que $a=c$ y de la segunda que $c=d$, esto es que $a=c$ y $b=d$.
            \item $a\neq b$, en cuyo caso se tiene que $\left\{c\right\}=\left\{a\right\}$ o $\left\{c\right\}=\left\{a,b\right\}$. Afirmamos que lo segundo no puede suceder, ya que en tal caso $a=c$ y $b=c$, lo cual implicaría que $a=b$\contradiction. Por tanto, $\left\{c\right\}=\left\{a\right\}$ y, por ende $a=c$.
            
            Ahora, tampoco puede suceder que $\left\{c,d \right\}=\left\{a\right\}$ ya que en tal caso $\left\{a,b \right\}=\left\{c\right\}$ lo que implicaría que $a=b$\contradiction, por lo que $\left\{c,d \right\}=\left\{a,b\right\}$. Nuevamente, debe suceder que $b=d$.
        \end{itemize}
        Los diso incisos anteriores prueban el resultado.

        $\Leftarrow):$ Es inmediata.
    \end{proof}

    \begin{excer}
        Suponga que decidimos usar una definición alternativa para los pares ordenados:
        \begin{equation*}
            (a,b)=\left\{a,\left\{a,b\right\} \right\}
        \end{equation*}
        \begin{enumerate}[label = \textit{(\alph*)}]
            \item Suponga que existen tres conjuntos distintos $x,y$ y $z$ tales que $x=\left\{\left\{x,y\right\},z \right\}$. Demuestre que entonces se tiene que $(x,y)=(\left\{x,y\right\},z)$.
            \item Demuestre que esta definición alternativa de par ordenado satisface que $(a,b)=(c,d)$ si y sólo si $a=b$ y $c=d$.
            
            \textit{Sugerencia}. Usar axioma de fundación.
        \end{enumerate}
    \end{excer}

    \begin{proof}
        De \textit{(1)}: veamos que:
        \begin{equation*}
            \begin{split}
                (x,y)&=\left\{x,\left\{x,y\right\} \right\}\\
                &=\left\{\left\{x,y\right\},\left\{\left\{x,y\right\},z \right\} \right\}\\
                &=\left\{a,\left\{a,z\right\} \right\}\\
                &=(a,z)\\
            \end{split}
        \end{equation*}
        donde $a=\left\{x,y\right\}$. Por tanto, $(x,y)=(\left\{x,y \right\},z)$.

        De \textit{(2)}: La suficiencia es inmediata. Suponga que $(a,b)=(c,d)$, es decir que $\left\{a,\left\{a,b\right\}\right\}=\left\{c,\left\{c,d\right\}\right\}$. Analicemos por casos:
        \begin{itemize}
            \item $a=b$, en cuyo caso se sigue que $(a,b)=\left\{a,\left\{a\right\}\right\}$, por tanto $c=a$ o $c=\left\{a\right\}$. Si $c=\left\{a\right\}$, entonces tenemos que:
            \begin{equation*}
                (c,d)=\left\{\left\{a\right\},\left\{\left\{a\right\},d\right\} \right\}
            \end{equation*}
            no puede pasar que $\left\{\left\{a\right\},d\right\}=\left\{a\right\}$, ya que esto implicaría que $\left\{a\right\}\in\left\{a\right\}$\contradiction. Por tanto, $a=\left\{\left\{a\right\},d \right\}$, así que en particular, $\left\{a\right\}\in a$.

            Por axioma de fundación, para el conjunto $\left\{a,\left\{a\right\} \right\}$ existe $b\in\left\{a,\left\{a\right\} \right\}$ tal que $b\cap\left\{a,\left\{a\right\} \right\}=\emptyset$, en particular $b\neq\left\{a\right\}$, por lo que $b=a$, así que $a\cap\left\{a,\left\{a\right\}\right\}=\emptyset$, en particular, $\left\{a\right\}\notin a$. Esto contradice lo de arriba.

            Por tanto, $c\neq\left\{a\right\}$, así que $a=c$. Ahora, se tiene que:
            \begin{equation*}
                \left\{a,\left\{a\right\} \right\}=\left\{a,\left\{a,d\right\} \right\}
            \end{equation*}
            se tiene que debe suceder que $\left\{a,d \right\}=\left\{a\right\}$, es decir que $d=a$. Por tano, $a=c$ y $b=d$.

            \item $a\neq b$, en cuyo caso se tiene que $(a,b)=\left\{a,\left\{a,b\right\}\right\}$ y $(c,d)=\left\{c,\left\{c,d\right\} \right\}$. Se tienen dos casos:
            \begin{itemize}
                \item $a=c$, en cuyo caso se sigue que:
                \begin{equation*}
                    \left\{a,\left\{a,b\right\}\right\}=\left\{a,\left\{a,d\right\} \right\}
                \end{equation*}
                por lo que $\left\{a,b\right\}=\left\{a,d \right\}$ (en caso contrario se llega a que $a\in a$\contradiction), se sigue así que $b=d$ ya que en caso contrario se tendría que $b=a$\contradiction.
                \item $a=\left\{a,d\right\}$, esto no puede suceder pues llegaríamos a una contradicción ya que implicaría que $a\in a$\contradiction.
            \end{itemize}
            Por los dos incisos anteriores, se sigue que $a=c$ y $b=d$.
        \end{itemize}
    \end{proof}

    \begin{excer}
        
    \end{excer}

    \begin{mydef}
        Un \textbf{ordinal límite} es un ordinal $\alpha$ no cero tal que $\alpha$ no es el sucesor de ningún ordinal.
    \end{mydef}

    \begin{excer}
        Demuestre que si $\gamma$ es ordinal límite, entonces $\alpha+\gamma$ también lo es.

        \textit{Sugerencia.} Proceder por contradicción.
    \end{excer}

    \begin{proof}
        Sea $\gamma$ un ordinal límite y $\alpha$ ordinal. Supongamos que $\alpha+\gamma$ no es ordinal límite, entonces existe un ordinal $\beta$ tal que:
        \begin{equation*}
            \alpha+\gamma=S(\beta)
        \end{equation*}
        en particular, $\beta<\alpha+\gamma$.

        Ahora, como $\gamma$ es ordinal límite, se tiene que:
        \begin{equation*}
            \alpha+\gamma=\sup\left\{\alpha+\eta\Big|\eta<\gamma \right\}
        \end{equation*}
        por lo anterior se tiene que existe $\eta<\gamma$ tal que:
        \begin{equation*}
            \beta=\alpha+\eta
        \end{equation*}
        luego,
        \begin{equation*}
            S(\beta)=S(\alpha+\eta)=\alpha+S(\eta)
        \end{equation*}
        en particular, se tiene que:
        \begin{equation*}
            \alpha+S(\eta)=\sup\left\{\alpha+\eta\Big|\eta<\gamma \right\}
        \end{equation*}
        pero, $\alpha+S(\eta)<\sup\left\{\alpha+\eta\Big|\eta<\gamma \right\}$ ya que $\gamma$ es ordinal límite\contradiction. Por tanto, $\alpha+\gamma$ es ordinal límite.
    \end{proof}

    \begin{excer}
        Demuestre que, para todos los ordinales $\alpha,\beta$ se cumple que $\beta\leq\alpha+\beta$ y, si $\beta>0$, entonces $\alpha<\alpha+\beta$.
    \end{excer}

    \begin{proof}
        Sea $\alpha$ ordinal. Probaremos la afirmación por inducción transfinita. En efecto, supongamos que para todo $\gamma$ ordinal tal que $\gamma<\beta$ se tiene que $\gamma\leq\alpha+\gamma$, probaremos que $\beta\leq\alpha+\beta$. Se tienen tres casos:
        \begin{itemize}
            \item $\beta=0$, en cuyo caso se tiene que como $\alpha$ es ordinal, entonces $0\leq\alpha$ y, $\alpha+0=\alpha$, por lo cual $0\leq\alpha+0$, es decir que $\beta\leq\alpha+\beta$.
            \item $\beta$ es el sucesor de alguien, en cuyo caso existe $\eta$ tal que $\beta=S(\eta)$, por hipótesis se tiene que $\eta\leq\alpha+\eta$. Como:
            \begin{equation*}
                \alpha+\eta<S(\alpha+\eta)=\alpha+S(\alpha)=\alpha+\beta
            \end{equation*}
            y además, debe suceder que $S(\eta)\leq\alpha+\beta$, esto es que $\beta\leq\alpha+\beta$.
            \item Suponga que $\beta$ no es el sucesor de nadie. Veamos que:
            \begin{equation*}
                \alpha+\beta=\sup\left\{\alpha+\eta\Big|\eta<\beta \right\}
            \end{equation*}
            en particular, se tiene que:
            \begin{equation*}
                \alpha+\eta<\alpha+\beta,\quad\forall\eta<\beta
            \end{equation*}
            y, por hipótesis de inducción:
            \begin{equation*}
                \eta\leq\alpha+\eta,\quad\forall\eta<\beta
            \end{equation*}
            por lo que:
            \begin{equation*}
                \eta<\alpha+\beta,\quad\forall\eta<\beta
            \end{equation*}
            Recordemos que:
            \begin{equation*}
                \beta=\sup\left\{\eta\Big|\eta<\beta \right\}=\left\{\eta\Big|\eta<\beta \right\}
            \end{equation*}
            así que:
            \begin{equation*}
                \beta\leq\alpha+\beta
            \end{equation*}
        \end{itemize} 
        aplicando inducción transfinita por los tres incisos anteriores se sigue el resultado.

        Ahora, probaremos que:
        \begin{equation*}
            \alpha<\alpha+\beta
        \end{equation*}
        para todo par de ordinales $\alpha,\beta$ tales que $\beta>0$. En efecto, procederemos por inducción transfinita sobre $\beta$. Suponga que para todo ordinal 
    \end{proof}

    \begin{excer}
        Para todo ordinal $\alpha$, $0+\alpha=\alpha$.
    \end{excer}

    \begin{proof}
        Procederemos por inducción transfinita sobre $\alpha$: Supongamos que existe un ordinal $\gamma$ tal que para todo $\beta<\gamma$ se tiene que $0+\beta=\beta$. Se tienen tres casos:
        \begin{itemize}
            \item $\gamma=0$, en cuyo caso se sigue que $0+\gamma=\gamma+\gamma=\gamma+0=\gamma$.
            \item $\gamma$ es el sucesor de algún ordinal $\eta$, en particular, como $\eta<\gamma$, entonces por hipótesis de inducción: $0+\eta=\eta$, luego:
            \begin{equation*}
                0+\gamma=0+S(\eta)=S(0+\eta)=S(\eta)=\gamma
            \end{equation*}
            \item $\gamma$ no es el sucesor de nadie, entonces es ordinal límite. Veamos que:
            \begin{equation*}
                \begin{split}
                    0+\gamma&=\sup\left\{0+\beta\Big|\beta<\gamma \right\}\\
                    &=\sup\left\{\beta\Big|\beta<\gamma \right\}\\
                    &=\gamma\\
                \end{split}
            \end{equation*}
        \end{itemize}
        por los tres incisos anterioes y aplicando inducción transfinita se sigue el resultado.
    \end{proof}

    \begin{excer}
        Demuestre que la adición ordinal satisface la \textit{propiedad cancelativa por la izquierda}: Para cualesquiera ordinales $\alpha,\beta,\gamma$ se tiene que:
        \begin{equation*}
            \beta+\alpha=\beta+\gamma\Rightarrow\alpha=\gamma
        \end{equation*}
    \end{excer}

    \begin{proof}
        Procederemos por inducción sobre $\beta$. Suponga que existe un ordinal $\eta$ tal que $(\forall\chi<\eta)(\chi+\alpha=\chi+\beta\Rightarrow\alpha=\gamma)$. Se tienen tres casos:
        \begin{itemize}
            \item $\eta=0$, en cuyo caso se tiene que:
            \begin{equation*}
                \eta+\alpha=\eta+\gamma\Rightarrow0+\alpha=0+\gamma\Rightarrow\alpha=\gamma
            \end{equation*}
            por el ejercicio anterior.
            \item Suponga que $\eta$ es el sucesor de algún ordinal $\chi$, entonces por hipótesis de inducción se sigue que:
            \begin{equation*}
                \chi+\alpha=\chi+\gamma\Rightarrow\alpha=\gamma
            \end{equation*}
            Se sigue así que:
            \begin{equation*}
                \begin{split}
                    \eta+\alpha=\eta+\gamma&\Rightarrow S(\chi)+\alpha=S(\chi)+\gamma\\
                \end{split}
            \end{equation*}
            Se tiene tres casos:
            \begin{itemize}
                \item $\alpha=0$, en cuyo caso se sigue que:
                \begin{equation*}
                    \begin{split}
                        S(\chi)=S(\chi)+\gamma
                    \end{split}
                \end{equation*}
                Si $\gamma>0$, entonces por el ejercicio anterior se seguiría que: $S(\chi)<S(\chi)+\gamma$\contradiction. Por tanto, $\gamma=0$.
                \item $\alpha$ es el sucesor de algún número, digamos $\upsilon$, entonces:
                \begin{equation*}
                    \begin{split}
                        S(\chi)+\alpha&=S(\chi)+S(\upsilon)\\
                        &=S(S(\chi)+\upsilon)\\
                    \end{split}
                \end{equation*}
                por lo que:
                \begin{equation*}
                    \begin{split}
                        S(S(\chi)+\upsilon)=S(\chi)+\gamma
                    \end{split}
                \end{equation*}
                se tiene que $\gamma\neq0$ (en caso contrario por un ejercicio llegaríamos a una contradicción). Si $\gamma$ no fuera el sucesor de nadie, por un ejercicio anterior se tendría que $S(\chi)+\gamma$ tampoco lo es\contradiction. Por ende, $\gamma$ es el sucesor de alguien, así que existe $\nu$ ordinal tal que $\gamma=S(\nu)$. Se tiene pues que:
                \begin{equation*}
                    S(S(\chi)+\upsilon)=S(S(\chi)+\nu)
                \end{equation*}
                por lo cual, $S(\chi)+\upsilon=S(\chi)$%TODO
            \end{itemize}
        \end{itemize}
    \end{proof}

    \begin{obs}
        Se modificó el siguiente ejercicio.
    \end{obs}

    \begin{excer}
        Demuestre que para un número ordinal $\alpha$ no cero, las siguientes condiciones son equivalentes:
        \begin{enumerate}[label = \textit{(\arabic*)}]
            \item $\alpha$ es ordinal límite.
            \item $\forall\beta<\alpha$ se tiene que $S(\beta)<\alpha$.
            \item $\alpha=\bigcup_{\beta\in\alpha}\beta$.
        \end{enumerate}
    \end{excer}

    \begin{proof}
        \textit{(1) $\Rightarrow$ (2)}: Suponga que $\alpha$ es ordinal límite, entonces no es sucesor de ningún ordinal. Suponga que existe un ordinal $\beta$ tal que $\beta<\alpha$ y $\alpha\leq S(\beta)$. Se tienen dos casos:
        \begin{itemize}
            \item $\alpha=S(\beta)$, lo cual es una contradicción ya que $\alpha$ no es sucesor de ningún ordinal.
            \item $\alpha<S(\beta)$, debe tenerse que $\alpha\leq\beta$\contradiction, pues $\beta<\alpha$.
        \end{itemize}
        en ambos incisos anteriores se llega a una contradicción, por lo que $\forall\beta<\alpha(S(\beta)<\alpha)$.

        \textit{(2) $\Rightarrow$ (3)}: Suponga que $\forall\beta<\alpha(S(\beta)<\alpha)$. Veamos que $\alpha=\bigcup_{\beta\in\alpha}\beta$. Antes de empezar, ambos conjuntos son ordinales:
        \begin{itemize}
            \item Para todo $\beta\in\alpha$, luego $\beta\subseteq\alpha$, lo cual implica que $\bigcup_{\beta\in\alpha}\subseteq\alpha$.
            \item Sea $\gamma\in\alpha$, entonces $\gamma<\alpha$, en particular $S(\gamma)<\alpha\Rightarrow S(\gamma)\in\alpha$ y $\gamma\in S(\gamma)$, por lo que $\gamma\in S(\gamma)\subseteq\bigcup_{\beta\in\alpha}\beta$. Así que $\alpha\subseteq\bigcup_{\beta\in\alpha}\beta$.
        \end{itemize}
        por ambos incisos se sigue la igualdad.

        \textit{(3) $\Rightarrow$ (1)}: Suponga que $\alpha=\bigcup_{\beta\in\alpha}\beta$. Si $\alpha$ no fuese ordinal límite, existiría $\eta<\alpha$ tal que $\alpha=S(\eta)$.

        Si sucede lo segundo, entonces:
        \begin{equation*}
            \begin{split}
                S(\eta)&=\alpha\\
                &=\bigcup_{\beta\in\alpha}\beta\\
            \end{split}
        \end{equation*}
        en particular, para todo $\beta\in\alpha$ se tiene que $\beta\leq\eta$ pues en caso contrario $\alpha$ no podría ser el sucesor de $\eta$, así que:
        \begin{equation*}
            \beta\subseteq\eta,\quad\forall\beta\in\alpha
        \end{equation*}
        por ende,
        \begin{equation*}
            \bigcup_{\beta\in\alpha}\beta\subseteq\eta
        \end{equation*}
        y, $\eta\subseteq\bigcup_{\beta\in\alpha}\beta$ pues $\eta\in\alpha$. Por tanto:
        \begin{equation*}
            \bigcup_{\beta\in\alpha}\beta=\eta
        \end{equation*}
        lo que implica:
        \begin{equation*}
            S(\eta)=\eta
        \end{equation*}
        cosa que es una contradicción\contradiction. Por tanto, $\alpha$ es ordinal límite.
    \end{proof}

    \begin{excer}[Nombre]
        
    \end{excer}

\end{document}