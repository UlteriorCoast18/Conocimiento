\documentclass[12pt]{report}
\usepackage[spanish]{babel}
\usepackage[utf8]{inputenc}
\usepackage{amsmath}
\usepackage{amssymb}
\usepackage{amsthm}
\usepackage{graphics}
\usepackage{subfigure}
\usepackage{lipsum}
\usepackage{array}
\usepackage{multicol}
\usepackage{enumerate}
\usepackage[framemethod=TikZ]{mdframed}
\usepackage[a4paper, margin = 1.5cm]{geometry}

%En esta parte se hacen redefiniciones de algunos comandos para que resulte agradable el verlos%

\renewcommand{\theenumii}{\roman{enumii}}

\def\proof{\paragraph{Demostración:\\}}
\def\endproof{\hfill$\blacksquare$}

\def\sol{\paragraph{Solución:\\}}
\def\endsol{\hfill$\square$}

%En esta parte se definen los comandos a usar dentro del documento para enlistar%

\newtheoremstyle{largebreak}
  {}% use the default space above
  {}% use the default space below
  {\normalfont}% body font
  {}% indent (0pt)
  {\bfseries}% header font
  {}% punctuation
  {\newline}% break after header
  {}% header spec

\theoremstyle{largebreak}

\newmdtheoremenv[
    leftmargin=0em,
    rightmargin=0em,
    innertopmargin=0pt,
    innerbottommargin=5pt,
    hidealllines = true,
    roundcorner = 5pt,
    backgroundcolor = gray!60!red!30
]{exa}{Ejemplo}[section]

\newmdtheoremenv[
    leftmargin=0em,
    rightmargin=0em,
    innertopmargin=0pt,
    innerbottommargin=5pt,
    hidealllines = true,
    roundcorner = 5pt,
    backgroundcolor = gray!50!blue!30
]{obs}{Observación}[section]

\newmdtheoremenv[
    leftmargin=0em,
    rightmargin=0em,
    innertopmargin=0pt,
    innerbottommargin=5pt,
    rightline = false,
    leftline = false
]{theor}{Teorema}[section]

\newmdtheoremenv[
    leftmargin=0em,
    rightmargin=0em,
    innertopmargin=0pt,
    innerbottommargin=5pt,
    rightline = false,
    leftline = false
]{propo}{Proposición}[section]

\newmdtheoremenv[
    leftmargin=0em,
    rightmargin=0em,
    innertopmargin=0pt,
    innerbottommargin=5pt,
    rightline = false,
    leftline = false
]{cor}{Corolario}[section]

\newmdtheoremenv[
    leftmargin=0em,
    rightmargin=0em,
    innertopmargin=0pt,
    innerbottommargin=5pt,
    rightline = false,
    leftline = false
]{lema}{Lema}[section]

\newmdtheoremenv[
    leftmargin=0em,
    rightmargin=0em,
    innertopmargin=0pt,
    innerbottommargin=5pt,
    roundcorner=5pt,
    backgroundcolor = gray!30,
    hidealllines = true
]{mydef}{Definición}[section]

\newmdtheoremenv[
    leftmargin=0em,
    rightmargin=0em,
    innertopmargin=0pt,
    innerbottommargin=5pt,
    roundcorner=5pt
]{excer}{Ejercicio}[section]

%En esta parte se colocan comandos que definen la forma en la que se van a escribir ciertas funciones%

\newcommand\abs[1]{\ensuremath{\biglvert#1\bigrvert}}
\newcommand\divides{\ensuremath{\bigm|}}
\newcommand\cf[3]{\ensuremath{#1:#2\rightarrow#3}}
\newcommand\contradiction{\ensuremath{\#_c}}

\begin{document}
    \title{Curso de Lógica Matemática}
    \author{Cristo Daniel Alvarado}
    \maketitle

    \tableofcontents %Con este comando se genera el índice general del libro%

    \setcounter{chapter}{-1} %En esta parte lo que se hace es cambiar la enumeración del capítulo%
    
    \chapter{Introducción}
    
    \section{Temario}
    

    Los siguientes temas se verán a lo largo del curso:

    \renewcommand{\theenumii}{\arabic{enumi}.\arabic{enumii}}

    \begin{enumerate}
        \item Lógica (Teoría de Modelos).
        \begin{enumerate}
            \item Lógica proposicional.
            \item Lógica de primer orden.
        \end{enumerate}
        \item Teoría de la Computabilidad.
        \item Teoría de Conjuntos.
    \end{enumerate}

    Y la bibliografía para el curso es la siguiente:

    \begin{itemize}
        \item Enderton, 'Introducción matemática a la lógica'.
        \item  Enderton, 'Teoría de la computabilidad'.
        \item Copi, 'Lógica Simbólica' o 'Computability Theory'.
        \item Rebeca Weber 'Computability Theory'.
    \end{itemize}

    \section{Conectivas Lógicas}

    La disyunción ($\land$), conjunción ($\lor$), negación ($\neg$), implicación ($\Rightarrow$) y si y sólo si ($\iff$) son las conectivas lógicas usadas usualmente. 

    (Se habló un poco de una cosa llamada forma normal disyuntiva).
    
    A $\left\{\land, \lor, \neg \right\}$ se le conoce como un conjunto completo de conectivas lógicas. Nos podemos quedar simplemente con conjuntos completos de disyuntivas con solo dos elementos, a saber: $\left\{\land, \neg \right\}$ y $\left\{\lor, \neg \right\}$, ya que $P\lor Q$ es $\neg(\neg P\land \neg Q)$. (de forma similar a lo otro $P\land Q$ es $\neg(\neg P\lor \neg Q)$).

    También $\left\{\Rightarrow, \neg \right\}$ es otro conjunto completo de conectivas lógicas, ya que $P\land Q$ es $\neg(P\Rightarrow\neq Q)$.

    Y, $\left\{|\right\}$ es un conjunto completo, donde $|$ es llamado la \textbf{barra de Scheffel}, que tiene la siguiente tabla de verdad.

    \begin{center}
        \begin{tabular}{c c | c}
            \hline
            $P$ & $Q$ & $P|Q$ \\
            \hline
            $V$ & $V$ & $F$ \\
            $V$ & $F$ & $V$ \\
            $F$ & $V$ & $V$ \\
            $F$ & $F$ & $V$ \\
        \end{tabular}
    \end{center}
    con este, se tiene un conjunto completo de conectivas lógicas.

    Como muchas veces se usan conectivas de este tipo:
    \begin{equation*}
        (P\Rightarrow \neg Q)\Rightarrow((P\Rightarrow R)\land\neg(Q\Rightarrow S)\land T)
    \end{equation*}
    al ser muy largas, a veces es más conveniente escribirlas en forma Polaca. De esta forma, lo anterior quedaría de la siguiente manera:
    \begin{equation*}
        \Rightarrow\Rightarrow P\neg Q\land\land PR\neg\Rightarrow Q S T 
    \end{equation*}

    Ahora empezamos con el estudio formal de la lógica.

    \chapter{Lógica Proposicional}

    \section{Alfabeto}

    El alfabeto de la lógica proposicional es un conjunto que consta de dos tipos de símbolos:
    \begin{enumerate}
        \item \textbf{Variables}, denotadas por $p_1,p_2,...,p_n,...$ (a lo más una cantidad numerable). Estas representan proposiciones o enunciados (tengo un paraguas, me caí de las escaleras, no tengo café en la cafetera, etc\dots).
        \item \textbf{Conectivas}, como $\Rightarrow$ y $\neg$.
    \end{enumerate}
    Aceptamos la existencia de estas cosas (pues, al menos debemos aceptar la existencia de algo).

    Se van a trabajar con sucesiones finitas de símbolos del alfabeto descrito anteriormente. Ahora necesitaremos especificar que tipos de sucesiones van a servirnos para tener un significado formal.

    \begin{mydef}
        En el conjunto de sucesiones finitas de símbolos del alfabeto, definimos una \textbf{fórmula bien formada} (abreviada como \textbf{FBF}) como sigue:
        \begin{enumerate}
            \item Cada variable es una \textbf{FBF}.
            \item Si $\varphi,\psi$ son \textbf{FBF}, entonces $\neg\varphi$ y $\Rightarrow\varphi\psi$ también lo son.
        \end{enumerate}
    \end{mydef}

    \begin{obs}
        Recordar que usamos la notación Polaca en la definición anterior.
    \end{obs}

    A continuación unos ejemplos:
    
    \begin{exa}
        $p_{17}$, $p_{54}$ y $\Rightarrow p_2p_{25}$ son FBF. Las primeras dos son llamadas \textbf{variables aisladas}. También lo es $\neg \Rightarrow p_2p_{25}$ (en este ejemplo, los $p_i$ son variables).

        Pero, por ejemplo $\Rightarrow \neg p_1 p_2 p_3$ y $\Rightarrow p_4$ no son FBF.
    \end{exa}

    Viendo el ejemplo anterior, notamos que el operador $\Rightarrow$ es binario (solo usa dos entradas) y $\neg$ es unario (solo una entrada). Por lo cual, añadir o no demás variables a los opeadores dentro de la fórmula, hace que la fórmula ya no sea una FBF.

    \begin{obs}
        Eventualmente se va a sustituir la notación Polaca por la normal, para que se pueda leer la FBF y el proceso no sea robotizado.
    \end{obs}

    Definiremos ahora más conectivas lógicas para poder trabajar más cómodamente.

    \begin{mydef}
        Se definirán tres conectivas lógicos adicionales.
        \begin{enumerate}
            \item Se define la \textbf{disyunción} $\varphi\lor\psi$ como $\Rightarrow\neg\psi\varphi$ (en notación Polaca).
            \item Se define la \textbf{conjunción} $\varphi\land\psi$ como $\neg(\neg\psi\lor\neg\varphi)$.
            \item Se define el \textbf{si sólo si} $\psi\iff\varphi$ como $(\psi\Rightarrow\varphi)\land(\varphi\Rightarrow\psi)$.
        \end{enumerate}
    \end{mydef}

    \section{Modeos o Estructuras}

    En el fondo, queremos que las FBF sean cosas verdaderas o falsas. Un Modelo o Estructura es algo que le va a dar significado a las FBF. De alguna manera va a ser una forma de asignarle el valor de verdadero o falso a cada una de las variables.

    \begin{mydef}
        Un \textbf{Modelo o Estructura} de la lógica proposicional es una función $\cf{m}{\textup{Var}}{\left\{V,F\right\}}$, donde Var denota al conjunto de símbolos que son variables. Básicamente estamos diciendo que hay variables que son verdaderas y otras que son falsas.
    \end{mydef}

    \begin{theor}
        Para todo modelo $m$, existe una única extensión $\cf{\overline{m}}{\textup{FBF}}{\left\{V,F\right\}}$, donde FBF denota al conjunto de las fórmulas bien formadas, tal que $\overline{m}(\neg\varphi)=V\iff \overline{m}(\varphi)=F$ y $\overline{m}(\neg\varphi\psi)=F\iff\overline{m}(\varphi)=V$ y $\overline{m}(\psi)=F$.
    \end{theor}

    \begin{mydef}
        Sea $m$ un modelo, $\varphi$ una fórmula y $\Sigma$ un cojunto de fórmulas. Definimos que
        \begin{enumerate}
            \item $m\vDash \varphi$ ($m$ satisface $\varphi$) si $\overline{m}(\varphi)=V$.
            \item $m\vDash \Sigma$ si $m\vDash\varphi$ para cada $\varphi$ elemento de $\Sigma$.
        \end{enumerate}
    \end{mydef}

    \begin{exa}
        Sea $m$ un modelo tal que $m(p_1)=V$ y $m(p_i)=F$, para todo $i\geq2$. En este caso $m\nvDash \Rightarrow p_1p_3$, pero $m\vDash \neg p_5$.
    \end{exa}
    
    \begin{mydef}
        Decimos que una fórmula $\varphi$ es:
        \begin{enumerate}
            \item \textbf{Satisfacible} si existe un modelo $m$ tal que $m\vDash\varphi$.
            \item \textbf{Contradictoria} si todo modelo cumple que $m\nvDash\varphi$.
            \item \textbf{Una tautología} si todo modelo $m$ cumple que $m\vDash\varphi$.
        \end{enumerate}
    \end{mydef}

    \begin{exa}
        Tomemos de ejemplo a $\Rightarrow p_1 p_2$. cualquier modelo que haga a $p_1$ y $p_3$ verdaderas, o ambas falsas satisfacen la FBF, $p_1$, $\neg\Rightarrow p_1 p_3$ o $\neg(p_1\Rightarrow\neg p_1)$. Por lo cual, esta fórmula es satisfacible.

        En cambio, $\neg(p_1\Rightarrow p_1)$ es contradictoria y, por ende $p_1\Rightarrow p_1$ y $\neg p_1\Rightarrow\neg p_1$ son tautologías.
    \end{exa}

    \begin{mydef}
        Sea $\Sigma$ un conjunto de fórmulas. Decimos que $\Sigma$ es
        \begin{enumerate}
            \item \textbf{Satisfacible} si existe un modelo $m$ tal que $m\vDash\Sigma$.
            \item \textbf{Contradictoria} si todo modelo cumple que $m\nvDash\Sigma$.
            \item \textbf{Una tautología} si todo modelo $m$ cumple que $m\vDash\Sigma$.
        \end{enumerate}
    \end{mydef}

    \begin{exa}
        El conjunto de fórmulas $\Sigma=\left\{\Rightarrow p_1p_2, p_1,\neg p_2 \right\}$ no es satisfacible (en este caso, es contradictorio).
    \end{exa}

    \begin{obs}
        Se tiene lo siguiente:
        \begin{enumerate}
            \item Una tautología $\Rightarrow$ satisfacible.
            \item $\varphi$ es satisfacible $\iff$ $\neg\varphi$ es una contradicción.
            \item Satisfacible es lo mismo que no contradictoria.
        \end{enumerate}
    \end{obs}

    \begin{mydef}
        Si $\Sigma$ es un conjunto de FBF y $\varphi$ es alguna otra fórmula, entonces decimos que $\varphi$ es \textbf{consecuencia lógica} de $\Sigma$, o que $\Sigma$ \textbf{implica lógicamente} a $\varphi$, escrito como $\Sigma\vDash\varphi$, si para todo modelo $m$ tal que $m\vDash\Sigma$ se tiene que $m\vDash\varphi$.
    \end{mydef}

    \begin{exa}
        El conjunto  de FBF $\left\{\Rightarrow p_1 p_2, p_1\right\}\vDash p_2$.
    \end{exa}

    \begin{obs}
        \begin{enumerate}
            Se tiene lo siguiente:
            \item Un conjunto de FBF $\Sigma\nvDash\varphi$ si y sólo si $\Sigma\cup\left\{\neg\varphi \right\}$ es satisfacible.
            \item Además, un conjunto de FBF $\Sigma\vDash\varphi$ si y sólo si $\Sigma\cup\left\{\neg\varphi \right\}$ no es satisfacible.
        \end{enumerate}
    \end{obs}

    \begin{lema}
        Sea $\Sigma$ un conjunto de fórmulas y sean Var$(\Sigma)$ el conjunto de las variables $p_i$ que aparecen en las fórmulas de $\Sigma$. Si $m_1$ y $m_2$ son dos modelos tales que
        \begin{equation*}
            m_1|_{\textup{Var}(\Sigma)}=m_2|_{\textup{Var}(\Sigma)}
        \end{equation*}
        entonces, $\overline{m_1}|_{\Sigma}=\overline{m_2}|_{\Sigma}$. En particular, para cada fórmula $\varphi$ que sea elemento de $\Sigma$, entonces $m_1\vDash\varphi$ si y sólo si $m_2\vDash\varphi$, más aún $m_1\vDash\Sigma$ si y sólo si $m_2\vDash\Sigma$.
    \end{lema}

    \begin{proof}
        Sin pérdida de generalidad, $\Sigma$ es cerrado bajo subformulas.
        
        Procederemos por inducción sobre $\varphi\in\Sigma$, demostraremos que $\overline{m_1}(\varphi)=\overline{m_2}(\varphi)$.
        Si $\varphi$ coincide con algún $p_i$, entonces $p_i\in\textup{Var}(\Sigma)$ y, por tanto
        \begin{equation*}
            \overline{m_1}(p_i)=m_1(p_i)=m_2(p_i)=\overline{m_2}(p_i)
        \end{equation*}
        Ahora hacemos el paso inductivo. 
        \begin{enumerate}
            \item Tenemos el caso en que $\varphi$ es de la forma $\neg\psi$ y suponemos que $\overline{m_1}(\psi)=\overline{m_2}(\psi)$. Se tiene que $\overline{m_1}(\neg \psi)=F\iff \overline{m_1}(\psi) = V \iff\overline{m_2}(\psi)=V\iff\overline{m_2}(\neg \psi)=F$. Por lo tanto, $\overline{m_1}(\psi)=\overline{m_2}(\psi)$. El caso en que sea verdadero es análogo.
            
            \item Tenemos el caso en que $\varphi$ es de la forma $\Rightarrow\varphi_1\psi$ y, supontemos que $\overline{m_1}(\varphi_1)=\overline{m_2}(\varphi_1)$ y $\overline{m_1}(\psi)=\overline{m_2}(\psi)$. Se tiene que $\overline{m_1}(\Rightarrow\varphi_1\psi)=F\iff\overline{m_1}(\varphi_1)=V$ y $\overline{m_1}(\psi)=F\iff$ (por hipótesis de inducción) $\overline{m_2}(\varphi_1)=V$ y $\overline{m_2}(\psi)=F\iff\overline{m_2}(\Rightarrow\varphi_1\psi)=F$. El caso en que sean verdaderas es análogo. Por tanto, $\overline{m_1}(\Rightarrow\varphi_1\psi)=\overline{m_2}(\Rightarrow\varphi_1\psi)$.
        \end{enumerate}
        Lo cual completa el paso inductivo.
    \end{proof}

    \begin{cor}
        Si $\Sigma$ es un conjunto finito de fórmulas, entonces se puede verificar 'Mecánicamente' si es el caso, que $\Sigma\vDash\varphi$.
    \end{cor}

    El procedimiento para verificar el modelo, se hace mediante la tabla de verdad de las variables y las FBF de $\Sigma$.

    \begin{mydef}
        Decimos que un conjunto de fórmulas bien formadas $\Sigma$ es \textbf{finitamente satisfacible} si cualquier subconjunto finito $\Delta\subseteq\Sigma$ es satisfacible.
    \end{mydef}

    \begin{theor}[Teorema de Compacidad de Gödel]
        Si $\Sigma$ es un conjunto (arbitrario) de fórmulas tal que $\Sigma\vDash\varphi$, entonces existe un $\Delta\subseteq\Sigma$ finito tal que $\Delta\vDash\varphi$.
    \end{theor}

    El teorema que Gödel probó originalmente fue este:

    \begin{theor}[Teorema de Gödel]
        Un conjunto de fórmulas $\Sigma$ es satisfacible si y sólo si es finitamente satisfacible.
    \end{theor}

    Veamos por qué el teorema de Gödel implica el teorema de compacidad de Gödel. Se tiene que $\Sigma\nvDash\varphi\iff$ existe un modelo $m$ tal que $m\vDash\Sigma\cup\left\{\neg\varphi \right\}$. Es decir, si y sólo si $\Sigma\cup\left\{\neg\varphi \right\}$ es satisfacible, es decir que es finitamente satisfacible (por el teorema de Gödel), es decir que para todo $\Delta\subseteq\Sigma$ finito se cumple que
    \begin{equation*}
        \Delta\cup\left\{\neg\varphi \right\}
    \end{equation*}
    es satisfacible. Y esto sucede si y sólo si para todo $\Delta\subseteq\Sigma$ finito existe $m$ tal que $m\vDash\Delta\cup\left\{\neg\varphi \right\}$, si y sólo si para todo $\Delta\subseteq\Sigma$ finito $\Delta\nvDash\varphi$, con lo cual
    \begin{equation*}
        \Sigma\nvDash\varphi\iff\Delta\nvDash\varphi
    \end{equation*}
    para todo $\Delta\subseteq\Sigma$ finito, que es el teorema de compacidad en su forma contrapositiva.

    \begin{lema}
        Sea $\Sigma$ un conjunto finitamente satisfacible, y sea $\varphi$ cualquier fórmula, entonces o bien $\Sigma\cup\left\{\varphi\right\}$ es finitamente satisfacible o $\Sigma\cup\left\{\neg\varphi\right\}$ lo es.
    \end{lema}

    \begin{proof}
        Supongamos que no, es decir que tanto $\Sigma\cup\left\{\varphi\right\}$ como $\Sigma\cup\left\{\neg\varphi\right\}$ no son finitamente satisfacibles, por lo cual existen $\Delta_1,\Delta_2\subseteq\Sigma$ finitos tales que $\Delta_1\cup\left\{\varphi \right\}$ y $\Delta_2\cup\left\{\neg\varphi \right\}$ no son satisfacibles.
        Entonces $\Delta_1\cup\Delta_2$ no puede ser satisfacible, pues si $m$ es un modelo tal que $m\vDash\Delta_1\cup\Delta_2$, entonces $m\vDash\varphi$ contradice el hecho de que $\Delta_1\cup\left\{\varphi \right\}$ es no satisfacible y si $m\vDash\neg\varphi$ contradice el hecho de que $\Delta_2\cup\left\{\neg\varphi \right\}$ no es satisfacible, siendo $\Delta_1\cup\Delta_2\subseteq\Sigma$, se contradice el hecho de que $\Sigma$ es finitamente satisfacible$\#_c$. Luego se tiene el resultado.
    \end{proof}

    Ahora procederemos a probar el teorema de Gödel. 

    \begin{proof}
        Se probará la doble implicación:
        
        $\Rightarrow)$: Es inmediato.

        $\Leftarrow)$: Sean $\varphi_1,\varphi_2,...$ una enumeración 'efectiva' de todas las fórmulas (checar la observación). Recursivamente, definimos conjuntos de fórmulas $\Sigma_0\subseteq\Sigma_1\subseteq\cdots$ tales que $\Sigma_0=\Sigma$, y
        \begin{enumerate}
            \item Cada $\Sigma_n$ es finitamente satisfacible.
            \item Para cada $n\in\mathbb{N}$, o bien $\varphi_n\in\Sigma_{n+1}$ o bien $\neg\varphi_n\in\Sigma_{n+1}$
        \end{enumerate}
        en este contexto, definimos:
        \begin{equation*}
            \Sigma_{n+1}=\left\{\begin{array}{lr}
                    \Sigma_n\cup\left\{\varphi_n \right\} & \textup{si este conjunto es finitamente satisfacible}\\
                    \Sigma_n\cup\left\{\neg \varphi_n \right\} & \textup{en caso contrario}\\
                \end{array}
            \right.
        \end{equation*}
        Esta definición es consistente con la recursión por el lema anterior.

        Ahora, definimos $\Sigma_\infty=\bigcup_{n\in\mathbb{N}}\Sigma_n$. Analicemos a este conjunto.

        \begin{enumerate}
            \item \textbf{$\Sigma_\infty$ es finitamente satisfacible}. En efecto, sea $\Delta\subseteq\Sigma$ un subconjunto finito, entonces existe $n\in\mathbb{N}$ tal que $\Delta\subseteq\Sigma_n$, luego como $\Sigma_n$ es finitamente satisfacible, $\Delta$ es satisfacible. Por lo cual $\Sigma_\infty$ es finitamente satisfacible.
            \item \textbf{Para cada fórmula $\psi$ o bien $\psi\in\Sigma_\infty$ ó $\neg\psi\in\Sigma_\infty$ y no ambas.} Esto es inmediato con la enumeración efectiva de todas las fórmulas bien formadas.
            \item \textbf{$\Sigma_\infty$ es maximal finitamente satisfacible}.
        \end{enumerate}

        Sea $\cf{m}{\textup{Var}(\Sigma_\infty)}{\left\{V,F\right\}}$, dado por $m(p_n)=V$ si y sólo si $p_n\in \Sigma_\infty$.

        Se probará el siguiente lema:

        \begin{lema}
            Para cualquier fórmula $\psi$, $\overline{m}(\psi)=V$ si y sólo si $\psi\in\Sigma_\infty$ y $\overline{m}(\psi)=F$ si y sólo si $\neg\psi\in\Sigma_\infty$.
        \end{lema}

        \begin{proof}
            Procederemos por inducción sobre $\psi$.
            \begin{itemize}
                \item El caso base es inmediato por definición.
                \item $\overline{m}(\neg\psi)=V\iff\overline{m}(\psi)=F\iff\psi\notin\Sigma_\infty\iff\neg\psi\in\Sigma_\infty$.
                \item $\overline{m}(\Rightarrow\xi\psi)=F\iff\overline{m}(\xi)=F$ y $\overline{m}(\psi)=V\iff\neg\xi,\psi\in\Sigma_\infty$ si y sólo si $\Rightarrow \psi\xi\notin\Sigma_\infty$ (esto es cierto por la maximalidad de $\Sigma_\infty$ al ser finitamente satisfacible).
            \end{itemize}
            por inducción se tiene lo deseado.
        \end{proof}

        En conclusión, el modelo definido cumple que $m\vDash\psi$ si y sólo si $\psi\in\Sigma_\infty$. En particular, $m\vDash\Sigma$, y $\Sigma$ es satisfacible.

    \end{proof}

    \begin{obs}
        Tuplas. Considere los números naturales. Podemos establecer una biyección entre las tuplas finitas de números naturales junto con el cero, y los números naturales, de esta forma:

        Si $n\in\mathbb{N}$, por el TFA podemos expresar a $n=q_1^{\alpha_1}\cdot...\cdot q_m^{\alpha_m}$. Establecemos la biyección dada como sigue: $n\mapsto(\alpha_1,...,\alpha_{m-1},\alpha_{m}-1)$. De esta forma podemos enumerar algo con tuplas. Lo que Gödel hace es que hace ciertas asignaciones: $\neg=0$, $\Rightarrow=1$, $2=p_1$, $3=p_2$, etc... Esta enumeración es llamada \textbf{enumeración de Gödel}.

        Cuando decimos lo de enumeración, nos referimos a esto. Básicamente enumeramos a todas las fórmulas bien formadas. Cuando decimos que la enumeración es efectiva, hacemos referencia a que podemos hacerlo de forma mecánica.
    \end{obs}

    \section{Lista de Axiomas Lógicos}    

    \begin{mydef}[Axiomas Lógicos]
        Se tienen los siguientes axiomas. Cualquier fórmula que caiga en alguno de los siguientes casos.
        \begin{enumerate}
            \item $\varphi\Rightarrow (\psi\Rightarrow \varphi)$.
            \item $\varphi\Rightarrow ((\psi\Rightarrow\neg\varphi)\Rightarrow \neg\psi)$.
            \item $\varphi\Rightarrow\varphi'$ siempre que $\varphi'$ sea el resultado de sustituir una subfórmula de la forma $\neg\neg\psi$ por $\psi$, o viceversa.
            \item $\varphi\Rightarrow \varphi[\psi\Rightarrow\xi \leftrightsquigarrow\neg\xi\Rightarrow \neg\psi]$.
            \item $\varphi\Rightarrow\varphi[\neg\psi\Rightarrow\psi\leftrightsquigarrow\psi]$.
            \item $(\varphi\Rightarrow(\xi\Rightarrow\psi))\Rightarrow((\varphi\Rightarrow\xi)\Rightarrow(\varphi\Rightarrow\psi))$.
        \end{enumerate}
        Junto con una única regla de inferencia, llamada \textbf{Modus Ponens}, la cual consiste en que
        \begin{center}
            \begin{tabular}{c c c}
                $\varphi$ & $\Rightarrow$ & $\psi$ \\
                $\varphi$ &  &  \\
                \hline
                 & $\therefore$ & $\psi$ \\
            \end{tabular}
        \end{center}
    \end{mydef}

    Un ejemplo de 3. sería que $(p_1\Rightarrow p_2)\Rightarrow(p_1\Rightarrow \neg\neg p_2)$. Cuando ponemos $[.]$ al lado de una fórmula, nos referimos a cualquier subfórmula interna dentro de la original. Cuando ponemos $\leftrightsquigarrow$ es que podemos sustituir uno por otro.

    \begin{mydef}
        Sea $\Gamma$ un conjunto de fórmulas,y sea $\varphi$ una fórmula.
        \begin{enumerate}
            \item Una \textbf{demostración} de $\varphi$ a partifr de $\Gamma$ es una sucesión finita de fórmulas $\left(\varphi_1,...,\varphi_n\right)$ tales que, para cada $i$ se cumple una de las siguientes:
            \begin{enumerate}
                \item $\varphi_i$ es un axioma lógico.
                \item $\varphi_i$ es un elemento de $\Gamma$.
                \item Existen $j,k<i$ tales que: $\varphi_j$ es la fórmula $\varphi_k\Rightarrow\varphi_i$.
            \end{enumerate}
            \item $\varphi$ es \textbf{demostrable a partir de $\Gamma$}, o bien $\varphi$ es un \textbf{teorema de $\Gamma$}, si existe una demostración de $\varphi$ a partir de $\Gamma$. Esto se simboliza por $\Gamma\vdash\varphi$. 
        \end{enumerate}
    \end{mydef}

    \begin{obs}
        $\varphi\lor\psi$ es $\neg\varphi\Rightarrow\psi$, y $\varphi\land\psi$ es $\neg(\psi\Rightarrow\neg \varphi)$.
        $\varphi\iff\psi$ es $(\varphi\Rightarrow \psi)\land( \psi\Rightarrow\varphi)$
    \end{obs}

    \begin{exa}
        Se cumple que $\left\{\neg C,A\Rightarrow C, A\lor(B\Rightarrow C),\neg C\Rightarrow (C\Rightarrow E), B \right\}\vdash E$. Probemos que esto es cierto:
        \begin{center}
            \begin{tabular}{l l c l r}
                1) & $(A\Rightarrow C)$ & $\Rightarrow$ & $(\neg C\Rightarrow\neg A)$ & Ax. 4 \\
                2) & $A$ & $\Rightarrow$ & $C$ & Premisa \\
                3) & $\neg C$ & $\Rightarrow$ & $\neg A$& Modus ponens \\
                4) & $\neg C$ &  &  & Premisa\\
                5) & $\neg A$ &  &  & 3,4 Modus ponens\\
                6) & $\neg A$ & $\Rightarrow$ & $(B\Rightarrow C)$ & Premisa\\
                7) & $B$ & $\Rightarrow$ & $C$ & 6,5 Modus ponens\\
                8) & $B$ &  &  & Premisa\\
                9) & $C$ &  &  & 7,8 Modus ponens\\
                10) & $\neg C$ & $\Rightarrow$ & $(C\Rightarrow E)$ & Premisa\\
                11) & $C$ & $\Rightarrow$ & $E$  & 10,4 Modus ponens\\
                12) & $E$ &  &  & 11,9 Modus ponens\\
                \hline
                & & $\therefore$ & $E$ & \\
            \end{tabular}
        \end{center}
    \end{exa}

    \begin{exa}
        $\left\{\varphi\land\psi \right\}\vdash\varphi$. En efecto:
        \begin{center}
            \begin{tabular}{l l c l r}
                1) & $\neg(\psi\Rightarrow\neg\varphi)$ &  &  & Premisa \\
                2) & $\neg\varphi$ & $\Rightarrow$ & $\psi\Rightarrow\neg\varphi$ & Ax. 1 \\
                3) & $(\neg\varphi\Rightarrow(\psi\Rightarrow\neg\varphi))$ & $\Rightarrow$ & $(\neg(\psi\Rightarrow\neg\varphi)\Rightarrow\neg\neg\varphi)$ & Ax. 4 \\
                4) & $\neg(\psi\Rightarrow\neg\varphi)$ & $\Rightarrow$ & $\neg\neg\varphi$ & 3,2 M.P. \\
                5) & $\neg\neg\varphi$ &  &  & 4,1 M.P. \\
                6) & $\neg\neg\varphi$ & $\Rightarrow$ & $\varphi$ & Ax. 3\\
                7) & $\varphi$ &  &  & 6,5 M.P.\\
                \hline
                & & $\therefore$ & $\varphi$ & \\
            \end{tabular}
        \end{center}
        esta demostración es llamada \textbf{simplificación}.
    \end{exa}

    Hay varias demostraciones que son de utilidad. Como las siguientes:
    \begin{excer}
        Pruebe lo siguiente: 
        \begin{enumerate}
            \item $\left\{\varphi\Rightarrow \psi,\neg\psi \right\}\vdash\neg\varphi$ (llamada \textbf{Modus Tollens}).
            \item $\left\{\varphi\right\}\vdash\varphi\lor\psi$ (llamada \textbf{Adición}).
            \item $\left\{\varphi\lor\psi,\neg\varphi \right\}\vdash\psi$ (llamada \textbf{Silogismo Disyuntivo}).
            \item $\left\{\varphi,\psi\right\}\vdash \varphi\land\psi$ (llamada \textbf{Conjunción}).
            \item $\left\{\varphi\Rightarrow\psi \right\}\vdash\neg\psi\Rightarrow\neg\varphi$. (llamada \textbf{Transposición}).
        \end{enumerate}
    \end{excer}

    \begin{proof}
        Probemos cada inciso.
        
        De (1): 
        \begin{center}
            \begin{tabular}{l l c l r}
                1) & $\varphi\Rightarrow\psi$ & & & Premisa \\
                2) & $(\varphi\Rightarrow \psi)$ & $\Rightarrow$ & $(\neg \psi\Rightarrow\neg\psi)$ & Ax. 4 \\
                3) & $\neg\psi$ & $\Rightarrow$ & $\neg\psi$ & 2,1 M.P. \\
                4) & $\neg\psi$ &  &  & Premisa \\
                5) & $\neg\varphi$ &  &  & 3,4 M.P. \\
                \hline
                & & $\therefore$ & $\neg\varphi$ & \\
            \end{tabular}
        \end{center}

        De (2): 

        \begin{center}
            \begin{tabular}{l l c l r}
                1) & $\varphi$ & & & Premisa \\
                2) & $\varphi$ & $\Rightarrow$ & $(\neg\psi\Rightarrow\varphi)$ & Ax. 1 \\
                3) & $\neg\psi\Rightarrow\varphi$ & & & 2,1 M.P. \\
                4) & $\neg\psi\Rightarrow\varphi$ & $\Rightarrow$ & $\neg\varphi\Rightarrow\neg\neg\psi$ & Ax.4. \\
                5) & $\neg\varphi\Rightarrow\neg\neg\psi$ & & & 4,3 M.P. \\
                6) & $\neg\varphi\Rightarrow\neg\neg\psi$ & $\Rightarrow$ & $\neg\varphi\Rightarrow\psi$ & Ax. 3 \\
                7) & $\neg\varphi\Rightarrow\psi$ & & & 6,5 M.P. \\
                8) & $\varphi\lor\psi$ & & & 7) \\
                \hline
                & & $\therefore$ & $\varphi\lor\psi$ & \\
            \end{tabular}
        \end{center}

        De (3):
        \begin{center}
            \begin{tabular}{l l c l r}
                1) & $\varphi\lor\psi$ & & & Premisa \\
                2) & $\neg\varphi$ & $\Rightarrow$ & $\psi$ & 1) \\
                3) & $\neg\varphi$ & & & Premisa\\
                4) & $\psi$ & & & 2,3 M.P.\\
                \hline
                & & $\therefore$ & $\psi$ & \\
            \end{tabular}
        \end{center}

        De (4):
        \begin{center}
            \begin{tabular}{l l c l r}
                1) & $\varphi$ &  &  & Premisa \\
                2) & $\psi$ &  &  & Premisa \\
                3) & $\psi$ & $\Rightarrow$ & $((\psi\Rightarrow\neg\varphi)\Rightarrow\neg\psi)$ & Ax. 2 \\
                4) & $(\psi\Rightarrow\neg\varphi)$ & $\Rightarrow$ & $\neg\psi$ & 1,3 M.P. \\
                5) & $\psi$ & $\Rightarrow$ & $\neg\neg\psi$ & Ax. 3 \\
                6) & $\neg\neg\psi$ &  &  & 2,5 M.P. \\
                7) & $\neg(\psi$ & $\Rightarrow$ & $\neg\varphi)$ & 4,6 M.T. \\
                8) & $\varphi$ & $\land$ & $\psi$ & 7) \\
                \hline
                & & $\therefore$ & $\varphi\land\psi$  & \\
            \end{tabular}
        \end{center}

    \end{proof}

    \begin{excer}
        Demuestre que existe una demostración de lo siguiente:
        \begin{enumerate}
            \item $\left\{F\lor(G\lor H),(G\lor H)\Rightarrow(I\lor J),(I\lor J)\Rightarrow(F\lor H),\neg F \right\}\vdash H$.
            \item $\left\{Q\Rightarrow(R\Rightarrow S), (R\Rightarrow S)\Rightarrow T, (S\lor U)\Rightarrow \neg V,\neg V\Rightarrow(R\iff \neg W),\neg T,\neg(R\iff \neg W) \right\}\vdash \neg Q\land\neg(S\lor U)$.
            \item $\left\{A\Rightarrow B, C\Rightarrow D, \neg B\lor\neg D,\neg\neg A, (E\land F)\Rightarrow C \right\}\vdash \neg(E\land F)$.
            \item $\left\{E\Rightarrow (F\land \neg G),(F\lor G)\Rightarrow H, E \right\}\vdash H$.
            \item $\left\{J\Rightarrow K, J\lor(L\lor\neg L),\neg K \right\}\vdash\neg L\land \neg K$.
            \item $\left\{(R\Rightarrow\neg S)\land(T\Rightarrow\neg U),(V\Rightarrow\neg W)\land(X\Rightarrow\neg Y),(T\Rightarrow W)\land(U\Rightarrow S), V, R \right\}\vdash \neg T\land\neg U$.
        \end{enumerate}
    \end{excer}

    \begin{proof}
        De (1):
        \begin{center}
            \begin{tabular}{l l c l r}
                1) & $F$ & $\lor$ & $(G\lor H)$ & Premisa \\
                2) & $G$ & $\lor$ & $H$ & Premisa \\
                3) & $(G\lor H)$ & $\Rightarrow$ & $(I\lor J)$ & Premisa \\
                4) & $(I\lor J)$ & $\Rightarrow$ & $(F\lor H)$ & Premisa \\
                5) & $F$ & $\lor$ & $H$ & Premisa \\
                6) & $\neg F$ &  &  & Premisa \\
                7) & $G$ & $\lor$ & $H$ & 1,6 S.D. \\
                8) & $I$ & $\lor$ & $J$ & 3,7 M.P. \\
                9) & $F$ & $\lor$ & $H$ & 4,8 M.P. \\
                10) & $H$ &  &  & 9,6 S.D. \\
                \hline
                & & $\therefore$ & $H$  & \\
            \end{tabular}
        \end{center}

        De (2):
        \begin{center}
            \begin{tabular}{l l c l r}
                1) & $Q$ & $\Rightarrow$ & $(R\Rightarrow S)$ & Premisa \\
                2) & $(R\Rightarrow S)$ & $\Rightarrow$ & $T$ & Premisa \\
                3) & $(S\lor U)$ & $\Rightarrow$ & $\neg V$ & Premisa \\
                4) & $\neg V$ & $\Rightarrow$ & $(R\iff \neg W)$ & Premisa \\
                5) & $\neg T$ &  &  & Premisa \\
                6) & $\neg(R$ & $\iff$ & $\neg W)$ & Premisa \\
                7) & $\neg\neg V$ &  &  & 4,6 M.T. \\
                8) & $\neg\neg V$ & $\Rightarrow$  & $V$ & Ax. 3. \\
                9) & $V$ &  &  & 8,7 M.P. \\
                10) & $\neg(S\lor U)$ &  &  & 3,9 M.T. \\
                11) & $\neg(R\Rightarrow S)$ &  &  & 2,5 M.T. \\
                12) & $\neg Q$ &  &  & 1,11 M.T. \\
                13) & $\neg Q$ & $\land$ & $\neg(S\lor U)$ & 12,10 Conj.\\
                \hline
                & & $\therefore$ & $\neg Q\land\neg(S\lor U)$  & \\
            \end{tabular}
        \end{center}

        De (3):
        \begin{center}
            \begin{tabular}{l l c l r}
                1) & $A$ & $\Rightarrow$ & $B$ & Premisa \\
                2) & $C$ & $\Rightarrow$ & $D$ & Premisa \\
                3) & $\neg B$ & $\lor$ & $\neg D$ & Premisa \\
                4) & $\neg\neg A$ &  &  & Premisa \\
                5) & $(E\land F)$ & $\Rightarrow$ & $C$ & Premisa \\
                6) & $\neg\neg A$ & $\Rightarrow$ & $A$ & Ax. 3 \\
                7) & $A$ &  &  & 6,4 M.P. \\
                8) & $B$ &  &  & 1,7 M.P. \\
                9) & $B$ & $\Rightarrow$ & $\neg\neg B$ & Ax. 3 \\
                10) & $\neg\neg B$ &  &  & 8,9 M.P. \\
                11) & $\neg D$ &  &  &  3,9 S.D.\\
                12) & $\neg C$ &  &  &  2,11 M.T.\\
                13) & $\neg(E\land F)$ &  &  &  5,12 M.T.\\
                \hline
                & & $\therefore$ & $\neg (E\lor F)$  & \\
            \end{tabular}
        \end{center}

        De (4):
        \begin{center}
            \begin{tabular}{l l c l r}
                1) & $E$ & $\Rightarrow$ & $(F\land \neg G)$ & Premisa \\
                2) & $(F\lor G)$ & $\Rightarrow$ & $H$ & Premisa \\
                3) & $E$ &  &  & Premisa \\
                4) & $F\and \neg G$ &  &  & 1,3 M.P. \\
                5) & $F$ &  &  & 4 Simp. \\
                6) & $F\lor G$ &  &  & 5 Ad. \\
                7) & $H$ &  &  & 2,6 M.P. \\
                \hline
                & & $\therefore$ & $H$  & \\
            \end{tabular}
        \end{center}

        De (5):
        \begin{center}
            \begin{tabular}{l l c l r}
                1) & $J$ & $\Rightarrow$ & $K$ & Premisa \\
                2) & $J$ & $\lor$ & $(K\lor\neg L)$ & Premisa \\
                3) & $\neg K$ &  &  & Premisa \\
                4) & $\neg J$ &  &  & 1,3 M.T. \\
                5) & $K$ & $\lor$ & $\neg L$ & 2,4 S.D.\\
                6) & $\neg L$ &  &  & 5,3 S.D.\\
                7) & $\neg L$ & $\land$ & $\neg K$ & 3,6 Conj.\\
                \hline
                & & $\therefore$ & $\neg L\land \neg K$  & \\
            \end{tabular}
        \end{center}

        De (6):
        \begin{center}
            \begin{tabular}{l l c l r}
                1) & $(R\Rightarrow\neg S)$ & $\land$ & $(T\Rightarrow\neg U)$ & Premisa \\
                2) & $(V\Rightarrow\neg W)$ & $\land$ & $(X\Rightarrow\neg Y)$ & Premisa \\
                3) & $(T\Rightarrow W)$ & $\land$ & $(U\Rightarrow S)$ & Premisa \\
                4) & $V$ &  &  & Premisa \\
                5) & $R$ &  &  & Premisa \\
                6) & $R$ & $\Rightarrow$ & $\neg S$ & 1 Simp. \\
                7) & $\neg S$ &  &  & 6,5 M.P. \\
                8) & $V$ & $\Rightarrow$ & $\neg W$ & 2 Simp. \\
                9) & $\neg W$ &  &  & 8,4 M.P. \\
                10) & $T$ & $\Rightarrow$ & $ W$ & 3 Simp. \\
                11) & $\neg W$ & $\Rightarrow$ & $\neg T$ & 10 Transp. \\
                12) & $\neg T$ &  &  & 11,9 M.P. \\
                13) & $U$ & $\Rightarrow$ & $S$ & 3 Simp. \\
                14) & $\neg S$ & $\Rightarrow$ & $\neg U$ & 13 Transp. \\
                15) & $\neg U$ &  &  & 14,7 M.P. \\
                15) & $\neg T$ & $\land$ & $\neg U$ & 12,15 Conj. \\
                \hline
                & & $\therefore$ & $\neg T\land \neg U$  & \\
            \end{tabular}
        \end{center}

    \end{proof}

    \begin{obs}[\textbf{Conmutatividad del $\land$ y $\lor$}]
        Es fácil de probar (teniendo en mente la definción) que:
        \begin{enumerate}
            \item $\varphi\Rightarrow\varphi[\xi\land\psi\leftrightsquigarrow \psi\land\xi]$.
            \item $\varphi\Rightarrow\varphi[\xi\lor\psi\leftrightsquigarrow \psi\lor\xi]$.
        \end{enumerate}
    \end{obs}

    \begin{proof}
        
    \end{proof}

    \begin{propo}[Leyes de Morgan]
        Se cumplen las siguiente (denominadas Leyes de Morgan):
        \begin{enumerate}
            \item $\neg(\xi\lor\psi)\iff \neg\xi\land\neg\psi$.
            \item $\neg(\xi\land\psi)\iff \neg\xi\lor\neg\psi$.
        \end{enumerate}
    \end{propo}

    \begin{proof}
        
    \end{proof}

    \begin{lema}
        $\emptyset\vdash\varphi\Rightarrow\varphi$. Es decir, que sin premisas es válido que $\varphi\Rightarrow\varphi$.
    \end{lema}

    \begin{proof}
        Veamos que:

        \begin{center}
            \begin{tabular}{l l c l r}
                1) & $\varphi$ & $\Rightarrow$ & $((\psi\Rightarrow\varphi)\Rightarrow\varphi)$ & Ax. 1 \\
                1) & $(\varphi\Rightarrow((\psi\Rightarrow\varphi)\Rightarrow\varphi))$ & $\Rightarrow$ & $((\varphi\Rightarrow(\psi\Rightarrow\varphi)\Rightarrow(\varphi\Rightarrow\varphi)))$ & Ax. 6 \\
                1) & $(\varphi\Rightarrow(\psi\Rightarrow\varphi))$ & $\Rightarrow$ & $(\varphi\Rightarrow\varphi)$ & Ax. 6 \\
                4) & $\varphi$ & $\Rightarrow$ & $(\psi\Rightarrow\varphi)$ & Ax. 1 \\
                4) & $\varphi$ & $\Rightarrow$ & $\varphi$ & 4,3 M.P. \\
                \hline
                & & $\therefore$ & $\varphi\Rightarrow\varphi$ & \\
            \end{tabular}
        \end{center}

        Lo cual termina la prueba

    \end{proof}

    \begin{theor}[Metateorema de Deducción]
        Sea $\Sigma$ un conjunto de fórmulas y $\varphi,\psi$ dos fórmulas. Entonces, $\Sigma\vdash(\varphi\Rightarrow\psi)$ si y sólo si $\Sigma\cup\left\{\varphi\right\}\vdash\psi$.
    \end{theor}

    \begin{proof}
        Probaremos las dos implicaciones:

        $\Rightarrow)$: Suponga que $\Sigma\vdash(\varphi\Rightarrow\psi)$, entonces en $\Sigma\cup\left\{\varphi\right\}$ como $\Sigma\vdash(\varphi\Rightarrow\psi)$ entonces por M.P. al tener que $\left\{\varphi\Rightarrow\psi,\varphi \right\}\vdash\psi$, se sigue que $\Sigma\cup\left\{\varphi\right\}\vdash\psi$.

        $\Leftarrow)$: Supongamos que $\Sigma\cup\left\{\varphi\right\}\vdash\psi$. La prueba se hará por inducción sobre la longitud de la demostración de $\psi$ a partir de $\Sigma\cup\left\{\varphi \right\}$.

        Sean $(\varphi_1,...,\varphi_n,\psi)$ la demostración, Entonces, la hipótesis inductiva es que: $\Sigma\vdash \varphi\Rightarrow\varphi_k$ para $k\in\left\{1,...,n\right\}$.

        Hay 4 casos para $\psi$:
        \begin{enumerate}
            \item $\psi\in\Sigma$. Se tiene que:
            \begin{center}
                \begin{tabular}{l l c l r}
                    1) & $\psi$ &  &  & Premisa \\
                    2) & $\psi$ & $\Rightarrow$ & $\varphi\Rightarrow\psi$ & Ax. 1 \\
                    3) & $\varphi$ & $\Rightarrow$ & $\psi$ & 2,1 M.P \\
                    \hline
                    & & $\therefore$ & $\varphi\Rightarrow\psi$ & \\
                \end{tabular}
            \end{center}

            \item $\psi=\varphi$) En este caso lo que se quiere probar es que $\Sigma\vdash(\varphi\Rightarrow\varphi)$. Para lo cual se usa el lema anterior se tiene el resultado de forma inmediata (tomando el conjunto vacío).
            
            \item $\psi$ es axioma lógico. Es inmediato pues si es un axioma lógico, siempre se tiene que $ \emptyset\vdash \varphi\Rightarrow\psi$. Luego, $\Sigma\vdash\varphi\Rightarrow\psi$.
            
            \item Algún $\varphi_i$ es $\varphi_k\Rightarrow\psi$. Como por inducción se tiene que $\varphi\Rightarrow\varphi_k$, en particular para $i$ se tiene que:
            
            \begin{center}
                \begin{tabular}{l l c l r}
                    1) & ... & ... & ... & Premisa \\
                    2) & $\varphi$ & $\Rightarrow$ & $\varphi_k$ & Ax. 1 \\
                    3) & $\varphi$ & $\Rightarrow$ & $(\varphi_k\Rightarrow\psi)$ & 2,1 M.P \\
                    3) & $(\varphi\Rightarrow(\varphi_k\Rightarrow\psi))$ & $\Rightarrow$ & $((\varphi\Rightarrow\varphi_k)\Rightarrow(\varphi\Rightarrow\psi))$ & Ax. 6 \\
                    3) & $(\varphi\Rightarrow\varphi_k)$ & $\Rightarrow$ & $(\varphi\Rightarrow\psi)$ & 3,2 M.P. \\
                    3) & $\varphi$ & $\Rightarrow$ & $\psi$ & 4,1 M.P. \\
                    \hline
                    & & $\therefore$ & $\varphi\Rightarrow\psi$ & \\
                \end{tabular}
            \end{center}

        \end{enumerate}

        Lo cual termina la demostración por inducción. Esto se abrevia con \textbf{P.C.}
    \end{proof}

    \begin{exa}
        Considere:

        \begin{center}
            \begin{tabular}{l l c l r}
                1) & $M$ & $\Rightarrow$ & $N$ & Premisa \\
                2) & $N$ & $\Rightarrow$ & $O$ & Premisa \\
                3) & $(M\Rightarrow O)$ & $\Rightarrow$ & $(N\Rightarrow P)$ & Premisa \\
                4) & $(M\Rightarrow P)$ & $\Rightarrow$ & $Q$ & Premisa \\
                5) & $M$ &  &  & Suposición \\
                6) & $N$ &  &  & 1,5 M.P. \\
                7) & $O$ &  &  & 2,6 M.P. \\
                8) & $M$ & $\Rightarrow$ & $O$ & 5-7 P.C. \\
                9) & $N$ & $\Rightarrow$ & $P$ & 3,8 M.P. \\
                10) & $M$ &  &  & Suposición. \\
                11) & $N$ &  &  & 1,10 M.P. \\
                12) & $P$ &  &  & 9,11 M.P. \\
                13) & $M$ & $\Rightarrow$ & $P$ & 10-12 P.C. \\
                \hline
                & & $\therefore$ & $M\Rightarrow P$ & \\
            \end{tabular}
        \end{center}

    \end{exa}

    \begin{excer}
        Complete la demostración:
        \begin{center}
            \begin{tabular}{l l c l r}
                1) & $V$ & $\Rightarrow$ & $W$ & Premisa \\
                2) & $X$ & $\Rightarrow$ & $Y$ & Premisa \\
                3) & $Z$ & $\Rightarrow$ & $W$ & Premisa \\
                4) & $X$ & $\Rightarrow$ & $A$ & Premisa \\
                5) & $W$ & $\Rightarrow$ & $X$ & Premisa \\
                6) & $(V\Rightarrow Y)\land(Z\Rightarrow A)$ & $\Rightarrow$ & $(V\lor Z)$ & Premisa \\
                7) & $V$ &  &  & Suposición \\
                8) & $W$ &  &  & 1,7 M.P. \\
                9) & $X$ &  &  & 5,8 M.P. \\
                10) & $Y$ &  &  & 2,9 M.P. \\
                11) & $V$ & $\Rightarrow$ & $Y$ & 7-10 P.C. \\
                12) & $Z$ &  &  & Suposición \\
                13) & $W$ &  &  & 3,12 M.P. \\
                14) & $X$ &  &  & 5,13 M.P. \\
                15) & $A$ &  &  & 4,14 M.P. \\
                16) & $Z$ & $\Rightarrow$ & $A$ & 12-15 P.C. \\
                17) & $(V\Rightarrow Y)$ & $\land$ & $(Z\Rightarrow A)$ & 11,16 Conj. \\
                \hline
                & & $\therefore$ & $Y\lor A$ & \\
            \end{tabular}
        \end{center}

    \end{excer}

    \begin{excer}
        Complete las demostraciones:
        
        1.
        \begin{center}
            \begin{tabular}{l l c l r}
                1) & $P$ & $\Rightarrow$ & $Q$ & Premisa \\
                2) & $Q$ & $\Rightarrow$ & $R$ & Premisa \\
                3) & $P$ &  &  & Suposición \\
                4) & $Q$ &  &  & 1,3 M.P. \\
                5) & $R$ &  &  & 2,4 M.P. \\
                6) & $P$ & $\Rightarrow$ & $R$ & 3-5 P.C. \\
                \hline
                & & $\therefore$ & $P\Rightarrow R$ & \\
            \end{tabular}
        \end{center}

        2.
        \begin{center}
            \begin{tabular}{l l c l r}
                1) & $Q$ &  &  & Premisa \\
                2) & $Q$ & $\Rightarrow$ & $(P\Rightarrow Q)$ & Ax. 2 \\
                3) & $P$ & $\Rightarrow$ & $Q$ & 2,1 M.P. \\
                \hline
                & & $\therefore$ & $P\Rightarrow Q$ & \\
            \end{tabular}
        \end{center}

        3.
        \begin{center}
            \begin{tabular}{l l c l r}
                1) & $P$ & $\Rightarrow$ & $(Q\Rightarrow R)$ & Premisa\\
                2) & $P$ &  &  & Suposición\\
                3) & $Q$ & $\Rightarrow$ & $R$ & 1,2 M.P.\\
                4) & $Q$ &  &  & Suposición\\
                5) & $R$ &  &  & 3,4 M.P.\\
                6) & $P$ & $\Rightarrow$ & $R$ & 2-5 P.C.\\
                7) & $Q$ & $\Rightarrow$ & $(P\Rightarrow R)$ & 6 Eje. 2\\
                \hline
                & & $\therefore$ & $Q\Rightarrow (P\Rightarrow R)$ & \\
            \end{tabular}
        \end{center}

        4.
        \begin{center}
            \begin{tabular}{l l c l r}
                1) & $P$ & $\Rightarrow$ & $(Q\land R)$ & Premisa \\
                2) & $P$ &  &  & Suposición \\
                3) & $Q\land R$ &  &  & 1,2 M.P. \\
                4) & $Q$ &  &  & 3 Simp. \\
                5) & $P$ & $\Rightarrow$ & $Q$ & 1-4 P.C. \\
                \hline
                & & $\therefore$ & $P\Rightarrow Q$ & \\
            \end{tabular}
        \end{center}

        5.
        \begin{center}
            \begin{tabular}{l l c l r}
                1) & $(P\Rightarrow Q)$ & $\land$ & $(C\Rightarrow D)$ & Premisa \\
                2) & $(Q\lor D)$ & $\Rightarrow$ & $((E\Rightarrow(E\lor F))\Rightarrow(P\land C))$ & Premisa \\
                3) & $P$ & $\Rightarrow$ & $Q$ & 1 Simp.\\
                4) & $C$ & $\Rightarrow$ & $D$ & 1 Simp.\\

                5) & $E$ &  &  & Suposición.\\
                6) & $E$ & $\lor$ & $F$ & 5 Ad.\\
                7) & $E$ & $\Rightarrow$ & $(E\lor F)$ & 5-6 P.C.\\

                8) & $P$ &  &  & Suposición.\\
                9) & $Q$ &  &  & 3,5 M.P.\\
                10) & $Q\lor D$ &  &  & 6 Ad.\\
                11) & $(E\Rightarrow(E\lor F))$ & $\Rightarrow$ & $(P\land C)$ & 7,2 M.P.\\
                \hline
                & & $\therefore$ & $P\iff R$ & \\
            \end{tabular}
        \end{center}
        
    \end{excer}

    \begin{exa}
        Este es un esquema general en el que se hacen las pruebas por contradicción:
        \begin{center}
            \begin{tabular}{l l c l r}
                1) & $P$ & $\lor$ & $(Q\land R)$ & Premisa \\
                2) & $P$ & $\Rightarrow$ & $R$ & Premisa \\
                3) & $\neg R$ &  &  & Suposición. \\
                4) & $\neg P$ &  &  & 2,3 M.T. \\
                5) & $Q$ & $\land$ & $R$ & 4,1 S.D. \\
                6) & $R$ &  &  & 5 Simp. \\
                7) & $R$ & $\land$ & $\neg R$ & 6,3 Ad. \\
                8) & $R$ &  &  & 3-7 P.I. \\

                \hline
                & & $\therefore$ & $R$ & \\
            \end{tabular}
        \end{center}
    \end{exa}

    \begin{excer}
        Complete las siguientes demostraciones:

        1.
        \begin{center}
            \begin{tabular}{l l c l r}
                1) & $(P\lor Q)$ & $\Rightarrow$ & $(R\Rightarrow D)$ & Premisa \\
                2) & $(\neg D\lor E)$ & $\Rightarrow$ & $(P\land R)$ & Premisa \\
                3) & $\neg D$ &  &  & Suposición \\
                4) & $\neg D$ & $\lor$ & $E$ & 3 Ad. \\
                5) & $P$ & $\land$ & $R$ & 2,4 M.P. \\
                6) & $P$ &  &  & 5 Simp. \\
                7) & $P$ & $\lor$ & $Q$ & 6 Ad. \\
                8) & $R$ & $\Rightarrow$ & $D$ & 1,7 M.P. \\
                9) & $R$ &  &  & 5 Simp. \\
                10) & $D$ &  &  & 8,9 M.P. \\
                11) & $D$ & $\land$ & $\neg D$ & 10, 3 Conj. \\
                12) & $\neg\neg D$ &  &  & 3-11 P.I. \\
                13) & $D$ &  &  & 12 Ax. 3 \\
                \hline
                & & $\therefore$ & $D$ & \\
            \end{tabular}
        \end{center}

        2.
        \begin{center}
            \begin{tabular}{l l c l r}
                1) & $(P\lor Q)$ & $\Rightarrow$ & $(R\land D)$ & Premisa \\
                2) & $(R\lor F)$ & $\Rightarrow$ & $(\neg F\land G)$ & Premisa \\
                3) & $(F\lor H)$ & $\Rightarrow$ & $(P\land I)$ & Premisa \\
                4) & $F$ &  &  & Suposición \\
                5) & $F$ & $\lor$ & $H$ & 4. Ad. \\
                6) & $P$ & $\land$ & $I$ & 3,5 M.P. \\
                7) & $P$ &  &  & 6 Simp. \\
                8) & $P$ & $\lor$ & $Q$ & 7 Ad. \\
                9) & $R$ & $\land$ & $D$ & 7 Ad. \\
                10) & $R$ &  &  & 9 Simp. \\
                11) & $R$ & $\lor$ & $ F$ & 10 Ad. \\
                12) & $\neg F$ & $\land$ & $G$ & 2,11 M.P. \\
                13) & $\neg F$ &  &  & 12 Simp. \\
                14) & $F$ & $\land$ & $\neg F$ & 4,13 Conj. \\
                15) & $\neg F$ &  &  & 4-14 P.I. \\
                \hline
                & & $\therefore$ & $\neg F$ & \\
            \end{tabular}
        \end{center}
    \end{excer}

    %Pasar notas martes%

    \begin{mydef}
        
    \end{mydef}

    \begin{theor}[\textbf{Teorema de Completud}]
        Cualquier conjunto de fórmulas $\Gamma$ que sea consistente, es satisfacible.
    \end{theor}

    \begin{proof}
        
    \end{proof}

    \begin{cor}
        Si $\Gamma$ es un conjunto de fórmulas, entonces $\Gamma\vDash\varphi$ implica $\Gamma\vdash\varphi$. 
    \end{cor}

    \begin{proof}
        
        Ya construimo sun conjunto $\Gamma_\infty$ con $\Gamma\subseteq\Gamma_\infty$ tal que:
        \begin{enumerate}
            \item $\Gamma_\infty$ es consistente.
            \item Para toda fórmula $\varphi$, o bien $\varphi\in\Gamma_\infty$ ó $\neg\varphi\in\Gamma_\infty$.
            \item $\varphi\in\Gamma_\infty$ si y sólo si $\Gamma_\infty\vdash\varphi$, y $\varphi\notin\Gamma_\infty$ si y sólo si $\neg\varphi\in\Gamma_\infty$, si y sólo si $\Gamma_\infty\nvdash\varphi$ si y sólo si $\Gamma_\infty\vdash\neg\varphi$.
        \end{enumerate}

        Definimos $\cf{M}{\textup{Var}}{\left\{V,F\right\}}$ de tal forma que $m(p_k)=V$ si y sólo si $p_k\in\Gamma_\infty$ (de forma análoga, $m(p_k)=F$ si y sólo si $p_k\notin\Gamma_\infty$).

        Afirmamos que para toda fórmula $\varphi$, se tiene que $\overline{m}(\varphi)=V$ si y sólo s i$\varphi\in\Gamma_\infty$. Procederemos por inducción sobre $\varphi$.
        \begin{enumerate}
            \item Si $\varphi$ es atómica, entonces se cumple por definición.
            \item Paso inductivo: supongamos que se cumple para $\varphi$ y $\psi$. Entonces,
            \begin{equation*}
                \begin{split}
                    \overline{m}(\neg\varphi)=V&\iff \overline{m}(\varphi)=F\\
                    &\iff \neg\varphi\in\Gamma\\
                \end{split}
            \end{equation*}
            además,
            \begin{equation*}
                \begin{split}
                    \overline{m}(\varphi\Rightarrow\psi)=F&\iff\overline{m}(\varphi)=V\textup{ y } \overline{m}(\psi)=F\\
                    &\iff \varphi\in\Gamma_\infty\textup{ y } \neg\psi\in\Gamma_\infty\\
                    &\iff \varphi\Rightarrow\psi\notin \Gamma_\infty\\
                \end{split}
            \end{equation*}
            probaremos una doble implicación.
            
            $\Rightarrow)$: Suponga que $\varphi,\neg\psi\in\Gamma_\infty$. Si $\varphi\Rightarrow\psi\in\Gamma_\infty$, entonces $\varphi\Rightarrow\psi\notin\Gamma_\infty$.

            $\Leftarrow)$: Suponga que $\varphi\Rightarrow\psi\notin\Gamma_\infty$, entonces $\neg(\varphi\Rightarrow\psi)\in\Gamma_\infty$, por lo cual $\neg(\varphi\Rightarrow\neg\neg \psi)\in\Gamma_\infty$, es decir que $\varphi\land\neg\psi\in\Gamma_\infty$, luego $\Gamma_\infty\vdash\varphi$ y $\Gamma_\infty\vdash\neg\psi$.
        \end{enumerate}
        por tanto, usando inducción se cumple que $m\vDash\Gamma_\infty$, en particular $m\vDash\Gamma$.
    \end{proof}

    \chapter{Lógica de primer orden}

    \section{Fundamentos}

    \begin{mydef}
        Un \textbf{lenguaje de primer orden} cuenta con un alfabeto que consta de lo siguiente:
        \begin{enumerate}
            \item \textit{Variables} (denotadas por $\textup{Var}$), denotadas por $v_1,v_2,...$ (a lo sumo una cantidad numerable).
            \item \textit{Conectivas lógicas} $\neg,\Rightarrow$.
            \item \textit{Símbolo de igualdad} $=$.
            \item \textit{Cuantificador} $\forall$, denominado \textbf{para todo}.
            \item \textit{Símbolos de predicado} (o \textit{Símbolo de relación}), $P_1,P_2,...$.
            \item \textit{Símbolos de función}, $f_1,F_2,...$.
            \item \textit{Símbolos de constante} $c_1,c_2,...$.
        \end{enumerate}
        los primeros cuatro son llamados \textbf{símbolos lógicos},y los últimos tres son llamados \textbf{símbolos no lógicos}. Puede que un lenguaje de primer orden no conste con alguno de los elementos de 5. a 7. o que conste de una cantidad finita. Cada uno de los 5. a 7. tiene asociada una \textbf{aridad} (que es un número entero).
    \end{mydef}

    Para que la idea quede más afianzada, se verán algunos ejemplos.

    \begin{exa}
        El lenguaje de la Teoría de Grupos consta de $\left\{*,(\cdot)^{-1}, e\right\}$ donde $*$ es una función binaria, $(\cdot)^{-1}$ es una función unaria y $e$ es una constante.
    \end{exa}

    \begin{exa}
        El lenguaje de la Teoría de Anillos consta de $\left\{\cdot,+,0,1 \right\}$ donde $\cdot$ y $+$ son función binaria, y $0,1$ son constantes.
    \end{exa}

    \begin{exa}
        El lenguaje de la Aritmética consta de $\left\{+,\cdot,s,<,1 \right\}$, donde $+,\cdot$ son funciones binarias, $s$ es una función unaria, $<$ es una relación binaria y $1$ es una constante.
    \end{exa}

    \begin{exa}
        El lenguaje de la Teoría de Conjuntos, consta de $\left\{\in \right\}$, la cual es una relación binaria.
    \end{exa}

    \begin{mydef}
        Definimos lo siguiente:
        \begin{enumerate}
            \item \textbf{Términos} son:
            \begin{enumerate}
                \item $v_i$ y $c_i$ son términos.
                \item Si $F_i$ es un símbolo de función $n$-aria, y $t_1,...,t_n$ son términos, entonces $F_i t_1\cdots t_n$ es un término (en notación polaca),
            \end{enumerate}
            \item \textbf{Fórmulas} son:
            \begin{enumerate}
                \item Si $t_1,t_2$ son términos, entonces $=t_1t_2$ es una fórmula.
                \item Si $R_i$ es un símbolo de relación de aridad $n$ y tengo $n$-términos, entonces $R_1t_1\cdots t_n$ es una fórmula.
                \item Si $\varphi,\psi$ son fórmulas y $v_i$ es una variable, entonces $\neg\varphi$, $\Rightarrow\varphi\psi$ y $\forall v_i\varphi$ son fórmulas.
            \end{enumerate}
        \end{enumerate}
    \end{mydef}

    \begin{exa}
        La asociatividad se puede escribir como la siguiente fórmula:
        \begin{equation*}
            \forall x\forall y\forall z=**xyz*x*yz
        \end{equation*}
        que básicamente es decir que:
        \begin{equation*}
            \forall x,y,z, (x*y)*z=x*(y*z)
        \end{equation*}
        en un grupo cualquiera.
    \end{exa}

    \section{Axiomas Lógicos}

    Cualquier generalización de 
    \begin{enumerate}
        \item Los de Lógica proposicional.
        \item $(\forall x)(\varphi\Rightarrow \psi)\Rightarrow (\forall x\varphi\Rightarrow\forall x\psi)$.
        \item $\varphi\Rightarrow\forall x\varphi$ si $x$ no es libre en $\varphi$.
        \item $x=x$.
        \item $z=y\Rightarrow(\varphi\Rightarrow\varphi\left[y/x\right])$ si $\varphi$ es atómica.
        \item $\forall x\varphi\Rightarrow\varphi\left[t/x \right]$ si $t$ es sustitubile por $x$.
    \end{enumerate}

    Reglas de inferencia: M.P.

    \begin{theor}[\textbf{Metateorema}]
        Se tiene lo siguiente:
        \begin{enumerate}
            \item \textbf{Instanciación universal}. Si $\Sigma\vdash(\forall x)\varphi$ entonces, $\Sigma\vdash\varphi[t/x]$ siempre que $t$ sea sustituible por $x$ en $\varphi$.
            \item \textbf{Generalización existencial}. Si $\Sigma\vdash\varphi[t/x]$ entonces, $\Sigma\vdash\left(\exists x \right)\varphi$ siempre que $t$ sea sustituible por $x$ en $\varphi$.
        \end{enumerate}
    \end{theor}

    \begin{proof}
        De (1): Como $\Sigma\vdash(\forall x)\varphi$, entonces existe una demostración que prueba $(\forall x)\varphi$. Por el axioma (5), se tiene que al ser $t$ sustituible: $\forall x\varphi\Rightarrow\varphi\left[t/x \right]$, luego existe una demostración que prueba a $\varphi\left[t/x \right]$, añadiendo esta línea al teorema anterior, se sigue que $\Sigma\vdash\varphi\left[t/x \right]$.

        De (2): Como $\Sigma\vdash\varphi[t/x]$, entonces existe una demostración que prueba $\Sigma\vdash\varphi[t/x]$. Procederemos por contradicción. Suponga que $\neg(\exists x)\varphi$, es decir $\neg(\exists x)\neg\neg\varphi$, luego $(\forall x)\neg\varphi$. Por (1), se sigue que $\neg\varphi[t/x]$, lo que es una contradicción del renglón de arriba.

        Luego, $(\exists x)\varphi$.
    \end{proof}

    \begin{excer}
        Demuestre que existe una demostración formal de válidez para lo siguiente:
        \begin{enumerate}
            \item $(\forall x)(Px\Rightarrow Qx)/\therefore Pc\Rightarrow((\forall y)(Qy\Rightarrow Sy)\Rightarrow Sc)$.
            \item $(\forall x)(Px\Rightarrow(\forall y)(Qy\Rightarrow Sy))/\therefore (\forall x)Px\Rightarrow(\forall y)(Qy\Rightarrow Sy)$.
            \item $(\exists x)Px\Rightarrow (\exists y)Qy/\therefore (\exists x)(Px\Rightarrow (\exists y)Qy)$.
        \end{enumerate}
    \end{excer}

    \begin{proof}
        De (1):
        \begin{center}
            \begin{tabular}{ c  l  l  c  }
                \hline
                No. &  &  &  \\
                \hline
                1) & $(\forall x)(Px\Rightarrow Qx)$ &  & Premisa \\
                2) & $Pc$ &  & Hipótesis \\
                3) & $(\forall y)(Qy\Rightarrow Sy)$ &  & Hipótesis \\
                4) & $Pc\Rightarrow Qc$ &  & 1 I.U. \\
                5) & $Qc\Rightarrow Sc$ &  & 3 I.U. \\
                6) & $Qc\Rightarrow Sc$ &  & 4,5, S.H. \\
                7) & $Sc$ &  & 6,2 M.P. \\
                8) & $(\forall y)(Qy\Rightarrow Sy)$ & $\Rightarrow Sc$ & 3-7 P.C. \\
                9) & $Pc\Rightarrow$ & $((\forall y)(Qy\Rightarrow Sy)\Rightarrow Sc)$ & 2-8 P.C. \\
                \hline
                  &  &  $\therefore Pc\Rightarrow ((\forall y)(Qy\Rightarrow Sy)\Rightarrow Sc)$ &  \\
            \end{tabular}
        \end{center}

        De (2):
        \begin{center}
            \begin{tabular}{ c  l  l  c  }
                \hline
                No. &  &  &  \\
                \hline
                1) & $(\forall x)(Px\Rightarrow(\forall y)(Qy\Rightarrow Sy))$ &  & Premisa \\
                2) & $(\forall x)Px$  &  & Hipótesis \\
                3) & $(\forall x)(Px\Rightarrow(\forall y)(Qy\Rightarrow Sy))$  & $\Rightarrow((\forall x) Px\Rightarrow (\forall x)((\forall y)(Qy\Rightarrow Sy)))$  & 1 Ax. 1 \\
                4) & $(\forall x) Px\Rightarrow (\forall x)((\forall y)(Qy\Rightarrow Sy))$ &  & 1,3 M.P.  \\
                5) & $(\forall x)((\forall y)(Qy\Rightarrow Sy))$ &  & 2,4 M.P.  \\
                6) & $(\forall y)(Qy\Rightarrow Sy)$ &  & 5 I.U. \\
                7) & $(\forall x)Px$ & $\Rightarrow((\forall y)(Qy\Rightarrow Sy))$ & 2-6 P.C. \\
                \hline
                  &  &  $\therefore (\forall x)Px\Rightarrow((\forall y)(Qy\Rightarrow Sy))$ &  \\
            \end{tabular}
        \end{center}

        De (3):
        \begin{center}
            \begin{tabular}{ c  l  l  c  }
                \hline
                No. &  &  &  \\
                \hline
                1) & $(\exists x)Px\Rightarrow (\exists y)Qy$ &  & Premisa \\
                2) & $\neg(\exists x)(Px\Rightarrow (\exists y)Qy)$ &  & Negación \\
                3) & $\neg(\exists x)(\neg\neg(Px\Rightarrow (\exists y)Qy))$ &  & 2 Ax. \\
                4) & $(\forall x)(\neg(Px\Rightarrow (\exists y)Qy))$ &  & Equiv. \\
                5) & $(\forall x)(\neg(\exists y)Qy\Rightarrow \neg Px)$ &  & Equiv. \\
                6) & $(\forall x)((\forall y)\neg Qy\Rightarrow \neg Px)$ &  & Equiv. \\
                7) & $(\forall x)(\forall y)\neg Qy\Rightarrow (\forall x)\neg Px$ &  & Equiv. \\
                8) & $(\forall y)\neg Qy$ & $\Rightarrow (\forall x)(\forall y)\neg Qy$ & Ax.2 \\
                9) & $(\forall y)\neg Qy \Rightarrow (\forall x)\neg Px$ &  & 7,8 I.U.  \\
                10) & $\neg((\exists y)Qy) \Rightarrow \neg((\exists x)Px)$ &  & Equiv  \\
                11) & $(\exists x)(Px\Rightarrow (\exists y)Qy)$ &  & 2-10 Contradicción  \\
                \hline
                  &  &  $\therefore (\exists x)(Px\Rightarrow (\exists y)Qy)$ &  \\
            \end{tabular}
        \end{center}

        Alternativa (y correcta):

        \begin{center}
            \begin{tabular}{ c  l  l  c  }
                \hline
                No. &  &  &  \\
                \hline
                1) & $(\exists x)Px\Rightarrow (\exists y)Qy$ &  & Premisa \\
                2) & $Px$ &  & Hipótesis \\
                3) & $(\exists x)Px$ &  & 2 G.E. \\
                4) & $(\exists y)Qy$ &  & 3,1 M.P. \\
                5) & $(Px\Rightarrow(\exists y)Qy)$ &  & 2-4 P.C. \\
                6) & $(\exists x)(Px\Rightarrow(\exists y)Qy)$ &  & 5 G.E. \\
                \hline
                  &  &  $\therefore (\exists x)(Px\Rightarrow (\exists y)Qy)$ &  \\
            \end{tabular}
        \end{center}

    \end{proof}

    \begin{obs}
        S.H. siginifca silogismo hipotético. El I.U y G.E son:
        $(\forall x)\varphi\Rightarrow \varphi[t/x]$ y $\varphi[t/x]\Rightarrow (\exists x)\varphi$.
    \end{obs}

\end{document}