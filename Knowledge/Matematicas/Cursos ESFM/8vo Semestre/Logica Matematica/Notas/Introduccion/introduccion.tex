\documentclass[12pt]{report}
\usepackage[spanish]{babel}
\usepackage[utf8]{inputenc}
\usepackage{amsmath}
\usepackage{amssymb}
\usepackage{amsthm}
\usepackage{graphics}
\usepackage{subfigure}
\usepackage{lipsum}
\usepackage{array}
\usepackage{multicol}
\usepackage{enumerate}
\usepackage[framemethod=TikZ]{mdframed}
\usepackage[a4paper, margin = 1.5cm]{geometry}
\usepackage{enumitem}
\usepackage{tikz}
\usepackage{pgffor}
\usepackage{ifthen}

\usetikzlibrary{shapes.multipart}

\newcounter{it}
\newcommand*\watermarktext[1]{\begin{tabular}{c}
    \setcounter{it}{1}%
    \whiledo{\theit<100}{%
    \foreach \col in {0,...,15}{#1\ \ } \\ \\ \\
    \stepcounter{it}%
    }
    \end{tabular}
    }

\AddToHook{shipout/foreground}{
    \begin{tikzpicture}[remember picture,overlay, every text node part/.style={align=center}]
        \node[rectangle,black,rotate=30,scale=2,opacity=0.08] at (current page.center) {\watermarktext{Cristo Daniel Alvarado ESFM\quad}}; 
  \end{tikzpicture}
}
%En esta parte se hacen redefiniciones de algunos comandos para que resulte agradable el verlos%

\renewcommand{\theenumii}{\roman{enumii}}

\def\proof{\paragraph{Demostración:\\}}
\def\endproof{\hfill$\blacksquare$}

\def\sol{\paragraph{Solución:\\}}
\def\endsol{\hfill$\square$}

%En esta parte se definen los comandos a usar dentro del documento para enlistar%

\newtheoremstyle{largebreak}
  {}% use the default space above
  {}% use the default space below
  {\normalfont}% body font
  {}% indent (0pt)
  {\bfseries}% header font
  {}% punctuation
  {\newline}% break after header
  {}% header spec

\theoremstyle{largebreak}

\newmdtheoremenv[
    leftmargin=0em,
    rightmargin=0em,
    innertopmargin=0pt,
    innerbottommargin=5pt,
    hidealllines = true,
    roundcorner = 5pt,
    backgroundcolor = gray!60!red!30
]{exa}{Ejemplo}[section]

\newmdtheoremenv[
    leftmargin=0em,
    rightmargin=0em,
    innertopmargin=0pt,
    innerbottommargin=5pt,
    hidealllines = true,
    roundcorner = 5pt,
    backgroundcolor = gray!50!blue!30
]{obs}{Observación}[section]

\newmdtheoremenv[
    leftmargin=0em,
    rightmargin=0em,
    innertopmargin=0pt,
    innerbottommargin=5pt,
    rightline = false,
    leftline = false
]{theor}{Teorema}[section]

\newmdtheoremenv[
    leftmargin=0em,
    rightmargin=0em,
    innertopmargin=0pt,
    innerbottommargin=5pt,
    rightline = false,
    leftline = false
]{propo}{Proposición}[section]

\newmdtheoremenv[
    leftmargin=0em,
    rightmargin=0em,
    innertopmargin=0pt,
    innerbottommargin=5pt,
    rightline = false,
    leftline = false
]{cor}{Corolario}[section]

\newmdtheoremenv[
    leftmargin=0em,
    rightmargin=0em,
    innertopmargin=0pt,
    innerbottommargin=5pt,
    rightline = false,
    leftline = false
]{lema}{Lema}[section]

\newmdtheoremenv[
    leftmargin=0em,
    rightmargin=0em,
    innertopmargin=0pt,
    innerbottommargin=5pt,
    roundcorner=5pt,
    backgroundcolor = gray!30,
    hidealllines = true
]{mydef}{Definición}[section]

\newmdtheoremenv[
    leftmargin=0em,
    rightmargin=0em,
    innertopmargin=0pt,
    innerbottommargin=5pt,
    roundcorner=5pt
]{excer}{Ejercicio}[section]

%En esta parte se colocan comandos que definen la forma en la que se van a escribir ciertas funciones%

\newcommand\abs[1]{\ensuremath{\biglvert#1\bigrvert}}
\newcommand\divides{\ensuremath{\bigm|}}
\newcommand\cf[3]{\ensuremath{#1:#2\rightarrow#3}}
\newcommand\contradiction{\ensuremath{\#_c}}
\newcommand\natint[1]{\ensuremath{\left[\!\left[ #1\right]\!\right]}}
\newcommand{\afa}{\:
    \begin{tikzpicture}
        \draw [line width = 0.17 mm, black] (0,0) -- (-0.115,0.29);
        \draw [line width = 0.17 mm, black] (0,0) -- (0.115,0.29);
        \draw [line width = 0.17 mm, black] (-0.12,0) arc (190:-10:0.12cm);
    \end{tikzpicture}
    \:
}
\newcommand{\pstable}[1]{\arabic{#1})\stepcounter{#1}}
\newcounter{tablec}
\newcommand{\free}{\textup{Free}}

\begin{document}
    \title{Curso de Lógica Matemática}
    \author{Cristo Daniel Alvarado}
    \maketitle

    \tableofcontents %Con este comando se genera el índice general del libro%

    \setcounter{chapter}{-1} %En esta parte lo que se hace es cambiar la enumeración del capítulo%
    
    \chapter{Introducción}
    
    \section{Temario}
    

    Los siguientes temas se verán a lo largo del curso:

    \renewcommand{\theenumii}{\arabic{enumi}.\arabic{enumii}}

    \begin{enumerate}
        \item Lógica (Teoría de Modelos).
        \begin{enumerate}
            \item Lógica proposicional.
            \item Lógica de primer orden.
        \end{enumerate}
        \item Teoría de la Computabilidad.
        \begin{enumerate}
            \item Conjuntos/Funciones computables.
            \item Teoremas de incompletitud.
        \end{enumerate}
        \item Teoría de Conjuntos.
        \begin{enumerate}
            \item Ordinales.
            \item Cardinalidad.
        \end{enumerate}
    \end{enumerate}

    Y la bibliografía para el curso es la siguiente:

    \begin{itemize}
        \item Enderton, 'Introducción matemática a la lógica'.
        \item  Enderton, 'Teoría de la computabilidad'.
        \item Copi, 'Lógica Simbólica' o 'Computability Theory'.
        \item Rebeca Weber 'Computability Theory'.
        \item Hrbacek, Seda.
        \item Herrnández Herrnández.
    \end{itemize}

    \section{Conectivas Lógicas}

    La disyunción ($\land$), conjunción ($\lor$), negación ($\neg$), implicación ($\Rightarrow$) y si y sólo si ($\iff$) son las conectivas lógicas usadas usualmente. Para las demostraciones se tienen que tomar los casos que se cumplan con estas implicaciones, por lo que si podemos simplificar el conjunto de conectivas lógicas, todo se simplificará.

    Para ello, veamos que
    \begin{equation*}
        P\Rightarrow Q \equiv \neg(P\land\neg Q)
    \end{equation*}
    tienen las mismas tablas de verdad. Por ejemplo también se tiene
    \begin{equation*}
        P\land Q\equiv \neg(\neg P\lor\neg Q)
    \end{equation*}
    ó
    \begin{equation*}
        P\lor Q\equiv\neg(\neg P\land\neg Q)
    \end{equation*}

    A $\left\{\land, \lor, \neg \right\}$ se le conoce como un \textbf{conjunto completo de conectivas lógicas} (es decir que toda conectiva se expresa como combinaciones de ellas). Nos podemos quedar simplemente con conjuntos completos de disyuntivas con solo dos elementos, a saber: $\left\{\land, \neg \right\}$ y $\left\{\lor, \neg \right\}$, ya que $P\lor Q$ es $\neg(\neg P\land \neg Q)$. (de forma similar a lo otro $P\land Q$ es $\neg(\neg P\lor \neg Q)$).

    También $\left\{\Rightarrow, \neg \right\}$ es otro conjunto completo de conectivas lógicas, ya que $P\land Q$ es $\neg(P\Rightarrow\neq Q)$.

    Y, $\left\{|\right\}$ es un conjunto completo, donde $|$ es llamado la \textbf{barra de Scheffel}, que tiene la siguiente tabla de verdad.

    \begin{center}
        \begin{tabular}{c c | c}
            \hline
            $P$ & $Q$ & $P|Q$ \\
            \hline
            $V$ & $V$ & $F$ \\
            $V$ & $F$ & $V$ \\
            $F$ & $V$ & $V$ \\
            $F$ & $F$ & $V$ \\
        \end{tabular}
    \end{center}
    con este, se tiene un conjunto completo de conectivas lógicas. Veamos que
    \begin{equation*}
        P|P\equiv\neg P
    \end{equation*}
    y,
    \begin{equation*}
        P\land Q\equiv \neg(P|Q)\equiv (P|Q)|(P|Q)
    \end{equation*}

    Como muchas veces se usan conectivas de este tipo:
    \begin{equation*}
        (P\Rightarrow \neg Q)\Rightarrow((P\Rightarrow R)\land\neg(Q\Rightarrow S)\land T)
    \end{equation*}
    al ser muy largas, a veces es más conveniente escribirlas en forma Polaca. De esta forma, lo anterior quedaría de la siguiente manera:
    \begin{equation*}
        \Rightarrow\Rightarrow P\neg Q\land\land PR\neg\Rightarrow Q S T 
    \end{equation*}

    Ahora empezamos con el estudio formal de la lógica.

    \section{Ejercicios}

    Convierta a/de notación Polaca, según sea el caso.

    \begin{enumerate}
        \item $P\Rightarrow(Q\land\neg A)$. Sería $\Rightarrow P\land Q \neg A$.
        \item $\Rightarrow  A\neg\land CD$. Sería $A\Rightarrow \neg(C\land D)$.
        \item $(A\Rightarrow B)\iff (\neg A\lor \neg(\neg B\lor C))$. Sería $\iff \Rightarrow A B \lor \neg A \neg \lor \neg B C$.
        \item $\lor\neg\neg\Rightarrow B C\land\Rightarrow A\land BCA$. Sería $(\neg\neg(B\Rightarrow C))\lor((A\Rightarrow (B\land C))\land A)$.
        \item $(P\Rightarrow Q)\iff (\neg Q\Rightarrow\neg P)$. Sería $\iff \Rightarrow P Q \Rightarrow \neg Q \neg P$.
    \end{enumerate}

    \chapter{Lógica Proposicional}

    \section{Alfabeto}

    Hablaremos un poco de sintaxis y semántica.

    \begin{itemize}
        \item Sintaxis: la forma en la que vamos a ordenar nuestras variables y conectivas.
        \item Semántica: da significado al orden de nuestras variables y conectivas.
    \end{itemize}

    Definiremos el lenguaje de la lógica proposicional. Para ello primero definiremos el alfabeto.

    El \textbf{alfabeto} de la lógica proposicional es un conjunto que consta de dos tipos de símbolos:
    \begin{enumerate}
        \item \textbf{Variables}, denotadas por $p_1,p_2,...,p_n,...$ (a lo más una cantidad numerable). Estas representan proposiciones o enunciados (tengo un paraguas, me caí de las escaleras, no tengo café en la cafetera, etc\dots).
        \item \textbf{Conectivas}, como $\Rightarrow$ y $\neg$.
    \end{enumerate}

    El alfabeto que usaremos es: $\left\{\Rightarrow,\neg,p_1,p_2,... \right\}$.

    \begin{obs}
        Podríamos usar otro alfabeto, pero dado a que $\left\{\Rightarrow,\neg \right\}$ es un conjunto completo de conectivas lógicas (y resulta más sencillo usarlas que la barra de Scheffel), se tomará este alfabeto con estas conectivas.
    \end{obs}

    Aceptamos la existencia de estas cosas (pues, al menos debemos aceptar la existencia de algo).

    Se van a trabajar con sucesiones finitas de símbolos del alfabeto descrito anteriormente. Ahora necesitaremos especificar que tipos de sucesiones van a servirnos para tener un significado formal.

    \begin{exa}
        Por ejemplo la sucesión $(p_3)$, $\emptyset$, $(\Rightarrow, p_2,\neg,p_5)$. Básicamente estas sucesiones finitas representan fórmulas en notación polaca.
    \end{exa}

    \begin{mydef}
        En el conjunto de sucesiones finitas de símbolos del alfabeto, definimos una \textbf{fórmula bien formada} (abreviada como \textbf{FBF}) como sigue:
        \begin{enumerate}
            \item Cada variable es una \textbf{FBF}.
            \item Si $\varphi,\psi$ son \textbf{FBF}, entonces $\neg\varphi$ y $\Rightarrow\varphi\psi$ también lo son.
        \end{enumerate}
    \end{mydef}

    \begin{obs}
        Recordar que usamos la notación Polaca en la definición anterior. Cuando se colocan en (2) $\neg\varphi$ y $\Rightarrow\varphi\psi$, hace referencia a concatenar estas sucesiones finitas.
    \end{obs}

    A continuación unos ejemplos:
    
    \begin{exa}
        $p_{17}$, $p_{54}$ y $\Rightarrow p_2p_{25}$ son FBF. Las primeras dos son llamadas \textbf{variables aisladas}. También lo es $\neg \Rightarrow p_2p_{25}$ (en este ejemplo, los $p_i$ son variables).

        Pero, por ejemplo $\Rightarrow \neg p_1 p_2 p_3$ y $\Rightarrow p_4$ no son FBF.
    \end{exa}

    Viendo el ejemplo anterior, notamos que el operador $\Rightarrow$ es binario (solo usa dos entradas) y $\neg$ es unario (solo una entrada). Por lo cual, añadir o no demás variables a los opeadores dentro de la fórmula, hace que la fórmula ya no sea una FBF.

    \begin{obs}
        Eventualmente se va a sustituir la notación Polaca por la normal, para que se pueda leer la FBF y el proceso no sea robotizado.
    \end{obs}

    Definiremos ahora más conectivas lógicas para poder trabajar más cómodamente.

    \begin{mydef}
        Se definirán tres conectivas lógicos adicionales.
        \begin{enumerate}
            \item Se define la \textbf{disyunción} $\varphi\lor\psi$ como $\Rightarrow\neg\psi\varphi$ (en notación Polaca).
            \item Se define la \textbf{conjunción} $\varphi\land\psi$ como $\neg(\neg\psi\lor\neg\varphi)$.
            \item Se define el \textbf{si sólo si} $\psi\iff\varphi$ como $(\psi\Rightarrow\varphi)\land(\varphi\Rightarrow\psi)$.
        \end{enumerate}
    \end{mydef}

    \section{Modeos o Estructuras}

    En el fondo, queremos que las FBF sean cosas verdaderas o falsas. Un Modelo o Estructura es algo que le va a dar significado a las FBF, esta es la parte de la semántica. De alguna manera va a ser una forma de asignarle el valor de verdadero o falso a cada una de las variables.

    \begin{mydef}
        Un \textbf{Modelo o Estructura} de la lógica proposicional es una función $\cf{m}{\textup{Var}}{\left\{V,F\right\}}$, donde Var denota al conjunto de símbolos que son variables. Básicamente estamos diciendo que hay variables que son verdaderas y otras que son falsas.
    \end{mydef}

    \begin{theor}
        Para todo modelo $m$, existe una única extensión $\cf{\overline{m}}{\textup{FBF}}{\left\{V,F\right\}}$, donde FBF denota al conjunto de las fórmulas bien formadas, tal que
        \begin{enumerate}
            \item $\overline{m}(p)=m(p)$ para todo $p\in\textup{Var}$.
            \item Para todo $\varphi,\psi\in\textup{FBF}$:
            \begin{equation*}
                \overline{m}(\neg\varphi)=V\textup{ si y sólo si } \overline{m}(\varphi)=F
            \end{equation*}
            y,
            \begin{equation*}
                \overline{m}(\Rightarrow\varphi\psi)=F\textup{ si y sólo si }\overline{m}(\varphi)=V\textup{ y } \overline{m}(\psi)=F
            \end{equation*}
        \end{enumerate}
    \end{theor}

    \begin{proof}
        La prueba aún no se hará, pero es por recursión.
    \end{proof}

    \begin{mydef}
        Sea $m$ un modelo, $\varphi$ una fórmula y $\Sigma$ un cojunto de fórmulas. Decimos que
        \begin{enumerate}
            \item \textbf{$m$ satisface $\varphi$} (denotado por $m\vDash \varphi$) si $\overline{m}(\varphi)=V$.
            \item \textbf{$m$ satisface $\Sigma$} (denotado por $m\vDash \Sigma$) si $m\vDash\varphi$ para cada $\varphi\in\Sigma$.
        \end{enumerate}
    \end{mydef}

    \begin{exa}
        Sea $m$ un modelo tal que $m(p_1)=V$ y $m(p_i)=F$, para todo $i\geq2$. En este caso $m\nvDash \Rightarrow p_1p_3$, pero $m\vDash \neg p_5$.
    \end{exa}

    \begin{mydef}
        Sea $\varphi$ una fórmula y $\Sigma$ un conjunto de fórmulas.
        \begin{enumerate}
            \item Decimos que $\varphi$ es \textbf{consecuencia lógica} de $\Sigma$ o que $\Sigma$ \textbf{lógicamente implica} $\varphi$ (denotado por $\Sigma\vDash\varphi$) si para todo modelo $m$ tal que $m\vDash\Sigma$, se cumple que $m\vDash\varphi$.
            \item Decimos que $\varphi$ es una \textbf{tautología} si $\emptyset\vDash\varphi$.
            \item Decimos que $\varphi$ es una \textbf{contradicción} si $\emptyset\vDash\neg\varphi$ es una tautología.
        \end{enumerate}
    \end{mydef}

    Veamos ejemplos para aclarar las ideas:

    \begin{exa}
        Se tiene $\left\{\Rightarrow p_1p_2,p_2 \right\}\nvDash p_1$. En efecto, para un modelo $m$ tal que $m(p_2)=V$ y $m(p_1)=F$ es tal que $m$ no satisface $p_1$.
    \end{exa}

    \begin{exa}
        Muestre que $\left\{\Rightarrow p_1p_2,\Rightarrow p_2p_3 \right\}\vDash\Rightarrow p_1p_3$.
    \end{exa}

    \newcommand{\rest}{\ensuremath{\upharpoonright}}

    \begin{obs}
        De ahora en adelante, $f\rest A$ denotará la reestricción de $f$ al conjunto $A$.
    \end{obs}

    \begin{lema}
        Sean $n,m$ dos modelos y sea $\varphi$ una fórmula. Sean $p_{ i_1},p_{ i_2},...,p_{ i_k}$ las variables que ocurren en $\varphi$. Si $n\rest\left\{p_{ i_1},p_{ i_2},...,p_{ i_k}\right\}=m\rest\left\{p_{ i_1},p_{ i_2},...,p_{ i_k}\right\}$, entonces
        \begin{equation*}
            n\vDash \varphi\textup{ si y sólo si } m\vDash\varphi
        \end{equation*}
    \end{lema}

    \begin{proof}
        Por la definición de $\vDash$, basta con demostrar que
        \begin{equation*}
            \overline{n}(\varphi)=\overline{m}(\varphi)
        \end{equation*}
        Esto lo haremos por inducción sobre $\varphi$.
        \begin{itemize}
            \item Si $\varphi$ es una variable $p_t$, entonces
            \begin{equation*}
                \begin{split}
                    \overline{n}(\varphi)&=\overline{n}(p_t)\\
                    &=n(p_t)\\
                    &=m(p_t)\\
                    &=\overline{m}(p_t)\\
                    &=\overline{m}(\varphi)\\
                \end{split}
            \end{equation*}
            \item Hay que ver que se cumple también para las conectivas:
            \begin{enumerate}
                \item Supongamos que $\varphi=\neg\psi$, siendo $\psi$ una FBF tal que $\overline{n}(\psi)=\overline{m}(\psi)$. Entonces,
                \begin{equation*}
                    \begin{split}
                        \overline{n}(\varphi)&=\overline{n}(\neg\psi)\\
                    \end{split}
                \end{equation*}
                se tiene que
                \begin{equation*}
                    \begin{split}
                        \overline{n}(\neg\psi)=V&\textup{ si y sólo si } \overline{n}(\psi)=F\\
                        &\textup{ si y sólo si } \overline{m}(\psi)=F\\
                        &\textup{ si y sólo si } \overline{m}(\neg\psi)=V\\
                        &\textup{ si y sólo si } \overline{m}(\varphi)=V\\
                    \end{split}
                \end{equation*}
                por tanto, $\overline{m}(\varphi)=V$ si y sólo si $\overline{m}(\varphi)=V$. De forma análoga se llega a que $\overline{m}(\varphi)=F$ si y sólo si $\overline{m}(\varphi)=F$. Por tanto:
                \begin{equation*}
                    \overline{m}(\varphi)=\overline{n}(\varphi)
                \end{equation*}
                \item Supongamos que $\varphi$ es de la forma $\Rightarrow\psi\chi$ y que $\overline{m}(\psi)=\overline{n}(\psi)$, y $\overline{m}(\chi)=\overline{n}(\chi)$.
                
                Se tiene que $\overline{m}(\Rightarrow\psi\chi)=F$ si y sólo si $\overline{m}(\psi)=V$ y $\overline{m}(\chi)=F$, si y sólo si $\overline{n}(\psi)=V$ y $\overline{n}(\chi)=F$, si y sólo si $\overline{n}(\Rightarrow\psi\chi)=F$ (que es el único caso en que es falso). Por tanto:
                \begin{equation*}
                    \overline{m}(\varphi)=\overline{n}(\varphi)
                \end{equation*}
            \end{enumerate}
        \end{itemize}
        por inducción se sigue que
        \begin{equation*}
            n\vDash\varphi\textup{ si y sólo si } m\vDash\varphi
        \end{equation*}
    \end{proof}

    Con este lema, se tiene que el ejemplo 1.2.2 ya tiene fundamentación, ya que únicamente basta que el modelo sea válido en las variables $p_1$ y $p_2$ (no en la cantidad infinita de variables que podemos llegar a tener).

    \begin{obs}
        Si $\Sigma$ es un conjunto finito de fórmulas (digamos $\Sigma=\left\{\varphi_1,...,\varphi_n \right\}$), entonces $\Sigma\vDash\varphi$ si y sólo si $(\varphi_1\land\cdots\land\varphi_n)\Rightarrow\varphi$ es una tautología.
    \end{obs}

    \section{Cálculo Proposicional}

    Nuestro cálculo proposicional se compondrá de lo siguiente:
    \begin{enumerate}
        \item \textbf{Axiomas Lógicos}.
        \item \textbf{Reglas de inferencia}.
    \end{enumerate}
    más adelante se probará que si hubiesemos elegido diferentes axiomas lógicos y reglas de inferencia, habríamos llegado al mismo resultado (siempre que se haya cumplido una hipótesis adicional ¿?). 

    \begin{mydef}[\textbf{Axiomas Lógicos}]
        Tenemos para nuestro cálculo proposicional los siguientes axiomas lógicos:
        \begin{enumerate}
            \item $\varphi\Rightarrow(\psi\Rightarrow\varphi)$.
            \item $\varphi\Rightarrow((\psi\Rightarrow\neg\varphi)\Rightarrow\neg\psi)$.
            \item $\varphi\Rightarrow\varphi'$ siempre que $\varphi$ resulte de sustituir $\psi$ por $\neg\neg\psi$ o viceversa (siendo $\psi$ una subfórmula de $\varphi$).
            \item $\varphi\Rightarrow\varphi'$ siempre que $\varphi$ resulte de sustituir $\psi\Rightarrow\chi$ por $\neg\chi\Rightarrow\neg\psi$ o viceversa (siendo $\psi$ y $\chi$ subfórmulas de $\varphi$).
            \item $\varphi\Rightarrow\varphi'$ siempre que $\varphi$ resulte de sustitur $\neg\psi\Rightarrow\psi$ por $\psi$ (siendo $\psi$ una subfórmula de $\varphi$).
            \item $(\varphi\Rightarrow(\chi\Rightarrow\psi))\Rightarrow((\varphi\Rightarrow\chi)\Rightarrow(\varphi\Rightarrow\psi))$.
        \end{enumerate}
        siendo $\varphi$ una FBF dada y $\psi$ una FBF arbitraria en 1 y 2, y $\varphi,\psi,\chi$ FBF dadas.
    \end{mydef}

    \begin{exa}
        Ejemplo del axioma 1: si $p_1$ es una variable,
        \begin{equation*}
            p_1\Rightarrow(p_3\Rightarrow p_1)
        \end{equation*}
        siendo $p_3$ una variable arbitraria.
    \end{exa}

    \begin{exa}
        Ejemplo del axioma 3:
        \begin{equation*}
            (p_1\Rightarrow\neg\neg p_3)\Rightarrow(p_1\Rightarrow p_3)
        \end{equation*}
        o
        \begin{equation*}
            (p_1\Rightarrow p_3)\Rightarrow(p_1\Rightarrow\neg\neg p_3)
        \end{equation*}
    \end{exa}

    \begin{mydef}[\textbf{Reglas de Inferencia}]
        Se define una única regla de inferencia, denominada \textbf{Modus Ponens} (abreviada \textbf{MP}) dada por:
        \begin{center}
            \begin{tabular}{c c c}
                $\varphi$ & $\Rightarrow$ & $\psi$ \\
                $\varphi$ &  &  \\
                \hline
                 & $\therefore$ & $\psi$ \\
            \end{tabular}
        \end{center}
    \end{mydef}

    \renewcommand{\theenumi}{\arabic{enumi}}

    \renewcommand{\theenumii}{\arabic{enumi}.\roman{enumii}}

    \renewcommand{\theenumiii}{\roman{enumi}.\roman{enumii}.\alph{enumiii}}

    \begin{mydef}
        Sea $\Sigma$ un conjunto y $\varphi$ una fórmula.
        \begin{enumerate}
            \item Una \textbf{demostración} de $\varphi$ a partir de $\Sigma$ es una sucesión finita de fórmulas $(\varphi_1,...,\varphi_n)$ tal que:
            \begin{enumerate}
                \item $\varphi_n=\varphi$.
                \item Para cada $i\in\left\{1,...,n\right\}$ se tiene una de las tres:
                \begin{enumerate}
                    \item $\varphi_i\in\Sigma$.
                    \item $\varphi_i$ es axioma lógico.
                    \item existen $k,j\in\left\{1,...,n\right\}$ con $k<j<i$ tales que $\varphi_j$ es $\Rightarrow\varphi_k\varphi_i$.
                \end{enumerate}
            \end{enumerate}
            \item Decimos que $\varphi$ es \textbf{demostrable a partir de $\Sigma$}, o que $\varphi$ es un \textbf{teorema de $\Sigma$} si existe una demostración de $\varphi$ a partir de $\Sigma$, esto se simboliza por $\Sigma\vdash\varphi$.
        \end{enumerate}
    \end{mydef}

    \begin{mydef}
        Convenimos que
        \begin{itemize}
            \item $\varphi\lor\psi\equiv\neg\varphi\Rightarrow\psi$.
            \item $\varphi\land\psi\equiv\neg(\varphi\Rightarrow\neg\psi)$.
            \item $\varphi\iff\psi\equiv(\varphi\Rightarrow\psi)\land(\psi\Rightarrow\varphi)$.
        \end{itemize}
    \end{mydef}

    \begin{obs}
        De ahora en adelante \textbf{R.E.} simplifica reescritura.
    \end{obs}

    \begin{exa}
        Se cumple que $\left\{\neg p_3,p_1\Rightarrow p_3,p_1\lor(p_2\Rightarrow p_3),\neg p_3\Rightarrow(p_3\Rightarrow p_5),p_2 \right\}\vdash p_5$.    
    \end{exa}

    \begin{proof}
        Se tiene la siguiente demostración de $p_5$:
        \begin{center}
            \begin{tabular}{l l c l r}
                1) & $\neg p_3$ &  &  & Premisa \\
                2) & $p_1$ & $\Rightarrow$ & $p_3$ & Premisa \\
                3) & $p_1$ & $\lor$ & $(p_2\Rightarrow p_3)$ & Premisa \\
                4) & $\neg p_3$ & $\Rightarrow$ & $(p_3\Rightarrow p_5)$ & Premisa \\
                5) & $p_2$ &  &  & Premisa \\
                6) & $(p_1\Rightarrow p_3)$ & $\Rightarrow$ & $(\neg p_3\Rightarrow \neg p_1)$ & 2 Ax. 4 \\
                7) & $\neg p_3$ & $\Rightarrow$ & $\neg p_1$ & 6,2 M.P. \\
                8) & $\neg p_1$ &  &  & 1,6 M.P. \\
                9) & $\neg(p_2\Rightarrow p_3)$ & $\Rightarrow$ & $p_1$ & 3 R.E. \\
                10) & $(\neg(p_2\Rightarrow p_3)\Rightarrow p_1)$ & $\Rightarrow$ & $(\neg p_1\Rightarrow\neg\neg (p_2\Rightarrow p_3))$ & 9 Ax. 4 \\
                11) & $\neg p_1$ & $\Rightarrow$ & $\neg\neg (p_2\Rightarrow p_3)$ & 10,9 M.P. \\
                12) & $(\neg p_1 \Rightarrow\neg\neg (p_2\Rightarrow p_3))$ & $\Rightarrow$ & $(\neg p_1 \Rightarrow (p_2\Rightarrow p_3))$ & 11 Ax. 3 \\
                13) & $\neg p_1$ & $\Rightarrow$ & $(p_2\Rightarrow p_3)$ & 11,12 M.P. \\
                14) & $p_2$ & $\Rightarrow$ & $p_3$ & 11,12 M.P. \\
                15) & $p_3$ &  &  & 13,5 M.P. \\
                16) & $p_3$ & $\Rightarrow$ & $p_5$ & 1,4 M.P. \\
                17) & $p_5$ &  &  & 16,15 M.P. \\
                \hline
                & & $\therefore$ & $p_5$ & \\
            \end{tabular}
        \end{center}
    \end{proof}

    Tenemos las siguientes reglas de inferencia adicionales:

    \begin{propo}
        Se cumple lo siguiente:
        \begin{center}
            \textbf{Modus Tollens}
        \end{center}
        \begin{center}
            \begin{tabular}{c c c}
                $\varphi$ & $\Rightarrow$ & $\psi$ \\
                $\neg\psi$ &  &  \\
                \hline
                 & $\therefore$ & $\neg\varphi$ \\
            \end{tabular}
        \end{center}
        \begin{center}
            \textbf{Silogismo Disyuntivo}
        \end{center}
        \begin{center}
            \begin{tabular}{c c c}
                $\varphi$ & $\lor$ & $\psi$ \\
                $\neg\varphi$ &  &  \\
                \hline
                 & $\therefore$ & $\psi$ \\
            \end{tabular}
        \end{center}
        \begin{center}
            \textbf{Adición}
        \end{center}
        \begin{center}
            \begin{tabular}{c c c}
                $\varphi$ &  &  \\
                \hline
                 & $\therefore$ & $\varphi\lor\psi$ \\
            \end{tabular}
        \end{center}
        siendo $\psi$ una FBF cualquiera.
        \begin{center}
            \textbf{Simplificación}
        \end{center}
        \begin{center}
            \begin{tabular}{c c c}
                $\varphi$ & $\land$ & $\psi$ \\
                \hline
                 & $\therefore$ & $\varphi$ \\
            \end{tabular}
        \end{center}
        \begin{center}
            \textbf{Conjunción}
        \end{center}
        \begin{center}
            \begin{tabular}{c c c}
                $\varphi$ &  &  \\
                $\psi$ &  &  \\
                \hline
                 & $\therefore$ & $\varphi\land\psi$ \\
            \end{tabular}
        \end{center}
    \end{propo}

    \begin{proof}
        En efecto, veamos que existen las demostraciones:
        \begin{itemize}
            \item \textbf{Modus Tollens}:
            \begin{center}
                \begin{tabular}{l l c l r}
                    1) & $\varphi$ & $\Rightarrow$ & $\psi$ & Premisa \\
                    2) & $\neg\psi$ &  &  & Premisa \\
                    3) & $(\varphi\Rightarrow\psi)$ & $\Rightarrow$ & $(\neg\psi\Rightarrow\neg\varphi)$ & 1 Ax. 4 \\
                    4) & $\neg\psi$ & $\Rightarrow$ & $\neg\varphi$ & 1,3 M.P. \\
                    5) & $\neg\varphi$ &  &  & 2,4 M.P. \\
                    \hline
                    & & $\therefore$ & $\neg\varphi$ & \\
                \end{tabular}
            \end{center}
            \item \textbf{Silogismo Disyuntivo}:
            \begin{center}
                \begin{tabular}{l l c l r}
                    1) & $\varphi$ & $\lor$ & $\psi$ & Premisa \\
                    2) & $\neg\varphi$ &  &  & Premisa \\
                    3) & $\neg\varphi$ & $\Rightarrow$ & $\psi$ & 1 R.E.\\
                    4) & $\psi$ &  &  & 2,1 M.P. \\
                    \hline
                    & & $\therefore$ & $\psi$ & \\
                \end{tabular}
            \end{center}
            \item \textbf{Adición}:
            \begin{center}
                \begin{tabular}{l l c l r}
                    1) & $\varphi$ &  &  & Premisa \\
                    2) & $\varphi$ & $\Rightarrow$ & $(\neg\psi\Rightarrow\varphi)$ & 1 Ax. 1 \\
                    3) & $\neg\psi$ & $\Rightarrow$ & $\varphi$ & 1,2 M.P. \\
                    4) & $(\neg\psi\Rightarrow\varphi)$ & $\Rightarrow$ & $(\neg\varphi\Rightarrow\neg\neg\psi)$ & 3 Ax. 4 \\
                    5) & $\neg\varphi$ & $\Rightarrow$ & $\neg\neg\psi$ & 3,4 M.P. \\
                    6) & $(\neg\varphi\Rightarrow\neg\neg\psi)$ & $\Rightarrow$ & $(\neg\varphi\Rightarrow\psi)$ & 5 Ax. 3 \\
                    7) & $\neg\varphi$ & $\Rightarrow$ & $\psi$ & 5,6 M.P. \\
                    8) & $\varphi$ & $\lor$ & $\psi$ & 7 R.E. \\
                    \hline
                    & & $\therefore$ & $\varphi\lor\psi$ & \\
                \end{tabular}
            \end{center}
            \item \textbf{Simplificación}:
            \begin{center}
                \begin{tabular}{l l c l r}
                    1) & $\varphi$ & $\land$ & $\psi$ & Premisa \\
                    2) & $\neg(\varphi$ & $\Rightarrow$ & $\neg\psi)$ & 1 R.E. \\
                    3) & $\neg(\varphi\Rightarrow\neg\psi)$ & $\Rightarrow$ & $\neg(\neg\neg\psi\Rightarrow\neg\varphi)$ & 2 Ax. 4 \\
                    4) & $\neg(\neg\neg\psi$ & $\Rightarrow$ & $\neg\varphi)$ & 3,2 M.P. \\
                    5) & $\neg(\neg\neg\psi\Rightarrow\neg\varphi)$ & $\Rightarrow$ & $\neg(\psi\Rightarrow\neg\varphi)$ & 4 Ax. 3 \\
                    6) & $\neg(\psi$ & $\Rightarrow$ & $\neg\varphi)$ & 5,4 M.P. \\
                    7) & $\neg\varphi$ & $\Rightarrow$ & $\neg(\psi\Rightarrow\neg\varphi)$ & 6 Ax. 1 y M.P. \\
                    8) & $(\neg\varphi\Rightarrow(\psi\Rightarrow\neg\varphi))$ & $\Rightarrow$ & $(\neg\varphi\Rightarrow(\neg\neg\varphi\Rightarrow\neg\psi))$ & 7 Ax. 1 \\
                    5) & $\neg\varphi$ & $\Rightarrow$ & $(\neg\neg\varphi\Rightarrow\neg\psi)$ & 8,7 M.P. \\
                    6) & $(\neg\varphi\Rightarrow(\neg\neg\varphi\Rightarrow\neg\psi))$ & $\Rightarrow$ & $(\neg\varphi\Rightarrow(\varphi\Rightarrow\neg\psi))$ &  5 Ax. 3 \\
                    7) & $\neg\varphi$ & $\Rightarrow$ & $(\varphi\Rightarrow\neg\psi)$ &  6,5 M.P. \\
                    8) & $\neg(\varphi\Rightarrow\neg\psi)$ & $\Rightarrow$ & $\neg\neg\varphi$ &  Ax. 4 + M.P. \\
                    9) & $\neg\neg\varphi$ &  &  &  M.P. \\
                    10) & $\varphi$ &  &  &  M.P. \\
                    \hline
                    & & $\therefore$ & $\varphi$ & \\
                \end{tabular}
            \end{center}
            La prueba está mal, pero el resultado es correcto (debo verificar que detalles de la prueba anoté mal).
            %TODO
            \item \textbf{Conjunción}:
            \begin{center}
                \begin{tabular}{l l c l r}
                    1) & $\varphi$ &  &  & Premisa \\
                    2) & $\psi$ &  &  & Premisa \\
                    3) & $\varphi$ & $\Rightarrow$ & $((\psi\Rightarrow\neg\varphi)\Rightarrow\neg\psi)$ & 1 Ax. 2 \\
                    4) & $(\psi\Rightarrow\neg\varphi)$ & $\Rightarrow$ & $\neg\psi$ & 3,1 M.P. \\
                    5) & $((\psi\Rightarrow\neg\varphi)\Rightarrow\neg\psi)$ & $\Rightarrow$ & $(\neg\neg\psi\Rightarrow\neg(\psi\Rightarrow\neg\varphi))$ & 4 Ax. 3\\
                    6) & $\neg\neg\psi$ & $\Rightarrow$ & $\neg(\psi\Rightarrow\neg\varphi)$ & 5,4 M.P.\\
                    7) & $\psi$ & $\Rightarrow$ & $\neg\neg\psi$ & 2 Ax. 3\\
                    8) & $\neg\neg\psi$ &  &  & 7,1 M.P.\\
                    9) & $\neg(\psi$ & $\Rightarrow$ & $\neg\varphi)$ & 6,8 M.P.\\
                    10) & $\psi$ & $\land$ & $\varphi$ & 10 R.E.\\
                    \hline
                    & & $\therefore$ & $\psi\land\varphi$ & \\
                \end{tabular}
            \end{center}
        \end{itemize}
    \end{proof}

    \begin{propo}
        Se cumple lo siguiente:
        \begin{center}
            \textbf{Conmutatividad de $\lor$}
        \end{center}
        \begin{center}
            \begin{tabular}{c c c}
                $\varphi$ & $\lor$ & $\psi$ \\
                \hline
                 & $\therefore$ & $\psi\lor\varphi$ \\
            \end{tabular}
        \end{center}
        \begin{center}
            \textbf{Conmutatividad de $\land$}
        \end{center}
        \begin{center}
            \begin{tabular}{c c c}
                $\varphi$ & $\land$ & $\psi$ \\
                \hline
                 & $\therefore$ & $\psi\land\varphi$ \\
            \end{tabular}
        \end{center}
    \end{propo}

    \begin{proof}
        \begin{itemize}
            \item \textbf{Conmutatividad de $\lor$}
            \begin{center}
                \begin{tabular}{l l c l r}
                    1) & $\varphi$ & $\lor$ & $\psi$ & Premisa \\
                    2) & $\neg\varphi$ & $\Rightarrow$ & $\psi$ & 1 R.E. \\
                    3) & $(\neg\varphi\Rightarrow\psi)$ & $\Rightarrow$ & $(\neg\psi\Rightarrow\neg\neg\varphi)$ & 2 Ax. 4 \\
                    4) & $\neg\psi$ & $\Rightarrow$ & $\neg\neg\varphi$ & 3,2 M.P.\\
                    5) & $(\neg\psi\Rightarrow\neg\neg\varphi)$ & $\Rightarrow$ & $(\neg\psi\Rightarrow\varphi)$ & 4 Ax. 3 \\
                    6) & $\neg\psi$ & $\Rightarrow$ & $\varphi$ & 5,4 M.P. \\
                    7) & $\psi$ & $\lor$ & $\varphi$ & 6 R.E. \\
                    \hline
                    & & $\therefore$ & $\psi\lor\varphi$ & \\
                \end{tabular}
            \end{center}
            \item \textbf{Conmutatividad de $\land$}
            \begin{center}
                \begin{tabular}{l l c l r}
                    1) & $\varphi$ & $\land$ & $\psi$ & Premisa \\
                    2) & $\neg(\varphi$ & $\Rightarrow$ & $\neg\psi)$ & 1 R.E. \\
                    3) & $\neg(\varphi\Rightarrow\neg\psi)$ & $\Rightarrow$ & $\neg(\neg\neg\psi\Rightarrow\neg\varphi)$ & 2 Ax. 4 \\
                    4) & $\neg(\neg\neg\psi$ & $\Rightarrow$ & $\neg\varphi)$ & 3,2 M.P. \\
                    5) & $\neg(\neg\neg\psi\Rightarrow\neg\varphi)$ & $\Rightarrow$ & $\neg(\psi\Rightarrow\neg\varphi)$ & 4 Ax. 3 \\
                    6) & $\neg(\psi$ & $\Rightarrow$ & $\neg\varphi)$ & 5,4 M.P. \\
                    7) & $\psi$ & $\land$ & $\varphi$ & 5,4 M.P. \\
                    \hline
                    & & $\therefore$ & $\psi\land\varphi$ & \\
                \end{tabular}
            \end{center}
        \end{itemize}
    \end{proof}

    \begin{excer}
        Complete las siguientes demostraciones (que es equivalente a probar que existe una demostración de los siguientes enunciados):
        \begin{enumerate}
            \item $\left\{p_1\lor(p_5\lor p_7),(p_5\lor p_7)\Rightarrow(p_{13}\lor p_{14}),(p_3\lor p_{14})\Rightarrow(p_1\lor p_7),\neg p_1 \right\}\vdash p_7$.
            \item $\left\{p_4\Rightarrow(p_5\Rightarrow p_6),(p_5\Rightarrow p_6)\Rightarrow p_{10},(p_{20}\lor p_{ 30})\Rightarrow \neg p_{ 40}, \neg p_{ 40}\Rightarrow( p_5\iff \neg p_{ 45}),\neg p_{10},\neg(p_5\iff\neg p_{45}) \right\}\vdash \neg p_4\land(p_{20}\lor p_{30})$.
        \end{enumerate}
    \end{excer}

    \begin{proof}
        De (1):
        \begin{center}
            \begin{tabular}{l l c l r}
                1) & $p_1$ & $\lor$ & $(p_5\lor p_7)$ & Premisa \\
                2) & $(p_5\lor p_7)$ & $\Rightarrow$ & $(p_{13}\lor p_{14})$ & Premisa \\
                3) & $(p_{13}\lor p_{14})$ & $\Rightarrow$ & $(p_{1}\lor p_{7})$ & Premisa \\
                4) & $\neg p_1$ &  &  & Premisa \\
                5) & $p_5$ & $\lor$ & $p_7$ & 1,4 S.D. \\
                6) & $p_{13}$ & $\lor$ & $p_{14}$ & 2,5 M.P.\\
                7) & $p_{1}$ & $\lor$ & $p_7$ & 3,6 M.P. \\
                8) & $p_{7}$ &  &  & 7,4 S.D.\\
                \hline
                & & $\therefore$ & $p_7$ & \\
            \end{tabular}
        \end{center}

        De (2):
        \begin{center}
            \begin{tabular}{l l c l r}
                1) & $p_4$ & $\Rightarrow$ & $(p_5\Rightarrow p_7)$ & Premisa \\
                2) & $(p_5\Rightarrow p_7)$ & $\Rightarrow$ & $p_{10}$ & Premisa \\
                3) & $(p_{20}\Rightarrow p_{30})$ & $\Rightarrow$ & $\neg p_{40}$ & Premisa \\
                4) & $\neg p_{40}$ & $\Rightarrow$ & $(p_5\iff\neg p_{ 45})$ & Premisa \\
                5) & $\neg p_{10}$ &  &  & Premisa \\
                6) & $\neg(p_5$ & $\iff$ & $\neg p_{45})$& Premisa \\
                7) & $p_{40}$ &  &  & 4,6 M.T. \\
                8) & $p_{40}$ & $\Rightarrow$ & $\neg\neg p_{40}$ & 7 Ax. 3 \\
                9) & $\neg\neg p_{40}$ &  &  & 8,7 M.P. \\
                10) & $\neg(p_{20}$ & $\Rightarrow$ & $p_{30})$ & 3,9 M.T. \\
                11) & $p_{20}$ & $\lor$ & $p_{30}$ & 10 R.E. \\
                12) & $((p_5\Rightarrow p_7)\Rightarrow p_{10})$ & $\Rightarrow$ & $(\neg p_{10}\Rightarrow\neg(p_5\Rightarrow p_7))$ & 2 Ax. 4 \\
                13) & $\neg p_{10}$ & $\Rightarrow$ & $\neg(p_5\Rightarrow p_7)$ & 12,2 M.P. \\
                14) & $\neg(p_5$ & $\Rightarrow$ & $p_7)$ & 13,5 M.P. \\
                15) & $\neg p_4$ &  &  & 1,14 M.T. \\
                16) & $\neg p_4$ & $\land$ & $(p_{20}\lor p_{30})$ & 15,11 Ad. \\
                \hline
                & & $\therefore$ & $\neg p_4\land(p_{20}\lor p_{30})$ & \\
            \end{tabular}
        \end{center}
    \end{proof}

    \begin{propo}
        Se cumple lo siguiente:
        \begin{center}
            \textbf{Doble Negación}
        \end{center}
        \begin{center}
            \begin{tabular}{c c c}
                $\varphi$ &  &  \\
                \hline
                 & $\therefore$ & $\neg\neg\varphi$ \\
            \end{tabular}
        \end{center}
        y,
        \begin{center}
            \begin{tabular}{c c c}
                $\neg\neg\varphi$ &  &  \\
                \hline
                 & $\therefore$ & $\varphi$ \\
            \end{tabular}
        \end{center}
        \begin{center}
            \textbf{Transposición}
        \end{center}
        \begin{center}
            \begin{tabular}{c c c}
                $\varphi$ &  &  \\
                $\varphi$ & $\Rightarrow$ & $\varphi'$ (con Ax. 4) \\
                \hline
                 & $\therefore$ & $\varphi'$ \\
            \end{tabular}
        \end{center}
        \begin{center}
            \textbf{Tautología}
        \end{center}
        \begin{center}
            \begin{tabular}{c c c}
                $\neg\psi$ & $\Rightarrow$ & $\psi$ \\
                \hline
                 & $\therefore$ & $\psi$ \\
            \end{tabular}
        \end{center}
    \end{propo}

    \begin{proof}
        \begin{itemize}
            \item \textbf{Doble Negación}:
            \begin{center}
                \begin{tabular}{l l c l r}
                    1) & $\varphi$ &  &  & Premisa \\
                    2) & $\varphi$ & $\Rightarrow$ & $\neg\neg\varphi$ & 1 Ax. 4 \\
                    3) & $\neg\neg\varphi$ &  &  & 2,1 M.P. \\
                    \hline
                    & & $\therefore$ & $\neg\neg\varphi$ & \\
                \end{tabular}
            \end{center}
            y,
            \begin{center}
                \begin{tabular}{l l c l r}
                    1) & $\neg\neg\varphi$ &  &  & Premisa \\
                    2) & $\neg\neg\varphi$ & $\Rightarrow$ & $\varphi$ & 1 Ax. 4 \\
                    3) & $\varphi$ &  &  & 2,1 M.P. \\
                    \hline
                    & & $\therefore$ & $\varphi$ & \\
                \end{tabular}
            \end{center}
            \item \textbf{Transposición}: Solo basta con usar un Modus Ponens y una instancia del axioma 4.
            \item \textbf{Tautología}:
            \begin{center}
                \begin{tabular}{l l c l r}
                    1) & $\neg\psi$ & $\Rightarrow$ & $\psi$ & Premisa \\
                    2) & $(\neg\psi\Rightarrow\psi)$ & $\Rightarrow$ & $\psi$ & 1 Ax. 5 \\
                    3) & $\psi$ &  &  & 2,1  M.P. \\
                    \hline
                    & & $\therefore$ & $\varphi$ & \\
                \end{tabular}
            \end{center}
        \end{itemize}
    \end{proof}

    \begin{excer}
        Complete las siguientes demostraciones.
    \end{excer}

    \begin{proof}
        De (a):
        \begin{center}
            \begin{tabular}{l l c l r}
                1) & $A$ & $\Rightarrow$ & $B$ & Premisa \\
                2) & $C$ & $\Rightarrow$ & $D$ & Premisa \\
                3) & $\neg B$ & $\lor$ & $\neg D$ & Premisa \\
                4) & $\neg\neg A$ &  &  & Premisa \\
                5) & $E\land F$ & $\Rightarrow$ & $C$ & Premisa \\
                6) & $A$ &  &  & 4 D.N. \\
                7) & $B$ &  &  & 1,6 M.P. \\
                8) & $\neg\neg B$ &  &  & 7 D.N. \\
                9) & $\neg D$ &  &  & 3,8 S.D.\\
                10) & $\neg D$ & $\Rightarrow$ & $\neg C$ & 2 Transp. \\
                11) & $\neg C$ &  &  & 9,10 M.P.\\
                12) & $\neg(E\land F)$ &  &  & 5,11 S.D.\\
                \hline
                & & $\therefore$ & $\neg (E\land F)$ & \\
            \end{tabular}
        \end{center}

        De (b):
        \begin{center}
            \begin{tabular}{l l c l r}
                1) & $E$ & $\Rightarrow$ & $(F\land\neg G)$ & Premisa \\
                2) & $(F\lor G)$ & $\Rightarrow$ & $H$ & Premisa \\
                3) & $E$ &  &  & Premisa \\
                4) & $F$ & $\land$ & $\neg G$ & 1,3 M.P. \\
                5) & $F$ &  &  & 4, Simp. \\
                6) & $F$ & $\lor$ & $G$ & 4 Ad. \\
                7) & $H$ &  &  & 2,6 M.P. \\
                \hline
                & & $\therefore$ & $H$ & \\
            \end{tabular}
        \end{center}
    \end{proof}

    \begin{excer}
        Demuestre que existe una demostración formal de los siguientes argumentos:
    \end{excer}

    \begin{proof}
        De (c):
        \begin{center}
            \begin{tabular}{l l c l r}
                1) & $J$ & $\Rightarrow$ & $K$ & Premisa \\
                2) & $J$ & $\lor$ & $(K\lor\neg L)$ & Premisa \\
                3) & $\neg K$ &  &  & Premisa \\
                4) & $\neg J$ &  &  & 1,3 M.T.\\
                5) & $K$ & $\lor$ & $\neg L$ & 2,4 S.D.\\
                6) & $\neg L$ &  &  & 3,5 S.D.\\
                7) & $\neg L$ & $\lor$ & $\neg K$ & 6 Ad. \\
                \hline
                & & $\therefore$ & $\neg L\lor\neg K$ & \\
            \end{tabular}
        \end{center}

        De (d):
        \begin{center}
            \begin{tabular}{l l c l r}
                1) & $(R\Rightarrow\neg S)$ & $\land$ & $(T\Rightarrow\neg U)$ & Premisa \\
                2) & $(V\Rightarrow\neg W)$ & $\land$ & $(X\Rightarrow\neg Y)$ & Premisa \\
                3) & $(T\Rightarrow W)$ & $\land$ & $(U\Rightarrow S)$ & Premisa \\
                4) & $V$ &  &  & Premisa \\
                5) & $R$ &  &  & Premisa \\
                6) & $R$ & $\Rightarrow$ & $\neg S$ & 1 Simp. \\
                7) & $\neg S$ &  &  & 6,5 M.P. \\
                8) & $(U\Rightarrow S)$ & $\land$ & $(T\Rightarrow W)$ & 3 Conm. \\
                9) & $U$ & $\Rightarrow$ & $S$ & 3 Simp. \\
                10) & $\neg U$ &  &  & 9,7 M.T. \\
                11) & $V$ & $\Rightarrow$ & $\neg W$ & 2 Simp. \\
                12) & $\neg W$ &  &  & 4,11 M.P. \\
                13) & $T$ & $\Rightarrow$ & $ W$ & 3 Simp. \\
                14) & $\neg T$ &  &  & 13,12 M.T. \\
                15) & $\neg T$ & $\land$ & $\neg U$ & 14,10 Conj. \\
                \hline
                & & $\therefore$ & $\neg T\land\neg U$ & \\
            \end{tabular}
        \end{center}
    \end{proof}

    \begin{propo}[\textbf{Leyes de DeMorgan}]
        Se cumple que:
        \begin{center}
            \begin{tabular}{c c c}
                $\neg(\varphi$ & $\land$ & $\psi)$ \\
                \hline
                 & $\therefore$ & $\neg\varphi\lor\neg\psi$ \\
            \end{tabular}
        \end{center}
        y,
        \begin{center}
            \begin{tabular}{c c c}
                $\neg(\varphi$ & $\lor$ & $\psi)$ \\
                \hline
                 & $\therefore$ & $\neg\varphi\land\neg\psi$ \\
            \end{tabular}
        \end{center}
    \end{propo}

    \begin{proof}
        En efecto, veamos que
        \begin{center}
            \begin{tabular}{l l c l r}
                1) & $\neg(\varphi$ & $\land$ & $\psi)$ & Premisa \\
                2) & $\neg(\neg(\varphi$ & $\Rightarrow$ & $\neg\psi))$ & 1 R.E. \\
                3) & $\varphi$ & $\Rightarrow$ & $\neg\psi$ & 2 D.N. \\
                4) & $\neg\neg\psi$ & $\Rightarrow$ & $\neg\varphi$ & 3 Transp. \\
                5) & $\neg\psi$ & $\lor$ & $\neg\varphi$ & 4 R.E.\\
                6) & $\neg\varphi$ & $\lor$ & $\neg\psi$ & 5 Conm.\\
                \hline
                & & $\therefore$ & $\neg\varphi\lor\neg\psi$ & \\
            \end{tabular}
        \end{center}
        y,
        \begin{center}
            \begin{tabular}{r l c l r}
                1) & $\neg(\varphi$ & $\lor$ & $\psi)$ & Premisa \\
                2) & $\neg(\neg\varphi$ & $\Rightarrow$ & $\psi)$ & 1 R.E. \\
                3) & $\neg(\neg\varphi$ & $\Rightarrow$ & $\neg\neg\psi)$ & 2 Ax. 3 y M.P. \\
                \hline
                & & $\therefore$ & $\neg \varphi\land\neg\psi$ & \\
            \end{tabular}
        \end{center}
        Falta completar esta prueba.%TODO
        %TODO Hacerlo
    \end{proof}

    \begin{lema}
        $\emptyset\vdash\varphi\Rightarrow\varphi$ para cualquier fórmula $\varphi$.
    \end{lema}

    \begin{proof}
        En efecto, veamos que
        \begin{center}
            \begin{tabular}{r l c l r}
                1) & $\varphi$ & $\Rightarrow$ & $((\psi\Rightarrow\varphi)\Rightarrow\varphi)$ & Ax. 1\\
                2) & $(\varphi\Rightarrow((\psi\Rightarrow\varphi)\Rightarrow\varphi))$ & $\Rightarrow$ & $((\varphi\Rightarrow(\psi\Rightarrow\varphi))\Rightarrow(\varphi\Rightarrow\varphi))$ & Ax. 3\\
                3) & $(\varphi\Rightarrow(\psi\Rightarrow\varphi))$ & $\Rightarrow$ & $(\varphi\Rightarrow\varphi)$ & 2,1 M.P.\\
                4) & $\varphi$ & $\Rightarrow$ & $(\psi\Rightarrow\varphi)$ & Ax. 1\\
                5) & $\varphi$ & $\Rightarrow$ & $\varphi$ & 3,4 M.P.\\
                \hline
                & & $\therefore$ & $\varphi\Rightarrow\varphi$ & \\
            \end{tabular}
        \end{center}
    \end{proof}

    \begin{theor}[\textbf{Metateorema de Deducción}]
        Si $\Sigma$ es un conjunto de fórmulas y $\varphi,\psi$ son fórmulas, entonces:
        \begin{equation*}
            \Sigma\cup\left\{\varphi\right\}\vdash\psi\textup{ si y sólo si }\Sigma\vdash\varphi\Rightarrow\psi
        \end{equation*}
    \end{theor}

    \begin{proof}
        Probemos la suficiencia. Suponga que $\Sigma\vdash\varphi\Rightarrow\psi$, entonces a partir de $\Sigma\cup\left\{\varphi\right\}$ tenemos:
        \begin{center}
            \begin{tabular}{r l c l r}
                $n$) & $\varphi$ &  &  & Premisa \\
                $\vdots$ & \vdots & \vdots & \vdots & Líneas de $\Sigma$ que prueban $\varphi\Rightarrow\psi$ \\
                $n+k$) & $\varphi$ & $\Rightarrow$ & $\psi$ & $\Sigma\vdash\varphi\Rightarrow\psi$ \\
                $n+k+1$) & $\psi$ &  &  & $n$,$n+k$ M.P. \\
                \hline
                & & $\therefore$ & $\psi$ & \\
            \end{tabular}
        \end{center}
        ya que se sabe que existe una demostración de $\varphi\Rightarrow\psi$ a partir de $\Sigma$. Por tanto, $\Sigma\cup\left\{\varphi \right\}\vdash\psi$.

        Para la necesidad, supongamos que existe una demostración $(\varphi_1,....,\varphi_{n-1},\psi)$ de $\psi$ a partir de $\Sigma\cup\left\{\varphi\right\}$, y supongamos por hipótesis de inductiva que $\Sigma\vdash\varphi\Rightarrow\varphi_i$ para cada $i\in\natint{1,n-1}$.

        $\psi$ cumple algunos de los siguiente casos:
        \begin{enumerate}
            \item $\psi$ fue un axioma lógico.
            \item $\psi\in\Sigma$.
            \item $\psi$ es $\varphi$.
            \item Existen $i,j\in\natint{1,n-1}$ tales que $i<j$ y $\varphi_j$ es $\varphi_i\Rightarrow\psi$.
        \end{enumerate}

        Consideremos los casos 1) y 2). En cualquier caso tenemos la siguiente demostración:
        \begin{center}
            \begin{tabular}{r l c l r}
                $\vdots$ & \vdots & \vdots & \vdots & Líneas de $\Sigma$ \\
                $n$) & $\psi$ &  &  & Premisa de $\Sigma$/Axioma Lógico \\
                $n+1$) & $\psi$ & $\Rightarrow$ & $(\varphi\Rightarrow\psi)$ & $n$ Ax.1 \\
                $n+2$) & $\varphi$ & $\Rightarrow$ & $\psi$ & $n+1$,$n$ M.P. \\
                \hline
                & & $\therefore$ & $\varphi\Rightarrow\psi$ & \\
            \end{tabular}
        \end{center}

        Consideremos el caso 3), en cuyo caso por el lema anterior se tiene que $\emptyset\vdash\varphi\Rightarrow\varphi$, esto es que $\emptyset\vdash\varphi\Rightarrow\psi$, en particular $\Sigma\vdash\varphi\Rightarrow\psi$.

        Consideremos ahora el caso 4), se tiene que:
        \begin{center}
            \begin{tabular}{r l c l r}
                $\vdots$ & \vdots & \vdots & \vdots & Líneas de $\Sigma$ \\
                $j$) & $\varphi_i$ & $\Rightarrow$ & $\psi$ & Línea $j$ de $\Sigma$ \\
                $\vdots$ & \vdots & \vdots & \vdots & Líneas de $\Sigma$ \\
                $k$) & $\varphi$ & $\Rightarrow$ & $\varphi_i$ & Deducción de $\varphi\Rightarrow\varphi_i$ \\
                $\vdots$ & $\vdots$ & $\vdots$ & $\vdots$ & Más demostraciones a partir de $\Sigma$ \\
                $l$) & $\varphi$ & $\Rightarrow$ & $(\varphi_i\Rightarrow\psi)$ & $j$  Ax. 1\\
                $l+1$) & $(\varphi\Rightarrow(\varphi_i\Rightarrow\psi))$ & $\Rightarrow$ & $((\varphi\Rightarrow\varphi_i)\Rightarrow(\varphi\Rightarrow\psi))$ & $l$ Ax. 6\\
                $l+2$) & $(\varphi\Rightarrow\varphi_i)$ & $\Rightarrow$ & $(\varphi\Rightarrow\psi)$ & $l+1,l$ M.P.\\
                $l+3$) & $\varphi$ & $\Rightarrow$ & $\psi$ & $l+2,l+2$ M.P.\\
                \hline
                & & $\therefore$ & $\varphi\Rightarrow\psi$ & \\
            \end{tabular}
        \end{center}
        aplicando inducción se tiene lo deseado.
    \end{proof}

    \begin{obs}
        Metateorema de Deducción será denotado por M.D. La palabra Sup. abrevia a Suposición.
    \end{obs}

    \begin{exa}
        Muestre que existe una demostración de $\left\{P\Rightarrow Q,Q\Rightarrow S \right\}\vdash P\Rightarrow S$.
    \end{exa}

    \begin{sol}
        La forma de hacer pruebas con el Metateorema de Deducción será la siguiente: si se quiere probar $P\Rightarrow S$, entonces se supondrá $P$ y se marcará a la izquierda con una linea que nos indicará a partir de suponer $P$, llegar a $S$. Llegando a $S$, cerramos la línea y ponemos línea abajo con $P\Rightarrow S$, indicando las líneas que marcan desde la suposición de $P$ hasta la deducción de $S$, como se muestra en la siguiente demostración:

        \begin{center}
            \begin{tabular}{l r l c l r}
                & 1) & $P$ & $\Rightarrow$ & $Q$ & Premisa \\
                & 2) & $Q$ & $\Rightarrow$ & $S$ & Premisa \\
                $|\longrightarrow$ & 3) & $P$ &  &  & Sup. \\
                $|$ & 4) & $Q$ &  &  & 1,3 M.P. \\
                $|$ & 5) & $S$ &  &  & 2,4 M.P. \\
                \hline
                & 6) & $P$ & $\Rightarrow$ & $S$ & 2-5 M.D.\\
                \hline
                & & & $\therefore$ & $Q$ & \\
            \end{tabular}
        \end{center}
    \end{sol}

    \begin{theor}[\textbf{Metateorema (Demostración por Contradicción)}]
        Si $\Sigma$ es un conjunto de fórmulas y $\varphi,\psi$ son fórmulas, entonces
        \begin{equation*}
            \Sigma\cup\left\{\neg\psi \right\}\vdash\varphi\land\neg\varphi\textup{ si y sólo si }\Sigma\vdash\psi
        \end{equation*}
    \end{theor}

    \begin{proof}
        Necesidad, supongamos que $\Sigma\cup\left\{\neg\psi \right\}\vdash\varphi\land\neg\varphi$. Entonces,
        \begin{center}
            \begin{tabular}{l r l c l r}
                & $\vdots$ & $\vdots$ & $\vdots$ & $\vdots$ & Líneas de $\Sigma$ \\
                $|\longrightarrow$ & $k$) & $\neg\psi$ &  &  & Suposición \\
                $|$ & $\vdots$ & $\vdots$ & $\vdots$ & $\vdots$ & Líneas de $\Sigma\cup\left\{\neg\psi\right\}$ que prueban $\varphi\land\neg\varphi$ \\
                $|$ & $m$) & $\varphi$ & $\land$ & $\neg\varphi$ &  \\
                \hline
                 & $m+1$) & $\neg\psi$ & $\Rightarrow$ & $(\varphi\land\neg\varphi)$ & $k$-$m$ M.D. \\
                 & $m+2$) & $\neg(\varphi\land\neg\varphi)$ & $\Rightarrow$ & $\neg\neg\psi$ & $m+1$ Transp. \\
                 & $m+3$) & $\neg(\varphi\land\neg\varphi)$ & $\Rightarrow$ & $\psi$ & $m+2$ M.P. y D.N. \\
                $|\longrightarrow$ & $m+4$) & $\neg\psi$ &  &  & Sup. \\
                $|$ & $m+5$) & $\varphi$ & $\land$ & $\neg\varphi$ & $m$,$m+4$ M.P. \\
                $|$ & $m+6$) & $\neg\varphi$ & $\land$ & $\varphi$ & $m+5$ Conm. \\
                $|$ & $m+7$) & $\neg\varphi$ &  &  & $m+6$ Simp. \\
                $|$ & $m+8$) & $\neg\varphi$ & $\lor$ & $\neg\neg\varphi$ & $m+7$ Ad. \\
                $|$ & $m+9$) & $(\neg\varphi\lor\neg\neg\varphi)$ & $\Rightarrow$ & $\neg(\varphi\land\neg\varphi)$ & $m+8$ DeMorgan \\
                $|$ & $m+10$) & $\neg(\varphi$ & $\land$ & $\neg\varphi)$ & $m+9$,$m+8$ M.P. \\
                $|$ & $m+11$) & $\psi$ &  &  & $m+3$,$m+10$ M.P. \\
                \hline
                 & $m+12$) & $\neg\psi$ & $\Rightarrow$ & $\psi$ & $m+4$-$m+11$ M.D. \\
                 & $m+13$) & $\psi$ &  &  & $m+12$ Tautología \\
                \hline
                & & & $\therefore$ & $\psi$ & \\
            \end{tabular}
        \end{center}
        por tanto, $\Sigma\vdash\psi$.

        Suficiencia, supongamos que $\Sigma\vdash\psi$. Partamos de $\Sigma\cup\left\{\neg\psi\right\}$:
        \begin{center}
            \begin{tabular}{l r l c l r}
                 & 1) & $\neg\psi$ &  &  & Premisa \\
                & $\vdots$ & $\vdots$ & $\vdots$ & $\vdots$ & Líneas de $\Sigma$ \\
                & $\vdots$ & $\vdots$ & $\vdots$ & $\vdots$ & Líneas de $\Sigma$ que prueban $\psi$ \\
                & $n$) & $\psi$ &  &  & Prueba de $\varphi$ a partir de $\Sigma$ \\
                & $n+1$) & $\psi$ & $\lor$ & $(\varphi\land\neg\varphi)$ & $n$ Ad. \\
                & $n+2$) & $\varphi$ & $\land$ & $\neg\varphi$ & $n+1$ S.D. \\
                \hline
                & & & $\therefore$ & $\varphi\land\neg\varphi$ & \\
            \end{tabular}
        \end{center}
        por tanto, $\Sigma\cup\left\{\neg\psi\right\}\vdash\varphi\land\neg\varphi$.
    \end{proof}

    \begin{excer}
        Complete las siguientes demostraciones.
    \end{excer}

    \begin{sol}
        De 1):
        \begin{center}
            \begin{tabular}{l r l c l r}
                & 1) & $M$ & $\Rightarrow$ & $N$ & Premisa \\
                & 2) & $N$ & $\Rightarrow$ & $O$ & Premisa \\
                & 3) & $(M\Rightarrow O)$ & $\Rightarrow$ & $(N\Rightarrow P)$ & Premisa \\
                & 4) & $(M\Rightarrow P)$ & $\Rightarrow$ & $Q$ & Premisa \\
                $|\longrightarrow$ & 5) & $M$ &  &  & Sup. \\
                $|$ & 6) & $N$ &  &  & 1,5 M.P. \\
                $|$ & 7) & $O$ &  &  & 2,6 M.P. \\
                \hline
                & 8) & $M$ & $\Rightarrow$ & $O$ & 5-7 M.D. \\
                & 9) & $N$ & $\Rightarrow$ & $P$ & 3,8 M.P. \\
                $|\longrightarrow$ & 10) & $M$ &  &  & Sup. \\
                $|$ & 11) & $N$ &  &  & 2,10 M.P. \\
                $|$ & 12) & $P$ &  &  & 9,11 M.P. \\
                \hline
                 & 13) & $M$ & $\Rightarrow$ & $P$ & 10-12 M.D.\\
                 & 14) & $Q$ &  &  & 13, 4 M.P.\\
                \hline
                & & & $\therefore$ & $Q$ & \\
            \end{tabular}
        \end{center}

        De 2):
        \begin{center}
            \begin{tabular}{l r l c l r}
                & 1) & $V$ & $\Rightarrow$ & $W$ & Premisas \\
                & 2) & $X$ & $\Rightarrow$ & $Y$ & Premisas \\
                & 3) & $Z$ & $\Rightarrow$ & $W$ & Premisas \\
                & 4) & $X$ & $\Rightarrow$ & $A$ & Premisas \\
                & 5) & $W$ & $\Rightarrow$ & $X$ & Premisas \\
                & 6) & $((V\Rightarrow Y)\land(Z\Rightarrow A))$ & $\Rightarrow$ & $V\lor Z$ & Premisas \\
                \hline
                & & & $\therefore$ & $\neg Y\Rightarrow A$ & \\
            \end{tabular}
        \end{center}
        Falta completar esta demostración, es un poco larga de hacer (el profesor la hizo en clase).
        %TODO

        De 3):
        \begin{center}
            \begin{tabular}{l r l c l r}
                & 1) & $P$ & $\Rightarrow$ & $(Q\land R)$ & Premisa \\
                & 2) & $R$ & $\Rightarrow$ & $(A\land B)$ & Premisa \\
                $|\longrightarrow$& 3) & $P$ &  &  & Sup. \\
                $|$& 4) & $Q$ & $\land$ & $R$ & 1,3 M.P.\\
                $|$& 5) & $R$ & $\land$ & $Q$ & 4 Conm.\\
                $|$& 6) & $R$ &  &  & 5 Simp.\\
                $|$& 7) & $A$ & $\land$ & $B$ & 6,2 M.P.\\
                $|$& 8) & $A$ &  &  & 6,2 M.P.\\
                $|$& 9) & $A$ & $\Rightarrow$ & $(Q\Rightarrow A)$ & Ax. 1\\
                $|$& 10) & $Q$ & $\Rightarrow$ & $A$ & 9,8 M.P.\\
                \hline
                & 10) & $P$ & $\Rightarrow$ & $(Q\Rightarrow A)$ & 3-10 M.D.\\
                \hline
                & & & $\therefore$ & $P\Rightarrow(Q\Rightarrow A)$ & \\
            \end{tabular}
        \end{center}
    \end{sol}

    \begin{excer}
        Complete las siguientes demostraciones.
    \end{excer}

    \begin{sol}
        De (1):
        \begin{center}
            \begin{tabular}{l r l c l r}
                & 1) & $Q$ &  &  & Premisa \\
                & 2) & $Q$ & $\Rightarrow$ & $(P\Rightarrow Q)$ & 1 Ax. 1 \\
                & 3) & $P$ & $\Rightarrow$ & $Q$ & 2,1 M.P. \\
                \hline
                & & & $\therefore$ & $P\Rightarrow Q$ & \\
            \end{tabular}
        \end{center}
        De (2):
        \begin{center}
            \begin{tabular}{l r l c l r}
                & 1) & $(P\Rightarrow Q)$ & $\land$ & $(C\Rightarrow D)$ & Premisa \\
                & 2) & $(Q\lor D)$ & $\Rightarrow$ & $((E\Rightarrow(E\lor F))\Rightarrow P\land C)$ & Premisa \\
                & 3) & $P$ & $\Rightarrow$ & $Q$ & 1 Simp. \\
                & 4) & $C$ & $\Rightarrow$ & $D$ & 1 Conm. y Simp. \\
                $|\longrightarrow$ & 5) & $E$ &  &  & Sup. \\
                $|$ & 6) & $E$ & $\lor$ & $F$ & 5 Ad. \\
                \hline
                 & 7) & $E$ & $\Rightarrow$ & $(E\lor F)$ & 5-6 M.D. \\
                $|\longrightarrow$ & 8) & $C$ &  &  & Sup. \\
                $|$ & 9) & $D$ &  &  & 8,4 M.P. \\
                $|$ & 10) & $Q$ & $\lor$ & $D$ & 9 Ad. y Conm. \\
                $|$ & 11) & $(E\Rightarrow(E\Rightarrow D)$ & $\Rightarrow$ & $P\land C$ & 2,10 M.P. \\
                $||\longrightarrow$ & 12) & $\neg P$ &  &  & Sup. \\
                $||$ & 13) & $\neg P$ & $\lor$ & $\neg C$ & 12 Ad. \\
                $||$ & 14) & $\neg (P$ & $\land$ & $C)$ & 13 DeMorgan \\
                $||$ & 15) & $\neg (E$ & $\Rightarrow$ & $(E\lor F))$ & 11,14 M.T. \\
                $||$ & 16) & $(E\Rightarrow(E\lor F))$ & $\land$ & $\neg(E\Rightarrow(E\lor F))$ & 7,15 Ad.\\
                \hline
                $|$ & 17) & $P$ &  &  & 12-16 D.C. \\
                \hline
                 & 18) & $C$ & $\Rightarrow$ & $P$ & 18-17 D.C. \\
                $|\longrightarrow$ & 19) & $P$ &  &  & Sup. \\
                & & & $\therefore$ & $P\iff C$ & \\
            \end{tabular}
        \end{center}
        De (3):
        \begin{center}
            \begin{tabular}{l r l c l r}
                & 1) & $P$ & $\lor$ & $(Q\land R)$ & Premisa \\
                & 2) & $P$ & $\Rightarrow$ & $R$ & Premisa \\
                $|\longrightarrow$& 3) & $\neg R$ &  &  & Sup. \\
                $|$& 4) & $\neg P$ &  &  & 2,3 M.T. \\
                $|$& 5) & $Q$ & $\land$ & $R$ & 1,4 S.D. \\
                $|$& 6) & $R$ &  &  & 5 Conm. y Simp. \\
                $|$& 7) & $R$ & $\land$ & $\neg R$ & 6,3 Conj. \\
                \hline
                & 8) & $R$ &  &  & 3-7 M.D.C. \\
                \hline
                & & & $\therefore$ & $R$ & \\
            \end{tabular}
        \end{center}
        De (4):
        \begin{center}
            \begin{tabular}{l r l c l r}
                & 1) & $(P\lor Q)$ & $\Rightarrow$ & $(R\Rightarrow D)$ & Premisa \\
                & 2) & $(\neg D\lor E)$ & $\Rightarrow$ & $(P\land R)$ & Premisa \\
                $|\longrightarrow$ & 3) & $\neg D$ &  &  & Sup. \\
                $|$ & 4) & $\neg D$ & $\lor$ & $E$ & 3 Ad. \\
                $|$ & 5) & $P$ & $\land$ & $R$ & 2,4 M.P. \\
                $|$ & 6) & $P$ &  &  & 5 Simp. \\
                $|$ & 7) & $P$ & $\lor$ & $Q$ & 6 Ad. \\
                $|$ & 8) & $R$ & $\Rightarrow$ & $D$ & 1,7 M.P. \\
                $|$ & 9) & $R$ &  &  & 5 Conm. y Simp. \\
                $|$ & 10) & $D$ &  &  & 8,9 M.P. \\
                $|$ & 11) & $D$ & $\land$ & $\neg D$ & 10,3 Conj. \\
                \hline
                & 12) & $D$ &  &  & 3-11 D.C.\\
                \hline
                & & & $\therefore$ & $D$ & \\
            \end{tabular}
        \end{center}
        De (5):
        \begin{center}
            \begin{tabular}{l r l c l r}
                & 1) & $(P\lor Q)$ & $\Rightarrow$ & $(R\land D)$ & Premisa \\
                & 2) & $(R\land F)$ & $\Rightarrow$ & $(\neg F\land G)$ & Premisa \\
                & 3) & $(F\lor H)$ & $\Rightarrow$ & $(P\land I)$ & Premisa \\
                $|\longrightarrow$& 4) & $F$ &  &  & Sup. \\
                $|$& 5) & $F$ & $\lor$ & $H$ & 4 Ad. \\
                $|$& 6) & $P$ & $\land$ & $I$ & 2,5 M.P. \\
                $|$& 7) & $P$ &  &  & 6 Simp. \\
                $|$& 8) & $R$ & $\land$ & $D$ & 1,7 M.P. \\
                $|$& 9) & $R$ &  &  & 8 Simp. \\
                $|$& 10) & $R$ & $\land$ & $F$ & 9,4 Conj. \\
                $|$& 11) & $\neg F$ & $\land$ & $G$ & 2,10 M.P. \\
                $|$& 12) & $\neg F$ &  &  & 11 Simp. \\
                $|$& 13) & $F$ & $\land$ & $\neg F$ & 4,12 Conj. \\
                \hline
                & 14) & $\neg F$ &  &  & 4-13 D.C. \\
                \hline
                & & & $\therefore$ & $\neg F$ & \\
            \end{tabular}
        \end{center}
    \end{sol}

    \begin{mydef}
        Un conjunto de fórmulas $\Sigma$ es \textbf{consistente} si para cualquier fórmula $\varphi$, $\Sigma\nvdash\varphi\land\neg\varphi$. $\Sigma$ es \textbf{inconsistente} si no es consistente.
    \end{mydef}

    \begin{theor}[\textbf{Explosividad de la Lógica}]
        Si $\Sigma$ es un conjunto de fórmulas bien formadas que es inconsistente, entonces para toda fórmula $\varphi$, $\Sigma\vdash\varphi$.
    \end{theor}

    \begin{proof}
        Sea $\varphi$ una fórmula. Como $\Sigma$ no es consistente, existe $\psi$ fórmula tal que $\Sigma\vdash\psi\land\neg\psi$. Se tiene que:
        \begin{center}
            \begin{tabular}{l r l c l r}
                & $\vdots$ & $\vdots$ & $\vdots$ & $\vdots$ & Líneas de $\Sigma$. \\
                & $n$) & $\psi$ & $\land$ & $\neg\psi$ & Suposición. \\
                $|\longrightarrow$ & $n+1$) & $\psi$ &  &  & $n$ Simp. \\
                $|$ & $n+2$) & $\psi$ & $\lor$ & $\varphi$ & $n+1$ Ad. \\
                $|$ & $n+3$) & $\neg\psi$ &  &  & $n$ Conm. y Simp. \\
                $|$ & $n+4$) & $\varphi$ &  &  & $n+2$,$n+3$ S.D. \\
                \hline
                & $n+5$) & $\psi\land\neg\psi$ & $\Rightarrow$ & $\varphi$ & $n$-$n+4$ M.D. \\
                & $\vdots$ & $\vdots$ & $\vdots$ & $\vdots$ & Líneas de $\Sigma$ que prueban $\psi\land\neg\psi$. \\
                & $m$) & $\psi\land\neg\psi$ &  &  &  \\
                & $m+1$) & $\varphi$ &  &  & $n+5$,$m$ M.P. \\
                \hline
                & & & $\therefore$ & $\varphi$ & \\
            \end{tabular}
        \end{center}
    \end{proof}

    \begin{cor}
        Un conjunto de fórmulas $\Sigma$ es consistente si y sólo si existe una fórmula $\varphi$ tal que $\Sigma\nvdash\varphi$.
    \end{cor}

    \begin{proof}
        Necesidad: Suponga que $\Sigma$ es consistente, entonces para una fórmula $\psi$ siempre sucede que $\Sigma\nvdash\psi\land\neg\psi$. Tomando $\varphi=\psi\land\neg\psi$ se sigue el resultado.

        Suficiencia: Suponga que $\Sigma$ no es consistente, entonces por el Teorema anterior $\Sigma\vdash\varphi$ para toda fórmula $\varphi$.
    \end{proof}

    \begin{lema}[\textbf{Demostración por casos}]
        \label{demPorCas}
        Si $\Sigma$ es un conjunto de FBF y $\varphi,\psi$ son fórmulas tales que $\Sigma\cup\left\{\varphi\right\}\vdash\psi$ y $\Sigma\cup\left\{\neg\varphi\right\}\vdash\psi$, entonces $\Sigma\vdash\psi$.
    \end{lema}

    \begin{proof}
        Por las hipótesis y el Metateorema de Deducción,
        \begin{equation*}
            \Sigma\vdash\varphi\Rightarrow\psi\quad\textup{y}\quad \Sigma\vdash\neg\varphi\Rightarrow\psi
        \end{equation*}
        Veamos que
        \begin{center}
            \begin{tabular}{l r l c l r}
                & $\vdots$ & $\vdots$ & $\vdots$ & $\vdots$ & Líneas de $\Sigma$. \\
                $|\longrightarrow$ & $n$) & $\neg\psi$ &  &  & Suposición. \\
                $|$ & $\vdots$ & $\vdots$ & $\vdots$ & $\vdots$ & Líneas de $\Sigma$ que prueban $\varphi\Rightarrow\psi$ \\
                $|$ & $m_1$) & $\varphi$ & $\Rightarrow$ & $\psi$ &  \\
                $|$ & $\vdots$ & $\vdots$ & $\vdots$ & $\vdots$ & Líneas de $\Sigma$ que prueban $\neg\varphi\Rightarrow\psi$ \\
                $|$ & $m_2$) & $\neg\varphi$ & $\Rightarrow$ & $\psi$ &  \\
                $|$ & $m_2+1$) & $\neg\varphi$ &  &  & $m_1$,$n$ M.T. \\
                $|$ & $m_2+2$) & $\neg\neg\varphi$ &  &  & $m_2$,$n$ M.T. \\
                $|$ & $m_2+3$) & $\varphi$ &  &  & $m_2+2$ D.N. \\
                $|$ & $m_2+4$) & $\varphi$ & $\land$ & $\neg\varphi$ & $m_2+3$,$m_2+1$ Ad. \\
                \hline
                & $m_2+5$) & $\psi$ &  &  & $n$-$m_2+4$ D.C. \\
                \hline
                & & & $\therefore$ & $\psi$ & \\
            \end{tabular}
        \end{center}
        lo que prueba el resultado.
    \end{proof}

    \begin{cor}
        Sea $\Sigma$ un conjunto de fórmulas bien formadas y sea $\varphi$ una fórmula. Entonces:
        \begin{enumerate}
            \item $\Sigma\nvdash\varphi$ si y sólo si $\Sigma\cup\left\{\neg\varphi\right\}$ es consistente.
            \item $\Sigma\vdash\varphi$ si y sólo si $\Sigma\cup\left\{\neg\varphi\right\}$ es inconsistente.
        \end{enumerate}
    \end{cor}

    \begin{proof}
        De (1):

        $\Rightarrow$): Suponga que $\Sigma\cup\left\{\varphi \right\}$ es inconsistente, entonces por el tereoam anterior
        \begin{equation*}
            \Sigma\cup\left\{\varphi \right\}\vdash\neg\varphi
        \end{equation*}
        y, de forma inmediata se tiene:
        \begin{equation*}
            \Sigma\cup\left\{\varphi\right\}\vdash\varphi
        \end{equation*}
        por tanto, por el Lema anterior se sigue que $\Sigma\vdash\varphi$.

        $\Leftarrow$): %TODO

        De (2): Veamos las dos implicaciones:

        $\Rightarrow$): Suponga que $\Sigma\vdash\varphi$, entonces existe una demostración $(\varphi_1,...,\varphi_{ n-1},\varphi)$ de $\varphi$ a partir de $\Sigma$.

        En particular, se tiene a partir de $\Sigma\cup\left\{\neg\varphi\right\}$ que:
        \begin{center}
            \begin{tabular}{l r l c l r}
                & $\vdots$ & $\vdots$ & $\vdots$ & $\vdots$ & Líneas de $\Sigma$ que prueban $\varphi$. \\
                & $n$) & $\varphi$ &  &  & \\
                & $n+1$) & $\neg\varphi$ &  &  & Premisa. \\
                & $n+2$) & $\varphi$ & $\land$ & $\neg\varphi$ & $n,n+1$ Ad. \\
                \hline
                & & & $\therefore$ & $\varphi\land\neg\varphi$ & \\
            \end{tabular}
        \end{center}
        por tanto, $\Sigma\cup\left\{\neg\phi\right\}\vdash\varphi\land\neg\varphi$. Se sigue que $\Sigma\cup\left\{\neg\varphi\right\}$ es inconsistente.

        $\Leftarrow$): Suponga que $\Sigma\cup\left\{\neg\varphi\right\}$ es inconsistente, entonces existe una fórmula $\psi$ tal que $\Sigma\cup\left\{\neg\varphi\right\}\vdash\psi\land\neg\psi$. A partir de $\Sigma$ tenemos que:
        \begin{center}
            \begin{tabular}{l r l c l r}
                & $\vdots$ & $\vdots$ & $\vdots$ & $\vdots$ & Líneas de $\Sigma$. \\
                $|\longrightarrow$ & $n$) & $\neg\varphi$ &  &  & Sup. \\
                $|$ & $\vdots$ & $\vdots$ & $\vdots$ & $\vdots$ & Líneas de $\Sigma\cup\left\{\neg\varphi\right\}$ que prueban $\psi\land\neg\psi$.\\
                $|$ & $m$) & $\psi$ & $\land$ & $\neg\psi$ & \\
                \hline
                & $m$) & $\varphi$ &  &  & $m$-$n$ D.C. \\
                \hline
                & & & $\therefore$ & $\varphi$ & \\
            \end{tabular}
        \end{center}
        por tanto, $\Sigma\vdash\varphi$.
    \end{proof}

    \begin{mydef}
        Un conjunto de FBF $\Sigma$ es \textbf{satisfacible} si existe un modelo $m$ tal que $m\vDash\Sigma$.
    \end{mydef}

    \begin{lema}
        Si $\Gamma$ es un conjunto de fórmulas consistente, y $\varphi$ es cualquier otra fórmula, entonces o bien $\Gamma\cup\left\{\varphi\right\}$ es consistente, o lo es $\Gamma\cup\left\{\neg\varphi\right\}$
    \end{lema}

    \begin{proof}
        Sea $\Gamma$ un conjunto de fórmulas consistente. Supongamos que no, es decir, tanto $\Gamma\cup\left\{\varphi\right\}$ y $\Gamma\cup\left\{\neg\varphi\right\}$ son ambos inconsistentes. Entonces
        \begin{equation*}
            \Gamma\cup\left\{\varphi\right\}\vdash\psi\land\neg\psi\quad\textup{y}\quad\Gamma\cup\left\{\neg\varphi\right\}\vdash\psi\land\neg\psi
        \end{equation*}
        para algúna FBF $\psi$. Por tanto, del Lema 1.3.2 se tiene que
        \begin{equation*}
            \Gamma\vdash\psi\land\neg\psi
        \end{equation*}
        por tanto $\Gamma$ es inconsistente\contradiction. Por tanto, uno de los dos:
        \begin{equation*}
            \Gamma\cup\left\{\varphi\right\}\quad\textup{y}\quad\Gamma\cup\left\{\neg\varphi\right\}
        \end{equation*}
        debe ser consistente.
    \end{proof}

    \begin{lema}
        \label{demConsist}
        Para todo conjunto de fórmulas $\Sigma$, si $\Sigma$ es consistente, entonces es satisfacible.
    \end{lema}

    \begin{proof}
        Haremos la demostración paso a paso:
        \begin{itemize}
            \item \textbf{Enumerar todas las fórmulas}: La idea va a ser usar \textit{codificación}. Hacemos lo siguiente:
            \begin{center}
                \begin{tabular}{ccc}
                    $\neg$ & $\longleftrightarrow$ & $1$ \\
                    $\Rightarrow$ & $\longleftrightarrow$ & $2$ \\
                    $p_1$ & $\longleftrightarrow$ & $3$ \\
                    $p_2$ & $\longleftrightarrow$ & $4$ \\
                    $p_3$ & $\longleftrightarrow$ & $5$ \\
                    $\vdots$ & $\vdots$ & $\vdots$ \\
                    $p_n$ & $\longleftrightarrow$ & $n+2$ \\
                    $\vdots$ & $\vdots$ & $\vdots$ \\
                \end{tabular}
            \end{center}
            siendo $p_1,p_2,p_3,...$ variables y $\neg,\Rightarrow$ conectivas.

            Recordemos antes que existe una biyección entre $\mathbb{N}\times\mathbb{N}$ y $\mathbb{N}$, dada por:
            \begin{equation*}
                (n,m)\mapsto 2^{ n-1}(2m-1)
            \end{equation*}
            ó
            \begin{equation*}
                (n,m)\mapsto\frac{(n+m-1)(n+m)}{2}+m
            \end{equation*}
            para todo $n,m\in\mathbb{N}$ (verifique que esto es cierto). En general, uno puede establecer una biyección entre $\mathbb{N}^n$ y $\mathbb{N}$, para todo $n\in\mathbb{N}$. Lo que nos va a interesar es codificar sucesiones finitas de esta manera.

            Uno puede hacer lo siguiente: si $n\in\mathbb{N}$, podemos escribir:
            \begin{equation*}
                n=p_1^{\alpha_1}\cdot...\cdot p_k^{\alpha_k}
            \end{equation*}
            donde $\left\{p_n \right\}_{ n=1}^\infty$ es la sucesión de números primos ordenados de menor a mayor y $\alpha_i\in\mathbb{N}\cup\left\{0\right\}$, para todo $i\in\natint{1.k}$. En este caso, $p_k$ es el mayor primo para el cual $\alpha_k\neq0$. De esta fórmula, lo que haremos es codificar tuplas, a $n$ le corresponde la $k$-tupla de números naturales:
            \begin{equation*}
                n\leftrightsquigarrow (\alpha_1+1,\alpha_2+1,...,\alpha_ {k-1}+1,\alpha_k)
            \end{equation*}
            por ejemplo, en este caso la fórmula $\neg p_3=(\neg,p_3)$ se codificaría como tupla en el número $243$ (asignándole lo establecido a $\neg$ y $p_3$).

            Con esta codificación, se tiene que:
            \begin{center}
                \begin{tabular}{ccccc}
                    $1$ & $\longleftrightarrow$ & $\emptyset$ & $\equiv$ & $\emptyset$ \\
                    $2$ & $\longleftrightarrow$ & $(1)$ & $\equiv$ & $\neg$ \\
                    $3$ & $\longleftrightarrow$ & $(1,1)$ & $\equiv$ & $\ne\neg$ \\
                    $4$ & $\longleftrightarrow$ & $(2)$ & $\equiv$ & $\Rightarrow$ \\
                    $5$ & $\longleftrightarrow$ & $(1,1,1)$ & $\equiv$ & $\neg\neg\neg$ \\
                    $6$ & $\longleftrightarrow$ & $(2,1)$ & $\equiv$ & $\Rightarrow\neg$ \\
                    $7$ & $\longleftrightarrow$ & $(1,1,1,1)$ & $\equiv$ & $\neg\neg\neg\neg$ \\
                    $8$ & $\longleftrightarrow$ & $(3)$ & $\equiv$ & $p_1$ \\
                    $\vdots$ & $\vdots$ & $\vdots$ \\
                    $p_n$ & $\longleftrightarrow$ & $n+2$ \\
                    $\vdots$ & $\vdots$ & $\vdots$ \\
                \end{tabular}
            \end{center}
            en este caso, la primera fórmula bien formada es la que codifica el número 8, que llamaremos $\varphi_1$. Con este procedimiento enumeramos cada una de las fórmulas bien formadas.

            \item Ahora si con la demostración. Sea $\Sigma$ un conjunto consistente de fórmulas. Sea $\varphi_1,\varphi_2,...,\varphi_n,...$ una \textbf{enumeración efectiva} de todas las fórmulas bien formadas (esto es, con el procedimiento hecho anteriormente y yendo en correspondencia con el mismo).
            
            Recursivamente definimos
            \begin{equation*}
                \begin{split}
                    \Sigma=\Sigma_0\subseteq\Sigma_1\subseteq\Sigma_2\subseteq\cdots\subseteq\Sigma_n\subseteq\cdots
                \end{split}
            \end{equation*}
            de la manera siguiente:
            \begin{equation*}
                \Sigma_{ n+1}=\left\{
                    \begin{array}{lcr}
                        \Sigma_n\cup\left\{\varphi_{n+1}\right\} & \textup{ si } & \Sigma_n\cup\left\{\varphi_{n+1}\right\} \textup{ es consistente.}\\
                        \Sigma_n\cup\left\{\neg\varphi_{n+1}\right\} & \textup{ e.o.c. } & \\
                    \end{array}
                \right.,\quad\forall n\in\mathbb{N}
            \end{equation*}
            del Lema 1.3.3 se sigue que $\Sigma_n$ es consistente, para todo $n\in\mathbb{N}$, y además cumple que:
            \begin{equation*}
                \Sigma=\Sigma_0\subseteq\Sigma_1\subseteq\Sigma_2\subseteq\cdots\subseteq\Sigma_n\subseteq\cdots
            \end{equation*}
            Definimos
            \begin{equation*}
                \Sigma_{\infty}=\bigcup_{ n=1}^\infty\Sigma_n
            \end{equation*}
            Se tiene pues que:
            \begin{enumerate}
                \item $\Sigma_{\infty}$ es consistente.
                \item Para cada fórmula $\varphi$, o bien $\varphi\in\Sigma_\infty$ o $\neg\varphi\in\Sigma_{\infty}$.
                \item $\Sigma_\infty\vdash\varphi$ si y sólo si $\varphi\in\Sigma_{\infty}$.
            \end{enumerate}
            Los incisos (2) y (3) son inmediatos de la construcción de cada $\Sigma_n$. Para el inciso (1): Suponga que $\Sigma_\infty$ es inconsistente, entonces existe una FBF tal que
            \begin{equation*}
                \Sigma_\infty\vdash\psi\land\neg\psi
            \end{equation*}
            Sean $\psi_1,...,\psi_m$ los elementos de $\Sigma_\infty$ usados en esta demostración. Para cada $i\in\natint{1,m}$, sea $n_i$ tal que $\psi_i\in\Sigma_{ n_i}$. Entonces
            \begin{equation*}
                \psi_i\in\Sigma_{n}\quad\forall i\in\natint{1,m}
            \end{equation*}
            donde $n=\max_{1\leq i\leq m}n_i$. Se sigue entonces que $\Sigma_n$ no es consistente\contradiction. Por tanto, $\Sigma_\infty$ es consistente.

            Definimos el modelo $\cf{m}{\textup{Var}}{\left\{V,F\right\}}$ mediante:
            \begin{equation*}
                m(p_i)=V\textup{ si y sólo si }p_i\in\Sigma_{\infty}
            \end{equation*}
            y, también
            \begin{equation*}
                m(\neg p_i)=V\textup{ si y sólo si }\neg p_i\in\Sigma_{\infty}\textup{ si y sólo si }\Sigma_\infty\nvdash p_i
            \end{equation*}

            Afirmamos que para toda fórmula $\varphi$, $m\vDash\varphi$ si y sólo si $\varphi\in\Sigma_{\infty}$. En efecto, procederemos por induccción sobre $\varphi$.
            \begin{itemize}
                \item El caso base es inmediato.
                \item Veamos que para una FBF:
                \begin{equation*}
                    \begin{split}
                        m\vDash\neg\varphi&\textup{ si y sólo si }\overline{m}(\neg\varphi)=V\\
                        &\textup{ si y sólo si }\overline{m}(\varphi)=F\\
                        &\textup{ si y sólo si }\varphi\notin\Sigma_{\infty}\\
                        &\textup{ si y sólo si }\neg\varphi\in\Sigma_{\infty}\\
                    \end{split}
                \end{equation*}
                y, si tenemos una fórmula de la forma $\varphi\Rightarrow\psi$, donde $m\vDash\varphi$ y $m\vDash\psi$. Veamos que:
                \begin{equation*}
                    \begin{split}
                        \overline{m}\nvDash\Rightarrow\varphi\psi&\textup{ si y sólo si }\overline{m}(\Rightarrow\varphi\psi)=F\\
                        &\textup{ si y sólo si }\overline{m}(\varphi)=V\textup{ y }\overline{m}(\psi)=F\\
                        &\textup{ si y sólo si }m\vDash\varphi\textup{ y }m\nvDash\psi\\
                        &\textup{ si y sólo si }\varphi\in\Sigma_{\infty}\textup{ y }\psi\notin\Sigma_{\infty}\\
                        &\textup{ si y sólo si }\varphi\in\Sigma_{\infty}\textup{ y }\neg\psi\in\Sigma_{\infty}\\
                    \end{split}
                \end{equation*}
                si $\Rightarrow\varphi\psi\in\Sigma_{\infty}$, entonces como $\varphi,\neg\psi\in\Sigma_\infty$, se puede demostrar que $\psi\land\neg\psi\in\Sigma_{\infty}$, es decir que $\Sigma_{\infty}$ es inconsistente. Por tanto:
                \begin{equation*}
                    \begin{split}
                        \overline{m}\nvDash\Rightarrow\varphi\psi&\textup{ implica }\Rightarrow\varphi\psi\notin\Sigma_{\infty}\\
                    \end{split}
                \end{equation*}
                para la otra implicación, si $\Rightarrow\varphi\psi\notin\Sigma_{\infty}$, entonces $\neg\Rightarrow\varphi\psi\in\Sigma_\infty$. Pero:
                \begin{equation*}
                    \begin{split}
                        \neg\Rightarrow\varphi\psi&\equiv\neg(\varphi\Rightarrow\psi)\\
                        &\equiv\neg(\neg\neg\varphi\Rightarrow\psi)\\
                        &\equiv\neg(\neg\varphi\lor\psi)\\
                        &\equiv\neg\neg\varphi\land\neg\psi\\
                        &\equiv\varphi\land\neg\psi\\
                    \end{split}
                \end{equation*}
                en particular, se sigue que $\varphi\in\Sigma_{\infty}$ y $\neg\psi\in\Sigma_{\infty}$, lo cual prueba
                \begin{equation*}
                    \begin{split}
                        \overline{m}\nvDash\Rightarrow\varphi\psi&\textup{ si }\Rightarrow\varphi\psi\notin\Sigma_{\infty}\\
                    \end{split}
                \end{equation*}
                por ende:
                \begin{equation*}
                    \begin{split}
                        \overline{m}\nvDash\Rightarrow\varphi\psi&\textup{ si y sólo si }\Rightarrow\varphi\psi\notin\Sigma_{\infty}\\
                    \end{split}
                \end{equation*}
            \end{itemize}
            aplicando inducción, se sigue que $m\vDash\varphi$ si y sólo si $\varphi\in\Sigma_{\infty}$.

            Por tanto, $m\vDash\Sigma_{\infty}$, en particular $m\vDash\Sigma$. Por ende, $\Sigma$ es satisfacible.
        \end{itemize}
    \end{proof}

    \begin{obs}
        Note a su vez que con el procedimiento del lema anterior, también se pueden codificar tuplas de tuplas (en esencia la idea es que podemos codificar una demostración). Esto servirá más adelante para computabilidad.
    \end{obs}

    \begin{obs}
        Probado este lema anterior, tenemos que:
        \begin{center}
            $\Sigma\vDash\varphi$ implica que no existe $m\vDash\Sigma$ tal que $m\nvDash\varphi$;

            lo cual implica que no existe $m\vDash\Sigma$ tal que $m\vDash\neg\varphi$.
        \end{center}
        Esto significa que
        \begin{center}
            $\Sigma\cup\left\{\neg\varphi \right\}$ no es satisfacible.
        \end{center}
        lo cual implica por el Lema (\ref{demConsist}) que $\Sigma\cup\left\{\neg\varphi\right\}$ es inconsistente. Esto implica que $\Sigma\vdash\varphi$.
    \end{obs}

    \begin{excer}
        Pruebe que $\emptyset\vDash\phi$ siendo $\phi$ axioma lógico.
    \end{excer}

    \begin{proof}
        Se tienen tres casos:
        \begin{enumerate}
            \item $\phi\equiv\varphi\Rightarrow(\psi\Rightarrow\varphi)$, siendo $\varphi$ y $\psi$ FBF. Sea $m$ un modelo, para ver que $\emptyset\vDash\phi$ basta con probar que $\overline{m}(\phi)=V$.
            
            Por el Lema 1.2.1. basta con ver los valores de $\overline{m}$ en $\varphi$ y $\psi$. Para el modelo $m$ se tienen los posibles valores de $\varphi$, $\psi$, $\psi\Rightarrow\varphi$ y $\phi$:
            \begin{center}
                \begin{tabular}{c c | c | c}
                    \hline
                    $\varphi$ & $\psi$ & $\psi\Rightarrow\varphi$ & $\varphi\Rightarrow(\psi\Rightarrow\varphi)$ \\
                    \hline
                    $V$ & $V$ & $V$ & $V$ \\
                    $V$ & $F$ & $V$ & $V$ \\
                    $F$ & $V$ & $F$ & $V$ \\
                    $F$ & $F$ & $V$ & $V$ \\
                \end{tabular}
            \end{center}
            en cualquier caso, $\overline{m}(\phi)=V$. Por tanto, $\emptyset\vDash\phi$.

            \item $\phi\equiv\varphi\Rightarrow((\psi\Rightarrow\neg\varphi)\Rightarrow\neg\psi)$. Com oen el caso anterior, sea $m$ un modelo. Veamos que se tienen los posibles valores:
            \begin{center}
                \begin{tabular}{c c | c | c | c}
                    \hline
                    $\varphi$ & $\psi$ & $\psi\Rightarrow\neg\varphi$ & $(\psi\Rightarrow\neg\varphi)\Rightarrow\neg\psi$ & $\varphi\Rightarrow((\psi\Rightarrow\neg\varphi)\Rightarrow\neg\psi)$ \\
                    \hline
                    $V$ & $V$ & $F$ & $V$ & $V$ \\
                    $V$ & $F$ & $V$ & $V$ & $V$ \\
                    $F$ & $V$ & $V$ & $F$ & $V$ \\
                    $F$ & $F$ & $V$ & $V$ & $V$ \\
                \end{tabular}
            \end{center}
            en cualquier caso, $\overline{m}(\phi)=V$. Por tanto, $\emptyset\vDash\phi$.

            \item Suponga que $\phi$ es de la forma $\varphi\Rightarrow\varphi'$, donde $\varphi'$ es una fórmula donde alguna subfórmula $\psi$ de $\varphi$ fue sustituída por $\neg\neg\psi$. Para esto, basta con ver que $\overline{m}(\psi)=\overline{m}(\neg\neg\psi)$ (lo cual siempre se cumple por definición de modelo). Para formalizar esta prueba, todo se debe hacer por inducción. %TODO
            
            \item %TODO
        \end{enumerate}
    \end{proof}

    \begin{theor}[\textbf{Teorema de Completud}]
        Sea $\Sigma$ un conjunto de fórmulas y $\varphi$ otra fórmula. Entonces,
        \begin{equation*}
            \Sigma\vdash\varphi\textup{ si y sólo si }\Sigma\vDash\varphi
        \end{equation*}
        La necesidad es llamada \textbf{correctud} y la suficiencia es \textbf{completud}.
    \end{theor}

    \begin{proof}
        Se probará la doble implicación:
        \begin{itemize}
            \item \textbf{Correctud}: Sea $(\varphi_1,...,\varphi_{ n-1},\varphi)$ una demostración de $\varphi$ a partir de $\Sigma$. La hipótesis inductiva es:
            \begin{equation*}
                \Sigma\vDash\varphi_{ i},\quad\forall i\in\natint{1,n-1}
            \end{equation*}
            Se tienen tres casos:
            \begin{itemize}
                \item \textit{$\varphi$ es un elemento de $\Sigma$}: Al tenerse que cualquier modelo que satisfaga $\Sigma$ en particular satisface $\varphi$ (checar la definición de satisfabilidad). Por tanto $\Sigma\vDash\varphi$.
                \item \textit{$\varphi$ es axioma lógico}: Afirmamos que $\emptyset\vDash\varphi$ (\textit{ejercicio}, checar con tabla de verdad y usar proposiciones pasadas). En particular se tiene que $\Sigma\vDash\varphi$.
                \item \textit{Existen $i,j\in\natint{1,n-1}$ con $i<j$ tales que $\varphi_j$ es $\varphi_i\Rightarrow\varphi$}: Se tiene por hipótesis inductiva $\Sigma\vDash\varphi_i$ y $\Sigma\vDash\varphi_j$, por tanto si $m$ es cualquier modelo que satisface $\Sigma$, esto es $m\vDash\Sigma$, entonces $m\vDash\varphi_i$ y $m\vDash\varphi_j$, se sigue que:
                \begin{equation*}
                    m\vDash\varphi_i\quad\textup{y}\quad m\vDash\varphi_i\Rightarrow\varphi
                \end{equation*}
                asi que $\overline{m}(\varphi_i)=V$ y $\overline{m}(\varphi_i\Rightarrow\varphi)=V$, lo cual implica que $\overline{m}(\varphi)=V$, por lo que $m\vDash\varphi$.

                Por ende, $\Sigma\vDash\varphi$.
            \end{itemize}
            por inducción se sigue que $\Sigma\vDash\varphi$.
            \item \textbf{Completud}: Suponga que $\Sigma\vDash\varphi$, entonces no existe modelo $m$ tal que $m\vDash\Sigma$ y que $m\nvDash\varphi$. Es decir que no existe modelo $m$ tal que $m\vDash\Sigma$ y $m\vDash\neg\varphi$.
            
            Por tanto, el conjunto $\Sigma\cup\left\{\neg\varphi\right\}$ no es satisfacible, por la Correctud, se sigue que $\Sigma\cup\left\{\neg\varphi\right\}$ es inconsistente por el Lema 1.3.4, lo cual a su vez implica por el Corolario 1.3.2 que $\Sigma\vdash\varphi$.
        \end{itemize}
    \end{proof}

    \begin{cor}[\textbf{Teorema de Compacidad}]
        Si $\Sigma\vDash\varphi$, entonces existe un conjunto $\Sigma_0\subseteq\Sigma$ finito tal que $\Sigma_0\vDash\varphi$.
    \end{cor}

    \begin{proof}
        Si $\Sigma\vDash\varphi$, entonces $\Sigma\vdash\varphi$. Si
        \begin{equation*}
            \Sigma_0=\left\{\psi\Big|\psi\textup{ aparece en una demostracioń de }\varphi \right\}\subseteq\Sigma
        \end{equation*}
        entonees $\Sigma_0$ es finito, y $\Sigma_0\vdash\varphi$, lo cual implica que $\Sigma_0\vDash\varphi$.
    \end{proof}

    \chapter{Lógica de Primer Orden}

    \section{Introducción}

    \begin{mydef}
        Un \textbf{Lenguaje de Primer Orden} consta de:
        \begin{itemize}
            \item \textbf{Alfabeto}: Mismo que tiene:
            \begin{enumerate}
                \item \textit{Variables}: denotadas por $v_1,...,v_n$.
                \item \textit{Conectivas}: a saber $\neg$ y $\Rightarrow$.
                \item \textit{Relación de Igualdad}: denotada por el símbolo $=$.
                \item \textit{Cuantificador}: denotado por $\exists$.
                \item[a)] \textit{Símbolos de predicado/relación}, denotados por $P_1,...,P_n$.
                \item[b)] \textit{Símbolos de función}, denotados por $F_1,...,F_n$.
                \item[c)] \textit{Símbolos de constante}, denotados por $C_1,...,C_n$.
            \end{enumerate}
        \end{itemize}
        $1$-$4$ son llamados \textbf{Símbolos Lógicos}, y los de $a)$-$b)$ son llamados \textbf{Símbolos no Lógicos}.

        Cada función lleva su ariedad, como lo muestra el ejemplo de abajo.
    \end{mydef}

    \begin{obs}
        En un modelo para un lenguaje de primer orden, se van a evaluar expresiones verdaderas, pero ¿qué significa que una expresión pueda ser evaluada? Más adelante se responderá esa pregunta.
    \end{obs}

    \begin{obs}
        Todos los lenguajes poseen Símbolos Lógicos. Lo que generalmente va a cambiar son los Símbolos no Lógicos. Estos se denotarán entre corchetes $\left\{\right\}$.
    \end{obs}

    \begin{exa}
        Un ejemplo de un lenguaje de primer orden es el de la Teoría de Grupos: $\left\{\cdot,()^{-1},e_G \right\}=\mathcal{L}_{TG}$, siendo $e_G$ la identidad del grupo, $\cdot$ la operación binaria sobre $G$ que es asociativa y la función $()^{-1}$ es una función unaria que a cada elemento le asigna su inverso.
        \begin{itemize}
            \item $\cdot$ es una función binaria.
            \item $()^{-1}$ es una función unaria.
            \item $e_G$ es una constante.
        \end{itemize}
    \end{exa}

    \begin{exa}
        Otro ejemplo de un lenguaje de primer orden es el de la Teoría de Anillos: $\left\{+,\cdot,1,0 \right\}=\mathcal{L}_{TA}$, $+$ y $\cdot$ funciónes binarias. 0 y 1 constantes.
    \end{exa}
    
    \begin{exa}
        El Lenguaje de la Aritmética, denotado por $\mathcal{L}_A=\left\{+,\cdot,<,1,Sc \right\}$, donde $+$ y $\cdot$ son funciones binarias, $<$ es un símbolo de predicado/relación, $1$ es constante y $Sc$ es la función unaria sucesor.
    \end{exa}

    \begin{exa}
        El Lenguaje de la Teoría de Conjuntos, denotado por $\mathcal{L}_{TC}=\left\{\in\right\}$, siendo $\in$ el único símbolo que es una relación binaria.
    \end{exa}

    \begin{mydef}
        Dado un lenguaje de primer orden $\left\{P_i,F_i,C_i \right\}$, definimos los \textbf{términos} como sigue:
        \begin{enumerate}
            \item Cada $v_i$ y cada $c_i$ son términos (variables y constantes). Realmente estamos abreviando la cadena de longitud uno con único elemento a $v_i$ o $c_i$ (dependiendo del caso, puede que abreviemos a $(v_i)$ o $(c_i)$).
            \item Si $F_i$ es un símbolo de función que sea $k$-ario, y $t_1,...,t_k$ son términos, entonces la sucesión o cadena de símbolos $(F_i,t_1,...,t_k)\equiv F_it_1\cdot st_k$ también es un término.
        \end{enumerate}
    \end{mydef}

    \begin{obs}
        El símbolo $\equiv$ denota que dos fórmulas son equivalentes (o que son la misma). En lo que sigue se intentará no usar para no confundir con $=$.
    \end{obs}

    \begin{exa}
        En $\mathcal{L}_{TG}$, $v_5,v_7,e_G$ son términos. También:
        \begin{equation*}
            \cdot v_t\cdot v_7 e_G
        \end{equation*}
        es un término (recuerde que está en notación polaca).

        En $\mathcal{L}_{ TA}$, $0,1,v_{1000},\cdot v_{57}+01$ son términos.

        En $\mathcal{L}_A$, $Sc+1v_{ 57}$ es un término.

        En $\mathcal{L}_{ TC}$, $v_{49},v_{79},v_{ 1985}$ son términos.
    \end{exa}

    \begin{mydef}
        Dado un lenguaje de primer orden $\left\{F_i,P_i,C_i \right\}$ definimos las fórmulas como sigue:
        \begin{enumerate}
            \item \textbf{Fórmulas Atómicas}:
            \begin{itemize}
                \item Si $t_1,t_2$ son términos, entonces $=t_1t_2$ es una fórmula.
                \item Si $R_i$ es un símbolo de relación $k$-aria, y $t_1,...,t_k$ son términos, entonces $R_it_1\cdots t_k$ es una fórmula.
            \end{itemize}
            \item Si $\varphi,\psi$ son fórmulas, entonces $\Rightarrow\varphi\psi$ y $\neg\varphi$ son fórmulas.
            \item Si $\varphi$ es una fórmula y $v_i$ es una variable, entonces $\exists v_i\varphi$ es una fórmula.
        \end{enumerate}
    \end{mydef}

    \begin{obs}
        Las fórmulas atómicas son las fórmulas más simples que podemos formar en nuestro lenguaje de primer orden.
    \end{obs}

    Los modelos que definiremos sobre la lógica de primer orden se harán sobre estos conjuntos de fórmulas, a los cuales les podremos asignar un valor de verdadero o falso (dependiendo del caso).

    \begin{mydef}
        Se define el cuantificador $\forall$, \textbf{Para Todo} como:
        \begin{equation*}
            \forall x\varphi \equiv \neg((\exists x)(\neg\varphi))
        \end{equation*}
        siendo $x$ una variable y $\varphi$ una fórmula.
    \end{mydef}

    \begin{exa}
        Un ejemplo de una fórmula en $\mathcal{L}_{TA}$ es:
        \begin{equation*}
            \neg(\exists v_6\neg (1+1)v_5\cdot v_5+(1+1+1)v_5\cdot v_7+v_7=0)
        \end{equation*}
    \end{exa}

    \begin{exa}
        Un ejemplo de fórmula en $\mathcal{L}_A$ es:
        \begin{equation*}
            \neg(\exists v_1\neg\neg \exists v_2\neg=+v_1Scv_2Sc+v_1v_2)
        \end{equation*}
    \end{exa}

    \begin{exa}
        Un ejemplo de fórmula en $\mathcal{L}_{TC}$ es:
        \begin{equation*}
            (\forall x)(\forall y)(((\forall z)( z\in x\iff z\in y))\Rightarrow x=y)
        \end{equation*}
        que, escrita en notación polaca sería:
        \begin{equation*}
            \neg\exists v_1\neg\neg\exists v_2\neg \Rightarrow\neg \exists v_3\neg\land\Rightarrow \in v_3v_1\in v_3v_2\Rightarrow \in v_3v_2\in v_3v_1=v_1v_2
        \end{equation*}
    \end{exa}

    Hablaremos ahora de variables libres. En la fórmula:
    \begin{equation*}
        (\forall v_5)(e_G=v_5(v_6e3_G))
    \end{equation*}
    $v_5$ es llamada variable ligada, y $v_6$ es llamada libre.
    
    En la fórmula:
    \begin{equation*}
        (\forall x\forall y)(x+Sc(y)=Sc(x)+y)
    \end{equation*}
    no hay variables libres, todas son ligadas. Pero la fórmula
    \begin{equation*}
        (\forall v_5)(v_5=e_G)\land(v_5\cdot v_7=v_6)
    \end{equation*}
    es tal que $v_5$ es una variable ligada en el lado izquierdo del $\land$, pero es libre en el lado derecho.

    \begin{mydef}
        Se define la función $\cf{\free}{\textup{Conjunto de Fórmulas}}{\mathcal{P}(\textup{Var})}$, donde $\textup{Conjunto de Fórmulas}$ (como su nombre lo indica) es el conjunto de fórmulas de nuestro lenguaje de primer orden, y $\textup{Var}$ es el conjunto de variables del mismo lenguaje. Esta función está definida de tal forma que:
        \begin{equation*}
            \free(\varphi)=\textup{Variables que aparecen en todos los términos}
        \end{equation*}
        para toda fórmula atómica $\varphi$. Además, hacemos que:
        \begin{enumerate}
            \item $\free(\varphi)=\free(\neg\varphi)$ para toda fórmula $\varphi$.
            \item $\free(\Rightarrow\varphi\psi)=\free(\varphi)\cup\free(\psi)$, para todo par de fórmulas $\varphi,\psi$.
            \item $\free(\exists x\varphi)=\free(\varphi)\setminus\left\{x\right\}$, para toda fórmula $\varphi$ y toda variable $x$.
        \end{enumerate}
    \end{mydef}

    \begin{obs}
        Del inciso (3) de la definición anterior se sigue que
        \begin{equation*}
            \free(\forall x\varphi)=\free(\varphi)\setminus\left\{x\right\}
        \end{equation*}
    \end{obs}

    \section{Cálculo Deductivo}

    Básicamente es lo análogo al cálculo proposicional en el capítulo pasado:

    \begin{mydef}[\textbf{Axiomas Lógicos}]
        Todo lenguaje de primer orden consta de 6 axiomas lógicos (llamados \textbf{Axiomas de la Lógica Proposicional}):
        \renewcommand{\theenumi}{\arabic{enumi}}
        \begin{enumerate}
            \item $\varphi\Rightarrow(\psi\Rightarrow\varphi)$.
            \item $\varphi\Rightarrow((\psi\Rightarrow\neg\varphi)\Rightarrow\neg\psi)$.
            \item $\varphi\Rightarrow\varphi'$ siempre que $\varphi$ resulte de sustituir $\psi$ por $\neg\neg\psi$ o viceversa (siendo $\psi$ una subfórmula de $\varphi$).
            \item $\varphi\Rightarrow\varphi'$ siempre que $\varphi$ resulte de sustituir $\psi\Rightarrow\chi$ por $\neg\chi\Rightarrow\neg\psi$ o viceversa (siendo $\psi$ y $\chi$ subfórmulas de $\varphi$).
            \item $\varphi\Rightarrow\varphi'$ siempre que $\varphi$ resulte de sustitur $\neg\psi\Rightarrow\psi$ por $\psi$ (siendo $\psi$ una subfórmula de $\varphi$).
            \item $(\varphi\Rightarrow(\chi\Rightarrow\psi))\Rightarrow((\varphi\Rightarrow\chi)\Rightarrow(\varphi\Rightarrow\psi))$.
        \end{enumerate}
        siendo $\varphi$ una fórmula dada y $\psi$ una fórmulas arbitraria en 1 y 2, y $\varphi,\psi,\chi$ fórmulas dadas.

        Además, consta de los siguientes axiomas adicionales, correspondeintes a la lógica de primer orden (llamados \textbf{Axiomas de la Lógica de Primer Orden}):
        \begin{enumerate}
            \item $(\forall x)(\varphi\Rightarrow\psi)\Rightarrow(\forall x\varphi\Rightarrow\forall x\psi)$.
            \item $\varphi\Rightarrow\forall x\varphi$ si $x\notin\free(\varphi)$.
            \item $x=x$.
            \item $x=y\Rightarrow \varphi=\varphi'$ donde $\varphi'$ resulta de sustituir una instancia por $y$ en $\varphi$ y $\varphi$ es atómica que contiene a la variable $x$.
            \item $\forall x\varphi\Rightarrow\varphi[t/x]$ si $t$ es \textit{sustituible} por $x$ en $\varphi$. (aquí $t$ es un término).
        \end{enumerate}
    \end{mydef}

    Los únicos axiomas nuevos en la lógica de primer orden son los 7-11.

    \begin{mydef}
        Nuestro lenguaje de primer orden consta de una única regla de inferencia \textbf{Modus Ponens} (abreviado M.P.), dada por:
        \begin{center}
            \begin{tabular}{c c c}
                $\varphi$ & $\Rightarrow$ & $\psi$ \\
                $\varphi$ &  &  \\
                \hline
                 & $\therefore$ & $\psi$ \\
            \end{tabular}
        \end{center}
    \end{mydef}

    Con los axiomas 1-6, se tiene que podemos retomar todas las proposiciones que hemos deducido en el capítulo anterior (únicamente aquellas demostradas en base a los axiomas de la lógica proposicional y a la regla de inferencia Modus Ponens). Por ejemplo, la conjunción, simplificación, adición, Metateorema de Deducción, etc...

    Veremos ahora algunos ejemplos de estos axiomas y de M.P.

    \begin{obs}
        $\varphi[t/x]$ significa: sustituir cada instancia de $x$ por $t$ en $\varphi$.
    \end{obs}

    \begin{exa}
        Ejemplo del axioma 5: Si
        \begin{equation*}
            \varphi\equiv x=y+1
        \end{equation*}
        entonces, $\varphi[t/x]$ con $t$ siendo $(1+0)\cdot\zeta$ sería $(1+0)\cdot\zeta=y+1$.
    \end{exa}

    \begin{exa}
        En la definición, ¿qué significa ser \textit{sustituible}? Si tenemos la fórmula
        \begin{equation*}
            (\forall x)(x=y+1)\Rightarrow(1+0)\cdot\zeta=y+1
        \end{equation*}
        es correcto poder hacer aplicación del axioma 5. También
        \begin{equation*}
            (\forall x)(x=y+1)\Rightarrow(0+y)=y+1
        \end{equation*}
        es una instancia correcta del axioma 5 (recuerde que no existen las reglas de cancelación ni de absolutamente nada, por lo que es correcto lo que estamos diciendo aunque sepamos que no puede suceder $y=y+1$).
    \end{exa}

    \begin{exa}
        ¿Cuándo no se puede hacer la sustitución? Considere la fórmula:
        \begin{equation*}
            (\forall x)(\neg(\forall y)(x=y))\Rightarrow(\neg(\forall y)(y=y))
        \end{equation*}
        es decir:
        \begin{equation*}
            (\forall x)(\exists y(x\neq y))\Rightarrow (\exists y)(y\neq y)
        \end{equation*}
        en este caso, del lado izquierdo la variable $x$ era libre y luego ya no, esto es que fue capturada. Se tiene que esta no es una aplicación correcta del axioma 5.
    \end{exa}

    Veremos ahora como hacer aplicaciones correctas del axioma 5.

    \begin{mydef}
        En el axioma 5 de la Lógica de Primer Orden, \textbf{$t$ es sustituíble por $x$ en $\varphi$} significa que ninguna variable de $t$ queda (después de la sustitución) bajo el alcance de un cuantificador de $\varphi$.

        En caso de que no se cumpla lo contrario, se tiene que $t$ no es sustituíble.
    \end{mydef}

    Con esta definición, los ejemplos anteriores toman sentido.

    \begin{mydef}
        Sea $\Sigma$ un conjunto y $\varphi$ una fórmula.
        \begin{enumerate}
            \item Una \textbf{demostración} de $\varphi$ a partir de $\Sigma$ es una sucesión finita de fórmulas $(\varphi_1,...,\varphi_n)$ tal que:
            \begin{enumerate}
                \item $\varphi_n=\varphi$.
                \item Para cada $i\in\left\{1,...,n\right\}$ se tiene una de las tres:
                \begin{enumerate}
                    \item $\varphi_i\in\Sigma$.
                    \item $\varphi_i$ es axioma lógico de la lógica proposicional o de la lógica de primer orden.
                    \item existen $k,j\in\left\{1,...,n\right\}$ con $k<j<i$ tales que $\varphi_j$ es $\Rightarrow\varphi_k\varphi_i$.
                \end{enumerate}
            \end{enumerate}
            \item Decimos que $\varphi$ es \textbf{demostrable a partir de $\Sigma$}, o que $\varphi$ es un \textbf{teorema de $\Sigma$} si existe una demostración de $\varphi$ a partir de $\Sigma$, esto se simboliza por $\Sigma\vdash\varphi$.
        \end{enumerate}
    \end{mydef}

    Es casi lo mismo que en lógica proposicional, pero cambia el hecho de que las fórmulas incluyen cuantificadores.

    \begin{exa}
        Completemos la siguente demostración:
        \begin{center}
            \setcounter{tablec}{1}
            \begin{tabular}{l r l c l r}
                & \pstable{tablec} & $(\forall x)(Px$ & $\Rightarrow$ & $Qx)$ & Premisa \\
                $|\longrightarrow$& \pstable{tablec} & $Pc$ &  &  & Sup. \\
                $||\longrightarrow$& \pstable{tablec} & $(\forall y)(Qy$ & $\Rightarrow$ & $Sy)$ & Sup. \\
                $||$& \pstable{tablec} & $Pc$ & $\Rightarrow$ & $Qc$ & 1 Ax. 5 LdPO y M.P. \\
                $||$& \pstable{tablec} & $Qc$ & $\Rightarrow$ & $Sc$ & 3 Ax. 5 LdPO y M.P. \\
                $||$& \pstable{tablec} & $Qc$ &  &  & 4,2 M.P. \\
                $||$& \pstable{tablec} & $Sc$ &  &  & 5,6 M.P. \\
                \hline
                $|$& \pstable{tablec} & $(\forall y)(Qy\Rightarrow Sy)$ & $\Rightarrow$ & $Sc$ & 3-7 M.D. \\
                \hline
                & \pstable{tablec} & $Pc$ & $\Rightarrow$ & $((\forall y)(Qy\Rightarrow Sy)\Rightarrow Sc)$ & 2-8 M.D. \\
                \hline
                & & & $\therefore$ & $Pc\Rightarrow((\forall y)(Qy\Rightarrow Sy)\Rightarrow Sc)$ & \\
            \end{tabular}
        \end{center}
        en este caso, $P$, $Q$ y $S$ son relaciones 1-arias. Además, $c$ es una constante. En esta prueba se usa el Metateorema de Deducción, el hecho de que se pueda usar este teorema fue fundamentado anteriormente.
    \end{exa}

    \begin{theor}[\textbf{Metateorema}]
        Se tiene lo siguiente:
        \begin{enumerate}
            \item \textbf{Instanciación Universal}:
            \begin{center}
                \begin{tabular}{c c c}
                    & $\forall x\varphi$  &  \\
                    \hline
                     & $\therefore$ & $\varphi[t/x]$ \\
                \end{tabular}
            \end{center}
            esto es, si $\Sigma\vdash\forall x\varphi$, entonces $\Sigma\vdash\varphi[t/x]$.
            \item \textbf{Generalización Existencial}:
            \begin{center}
                \begin{tabular}{c c c}
                    & $\varphi[t/x]$  &  \\
                    \hline
                     & $\therefore$ & $\exists x\varphi$ \\
                \end{tabular}
            \end{center}
            esto es, si $\Sigma\vdash\varphi[t/x]$, entonces $\Sigma\vdash\exists x\varphi$.
        \end{enumerate}
        Los dos incisos siempre y cuando $t$ sea sustituíble por $x$ en $\varphi$. 
    \end{theor}

    \begin{proof}
        De (1):
        \begin{center}
            \begin{tabular}{l r l c l r}
                & $\vdots$ & $\vdots$ & $\vdots$ & $\vdots$ & Líneas de $\Sigma$. \\
                 & $n$) & $\forall x\varphi$ &  &  &  \\
                 & $n+1$) & $\forall x\varphi$ & $\Rightarrow$ & $\varphi[t/x]$ & $n$ Ax. 5 \\
                 & $n+2$) & $\varphi[t/x]$ &  &  & $n+1$,$n$ M.P. \\
                \hline
                & & & $\therefore$ & $\varphi[t/x]$ & \\
            \end{tabular}
        \end{center}
        De (2):
        \begin{center}
            \begin{tabular}{l r l c l r}
                & $\vdots$ & $\vdots$ & $\vdots$ & $\vdots$ & Líneas de $\Sigma$. \\
                 & $n$) & $\varphi[t/x]$ &  &  &  \\
                $|\longrightarrow$ & $n+1$) & $\neg\exists x\varphi$ &  &  & Sup. \\
                $|$ & $n+2$) & $\neg\exists x\neg\neg\varphi$ &  &  & $n+1$ D.N. \\
                $|$ & $n+3$) & $\forall x\neg\varphi$ &  &  & $n+2$ R.E. \\
                $|$ & $n+4$) & $\forall x\neg\varphi$ & $\Rightarrow$ & $\neg\varphi[t/x]$ & $n+3$ I.U. \\
                $|$ & $n+5$) & $\varphi[t/x]$ & $\land$ & $\neg\varphi[t/x]$ & $n$,$n+5$ Conj. \\
                \hline
                 & $n+6$) & $\exists x\varphi$ &  &  & $n$-$n+6$ M.D.p.C. \\
                \hline
                & & & $\therefore$ & $\exists x\varphi$ & \\
            \end{tabular}
        \end{center}
    \end{proof}

    \begin{obs}
        \textit{Instanciación} es quitar cuantificador, \textit{Generalización} es añadir cuantificador.
    \end{obs}

    \begin{excer}
        Complete las demostraciones siguiente:

        \begin{enumerate}
            \item
                \begin{center}
                    \setcounter{tablec}{1}
                    \begin{tabular}{l r l c l r}
                        & \pstable{tablec} & $(\forall x)(Px$ & $\Rightarrow$ & $(\forall y)(Qy\Rightarrow Sy))$ & Premisa \\
                        & \pstable{tablec} & $(\forall x)(Px\Rightarrow (\forall y)(Qy\Rightarrow Sy))$ & $\Rightarrow$ & $(\forall x)Px\Rightarrow (\forall x)(\forall y)(Qy\Rightarrow Sy)$ & 1 Ax. 1 \\
                        & \pstable{tablec} & $(\forall x)Px$ & $\Rightarrow$ & ($\forall x)(\forall y)(Qy\Rightarrow Sy)$ & 2,1 M.P.\\
                        $|\longrightarrow$ & \pstable{tablec} & $(\forall x)Px$ &  &  & Sup.\\
                        $|$ & \pstable{tablec} & $(\forall x)(\forall y)(Qy$ & $\Rightarrow$ & $Sy)$ & 4,3 M.P.\\
                        $|$ & \pstable{tablec} & $(\forall x)(\forall y)(Qy\Rightarrow Sy)$ & $\Rightarrow$ & $(\forall y)(Qy\Rightarrow Sy)$ & 5 Ax. 5\\
                        $|$ & \pstable{tablec} & $(\forall y)(Qy$ & $\Rightarrow$ & $Sy)$ & 7,6 M.P.\\
                        \hline
                        & \pstable{tablec} & $(\forall x)Px$ & $\Rightarrow$ & $(\forall y)(Qy\Rightarrow Sy)$ & 4-7 M.D.\\
                        \hline
                        & & & $\therefore$ & $(\forall x)Px\Rightarrow (\forall y)(Qy\Rightarrow Sy)$ & \\
                    \end{tabular}
                \end{center}
            \item
                \begin{center}
                    \setcounter{tablec}{1}
                    \begin{tabular}{l r l c l r}
                        & \pstable{tablec} & $(\exists x)Px$ & $\Rightarrow$ & $(\exists y)Qy$ & Premisa \\
                        $|\longrightarrow$ & \pstable{tablec} & $Pt$ &  &  & Sup. \\
                        $|$ & \pstable{tablec} & $(\exists x)Px$ &  &  & 2 G.E. \\
                        $|$ & \pstable{tablec} & $(\exists y) Qy$ &  &  & 1,3 M.P.\\
                        \hline
                        & \pstable{tablec} & $Pt$ & $\Rightarrow$ & $(\exists y)Qy$ & 2-4 M.D. \\
                        & \pstable{tablec} & $(\exists x)(Px$ & $\Rightarrow$ & $(\exists y)Qy)$ & 5 G.E. \\
                        \hline
                        & & & $\therefore$ & $(\exists x)(Px\Rightarrow (\exists y)Qy)$ & \\
                    \end{tabular}
                \end{center}
        \end{enumerate}

        Hay una forma alternativa de probar 1):


    \end{excer}

    \begin{lema}
        Usando 1. del siguiente Teorema, si $\Sigma\vdash\forall x\varphi$ y $z$ no aparece libre en ningún elemento de $\Sigma$ y es sustituíble por $x$ en $\varphi$, entonces $\Sigma\vdash\forall z\varphi[z/x]$.
    \end{lema}

    \begin{proof}
        Se tiene que:
        \begin{center}
            \setcounter{tablec}{1}
            \begin{tabular}{l r l c l r}
                & \pstable{tablec} & $\vdots$ & $\vdots$ & $\vdots$ & Líneas de $\Sigma$ \\
                & \pstable{tablec} & $(\forall x)\varphi$ &  &  & Prueba de $\forall x\varphi$ \\
                & \pstable{tablec} & $\varphi[z/x]$ &  &  & I.U. \\
                & \pstable{tablec} & $(\forall z)\varphi[z/x]$ &  &  & G.U. \\
                \hline
                & & & $\therefore$ & $(\forall z)\varphi[z/x]$ & \\
            \end{tabular}
        \end{center}
    \end{proof}

    \begin{theor}[\textbf{Metateorema}]
        Se tiene lo siguiente:
        \begin{enumerate}
            \item \textbf{Generalización Universal}: Si $\Sigma\vdash\varphi$ y $x$ no aparece libre en ninguna fórmula de $\Sigma$, entonces $\Sigma\vdash(\forall x)\varphi$. Esto es:
            \begin{center}
                \begin{tabular}{c c c}
                    & $\varphi$  &  \\
                    \hline
                     & $\therefore$ & $(\forall x)\varphi$ \\
                \end{tabular}
            \end{center}
            siempre que $x$ no aparezca libre en ninguna premisa, absolutas o relativas.
            \item \textbf{Instanciación Existencial}: Si $\Sigma\cup\left\{\varphi[w/y]\right\} \vdash\psi$, entonces $\Sigma\cup\left\{\exists y\varphi \right\}\vdash\psi$ si $w$ no aparece libre en ninguna fórmula de $\Sigma$. Esto es:
            \begin{center}
                \begin{tabular}{l c c}
                    & $\exists y\varphi$  &  \\
                   $|\longrightarrow$ & $\varphi[w/y]$  & Sup. \\
                    $|$ & $\vdots$  &  \\
                    $|$ & $\psi$  &  \\
                    \hline
                     & $\therefore$ & $\varphi[w/y]$ \\
                \end{tabular}
            \end{center}
            siempre que $w$ no aparezca libre ni en $\psi$ ni en ninguna premisa de $\Sigma$.
        \end{enumerate}
    \end{theor}

    \begin{proof}
        De (1): Sea $\Gamma=\left\{\varphi|\Sigma\vdash\forall x\varphi \right\}$. Probaremos que $\Sigma\subseteq\Gamma$, $\Gamma$ contiene todos los axiomas lógicos y $\Gamma$ es cerrado bajo M.P. En efecto:
        \begin{itemize}
            \item \textbf{$\Gamma$ contiene a todos los axiomas lógicos}: es inmediato por la definición de $\Gamma$.
            \item Sea $\alpha\in\Sigma$. Entonces, a partir de $\Sigma$, $\alpha$ es premisa. Luego $\alpha\Rightarrow(\forall x)\alpha$ (por una instancia del axioma 2 ya que $x$ no es libre en $\alpha$). Luego $\alpha\in\Gamma$.
            \item Supongamos que $\alpha,\alpha\Rightarrow\beta\in\Gamma$. A partir de $\Sigma$, $\Sigma\vdash(\forall x)\alpha$ y $\Sigma\vdash(\forall x)(\alpha\Rightarrow\beta)$. Por el axioma 1, se sigue que $\Sigma\vdash(\forall x)\alpha\Rightarrow(\forall x)\beta$. Se puede entonces aplicar M.P. en $\Sigma$ para que se deduzca que $\Sigma\vdash(\forall x)\beta$.
        \end{itemize}

        De (2): Se tiene que:
        \begin{center}
            \setcounter{tablec}{1}
            \begin{tabular}{l r l c l r}
                & \pstable{tablec} & $\vdots$ & $\vdots$ & $\vdots$ & Líneas de $\Sigma\cup\left\{(\exists y)\varphi\right\}$ \\
                & \pstable{tablec} & $(\exists y)\varphi$ &  &  &  \\
                & \pstable{tablec} & $\varphi[w/y]$ & $\Rightarrow$ & $\psi$ & Hipótesis + M.D. \\
                $|\longrightarrow$ & \pstable{tablec} & $\neg\psi$ &  &  & Sup. \\
                $|$ & \pstable{tablec} & $\neg\varphi[w/y]$ &  &  & 3,4 M.T. \\
                $|$ & \pstable{tablec} & $(\forall w)\neg\varphi[w/y]$ &  &  & 5 G.U. \\
                $|$ & \pstable{tablec} & $(\forall y)\neg\varphi$ &  &  & 6 Lema 2.2.1 \\
                $|$ & \pstable{tablec} & $\neg(\forall y)\neg\varphi$ &  &  & 2 R.E. \\
                $|$ & \pstable{tablec} & $((\forall y)\neg\varphi)$ & $\land$ & $\neg((\forall y)\neg\varphi)$ & 8,7 Conj.\\
                \hline
                $|$ & \pstable{tablec} & $\psi$ &  &  & 4-9 D.C.\\
                & & & $\therefore$ & $\psi$ & \\
            \end{tabular}
        \end{center}
        la línea 3 se tiene pues: $\Sigma\cup\left\{\varphi[w/y]\right\} \vdash\psi$ si y sólo si $\Sigma\vdash\varphi[w/y]\Rightarrow\psi$.
    \end{proof}

    \begin{exa}
        Complete la demostración siguiente:
        \begin{center}
            \setcounter{tablec}{1}
            \begin{tabular}{l r l c l r}
                & \pstable{tablec} & $(\forall x)(Px$ & $\Rightarrow $ & $Qx)$ & Premisa \\
                $|\longrightarrow$ & \pstable{tablec} & $Sz$ &  &  & Sup. \\
                $||\longrightarrow$ & \pstable{tablec} & $(\forall y)(Sy$ & $\Rightarrow$ & $Py)$ & Sup. \\
                $||$ & \pstable{tablec} & $Sz$ & $\Rightarrow$ & $Pz$ & 3 I.U. \\
                $||$ & \pstable{tablec} & $Pz$ & $\Rightarrow$ & $Qz$ & 3 I.U. \\
                $||$ & \pstable{tablec} & $Pz$ &  &  & 4,2 M.P. \\
                $||$ & \pstable{tablec} & $Qz$ &  &  & 5,7 M.P. \\
                \hline
                $|$ & \pstable{tablec} & $(\forall y)(Sy\Rightarrow Py)$ & $\Rightarrow$ & $Qz$ & 3-7 M.D. \\
                $|$ & \pstable{tablec} & $Sz$ & $\Rightarrow$ & $((\forall y)(Sy\Rightarrow Py)\Rightarrow Qz)$ & 2-8 M.D. \\
                $|$ & \pstable{tablec} & $(\forall x)(Sx$ & $\Rightarrow$ & $((\forall y)(Sy\Rightarrow Py)\Rightarrow Qx))$ & 9 G.U. \\
                \hline
                & & & $\therefore$ & $(\forall x)(Sx\Rightarrow ((\forall y)(Sy\Rightarrow Py)\Rightarrow Qx))$ & \\
            \end{tabular}
        \end{center} 
    \end{exa}

    \begin{exa}
        Complete la demostración siguiente:
        \begin{center}
            \setcounter{tablec}{1}
            \begin{tabular}{l r l c l r}
                & \pstable{tablec} & $(\forall x)(Px$ & $\Rightarrow $ & $Qx)$ & Premisa \\
                & \pstable{tablec} & $(\forall x)(Sx$ & $\Rightarrow $ & $Tx)$ & Premisa \\
                $|\longrightarrow$ & \pstable{tablec} & $(\forall x)(Qx$ & $\Rightarrow $ & $Sx)$ & Sup. \\
                $||\longrightarrow$ & \pstable{tablec} & $Pz$ &  &  & Sup. \\
                $||$ & \pstable{tablec} & $Pz$ & $\Rightarrow$ & $Qz$ & 1 I.U. \\
                $||$ & \pstable{tablec} & $Qz$ & $\Rightarrow$ & $Sz$ & 3 I.U. \\
                $||$ & \pstable{tablec} & $Sz$ & $\Rightarrow$ & $Tz$ & 2 I.U. \\
                $||$ & \pstable{tablec} & $Qz$ &  &  & 5,4 M.P. \\
                $||$ & \pstable{tablec} & $Sz$ &  &  & 6,8 M.P. \\
                $||$ & \pstable{tablec} & $Tz$ &  &  & 9,7 M.P. \\
                \hline
                $|$ & \pstable{tablec} & $Pz$ & $\Rightarrow$ & $Tz$ & 4-10 M.D. \\
                $|$ & \pstable{tablec} & $(\forall y)(Py$ & $\Rightarrow$ & $Ty)$ & 11 G.U. \\
                \hline
                & \pstable{tablec} & $(\forall x)(Qx\Rightarrow Sx)$ & $\Rightarrow$ & $(\forall y)(Py\Rightarrow Ty)$ & 3-12 M.D. \\
                \hline
                & & & $\therefore$ & $(\forall x)(Qx\Rightarrow Sx)\Rightarrow(\forall y)(Py\Rightarrow Ty)$ & \\
            \end{tabular}
        \end{center} 
    \end{exa}

    \begin{exa}
        Complete la demostración siguiente:
        \begin{center}
            \setcounter{tablec}{1}
            \begin{tabular}{l r l c l r}
                & \pstable{tablec} & $(\exists x)Px$ & $\Rightarrow $ & $(\forall y)((Py\lor Qy)\Rightarrow Sy)$ & Premisa \\
                & \pstable{tablec} & $(\exists x)Px$ & $\land$ & $(\exists x)Sx$ & Premisa \\
                & \pstable{tablec} & $(\exists x)Px$ &  &  & 2 Simp. \\
                & \pstable{tablec} & $(\forall y)((Py\lor Qy)$ & $\Rightarrow$ & $Sy)$ & 1,3 M.P. \\
                $|\longrightarrow$ & \pstable{tablec} & $Pz$ &  &  & 3 I.E. \\
                $|$ & \pstable{tablec} & $Pz$ & $\lor$ & $Qz$ & 5 Ad. \\
                $|$ & \pstable{tablec} & $(Pz\lor Qz)$ & $\Rightarrow$ & $Sz$ & 4 I.U. \\
                $|$ & \pstable{tablec} & $Sz$ &  &  & 7,6 M.P. \\
                $|$ & \pstable{tablec} & $Pz$ & $\land$ & $Sz$ & 5,8 Conj. \\
                \hline
                & \pstable{tablec} & $(\exists x)(Px$ & $\land$ & $Sx)$ & 5-9 G.E. \\
                \hline
                & & & $\therefore$ & $\exists x(Px\land Sx)$ & \\
            \end{tabular}
        \end{center}
        Este es un buen ejemplo del uso de instanciación existencial.
    \end{exa}

    \begin{exa}
        Complete la demostración siguiente:
        \begin{center}
            \setcounter{tablec}{1}
            \begin{tabular}{l r l c l r}
                & \pstable{tablec} & $(\exists x)Px$ & $\Rightarrow $ & $(\forall y)(Qy\Rightarrow Sy)$ & Premisa \\
                $|\longrightarrow$ & \pstable{tablec} & $(\exists x)(Px$ & $\land $ & $Qx)$ & Sup. \\
                $||\longrightarrow$ & \pstable{tablec} & $Pz$ & $\land $ & $Qz$ & 2 I.E. con $z$\\
                $||$ & \pstable{tablec} & $Pz$ &  &  & 3 Simp. \\
                \hline
                $|$ & \pstable{tablec} & $(\exists x)Px$ &  &  & 3-4 G.E. \\
                $|$ & \pstable{tablec} & $(\forall y)(Qy$ & $\Rightarrow$ & $Sy)$ & 1,5 M.P. \\
                $|$ & \pstable{tablec} & $Qz$ & $\Rightarrow$ & $Sz$ & 6 I.U. \\
                $||\longrightarrow$ & \pstable{tablec} & $Pz$ & $\land $ & $Qz$ & 2 I.E. con $z$ \\
                $||$ & \pstable{tablec} & $Pz$ &  &  & 8 Simp. \\
                $||$ & \pstable{tablec} & $Qz$ &  &  & 8 Conm. y Simp. \\
                $||$ & \pstable{tablec} & $Sz$ &  &  & 7,9 M.P. \\
                $||$ & \pstable{tablec} & $Pz$ & $\land$ & $Sz$ & 9,11 Conj. \\
                \hline
                $|$ & \pstable{tablec} & $(\exists y)(Py$ & $\land$ & $Sy)$ & 8-12 G.E. \\
                \hline
                & \pstable{tablec} & $(\exists x)(Px\land Qx)$ & $\Rightarrow$ & $(\exists y)(Py\land Sy)$ & 2-13 M.D. \\
                \hline
                & & & $\therefore$ & $(\exists x)(Px\land Qx)\Rightarrow(\exists y)(Py\land Sy)$ & \\
            \end{tabular}
        \end{center}
    \end{exa}

    \begin{exa}
        Complete la demostración siguiente:
        \begin{center}
            \setcounter{tablec}{1}
            \begin{tabular}{l r l c l r}
                & \pstable{tablec} & $(\forall x)(\exists y)(Px$ & $\lor $ & $Qy)$ & Premisa \\
                $|\longrightarrow$ & \pstable{tablec} & $\neg((\forall x)Px)$ &  &  & Sup. \\
                $|$ & \pstable{tablec} & $(\exists x)\neg Px$ &  &  & 2 R.E. \\
                $|$ & \pstable{tablec} & $\neg Pz$ &  &  & 3 I.E. \\
                $|$ & \pstable{tablec} & $(\exists y)(Pz$ & $\lor$ & $Qy)$ & 1 I.U. \\
                $|$ & \pstable{tablec} & $Pz$ & $\lor$ & $Qw$ & 6 I.E. \\
                $|$ & \pstable{tablec} & $Qw$ &  &  & 7,5 S.D. \\
                $|$ & \pstable{tablec} & $(\exists y)Qy$ &  &  & 8 G.E.\\
                \hline
                & \pstable{tablec} & $\neg((\forall x)Px)$ & $\Rightarrow$ & $(\exists y)Qy$ & 2-9 M.D. \\
                & \pstable{tablec} & $(\forall x)Px$ & $\lor$ & $(\exists y)Qy$ & 10 R.E. \\
                \hline
                & & & $\therefore$ & $(\forall x)Px\lor(\exists y)Qy$ & \\
            \end{tabular}
        \end{center}
    \end{exa}

    \begin{exa}
        Complete la demostración siguiente:
        \begin{center}
            \setcounter{tablec}{1}
            \begin{tabular}{l r l c l r}
                & \pstable{tablec} & $(\exists x)Px$ & $\lor$ & $(\forall y)(Py\Rightarrow Qy)$ & Premisa \\
                & \pstable{tablec} & $(\forall x)(Sx$ & $\Rightarrow$ & $\neg Px)$ & Premisa \\
                $|\longrightarrow$ & \pstable{tablec} & $(\forall x)(Px$ & $\Rightarrow$ & $Sx)$ & Sup. \\
                $||\longrightarrow$ & \pstable{tablec} & $Pu$ &  &  & Sup. \\
                $||$ & \pstable{tablec} & $Pu$ & $\Rightarrow$ & $Su$ & 3 I.U. \\
                $||$ & \pstable{tablec} & $Su$ & $\Rightarrow$ & $\neg Pu$ & 2 I.U. \\
                $||$ & \pstable{tablec} & $Su$ &  &  & 4,5 M.P. \\
                $||$ & \pstable{tablec} & $\neg Pu$ &  &  & 6,7 M.P. \\
                $||$ & \pstable{tablec} & $Pu$ & $\land$ & $\neg Pu$ & 4,8 Conj. \\
                \hline
                $|$ & \pstable{tablec} & $\neg Pu$ &  &  & 4-9 D.C. \\
                $|$ & \pstable{tablec} & $\forall x \neg Px$ &  &  & 10 G.U. \\
                $|$ & \pstable{tablec} & $\neg((\exists x)\neg\neg Px)$ &  &  & 11 R.E. \\
                $|$ & \pstable{tablec} & $\neg((\exists x) Px)$ &  &  & 12 D.N. \\
                $|$ & \pstable{tablec} & $(\forall y)(Py$ & $\Rightarrow$ & $Qy)$ & 1,13 S.D. \\
                \hline
                $|$ & \pstable{tablec} & $(\forall x)(Px\Rightarrow Sx)$ & $\Rightarrow$ & $(\forall y)(Py\Rightarrow Qy)$ & 3-14 M.D. \\
                \hline
                & & & $\therefore$ & $(\forall x)(Px\Rightarrow Sx)\Rightarrow(\forall y)(Py\Rightarrow Qy)$ & \\
            \end{tabular}
        \end{center}
    \end{exa}

    \begin{exa}
        Complete la demostración siguiente:
        \begin{center}
            \setcounter{tablec}{1}
            \begin{tabular}{l r l c l r}
                & \pstable{tablec} & $(\exists x)Px$ & $\lor$ & $(\exists y)Qy$ & Premisa \\
                & \pstable{tablec} & $(\forall x)(Px$ & $\Rightarrow$ & $Qx)$ & Premisa \\
                $|\longrightarrow$ & \pstable{tablec} & $\neg((\exists y)Qy)$ &  &  & Sup. \\
                $|$ & \pstable{tablec} & $(\exists x)Px$ &  &  & 1,3 S.D. \\
                $||\longrightarrow$ & \pstable{tablec} & $Pz$ &  &  & 4 I.E. \\
                $||$ & \pstable{tablec} & $Pz$ & $\Rightarrow$ & $Qz$ & 2 I.U. \\
                $||$ & \pstable{tablec} & $Qz$ &  &  & 6,5 M.P. \\
                \hline
                $|$ & \pstable{tablec} & $(\exists y)Qy$ &  &  & 5-7 I.E. \\
                $|$ & \pstable{tablec} & $\neg((\exists y)Qy)$ & $\land$ & $((\exists y)Qy)$ & 8,3 Conj. \\
                \hline
                  & \pstable{tablec} & $(\exists y)Qy$ &  &  & 3-9 D.C. \\
                \hline
                & & & $\therefore$ & $(\exists y)Qy$ & \\
            \end{tabular}
        \end{center}
    \end{exa}

    \section{Semántica de la Lógica de Primer Orden}

    \begin{mydef}
        Sea $\mathcal{L}=\left\{c_i,R_i,F_i \right\}$ un lenguaje de primer orden. Un \textbf{$\mathcal{L}$-modelo} o \textbf{$\mathcal{L}$-estructura} es una tupla $\mathfrak{A}=(A,\alpha_i,r_i,f_i)$ (tiene una entrada más que símbolos tiene nuestro lenguaje, $\alpha_i$ corresponde biyectivamente a $c_i$, lo mismo con $r_i$ y $R_i$, y $f_i$ y $F_i$, respectivamente), tal que
        \begin{enumerate}[label=(\textit{\alph*})]
            \item $A\neq\emptyset$, $A$ es llamado \textbf{universo de discurso/dominio}.
            \item Cada $a_i\in A$.
            \item Si $R_i$ es de aridad $K$, entonces $r_i\subseteq A^k$.
            \item Si $f_i$ es de aridad  
        \end{enumerate}
    \end{mydef}

    \begin{exa}
        En el lenguaje de la teoría de grupos, $\mathcal{L}_{TG}=\left\{e_G,\cdot,(\cdot)^{-1} \right\}$. Una $\mathcal{L}_{TG}$-estructura sería:
        \begin{equation*}
            \mathfrak{A}=(\mathbb{N},3,{}^{\wedge},Sc)
        \end{equation*}
        donde $Sc$ es la función sucesor. Un símbolo de este lenguaje sería:
        \begin{equation*}
            (\forall x)(E\cdot x=x\cdot x)\textup{ y }(\forall n\in\mathbb{N})(3^n=n^n)
        \end{equation*}
        Otro ejemplo sería:
        \begin{equation*}
            (\mathbb{R},\sqrt{2},+,-)
        \end{equation*}
        donde $-$ es la operación que toma inverso. Un símbolo de este lenguaje sería:
        \begin{equation*}
            (\forall r\in\mathbb{R})(\sqrt{2}+r=r+r)
        \end{equation*}        
        Otro ejemplo sería:
        \begin{equation*}
            (S_{19},id,\circ,\pi\mapsto\pi^{-1})
        \end{equation*}
        (el cual técnicamente si es un grupo). Un símbolo de este lenguaje sería:
        \begin{equation*}
            (\forall \sigma\in S_{19})(id\circ\sigma=\sigma\circ\sigma)
        \end{equation*}
    \end{exa}

    Notemos que en estos ejemplos, puede que una fórmula sea válida (o sea satisfacible) en algunos modelos, pero en otros no. Solo basta ver la primer estructura.

    \begin{exa}
        En $\mathcal{L}_{A}=\left\{1,+,\cdot,Sc,< \right\}$ unos $\mathcal{L}_A$-estructuras serían:
        \begin{equation*}
            (\mathbb{N},+,\cdot,n\mapsto n+1,<)
        \end{equation*}
        y,
        \begin{equation*}
            (\mathbb{Z},-3,\cdot,+,n\mapsto n-1,\divides)
        \end{equation*}
    \end{exa}

    \begin{exa}
        En $\mathcal{L}_{TC}=\left\{\in \right\}$, tenemos las siguientes $\mathcal{L}_{TC}$-estructuras:
        \begin{equation*}
            (\mathbb{Z},\divides),(\mathbb{Q},\leq),(\mathcal{P}(X),\subseteq), (\mathbb{R},=) \textup{ y } (\mathbb{R}[x],r)
        \end{equation*}
        (siendo $X$ un conjunto), donde
        \begin{equation*}
            frg\iff \deg(f)>\deg(g)
        \end{equation*}
    \end{exa}

    \begin{mydef}
        Sea $\mathcal{L}=\left\{c_i,R_i,F_i \right\}$ un lenguaje de primer orden, y sea $\mathfrak{A}=(A,a_i,r_i,f_i)$ un $\mathcal{L}$-modelo.
        \begin{enumerate}[label=(\textit{\arabic*})]
            \item Una \textbf{interpretación} es una función $\cf{\iota}{\textup{Var}}{A}$.
            \item Definimos la \textbf{satisfacción}, o \textbf{relación de satisfacción}, denotada por $\mathfrak{A}\vDash\varphi[\iota]$ por induccioń de la manera siguiente:
            \begin{itemize}
                \item Primero, definimos $\cf{\hat{\iota}}{\textup{Term}}{A}$ (donde $\textup{Term}$ es el conjunto de términos) que extiende a $\iota$ y está dada por:
                \begin{enumerate}[label=(\textit{\alph*})]
                    \item $\hat{\iota}(c_i)=a_i$ y $\hat{i}(v_i)=\iota(v_i)$.
                    \item $\hat{\iota}(F_i,t_1,...,t_k)=f_i(\hat{\iota}(t_1),...,\hat{\iota}(t_k))$
                \end{enumerate}
                \item Ahora, sí
                \begin{enumerate}[label=(\textit{\alph*})]
                    \item Si $\varphi$ es $t_1=t_2$, entonces
                    \begin{equation*}
                        \mathfrak{A}\vDash\varphi[\iota]\textup{ si y sólo si }\hat{\iota}(t_1)\textup{ es lo mismo que }\hat{\iota}(t_2)
                    \end{equation*}
                    y, si $\varphi$ es $R_it_1\cdots t_k$, entonces
                    \begin{equation*}
                        \mathfrak{A}\vDash\varphi[\iota]\textup{ si y sólo si }(\hat{\iota}(t_1),...,\hat{\iota}(t_k))\in r_i
                    \end{equation*}
                    \item Se tiene que
                    \begin{equation*}
                        \mathfrak{A}\vDash(\varphi\Rightarrow\psi)[\iota]\textup{ si y sólo si }\mathfrak{A}\nvDash\varphi[\iota]\textup{ o bien }\mathfrak{A}\vDash\psi[\iota]
                    \end{equation*}
                    y,
                    \begin{equation*}
                        \mathfrak{A}\vDash(\neg\varphi)(\iota)\textup{ si y sólo si }\mathfrak{A}\nvDash\varphi[\iota]
                    \end{equation*}
                    \item Se tiene que
                    \begin{equation*}
                        \mathfrak{A}\vDash(\exists x\varphi)[\iota]\textup{ si y sólo si }
                    \end{equation*}
                    para algún $a\in A$, se cumple que
                    \begin{equation*}
                        \mathfrak{A}\vDash\varphi[\iota_{a/x}]
                    \end{equation*}
                    donde $\iota_{ a/x}=\left(\iota\setminus\left\{(x,\iota(x)) \right\} \right)\cup\left\{(x,a) \right\}$ (recuerde la interpretación de $\iota$ como función dada como subconjunto del producto cartesiano del dominio con el codominio).
                \end{enumerate}
            \end{itemize}
        \end{enumerate}
    \end{mydef}

    \begin{exa}
        Retomando el ejemplo 2.3.1, tomando como lenguaje $\mathcal{L}=\left\{e_G,\cdot,(\cdot)^{-1} \right\}$, como modelo a $\mathfrak{N}=(\mathbb{N},3,{}^{\wedge},Sc)$ una interpretación sería:
        \begin{equation*}
            \iota(v_i)=i+3
        \end{equation*}
        para el término $e_G\cdot(v_3)^{-1}$, tenemos que:
        \begin{equation*}
            \begin{split}
                \hat{\iota}(e_G\cdot(v_3))&=(\hat{\iota}(e_G))^{\hat{\iota}((v_3)^{-1})}\\
                &=(\iota(e_G))^{Sc(\hat{\iota}(v_3))}\\
                &=(3)^{(3+3)+1}\\
                &=3^7\\
            \end{split}
        \end{equation*}
        en el modelo $\mathfrak{R}=(\mathbb{R},\sqrt{2},+,-)$, el término tendría el valor de (haciendo $\iota(v_i)=\sqrt{i}$):
        \begin{equation*}
            \begin{split}
                \hat{\iota}(e_G\cdot(v_3))&=\sqrt{2}-\sqrt{3}
            \end{split}
        \end{equation*}

        Considere ahora la fórmula $\varphi$ dada por:
        \begin{equation*}
            e_G\cdot(v_3)^{-1}=v_4
        \end{equation*}
        se tiene que $\hat{\iota}(e_G\cdot(v_3)^{-1})=3^7$ y $\hat{i}(v_4)=7$. Por tanto, $\mathfrak{N}\nvDash\varphi$. Ahora, considere la fórmula $\varphi$ dada por:
        \begin{equation*}
            (\exists v_1)(e_G\times e_G=v_1 )
        \end{equation*}
        Se sabe que $\mathfrak{N}\vDash e_G\cdot e_G=v_{24}$ ($\hat{\iota}(e_G\cdot e_G)=3^3=27$ y $\hat{\iota}(v_{24})=24+3=27$), por ende
        \begin{equation*}
            \mathfrak{N}\vDash (\exists v_1)(e_G\cdot e_G=v_{1})
        \end{equation*}
    \end{exa}

    \begin{mydef}
        Sea $\mathcal{L}$ un modelo. Una $\mathcal{L}$-fórmula, es una fórmula formada a partir de $\mathcal{L}$.
    \end{mydef}

    \begin{mydef}
        Sea $\mathcal{L}=\left\{c_i,R_i,F_i \right\}$ un lenguaje de primer orden, y sea $\Sigma$ un conjunto de $\mathcal{L}$-fórmulas.
        \begin{enumerate}[label=(\textit{\arabic*})]
            \item Decimos que $\mathfrak{A}\vDash\Sigma[\iota]$ si para todo $\varphi\in\Sigma$, $\mathfrak{A}\vDash\varphi[\iota]$, siendo $\iota$ una interpretación.
            \item Decimos que un $\mathcal{L}$-modelo $\mathfrak{A}$ \textbf{satisface $\Sigma$} (denotado por $\mathfrak{A}\vDash\Sigma$), si para toda $\varphi\in\Sigma$ y para toda intepretación $\iota$, $\mathfrak{A}\vDash\varphi[i]$.
            \item Decimos que $\varphi$ es \textbf{consecuencia lógica de $\Sigma$}, escrito $\Sigma\vDash\varphi$, si par todo $\mathcal{L}$-modelo y para toda interpretación $\iota$ tal que $\mathfrak{A}\vDash\Sigma[\iota]$, se tiene que $\mathfrak{A}\vDash\varphi[\iota]$.
        \end{enumerate}
    \end{mydef}

    \begin{lema}
        Sea $\mathcal{L}$ un lenguaje de primer orden, $\mathfrak{A}$ una $\mathcal{L}$-estructura y $\cf{\iota,\upsilon}{\textup{Var}}{A}$ dos interpretaciones.
        \begin{enumerate}[label=(\textit{\arabic*})]
            \item Si $t$ es un término, con $\left\{v_{ i_1},...,v_{ i_k} \right\}$ el conjunto de variables que aparecen en $t$ y,
            \begin{equation*}
                \iota\upharpoonright\left\{v_{ i_1},...,v_{ i_k} \right\}=\upsilon\upharpoonright\left\{v_{ i_1},...,v_{ i_k} \right\}
            \end{equation*}
            entonces,
            \begin{equation*}
                \hat{\iota}(t)=\hat{\upsilon}(t)
            \end{equation*}
            \item Si $\varphi$ es una fórmula, $\left\{v_1,...,v_n \right\}$ es el conjunto de variables que aparecen libres en $\varphi$ y,
            \begin{equation*}
                \iota\upharpoonright\left\{v_{ i_1},...,v_{ i_k} \right\}=\upsilon\upharpoonright\left\{v_{ i_1},...,v_{ i_k} \right\}
            \end{equation*}
            entonces
            \begin{equation*}
                \mathfrak{A}\vDash\varphi[\iota]\textup{ si y solo si }\mathfrak{A}\vDash\varphi[\upsilon]
            \end{equation*}
        \end{enumerate}
    \end{lema}

    \begin{proof}
        De \textit{(1)}: Procederemos por inducción sobre $t$.
        \begin{itemize}
            \item[\textit{(b)}] Suponga que $t$ es una constante. De forma inmediata se tiene que $\iota(t)=\upsilon(t)$, es decir que $\hat{\iota}(t)=\hat{\upsilon}(t)$.
            \item[\textit{(b)}] Suponga que $t$ es una variable, entonces $t$ es $v_i$ para algún $i\in\left\{i_1,...,i_k \right\}$, así que este conjunto se reduce a un sólo elemento. De forma análoga al caso anterior se sigue $\hat{\iota}(t)=\hat{\upsilon}(t)$.
            \item[\textit{(i)}] Si $F_i$ es símbolo de función de aridad $l$, y $t_1,...,t_l$ son términos tales que
            \begin{equation*}
                \hat{\iota}(t_j)=\hat{\upsilon}(t_j),\quad\forall j\in\natint{1,l}
            \end{equation*}
            entonces:
            \begin{equation*}
                \begin{split}
                    \hat{\iota}\left(F_it_1\cdots t_l\right)&=f_i\left(\hat{\iota}(t_1),\cdots,\hat{\iota}(t_l) \right)\\
                    &=f_i\left(\hat{\upsilon}(t_1),\cdots,\hat{\upsilon}(t_l) \right)\\
                    &=\hat{\upsilon}\left(F_it_1\cdots t_l\right)\\
                \end{split}
            \end{equation*}
        \end{itemize}
        por inducción se sigue el resultado.
        
        De \textit{(2)}: Se procederá por inducción sobre $\varphi$:
        \begin{itemize}
            \item[\textit{(b)}] Si $\varphi$ es $=t_1t_2$, entonces $\mathfrak{A}\vDash\varphi[\iota]$ si y śolo si $\hat{\iota}(t_1)$ es igual a $\hat{\iota}(t_2)$, lo cual sucede si y sólo si $\hat{\upsilon}(t_1)$ es igual a $\hat{\upsilon}(t_2)$, es decir que $\mathfrak{A}\vDash\varphi[\upsilon]$.
            \item[\textit{(b)}] Si $\varphi$ es $R_it_1\cdots t_l$ donde $R_i$ es un símbolo de relación de aridad $l$ y, $t_1,...,t_l$ son términos, entonces $\mathfrak{A}\vDash\varphi[\iota]$ si y sólo si $(\hat{\iota}(t_1),...,\hat{\iota}(t_l))\in r_i$, es decir si y sólo si $(\hat{\upsilon}(t_1),...,\hat{\upsilon}(t_l))\in r_i$, lo cual sucede si y sólo si $\mathfrak{A}\vDash\varphi[\upsilon]$.
            \item[\textit{(i)}] Se tiene que $\mathfrak{A}\vDash\neg\varphi[\iota]$ si y sólo si $\mathfrak{A}\nvDash\varphi[\iota]$, si y sólo si (por hipótesis de inducción) $\mathfrak{A}\nvDash\varphi[\upsilon]$, si y sólo si $\mathfrak{A}\vDash\neg\varphi[\upsilon]$.
            \item[\textit{(i)}] Se tiene que $\mathfrak{A}\vDash(\varphi\Rightarrow\psi)[\iota]$ si y sólo si o bien $\mathfrak{A}\nvDash\varphi[\iota]$ o bien $\mathfrak{A}\vDash\psi[\iota]$, si y sólo (por hipótesis de inducción) si o bien $\mathfrak{A}\nvDash\varphi[\upsilon]$ o bien $\mathfrak{A}\vDash\psi[\upsilon]$, si y sólo si $\mathfrak{A}\vDash(\varphi\Rightarrow\psi)[\upsilon]$.
            \item[\textit{(i)}] Se tiene que $\mathfrak{A}\vDash(\exists v_i)\varphi[\iota]$, si y sólo si para algún $a\in A$, $\mathfrak{A}\vDash\varphi[\iota_{a/v_i}]$, si y sólo si $\mathfrak{A}\vDash\varphi[\upsilon_{a/v_i}]$ (pues $v_i$ es variable libre en $\varphi$, al tenerse que $\iota_{ a/v_i}\left(v_i\right)=a=\upsilon_{ a/v_i}\left(v_i\right)$, luego $\iota_{ a/v_i}\upharpoonright\left\{v_1,...,v_k \right\}=\upsilon_{ a/v_i}\upharpoonright\left\{v_1,...,v_k \right\}$, por la hipótesis de inducción se sigue que $\mathfrak{A}\vDash\varphi[\iota_{ a/v_i}]$ si y sólo si $\mathfrak{A}\vDash\varphi[\upsilon_{ a/v_i}]$), si y sólo si para algún $a\in A$, $\mathfrak{A}\vDash\varphi[\upsilon_{ a/v_i}]$, si y sólo si $\mathfrak{A}\vDash(\exists v_i)\varphi[\upsilon]$.
        \end{itemize}
        por inducción se sigue el resultado.
    \end{proof}

    \begin{obs}
        Se denotará por $\mathfrak{A}\vDash\varphi[a_1,...,a_k]$, tomando a una interpretación $\iota$ tal que $\iota(v_i)=a_i$ para $i\in\natint{1,k}$ y con $\left\{v_1,...,v_k \right\}$ las variables libres de $\varphi$. Esto por el lema anterior es válido, ya que las interpretaciones dependen solo del valor que tomen en las variables libres.

        Los corchetes nos dicen el valor de las variables libres.
    \end{obs}

    \begin{exa}
        Considere el lenguaje de primer orden de la teoría de grupos, $\mathcal{L}_{TG}=\left\{E,*,(\cdot)^{-1} \right\}$ y sea $\mathfrak{A}=\left(\mathbb{N},3,{}^{\wedge},S \right)$ una $\mathcal{L}_{ TG}$-estructura. Entonces
        \begin{equation*}
            \mathfrak{A}\nvDash(\exists v_5)(\forall v_7)(v_5*E=v_7*v_8)[10]
        \end{equation*}
        donde se cambia la única variable libre en la fórmula $v_8$ por el valor de $10$. Es decir, estamos diciendo que
        \begin{equation*}
            (\exists n\in\mathbb{N})(\forall m\in\mathbb{N})(n^3=m^{10})
        \end{equation*}
        y,
        \begin{equation*}
            \mathfrak{A}\vDash(\exists v_2)(v_2^{-1}*E=v_5*v_7)[2,3]
        \end{equation*}
        sería
        \begin{equation*}
            (\exists n\in\mathbb{N})((n+1)^2=2^3)
        \end{equation*}
    \end{exa}

    \begin{mydef}
        Un \textbf{enunciado} es una fórmula $\varphi$ tal que $\free\left(\varphi\right)=\emptyset$.
    \end{mydef}

    \begin{exa}
        Se tienen los siguientes ejemplos de enunciados:
        \begin{itemize}
            \item $(\forall v_1)(E*v_1=v_1)$.
            \item $(\forall v_1)(\forall v_2)(v_1*v_2=v_2*v_1)$.
        \end{itemize}
        al no depender de la interpretación, estos se toman como una especie de axiomas.
    \end{exa}

    \begin{obs}
        Si $\varphi$ es un enunciado, entonces o bien $\mathfrak{A}\vDash\varphi$ o bien $\mathfrak{A}\nvDash\varphi$, independientemente de la interpretación.
    \end{obs}

    '\begin{mydef}
        Sea $\mathcal{L}$ un lenguaje de primer orden.
        \begin{enumerate}
            \item Un conjunto de fórmulas $\Sigma$ es \textbf{consistente}, si $\Sigma\nvdash\varphi\land\neg\varphi$ para toda fórmula $\varphi$.

            Equivalentemente, existe una fórmula $\psi$ tal que $\Sigma\nvdash\psi$.
            \item Un conjunto de fórmulas $\Sigma$ es \textbf{satisfacible} (denotado por $\Sigma\vDash\Sigma$), si existe una $\mathcal{L}$-estructura (o $\mathcal{L}$-modelo) y una interpretación $\iota$ tal que
            \begin{equation*}
                \mathfrak{A}\vDash\varphi[\iota]
            \end{equation*}
            para todo $\varphi\in\Sigma$.
        \end{enumerate}
    \end{mydef}

    \begin{obs}
        Se tiene lo siguiente:
        \begin{itemize}
            \item Si Consistente implica Satisfacible, entonces $\Sigma\nvdash\varphi$ implica que $\Sigma\nvDash\varphi$.
            \item Si Satisfacible implica Consistente, entonces $\Sigma\nvDash\varphi$ implica que $\Sigma\nvdash\varphi$.
        \end{itemize}
        la primera parte se llama \textbf{Completud} ya siguiente \textbf{Correctud}.
    \end{obs}

    \begin{lema}[Nombre]
        Sea $\mathcal{L}$ un lenguaje de primer orden, $\Sigma$ un conjunto de $\mathcal{L}$-fórmulas consistente, entonces si $D_n$ es símbolo de constante, para todo $n\in\mathbb{N}$, se tiene que en el lenguaje:
        \begin{equation*}
            \mathcal{L}^*=\mathcal{L}\cup\left\{D_n\Big|n\in\mathbb{N} \right\}
        \end{equation*}
        $\Sigma$ sigue siendo consistente.
    \end{lema}

    \begin{proof}
        Supongamos que $\Sigma$ no es consistente en $\mathcal{L}^*$. Enotnces para algúna fórmula $\Sigma\vdash\varphi\land\neg\varphi$, como $\Sigma$ es consistente en $\mathcal{L}$, en la demostración de $\varphi\land\neg\varphi$ aparecen fórmulas involucrando a algunos $D_n$, $n\in\mathbb{N}$.
        
        La única forma en que pudo aparecer la constante $D_n$ es por que en algún momento se usó la instanciación universal. Existe $v_i$ tal que no aparece en la demostración, luego en la demostración podemos cambiar $D_n$ por $v_i$.

        Repitiendo este proceso inductivamente podemos cambiar todos los $D_{ n_j}$ por algún $v_{ n_j}$ que no aparecía inicialmente en la demostración, por tanto, la demostración en $\mathcal{L}^*$ se transforma en una demostración en $\mathcal{L}$, así que $\Sigma$ no es consistente en $\mathcal{L}$.
    \end{proof}

    \begin{lema}
        Sea $\mathcal{L}$ un lenguaje de primer orden, $\Sigma$ un conjunto consistente de $\mathcal{L}$-fórmulas y una $\mathcal{L}$-fórmula $\varphi$. Entonces o bien $\Sigma\cup\left\{\varphi\right\}$ es consistente ó bien $\Sigma\cup\left\{\neg\varphi \right\}$ es consistente. 
    \end{lema}

    \begin{proof}
        Análogo a lo que se hizo el capítulo pasado.
    \end{proof}

    \begin{theor}[\textbf{Teorema de Completud de Gödel, 1929}]
        Dado un lenguaje de primer orden $\mathcal{L}$, $\Sigma$ un conjunto de $\mathcal{L}$-fórmulas y una $\mathcal{L}$-fórmula $\varphi$ se tiene que:
        \begin{equation*}
            \Sigma\vDash\varphi\textup{ si y sólo si }\Sigma\vdash\varphi
        \end{equation*}
    \end{theor}

    \begin{proof}
        Basta probar que para todo conjunto de fórmulas $\Sigma$, $\Sigma$ es consistente si y sólo si es satisfacible.

        En efecto, si consistente es equivalente a satisfacible, entonces se tiene que $\Sigma\nvDash\varphi$, si y sólo si existe un $\mathcal{L}$-modelo $\mathfrak{A}$ y una interpretación $\iota$ tal que
        \begin{equation*}
            \mathfrak{A}\vDash\Sigma[\iota]
        \end{equation*}
        tal que
        \begin{equation*}
            \mathfrak{A}\nvDash\varphi[\iota]
        \end{equation*}
        lo cual sucede si y sólo si existe un $\mathcal{L}$-modelo $\mathfrak{A}$ y una interpretación $\iota$ tal que
        \begin{equation*}
            \mathfrak{A}\vDash\left(\Sigma\cup\left\{\neg\varphi \right\}\right)[\iota]
        \end{equation*}
        si y sólo si $\Sigma\cup\left\{\neg\varphi \right\}$ es satisfacible, si y sólo si $\Sigma\cup\left\{\neg\varphi\right\}$ es consistente, si y sólo si $\Sigma\nvdash\varphi$. Por lo que en efecto, solo basta probar lo enunciado.

        $\Rightarrow):$ Suponga que $\Sigma$ es un conjunto consistente de fórmulas en $\mathcal{L}$, debemos probar que $\Sigma$ es satisfacible. Lo haremos por pasos:
        \begin{enumerate}[label = \textit{(\alph*)}]
            \item Extendemos $\mathcal{L}$ a otro lenguaje
            \begin{equation*}
                \mathcal{L}^*=\mathcal{L}\cup\left\{D_n\Big|n\in\mathbb{N} \right\}
            \end{equation*}
            siendo $D_n$ símbolos de constante. Por un lema anterior se tiene que $\Sigma$ también es consistente en $\mathcal{L}^*$.
            \item Sea $\varphi_1,...,\varphi_n,...$ una enumeración efectiva de las $\mathcal{L}^*$-fórmulas (como lo hecho en lógica proposicional).
            \item Extendemos $\Sigma$ a
            \begin{equation*}
                \Sigma^*=\Sigma\cup\left\{\exists x\varphi_n\Rightarrow\varphi_n[D_n/x]\Big|n\in\mathbb{N} \right\}
            \end{equation*}
            afirmamos que $\Sigma^*$ sigue siendo consistente. En efecto, sea $\theta_n$ la fórmula $\exists x\varphi_n\Rightarrow\varphi_n[D_n/x]$. Si $\Sigma^*$ fuese inconsistente, tomamos $N$ el mínimo natural tal que
            \begin{equation*}
                \Sigma\cup\left\{\theta_n\Big|n\leq N \right\} 
            \end{equation*}
            es inconsistente. Entonces, $\Sigma\cup\left\{\theta_n\Big|n<N \right\}$ es consistente y $\Sigma\cup\left\{\theta_n\Big|n<N \right\}\vdash\neg\theta_N$ (por el Metateorema Demostración por Contradicción). En particular, a partir de este conjunto estamos demostrando que
            \begin{equation*}
                (\exists x)\varphi_N\land\neg\varphi_N[D_N/x]
            \end{equation*}
            simplificando se obtiene una demostración de
            \begin{equation*}
                \neg\varphi_N[D_N/x]
            \end{equation*}
            lo cual quiere decir que en las premisas no aparece $D_N$, así que intercambiamos $D_n$ por una variable $y$ que no apareció en la demostración. Así que a partir de $\Sigma\cup\left\{\theta_n\Big|n<N \right\}$ demostramos que $\neg\varphi_N[y/x]$ y $\exists x\varphi_N$.

            Hacemos generalización universal, ya que $y$ nunca apareció anteriormente, luego se demuestra que:
            \begin{equation*}
                \begin{split}
                    \forall x&\neg\varphi_N\\
                    \neg\exists x\neg\neg\varphi_N\\
                    \neg\exists x\varphi_N
                \end{split}
            \end{equation*}
            por tanto, $\exists x\varphi_N$ y $\neg\exists x\varphi_N$, lo cual es una contradicción, ya que esto implicaría que
            \begin{equation*}
                \Sigma\cup\left\{\theta_n\Big|n<N \right\}
            \end{equation*}
            es inconsistente. Por tanto, no existe mínimo $N$ tal que $\Sigma\cup\left\{\theta_n\Big|n\leq N \right\}$ sea inconsistente, lo cual implica que $\Sigma^*$ debe ser consistente.
            \item Definimos conjuntos de fórmulas $\Sigma\subseteq\Sigma^*=\Sigma_1\subseteq\Sigma_2\subseteq\cdots\subseteq\Sigma_n\subseteq\cdots$ tales que
            \begin{equation*}
                \Sigma_{ n+1}=\left\{
                    \begin{array}{lr}
                        \Sigma_n\cup\left\{\varphi_n \right\} & \textup{ si es consistente}.\\
                        \Sigma_n\cup\left\{\neg\varphi_n \right\} & \textup{ en caso contrario}.\\
                    \end{array}
                \right.,\quad\forall n\in\mathbb{N}
            \end{equation*}
            por inducción, al ser $\Sigma_1$ consistente se sigue de forma inmediata que $\Sigma_n$ es consistente, para todo $n\in\mathbb{N}$. Definimos
            \begin{equation*}
                \Sigma^{\infty}=\bigcup_{n\in\mathbb{N}}\Sigma_n
            \end{equation*}
            Afirmamos lo siguiente:
            \begin{itemize}
                \item \textbf{$\Sigma^{\infty}$ es consistente}, en efecto, de lo contrario se tendría que algún $\Sigma_n$ sería consistente.
                \item \textbf{$\Sigma^\infty$ es maximal respecto de ser consistente} (equivalentemente, para toda fórmula $\varphi$ o bien $\varphi\in\Sigma^\infty$ o bien $\neg\varphi\in\Sigma^\infty$). Esto es inmediato de la definición de $\Sigma^\infty$ y por ser $\varphi_1,...,\varphi_n,...$ una enumeración efectiva de todas las posibles $\mathcal{L}^*$-fórmulas.
                \item Para cada fórmula $\varphi$:
                \begin{equation*}
                    \begin{split}
                        \varphi\in\Sigma^\infty &\textup{ si y sólo si }\neg\varphi\notin\Sigma^\infty\\
                        &\textup{ si y sólo si }\Sigma^\infty\vdash\varphi \\
                    \end{split}
                \end{equation*}
            \end{itemize}
        \end{enumerate}

        $\Leftarrow$): Suponga que $\Sigma$ satisfacible, hay que probar que es consistente. En efecto, lo haremos por pasos:
        \begin{itemize}
            \item Si $\mathfrak{M}\vDash\varphi\Rightarrow\psi$ y $\mathfrak{M}\vDash\varphi$, entonces $\mathfrak{M}\vDash\psi$.
            \item Si $\phi$ es axioma lógico, entonces $\mathfrak{M}\vDash\phi$.
            \begin{enumerate}[label = \textit{(\alph*)}]
                \item Los de lógica proposicional.
                \item Los de lógica de primer orden:
                \begin{enumerate}[label = \textit{(\arabic*)}]
                    \item Suponga que $\phi$ es $(\forall x)(\varphi\Rightarrow\psi)\Rightarrow (\forall x\varphi\Rightarrow\forall x\psi)$. Sea $\mathfrak{M}$ un modelo tal que $\mathfrak{M}\vDash(\forall x)(\varphi\Rightarrow\psi)$ y supongamos que $\mathfrak{M}\vDash\varphi$. Esto significa que para cada $a\in M$, $\mathfrak{M}\vDash(\varphi\Rightarrow\psi)[a]$ y $\mathfrak{M}\vDash\varphi[a]$, por el inciso anterior se sigue que $\mathfrak{M}\vDash\psi[a]$. Lo cual implica que $\mathfrak{M}\vDash\forall x\psi$.
                \end{enumerate}
            \end{enumerate}
        \end{itemize}
        lo que sigue se prueba de forma análoga a lo hecho anteriormente.
    \end{proof}

    \chapter*{Ejemplo Nuevo tipo de Tabla}

    \begin{center}
        \setcounter{tablec}{1}
        \begin{tabular}{l r l c l r}
            & \pstable{tablec} & $(A\lor B)$ & $\Rightarrow$ & $(C\land D)$ & Premisa \\
            & \pstable{tablec} & $(C\lor E)$ & $\Rightarrow$ & $(\neg F\land H)$ & Premisa \\
            & \pstable{tablec} & $(F\lor G)$ & $\Rightarrow$ & $(A\land I)$ & Premisa \\
            \hline
            & & & $\therefore$ & $\neg F$ & \\
        \end{tabular}
    \end{center}
    

\end{document}