% Turing Machine
% Author: Sebastian Sardina
\documentclass[a4paper,10pt]{standalone}
\usepackage{tikz}
%%%<
\usepackage{verbatim}
\usepackage[active,tightpage]{preview}
\PreviewEnvironment{tikzpicture}
\setlength\PreviewBorder{5pt}%
%%%>
\begin{comment}
:Title: Turing Machine
:Tags: Chains;Computer science
:Author: Sebastian Sardina
:Slug: turing-machine-2
Partly based on Ludger Humbert's pics of Unviersal Turing Machine at
https://haspe.homeip.net/projekte/ddi/browser/tex/pgf2/turingmaschine-schema.tex
\end{comment}

\usetikzlibrary{chains,fit,shapes}
\begin{document}
\begin{tikzpicture}
    \tikzstyle{every path}=[very thick]

    \edef\sizetape{0.7cm}
    \tikzstyle{tmtape}=[draw,minimum size=\sizetape]
    \tikzstyle{tmhead}=[arrow box,draw,minimum size=.5cm,arrow box
    arrows={east:.25cm, west:0.25cm}]

    %% Draw TM tape
    \begin{scope}[start chain=1 going right,node distance=-0.15mm]
        \node [on chain=1,tmtape,draw=none] {$\ldots$};
        \node [on chain=1,tmtape] {};
        \node [on chain=1,tmtape] (input) {1};
        \node [on chain=1,tmtape] {1};
        \node [on chain=1,tmtape] {0};
        \node [on chain=1,tmtape] {0};
        \node [on chain=1,tmtape] {0};
        \node [on chain=1,tmtape] {0};
        \node [on chain=1,tmtape] {};
        \node [on chain=1,tmtape,draw=none] {$\ldots$};
        \node [on chain=1] {\textbf{Banda de Entrada y Salida}};
    \end{scope}

    %% Draw TM Finite Control
    \begin{scope}
    [shift={(3cm,-5cm)},start chain=circle placed {at=(-\tikzchaincount*60:1.5)}]
    \foreach \i in {s_i,s_1,s_2,s_3,\ddots,s_f}
        \node [on chain] {$\i$};

    % Arrow to current state
    \node (center) {};
    \draw[->] (center) -- (circle-2);

    \node[rounded corners,draw=black,thick,fit=(circle-1) (circle-2) (circle-3) 
        (circle-4) (circle-5) (circle-6),
                label=below:\textbf{Posibles Estados}] (fsbox)
            {};
    \end{scope}

    %% Draw TM head below (input) tape cell
    \node [tmhead,yshift=-.3cm] at (input.south) (head) {$s_1$};

    %% Link Finite Control with Head
    \path[->,draw] (fsbox.north) .. controls (4.5,-1) and (0,-2) .. node[right] 
                (headlinetext)
                {} 
                (head.south);
    \node[xshift=5cm] at (headlinetext)  
                {\begin{tabular}{c} 
                    \textbf{El cabezal se mueve a la derecha o izquierda} \\  
                    \textbf{(puede no moverse)} 
                \end{tabular}};
    \end{tikzpicture}
\end{document}