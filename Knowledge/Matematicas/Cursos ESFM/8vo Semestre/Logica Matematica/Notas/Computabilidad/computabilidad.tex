\documentclass[12pt]{report}
\usepackage[spanish]{babel}
\usepackage[utf8]{inputenc}
\usepackage{amsmath}
\usepackage{amssymb}
\usepackage{amsthm}
\usepackage{graphics}
\usepackage{subfigure}
\usepackage{lipsum}
\usepackage{array}
\usepackage{multicol}
\usepackage{enumerate}
\usepackage[framemethod=TikZ]{mdframed}
\usepackage[a4paper, margin = 1.5cm]{geometry}
\usepackage{tikz}
\usepackage{pgffor}
\usepackage{ifthen}
\usetikzlibrary{positioning,calc}

\newcounter{it}
\newcommand*\watermarktext[1]{\begin{tabular}{c}
    \setcounter{it}{1}%
    \whiledo{\theit<100}{%
    \foreach \col in {0,...,15}{#1\ \ } \\ \\ \\
    \stepcounter{it}%
    }
    \end{tabular}
    }

\AddToHook{shipout/foreground}{
    \begin{tikzpicture}[remember picture,overlay, every text node part/.style={align=center}]
        \node[rectangle,black,rotate=30,scale=2,opacity=0.04] at (current page.center) {\watermarktext{Cristo Daniel Alvarado ESFM\quad}};
  \end{tikzpicture}
}
%En esta parte se hacen redefiniciones de algunos comandos para que resulte agradable el verlos%

\def\proof{\paragraph{Demostración:\\}}
\def\endproof{\hfill$\blacksquare$}

\def\sol{\paragraph{Solución:\\}}
\def\endsol{\hfill$\square$}

%En esta parte se definen los comandos a usar dentro del documento para enlistar%

\newtheoremstyle{largebreak}
  {}% use the default space above
  {}% use the default space below
  {\normalfont}% body font
  {}% indent (0pt)
  {\bfseries}% header font
  {}% punctuation
  {\newline}% break after header
  {}% header spec

\theoremstyle{largebreak}

\newmdtheoremenv[
    leftmargin=0em,
    rightmargin=0em,
    innertopmargin=0pt,
    innerbottommargin=5pt,
    hidealllines = true,
    roundcorner = 5pt,
    backgroundcolor = gray!60!red!30
]{exa}{Ejemplo}[section]

\newmdtheoremenv[
    leftmargin=0em,
    rightmargin=0em,
    innertopmargin=0pt,
    innerbottommargin=5pt,
    hidealllines = true,
    roundcorner = 5pt,
    backgroundcolor = gray!50!blue!30
]{obs}{Observación}[section]

\newmdtheoremenv[
    leftmargin=0em,
    rightmargin=0em,
    innertopmargin=0pt,
    innerbottommargin=5pt,
    rightline = false,
    leftline = false
]{theor}{Teorema}[section]

\newmdtheoremenv[
    leftmargin=0em,
    rightmargin=0em,
    innertopmargin=0pt,
    innerbottommargin=5pt,
    rightline = false,
    leftline = false
]{propo}{Proposición}[section]

\newmdtheoremenv[
    leftmargin=0em,
    rightmargin=0em,
    innertopmargin=0pt,
    innerbottommargin=5pt,
    rightline = false,
    leftline = false
]{cor}{Corolario}[section]

\newmdtheoremenv[
    leftmargin=0em,
    rightmargin=0em,
    innertopmargin=0pt,
    innerbottommargin=5pt,
    rightline = false,
    leftline = false
]{lema}{Lema}[section]

\newmdtheoremenv[
    leftmargin=0em,
    rightmargin=0em,
    innertopmargin=0pt,
    innerbottommargin=5pt,
    roundcorner=5pt,
    backgroundcolor = gray!30,
    hidealllines = true
]{mydef}{Definición}[section]

\newmdtheoremenv[
    leftmargin=0em,
    rightmargin=0em,
    innertopmargin=0pt,
    innerbottommargin=5pt,
    roundcorner=5pt
]{excer}{Ejercicio}[section]

%En esta parte se colocan comandos que definen la forma en la que se van a escribir ciertas funciones%

\newcommand\abs[1]{\ensuremath{\left|#1\right|}}
\newcommand\divides{\ensuremath{\bigm|}}
\newcommand\cf[3]{\ensuremath{#1:#2\rightarrow#3}}
\newcommand\contradiction{\ensuremath{\#_c}}
\newcommand\natint[1]{\ensuremath{\left[\big|#1\big|\right]}}

\begin{document}
    \setlength{\parskip}{5pt} % Añade 5 puntos de espacio entre párrafos
    \setlength{\parindent}{12pt} % Pone la sangría como me gusta
    \title{Curso de Lógica Matemática
    
    Teoría de la Computabilidad}
    \author{Cristo Daniel Alvarado}
    \maketitle

    \tableofcontents %Con este comando se genera el índice general del libro%

    %\setcounter{chapter}{3} %En esta parte lo que se hace es cambiar la enumeración del capítulo%

    \newpage

    \setcounter{chapter}{2}

    \chapter{Conjuntos y Funciones computables}

    Todo de lo que se va a tratar esta parte es de: ¿Cómo formalizar la noción de \textit{procedimiento mecánico, efectivo} o \textit{sistemático}? Con esto nos referimos a:
    \begin{itemize}
        \item Tener un número finito de instrucciones.
        \item Terminar el procedimiento en un número finito de pasos.
        \item Usar únicamente \textit{papel y lápiz}.
        \item No requiere razonamiento, solo se siguen reglas.
    \end{itemize}

    Básicamente se pretendía que dada una fórmula, encontrar un algoritmo que nos diga si esa fórmula es verdadera o falsa. Básicamente se pretendía formalizar las demostraciones para ver lo que nosotros podemos demostrar únicamente usando los axiomas.

    Turing y Alonzo Church eventualmente se hicieron preguntas en la misma dirección. En la Tesis de Church-Turing se probó que estas tres preguntas en realidad se reducen a un mismo problema.

    \section{Máquinas de Turing}

    \begin{mydef}
        Una \textbf{máquina de Turing} consta de:
        \begin{itemize}
            \item Un \textit{alfabeto}, un conjunto finito $L$.
            \item Un conjunto $S$ de \textit{estados}.
            \item Una función parcial $\cf{T}{L^*\times S}{L^*\times S\times\left\{<,-,> \right\}}$ llamada \textit{función de transición}.
        \end{itemize}
        donde $L^*=L\cup\left\{* \right\}$.
    \end{mydef}

    Intuitivamente, uno debe imaginar que esto es una especie de \textit{computadora rudimentaria}. Generalmente esto se conceptualiza como una cinta.

    \begin{tikzpicture}[every node/.style={block},
        block/.style={minimum height=1.5em,outer sep=0pt,draw,rectangle,node distance=0pt}]
    \node (A) {$\sigma$};
    \node (B) [left=of A] {$\ldots$};
    \node (C) [left=of B] {$B$};
    \node (D) [right=of A] {$\ldots$};
    \node (E) [right=of D] {$\$ $};
    \node (F) [above = 0.75cm of A,draw=red,thick] {\textsf q};
    \draw[-latex] (F) -- (A);
    \draw[-latex,blue] ($(F.east)!0.5!(A.east)$) -- ++(7mm,0);
    \draw (C.north west) -- ++(-1cm,0) (C.south west) -- ++ (-1cm,0) 
                    (E.north east) -- ++(1cm,0) (E.south east) -- ++ (1cm,0);
    \end{tikzpicture}

    \begin{exa}
        Considere $L=\left\{1 \right\}$, $S=\left\{s_i,s_1,s_2 \right\}$ y,
        \begin{equation*}
            T=\left\{(s_i,*,s_1,*,>),(s_i,1,s_1,1,>),(s_1,1,s_1,1,>),(s_1,1,s_2,1,-) \right\}
        \end{equation*}
        La cinta se ve más o menos así:

        \begin{center}
            \begin{tikzpicture}[every node/.style={block},
                block/.style={minimum height=1.5em,outer sep=0pt,draw,rectangle,node distance=0pt}]
            \node (1) {$*$};
            \foreach \x in {2,3,4} \node (\x) [right = of 1+\x] {$*$};
            \end{tikzpicture}
        \end{center}
    \end{exa}


    \chapter{Teoremas de Completud}



\end{document}