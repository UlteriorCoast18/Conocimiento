\documentclass[12pt]{report}
\usepackage[spanish]{babel}
\usepackage[utf8]{inputenc}
\usepackage{amsmath}
\usepackage{amssymb}
\usepackage{amsthm}
\usepackage{graphics}
\usepackage{subfigure}
\usepackage{lipsum}
\usepackage{array}
\usepackage{multicol}
\usepackage{enumerate}
\usepackage[framemethod=TikZ]{mdframed}
\usepackage[a4paper, margin = 1.5cm]{geometry}
\usepackage{tikz}
\usepackage{pgffor}
\usepackage{ifthen}
\usepackage{enumitem}
\usepackage{hyperref}
\usepackage{setspace}
\usepackage{bbm}

\usetikzlibrary{shapes.multipart}

\newcounter{it}
\newcommand*\watermarktext[1]{\begin{tabular}{c}
    \setcounter{it}{1}%
    \whiledo{\theit<100}{%
    \foreach \col in {0,...,15}{#1\ \ } \\ \\ \\
    \stepcounter{it}%
    }
    \end{tabular}
    }

\AddToHook{shipout/foreground}{
    \begin{tikzpicture}[remember picture,overlay, every text node part/.style={align=center}]
        \node[rectangle,black,rotate=30,scale=2,opacity=0.04] at (current page.center) {\watermarktext{Cristo Daniel Alvarado ESFM\quad}};
  \end{tikzpicture}
}
%En esta parte se hacen redefiniciones de algunos comandos para que resulte agradable el verlos%

\def\proof{\paragraph{Demostración:\\}}
\def\endproof{\hfill$\blacksquare$}

\def\sol{\paragraph{Solución:\\}}
\def\endsol{\hfill$\square$}

%En esta parte se definen los comandos a usar dentro del documento para enlistar%

\newtheoremstyle{largebreak}
  {}% use the default space above
  {}% use the default space below
  {\normalfont}% body font
  {}% indent (0pt)
  {\bfseries}% header font
  {}% punctuation
  {\newline}% break after header
  {}% header spec

\theoremstyle{largebreak}

\newmdtheoremenv[
    leftmargin=0em,
    rightmargin=0em,
    innertopmargin=0pt,
    innerbottommargin=5pt,
    hidealllines = true,
    roundcorner = 5pt,
    backgroundcolor = gray!60!red!30
]{exa}{Ejemplo}[section]

\newmdtheoremenv[
    leftmargin=0em,
    rightmargin=0em,
    innertopmargin=0pt,
    innerbottommargin=5pt,
    hidealllines = true,
    roundcorner = 5pt,
    backgroundcolor = gray!50!blue!30
]{obs}{Observación}[section]

\newmdtheoremenv[
    leftmargin=0em,
    rightmargin=0em,
    innertopmargin=0pt,
    innerbottommargin=5pt,
    rightline = false,
    leftline = false
]{theor}{Teorema}[section]

\newmdtheoremenv[
    leftmargin=0em,
    rightmargin=0em,
    innertopmargin=0pt,
    innerbottommargin=5pt,
    rightline = false,
    leftline = false
]{propo}{Proposición}[section]

\newmdtheoremenv[
    leftmargin=0em,
    rightmargin=0em,
    innertopmargin=0pt,
    innerbottommargin=5pt,
    rightline = false,
    leftline = false
]{cor}{Corolario}[section]

\newmdtheoremenv[
    leftmargin=0em,
    rightmargin=0em,
    innertopmargin=0pt,
    innerbottommargin=5pt,
    rightline = false,
    leftline = false
]{lema}{Lema}[section]

\newmdtheoremenv[
    leftmargin=0em,
    rightmargin=0em,
    innertopmargin=0pt,
    innerbottommargin=5pt,
    roundcorner=5pt,
    backgroundcolor = gray!30,
    hidealllines = true
]{mydef}{Definición}[section]

\newmdtheoremenv[
    leftmargin=0em,
    rightmargin=0em,
    innertopmargin=0pt,
    innerbottommargin=5pt,
    roundcorner=5pt
]{excer}{Ejercicio}[section]

%En esta parte se colocan comandos que definen la forma en la que se van a escribir ciertas funciones%

\makeatletter
\def\thickhrulefill{\leavevmode \leaders \hrule height 1ex \hfill \kern \z@}
\def\@makechapterhead#1{%
  {\parindent \z@ \raggedright
    \reset@font
    \hrule
    \vspace*{10\p@}%
    \par
    \center \LARGE \scshape \@chapapp{} \huge \thechapter
    \vspace*{10\p@}%
    \par\nobreak
    \vspace*{10\p@}%
    \par
    \vspace*{1\p@}%
    \hrule
    %\vskip 40\p@
    \vspace*{60\p@}
    \Huge #1\par\nobreak
    \vskip 50\p@
  }}

\def\section#1{%
  \par\bigskip\bigskip
  \hrule\par\nobreak\noindent
  \refstepcounter{section}%
  \addcontentsline{toc}{chapter}{#1}%
  \reset@font
  { \large \scshape
    \strut\S \thesection \quad
    #1}% 
    \hrule   
  \par
  \medskip
}

\newcommand\abs[1]{\ensuremath{\left|#1\right|}}
\newcommand\divides{\ensuremath{\bigm|}}
\newcommand\cf[3]{\ensuremath{#1:#2\rightarrow#3}}
\newcommand\contradiction{\ensuremath{\#_c}}
\newcommand\natint[1]{\ensuremath{\left[\big|#1\big|\right]}}
\newcommand\pot[1]{\ensuremath{\mathcal{P}\left(#1\right)}}
\newcommand\dom[1]{\ensuremath{\textup{dom}\left(#1\right)}}
\newcommand\ran[1]{\ensuremath{\textup{ran}\left(#1\right)}}
\newcommand{\bbm}[1]{\ensuremath{\mathbbm{#1}}}
\newcommand{\winecomma}[1]{\ensuremath{\ulcorner#1\urcorner}}

\begin{document}
    \setlength{\parskip}{5pt} % Añade 5 puntos de espacio entre párrafos
    \setlength{\parindent}{12pt} % Pone la sangría como me gusta
    \title{Curso de Lógica Matemática
    
    Teoría de Conjuntos}
    \author{Cristo Daniel Alvarado}
    \maketitle

    \tableofcontents %Con este comando se genera el índice general del libro%

    \newpage

    \setcounter{chapter}{4}

    \chapter{Teoría de Conjuntos}

    \section{Introducción}

    Recordemos que el lenguaje de la teoría de conjuntos consta de:
    \begin{equation*}
        \mathcal{L}_{TC}=\left\{\in \right\}
    \end{equation*}
    De ahora en adelante haremos las siguientes abreviaciones:
    \begin{center}
        \begin{tabular}{cc}
            \hline
            Fórmula & Abreviación \\
            \hline
            $\neg(x\in y)$ & $x\notin y$ \\
            $\forall z(z\in x\Rightarrow z\in y)$ & $x\subseteq y$\\
            $\exists!y\chi(y)$ & $\exists y(\chi(y)\land\forall z(\chi(z)\Rightarrow z=y))$\\
            $\forall a\in u\chi(a)$ & $(\forall a)(a\in u\Rightarrow\chi(a))$ \\
            $\exists a\in u\chi(a)$ & $\exists a(a\in u\land\chi(a))$ \\
            \hline
        \end{tabular}
    \end{center}

    \begin{mydef}[\textbf{Axiomas de la teoría de conjuntos}]
        Tenemos los siguientes axiomas en $\mathcal{L}_{TC}$:
        \begin{enumerate}[label = \textit{(\arabic*)}]
            \setcounter{enumi}{-1}
            \item \textbf{Existencia}: $\exists x(x=x)$.
            \item \textbf{Extensionalidad}: $\forall x\forall y((\forall z)(z\in x\iff z\in y)\Rightarrow x=y)$. En otras palabras, $\forall x\forall y((x\subseteq y\land y\subseteq x)\Rightarrow x=y)$.
            \item \textbf{Comprensión/Separación}: Para cada fórnula $\varphi(x)$ con una variable libre:
            \begin{equation*}
                \forall a\exists b\forall z(z\in b\iff (z\in a\land \varphi[z/x]))
            \end{equation*}
            informalmente, existe el conjunto $\left\{x\in a\Big|\varphi(x) \right\}$.
            \item \textbf{Par}: Se tiene que:
            \begin{equation*}
                \forall z\forall b\exists u\forall z(z\in u\iff(z= a\lor z\in b))
            \end{equation*}
            informalmente, dados $a,b$ existe el conjunto $\left\{a,b\right\}$.
            \item \textbf{Unión}: $\forall z\exists u\forall z(z\in u\iff(\exists b)(b\in a\land z\in b))$.
            informalmente, dado $a$ existe el conjunto $\bigcup_{ b\in a}b$.
            \item \textbf{Potencia}: $\forall z\exists b\forall z(z\in b\iff z\subseteq a)$.
            informalmente, dado $a$, existe $\pot{}$.
            \item \textbf{Infinitud}: $\exists x(\emptyset\in x\land \forall a\in x(a\cup\left\{a \right\}\in x))$.
            informalmente, decimos que existe un conjunto inductivo (por ejemplo, $\mathbb{N}$).
            \item \textbf{Fundación/Regularidad}: $\forall a\exists b(b\in a \land a\cap b=\emptyset)$.
            \item \textbf{Reemplazo}: Para cada fórmula $\psi(x,y)$ con dos variables libres:
            \begin{equation*}
                \forall x\exists!y\psi(x,y)\Rightarrow\forall y\exists v\forall y(y\in v\iff \exists x\in u\psi(x,y))
            \end{equation*}
            \item \textbf{Elección}: $\forall u(u\neq\emptyset\land\forall a(a\in u\Rightarrow a\neq\emptyset)\land \forall a,b\in u(a\neq b\Rightarrow a\cap b=\emptyset)\Rightarrow\exists s\forall a\in u(\exists! z(z\in a\cap s)))$.
        \end{enumerate}
    \end{mydef}

    \begin{theor}
        Existe un único conjunto sin elementos. En otras palabras, $(\exists! x)(\forall z)(z\notin x)$.
    \end{theor}

    \begin{proof}
        Sea $A$ un conjunto. Por el esquema de comprensión existe el conjunto:
        \begin{equation*}
            \emptyset=\left\{x\in A\Big|x\neq x \right\}
        \end{equation*}
        como por un axioma $(\forall z)z=z$, entonces $(\forall z)(z\notin \emptyset)$. Si $A$ es otro conjunto tal que $(\forall z)(z\notin A)$. Ahora veamos la unicidad. Sea $\emptyset^*$ un conjunto tal que $(\forall z)(z\notin \emptyset^*)$. Se tiene que:
    \end{proof}

    \begin{mydef}
        Al único conjunto sin elementos lo denominaremos por \textbf{conjunto vacío} o simplemente \textbf{vacío} y lo denotaremos por $\emptyset$.
    \end{mydef}

    Ahora, también haremos las siguientes convenciones:
    \begin{center}
        \begin{tabular}{cc}
            \hline
            Fórmula & Abreviación \\
            \hline
            $\emptyset\in x$ & $(\exists y)(\forall z(z\notin y)\land y\in x )$ \\
            $x\in\emptyset$ & $x\neq x$ \\
            $\emptyset = x$ & $(\forall z)(z\notin x)$ \\
            \hline
        \end{tabular}
    \end{center}
    
    \begin{theor}
        No existe un conjunto que contenga a todos los conjuntos.
    \end{theor}

    \begin{proof}
        Supongamos que $V$ es un conjunto tal que $(\forall x)(x\in V)$. Por el esquema de comprensión, existe el conjunto:
        \begin{equation*}
            A=\left\{x\in V\Big|x\notin x \right\}
        \end{equation*}
        Si $A\in A$, entonces $A\notin A$. En caso de que $A\notin A$ entonces $A\in A$. Por tanto, $A\in A\land \neg(A\in A)$, lo cual es una contradicción. Así que $V$ no existe.
    \end{proof}

    \begin{propo}
        Sean $a,b$ conjuntos. Entonces los conjuntos:
        \begin{equation*}
            a\cap b=\left\{x\in a\Big|x\in b \right\}=\left\{x\in b\Big|x\in a \right\}
        \end{equation*}
        y,
        \begin{equation*}
            a\setminus b=\left\{x\in a\Big|x\notin b \right\}
        \end{equation*}
        Además, también existe el conjunto:
        \begin{equation*}
            a\cup b=\left\{x\Big|x\in a\land x\in b \right\}
        \end{equation*}
    \end{propo}

    \begin{proof}
        Los primeros dos son inmediatos del esquema de comprensión. El tercero, por el axioma del par existe el conjunto $\left\{a,b \right\}$, luego del axioma de unión existe el conjunto:
        \begin{equation*}
            \bigcup_{ x\in\left\{a,b \right\}}x=a\cup b
        \end{equation*}
    \end{proof}

    \begin{propo}
        Existen los conjuntos $\left\{\emptyset \right\},\left\{\left\{\emptyset \right\}\right\},\left\{\left\{\left\{\emptyset \right\}\right\}\right\},...$.
    \end{propo}

    \begin{proof}
        Es inmediata del axioma del par.
    \end{proof}

    \begin{lema}[\textbf{Metalema}]
        Dados $a_1,...,a_n$ conjuntos, existe el conjunto $\left\{a_1,...,a_n\right\}$.
    \end{lema}

    \begin{proof}
        Se hace inducción sobre $n\in\mathbb{N}$, solo que note que en el lenguaje de teoría de conjuntos no existe el esquema de inducción (aún, solo hay que hacer algunos ajustes), por lo que habría una demostración para cada $n$. La prueba no es complicada y se hace rápida.
    \end{proof}

    \begin{obs}
        Notemos que en el axioma del conjunto potencia, solo sirve en el caso en que queramos obtener potencia de conjuntos infinitos, ya que en conjuntos finitos por el metateorema anterior es posible para cada $n\in\mathbb{N}$ y $a_1,...,a_n$ conjuntos, el conjunto:
        \begin{equation*}
            \pot{a_1,...,a_n}=\left\{\left\{a_1 \right\},...,\left\{a_n \right\},\left\{a_1,a_2 \right\},...,\left\{a_{ n-1},a_n \right\},\left\{a_1,a_2,a_3\right\},...,\left\{a_1,...,a_n \right\} \right\}
        \end{equation*}
        siempre existe (usando repetidamente el metalema anterior).
    \end{obs}

    \begin{mydef}
        Dados dos conjuntos $a,b$, definimos la \textbf{pareja ordenada $(a,b)$} como:
        \begin{center}
            \begin{tabular}{cc}
                \hline
                Fórmula & Abreviación \\
                \hline
                $(a,b)$ & $\left\{\left\{a \right\},\left\{a,b \right\} \right\}$ \\
                \hline
            \end{tabular}
        \end{center}
    \end{mydef}

    \begin{excer}
        Dados $a,b,c,d$ conjuntos se tiene que $(a,b)=(c,d)$ si y sólo si $a=c$ y $b=d$.
    \end{excer}

    \begin{proof}
        Ejercicio.
    \end{proof}

    \begin{mydef}
        Dados $A,B$ conjuntos definimos el \textbf{producto cartesiano de $A$ con $B$} por:
        \begin{equation*}
            A\times B=\left\{x\in\pot{\pot{A\cup B}} \Big| \exists a\in A\exists b\in B(x=(a,b)) \right\}
        \end{equation*}
    \end{mydef}
        
    \begin{mydef}
        Una \textbf{relación binaria $R$ entre $A$ y $B$} es un conjunto $R$ tal que $R\subseteq A\times B$.

        Cuando decimos \textbf{relación binaria $R$} nos referimos a que $(\exists A)(R\subseteq A\times A)$.

        Si $R$ es una relación binaria, se definen los conjuntos:
        \begin{equation*}
            \begin{split}
                \dom{R}&=\left\{x\in\bigcup_{ A\in \bigcup_{ B\in R}B}A\Big|\exists b (xRb) \right\} \\
            \dom{R}&=\left\{y\in\bigcup_{ A\in \bigcup_{ B\in R}B}A\Big|\exists a (aRy)  \right\} \\
            \end{split}
        \end{equation*}
        (aqui los conjuntos $A,B$ son diferentes a los de arriba).
    \end{mydef}

    \begin{mydef}
        Una \textbf{función $f$} es una relación binaria tal que:
        \begin{equation*}
            (\forall x\in\dom{d})(\exists !y)((x,y)\in f)
        \end{equation*}
    \end{mydef}

    \begin{obs}
        En otras palabras, si $f$ es una función y $x\in\dom{f}$ entonces $f(x)$ es el único $y$ tal que $(x,y)\in f$.
    \end{obs}

    \begin{mydef}
        Denotamos las funciones por $\cf{f}{A}{B}$. Decimos que:
        \begin{itemize}
            \item \textbf{$f$ es inyectiva}, si $f$ es función y $(\forall x,y\in\dom{f})(f(x)=f(y)\Rightarrow x=y)$.
            \item \textbf{$f$ es suprayectiva sobre $B$}, si $f$ es función y $(\forall y\in B\exists x\in\dom{f})(f(x)=y)$.
        \end{itemize}
    \end{mydef}

    \begin{propo}
        Dada una relación binaria $R$, el conjunto:
        \begin{equation*}
            R^{ -1}=\left\{x\in\ran{R}\times\dom{f}\Big|\exists a,b(x=(b,a)\land(a,b)\in R) \right\}
        \end{equation*}
        es una relación binaria sobre $A\times A$.
    \end{propo}

    \begin{proof}
        
    \end{proof}

    \begin{cor}
        Dada una función $f$, se tiene que $f^{-1}$ es una función si y sólo si $f$ es inyectiva. En este caso $\cf{f^{-1}}{\ran{f}}{\dom{f}}$.
    \end{cor}

    \begin{proof}
        
    \end{proof}

    Analicemos el axioma de infinitud. Nos dice que:
    \begin{equation*}
        (\exists X)(\emptyset \in X\land(\forall x\in X)(x\cup\left\{x \right\}\in X))
    \end{equation*}
    En este caso, $X$ se vería más o menos así:
    \begin{equation*}
        X=\left\{\emptyset,\left\{\emptyset \right\},\left\{\emptyset,\left\{\emptyset \right\}\right\},...\right\}
    \end{equation*}

    \begin{mydef}
        Definimos los conjuntos:
        \begin{center}
            \begin{tabular}{cc}
                $\winecomma{0}$ & $\emptyset$ \\
                $\winecomma{1}$ & $\left\{\emptyset\right\}=\left\{\winecomma{0}\right\}$ \\
                $\winecomma{2}$ & $\left\{\emptyset,\left\{\emptyset\right\}\right\}=\left\{\winecomma{0},\winecomma{1}\right\}$ \\
                $\vdots$ & $\vdots$ \\
                $\winecomma{n}$ & $\left\{\winecomma{0},...,\winecomma{n-1} \right\}$ \\
                $\vdots$ & $\vdots$ \\
            \end{tabular}
        \end{center}
    \end{mydef}

    \begin{obs}
        Por el axioma de infinitud todos estos conjuntos existen y más aún, existe un conjunto que los contiene a todos.
    \end{obs}

    \begin{mydef}
        Decimos que un conjunto $A$ es \textbf{inductivo} si $\emptyset\in A\land(\forall a\in A)(a\cup\left\{a \right\}\in A)$ 
    \end{mydef}

    \begin{cor}
        El conjunto obtenido a partir del axioma de infinitud, $X=\left\{\emptyset,\left\{\emptyset \right\},\left\{\emptyset,\left\{\emptyset \right\}\right\},...\right\}$ es inductivo.
    \end{cor}

    \begin{proof}
        Inmediata de su definición.
    \end{proof}

    \begin{mydef}
        Sea $X$ un conjunto inductivo. Se define:
        \begin{equation*}
            \omega=\left\{x\in X\Big|\forall Y, Y\textup{ inductivo implica }x\in Y \right\}
        \end{equation*}
        y hacemos $\mathbb{N}=\omega\setminus\left\{0\right\}$.
    \end{mydef}

    \begin{obs}
        En otras palabras, $\omega$ es el mínimo conjunto inductivo.
    \end{obs}

    \begin{theor}[\textbf{Teorema de Inducción}]
        Para todo $A\subseteq\omega$ si $\winecomma{0}\in A$ y $\forall\:\winecomma{n}\in A,\winecomma{n+1}\in A$ entonces $A=\omega$.
    \end{theor}

    \begin{proof}
        Por hipótesis $A$ es inductivo (se verifica rápidamente), luego por la observación anterior se sigue que $\omega\subseteq A$. Así que al tenerse que $A\subseteq\omega$ entonces $A=\omega$.
    \end{proof}

    \begin{theor}[\textbf{Teorema de Recursión}]
        Sea $Z$ un conjunto y sean $\zeta_0\in Z$ y $\cf{g}{Z}{Z}$. Entonces existe una única función $\cf{f}{\omega}{Z}$ tal que $f(0)=\zeta_0$ y $\forall\winecomma{n}\in\omega(f(\winecomma{n+1})=g(f(\winecomma{n})))$.
    \end{theor}

    \begin{proof}
        Ejercicio.
    \end{proof}

    \begin{obs}
        De ahora en adelante para no escribir tanto, denotaremos simplemente por $n=\winecomma{n}$.
    \end{obs}

    \begin{mydef}
        Para cada $n\in\omega$ definimos con el teorema anterior la función:
        \begin{equation*}
            \begin{split}
                A_n(0)&=n\\
                A_n(m+1)&=A_n(m)+1,\quad\forall m\in\omega \\
            \end{split}
        \end{equation*}
        en particular, definimos:
        \begin{equation*}
            n+m=A_n(m)
        \end{equation*}
        para todo $n,m\in\omega$. También, definimos un orden sobre $\omega\times\omega$ com la relación dada por:
        \begin{equation*}
            \leq \:= \left\{(n,m)\in\omega\times\omega\Big|\exists k\in\omega(n=m+k)\right\}
        \end{equation*}
    \end{mydef}

    \begin{propo}
        \label{noPertenenciaSiMismo}
        Para todo conjunto $x$, $x\notin x$.
    \end{propo}

    \begin{proof}
        Por el axioma del Par, $\left\{x\right\}$ es conjunto, luego por el axioma de fundación/regularidad existe un conjunto $y$ tal que $y\in\left\{x\right\}$ y $y\cap\left\{x\right\}=\emptyset$, en particular, $y=x$ y $x\cap\left\{x\right\}=\emptyset$, por lo que $x\notin x$.
    \end{proof}

    \begin{propo}
        No existen $x,y,z$ conjuntos tales que $x\in y\in z\in x$.
    \end{propo}

    \begin{proof}
        Considere el conjunto $\left\{x,y,z\right\}$. Por el axioma de fundación debe existir $a\in\left\{x,y,z\right\}$ tal que $a\cap\left\{x,y,z \right\}\neq\emptyset$, es decir que:
        \begin{equation*}
            x\notin a,y\notin a,z\notin a
        \end{equation*}
        Como $a\in\left\{x,y,z\right\}$, se tienen tres casos: $a=x$, entonces $z\notin x$, si $a=y$ entonces $x\notin y$ y, si $a=z$, entonces $y\notin z$.
    \end{proof}

    \begin{propo}
        No existe una función $f$ tal que $\dom{f}=\omega$ y que satisface $(\forall n\in\omega)(f(n+1)\in f(n))$.
    \end{propo}

    \begin{proof}
        Suponga que existe tal función $f$, entonces por el axioma de fundación, existe $x\in\ran{f}$ tal que $x\cap\ran{f}=\emptyset$. Como $x\in\ran{f}$, entonces existe $n\in\omega$ tal que $x=f(n)$, luego como $f(n+1)\in f(n)$, se sigue que:
        \begin{equation*}
            f(n+1)\in x\textup{ y }f(n+1)\in\ran{f}
        \end{equation*}
        por lo que $x\cap\ran{f}\neq\emptyset$\contradiction. Así que tal función $f$ no puede existir.
    \end{proof}

    \begin{mydef}
        Un conjunto es \textbf{transitivo} si $\forall y\in x(y\subseteq x)$.
    \end{mydef}

    \begin{exa}
        $\emptyset$, $\omega$ y $n$ son transitivos, para todo $n\in\mathbb{N}$.
    \end{exa}

    \begin{propo}
        Dado un conjunto $x$, los siguientes son equivalentes:
        \begin{enumerate}[label = \textit{(\arabic*)}]
            \item $x$ es transitivo.
            \item $z\in y\in x \Rightarrow z\in x$.
            \item $x\subseteq\pot{x}$.
            \item $\bigcup_{y\in x}y=x$.
        \end{enumerate}
    \end{propo}

    \begin{proof}
        Ejercicio.
    \end{proof}

    \begin{lema}
        Si $x$ es transitivo y $\forall y\in x$, $y$ es transitivo, entonces $\bigcup_{ y\in x}y$ y $\bigcap_{ y\in x}y$ son transitivos.
    \end{lema}

    \begin{proof}
        Sea $x$ tal que todos sus elementos son transitivos. Sea $a\in b\in\bigcup_{ y\in x}y$, entonces existe $y\in x$ tal que $b\in y$. Por ser $y$ transitivo se sigue que $a\in y$, luego $a\in\bigcup_{ y\in x}y$. Por tanto, de la proposición anterior se sigue que $\bigcup_{ y\in x}y$ es transitivo.

        Para probar que $\bigcap_{y\in x}y$ es transitivo se procede de forma análoga.
    \end{proof}

    \begin{mydef}[\textbf{Von Newmann}]
        Un \textbf{número ordinal} o simplemente \textbf{ordinal} es un conjunto transitivo y totalmente ordenado de manera estricta por la relación $\in$. Esto es, si $\alpha$ es un número ordinal:
        \begin{enumerate}[label = \textit{(\arabic*)}]
            \item (\textbf{Irreflexividad}) Para todo $\beta\in\alpha$, $\beta\notin\beta$.
            \item (\textbf{Transitividad}) Para todo $\beta,\gamma,\epsilon\in\alpha$, $\beta\in\gamma\in\epsilon$ implica $\beta\in\epsilon$.
            \item (\textbf{Totalmente ordenado}) Para todo $\beta,\gamma\in\alpha$ se tiene que $\beta\in\gamma$ o $\gamma\in\beta$ o $\beta=\gamma$.
        \end{enumerate}
    \end{mydef}

    \begin{exa}
        $\emptyset$, $n\in\omega$ y $\omega$ son números ordinales.
    \end{exa}

    \begin{lema}
        Si $\alpha$ es un ordinal, entonces $\alpha$ es bien ordenado por $\in$.
    \end{lema}

    \begin{proof}
        Sea $x\subseteq\alpha$ con $x\neq\emptyset$. Por el Axioma de Fundación existe $\beta\in x$ tal que $x\cap\beta=\emptyset$. Sea $\gamma\in x$, afirmamos que $\beta=\gamma$ o $\beta\in\gamma$.

        En efecto, tenemos tres casos: $\beta\in\gamma$ o $\beta=\gamma$ o $\gamma\in\beta$, esto último no puede suceder ya que ello implicaría que $\gamma\in x\cap\beta=\emptyset$\contradiction. Por tanto, $x$ tiene primer elemento, es decir:
        \begin{equation*}
            \beta=\min_{\in}(x)
        \end{equation*}

        Así que $\alpha$ es bien ordenado por $\in$.
    \end{proof}

    \begin{obs}
        En otras palabras, en el lema anterior estamos diciendo que todo subconjunto de $\alpha$ tiene un primer elemento bajo la relación $\in$.
    \end{obs}

    \begin{theor}
        Se tiene lo siguiente:
        \begin{enumerate}[label = \textit{(\arabic*)}]
            \item Si $\alpha$ es un ordinal y $x\in\alpha$, entonces $x$ también es un número ordinal.
            \item Si $\alpha$ es un número ordinal, entonces $\alpha\notin\alpha$.
            \item Si $\alpha,\beta,\gamma$ son números ordinales, entonces $\alpha\in\beta$ y $\beta\in\gamma$ implica que $\alpha\in\gamma$.
            \item Para cualesquiera $\alpha,\beta$ ordinales, entonces: $\alpha\in\beta$ o $\beta\in\alpha$ o $\alpha=\beta$.
        \end{enumerate}
    \end{theor}

    \begin{proof}
        De \textit{(1)}: Sea $\alpha$ un ordinal y $x\in\alpha$. Veamos que $x$ es transitivo y es totalmente ordenado de manera estricta por $\in$:
        \begin{itemize}
            \item (\textbf{Es transitivo}) Sea $z\in y\in x$. Como $\alpha$ es ordinal, entonces es transitivo, luego $y\in x\in\alpha$ implica que $y\in\alpha$, lo cual a su vez implica $z\in\alpha$ pues $z\in y\in\alpha$. Por tanto, $x,y,z\in\alpha$. Al ser $\in$ un orden total en $\alpha$, se sigue que $z\in x$, $z=x$ o $x\in z$. No pueden suceder ninguna de las dos últimas pues eso llevaría a una contradicción (por un lema anterior), por lo que $z\in x$.
            \item (\textbf{Es totalmente ordenado}). Como $x\in\alpha$, al ser $\alpha$ transitivo por definición se sigue que $x\subseteq\alpha$, por lo cual, como $\in$ es un orden total en $\alpha$, se sigue que también lo es en $x$.
        \end{itemize}
        por los dos incisos anteriores se sigue que $x$ es un número ordinal.

        De \textit{(2)}: Sea $\alpha$ un ordinal, en particular $\alpha$ es un conjunto, luego de la proposición (\ref{noPertenenciaSiMismo}) se sigue que $\alpha\notin\alpha$.

        De \textit{(3)}: Sean $\alpha,\beta,\gamma$ números ordinales tales que $\alpha\in\beta\in\gamma$, como $\gamma$ es un conjunto transitivo se sigue que $\alpha\in\gamma$.

        De \textit{(4)}: Sean $\alpha,\beta$ ordinales. Tomemos $\gamma=\alpha\cap\beta$. $\gamma$ es un número ordinal por (pues es transitivo por ser intersección de conjuntos transitivos y hereda el orden total de $\in$).

        Afirmamos que si $\gamma\neq\alpha$, entonces $\gamma\in\alpha$. Suponga que $\gamma\neq\alpha$, como $\gamma=\alpha\cap\beta$, entonces $\gamma\nsubseteq\alpha$, así que $\alpha\setminus\gamma$ es un subconjunto de $\alpha$ no vacío. Por el lema anterior este elemento tiene elemento mínimo, digamos:
        \begin{equation*}
            \delta=\min_{\in}(\alpha\setminus\gamma)
        \end{equation*}
        Probaremos que $\delta=\gamma$.
        \begin{itemize}
            \item Sea $\xi\in\delta$, entonces por minimalidad de $\delta$ debe suceder que $\xi\notin \alpha\setminus\gamma$, por ende $\xi\in\delta$. Se sigue así que $\delta\subseteq\gamma$.
            \item Sea $\xi\in\gamma\subseteq\alpha$. Por la linealidad de $\in$ tenemos tres casos:
            \begin{itemize}
                \item $\xi\in\delta$, en cuyo caso se sigue la contención.
                \item $\xi=\delta$, luego esto implica que $\delta=\xi\in\gamma\cap(\alpha\setminus\gamma)=\emptyset$\contradiction. Por tanto, esto no puede suceder.
                \item $\delta\in\xi$, en cuyo caso se sigue que $\delta\in\gamma\in\gamma$, al ser $\gamma$ ordinal se tiene que $\delta\in\gamma$ y $\delta\in\alpha\setminus\gamma$\contradiction. Por tanto, esto no puede suceder.
            \end{itemize}
            por los tres incisos anteriores, se sigue que $\xi\in\delta$, es decir que $\gamma\subseteq\delta$.
        \end{itemize}
        De los tres incisos anteriores se sigue que $\delta=\gamma$, en particular:
        \begin{equation*}
            \gamma=\delta\in\alpha\setminus\gamma\subseteq\alpha
        \end{equation*}
        es decir, $\gamma\in\alpha$. De forma análoga se prueba que si $\gamma\neq\beta$ entonces $\gamma\in\beta$.

        En conclusión, si $\gamma\neq\alpha$ y $\gamma\neq\beta$, entonces $\gamma\in\alpha\cap\beta=\gamma$\contradiction. Por tanto, $\gamma=\alpha$ o bien $\gamma=\beta$, es decir que $\beta=\gamma\in\alpha$ (si $\gamma\neq\alpha$ y $\gamma=\beta$) o bien $\alpha=\gamma\in\gamma$ (si $\gamma\neq\beta$ y $\gamma=\alpha$) o bien $\alpha=\gamma=\beta$ (si $\gamma=\alpha$ y $\gamma=\beta$), lo cual prueba el resultado.
    \end{proof}
    
    \begin{cor}[\textbf{Paradoja de Burali-Forti}]
        No existe un conjunto $X$ que contenga a todos los ordinales.
    \end{cor}

    \begin{proof}
        Suponga que tal conjunto existe, entonces existiría el conjunto:
        \begin{equation*}
            O=\left\{x\in X\Big|x\textup{ es un ordinal} \right\}
        \end{equation*}
        este es conjunto es un número ordinal por el Teorema anterior (es un conjunto transitivo y totalmente ordenado por $\in$), en particular, $O\in O$\contradiction. Por ende, $X$ no puede existir.
    \end{proof}

    \begin{mydef}
        Sean $\alpha,\beta$ ordinales. Decimos que $\alpha<\beta$ si $\alpha\in\beta$.
    \end{mydef}

    \begin{propo}
        La relación $<$ definida anteriormente es antisimétrica, transitiva, lineal y es total.
    \end{propo}

    \begin{proof}
        Es un reparrafraseo de las propiedades de los ordinales obtenidas en el teorema anterior.
    \end{proof}

    \begin{propo}
        Sean $\alpha,\beta$ ordianales. Entonces:
        \begin{equation*}
            \alpha\leq\beta\textup{ si y sólo si }\alpha\subseteq\beta
        \end{equation*}
    \end{propo}

    \begin{proof}
        $\Rightarrow):$ Si $\alpha\leq\beta$, entonces $\alpha<\beta$ o $\alpha=\beta$, es decir que $\alpha\in\beta$ o $\alpha=\beta$, como $\beta$ es transitivo se sigue que $\alpha\subset\beta$ o $\alpha=\beta$, es decir que $\alpha\subseteq\beta$.
        
        $\Leftarrow):$ Supongamos que $\alpha\subseteq\beta$. Por ser ordinales se tienen tres casos:
        \begin{itemize}
            \item $\alpha\in\beta$, en cuyo caso se sigue de forma inmediata que $\alpha<\beta$ lo cual implica que $\alpha\leq\beta$.
            \item $\alpha=\beta$, lo cual implica que $\alpha\leq\beta$.
            \item $\beta\in\alpha$, lo cual implica que $\beta\in\beta$\contradiction. Por ende, esto no puede suceder.
        \end{itemize}
        se sigue de los tres incisos anteriores que $\alpha\leq\beta$.
    \end{proof}

    \begin{obs}
        Chance y hay algo mal ya que hago la distinción entre $\subset$ y $\subseteq$ (una es contención propia y la otra es normal).
    \end{obs}

    \begin{mydef}
        Si $\alpha$ es un ordinal, definimos:
        \begin{equation*}
            S(\alpha)=\alpha\cup\left\{\alpha\right\}
        \end{equation*}
    \end{mydef}

    \begin{theor}
        Dado un ordinal $\alpha$ se tiene lo siguiente:
        \begin{enumerate}[label = \textit{(\arabic*)}]
            \item $S(\alpha)$ es un ordinal.
            \item $\alpha<S(\alpha)$.
            \item $\neg\exists\beta(\alpha<\beta<S(\alpha))$.
        \end{enumerate}
    \end{theor}

    \begin{proof}
        De \textit{(1)}: Por el teorema demostrado anteriormente la relación $\in$ es irreflexiva, transitiva y satisface la tricotomía entre los ordinales, en particular entre los elementos de $S(\alpha)$, que son todos ellos ordinales. Por tanto, $S(\alpha)$ es linealmente ordenado estricto por $\in$.

        Veamos que $S(\alpha)$ es un conjunto transitivo. Sea $\beta\in S(\alpha)=\alpha\cup\left\{\alpha\right\}$. Se tienen dos casos:
        \begin{itemize}
            \item $\beta\in\alpha$, en cuyo caso se sigue que $\beta\subseteq\alpha\subseteq\alpha\cup\left\{\alpha\right\}=S(\alpha)$.
            \item $\beta\in\left\{\alpha\right\}$, en cuyo caso se sigue que $\beta=\alpha\subseteq\alpha\cup\left\{\alpha\right\}=S(\alpha)$.            
        \end{itemize}
        por ambos incisos se sigue que $\beta\subseteq S(\alpha)$. Por ende, $S(\alpha)$ es transitivo. De esto y lo anterior se sigue que $S(\alpha)$ es un ordinal.

        De \textit{(2)}: Es inmediata de la definición.

        De \textit{(3)}: Probaremos que $\forall\beta(\beta\leq\alpha\left\{S(\alpha)\leq\beta \right\})$. Sea $\beta$ un ordinal arbitrario. Por tricotomía se tienen tres casos:
        \begin{itemize}
            \item $\beta<\alpha$ o $\beta=\alpha$, en cuyo caso se sigue que $\beta\leq\alpha$.
            \item Si $\alpha<\beta$, entonces $\alpha\in\beta$ y, por ser $\beta$ ordinal se tiene que $\alpha\subseteq\beta$, es decir que $\alpha\cup\left\{\alpha\right\}\subseteq\beta$. Por la proposición anterior se sigue que $S(\alpha)\leq\beta$.
        \end{itemize}
        por ambos casos se sigue el resultado.
    \end{proof}

    \begin{theor}
        Sea $x$ un conjunto tal que $(\forall\alpha)(\alpha\in x\Rightarrow\alpha\textup{ es un ordinal})$. Entonces:
        \begin{enumerate}[label = \textit{(\arabic*)}]
            \item $\bigcup_{\alpha\in x}\alpha$ es un ordinal, que además es el supremo de $x$.
            \item Si $x\neq\emptyset$, entonces $\bigcap_{\alpha\in x}\alpha$ es un ordinal que además es el mínimo de $x$.
        \end{enumerate}
    \end{theor}

    \begin{proof}
        De \textit{(1)}: Sea $\gamma=\bigcup_{\alpha\in x}\alpha$. Este conjunto es transitivo por ser la unión de conjuntos transitivos, además, $\gamma$ está linealmente ordenado por $\in$ de manera estricta ya que todos los ordinales lo están y tanto $\gamma$ como $\delta$ consisten de números ordinales.

        Por tanto, $\gamma$ es número ordinal. Veamos que $\gamma$ es el supremo de $x$. Se tiene que para todo $\alpha\in x$, $\alpha\subseteq x$ luego $\gamma\subseteq x$ y, si $\xi$ es otro ordinal tal que $\forall\alpha\in x\Rightarrow\alpha\subseteq\xi$ entonces $\gamma\subseteq\xi$ es decir $\gamma\in\xi$, por lo cual, $\gamma=\sup(x)$.

        De \textit{(2)}: Ejercicio. Es análogo a la prueba del inciso anterior.
    \end{proof}

    \begin{cor}[\textbf{Principio de Inducción Transfinita}]
        Dada una fórmula $\varphi(x)$, $(\forall\alpha\textup{ ordinal})((\forall\beta<\alpha)\varphi(\beta)\Rightarrow\varphi(\alpha))\Rightarrow(\forall\alpha)\varphi(\alpha)$.
    \end{cor}

    \begin{proof}
        Probaremos la contrapositiva, supongamos que $\neg\forall\alpha$ ordinal se tiene $\varphi(\alpha)$. Entonces, sea $\beta$ ordinal tal que $\neg\varphi(\beta)$.
        
        Se tienen dos casos:
        \begin{itemize}
            \item $\forall\xi<\beta \varphi(\xi)$. Sea $\alpha=\beta$.
            \item $\exists\xi<\beta\neg\varphi(\xi)$. Sea
            \begin{equation*}
                A=\left\{\xi<\beta\Big|\neg\varphi(\xi) \right\}
            \end{equation*}
            este conjunto es no vacío por hipótesis. Tomemos $\alpha=\min(A)$.
        \end{itemize}
        En ambos casos $\alpha$ es un mínimo tal que $\neg\varphi(\alpha)$, esto es:
        \begin{equation*}
            \neg\varphi(\alpha)\textup{ y }(\forall\gamma<\alpha)\varphi(\gamma)
        \end{equation*}
        Por tanto, $\exists\alpha\textup{ ordinal}((\forall\gamma<\alpha\varphi(\gamma)\land\neg\varphi(\alpha)))$, lo cual prueba la contrapositiva.
    \end{proof}

    \begin{mydef}
        Sean $(X,\leq)$ y $(Y,\preceq)$ son dos conjuntos parcialmente ordenados. Un \textbf{isomorfismo} entre ellos es una biyección $\cf{\varphi}{X}{Y}$ tal que $\forall a,b\in X$, $a\leq b\iff\varphi(a)\preceq\varphi(b)$.
    \end{mydef}

    \begin{lema}
        Sean $\alpha,\beta$ ordinales y sea $\cf{h}{\alpha}{\beta}$ un isomorfismo entre ellos. Entonces $\beta=\alpha$ y $h=\textup{id}_\alpha$.
    \end{lema}

    \begin{proof}
        Demostraremos que:
        \begin{equation*}
            (\forall\xi<\alpha) h(\xi)=\xi
        \end{equation*}
        Sea $\xi<\alpha$ arbitrario y supongamos que $\forall \eta<\xi$, $h(\eta)=\eta$. Entonces:
        \begin{equation*}
            h(\xi)=\left\{\gamma\Big|\gamma<h(\xi) \right\}
        \end{equation*}
        como $h$ es suprayectiva, entonces:
        \begin{equation*}
            \begin{split}
                h(\xi)&=\left\{h(\sigma)\Big|h(\delta)<h(\xi) \right\}\\
                &=\left\{\delta\Big|h(\delta)<h(\xi) \right\}\\
                &=\left\{\delta\Big|\delta<\xi \right\}\\
                &=\xi\\
            \end{split}
        \end{equation*}
        Por inducción transfinita se sigue que $h(\xi)=\xi$ para todo $\xi<\alpha$, es decir que $h=\textup{id}_\alpha$ y, por ende $\beta=h(\alpha)=\alpha$.
    \end{proof}

\end{document}