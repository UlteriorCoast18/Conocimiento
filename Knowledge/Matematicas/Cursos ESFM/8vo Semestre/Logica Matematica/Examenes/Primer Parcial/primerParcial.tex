\documentclass[12pt]{article}
\usepackage[spanish]{babel}
\usepackage[utf8]{inputenc}
\usepackage{amsmath,bbm}
\usepackage{amssymb}
\usepackage{amsthm}
\usepackage{graphics}
\usepackage{subfigure}
\usepackage{lipsum}
\usepackage{array}
\usepackage{multicol}
\usepackage{enumerate}
\usepackage[framemethod=TikZ]{mdframed}
\usepackage[a4paper, margin = 1.5cm]{geometry}
\usepackage{enumitem}
\usepackage{tikz}
\usepackage{pgffor}
\usepackage{ifthen}
\usepackage{forest}
\usetikzlibrary{shapes.multipart}

\newcounter{it}
\newcommand*\watermarktext[1]{\begin{tabular}{c}
    \setcounter{it}{1}%
    \whiledo{\theit<10}{%
    \foreach \col in {0,...,4}{#1\ \ } \\ \\ \\ \\
    \stepcounter{it}%
    }
    \end{tabular}
    }

\AddToHook{shipout/foreground}{
    \begin{tikzpicture}[remember picture,overlay, every text node part/.style={align=center}]
        \node[rectangle,black,rotate=30,scale=2,opacity=0.08] at (current page.center) {\watermarktext{Cristo Daniel Alvarado ESFM\quad}}; 
  \end{tikzpicture}
}

%En esta parte se hacen redefiniciones de algunos comandos para que resulte agradable el verlos%

\renewcommand{\theenumii}{\roman{enumii}}

\def\proof{\paragraph{Demostración:\\}}
\def\endproof{\hfill$\blacksquare$}

\def\sol{\paragraph{Solución:\\}}
\def\endsol{\hfill$\square$}

%En esta parte se definen los comandos a usar dentro del documento para enlistar%

\newtheoremstyle{largebreak}
  {}% use the default space above
  {}% use the default space below
  {\normalfont}% body font
  {}% indent (0pt)
  {\bfseries}% header font
  {}% punctuation
  {\newline}% break after header
  {}% header spec

\theoremstyle{largebreak}

\newmdtheoremenv[
    leftmargin=0em,
    rightmargin=0em,
    innertopmargin=-2pt,
    innerbottommargin=8pt,
    hidealllines = true,
    roundcorner = 5pt,
    backgroundcolor = gray!60!red!30
]{exa}{Ejemplo}[section]

\newmdtheoremenv[
    leftmargin=0em,
    rightmargin=0em,
    innertopmargin=-2pt,
    innerbottommargin=8pt,
    hidealllines = true,
    roundcorner = 5pt,
    backgroundcolor = gray!50!blue!30
]{obs}{Observación}[section]

\newmdtheoremenv[
    leftmargin=0em,
    rightmargin=0em,
    innertopmargin=-2pt,
    innerbottommargin=8pt,
    rightline = false,
    leftline = false
]{theor}{Teorema}[section]

\newmdtheoremenv[
    leftmargin=0em,
    rightmargin=0em,
    innertopmargin=-2pt,
    innerbottommargin=8pt,
    rightline = false,
    leftline = false
]{propo}{Proposición}[section]

\newmdtheoremenv[
    leftmargin=0em,
    rightmargin=0em,
    innertopmargin=-2pt,
    innerbottommargin=8pt,
    rightline = false,
    leftline = false
]{cor}{Corolario}[section]

\newmdtheoremenv[
    leftmargin=0em,
    rightmargin=0em,
    innertopmargin=-2pt,
    innerbottommargin=8pt,
    rightline = false,
    leftline = false
]{lema}{Lema}[section]

\newmdtheoremenv[
    leftmargin=0em,
    rightmargin=0em,
    innertopmargin=-2pt,
    innerbottommargin=8pt,
    roundcorner=5pt,
    backgroundcolor = gray!30,
    hidealllines = true
]{mydef}{Definición}[section]

\newmdtheoremenv[
    leftmargin=0em,
    rightmargin=0em,
    innertopmargin=-2pt,
    innerbottommargin=8pt,
    roundcorner=5pt
]{excer}{Problema}

%En esta parte se colocan comandos que definen la forma en la que se van a escribir ciertas funciones%

\newcommand\abs[1]{\ensuremath{\left|#1\right|}}
\newcommand\divides{\ensuremath{\bigm|}}
\newcommand\cf[3]{\ensuremath{#1:#2\rightarrow#3}}
\newcommand\natint[1]{\ensuremath{\left[\!\left[ #1\right]\!\right]}}
\newcommand{\afa}{\:
    \begin{tikzpicture}
        \draw [line width = 0.17 mm, black] (0,0) -- (-0.115,0.29);
        \draw [line width = 0.17 mm, black] (0,0) -- (0.115,0.29);
        \draw [line width = 0.17 mm, black] (-0.12,0) arc (190:-10:0.12cm);
    \end{tikzpicture}
    \:
}
\newcommand{\bbm}[1]{\mathbbm{#1}}
\newcommand{\pstable}[1]{\arabic{#1})\stepcounter{#1}}
\newcounter{tablec}
%Este símvolo es para casi todo salvo una cantidad finita

%recuerda usar \clearpage para hacer un salto de página

\begin{document}
    \setlength{\parskip}{5pt} % Añade 5 puntos de espacio entre párrafos
    \setlength{\parindent}{12pt} % Pone la sangría como me gusta
    \title{Primer Examen Parcial Lógica Matemática}
    \author{Cristo Daniel Alvarado}
    \maketitle

    \begin{excer}
        Las siguientes son tres fórmulas bien formadas en notación polaca, y utlizando las conectivas $\neg$, $\land$, $\lor$, $\Rightarrow$ y $\iff$. Escriba estas fórmulas en la notación \textit{usual} (usando tantos paréntesis como sea necesairo para evitar cualquier tipo de ambigüedad). Posteriormente, escriba la fórmula del primer inciso utilizando únicamente las conectivas $\neg$, $\land,\lor$, y escriba la fórmula del segundo inciso utilizando únicamente las conectivas $\neg$, $\Rightarrow$.
        \begin{enumerate}[label = \textit{(\alph*)}]
            \item $\Rightarrow\neg\land p_1p_4\lor p_3\land p_4\neg p_{17}$.
            \item $\land\Rightarrow p_2\neg\land p_{12}p_4\lor\neg p_{ 23}p_7$.
            \item $\lor p_{ 25}\land\Rightarrow p_3\neg p_4\lor\land\neg p_4p_{10}$.
        \end{enumerate}
    \end{excer}

    \begin{sol}
        Haremos la primera parte por cada inciso:
        \begin{enumerate}
            \item[De \textit{(a)}:]  Formemos el árbol a partir de la primera fórmula en notación polaca:
            \begin{center}
                \begin{tikzpicture}[level/.style={sibling distance = 4 cm}]
                    \node[circle,draw]{$\Rightarrow$}
                        child {node[circle, draw]{$\neg$} 
                            child {node[circle,draw]{$\land$}
                                child {node[circle,draw]{$p_1$}}
                                child {node[circle,draw]{$p_4$}}
                            }
                        }
                        child {node[circle,draw]{$\lor$} 
                            child{node[circle,draw]{$p_3$}}
                            child{node[circle,draw]{$\land$}
                                child{node[circle,draw]{$p_4$}}
                                child{node[circle,draw]{$\neg$}
                                    child {node[circle,draw]{$p_{17}$}}
                                }
                            }
                        };
                \end{tikzpicture}

                Figura 1. Árbol de la fórmula \textit{(a)}.
            \end{center}
            traduciendo mediante este árbol la fórmula a la notación usual obtenemos la fórmula siguiente:
            \begin{equation*}
                \neg(p_1\land p_4)\Rightarrow(p_3 \lor(p_4 \land\neg p_{ 17}))
            \end{equation*}
            \item[De \textit{(b)}:] Nuevamente, formemos ahora el árbol a partir de la segunda fórmula en notación polaca:
            \begin{center}
                \begin{tikzpicture}[
                    level 1/.style={sibling distance = 4 cm},
                    level 2/.style={sibling distance = 2 cm},
                    level 3/.style={sibling distance = 3 cm},
                    level 4/.style={sibling distance = 2 cm}
                    ]
                    \node[circle,draw]{$\land$}
                        child{node[circle,draw]{$\Rightarrow$}
                                child{node[circle,draw]{$p_2$}}
                                child{node[circle,draw]{$\neg$}
                                    child{node[circle,draw]{$\land$}
                                        child{node[circle,draw]{$p_{12}$}}
                                        child{node[circle,draw]{$p_{4}$}}
                                    }
                                }
                            }
                        child{node[circle,draw]{$\lor$}
                            child{node[circle,draw]{$\neg$}
                                child{node[circle,draw]{$p_{23}$}}
                            }
                            child{node[circle,draw]{$p_7$}}
                    };
                \end{tikzpicture}

                Figura 2. Árbol de la fórmula \textit{(b)}.
            \end{center}
            traduciendo mediante este árbol la fórmula a la notación usual obtenemos la fórmula siguiente:
            \begin{equation*}
                (p_2 \Rightarrow\neg(p_{12}\land p_4))\land(\neg p_{23}\lor p_7)
            \end{equation*}
            \item[De \textit{(c)}:] Nuevamente, formemos ahora el árbol a partir de la terera fórmula en notación polaca:
            \begin{center}
                \begin{tikzpicture}[
                    level 1/.style={sibling distance = 4 cm},
                    level 2/.style={sibling distance = 4 cm},
                    level 3/.style={sibling distance = 2 cm},
                    level 4/.style={sibling distance = 1 cm}
                    ]
                    \node[circle,draw]{$\lor$}
                        child{node[circle,draw]{$p_{25}$}}
                        child{node[circle,draw]{$\land$}
                            child{node[circle,draw]{$\Rightarrow$}
                                child{node[circle,draw]{$p_3$}}
                                child{node[circle,draw]{$\neg$}
                                    child{node[circle,draw]{$p_4$}}
                                }
                            }
                            child{node[circle,draw]{$\lor$}
                                child{node[circle,draw]{$\land$}
                                    child{node[circle,draw]{$\neg$}
                                        child{node[circle,draw]{$p_4$}}
                                    }
                                    child{node[circle,draw]{$p_{10}$}}
                                }
                                child{node[circle,draw]{}}
                            }
                        };
                \end{tikzpicture}

                Figura 3. Árbol de la fórmula \textit{(c)}.
            \end{center}
            traduciendo mediante este árbol la fórmula a la notación usual obtenemos la fórmula siguiente:
            \begin{equation*}
                p_{25}\lor((p_3\Rightarrow\neg p_4)\land((\neg p_4\land p_{10})\lor\textup{  }))
            \end{equation*}
            donde, notemos que el nodo hijo del nodo que tiene a $\lor$ en la 3era fila está vacío (el nodo se dejó pues se sabe que la opearción $\lor$ es binaria) y también en la reescritura de la ecuación, por lo que esta tercera fórmula no es una fórmula bien formada. 
        \end{enumerate}
        Ahora, escribamos la fórmula del primer inciso usando las conectivas $\neg$, $\land$ y $\lor$. Notemos que tenemos una implicación y, recordemos que si $\varphi$ y $\psi$ son dos fórmulas se puede escribir $\varphi\lor\psi$ como:
        \begin{equation*}
            \neg\varphi\Rightarrow\psi
        \end{equation*}
        por lo cual, si en la primera fórmula hacemos a $\varphi$ la fórmula $p_1\land p_4$ y a $\psi$ la fórmula $p_3\lor(p_4\land\neg p_{17})$, obtenemos que la primera fórmula puede ser reescrita como:
        \begin{equation*}
            (p_1\land p_4)\lor(p_3\lor(p_4\land\neg p_{17}))
        \end{equation*}

        Para reescribir la segunda fórmula usando únicamente las conectivas, recordemos que si $\varphi$ y $\psi$, entonces $\varphi\land\psi$ puede ser reescrita como
        \begin{equation*}
            \neg(\varphi\Rightarrow\neg\psi)
        \end{equation*}
        por lo cual, usando este hecho y el anterior, podemos reescribir la fórmula del segundo inciso como:
        \begin{equation*}
            \neg\left((p_2\Rightarrow\neg\neg(p_{12}\Rightarrow \neg p_4)) \Rightarrow\neg(\neg\neg p_{23}\Rightarrow p_7) \right)
        \end{equation*}
        y, si se nos permite quitar las dobles negaciones mediante los axiomas de la lógica proposicional y la única regla de inferencia Modus Ponens, obtenemos la fórmula reescrita:
        \begin{equation*}
            \neg\left((p_2\Rightarrow(p_{12}\Rightarrow\neg p_4)) \Rightarrow\neg(p_{23}\Rightarrow p_7) \right)
        \end{equation*}
        que a su vez usando otro axioma de la lógica proposicional puede ser reescrita como:
        \begin{equation*}
            (p_{23}\Rightarrow p_7)\Rightarrow(p_2\Rightarrow(p_{12}\Rightarrow\neg p_4))
        \end{equation*}
    \end{sol}

    \begin{excer}
        Escriba una demostración formal de validez (renglón por renglón, y justificando apropiadamente cada paso) del siguiente argumento:
        \begin{center}
            \setcounter{tablec}{1}
            \begin{tabular}{l r l c l r}
                & \pstable{tablec} & $(\alpha\lor\beta)$ & $\Rightarrow$ & $(\gamma\Rightarrow\delta)$ & Premisa \\
                & \pstable{tablec} & $\delta$ & $\Rightarrow$ & $\varepsilon$ & Premisa \\
                & \pstable{tablec} & $\varepsilon$ & $\Rightarrow$ & $(\alpha\land\zeta)$ & Premisa \\
                & \pstable{tablec} & $\gamma$ &  &  & Premisa \\
                \hline
                & & & $\therefore$ & $\alpha\iff\varepsilon$ & \\
            \end{tabular}
        \end{center}
    \end{excer}

    \begin{sol}
        Completemos la demostración:
        \begin{center}
            \setcounter{tablec}{1}
            \begin{tabular}{l r l c l r}
                & \pstable{tablec} & $(\alpha\lor\beta)$ & $\Rightarrow$ & $(\gamma\Rightarrow\delta)$ & Premisa \\
                & \pstable{tablec} & $\delta$ & $\Rightarrow$ & $\varepsilon$ & Premisa \\
                & \pstable{tablec} & $\varepsilon$ & $\Rightarrow$ & $(\alpha\land\zeta)$ & Premisa \\
                & \pstable{tablec} & $\gamma$ &  &  & Premisa \\
                $|\longrightarrow$ & \pstable{tablec} & $\alpha$ &  &  & Suposición \\
                $|$ & \pstable{tablec} & $\alpha\lor\beta$ &  &  & 5) Adición \\
                $|$ & \pstable{tablec} & $\gamma$ & $\Rightarrow$ & $\delta$ & 1) y 6) Modus Ponens \\
                $|$ & \pstable{tablec} & $\delta$ &  &  & 7) y 4) Modus Ponens \\
                $|$ & \pstable{tablec} & $\varepsilon$ &  &  & 2) y 8) Modus Ponens \\
                \hline
                 & \pstable{tablec} & $\alpha$ & $\Rightarrow$ & $\varepsilon$ & Lineas 5)-9) Metateorema de Deducción \\
                $|\longrightarrow$ & \pstable{tablec} & $\varepsilon$ &  &  & Suposición \\
                $|$ & \pstable{tablec} & $\alpha\land\zeta$ &  &  & 3) y 11) Modus Ponens \\
                $|$ & \pstable{tablec} & $\alpha$ &  &  & 12) Simplificación \\
                \hline
                 & \pstable{tablec} & $\varepsilon$ & $\Rightarrow$ & $\alpha$ & Lineas 11)-13) Metateorema de Deducción \\
                & \pstable{tablec} & $(\alpha\Rightarrow\varepsilon)$ & $\land$ & $(\varepsilon\Rightarrow\alpha)$ & 10) y 14) Conjunción \\
                & \pstable{tablec} & $\alpha$ & $\iff$ & $\varepsilon$ & 15) Reescritura \\
                \hline
                & & & $\therefore$ & $\alpha\iff\varepsilon$ & \\
            \end{tabular}
        \end{center}
    \end{sol}

    \begin{excer}
        Sea $\mathcal{L}$ un lenguaje de primer oden equipado con un símbolo de función binaria $*$. Muestre que una de las siguientes dos $\mathcal{L}$-fórmulas implica lógicamente a la otra (¿cuál implica cuál?), y que esta implicación no es reversible.
        \begin{enumerate}[label = \textit{(\alph*)}]
            \item $(\exists x)(\forall y)(x*y=y)$.
            \item $(\exists x)(x*x=x)$.
        \end{enumerate}
    \end{excer}

    \begin{sol}
        Afirmamos que la fórmula \textit{(a)} implica a la fórmula \textit{(b)}. En efecto, se tiene la siguiente demostración de \textit{(b)} a partir de \textit{(a)}:
        \begin{center}
            \setcounter{tablec}{1}
            \begin{tabular}{l r l c l r}
                & \pstable{tablec} & $(\exists x)(\forall y)(x*y$ & $=$ & $y)$ & Premisa \\
                $|\longrightarrow$ & \pstable{tablec} & $(\forall y)(z*y$ & $=$ & $y)$ & 1) Instanciación Existencial \\
                $|$ & \pstable{tablec} & $z*z$ & $=$ & $z$ & 2) Instanciación Universal \\
                \hline
                & \pstable{tablec} & $(\exists x)(x*x$ & $=$ & $x)$ & 1)-3) Generalización Existencial \\
                \hline
                & & & $\therefore$ & $(\exists x)(x*x=x)$ & \\
            \end{tabular}
        \end{center}
        pues, en en la tercera fila la variable $z$ es sustituíble por $y$, ya que $z$ no queda bajo el alcance de ningún cuanificador de la fórmula de la fila 2).

        Ahora veamos que la fórmula \textit{(b)} no implica lógicamente a la fórmula \textit{(a)}. Esto es equivalente a probar que
        \begin{equation*}
            \Sigma\nvdash (\exists x)(\forall y)(x*y=y)
        \end{equation*}
        donde $\Sigma=\left\{(\exists x)(x*x=x) \right\}$. Por el Teorema de Completud de Gödel, basta con probar que
        \begin{equation*}
            \Sigma\nvDash (\exists x)(\forall y)(x*y=y)
        \end{equation*}
        así que, tenemos que encontrar una $\mathcal{L}$-estructura $\mathfrak{A}$ y una interpretación $\iota$ tales que
        \begin{equation*}
            \mathfrak{A}\vDash\Sigma[\iota]\textup{, pero }\mathfrak{A}\nvDash (\exists x)(\forall y)(x*y=y)[\iota]
        \end{equation*}
        Considere la $\mathcal{L}$-estructura
        \begin{equation*}
            \mathfrak{A}=(\mathbb{N},\bullet)
        \end{equation*}
        donde $\mathbb{N}$ es el conjunto de los números naturales y $\bullet$ es la función de $\mathbb{N}\times\mathbb{N}$ en $\mathbb{N}$ dada por:
        \begin{equation*}
            m\bullet n=1,\quad\forall m,n\in\mathbb{N}
        \end{equation*}
        Considere además la interpretación $\cf{\iota}{\textup{Var}}{\mathbb{N}}$ dada por:
        \begin{equation*}
            \iota(v_i)=i
        \end{equation*}
        para toda variable $v_i$. Veamos que $\mathfrak{A}\vDash\Sigma[\iota]$, esto es que
        \begin{equation*}
            \mathfrak{A}\vDash(\exists x)(x*x=x)[\iota]
        \end{equation*}
        en efecto, esto se cumple si y sólo si
        \begin{equation*}
            \mathfrak{A}\vDash(x*x=x)[\iota_{n/x}]
        \end{equation*}
        para algún $n\in\mathbb{N}$. Tomando $n=1$ obtenemos que:
        \begin{equation*}
            \begin{split}
                \hat{\iota}_{1/x}(x*x)&=\hat{\iota}_{1/x}(x)\bullet\hat{\iota}_{1/x}(x)\\
                &=1\bullet 1\\
                &=1\\
            \end{split}
        \end{equation*}
        y,
        \begin{equation*}
            \begin{split}
                \hat{\iota}_{1/x}(x)&=1\\
            \end{split}
        \end{equation*}
        por tanto, $\hat{\iota}_{1/x}(x)$ es lo mismo que $\hat{\iota}_{1/x}(x*x)$, se sigue así
        \begin{equation*}
            \mathfrak{A}\vDash(x*x=x)[\iota_{n/x}]
        \end{equation*}

        Veamos ahora que
        \begin{equation*}
            \begin{split}
                \mathfrak{A}\nvDash(\exists x)(\forall y)(x*y=y)[\iota]&\textup{ si y sólo si }\mathfrak{A}\nvDash(\exists x)(\neg((\exists y)\neg(x*y= y)))[\iota]\\
                &\textup{ si y sólo si }\mathfrak{A}\nvDash\neg((\exists y)(x*y\neq y))[\iota_{n/x}],\textup{ para todo }n\in\mathbb{N} \\
                &\textup{ si y sólo si }\mathfrak{A}\vDash(\exists y)(x*y\neq y)[\iota_{n/x}],\textup{ para todo }n\in\mathbb{N} \\
                &\textup{ si y sólo si }\mathfrak{A}\vDash(x*y\neq y)[(\iota_{n/x})_{ m/y}],\textup{ para todo }n\in\mathbb{N}\textup{ y para algún }m\in\mathbb{N} \\
            \end{split}
        \end{equation*}
        en efecto, sea $n\in\mathbb{N}$ y $m=1729\in\mathbb{N}$, tenemos que:
        \begin{equation*}
            \begin{split}
                (\hat{\iota}_{ n/x})_{ m/y}(x*y)&=(\hat{\iota}_{ n/x})_{ m/y}(x)\bullet(\hat{\iota}_{ n/x})_{ m/y}(y)\\
                &=n\bullet m\\
                &=n\bullet 1729\\
                &=1\\
            \end{split}
        \end{equation*}
        y,
        \begin{equation*}
            \begin{split}
                (\hat{\iota}_{ n/x})_{ m/y}(y)&=m\\
                &=1729\\
            \end{split}
        \end{equation*}
        donde claramente 1 no es 1729, se sigue pues de las equivalencias anteriores que:
        \begin{equation*}
            \mathfrak{A}\nvDash(\exists x)(\forall y)(x*y=y)[\iota]
        \end{equation*}
        
        De esta forma, $\mathfrak{A}$ y $\iota$ son la $\mathcal{L}$-estructura e interpretación buscadas, respectivamente.
    \end{sol}

    \begin{excer}
        Sea $\mathcal{L}=\left\{\in\right\}$ el lenguaje de la teoría de conjuntos (es decir, $\in$ es simplemene el símbolo de relación binaria). Demuestre que cualquier fórmula en la cual no aparezca el símbolo de negación $\neg$ debe tener longitud impar.
    \end{excer}

    \begin{proof}
        Procederemos por inducción sobre la longitud de la $\mathcal{L}$-fórmula. Antes de ello, notemos que como nuestro lenguaje $\mathcal{L}$ no posee constantes ni funciones, los únicos posibles términos serán las variables $v_i$. 
        \begin{itemize}
            \item \textbf{Caso base}: Sea $\varphi$ una $\mathcal{L}$-fórmula de longitud menor o igual a 3 en la que no aparece el símbolo de negación $\neg$. Se tienen 2 posibilidades:
            \begin{itemize}
                \item $\varphi$ es $v_1=v_2$, donde $v_1$ y $v_2$ son variables.
                \item $\varphi$ es $v_1\in v_2$, donde $v_1$ y $v_2$ son variables. 
            \end{itemize}
            esto pues $\varphi$ en su forma más simple es atómica. En cualquiera de los tres casos, la longitud de $\varphi$ es impar (más aún, es 3).
            \item \textbf{Paso Inductivo}: Suponga que existe $n\in\mathbb{N}$ tal que para toda $\mathcal{L}$-fórmula $\varphi$ de longitud menor o igual a $2n+1$ en la que no aparezca el símbolo de negación $\neg$, implica que la longitud de $\varphi$ es impar.
            
            Sea $\varphi$ una fórmula de longitud menor o igual a $2n+3$, se tienen tres posibles casos para $\varphi$:
            \begin{itemize}
                \item $\varphi$ tiene longitud menor o igual a $2n+1$, por lo que de la hipótesis de inducción se sigue que $\varphi$ tiene longitud impar.
                \item $\varphi$ es $\psi\Rightarrow\phi$, donde $\psi$ y $\phi$ son $\mathcal{L}$-fórmulas en las que no aparece el símbolo negación $\neg$, pues éste no aparece en $\varphi$. En particular, al tenerse que $\varphi$ es de longitud menor o igual a $2n+3$, se sigue que $\psi$ y $\phi$ tienen longitud menor o igual a $2n+1$. Por hipótesis de inducción se sigue que ambas tienen longitud impar, digamos:
                \begin{equation*}
                    2n_\psi+1\quad\textup{y}\quad 2n_\phi+1
                \end{equation*}
                respectivamente, con $n_\psi,n_\phi\in\mathbb{N}$. Por ende, la longitud de $\varphi$ es
                \begin{equation*}
                    (2n_\psi+1)+1+(2n_\phi+1)=2(n_\psi+n_\phi+1)+1
                \end{equation*}
                es decir, la longitud de $\varphi$ es impar.
                \item $\varphi$ es $(\exists x)\psi$, donde al tenerse que $\varphi$ no tiene al símbolo negación $\neg$, tampoco lo puede tener $\psi$, así que como $\varphi$ es de longitud menor o igual a $2n+3$, debe tenerse que la longitud de $\psi$ es menor o igual a $2n+1$. Por hipotesis de inducción $\psi$ es de longitud impar, digamos $2n_\psi+1$ con $n_\psi\in\mathbb{N}$, luego la longitud de $\varphi$ será:
                \begin{equation*}
                    (2n_\psi+1)+2=2(n_\psi+1)+1
                \end{equation*}
                es decir, la longitud de $\varphi$ es impar.
            \end{itemize}
            En cualquier caso, se sigue que la longitud de $\varphi$ es impar.
        \end{itemize}
        Aplicando inducción, se sigue que para todo $n\in\mathbb{N}$ y para toda $\mathcal{L}$-fórmula $\varphi$ en la que no aparezca $\neg$ que tenga longitud menor o igual a $2n+1$, implica que $\varphi$ tiene longitud impar.
        
        En particular, si $\varphi$ es una $\mathcal{L}$-fórmula en la que no aparece $\neg$, existe un número $m\in\mathbb{N}$ tal que la longitud de $\varphi$ es menor o igual a $2m+1$ (pues toda fórmula en este curso tiene longitud finita) luego, por lo anterior se sigue que $\varphi$ tiene longitud impar, lo cual prueba el resultado.
    \end{proof}
    
\end{document}