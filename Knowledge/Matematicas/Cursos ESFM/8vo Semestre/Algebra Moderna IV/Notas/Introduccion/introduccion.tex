\documentclass[12pt]{report}
\usepackage[spanish]{babel}
\usepackage[utf8]{inputenc}
\usepackage{amsmath}
\usepackage{amssymb}
\usepackage{amsthm}
\usepackage{graphics}
\usepackage{subfigure}
\usepackage{lipsum}
\usepackage{array}
\usepackage{multicol}
\usepackage{enumerate}
\usepackage[framemethod=TikZ]{mdframed}
\usepackage[a4paper, margin = 1.5cm]{geometry}

%En esta parte se hacen redefiniciones de algunos comandos para que resulte agradable el verlos%

\renewcommand{\theenumii}{\roman{enumii}}

\def\proof{\paragraph{Demostración:\\}}
\def\endproof{\hfill$\blacksquare$}

\def\sol{\paragraph{Solución:\\}}
\def\endsol{\hfill$\square$}

%En esta parte se definen los comandos a usar dentro del documento para enlistar%

\newtheoremstyle{largebreak}
  {}% use the default space above
  {}% use the default space below
  {\normalfont}% body font
  {}% indent (0pt)
  {\bfseries}% header font
  {}% punctuation
  {\newline}% break after header
  {}% header spec

\theoremstyle{largebreak}

\newmdtheoremenv[
    leftmargin=0em,
    rightmargin=0em,
    innertopmargin=0pt,
    innerbottommargin=5pt,
    hidealllines = true,
    roundcorner = 5pt,
    backgroundcolor = gray!60!red!30
]{exa}{Ejemplo}[section]

\newmdtheoremenv[
    leftmargin=0em,
    rightmargin=0em,
    innertopmargin=0pt,
    innerbottommargin=5pt,
    hidealllines = true,
    roundcorner = 5pt,
    backgroundcolor = gray!50!blue!30
]{obs}{Observación}[section]

\newmdtheoremenv[
    leftmargin=0em,
    rightmargin=0em,
    innertopmargin=0pt,
    innerbottommargin=5pt,
    rightline = false,
    leftline = false
]{theor}{Teorema}[section]

\newmdtheoremenv[
    leftmargin=0em,
    rightmargin=0em,
    innertopmargin=0pt,
    innerbottommargin=5pt,
    rightline = false,
    leftline = false
]{propo}{Proposición}[section]

\newmdtheoremenv[
    leftmargin=0em,
    rightmargin=0em,
    innertopmargin=0pt,
    innerbottommargin=5pt,
    rightline = false,
    leftline = false
]{cor}{Corolario}[section]

\newmdtheoremenv[
    leftmargin=0em,
    rightmargin=0em,
    innertopmargin=0pt,
    innerbottommargin=5pt,
    rightline = false,
    leftline = false
]{lema}{Lema}[section]

\newmdtheoremenv[
    leftmargin=0em,
    rightmargin=0em,
    innertopmargin=0pt,
    innerbottommargin=5pt,
    roundcorner=5pt,
    backgroundcolor = gray!30,
    hidealllines = true
]{mydef}{Definición}[section]

\newmdtheoremenv[
    leftmargin=0em,
    rightmargin=0em,
    innertopmargin=0pt,
    innerbottommargin=5pt,
    roundcorner=5pt
]{excer}{Ejercicio}[section]

%En esta parte se colocan comandos que definen la forma en la que se van a escribir ciertas funciones%

\newcommand\abs[1]{\ensuremath{\big|#1\big|}}
\newcommand\divides{\ensuremath{\bigm|}}
\newcommand\cf[3]{\ensuremath{#1:#2\rightarrow#3}}
\newcommand\contradiction{\ensuremath{\#_c}}
\newcommand{\Obj}[1]{\ensuremath{\textup{Obj}\left(#1\right)}}
\newcommand{\Hom}[3]{\ensuremath{\textup{Hom}_{#1}\left(#2,#3\right)}}

\newcommand{\Cat}[1]{\ensuremath{\textup{\textbf{#1}}}}
\newcommand{\Iso}[3]{\ensuremath{\textup{I}_{\textup{SO}_{#1}}\left(#2,#3\right)}}
\newcommand{\SO}[1]{\ensuremath{\textup{SO}\left(#1\right)}}

%recuerda usar \clearpage para hacer un salto de página

\begin{document}
    \setlength{\parskip}{5pt} % Añade 5 puntos de espacio entre párrafos
    \setlength{\parindent}{12pt} % Pone la sangría como me gusta
    \title{Notas de Álgebra Moderna IV.
    
    Una introducción a la teoría de categorías.}
    \author{Cristo Daniel Alvarado}
    \maketitle

    \tableofcontents %Con este comando se genera el índice general del libro%

    %\setcounter{chapter}{3} %En esta parte lo que se hace es cambiar la enumeración del capítulo%

    \chapter{Clases y conjuntos}

    \section{Axiomas de Von-Newmann-Gödel}

    Antes de decantarnos totalmente a nuestro estudio de las categorías, primero nos enfocaremos en estudiar a los objetos que se van a usar (las clases).

    Aceptamos la existencia de \textit{objetos primitivos}, las cuales son clases y conjuntos, dotadas de dos relaciones primitivas, la pertenencia $\in$ e igualdad $=$. Denotamos a los objetos primitivos por letras en mayúsculas.

    \begin{mydef}[\textbf{Axiomas de NBG}]
        \renewcommand{\theenumi}{A\arabic{enumi}}
        Se tienen los siguientes axiomas:
        \begin{enumerate}
            \item Todo conjunto es una clase.
            \item Si $x\in A$, $\forall x\in B$ y $x\in B$, $\forall x\in A$, entonces $A=B$.
            \item Si $A\in B$ donde $B$ es una clase, entonces $A$ es un conjunto.
            \item Si $P(x)$ es una propiedad definida sobre el parámetro $x$ que se recorre sobre conjuntos, entonces existe una clase $[x\big| P(x)]$ tal que para cada conjunto $y$.
            \begin{equation*}
                \begin{split}
                    y\in[x\big| P(x)]\iff P(y)
                \end{split}
            \end{equation*}
            \item Si $X,Y$ son conjuntos, entonces $[X,Y]$ es un conjunto y se denota por $\left\{X, Y\right\}$ (ver ejemplo 1.1.3).
            \item Si $X$ es un conjunto, entonces $\left\{X\right\}$, $\left\{X,\left\{X\right\} \right\}$,... son conjuntos.
            \item Existe un conjunto inductivo.
            \item Sea $A$ conjunto, entonces existe un conjunto denotado por $\mathcal{P}(A)$ tal que $B\in\mathcal{P}(A)$ si y sólo si $B\subseteq A$.
            \item Si $\cf{f}{A}{B}$ donde $A$ es un conjunto, entonces $f(A)$ es un conjunto.
        \end{enumerate}
    \end{mydef}

    \begin{exa}
        Construimos al \textbf{conjunto vacío} $\emptyset$ como $\emptyset=[x\big| x\neq x]$ (usando a A4).
    \end{exa}
    
    \begin{exa}
        $\textup{Set}=[x| x=x]$ (usando a A4).
    \end{exa}

    \begin{exa}
        Si $X$ y $Y$ son conjuntos, entonces
        \begin{equation*}
            [X,Y]=[Z\big| Z=X\textup{ o }Z=Y]
        \end{equation*}
        (construida por el A4).
    \end{exa}

    \begin{exa}
        Si $X$ es un conjunto, entonces $X\cup\left\{X\right\}$ es un conjunto y se denomina el \textbf{sucesor de $X$}.
    \end{exa}

    \begin{mydef}
        Sea $A$ una clase. Se define
        \begin{equation*}
            \bigcup A=\bigcup_{X\in A}X=[x\big| \exists X\in A\textup{ tal que }x\in X]
        \end{equation*}
        Si $A$ es un conjunto, $\bigcup A$ es un conjunto.
    \end{mydef}

    \begin{mydef}
        Un conjunto $A$ se denomina \textbf{inductivo} si
        \renewcommand{\theenumi}{\roman{enumi}}
        \begin{enumerate}
            \item $\emptyset\in A$.
            \item $X\in A\Rightarrow X\cup\left\{X\right\}\in A$.
        \end{enumerate}
    \end{mydef}

    \begin{propo}
        $\emptyset$ es un conjunto.
    \end{propo}

    \begin{proof}
        Sea $A$ un conjunto inductivo (el cual existe por A7), entonces $\emptyset\in A$, luego por A3, $\emptyset$ es un conjunto.
    \end{proof}

    \begin{mydef}
        Se dice que $B$ es subclase de $A$, si $x\in A$ para todo $x\in B$, y se denota por $B\subseteq A$.
    \end{mydef}

    \begin{propo}
        Si $B\subseteq A$ y $A$ es conjunto, entonces $B$ es conjunto.
    \end{propo}

    \begin{proof}
        Como $B\subseteq A$, entonces $B\in\mathcal{P}(A)$, luego $B$ por A3, $B$ es conjunto.
    \end{proof}

    Esta proposición es necesaria pues no sabemos si las subclases de conjuntos son conjuntos.

    \begin{mydef}
        Si $x,y$ son conjuntos, se define:
        \begin{equation*}
            (x,y)=\left\{\left\{x\right\},\left\{x,y\right\} \right\}
        \end{equation*}
        Si $A$ y $B$ son clases, se define
        \begin{equation*}
            A\times B = [(x,y)\big| x\in A\textup{ y }y\in B ]
        \end{equation*}
    \end{mydef}

    \begin{excer}
        Si $A$ y $B$ son conjuntos, entonces $A\times B$ es conjunto.
    \end{excer}

    \begin{proof}
        
    \end{proof}

    \begin{mydef}
        Una función de $A$ en $B$ es una subclase $F\subseteq A\times B$ tal que $(x,y),(x,z)\in F\Rightarrow y=z$.
    \end{mydef}

    \begin{exa}
        $\textup{Set}$ no es un conjunto.
    \end{exa}

    \begin{proof}
        Supóngase que Set es un conjunto. Sea
        \begin{equation*}
            X=[x\big| x\notin x]
        \end{equation*}
        Si $x\in X$, entonces $x$ es un conjunto (por A3) luego $x\in\textup{Set}$, es decir que $x$ es un conjunto. Por tanto, $X\subseteq\textup{Set}$, esto es que $X$ es un conjunto. Luego sucede que $X\in X$ o $X\notin X$ (por como se formó la clase $X$ a partir de A4).
        
        Por ende, $X\in X\iff X\notin X$\contradiction. Luego $\textup{Set}$ no es un conjunto.
    \end{proof}

    \begin{exa}
       Denotamos por $\mathcal{G}=[G| G\textup{ es grupo}]$, y $\mathcal{S}=[S_X\big| X\in\textup{Set}]$. Si sucediera que $S$ fuese conjunto, tomando $\cf{f}{\mathcal{S}}{\textup{Set}}$, $S_X\mapsto X$ es una función, luego $F(\mathcal{S})=\textup{Set}$ es un conjunto, lo cual no puede ser. Por tanto, como $\mathcal{S}\subseteq\mathcal{G}$, se sigue que $\mathcal{G}$ es clase.
    \end{exa}

    \chapter{Categorias}

    \section{Conceptos Fundamentales}

    Antes de comenzar aceptaremos como válido al siguiente axioma:

    \renewcommand{\theenumi}{A\arabic{enumi}}

    \begin{enumerate}
        \setcounter{enumi}{9}
        \item \textbf{Limitación de tamaño}. Una clase es un conjunto si y sólo si no es biyectivo con $\textup{Set}$.
    \end{enumerate}

    Ahora si con la parte de categorías.
    
    \renewcommand{\theenumi}{\arabic{enumi}}

    \begin{mydef}
        Una \textbf{categoría} $\mathcal{C}$ consta de lo siguiente:
        \begin{enumerate}
            \item Una clase $\textup{Obj}(\mathcal{C})$ cuyos elementos son llamados \textbf{objetos}.
            \item Para cada par $A,B\in\Obj{\mathcal{C}}$ existe un conjunto $\Hom{\mathcal{C}}{A}{B}$ cuyos elementos llamaremos morfismos y, dado un morfismo $f\in\Hom{\mathcal{C}}{A}{B}$ lo denotaremos por $\cf{f}{A}{B}$.
            \item Para cada objeto $A\in\Obj{\mathcal{C}}$ hay un morfismo $1_A\in\Hom{\mathcal{C}}{A}{A}$ llamado la \textbf{identidad de $A$}.
            \item Hay una ley de composición para una terna de objetos $A$, $B$ y $C$:
            \begin{equation*}
                \begin{split}
                    \Hom{\mathcal{C}}{A}{B}\times\Hom{\mathcal{C}}{B}{C}\rightarrow&\Hom{\mathcal{C}}{A}{C}\\
                    (f,g)\mapsto&g\circ f\\
                \end{split}
            \end{equation*}
            que satisface lo siguiente:
            \begin{enumerate}
                \item (\textit{Asociatividad}). Dado $f\in\Hom{\mathcal{C}}{A}{B}$ y $g\in\Hom{\mathcal{C}}{B}{C}$ y $h\in\Hom{\mathcal{C}}{C}{D}$ se cumple que:
                \begin{equation*}
                    h\circ (g\circ f)=(h\circ g)\circ f
                \end{equation*}
                \item Dado un morfismo $f\in\Hom{\mathcal{C}}{A}{B}$, se tiene que:
                \begin{equation*}
                    f\circ 1_A=f=1_B\circ f
                \end{equation*}
            \end{enumerate}
        \end{enumerate}
    \end{mydef}

    \begin{mydef}
        Si la clase de objetos de la categoría $\mathcal{C}$ es un conjunto, diremos que $\mathcal{C}$ es una \textbf{categoría pequeña}. 
        
        Más aún, si tenemos un número finito de morfismos (hablando de todos los que puede haber en la categoría y entre todos los objetos de la categoría), diremos que $\mathcal{C}$ es una \textbf{categoría finita}.
    \end{mydef}

    Dadas las definciones anteriores, no se nos da ejemplos concretos de lo que es una categoría, por lo cual procederemos a dar ejemplos de la misma.

    \begin{obs}
        Denotamos por $\mathbb{N}_0=\mathbb{N}^*$ a $\mathbb{N}\cup\left\{0\right\}$.
    \end{obs}

    \begin{exa}
        Sea $X$ un conjunto. Denotamos por $\mathcal{C}_X$ a una categoría formada por $\Obj{\mathcal{C}_X}=X$, y tendremos para cualquier par de elementos $x,y\in\Obj{\mathcal{C}_X}$ definimos:
        \begin{equation*}
            \Hom{\mathcal{C}_X}{x}{y}=\left\{
                \begin{array}{lcr}
                    \emptyset & \textup{ si } & x\neq y\\
                    \left\{1_x\right\} & \textup{ si } & x=y\\
                \end{array}
            \right.
        \end{equation*}
        Definimos la ley de composición de la siguiente forma:
        \begin{equation*}
            1_x\circ 1_x=1_x,\quad\forall x\in X
        \end{equation*}
        Observemos que el único caso en el que está definido es cuando los tres objetos de la categoría son el mismo, en caso contrario el conjunto de morfismos es vacío. Estos objetos que tomamos aquí hacemos $1_x=\left\{x\right\}$ para todo $x\in X$. Este ejemplo da una razón para decir que es lo que son los morfismos y objetos de la categoría.

        $\mathcal{C}_X$ es una categoría pequeña la cual no necesariamente es finita (es finita en el caso que la cardinalidad de $X$ sea finita).

        Si $X=\emptyset$ entonces $\mathcal{C}_X$ es la \textbf{categoría vacía} (que coincide cuando $n=0$ en la siguiente parte).
        
        Denotamos $\Cat{n}$ a la categoría como la de este ejemplo del conjunto con $n$ elementos, donde $n\in\mathbb{N}$.
    \end{exa}
    
    \begin{mydef}
        Un conjunto no vacío $X$ es preordenado si...
    \end{mydef}

    \begin{exa}
        Consideremos al conjunto preordenado $(X,\leq)$ (llamado a veces \textbf{PoSet}). Definimos la categoría $\mathcal{C}_{(X,\leq)}$ en donde:
        \begin{itemize}
            \item $\Obj{\mathcal{C}_{(X,\leq)}}=X$.
            \item Se define el conjunto de morfismos por:
            \begin{equation*}
                \Hom{\mathcal{C}_{(X,\leq)}}{x}{y}=\left\{
                    \begin{array}{lcr}
                        \emptyset & \textup{ si } & x\nleq y\\
                        \left\{\varphi_{x,y} \right\} & \textup{ si } & x\leq y\\
                    \end{array}
                \right.
            \end{equation*}
            \item Sean $x,y,z\in \Obj{\mathcal{C}_{(X,\leq)}}$. Se define la Ley de Composición de la siguiente manera:
            \begin{equation*}
                \begin{split}
                    \Hom{\mathcal{C}_{(X,\leq)}}{x}{y}\times\Hom{\mathcal{C}_{(X,\leq)}}{y}{z}\rightarrow&\Hom{\mathcal{C}_{(X,\leq)}}{x}{z}\\
                    (\varphi_{x,y},\varphi_{y,z})\mapsto&\varphi_{x,z}\\
                \end{split}
            \end{equation*}
            La Ley de composición anterior está definida solamente cuando $x\leq y\leq z$, pues es el único caso en el que el conjunto de morfismos es no vacío.
        \end{itemize}

        El morfismo es un objeto que cumple las leyes anteriores.
    \end{exa}
    
    \begin{exa}
        Construimos la categoría $\Cat{Set}$, donde la clase objetos es la clase de todos los conjuntos y los morfismos son las funciones entre cada conjunto. La ley de composición es la composición usual de funciones.
    \end{exa}

    \begin{exa}
        Sea $(M,\cdot)$ un monoide, es decir, $M$ es un conjunto no vacío en el que $\cdot$ es una operación binaria que es asociativa, para la cual existe un elemento $e_M\in M$ tal que:
        \begin{equation*}
            e_M\cdot x=x\cdot e_M=x\quad\forall x\in M
        \end{equation*}
        Denotamos por $\mathcal{M}$ a la categoría en donde:
        \begin{itemize}
            \item $\Obj{\mathcal{M}}=\left\{\cdot \right\}$.
            \item $\Hom{\mathcal{M}}{\cdot}{\cdot}=M$.
            \item La ley de composición $\circ$, se define de la siguiente forma:
            \begin{equation*}
                x\circ y=x\cdot y
            \end{equation*}
            donde $x,y\in \Hom{\mathcal{M}}{\cdot}{\cdot}$ (en este caso $\cdot$ son el dominio y codominio de $x,y\in X$).
        \end{itemize}
        En este caso, el morfismo identidad es la identidad del monoide.

        Este ejemplo se puede extender a grupos, y en el caso de grupos abelianos se tendría que la ley de composición es conmutativa.
    \end{exa}

    \begin{exa}
        La categoría $\Cat{Grp}$ es la categoría de todos los grupos, donde la clase de objetos de la categoría es la clase de todos los grupos y los morfismos son los homomorfismos de grupos.

        De forma similar $\Cat{Grp}=\Cat{AbGrp}$ y $\Cat{SiGrp}$ son las categorías de grupos abelianos y simples.
    \end{exa}

    \begin{exa}
        Sea $R$ un anillo con identidad, denotamos por $R_\mathcal{M}$ a la categoría
    \end{exa}

    \begin{exa}
        La categoría $\Cat{Top}$ es la categoría de todos los espacios topológicos con funciones continuas entre los espacios como los morfismos.

        De forma análoga con $\Cat{HausTop}$, que es la categoría de todos los espacios topológicos que en particular son Hausdorff y los morfismos son las funciones continuas entre los espacios.
    \end{exa}

    \begin{mydef}
        Sean $\mathcal{C}$ y $\mathcal{C}'$ dos categorías. Si
        \begin{enumerate}
            \item $\Obj{\mathcal{C}'}\subseteq\Obj{\mathcal{C}}$.
            \item $\Hom{\mathcal{C}}{A}{B}\subseteq\Hom{\mathcal{C}'}{A}{B}$ para todo $A,B\in\Obj{\mathcal{C}'}$.
            \item La ley de composición de morfismos en $\mathcal{C}'$ está inducida por la ley de composición de morfismos en $\mathcal{C}$.
            \item Los morfismos identidad en $\mathcal{C}'$ son los morfismos identidad en $\mathcal{C}$.
        \end{enumerate}
        se dice entonces que $\mathcal{C}'$ es una \textbf{subcategoría} de $\mathcal{C}$.

        Más aún, $\mathcal{C}'$ se dice que es una \textbf{subcategoría llena} de $\mathcal{C}$ si para cada par $(A,B)$ de objetos de $\mathcal{C}'$ tenemos que
        \begin{equation*}
            \Hom{\mathcal{C}'}{A}{B}=\Hom{\mathcal{C}}{A}{B}
        \end{equation*}
    \end{mydef}

    \begin{exa}
        \begin{enumerate}
            Se cumple lo siguiente:
            \item Las categorías $\Cat{Ab}$ y $\Cat{SiGrp}$ son subcategorías llenas de $\Cat{Grp}$.
            \item La categoría $\Cat{HausTop}$ es una subcategoría llena de $\Cat{Top}$.
            \item $\Cat{Ring}^c$ y $\Cat{Field}$ son subcategorísa de $\Cat{Ring}$, más aún son subcategorías llenas.
        \end{enumerate}
    \end{exa}

    \section{Objetos especiales y Morfismos en una Categoría}

    Las nociones de monomirfismo y epimorfismo que se van a introducir son generalizaciones a categorías arbitrarias de las funciones familiares inyectiva y suprayectiva que van de $\Cat{Set}$.

    \begin{mydef}
        Sea $\mathcal{C}$ una categoría y $f\in\Hom{\mathcal{C}}{A}{B}$.
        \begin{enumerate}
            \item $f$ es llamado \textbf{monomorfismo} si para cualesquiera $g_1,g_2\in\Hom{\mathcal{C}}{C}{A}$ (siendo $C$ un objeto de la categoría) tales que $f\circ g_1=f\circ g_2$ tenemos que $g_1=g_2$.
            \item $g$ es llamado \textbf{epimorfismo} si para cualesquiera $h_1,h_2\in\Hom{\mathcal{C}}{B}{C}$ (siendo $C$ un objeto de la categoría) tales que $h_1\circ f=h_2\circ f$ tenemos que $h_1=h_2$.
            \item $f$ es llamado \textbf{isomorfismo} si existe $f'\in\Hom{\mathcal{C}}{B}{A}$ tal que $f\circ f'=1_B$ y $f'\circ f=1_A$. En este caso decimos que $A$ y $B$ son \textbf{objetos isomorfos}.
        \end{enumerate}
    \end{mydef}

    A pesar de que en la categoría $\Cat{Set}$ los monomorfismos (respectivamente, epimorfismos) coinciden con funciones inyectivas (respectivamente, suprayectivas), esto no es cierto en cualquier categoría arbitraria cuyos objetos y morfismos pueden ser conjuntos y funciones, respectivamente. Esto se verá más a fondo en los siguientes ejemplos.

    \begin{exa}
        En cada una de las categorías $\Cat{Set},\Cat{Grp},\Cat{Ab},_R\mathcal{M}$ los monomorfismos coinciden con los homomorfismos inyectivos, mientras que en $\Cat{Top}$ y $\Cat{KHaus}$ los monomorfismos coinciden con los mapeos continuos inyectivos. Solo se probará que los monomorfismos en la categoría $ \Cat{Set}$ coinciden con las funciones inyectivas.
    \end{exa}

    \begin{proof}
        Sea $f\in\Hom{\Cat{Set}}{A}{B}$ es morfismo.

        Suponga que $f$ es monomorfismo, decir que $\cf{f}{A}{B}$ es una función que cumple el primer inciso de la definición anterior. Hay que probar que $f$ es inyectiva. Sean $a\in A$ y $a'\in A$ elementos tales que $f(a)=f(a')$. Consideremos a $g_1$ y $g_2$ los morfismos tales que $g_1(x)=a$ y $g_2(x)=a'$, estos pertenecen a $\Hom{\Cat{Set}}{\left\{x\right\}}{A}$ (en este contexto $x$ es un elemento de un conjunto arbitrario). Se tiene entonces que
        \begin{equation*}
            \begin{split}
                f\circ g_1(x)=&f(a) \\
                =&f(a') \\
                =&f\circ g_2(x) \\
            \end{split}
        \end{equation*}
        por tanto, $f\circ g_1=f\circ g_2$ (ya que coinciden en $x$). Por tanto, al suponer que fue monomorfismo, se sigue que $g_1=g_2$, es decir que $a=a'$. Luego $f$ es inyectiva.

        Suponga ahora que $f$ es inyectiva. Sean $g_1,g_2\in\Hom{\Cat{C}}{C}{A}$ tales que $f\circ g_1 = f\circ g_2$. Si $c\in C$, tenemos que:
        \begin{equation*}
            \begin{split}
                f\circ g_1(c)=&f\circ g_2(c)\\
                f(g_1(c))=&f(g_2(c))\\
                \Rightarrow g_1(c)=&g_2(c)\\
            \end{split}
        \end{equation*}
        pues $f$ es inyectiva. Como el $c\in C$ fue arbitrario, se sigue que $g_1=g_2$, por ende, $f$ es monomorfismo.

    \end{proof}

    De forma idéntica se prueba lo anterior para las categorías $\Cat{Top}$ y $\Cat{KHaus}$ (pues los morfismos en estas son simplemente las funciones continuas).

    \begin{exa}
        
    \end{exa}

    \begin{proof}
        
    \end{proof}

    \begin{exa}
        En cada una de las categorías $\Cat{Set}$, $\Cat{Grp}$, $\Cat{Ab}$ y $_R\mathcal{M}$, los isomorfismos coinciden con los homomorfismos biyectivos (por los ejemplos anteriores). En $\Cat{Top}$, los isomorfismos son exactamente los \textit{homeomorfismos}, esto es, biyecciones continuas cuyas inversas también son continuas.
    \end{exa}

    \begin{proof}
        Sea $\cf{f}{G}{G'}$ un homomorfismo biyectivo de grupos y considere a $\cf{f'}{G'}{G}$ su mapeo inverso, es decir que:
        \begin{equation*}
            f\circ f'=1_{G'}\quad\textup{y}\quad f'\circ f=1_G
        \end{equation*}
        veamos que en efecto, $f'$ es homomorfismo biyectivo (el cual ya es biyectivo y por tanto, será homomorfismo biyectivo). Sean $a',b'\in G'$, se tiene que:
        \begin{equation*}
            \begin{split}
                f'(a\cdot b)=&f'(f(a)\cdot f(b))\\
                =&f'(f(a\cdot b))\\
                =&a\cdot b\\
            \end{split}
        \end{equation*}
        luego, $f'$ es homomorfismo que, al ser biyectivo es homomorfismo biyectivo. 
    \end{proof}

    De forma natural, uno puede hacer una analogía de este tipo de objetos con las funciones inyectivas, suprayectivas y biyectivas, respectivamente. Pero surge la cuestión: ¿Todo isomorfismo es monomorfismo y epimorfismo? y viceversa? Una repuesta parcial la da la siguiente proposición:

    \begin{propo}
        Sea $\mathcal{C}$ una categoría y $A,B\in\Obj{\mathcal{C}}$ objetos isomorfos, entonces los isomorfismos $f\in\Hom{\mathcal{C}}{A}{B}$ y $f'\in\Hom{\mathcal{C}}{B}{A}$ tales que:
        \begin{equation*}
            f\circ f'=1_B\quad\textup{y}\quad f'\circ f=1_A
        \end{equation*}
        son monomorfismos y epimorfismos.
    \end{propo}
    
    \begin{proof}
        Basta probar el resultado para $f$, pues por simetría se tiene de forma inmediata que, al cumplirse para $f$ también se cumple para $f'$.
        \begin{enumerate}
            \item Sean $g_1,g_2\in\Hom{\mathcal{C}}{C}{A}$ tales que $f\circ g_1=f\circ g_2$. Entonces:
            \begin{equation*}
                \begin{split}
                    f'\circ (f\circ g_1)&=f'\circ(f\circ g_2)\\
                    \Rightarrow(f'\circ f)\circ g_1&=(f'\circ f)\circ g_2\\
                    \Rightarrow 1_A\circ g_1&=1_A\circ g_2\\
                    \Rightarrow g_1&= g_2\\
                \end{split}
            \end{equation*}
            pues, la ley de composición es asociativa, luego por la hipótesis y finalmente por ser el morfismo identidad.
            \item Es análogo a la parte anterior.
        \end{enumerate}
        Por ambos incisos, se sigue que $f$ es monomorfismo y epimorfismo.

    \end{proof}

    Surgen muchas dudas, todo morfismo que es monomorfismo y epimorfismo, es isomorfismo? Esa duda se responderá más adelante, ahora veamos un resultado que puede ser de utilidad a la hora de determinar si en un tipo particular de categoría, un morfismo es un monomorfismo, epimorfismo o isomorfismo.

    \begin{propo}
        Sea $\mathcal{C}$ una categoría donde las morfismos son (en particular) funciones que van de un conjunto a otro y la ley de composición coincide con la composición usual de funciones. Entonces se cumple lo siguiente:
        \begin{enumerate}
            \item Las funciones inyectivas son monomorfismos.
            \item Las funciones suprayectivas son epimorfismos.
            \item Las funciones biyectivas $f$ para las cuales se cumple que $f\in \Hom{\mathcal{C}}{A}{B}$ si y sólo si $f^{-1}\in\Hom{\mathcal{C}}{B}{A}$, son isomorfismos. 
        \end{enumerate}
    \end{propo}

    \begin{proof}
        De (1): Sean $A,B\in\Obj{\mathcal{C}}$ y $f\in\Hom{\mathcal{C}}{A}{B}$ una función inyectiva, entonces se cumple que:
        \begin{equation*}
            f(a)=f(a')\Rightarrow a=a'
        \end{equation*}
        para todo $a\in D(f)$ (aclarar porqué no necesariamente el dominio de $f$ es $A$, ver ejemplo 2.1.1 de la categoría creada a partir de un conjunto). Entonces, si $g_1,g_2\in \Hom{\mathcal{C}}{C}{A}$, se tiene que:
        \begin{equation*}
            f\circ g_1(c)=f\circ g_2(c)\Rightarrow g_1(c)=g_2(c)
        \end{equation*}
        para todo $c\in D(g_1)=D(g_2)$ (lo anterior debe suceder para que esté bien definida la ley de composición de morfismos en la categoría). Es decir, que $g_1=g_2$.

        De (2): Ejercicio.

        De (3): Ejercicio.
    \end{proof}

    Más adelante se verá que la recíproca de la proposición anterior no es cierta en general. Antes de llegar a ello, posiblemente surgió la duda: en una categoría donde los morfismos son funciones, ¿todo monomorfismo (respectivamente, epimorfismo e isomorfismo) es inyectivo (respectivamente, suprayectivo e inyectivo)? La respuesta a esta pregunta es que no, analizaremos 3 ejemplos donde esto no es cierto. 

    \begin{mydef}
        Sea $(G,+)$ un grupo abeliano. Decimos que $G$ es un \textbf{grupo divisible} si para cualesquiera $n\in\mathbb{N}$ y $g\in G$ existe $h\in H$ tal que $nh=g$.
    \end{mydef}

    \begin{exa}
        $\langle 0\rangle$, $(\mathbb{Q},+)$ y $(\mathbb{R},+)$ son grupos divisibles.
    \end{exa}

    No digas el siguiente ejemplo:

    \begin{exa}
        Sea $p\in\mathbb{N}$ primo y considere el siguiente subconjunto del grupo cociente $\mathbb{Q}/\mathbb{Z}$:
        \begin{equation*}
            Z(p^\infty)=\left\{\left[\frac{a}{b} \right]\in\mathbb{Q}/\mathbb{Z} \Big|a,b\in\mathbb{Z},b=p^i\textup{ para algún }i\geq0 \right\}
        \end{equation*}
        entonces, $(Z(p^\infty),+)$ es grupo divisible.
    \end{exa}
    
    \begin{exa}
        En la categoría $\Cat{Div}$ de grupos divisibles, el mapeo cociente $\cf{q}{\mathbb{Q}}{\mathbb{Q}/\mathbb{Z}}$ entre los grupos $(\mathbb{Q},+)$ y $(\mathbb{Q}/\mathbb{Z},+)$ no es inyectivo pero es un monomorfismo.
    \end{exa}

    \begin{sol}
        Recordemos que $(\mathbb{Q},+)$ es grupo con la operación de suma de elementos y $(\mathbb{Q}/\mathbb{Z},+)$ es el grupo cuyos elementos están dados de la siguiente forma; si $x\in\mathbb{Q}$, entonces $[x]\in\mathbb{Q}/\mathbb{Z}$ es el conjunto formado por:
        \begin{equation*}
            [x]=\left\{y\in\mathbb{Q}\big|y-x\in\mathbb{Z}  \right\}
        \end{equation*}
        Es claro que el mapeo cociente no es inyectivo, pues $q(0)=q(1)$ y $0\neq 1$. Pero sí es un monomorfismo. En efecto, sea $G$ otro grupo divisible y sean $f.g\in\Hom{\Cat{Div}}{G}{\mathbb{Q}}$ tales que $q\circ f = q\circ g$ (la composición de morfismos en esta categoría coincide con la composición usual de funciones), debemos probar que $f=g$.

        Como $q\circ f = q\circ g$ entonces, $q\circ(f-g)=[0]$ donde $[0]$ denota al morfismo que a cada elemento de $G$ lo envía a la clase del $0$, $[0]$. Denotemos por $h=f-g$, se sigue entonces que $q\circ h=[0]$. Entonces, para cualquier $x\in G$ tenemos que $q(h(x))=[0]$, por tanto $h(x)\in\mathbb{Z}$.
        
        Para probar el resultado basta con ver que $h(x)=0$ para todo $x\in G$. Procederemos por contradicción. Suponga que existe $x_0\in G$ tal que $h(x_0)\neq 0$, podemos asumir que $h(x_0)\in\mathbb{Z}^+$ (en caso de que no sea así, tomamos $-x_0\in G$ para el cual se cumple que $h(-x_0)=-h(x_0)\in\mathbb{Z}^+$). Como estamos trabajando con grupos divisibles, para $2h(x_0) \in\mathbb{N}$ y $x_0\in G$ existe $y_0\in G$ tal que
        \begin{equation*}
            \begin{split}
                x_0=&2h(x_0)y_0\\
                h(x_0)=&2h(x_0)h(y_0)\\
            \end{split}
        \end{equation*}
        como $h(x_0)\neq 0$, entonces $h(y_0)\neq 0$, luego lo anterior no puede suceder ya que debería suceder que $h(y_0)=\frac{1}{2}$, pero $h(y_0)\in\mathbb{Z}$. Luego tal $x_0$ no puede existir. Así, $h(x)=0$ para todo $x\in G$, lo cual prueba el resultado. Esto prueba que, de hecho, $q$ es un monomorfismo que no es inyectivo.
    \end{sol}

    \begin{exa}
        En la categoría $\Cat{Ring}^c$ de los anillos conmutativos con identidad, el mapeo inclusión $\cf{i}{\mathbb{Z}}{\mathbb{Q}}$ no es suprayectivo, pero sí es un epimorfismo.
    \end{exa}

    \begin{sol}
        En efecto, el mapeo inclusión no es suprayectivo pero es un epimorfismo. En efecto, sea $R$ otro anillo conmutativo con identidad, y sean $\cf{f,g}{\mathbb{Q}}{R}$ dos morfismos de anillos (que en este caso, coinciden con los homomorfismos de anillos) tales que
        \begin{equation*}
            f\circ i=g\circ i
        \end{equation*}
        Hay que ver que $f=g$. Sea $z\in\mathbb{Z}\backslash\left\{0\right\}$, entonces:
        \begin{equation*}
            1=f(1)=f(z)f(1/z)
        \end{equation*}
        por tanto, como $f(z)\neq 0$, entonces $f(1/z)=1/f(z)$. De forma totalmente análoga se tiene que $g(1/z)=1/g(z)$. Por tanto, como $f$ y $g$ coinciden en $\mathbb{Z}$, se tiene entonces que:
        \begin{equation*}
            f(1/z)=g(1/z),\quad\forall z\in\mathbb{Z}
        \end{equation*}
        Ahora, si $z'\in\mathbb{Z}$, entonces:
        \begin{equation*}
            f(z'/z)=f(z')f(1/z)=g(z')g(1/z)=g(z'/z)
        \end{equation*}
        por tanto, $f=g$. Luego, $i$ es un epimorfismo en $\Cat{Ring}^c$.

        Esta idea se puede generalizar en el caso de que si tenemos un anillo conmutativo con identidad $R$ y $(S^{-1}R,j)$ su localización con respecto al conjunto multiplicativo $S\subseteq R$, entonces la función:
        \begin{equation*}
            \begin{split}
                j:R\rightarrow& S^{-1}R\\
                r\mapsto &\frac{r}{1}\\
            \end{split}
        \end{equation*}
        para todo $r\in R$ es también un epimorfismo en $\Cat{Ring}^c$.
    \end{sol}
    
    \begin{exa}
        El mapeo inclusión $\cf{i}{\mathbb{Z}}{\mathbb{Q}}$ es un monomorfismo en la categoría  $\Cat{Ring}^c$ y también por el ejemplo anterior, un epimorfismo. Pero, no es un isomorfismo.
    \end{exa}

    \begin{proof}
        Veamos que es un monomorfismo, para ello hay que probar que si $\cf{f,g}{R}{\mathbb{Z}}$ son tales que
        \begin{equation*}
            i\circ f=i\circ g
        \end{equation*}
        entonces, $f=g$. En efecto, como el mapeo $i$ es inyectivo, entonces si $r\in R$:
        \begin{equation*}
            i(f(r))=i(g(r))\Rightarrow f(r)=g(r)
        \end{equation*}
        por tanto, $f=g$. Luego $i$ es un monomorfismo. Pero no es isomorfismo ya que si existiera $\cf{j}{\mathbb{Q}}{\mathbb{Z}}$ tal que $i\circ j=1_{\mathbb{Q}}$ y $j\circ i=1_{\mathbb{Z}}$, se tendría entonces que:
        \begin{equation*}
            i(j(\frac{1}{2}))=1_{\mathbb{Q}}(\frac{1}{2})=\frac{1}{2}
        \end{equation*}
        lo cual no puede suceder, pues $i(\mathbb{Z})=\mathbb{Z}\subseteq\mathbb{Q}$. Por tanto tal función no existe, luego $i$ no es isomorfismo, es decir que $i$ es un morfismo que es monomorfismo y epimorfismo, pero que no es isomorfismo.

    \end{proof}

    Todo lo anterior da pie a la siguiente definición:

    \begin{mydef}
        Sea $\mathcal{C}$ una categoría. Un morfismo $f$ de $\mathcal{C}$ se dice \textbf{bimorfismo} si es un epimorfismo y un monomorfismo.

        Además, diremos que una categoría $\mathcal{C}$ se dice \textbf{balanceada} si todo bimorfismo es un isomorfismo.
    \end{mydef}

    \begin{exa}
        Las categorías $\Cat{Set},\Cat{Grp}$ son balanceada.
    \end{exa}

    \begin{exa}
        Las cateogrías $\Cat{Div}$ y $\Cat{Ring}^c$ no son balanceadas, pues en embas no todo bimorfismo es isomorfismo (revisar ejemplos 2.2.6 a 2.2.8).
    \end{exa}

    \begin{exa}
        La categoría $\Cat{Top}$ no es balanceada.
    \end{exa}

    \begin{proof}
        En efecto, sea $X$ un conjunto tal que $\abs{X}\geq 2$. Considere a $\tau_D$ la topología discreta sobre $X$, y sea $\tau$ una topología diferente a la discreta. Es claro que $\tau\subsetneq\tau_D$. Tomemos $\cf{f}{(X,\tau_D)}{(X,\tau)}$ tal que $x\mapsto x$. Esta función es inyectiva y sobreyectiva. Además, es continua pues la imagen inversa de todo abierto en $(X,\tau)$ es abierta en $(X,\tau_D)$.

        Por tanto, $f$ (visto en la categoría $\Cat{Top}$) es bimorfismo, pero no es isomorfismo ya que su inversa $f^{-1}$ no es continua, y por ende no puede ser un morfismo en $\Hom{\Cat{Top}}{(X,\tau)}{(X,\tau_D)}$.

        Luego, $\Cat{Top}$ no es balanceada.
    \end{proof}

    \begin{mydef}
        Una categoría $\mathcal{C}$ se dice que es un \textbf{grupoide} si para todo morfismo entre dos objetos $A$ y $B$, este es un isomorfismo.
    \end{mydef}

    \begin{exa}
        Sea $(G,\cdot)$ grupo. Denotamos por $\mathcal{C}_G$ a la categoría cuyos objetos son $\Obj{\mathcal{C}_G}=\left\{\cdot \right\}$ (donde $e$ denota la identidad del grupo) y, el conjunto de morfismos $\Hom{\mathcal{C}_G}{\cdot}{\cdot}=G$ en el cual $e=1_e\in\Hom{\mathcal{C}_G}{\cdot}{\cdot}$, en donde la Ley de composición está dada por:
        \begin{equation*}
            \forall x,y\in\Hom{\mathcal{C}_G}{\cdot}{\cdot}, x\circ y =xy
        \end{equation*}
        en esta categoría, todo morfismo es un isomorfismo, a saber si $x\in\Hom{\mathcal{C}_G}{\cdot}{\cdot}$, $x^{-1}$ es su isomorfismo inverso. Luego, $\mathcal{C}_G$ es una categoría que es además, un grupoide.

        Este ejemlo es básicamente una particularización del ejemplo de la categoría constriuda a partir de un monoide, en esa categoría lo que faltaba para que fuera grupoide es que cada elemento fuera invertible, lo cual es garantizado por las propiedades del grupo $(G,\cdot)$.
    \end{exa}

    \begin{exa}
        A cada categoría $\mathcal{C}$ se le puede asociar un grupoide, el cual es una subcategoría de ella misma. A saber, se definen para cualesquiera $A,B\in\Obj{\mathcal{C}}$ el conjunto de isomorfismos:
        \begin{equation*}
            \Iso{\mathcal{C}}{A}{B}=\left\{f\in\Hom{\mathcal{C}}{A}{B}\Big| f\textup{ es isomorfismo} \right\}
        \end{equation*}
        tomamos a la categoría $\mathcal{C}'$ tal que $\Obj{\mathcal{C}'}=\Obj{\mathcal{C}}$ y $\forall A,B\in\Obj{\mathcal{C}'}$, $\Hom{\mathcal{C}'}{A}{B}=\Iso{\mathcal{C}}{A}{B}$, donde la ley de correspondencia es la misma que la de $\mathcal{C}$.

        Es claro que $\mathcal{C}'$ es un grupoide.
    \end{exa}

    \begin{mydef}
        Sea $\mathcal{C}$ una categoría y $A\in\Obj{\mathcal{C}}$.
        \begin{enumerate}
            \item $A$ es llamado \textbf{objeto inicial} si para cada $B\in\Obj{\mathcal{C}}$, el conjunto $\Hom{\mathcal{C}}{A}{B}$ consta de un único elemento.
            \item $A$ es llamado \textbf{objeto final} si para cada $C\in\Obj{\mathcal{C}}$, el conjunto $\Hom{\mathcal{C}}{C}{A}$ consta de un único elemento.
            \item $A$ es llamado un \textbf{objeto cero} si $A$ es un objeto inicial y final.
        \end{enumerate}
        Si la categoría $\mathcal{C}$ admite un objeto cero, decimos que $\mathcal{C}$ es una categoría \textbf{puntuada}.
    \end{mydef}

    \begin{obs}
        Sea $\mathcal{C}$ una categoría. Si $\mathcal{C}$ es puntuada, entonces para cualesquier $A,B\in\Obj{\mathcal{C}}$, el conjunto $\Hom{\mathcal{C}}{A}{B}\neq\emptyset$ (de $A$ nos movemos al objeto cero, y del objeto cero a $B$, luego por la ley de composición se sigue que este conjunto contiene al menos a este morfismo).
    \end{obs}

    \begin{propo}
        Sea $\mathcal{C}$ una categoría. Si $A$ y $B$ son objetos iniciales (respectivamente, finales) de $\mathcal{C}$, entonces $A$ es isomorfo a $B$.
    \end{propo}

    \begin{proof}
        Si $A$ y $B$ son objetos iniciales o finales, ya sabe que existen únicos morfismos $f\in\Hom{\mathcal{C}}{A}{B}$ y $g\in\Hom{\mathcal{C}}{B}{A}$ (por ser ambos objetos iniciales o finales). Entonces, $g\circ f\in \Hom{\mathcal{C}}{A}{A}$ $f\circ g\in \Hom{\mathcal{C}}{B}{B}$, pero $\Hom{\mathcal{C}}{A}{A}$ y $\Hom{\mathcal{C}}{B}{B}$ tienen un solo elemento, a saber $1_A$ y $1_B$ (por ser ambos objetos iniciales o finales).
        
        Luego:
        \begin{equation*}
            g\circ f =1_A\quad\textup{y}\quad f\circ g=1_B
        \end{equation*}
        por ende, $f$ y $g$ son isomorfismos, lo que implica que $A$ y $B$ son isomorfos.
    \end{proof}

    \begin{cor}
        Sea $\mathcal{C}$ una categoría puntuada. Entonces, los objetos de $\mathcal{C}$ son únicos hasta isomorfismos.
    \end{cor}

    \begin{proof}
        Es inmedaita de la proposición anterior.
    \end{proof}

    \begin{exa}
        Considere la categoría $\Cat{Set}$. Se tiene que el conjunto vacío es un objeto inicial de esta categoría.
    \end{exa}

    \begin{sol}
        En efecto, sean $A,B\in\Obj{\Cat{Set}}$, tal que $B=\left\{b_1,b_2 \right\}$. Tenemos que las funciones $\cf{f_1}{A}{B}$ y $\cf{f_2}{A}{B}$ dadas por $a\mapsto b_1$ y $a\mapsto b_2$, respectivamente. $A$ no puede ser un objeto inicial ya que $\abs{\Hom{\Cat{Set}}{A}{B}}\geq 2$, esto si $A$ es no vacío.

        Por lo cual, $A$ tiene que ser vacío para que en este conjunto haya solamente una función, a saber la función vacía, que tiene como dominio al vacío.
    \end{sol}

    \begin{exa}
        En la categoría del ejemplo anterior, los conjuntos que constan de un solo elemento son objetos finales de esta categoría.

        Luego, por el ejemplo anterior al tenerse que el único objeto inicial es el vacío, y este no puede ser objeto final, se sigue que $\Cat{Set}$ no tiene objetos cero.
    \end{exa}

    \begin{exa}
        En las categorías $\Cat{Grp},\Cat{Ab}$ y $_R\mathcal{M}$, el grupo trivial, respectivamente el módulo trivial, son objetos cero.
    \end{exa}

    \begin{exa}
        En la categoría $\Cat{Field}$ (los morfismos son homomorfismos de campos que mandan al $1$ en el $1$) no existen objetos iniciales ni finales.
    \end{exa}

    \begin{sol}
        En efecto, sean $E,F\in\Obj{\Cat{Field}}$ tales que $\textup{Car}(E)\neq\textup{Car}(F)$. Supongamos que $\cf{f}{E}{F}$ es un homomorfismo tal que $f(1)=1$. Se tienen los siguientes casos:
        \begin{enumerate}
            \item $\textup{Car}(E)=0$. En este caso, se tiene que $\textup{Car}(F)>0$, digamos que $q=\textup{Car}(F)$. Se tiene que:
            \begin{equation*}
                1=f(1)=f(q\cdot 1)=f(\underbrace{1+...+1}_{q-veces})=qf(1)=0=f(0)
            \end{equation*}
            por tanto, $1=0$\contradiction, ya que estaríamos diciendo que $0=q\cdot 1$.
            \item $\textup{Car}(E)=p$. Es análogo al anterior.
        \end{enumerate}
    \end{sol}

    \begin{obs}
        En la categoría $\Cat{Field}$ no se toma en cuenta el campo trivial con la conidición de que $f(1)=1$ ya que se llega a una contradicción. Así lo que sucede con el campo que solo contiene al $\left\{0\right\}$ no hay homomorfismos más que consigo misma.
    \end{obs}

    \begin{exa}
        En la categoría $\Cat{Ring}$ el anillo $\mathbb{Z}$ es un objeto inicial.
    \end{exa}

    \begin{proof}
        En efecto, sea $R\in\Obj{\Cat{Ring}}$. y $\cf{f}{\mathbb{Z}}{R}$ un homomorfismo. Entonces se tiene para todo $n\in\mathbb{Z}$ que:
        \begin{equation*}
            f(n)=nf(1)=n1_R
        \end{equation*}
        es decir, que queda unívocamente determinada la $f$ por el $1_R$.Así esta función queda determinada únicamente por el campo $R$. Así, solo puede haber un homomorfismo de $\mathbb{Z}$ en $R$.
    \end{proof}

    \begin{mydef}
        Sean $\mathcal{C}$ una categoría y $C\in\Obj{\mathcal{C}}$. Si $A,B\in\Obj{\mathcal{C}}$ y, $f,g$ son monomorfismos tales que $f\in\Hom{\mathcal{C}}{A}{C}$ y $g\in\Hom{\mathcal{C}}{B}{C}$, diremos entonces que $f$ y $g$ son \textbf{equivalentes} si existe un isomorfismo $u\in\Hom{\mathcal{C}}{A}{B}$ tal que
        \begin{equation*}
            g\circ u=f
        \end{equation*}
        es decir, hace del diagrama de abajo un diagrama conmutativo.

        La relación de ser equivalentes es una relación de equivalencia sobre la clase de todos los monomorfimos con contradominio $C$.
    \end{mydef}

    \begin{proof}
        Se tienen que verificar tres cosas:
        \begin{enumerate}
            \item \textbf{Reflexividad}. Si $f\in\Hom{\mathcal{C}}{A}{C}$ es monomorfismo, entonces $f$ es equivalente a sí misma ya que:
            \begin{equation*}
                f\circ 1_A=f
            \end{equation*}
            donde $1_A\in\Hom{\mathcal{C}}{A}{A}$ es un isomorfismo de $A$ en $A$.
            \item \textbf{Simetría}. Si $f\in\Hom{\mathcal{C}}{A}{C}$ y $g\in\Hom{\mathcal{C}}{B}{C}$ son monomorfismos tales que existe $u\in\Hom{\mathcal{C}}{A}{B}$ isomorfismo que cumple:
            \begin{equation*}
                g\circ u=f
            \end{equation*}
            como $u$ es isomorfismo, existe $u^*\in\Hom{\mathcal{C}}{B}{A}$ tal que:
            \begin{equation*}
                u^*\circ u=1_A \quad\textup{y}\quad u\circ u^*=1_B
            \end{equation*}
            se tiene entonces que:
            \begin{equation*}
                \begin{split}
                    f\circ u^*&=(f)\circ u^*\\
                    &=(g\circ u)\circ u^*\\
                    &=g\circ(u\circ u^*)\\
                    &=g\\
                \end{split}
            \end{equation*}
            luego, $g$ y $f$ son equivalentes.
            \item \textbf{Transitividad}. Sean $f\in\Hom{\mathcal{C}}{A}{C}$, $g\in\Hom{\mathcal{C}}{B}{C}$ y $h\in\Hom{\mathcal{C}}{D}{C}$ monomorfismos tales que existen $u\in\Hom{\mathcal{C}}{A}{B}$ y $v\in\Hom{\mathcal{C}}{B}{D}$ isomorfismos con:
            \begin{equation*}
                g\circ u=f\quad\textup{y}\quad h\circ v=g
            \end{equation*}
            veamos que $v\circ u$ es un isomorfismo. En efecto, por ser cada uno isomorfismos, existen $u^*\in\Hom{\mathcal{C}}{B}{A}$ y $v^*\in\Hom{\mathcal{C}}{D}{B}$ tales que:
            \begin{equation*}
                u^*\circ u=1_A\quad\textup{y}\quad u\circ u^*=1_B
            \end{equation*}
        \end{enumerate}
        por los tres incisos anteriores, esta relación es una relación de equivalencia.
    \end{proof}

    \begin{mydef}
        Sea $\mathcal{C}$ una categoría y $C\in\Obj{\mathcal{C}}$. A las clases de equivalencia de la relación anterior las llamaremos los \textbf{sub-objetos} de $C$.
    \end{mydef}

    \begin{exa}
        Consideremos las categorías $\Cat{Set}$ y sea $X\in\Obj{\Cat{Set}}$ fijo. Definimos la siguiente función:
        \begin{equation*}
            \begin{split}
                \psi_X:\mathcal{P}(X)&\rightarrow \SO{X}\\
                Y&\mapsto \hat{i_Y}\\
            \end{split}
        \end{equation*}
        para cada $Y\subseteq X$, donde $\SO{X}$ denota la clase de sub-objetos de $X$ y, $i_Y$ es la función dada por la inclusión $\cf{i_Y}{Y}{X}$, y el sombrerito denota su clase de equivalencia.
    \end{exa}

    \begin{sol}
        Claramente esta función está bien definida. Afirmamos que $\psi_X$ es una biyección. En efecto, sea $Y\subseteq X$, $V\in\Obj{\Cat{Set}}$ y $\cf{f}{V}{X}$ un monomorfismo tal que $f(V)=Y$. Como $f$ es una función inyectiva, la función $\cf{u}{V}{Y}$ tal que $v\mapsto f(v)$ es una función biyectiva, y
        \begin{equation*}
            i_Y\circ u=f
        \end{equation*}
        luego, $\hat{i_Y}=\hat{f}$. De aquí se deduce que $\psi_X$ es suprayectiva.

        Veamos que es inyectiva. Sean $Y,Z\subseteq X$ tales que $\hat{i_Y}=\hat{i_Z}$, luego existe una función biyectiva $\cf{u}{Y}{Z}$ tal que:
        \begin{equation*}
            i_Z\circ u = i_Y
        \end{equation*}
        Sea $y\in Y$, entonces por la igualdad $y=i_Y(y)=i_Z(u(y))=u(y)\in Z$. Por tanto, $Y\subseteq Z$. Considerando la inversa de $u$ se llega a la otra contención, así $Y=Z$.

        Por tanto, $\psi_X$ es biyección.
    \end{sol}

    \section{Construcción de Categorías}

    \begin{mydef}
        Sea $\mathcal{C}$ una categoría, llamaremos por \textbf{Categoría dual u opuesta}, denotada como $\mathcal{C}^{op}$,
        \begin{enumerate}
            \item $\Obj{\mathcal{C}^{op}}=\Obj{\mathcal{C}}$.
            \item \item $\Hom{\mathcal{C}^{op}}{A}{B}=\Hom{\mathcal{C}}{B}{A}$, para todo $A,B\in\Obj{\mathcal{C}^{op}}$.
            \item $\cf{\circ^{op}}{\Hom{\mathcal{C}^{op}}{A}{B}\times \Hom{\mathcal{C}^{op}}{B}{C}}{\Hom{\mathcal{C}^{op}}{A}{C}}$ está dada por: si $g^{op}\in\Hom{\mathcal{C}^{op}}{A}{B}$ y $f^{op}\in\Hom{\mathcal{C}^{op}}{B}{C}$, entonces se corresponde a $f^{op}$ y $g^{op}$ dos morfismos $f\in\Hom{\mathcal{C}}{C}{B}$ y $g\in\Hom{\mathcal{C}}{B}{A}$ tales que:
            \begin{equation*}
                f^{op}=f\quad\textup{ y }g^{op}=g
            \end{equation*}
            se tiene que:
            \begin{equation*}
                g^{op}\circ^{op} f^{op}=(f\circ g)^{op}
            \end{equation*}
            \item $1_A^{op}=1_A$, para todo $A\in\Obj{\mathcal{C}^{op}}$
        \end{enumerate}
    \end{mydef}

    \begin{exa}
        Se cumple lo siguiente:
        \begin{enumerate}
            \item $(\mathcal{C}^{op})^{op}=\mathcal{C}$.
            \item $\Cat{PoSet}(X,\subseteq)^{op}=\Cat{PoSet}(X,\supseteq)$.
        \end{enumerate}
    \end{exa}

    \begin{propo}[\textbf{Propiedades duales de los morfismos}]
        Sea $\mathcal{C}$ una categoría.
        \begin{enumerate}
            \item $f$ es monomorfismo de $\mathcal{C}$ si y sólo si $f^{op}$ es epimorfismo de $\mathcal{C}^{op}$.
            \item $I$ es elemento inicial de $\mathcal{C}$ si y sólo si $I$ es elemento final de $\mathcal{C}^{op}$.
            \item $f$ es isomorfismo en $\mathcal{C}$ si y sólo si $f^{op}$ es isomorfismo de $\mathcal{C}^{op}$.
        \end{enumerate}
    \end{propo}

    \begin{proof}
        Sean $A,B\in\Obj{\mathcal{C}}$. Entonces $f\in\Hom{\mathcal{C}}{A}{B}$ es monomorfismo si y sólo si para todo $C\in\Obj{\mathcal{C}}$ se tiene que si $g,h\in\Hom{\mathcal{C}}{B}{C}$ son tales que:
        \begin{equation*}
            f\circ 
        \end{equation*}
    \end{proof}

    \begin{mydef}
        Sean $\mathcal{C},\mathcal{D}$ categorías. Llamaremos su \textbf{Categoría Producto}, denotada por $\mathcal{C}\times\mathcal{D}$ a:
        \begin{enumerate}
            \item $\Obj{\mathcal{C}\times\mathcal{D}}=\Obj{\mathcal{C}}\times\Obj{\mathcal{D}}$.
            \item Sean $(C,D),(C',D')\in\Obj{\mathcal{C}\times\mathcal{D}}$, entonces $\Hom{\mathcal{C}\times\mathcal{D}}{(C,D)}{(C',D')}=\Hom{\mathcal{C}}{C}{C}\times\Hom{\mathcal{D}}{D}{D'}$.
            \item $\cf{\circ_{\mathcal{C}\times\mathcal{D}}}{\Hom{\mathcal{C}\times\mathcal{D}}{(C,D)}{(C',D')}\times\Hom{\mathcal{C}\times\mathcal{D}}{(C',D')}{(C'',D'')}}{\Hom{\mathcal{C}\times\mathcal{D}}{(C,D)}{(C'',D'')}}$ dada como:
            \begin{equation*}
                (f',g')\circ_{\mathcal{C}\times\mathcal{D}}(f,g)=(f'\circ_{\mathcal{C}} f,g'\circ_{\mathcal{D}} g)
            \end{equation*}
            \item $1_{C\times D}=(1_C,1_D)$.
        \end{enumerate}
    \end{mydef}

    \begin{exa}
        Sean $G_1,G_2$ grupos y $\mathcal{G}_1,\mathcal{G}_2$ son sus categorías respectivas (formadas como con el monoide), donde:
        \begin{enumerate}
            \item $\Obj{\mathcal{G}_i}=\left\{\cdot_i \right\}$.
            \item $\Hom{\mathcal{G}_i}{\cdot_i}{\cdot_i}=G_i$.
        \end{enumerate}
        para $i=1,2$. Entonces, en la categoría producto $\mathcal{G}_1\times\mathcal{G}_2$ el único objeto es $(\cdot_1,\cdot_2)$. En este caso, esta categoría se identifica con la categoría del producto directo de los grupos $G_1$ y $G_2$.
    \end{exa}

\end{document}