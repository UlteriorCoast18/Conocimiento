\documentclass[12pt]{report}
\usepackage[spanish]{babel}
\usepackage[utf8]{inputenc}
\usepackage{amsmath}
\usepackage{amssymb}
\usepackage{amsthm}
\usepackage{graphics}
\usepackage{subfigure}
\usepackage{lipsum}
\usepackage{array}
\usepackage{multicol}
\usepackage{enumerate}
\usepackage[framemethod=TikZ]{mdframed}
\usepackage[a4paper, margin = 1.5cm]{geometry}

%En esta parte se hacen redefiniciones de algunos comandos para que resulte agradable el verlos%

\renewcommand{\theenumii}{\roman{enumii}}

\def\proof{\paragraph{Demostración:\\}}
\def\endproof{\hfill$\blacksquare$}

\def\sol{\paragraph{Solución:\\}}
\def\endsol{\hfill$\square$}

%En esta parte se definen los comandos a usar dentro del documento para enlistar%

\newtheoremstyle{largebreak}
  {}% use the default space above
  {}% use the default space below
  {\normalfont}% body font
  {}% indent (0pt)
  {\bfseries}% header font
  {}% punctuation
  {\newline}% break after header
  {}% header spec

\theoremstyle{largebreak}

\newmdtheoremenv[
    leftmargin=0em,
    rightmargin=0em,
    innertopmargin=0pt,
    innerbottommargin=5pt,
    hidealllines = true,
    roundcorner = 5pt,
    backgroundcolor = gray!60!red!30
]{exa}{Ejemplo}[section]

\newmdtheoremenv[
    leftmargin=0em,
    rightmargin=0em,
    innertopmargin=0pt,
    innerbottommargin=5pt,
    hidealllines = true,
    roundcorner = 5pt,
    backgroundcolor = gray!50!blue!30
]{obs}{Observación}[section]

\newmdtheoremenv[
    leftmargin=0em,
    rightmargin=0em,
    innertopmargin=0pt,
    innerbottommargin=5pt,
    rightline = false,
    leftline = false
]{theor}{Teorema}[section]

\newmdtheoremenv[
    leftmargin=0em,
    rightmargin=0em,
    innertopmargin=0pt,
    innerbottommargin=5pt,
    rightline = false,
    leftline = false
]{propo}{Proposición}[section]

\newmdtheoremenv[
    leftmargin=0em,
    rightmargin=0em,
    innertopmargin=0pt,
    innerbottommargin=5pt,
    rightline = false,
    leftline = false
]{cor}{Corolario}[section]

\newmdtheoremenv[
    leftmargin=0em,
    rightmargin=0em,
    innertopmargin=0pt,
    innerbottommargin=5pt,
    rightline = false,
    leftline = false
]{lema}{Lema}[section]

\newmdtheoremenv[
    leftmargin=0em,
    rightmargin=0em,
    innertopmargin=0pt,
    innerbottommargin=5pt,
    roundcorner=5pt,
    backgroundcolor = gray!30,
    hidealllines = true
]{mydef}{Definición}[section]

\newmdtheoremenv[
    leftmargin=0em,
    rightmargin=0em,
    innertopmargin=0pt,
    innerbottommargin=5pt,
    roundcorner=5pt
]{excer}{Ejercicio}[section]

%En esta parte se colocan comandos que definen la forma en la que se van a escribir ciertas funciones%

\newcommand\abs[1]{\ensuremath{\biglvert#1\bigrvert}}
\newcommand\divides{\ensuremath{\bigm|}}
\newcommand\cf[3]{\ensuremath{#1:#2\rightarrow#3}}
\newcommand\contradiction{\ensuremath{\#_c}}
\newcommand{\Obj}[1]{\ensuremath{\textup{Obj}\left(#1\right)}}
\newcommand{\Hom}[3]{\ensuremath{\textup{Hom}_{#1}\left(#2,#3\right)}}

\newcommand{\Cat}[1]{\ensuremath{\textup{\textbf{#1}}}}

%recuerda usar \clearpage para hacer un salto de página

\begin{document}
    \title{Notas de Álgebra Moderna IV.
    
    Una introducción a la teoría de categorías.}
    \author{Cristo Daniel Alvarado}
    \maketitle

    \tableofcontents %Con este comando se genera el índice general del libro%

    %\setcounter{chapter}{3} %En esta parte lo que se hace es cambiar la enumeración del capítulo%

    \chapter{Clases y conjuntos}

    \section{Axiomas de Von-Newmann-Gödel}

    Antes de decantarnos totalmente a nuestro estudio de las categorías, primero nos enfocaremos en estudiar a los objetos que se van a usar (las clases).

    Aceptamos la existencia de \textit{objetos primitivos}, las cuales son clases y conjuntos, dotadas de dos relaciones primitivas, la pertenencia $\in$ e igualdad $=$. Denotamos a los objetos primitivos por letras en mayúsculas.

    \begin{mydef}[\textbf{Axiomas de NBG}]
        \renewcommand{\theenumi}{A\arabic{enumi}}
        Se tienen los siguientes axiomas:
        \begin{enumerate}
            \item Todo conjunto es una clase.
            \item Si $x\in A$, $\forall x\in B$ y $x\in B$, $\forall x\in A$, entonces $A=B$.
            \item Si $A\in B$ donde $B$ es una clase, entonces $A$ es un conjunto.
            \item Si $P(x)$ es una propiedad definida sobre el parámetro $x$ que se recorre sobre conjuntos, entonces existe una clase $[x\big| P(x)]$ tal que para cada conjunto $y$.
            \begin{equation*}
                \begin{split}
                    y\in[x\big| P(x)]\iff P(y)
                \end{split}
            \end{equation*}
            \item Si $X,Y$ son conjuntos, entonces $[X,Y]$ es un conjunto y se denota por $\left\{X, Y\right\}$ (ver ejemplo 1.1.3).
            \item Si $X$ es un conjunto, entonces $\left\{X\right\}$, $\left\{X,\left\{X\right\} \right\}$,... son conjuntos.
            \item Existe un conjunto inductivo.
            \item Sea $A$ conjunto, entonces existe un conjunto denotado por $\mathcal{P}(A)$ tal que $B\in\mathcal{P}(A)$ si y sólo si $B\subseteq A$.
            \item Si $\cf{f}{A}{B}$ donde $A$ es un conjunto, entonces $f(A)$ es un conjunto.
        \end{enumerate}
    \end{mydef}

    \begin{exa}
        Construimos al \textbf{conjunto vacío} $\emptyset$ como $\emptyset=[x\big| x\neq x]$ (usando a A4).
    \end{exa}
    
    \begin{exa}
        $\textup{Set}=[x| x=x]$ (usando a A4).
    \end{exa}

    \begin{exa}
        Si $X$ y $Y$ son conjuntos, entonces
        \begin{equation*}
            [X,Y]=[Z\big| Z=X\textup{ o }Z=Y]
        \end{equation*}
        (construida por el A4).
    \end{exa}

    \begin{exa}
        Si $X$ es un conjunto, entonces $X\cup\left\{X\right\}$ es un conjunto y se denomina el \textbf{sucesor de $X$}.
    \end{exa}

    \begin{mydef}
        Sea $A$ una clase. Se define
        \begin{equation*}
            \bigcup A=\bigcup_{X\in A}X=[x\big| \exists X\in A\textup{ tal que }x\in X]
        \end{equation*}
        Si $A$ es un conjunto, $\bigcup A$ es un conjunto.
    \end{mydef}

    \begin{mydef}
        Un conjunto $A$ se denomina \textbf{inductivo} si
        \renewcommand{\theenumi}{\roman{enumi}}
        \begin{enumerate}
            \item $\emptyset\in A$.
            \item $X\in A\Rightarrow X\cup\left\{X\right\}\in A$.
        \end{enumerate}
    \end{mydef}

    \begin{propo}
        $\emptyset$ es un conjunto.
    \end{propo}

    \begin{proof}
        Sea $A$ un conjunto inductivo (el cual existe por A7), entonces $\emptyset\in A$, luego por A3, $\emptyset$ es un conjunto.
    \end{proof}

    \begin{mydef}
        Se dice que $B$ es subclase de $A$, si $x\in A$ para todo $x\in B$, y se denota por $B\subseteq A$.
    \end{mydef}

    \begin{propo}
        Si $B\subseteq A$ y $A$ es conjunto, entonces $B$ es conjunto.
    \end{propo}

    \begin{proof}
        Como $B\subseteq A$, entonces $B\in\mathcal{P}(A)$, luego $B$ por A3, $B$ es conjunto.
    \end{proof}

    Esta proposición es necesaria pues no sabemos si las subclases de conjuntos son conjuntos.

    \begin{mydef}
        Si $x,y$ son conjuntos, se define:
        \begin{equation*}
            (x,y)=\left\{\left\{x\right\},\left\{x,y\right\} \right\}
        \end{equation*}
        Si $A$ y $B$ son clases, se define
        \begin{equation*}
            A\times B = [(x,y)\big| x\in A\textup{ y }y\in B ]
        \end{equation*}
    \end{mydef}

    \begin{excer}
        Si $A$ y $B$ son conjuntos, entonces $A\times B$ es conjunto.
    \end{excer}

    \begin{proof}
        
    \end{proof}

    \begin{mydef}
        Una función de $A$ en $B$ es una subclase $F\subseteq A\times B$ tal que $(x,y),(x,z)\in F\Rightarrow y=z$.
    \end{mydef}

    \begin{exa}
        $\textup{Set}$ no es un conjunto.
    \end{exa}

    \begin{proof}
        Supóngase que Set es un conjunto. Sea
        \begin{equation*}
            X=[x\big| x\notin x]
        \end{equation*}
        Si $x\in X$, entonces $x$ es un conjunto (por A3) luego $x\in\textup{Set}$, es decir que $x$ es un conjunto. Por tanto, $X\subseteq\textup{Set}$, esto es que $X$ es un conjunto. Luego sucede que $X\in X$ o $X\notin X$ (por como se formó la clase $X$ a partir de A4).
        
        Por ende, $X\in X\iff X\notin X$\contradiction. Luego $\textup{Set}$ no es un conjunto.
    \end{proof}

    \begin{exa}
       Denotamos por $\mathcal{G}=[G| G\textup{ es grupo}]$, y $\mathcal{S}=[S_X\big| X\in\textup{Set}]$. Si sucediera que $S$ fuese conjunto, tomando $\cf{f}{\mathcal{S}}{\textup{Set}}$, $S_X\mapsto X$ es una función, luego $F(\mathcal{S})=\textup{Set}$ es un conjunto, lo cual no puede ser. Por tanto, como $\mathcal{S}\subseteq\mathcal{G}$, se sigue que $\mathcal{G}$ es clase.
    \end{exa}

    \chapter{Categorias}

    \section{Conceptos Fundamentales}

    Antes de comenzar aceptaremos como válido al siguiente axioma:

    \renewcommand{\theenumi}{A\arabic{enumi}}

    \begin{enumerate}
        \setcounter{enumi}{9}
        \item \textbf{Limitación de tamaño}. Una clase es un conjunto si y sólo si no es biyectivo con $\textup{Set}$.
    \end{enumerate}

    Ahora si con la parte de categorías.
    
    \renewcommand{\theenumi}{\arabic{enumi}}

    \begin{mydef}
        Una \textbf{categoría} $\mathcal{C}$ consta de lo siguiente:
        \begin{enumerate}
            \item Una clase $\textup{Obj}(\mathcal{C})$ cuyos elementos son llamados \textbf{objetos}.
            \item Para cada par $A,B\in\Obj{\mathcal{C}}$ existe un conjunto $\Hom{\mathcal{C}}{A}{B}$ cuyos elementos llamaremos morfismos y, dado un morfismo $f\in\Hom{\mathcal{C}}{A}{B}$ lo denotaremos por $\cf{f}{A}{B}$.
            \item Para cada objeto $A\in\Obj{\mathcal{C}}$ hay un morfismo $1_A\in\Hom{\mathcal{C}}{A}{A}$ llamado la \textbf{identidad de $A$}.
            \item Hay una ley de composición para una terna de objetos $A$, $B$ y $C$:
            \begin{equation*}
                \begin{split}
                    \Hom{\mathcal{C}}{A}{B}\times\Hom{\mathcal{C}}{B}{C}\rightarrow&\Hom{\mathcal{C}}{A}{C}\\
                    (f,g)\mapsto&g\circ f\\
                \end{split}
            \end{equation*}
            que satisface lo siguiente:
            \begin{enumerate}
                \item (\textit{Asociatividad}). Dado $f\in\Hom{\mathcal{C}}{A}{B}$ y $g\in\Hom{\mathcal{C}}{B}{C}$ y $h\in\Hom{\mathcal{C}}{C}{D}$ se cumple que:
                \begin{equation*}
                    h\circ (g\circ f)=(h\circ g)\circ f
                \end{equation*}
                \item Dado un morfismo $f\in\Hom{\mathcal{C}}{A}{B}$, se tiene que:
                \begin{equation*}
                    f\circ 1_A=f=1_B\circ f
                \end{equation*}
            \end{enumerate}
        \end{enumerate}
    \end{mydef}

    \begin{mydef}
        Si la clase de objetos de la categoría $\mathcal{C}$ es un conjunto, diremos que $\mathcal{C}$ es una \textbf{categoría pequeña}. 
        
        Más aún, si tenemos un número finito de morfismos (hablando de todos los que puede haber en la categoría y entre todos los objetos de la categoría), diremos que $\mathcal{C}$ es una \textbf{categoría finita}.
    \end{mydef}

    Dadas las definciones anteriores, no se nos da ejemplos concretos de lo que es una categoría, por lo cual procederemos a dar ejemplos de la misma.

    \begin{obs}
        Denotamos por $\mathbb{N}_0=\mathbb{N}^*$ a $\mathbb{N}\cup\left\{0\right\}$.
    \end{obs}

    \begin{exa}
        Sea $X$ un conjunto. Denotamos por $\mathcal{C}_X$ a una categoría formada por $\Obj{\mathcal{C}_X}=X$, y tendremos para cualquier par de elementos $x,y\in\Obj{\mathcal{C}_X}$ definimos:
        \begin{equation*}
            \Hom{\mathcal{C}_X}{x}{y}=\left\{
                \begin{array}{lcr}
                    \emptyset & \textup{ si } & x\neq y\\
                    \left\{1_x\right\} & \textup{ si } & x=y\\
                \end{array}
            \right.
        \end{equation*}
        Definimos la ley de composición de la siguiente forma:
        \begin{equation*}
            1_x\circ 1_x=1_x,\quad\forall x\in X
        \end{equation*}
        Observemos que el único caso en el que está definido es cuando los tres objetos de la categoría son el mismo, en caso contrario el conjunto de morfismos es vacío. Estos objetos que tomamos aquí hacemos $1_x=\left\{x\right\}$ para todo $x\in X$. Este ejemplo da una razón para decir que es lo que son los morfismos y objetos de la categoría.

        $\mathcal{C}_X$ es una categoría pequeña la cual no necesariamente es finita (es finita en el caso que la cardinalidad de $X$ sea finita).

        Si $X=\emptyset$ entonces $\mathcal{C}_X$ es la \textbf{categoría vacía} (que coincide cuando $n=0$ en la siguiente parte).
        
        Denotamos $\Cat{n}$ a la categoría como la de este ejemplo del conjunto con $n$ elementos, donde $n\in\mathbb{N}$.
    \end{exa}
    
    \begin{mydef}
        Un conjunto no vacío $X$ es preordenado si...
    \end{mydef}

    \begin{exa}
        Consideremos al conjunto preordenado $(X,\leq)$ (llamado a veces \textbf{PoSet}). Definimos la categoría $\mathcal{C}_{(X,\leq)}$ en donde:
        \begin{itemize}
            \item $\Obj{\mathcal{C}_{(X,\leq)}}=X$.
            \item Se define el conjunto de morfismos por:
            \begin{equation*}
                \Hom{\mathcal{C}_{(X,\leq)}}{x}{y}=\left\{
                    \begin{array}{lcr}
                        \emptyset & \textup{ si } & x\nleq y\\
                        \left\{\varphi_{x,y} \right\} & \textup{ si } & x\leq y\\
                    \end{array}
                \right.
            \end{equation*}
            \item Sean $x,y,z\in \Obj{\mathcal{C}_{(X,\leq)}}$. Se define la Ley de Composición de la siguiente manera:
            \begin{equation*}
                \begin{split}
                    \Hom{\mathcal{C}_{(X,\leq)}}{x}{y}\times\Hom{\mathcal{C}_{(X,\leq)}}{y}{z}\rightarrow&\Hom{\mathcal{C}_{(X,\leq)}}{x}{z}\\
                    (\varphi_{x,y},\varphi_{y,z})\mapsto&\varphi_{x,z}\\
                \end{split}
            \end{equation*}
            La Ley de composición anterior está definida solamente cuando $x\leq y\leq z$, pues es el único caso en el que el conjunto de morfismos es no vacío.
        \end{itemize}

        El morfismo es un objeto que cumple las leyes anteriores.
    \end{exa}
    
    \begin{exa}
        Construimos la categoría $\Cat{Set}$, donde la clase objetos es la clase de todos los conjuntos y los morfismos son las funciones entre cada conjunto. La ley de composición es la composición usual de funciones.
    \end{exa}

    \begin{exa}
        Sea $(M,\cdot)$ un monoide, es decir, $M$ es un conjunto no vacío en el que $\cdot$ es una operación binaria que es asociativa, para la cual existe un elemento $e_M\in M$ tal que:
        \begin{equation*}
            e_M\cdot x=x\cdot e_M=x\quad\forall x\in M
        \end{equation*}
        Denotamos por $\mathcal{M}$ a la categoría en donde:
        \begin{itemize}
            \item $\Obj{\mathcal{M}}=\left\{\cdot \right\}$.
            \item $\Hom{\mathcal{M}}{\cdot}{\cdot}=\left\{M \right\}$.
            \item La ley de composición $\circ$, se define de la siguiente forma:
            \begin{equation*}
                x\circ y=x\cdot y
            \end{equation*}
            donde $x,y\in \Hom{\mathcal{M}}{\cdot}{\cdot}$ (en este caso $\cdot$ son el dominio y codominio de $x,y\in X$).
        \end{itemize}
        En este caso, el morfismo identidad es la identidad del monoide.

        Este ejemplo se puede extender a grupos, y en el caso de grupos abelianos se tendría que la ley de composición es conmutativa.
    \end{exa}

    \begin{exa}
        La categoría $\Cat{Grp}$ es la categoría de todos los grupos, donde la clase de objetos de la categoría es la clase de todos los grupos y los morfismos son los homomorfismos de grupos.

        De forma similar $\Cat{Grp}=\Cat{AbGrp}$ y $\Cat{SiGrp}$ son las categorías de grupos abelianos y simples.
    \end{exa}

    \begin{exa}
        Sea $R$ un anillo con identidad, denotamos por $R_\mathcal{M}$ a la categoría
    \end{exa}

    \begin{exa}
        La categoría $\Cat{Top}$ es la categoría de todos los espacios topológicos con funciones continuas entre los espacios como los morfismos.

        De forma análoga con $\Cat{HausTop}$, que es la categoría de todos los espacios topológicos que en particular son Hausdorff y los morfismos son las funciones continuas entre los espacios.
    \end{exa}

    \begin{mydef}
        Sean $\mathcal{C}$ y $\mathcal{C}'$ dos categorías. Si
        \begin{enumerate}
            \item $\Obj{\mathcal{C}'}\subseteq\Obj{\mathcal{C}}$.
            \item $\Hom{\mathcal{C}}{A}{B}\subseteq\Hom{\mathcal{C}'}{A}{B}$ para todo $A,B\in\Obj{\mathcal{C}'}$.
            \item La ley de composición de morfismos en $\mathcal{C}'$ está inducida por la ley de composición de morfismos en $\mathcal{C}$.
            \item Los morfismos identidad en $\mathcal{C}'$ son los morfismos identidad en $\mathcal{C}$.
        \end{enumerate}
        se dice entonces que $\mathcal{C}'$ es una \textbf{subcategoría} de $\mathcal{C}$.

        Más aún, $\mathcal{C}'$ se dice que es una \textbf{subcategoría llena} de $\mathcal{C}$ si para cada par $(A,B)$ de objetos de $\mathcal{C}'$ tenemos que
        \begin{equation*}
            \Hom{\mathcal{C}'}{A}{B}=\Hom{\mathcal{C}}{A}{B}
        \end{equation*}
    \end{mydef}

    \begin{exa}
        \begin{enumerate}
            Se cumple lo siguiente:
            \item Las categorías $\Cat{Ab}$ y $\Cat{SiGrp}$ son subcategorías llenas de $\Cat{Set}$.
            \item La categoría $\Cat{HausTop}$ es una subcategoría llena de $\Cat{Top}$.
            \item $\Cat{Ring}^c$ y $\Cat{Field}$ son subcategorísa de $\Cat{Ring}$, más aún son subcategorías llenas.
        \end{enumerate}
    \end{exa}

    \section{Objetos especiales y Morfismos en una Categoría}

    Las nociones de monomirfismo y epimorfismo que se van a introducir son generalizaciones a categorías arbitrarias de las funciones familiares inyectiva y suprayectiva que van de $\Cat{Set}$.

    \begin{mydef}
        Sea $\mathcal{C}$ una categoría y $f\in\Hom{\mathcal{C}}{A}{B}$.
        \begin{enumerate}
            \item $f$ es llamado \textbf{monomorfismo} si para cualesquiera $g_1,g_2\in\Hom{\mathcal{C}}{C}{A}$ (siendo $C$ un objeto de la categoría) tales que $f\circ g_1=f\circ g_2$ tenemos que $g_1=g_2$.
            \item $g$ es llamado \textbf{epimorfismo} si para cualesquiera $h_1,h_2\in\Hom{\mathcal{C}}{B}{C}$ (siendo $C$ un objeto de la categoría) tales que $h_1\circ f=h_2\circ f$ tenemos que $h_1=h_2$.
            \item $f$ es llamado \textbf{isomorfismo} si existe $f'\in\Hom{\mathcal{C}}{B}{A}$ tal que $f\circ f'=1_B$ y $f'\circ f=1_A$. En este caso decimos que $A$ y $B$ son \textbf{objetos isomorfos}.
        \end{enumerate}
    \end{mydef}

    A pesar de que en la categoría $\Cat{Set}$ los monomorfismos (respectivamente, epimorfismos) coinciden con funciones inyectivas (respectivamente, suprayectivas), esto no es cierto en cualquier categoría arbitraria cuyos objetos y morfismos pueden ser conjuntos y funciones, respectivamente. Esto se verá más a fondo en los siguientes ejemplos.

    \begin{exa}
        En cada una de las categorías $\Cat{Set},\Cat{Grp},\Cat{Ab},_R\mathcal{M}$ los monomorfismos coinciden con los homomorfismos inyectivos, mientras que en $\Cat{Top}$ y $\Cat{KHaus}$ los monomorfismos coinciden con los mapeos continuos inyectivos. Solo se probará que los monomorfismos en la categoría $ \Cat{Set}$ coinciden con las funciones inyectivas.
    \end{exa}

    \begin{proof}
        Sea $f\in\Hom{\Cat{Set}}{A}{B}$ es morfismo.

        Suponga que $f$ es monomorfismo, decir que $\cf{f}{A}{B}$ es una función que cumple el primer inciso de la definición anterior. Hay que probar que $f$ es inyectiva. Sean $a\in A$ y $a'\in A$ elementos tales que $f(a)=f(a')$. Consideremos a $g_1$ y $g_2$ los morfismos tales que $g_1(x)=a$ y $g_2(x)=a'$, estos pertenecen a $\Hom{\Cat{Set}}{\left\{x\right\}}{A}$ (en este contexto $x$ es un elemento de un conjunto arbitrario). Se tiene entonces que
        \begin{equation*}
            \begin{split}
                f\circ g_1(x)=&f(a) \\
                =&f(a') \\
                =&f\circ g_2(x) \\
            \end{split}
        \end{equation*}
        por tanto, $f\circ g_1=f\circ g_2$ (ya que coinciden en $x$). Por tanto, al suponer que fue monomorfismo, se sigue que $g_1=g_2$, es decir que $a=a'$. Luego $f$ es inyectiva.

        Suponga ahora que $f$ es inyectiva. Sean $g_1,g_2\in\Hom{\Cat{C}}{C}{A}$ tales que $f\circ g_1 = f\circ g_2$. Si $c\in C$, tenemos que:
        \begin{equation*}
            \begin{split}
                f\circ g_1(c)=&f\circ g_2(c)\\
                f(g_1(c))=&f(g_2(c))\\
                \Rightarrow g_1(c)=&g_2(c)\\
            \end{split}
        \end{equation*}
        pues $f$ es inyectiva. Como el $c\in C$ fue arbitrario, se sigue que $g_1=g_2$, por ende, $f$ es monomorfismo.

    \end{proof}

    De forma idéntica se prueba lo anterior para las categorías $\Cat{Top}$ y $\Cat{KHaus}$ (pues los morfismos en estas son simplemente las funciones continuas).

    \begin{exa}
        
    \end{exa}

    \begin{proof}
        
    \end{proof}

    \begin{exa}
        En cada una de las categorías $\Cat{Set}$, $\Cat{Grp}$, $\Cat{Ab}$ y $_R\mathcal{M}$, los isomorfismos coinciden con los homomorfismos biyectivos (por los ejemplos anteriores). En $\Cat{Top}$, los isomorfismos son exactamente los \textit{homeomorfismos}, esto es, biyecciones continuas cuyas inversas también son continuas.
    \end{exa}

    \begin{mydef}
        Sea $(G,+)$ un grupo abeliano. Decimos que $G$ es un \textbf{grupo divisible} si para cualesquiera $n\in\mathbb{N}$ y $g\in G$ existe $h\in H$ tal que $nh=g$.
    \end{mydef}

    \begin{exa}
        En la categoría $\Cat{Div}$ de grupos divisibles, el mapeo cociente $\cf{q}{\mathbb{Q}}{\mathbb{Q}/\mathbb{Z}}$ entre los grupos $(\mathbb{Q},+)$ y $(\mathbb{Q}/\mathbb{Z},+)$ no es inyectivo pero es un monomorfismo.
    \end{exa}

    \begin{sol}
        Recordemos que $(\mathbb{Q},+)$ es grupo con la operación de suma de elementos y $(\mathbb{Q}/\mathbb{Z},+)$ es el grupo cuyos elementos están dados de la siguiente forma; si $x\in\mathbb{Q}$, entonces $[x]\in\mathbb{Q}/\mathbb{Z}$ es el conjunto formado por:
        \begin{equation*}
            [x]=\left\{y\in\mathbb{Q}\big|y-x\in\mathbb{Z}  \right\}
        \end{equation*}
        Es claro que el mapeo cociente no es inyectivo, pues $q(0)=q(1)$ y $0\neq 1$. Pero sí es un monomorfismo. En efecto, sea $G$ otro grupo divisible y sean $f.g\in\Hom{\Cat{Div}}{G}{\mathbb{Q}}$ tales que $q\circ f = q\circ g$ (la composición de morfismos en esta categoría coincide con la composición usual de funciones), debemos probar que $f=g$.

        Como $q\circ f = q\circ g$ entonces, $q\circ(f-g)=[0]$ donde $[0]$ denota al morfismo que a cada elemento de $G$ lo envía a la clase del $0$, $[0]$. Denotemos por $h=f-g$, se sigue entonces que $q\circ h=[0]$. Entonces, para cualquier $x\in G$ tenemos que $q(h(x))=[0]$, por tanto $h(x)\in\mathbb{Z}$.
        
        Para probar el resultado basta con ver que $h(x)=0$ para todo $x\in G$. Procederemos por contradicción. Suponga que existe $x_0\in G$ tal que $h(x_0)\neq 0$, podemos asumir que $h(x_0)\in\mathbb{Z}^+$ (en caso de que no sea así, tomamos $-x_0\in G$ para el cual se cumple que $h(-x_0)=-h(x_0)\in\mathbb{Z}^+$). Como estamos trabajando con grupos divisibles, para $2h(x_0) \in\mathbb{N}$ y $x_0\in G$ existe $y_0\in G$ tal que
        \begin{equation*}
            \begin{split}
                x_0=&2h(x_0)y_0\\
                h(x_0)=&2h(x_0)h(y_0)\\
            \end{split}
        \end{equation*}
        como $h(x_0)\neq 0$, entonces $h(y_0)\neq 0$, luego lo anterior no puede suceder ya que debería suceder que $h(y_0)=\frac{1}{2}$, pero $h(y_0)\in\mathbb{Z}$. Luego tal $x_0$ no puede existir. Así, $h(x)=0$ para todo $x\in G$, lo cual prueba el resultado. Esto prueba que, de hecho, $q$ es un monomorfismo que no es inyectivo.
    \end{sol}

    \begin{exa}
        En la categoría $\Cat{Ring}^c$ de los anillos conmutativos con identidad, el mapeo inclusión $\cf{i}{\mathbb{Z}}{\mathbb{Q}}$ no es suprayectivo, pero sí es un epimorfismo.
    \end{exa}

    \begin{sol}
        
    \end{sol}
    


\end{document}