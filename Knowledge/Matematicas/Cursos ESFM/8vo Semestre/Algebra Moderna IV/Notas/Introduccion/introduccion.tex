\documentclass[12pt]{report}
\usepackage[spanish]{babel}
\usepackage[utf8]{inputenc}
\usepackage{amsmath}
\usepackage{amssymb}
\usepackage{amsthm}
\usepackage{graphics}
\usepackage{subfigure}
\usepackage{lipsum}
\usepackage{array}
\usepackage{multicol}
\usepackage{enumerate}
\usepackage[framemethod=TikZ]{mdframed}
\usepackage[a4paper, margin = 1.5cm]{geometry}

%En esta parte se hacen redefiniciones de algunos comandos para que resulte agradable el verlos%

\renewcommand{\theenumii}{\roman{enumii}}

\def\proof{\paragraph{Demostración:\\}}
\def\endproof{\hfill$\blacksquare$}

\def\sol{\paragraph{Solución:\\}}
\def\endsol{\hfill$\square$}

%En esta parte se definen los comandos a usar dentro del documento para enlistar%

\newtheoremstyle{largebreak}
  {}% use the default space above
  {}% use the default space below
  {\normalfont}% body font
  {}% indent (0pt)
  {\bfseries}% header font
  {}% punctuation
  {\newline}% break after header
  {}% header spec

\theoremstyle{largebreak}

\newmdtheoremenv[
    leftmargin=0em,
    rightmargin=0em,
    innertopmargin=-2pt,
    innerbottommargin=8pt,
    hidealllines = true,
    roundcorner = 5pt,
    backgroundcolor = gray!60!red!30
]{exa}{Ejemplo}[section]

\newmdtheoremenv[
    leftmargin=0em,
    rightmargin=0em,
    innertopmargin=-2pt,
    innerbottommargin=8pt,
    hidealllines = true,
    roundcorner = 5pt,
    backgroundcolor = gray!50!blue!30
]{obs}{Observación}[section]

\newmdtheoremenv[
    leftmargin=0em,
    rightmargin=0em,
    innertopmargin=-2pt,
    innerbottommargin=8pt,
    rightline = false,
    leftline = false
]{theor}{Teorema}[section]

\newmdtheoremenv[
    leftmargin=0em,
    rightmargin=0em,
    innertopmargin=-2pt,
    innerbottommargin=8pt,
    rightline = false,
    leftline = false
]{propo}{Proposición}[section]

\newmdtheoremenv[
    leftmargin=0em,
    rightmargin=0em,
    innertopmargin=-2pt,
    innerbottommargin=8pt,
    rightline = false,
    leftline = false
]{cor}{Corolario}[section]

\newmdtheoremenv[
    leftmargin=0em,
    rightmargin=0em,
    innertopmargin=-2pt,
    innerbottommargin=8pt,
    rightline = false,
    leftline = false
]{lema}{Lema}[section]

\newmdtheoremenv[
    leftmargin=0em,
    rightmargin=0em,
    innertopmargin=-2pt,
    innerbottommargin=8pt,
    roundcorner=5pt,
    backgroundcolor = gray!30,
    hidealllines = true
]{mydef}{Definición}[section]

\newmdtheoremenv[
    leftmargin=0em,
    rightmargin=0em,
    innertopmargin=-2pt,
    innerbottommargin=8pt,
    roundcorner=5pt
]{excer}{Ejercicio}[section]

%En esta parte se colocan comandos que definen la forma en la que se van a escribir ciertas funciones%

\newcommand\abs[1]{\ensuremath{\biglvert#1\bigrvert}}
\newcommand\divides{\ensuremath{\bigm|}}
\newcommand\cf[3]{\ensuremath{#1:#2\rightarrow#3}}
\newcommand\contradiction{\ensuremath{\#_c}}
\newcommand{\Obj}[1]{\ensuremath{\textup{Obj}\left(#1\right)}}
\newcommand{\Hom}[3]{\ensuremath{\textup{Hom}_{#1}\left(#2,#3\right)}}

%recuerda usar \clearpage para hacer un salto de página

\begin{document}
    \title{Notas de Álgebra Moderna IV.
    
    Una introducción a la teoría de categorías.}
    \author{Cristo Daniel Alvarado}
    \maketitle

    \tableofcontents %Con este comando se genera el índice general del libro%

    %\setcounter{chapter}{3} %En esta parte lo que se hace es cambiar la enumeración del capítulo%

    \chapter{Clases y conjuntos}

    \section{Axiomas de Von-Newmann-Gödel}

    Antes de decantarnos totalmente a nuestro estudio de las categorías, primero nos enfocaremos en estudiar a los objetos que se van a usar (las clases).

    Aceptamos la existencia de \textit{objetos primitivos}, las cuales son clases y conjuntos, dotadas de dos relaciones primitivas, la pertenencia $\in$ e igualdad $=$. Denotamos a los objetos primitivos por letras en mayúsculas.

    \begin{mydef}[\textbf{Axiomas de NBG}]
        \renewcommand{\theenumi}{A\arabic{enumi}}
        Se tienen los siguientes axiomas:
        \begin{enumerate}
            \item Todo conjunto es una clase.
            \item Si $x\in A$, $\forall x\in B$ y $x\in B$, $\forall x\in A$, entonces $A=B$.
            \item Si $A\in B$ donde $B$ es una clase, entonces $A$ es un conjunto.
            \item Si $P(x)$ es una propiedad definida sobre el parámetro $x$ que se recorre sobre conjuntos, entonces existe una clase $[x\big| P(x)]$ tal que para cada conjunto $y$.
            \begin{equation*}
                \begin{split}
                    y\in[x\big| P(x)]\iff P(y)
                \end{split}
            \end{equation*}
            \item Si $X,Y$ son conjuntos, entonces $[X,Y]$ es un conjunto y se denota por $\left\{X, Y\right\}$ (ver ejemplo 1.1.3).
            \item Si $X$ es un conjunto, entonces $\left\{X\right\}$, $\left\{X,\left\{X\right\} \right\}$,... son conjuntos.
            \item Existe un conjunto inductivo.
            \item Sea $A$ conjunto, entonces existe un conjunto denotado por $\mathcal{P}(A)$ tal que $B\in\mathcal{P}(A)$ si y sólo si $B\subseteq A$.
            \item Si $\cf{f}{A}{B}$ donde $A$ es un conjunto, entonces $f(A)$ es un conjunto.
        \end{enumerate}
    \end{mydef}

    \begin{exa}
        Construimos al \textbf{conjunto vacío} $\emptyset$ como $\emptyset=[x\big| x\neq x]$ (usando a A4).
    \end{exa}
    
    \begin{exa}
        $\textup{Set}=[x| x=x]$ (usando a A4).
    \end{exa}

    \begin{exa}
        Si $X$ y $Y$ son conjuntos, entonces
        \begin{equation*}
            [X,Y]=[Z\big| Z=X\textup{ o }Z=Y]
        \end{equation*}
        (construida por el A4).
    \end{exa}

    \begin{exa}
        Si $X$ es un conjunto, entonces $X\cup\left\{X\right\}$ es un conjunto y se denomina el \textbf{sucesor de $X$}.
    \end{exa}

    \begin{mydef}
        Sea $A$ una clase. Se define
        \begin{equation*}
            \bigcup A=\bigcup_{X\in A}X=[x\big| \exists X\in A\textup{ tal que }x\in X]
        \end{equation*}
        Si $A$ es un conjunto, $\bigcup A$ es un conjunto.
    \end{mydef}

    \begin{mydef}
        Un conjunto $A$ se denomina \textbf{inductivo} si
        \renewcommand{\theenumi}{\roman{enumi}}
        \begin{enumerate}
            \item $\emptyset\in A$.
            \item $X\in A\Rightarrow X\cup\left\{X\right\}\in A$.
        \end{enumerate}
    \end{mydef}

    \begin{propo}
        $\emptyset$ es un conjunto.
    \end{propo}

    \begin{proof}
        Sea $A$ un conjunto inductivo (el cual existe por A7), entonces $\emptyset\in A$, luego por A3, $\emptyset$ es un conjunto.
    \end{proof}

    \begin{mydef}
        Se dice que $B$ es subclase de $A$, si $x\in A$ para todo $x\in B$, y se denota por $B\subseteq A$.
    \end{mydef}

    \begin{propo}
        Si $B\subseteq A$ y $A$ es conjunto, entonces $B$ es conjunto.
    \end{propo}

    \begin{proof}
        Como $B\subseteq A$, entonces $B\in\mathcal{P}(A)$, luego $B$ por A3, $B$ es conjunto.
    \end{proof}

    Esta proposición es necesaria pues no sabemos si las subclases de conjuntos son conjuntos.

    \begin{mydef}
        Si $x,y$ son conjuntos, se define:
        \begin{equation*}
            (x,y)=\left\{\left\{x\right\},\left\{x,y\right\} \right\}
        \end{equation*}
        Si $A$ y $B$ son clases, se define
        \begin{equation*}
            A\times B = [(x,y)\big| x\in A\textup{ y }y\in B ]
        \end{equation*}
    \end{mydef}

    \begin{excer}
        Si $A$ y $B$ son conjuntos, entonces $A\times B$ es conjunto.
    \end{excer}

    \begin{proof}
        
    \end{proof}

    \begin{mydef}
        Una función de $A$ en $B$ es una subclase $F\subseteq A\times B$ tal que $(x,y),(x,z)\in F\Rightarrow y=z$.
    \end{mydef}

    \begin{exa}
        $\textup{Set}$ no es un conjunto.
    \end{exa}

    \begin{proof}
        Supóngase que Set es un conjunto. Sea
        \begin{equation*}
            X=[x\big| x\notin x]
        \end{equation*}
        Si $x\in X$, entonces $x$ es un conjunto (por A3) luego $x\in\textup{Set}$, es decir que $x$ es un conjunto. Por tanto, $X\subseteq\textup{Set}$, esto es que $X$ es un conjunto. Luego sucede que $X\in X$ o $X\notin X$ (por como se formó la clase $X$ a partir de A4).
        
        Por ende, $X\in X\iff X\notin X$\contradiction. Luego $\textup{Set}$ no es un conjunto.
    \end{proof}

    \begin{exa}
       Denotamos por $\mathcal{G}=[G| G\textup{ es grupo}]$, y $\mathcal{S}=[S_X\big| X\in\textup{Set}]$. Si sucediera que $S$ fuese conjunto, tomando $\cf{f}{\mathcal{S}}{\textup{Set}}$, $S_X\mapsto X$ es una función, luego $F(\mathcal{S})=\textup{Set}$ es un conjunto, lo cual no puede ser. Por tanto, como $\mathcal{S}\subseteq\mathcal{G}$, se sigue que $\mathcal{G}$ es clase.
    \end{exa}

    \chapter{Categorias}

    \section{Conceptos Fundamentales}

    Antes de comenzar aceptaremos como válido al siguiente axioma:

    \renewcommand{\theenumi}{A\arabic{enumi}}

    \begin{enumerate}
        \setcounter{enumi}{9}
        \item \textbf{Limitación de tamaño}. Una clase es un conjunto si y sólo si no es biyectivo con $\textup{Set}$.
    \end{enumerate}

    Ahora si con la parte de categorías.
    
    \renewcommand{\theenumi}{\arabic{enumi}}

    \begin{mydef}
        Una \textbf{categoría} $\mathcal{C}$ consta de lo siguiente:
        \begin{enumerate}
            \item Una clase $\textup{Obj}(\mathcal{C})$ cuyos elementos son llamados \textbf{objetos}.
            \item Para cada par $A,B\in\Obj{\mathcal{C}}$ existe un conjunto $\Hom{\mathcal{C}}{A}{B}$ cuyos elementos llamaremos morfismos y, dado un morfismo $f\in\Hom{\mathcal{C}}{A}{B}$ lo denotaremos por $\cf{f}{A}{B}$.
            \item Para cada objeto $A\in\Obj{\mathcal{C}}$ hay un morfismo $1_A\in\Hom{\mathcal{C}}{A}{A}$ llamado la \textbf{identidad de $A$}.
            \item Hay una ley de composición para una terna de objetos $A$, $B$ y $C$:
            \begin{equation*}
                \begin{split}
                    \Hom{\mathcal{C}}{A}{B}\times\Hom{\mathcal{C}}{B}{C}\rightarrow&\Hom{\mathcal{C}}{A}{C}\\
                    (f,g)\mapsto&g\circ f\\
                \end{split}
            \end{equation*}
            que satisface lo siguiente:
            \begin{enumerate}
                \item (\textit{Asociatividad}). Dado $f\in\Hom{\mathcal{C}}{A}{B}$ y $g\in\Hom{\mathcal{C}}{B}{C}$ y $h\in\Hom{\mathcal{C}}{C}{D}$ se cumple que:
                \begin{equation*}
                    h\circ (g\circ f)=(h\circ g)\circ f
                \end{equation*}
                \item Dado un morfismo $f\in\Hom{\mathcal{C}}{A}{B}$, se tiene que:
                \begin{equation*}
                    f\circ 1_A=f=1_B\circ f
                \end{equation*}
            \end{enumerate}
        \end{enumerate}
    \end{mydef}

    \begin{mydef}
        Si la clase de objetos de la categoría $\mathcal{C}$ es un conjunto, diremos que $\mathcal{C}$ es una \textbf{categoría pequeña}. Más aún, si tenemos un número finito de morfismos, diremos que $\mathcal{C}$ es una \textbf{categoría finita}.
    \end{mydef}

    Dadas las definciones anteriores, no se nos da ejemplos concretos de lo que es una categoría, por lo cual procederemos a dar ejemplos de la misma.

    \begin{exa}
        Sea $X$ un conjunto. Denotamos por $\mathcal{C}_X$ a una categoría formada por $\Obj{\mathcal{C}_X}=X$, y tendremos para cualquier par de elementos $x,y\in\Obj{\mathcal{C}_X}$ definimos:
        \begin{equation*}
            \Hom{\mathcal{C}_X}{x}{y}=\left\{
                \begin{array}{lcr}
                    \emptyset & \textup{ si } & x\neq y\\
                    1_x & \textup{ si } & x=y\\
                \end{array}
            \right.
        \end{equation*}
    \end{exa}

    \begin{exa}
        Definimos a $n$ por la categoría de un conjunto con $n$ elementos, donde $n\in\mathbb{N}$.
    \end{exa}
    
\end{document}