\documentclass[12pt]{report}
\usepackage[spanish]{babel}
\usepackage[utf8]{inputenc}
\usepackage{amsmath}
\usepackage{amssymb}
\usepackage{amsthm}
\usepackage{graphics}
\usepackage{subfigure}
\usepackage{lipsum}
\usepackage{array}
\usepackage{multicol}
\usepackage{enumerate}
\usepackage[framemethod=TikZ]{mdframed}
\usepackage[a4paper, margin = 1.5cm]{geometry}

%En esta parte se hacen redefiniciones de algunos comandos para que resulte agradable el verlos%

\renewcommand{\theenumii}{\roman{enumii}}

\def\proof{\paragraph{Demostración:\\}}
\def\endproof{\hfill$\blacksquare$}

\def\sol{\paragraph{Solución:\\}}
\def\endsol{\hfill$\square$}

%En esta parte se definen los comandos a usar dentro del documento para enlistar%

\newtheoremstyle{largebreak}
  {}% use the default space above
  {}% use the default space below
  {\normalfont}% body font
  {}% indent (0pt)
  {\bfseries}% header font
  {}% punctuation
  {\newline}% break after header
  {}% header spec

\theoremstyle{largebreak}

\newmdtheoremenv[
    leftmargin=0em,
    rightmargin=0em,
    innertopmargin=-2pt,
    innerbottommargin=8pt,
    hidealllines = true,
    roundcorner = 5pt,
    backgroundcolor = gray!60!red!30
]{exa}{Ejemplo}[section]

\newmdtheoremenv[
    leftmargin=0em,
    rightmargin=0em,
    innertopmargin=-2pt,
    innerbottommargin=8pt,
    hidealllines = true,
    roundcorner = 5pt,
    backgroundcolor = gray!50!blue!30
]{obs}{Observación}[section]

\newmdtheoremenv[
    leftmargin=0em,
    rightmargin=0em,
    innertopmargin=-2pt,
    innerbottommargin=8pt,
    rightline = false,
    leftline = false
]{theor}{Teorema}[section]

\newmdtheoremenv[
    leftmargin=0em,
    rightmargin=0em,
    innertopmargin=-2pt,
    innerbottommargin=8pt,
    rightline = false,
    leftline = false
]{propo}{Proposición}[section]

\newmdtheoremenv[
    leftmargin=0em,
    rightmargin=0em,
    innertopmargin=-2pt,
    innerbottommargin=8pt,
    rightline = false,
    leftline = false
]{cor}{Corolario}[section]

\newmdtheoremenv[
    leftmargin=0em,
    rightmargin=0em,
    innertopmargin=-2pt,
    innerbottommargin=8pt,
    rightline = false,
    leftline = false
]{lema}{Lema}[section]

\newmdtheoremenv[
    leftmargin=0em,
    rightmargin=0em,
    innertopmargin=-2pt,
    innerbottommargin=8pt,
    roundcorner=5pt,
    backgroundcolor = gray!30,
    hidealllines = true
]{mydef}{Definición}[section]

\newmdtheoremenv[
    leftmargin=0em,
    rightmargin=0em,
    innertopmargin=-2pt,
    innerbottommargin=8pt,
    roundcorner=5pt
]{excer}{Ejercicio}[section]

%En esta parte se colocan comandos que definen la forma en la que se van a escribir ciertas funciones%

\newcommand\abs[1]{\ensuremath{\big|#1\big|}}
\newcommand\divides{\ensuremath{\bigm|}}
\newcommand\cf[3]{\ensuremath{#1:#2\rightarrow#3}}
\newcommand\contradiction{\ensuremath{\#_c}}
\newcommand{\Obj}[1]{\ensuremath{\textup{Obj}\left(#1\right)}}
\newcommand{\Hom}[3]{\ensuremath{\textup{Hom}_{#1}\left(#2,#3\right)}}

\newcommand{\Cat}[1]{\ensuremath{\textup{\textbf{#1}}}}
\newcommand{\Iso}[3]{\ensuremath{\textup{I}_{\textup{SO}_{#1}}\left(#2,#3\right)}}
\newcommand{\SO}[1]{\ensuremath{\textup{SO}\left(#1\right)}}
\newcommand{\Quo}[1]{\ensuremath{\textup{Quo}\left(#1 \right)}}

%recuerda usar \clearpage para hacer un salto de página

\begin{document}
    \setlength{\parskip}{5pt} % Añade 5 puntos de espacio entre párrafos
    \setlength{\parindent}{12pt} % Pone la sangría como me gusta
    \title{Notas de Álgebra Moderna IV.
    
    Una introducción a la teoría de categorías.}
    \author{Cristo Daniel Alvarado}
    \maketitle

    \tableofcontents %Con este comando se genera el índice general del libro%

    \setcounter{chapter}{2} %En esta parte lo que se hace es cambiar la enumeración del capítulo%
    
    \chapter{Funtores}

    \section{Conceptos Fundamentales}
    
    \begin{mydef}
        Sean $\mathcal{C}$ y $\mathcal{D}$ dos categorías. Un \textbf{funtor covariante} (respectivamente, \textbf{funtor contravariante}), denotado por $\cf{F}{\mathcal{C}}{\mathcal{D}}$, consta de
        \begin{enumerate}
            \item Un mapeo $\cf{F}{\Obj{\mathcal{C}}}{\mathcal{D}}$ tal que $A\mapsto F(A)$.
            \item Para cualesquier dos pares de objetos $A,B\in\Obj{\mathcal{C}}$, un mapeo $\cf{F}{\Hom{\mathcal{C}}{A}{B}}{\Hom{\mathcal{D}}{F(A)}{F(B)}}$ (resp. $\cf{F}{\Hom{\mathcal{C}}{A}{B}}{\Hom{\mathcal{D}}{F(B)}{F(A)}}$) tal que $f\mapsto F(f)$, que cumple las condiciones siguientes:
            \begin{enumerate}
                \item Para cada $A\in\Obj{\mathcal{C}}$, $F(1_A)=1_{F(A)}$.
                \item Para cada $f\in\Hom{\mathcal{C}}{A}{B}$ y $g\in\Hom{\mathcal{C}}{B}{C}$, se tiene que
                \begin{equation*}
                    F(g\circ f)=F(g)\circ F(f)
                \end{equation*}
                (resp. $F(g\circ f)=F(f)\circ F(g)$).
            \end{enumerate}
        \end{enumerate}
        Un \textbf{bifuntor} es un funtor que va del producto de dos categorías en una categoría.
    \end{mydef}

    \begin{mydef}
        La \textbf{imagen de un funtor $F$ entre las categorías $\mathcal{C}$ y $\mathcal{D}$}, consta de una clase $\left\{F(C)\Big|C\in\Obj{\mathcal{C}} \right\}$ junto con todos los conjuntos $\left\{F(f)\Big|f\in\Hom{\mathcal{C}}{A}{B}\textup{ con }A,B\in\Obj{\mathcal{C}} \right\}$.
    \end{mydef}

    \begin{obs}
        La imagen de un funtor no necesariamente es una categoría. En cambio, si el funtor es inyectivo sobre objetos, se tiene que la imagen de un funtor si es una categoría.
    \end{obs}

    \begin{proof}
        En efecto, sean $\mathcal{C}$ y $\mathcal{D}$ dos categorías. Para cualesquiera $C_1,C_2,C_3,C_4\in\Obj{\mathcal{C}}$ y $D_1,D_2,D_3\in\Obj{\mathcal{D}}$, $f\in\Hom{\mathcal{C}}{C_1}{C_2}$ y $g\in\Hom{\mathcal{C}}{C_3}{C_4}$, $h\in\Hom{\mathcal{D}}{D_1}{D_2}$ y $k\in\Hom{\mathcal{D}}{D_2}{D_3}$. Se tiene lo siguiente:
        \begin{equation*}
            \begin{split}
                C_1\longrightarrow C_2\textup{ y }C_3\longrightarrow C_4\\
                D_1\longrightarrow D_2\longrightarrow D_3\\
            \end{split}
        \end{equation*}
        la imagen de $\cf{F}{\mathcal{C}}{\mathcal{D}}$ noes una categoría, pues si hacemos que
        \begin{equation*}
            F(C_1)=D_1, F(C_2)=F(C_3)=D_2\textup{ y }F(C_4)=D_4
        \end{equation*}
        haciendo
        \begin{equation*}
            F(f)=h,F(g)=k
        \end{equation*}
        además,
        \begin{equation*}
            F(1_{C_1})=1_{D_1}\quad F(1_{C_2})=F(1_{C_3})=1_{D_2}\quad F(1_{C_4})=1_{D_3} 
        \end{equation*}
        pues, $h$ y $k$ paretenecen a la imagen de $F$, pero su composición no lo está.
    \end{proof}

    \begin{obs}
        Si $F$ es inyectiva, entonces la imagen de $F$ será una categoría.
    \end{obs}

    \begin{propo}
        Si $\cf{F}{\mathcal{C}}{\mathcal{D}}$ es un funtor covariante (resp. contravariante), entonces $\cf{F'}{\mathcal{C}^{op}}{\mathcal{D}}$ es un funtor contravariante (resp. covariante).
    \end{propo}

    \begin{proof}
        Se hará el segundo caso. Sean $\mathcal{C}$ y $\mathcal{D}$ dos categorías y $\cf{F}{\mathcal{C}}{\mathcal{D}}$ un funtor contravariante. Definimos $F'$ tal que
        \begin{equation*}
            F(A)=F'(A),\quad\forall A\in\Obj{\mathcal{C}^{op}}
        \end{equation*}
        y,
        \begin{equation*}
            F(1_A)=F'(1_A),\quad\forall A\in\Obj{\mathcal{C}^{op}}
        \end{equation*}
        \begin{enumerate}
            \item Sea $f$ un morfismo de $A$ en $B$, con $A,B\in\Obj{\mathcal{C}^{op}}=\Obj{\mathcal{C}}$. Entonces su respectivo elemento $f^{op}$ el $\mathcal{C}$
            \begin{equation*}
                F(f)=F'(f^{op})
            \end{equation*}
            Claramente esto está bien definido. Con lo cual se tiene que $\cf{F'(f^{op})}{F(B)}{F(A)}$, que es la primera parte para probar que $F$ es funtor covariante.
            \item Sean $f^{op}\in\Hom{\mathcal{C}^{op}}{B}{A}$ y $g^{op}\in\Hom{\mathcal{C}^{op}}{C}{B}$. Entonces, existen $f\in\Hom{\mathcal{C}}{A}{B}$ y $g\in\Hom{\mathcal{C}}{B}{C}$. Se tiene entonces que
            \begin{equation*}
                \begin{split}
                    F'(f^{op}\circ g^{op})&=F'((g\circ f)^{op})\\
                    &=F(g\circ f)\\
                    &=F(f)\circ F(g)\\
                    &=F'(f^{op})\circ F'(g^{op})\\
                \end{split}
            \end{equation*}
        \end{enumerate}
        por tanto, de los dos incisos anteriores se sigue que $F'$ es un funtor covariante.
    \end{proof}

    \begin{mydef}
        Sea $\mathcal{C}$ una categoría.
        \begin{enumerate}
            \item Si $\mathcal{C}'$ es una subcategoría de $\mathcal{C}$, definimos el \textbf{funtor inclusión} $\cf{I}{\mathcal{C}'}{\mathcal{C}}$ el cual asigna a cada objeto y cada morfismo a sí mismo. En el caso que $\mathcal{C}'=\mathcal{C}$ ,tendremos simplemente que $\cf{I}{\mathcal{C}}{\mathcal{C}}$ es el \textbf{funtor identidad}, denotado por $1_{\mathcal{C}}$.
            \item Sea $\sim$ una congruencia en $\mathcal{C}$ categoría y, $\mathcal{C}/\sim$ la categoría cociente correspondiente. Se define el \textbf{funtor cociente}, denotado por $\cf{\pi}{\mathcal{C}}{\mathcal{C}/\sim}$ de la siguiente manera:
            \begin{itemize}
                \item $\pi(C)=C$ para todo $C\in\Obj{\mathcal{C}}$.
                \item $\pi(f)=\overline{f}$ para todo morfismo $f$ en la categoría $\mathcal{C}$.
            \end{itemize}
            donde $\overline{f}$ denota a la clase de equivalencia de los morfismos de $f$ en $\mathcal{C}$.

            Además, si $\cf{f}{A}{B}$ y $\cf{g}{B}{C}$ son morfismos en $\mathcal{C}$, con $A,B\in\Obj{\mathcal{C}}$, se tiene que
            \begin{equation*}
                \pi(g\circ f)=\overline{g\circ f}=\overline{g}\circ\overline{f}=\pi(g)\circ\pi(f)
            \end{equation*}
            \item Si $\cf{F}{\mathcal{C}}{\mathcal{D}}$ y $\cf{G}{\mathcal{D}}{\mathcal{E}}$ son dos funtores, siendo $\mathcal{C},\mathcal{D},\mathcal{E}$ categorías, podemos definirl el \textbf{funtor composición puntual} $\cf{G\circ F}{\mathcal{C}}{\mathcal{E}}$ como sigue:
            \begin{equation*}
                G\circ F(C)=G(F(C)),\quad\forall C\in\Obj{\mathcal{C}}
            \end{equation*}
            y,
            \begin{equation*}
                G\circ F(f)=G(F(f)),\quad\forall f\textup{ morfismo en }\mathcal{C}
            \end{equation*}
            \begin{enumerate}
                \item Si $F$ y $G$ son ambos covariantes ó contravariantes, entonces $G\circ F$ es un funtor covariante.
                \item Si uno de ellos es covariante y el otro contravariante, entonces $G\circ F$ es contravariante. 
            \end{enumerate}
        \end{enumerate}
    \end{mydef}

    \begin{proof}
        Verifiquemos en 3 que es un funtor. En efecto, claramente manda objetos en objetos y morfismos en morfismos de $\mathcal{C}$ en $\mathcal{E}$.
        
        Suponga que $F$ y $G$ son ambos contravariantes (el caso en el que son covariantes es inmediato). Entonces para $\cf{F}{A}{B}$ morfismo en $\mathcal{C}$:
        \begin{equation*}
            \begin{split}
                &\cf{F(f)}{F(B)}{F(A)}\\
                \Rightarrow &\cf{G(F(f))}{G(F(A))}{G(F(B))}\\
            \end{split}
        \end{equation*}
        además, si $f,g$ son morifsmos en $\mathcal{C}$:
        \begin{equation*}
            \begin{split}
                G\circ F(g\circ f)&=G(F(g\circ f))\\
                &=G(F(f)\circ F(g))\\
                &=G(F(g))\circ G(F(f))\\
            \end{split}
        \end{equation*}
        luego, el funtor composición puntual es covariante.

        Para el caso en el que uno sea covariante y otro contravariante, el caso es similar.
    \end{proof}

    \begin{mydef}
        Sean $\mathcal{C}$ y $\mathcal{D}$ dos categorías.
        \begin{enumerate}
            \item Fijemos un objeto $D_0\in\Obj{\mathcal{D}}$. Definimos el \textbf{funtor constante}, $\cf{A_{D_0}}{\mathcal{D}}{\mathcal{C}}$ el cual asigna a cada objeto $C\in\Obj{\mathcal{C}}$ al objeto $D_0$ y a cada morfismo $f$ de $\mathcal{C}$ el morfismo $1_{D_0}$ de $\mathcal{D}$.
            \item Considere la categoría producto $\mathcal{C}\times\mathcal{D}$. Definimos los \textbf{funtores proyección} de la siguiente mantera:
            \begin{equation*}
                \begin{split}
                    \cf{\rho_{\mathcal{C}}}{\mathcal{C}\times\mathcal{D}}{\mathcal{C}}, &\quad(C,D)\mapsto C\quad (f,g)\mapsto f \\
                    \cf{\rho_{\mathcal{D}}}{\mathcal{C}\times\mathcal{D}}{\mathcal{D}}, &\quad(C,D)\mapsto D\quad (f,g)\mapsto g \\
                \end{split}
            \end{equation*}
        \end{enumerate}
    \end{mydef}

    \newcommand{\HOM}[1]{\ensuremath{\textup{HOM}_{#1}}}

    \begin{mydef}
        Sea $\mathcal{C}$ una categoría. Definimos el \textbf{bifuntor} $\cf{\HOM{\mathcal{C}}}{\mathcal{C}^{op}\times \mathcal{C}}{\Cat{Set}}$ como $\HOM{\mathcal{C}}(A,B)=\Hom{\mathcal{C}}{A}{B}$, para todo $(A,B)\in\mathcal{C}^{op}\times\mathcal{C}$.
        
        Además, si $(f^{op},g)\in\Hom{\mathcal{C}^{op}\times\mathcal{C}}{(A,B)}{(C,D)}$, entonces se tiene que $f\in\Hom{\mathcal{C}}{C}{A}$ y $g\in\Hom{\mathcal{C}}{B}{D}$.

        Luego
        \begin{equation*}
            \begin{split}
                \HOM{\mathcal{C}}:\Hom{\mathcal{C}^{op}\times\mathcal{C}}{(A,B)}{(C,D)}&\rightarrow \Hom{\Cat{Set}}{\HOM{\mathcal{C}(A,B)}}{\HOM{\mathcal{C}(C,D)}}\\
                (f^{op},g)&\mapsto \HOM{\mathcal{C}}(f^{op},g)
            \end{split}
        \end{equation*}
        donde $\HOM{\mathcal{C}}(f^{op},g)(h)=g\circ h\circ f$, para todo $h\in\Hom{\mathcal{C}}{A}{B}$.
    \end{mydef}

    \begin{proof}
        Probaremos que $\HOM{\mathcal{C}}$ es un funtor. Hay que ver que se cumplen algunas condiciones:
        \begin{enumerate}
            \item $\cf{\HOM{\mathcal{C}}}{\mathcal{C}^{op}\times\mathcal{C}}{\Cat{Set}}$, tal que $(A,B)\mapsto\Hom{\mathcal{C}}{A}{B}$, y $(f^{op},g)\mapsto (h\mapsto g\circ h\circ f)$ para todo $h\in\Hom{\mathcal{C}}{A}{B}$ (siendo los $f$ y $g$ como se dió en la definición). Así, está bien definido.
            \item Sean $(A,B)\in\mathcal{C}^{op}\times\mathcal{C}$, $h\in\Hom{\mathcal{C}}{A}{B}$, entonces:
            \begin{equation*}
                \begin{split}
                    \HOM{\mathcal{C}}((1_A,1_B))(h)&=\HOM{\mathcal{C}}((1^{op}_A,1_B))(h)\\
                    &=1_b\circ g\circ 1_A\\
                    &=h\\
                    &=1_{ \Hom{\mathcal{C}}{A}{B}}(h)\\
                    &=1_{\HOM{\mathcal{C}}(A,B)(h)}
                \end{split}
            \end{equation*}
            luego manda identidades en identidades.
            \item Sean ahora $(f^{op},g)\in\Hom{\mathcal{C}^{op}\times\mathcal{C}}{(A,B)}{(C,D)}$ y $(r^{op},t)\in\Hom{\mathcal{C}^{op}\times\mathcal{C}}{(C,D)}{(E,F)}$. Tomemos $h\in\Hom{\mathcal{C}}{A}{B}$.
            
            Se tiene que $\HOM{\mathcal{C}}(r^{op},t)\in\Hom{\Cat{Set}}{\Hom{\mathcal{C}}{A}{B}}{\Hom{\mathcal{C}}{E}{F}}$ y $g\circ h\circ f\in \Hom{\mathcal{C}}{C}{D}$, de manera que
            \begin{equation*}
                \begin{split}
                    \HOM{\mathcal{C}}(r^{op},t)(g\circ h\circ f)&=t\circ (g\circ h\circ f)\circ r\\                    
                \end{split}
            \end{equation*}
            luego,
            \begin{equation*}
                \begin{split}
                    \HOM{\mathcal{C}}((r^{op},t)\circ (f^{op},g))(h)
                    &=\HOM{\mathcal{C}}(r^{op}\circ f^{op},t\circ g)(h)\\
                    &=\HOM{\mathcal{C}}((f\circ r)^{op},t\circ g)(h)\\
                    &=(t\circ g)\circ h\circ (f\circ r)\\
                    &=\HOM{\mathcal{C}}(r^{op}, t)(g\circ h\circ f)\\
                    &=\HOM{\mathcal{C}}(r^{op},t)\circ\HOM{\mathcal{C}}(f^{op},g)(h)\\
                    \Rightarrow \HOM{\mathcal{C}}((r^{op},t)\circ (f^{op},g))&=\HOM{\mathcal{C}}(r^{op},t)\circ\HOM{\mathcal{C}}(f^{op},g)(h)\\
                \end{split}
            \end{equation*}
        \end{enumerate}
        Por los incisos anteriores, se sigue que este es un bifuntor, denominado \textbf{bifuntor Hom}.
    \end{proof}

    \begin{exa}
        Los funtores olvidados (o \textit{forgetful}). Consideremos por ejemplo a las categorías $\Cat{Rng}$ y $\Cat{Ab}$. Entonces:
        \begin{enumerate}
            \item $\cf{F}{\Cat{Ring}}{\Cat{Ab}}$ tal que $(A,+,\cdot)\mapsto (A,+)$ y a cada morfismo lo manda a sí mismo es un funtor, el cual "olvida" propiedades del objeto original del que partió. En este caso, olvida el producto entre dos elementos del anillo.
            \item $\Cat{Grp}\rightarrow\Cat{Set}$. A cada grupo lo manda al conjunto de sus elementos y morfismo a la función entre grupos.
        \end{enumerate}
    \end{exa}

    \begin{propo}
        Las categorías pequeñas y los funtores entre ellas forman una categoría (localmente pequeña), la cual denotamos como $\Cat{Cat}$, definiendo:
        \begin{enumerate}
            \item $\Obj{\Cat{Cat}}=\left\{\mathcal{C}\Big|\mathcal{C}\textup{ es una categoría pequeña} \right\}$.
            \item Para cada par de objetos $\mathcal{C},\mathcal{D}\in\Obj{\Cat{Cat}}$, el conjunto:
            \begin{equation*}
                \Hom{\Cat{Cat}}{\mathcal{C}}{\mathcal{D}}=\left\{F\Big|\cf{F}{\mathcal{C}}{\mathcal{D}}\textup{ es un funtor} \right\}
            \end{equation*}
            (como $\mathcal{C}$ y $\mathcal{D}$ son conjuntos, entonces la clase anterior debe ser un conjunto).
            \item La composición de morfismos en esta categoría se define como la composición funtorial puntual de dos funtores. Además, la identidad es el funtor identidad de la misma categoría.
        \end{enumerate}
    \end{propo}

    \begin{proof}
        Basta con ver que la composición es asociativa en la categoría $\Cat{Cat}$. En efecto, sean $\mathcal{C},\mathcal{D},\mathcal{E},\mathcal{T}$ categorías pequeñas y, $\cf{F}{\mathcal{C}}{\mathcal{D}}$, $\cf{G}{\mathcal{D}}{\mathcal{E}}$ y $\cf{H}{\mathcal{E}}{\mathcal{T}}$ funtores. Se tiene que:
        \begin{equation*}
            \begin{split}
                (H\circ (G\circ F))(A)&=H(G\circ F(A))\\
                &=(H\circ G) (F(A))\\
                &=H\circ G\circ F(A)\\
            \end{split}
        \end{equation*}
        (pues la composición de funtores como se definió es asociativa en objetos).
        %TODO puede que también falte ver para que se cumpla algo con la identidad.
    \end{proof}

    \begin{propo}
        Sea $\cf{F}{\mathcal{A}\times\mathcal{B}}{\mathcal{C}}$ un bifuntor. Entonces, para todo $A\in\Obj{\mathcal{A}}$ existe un funtor $\cf{F_A}{\mathcal{B}}{\mathcal{C}}$, llamado el \textbf{funtor asociado derecho respecto a $A$} y está definido como sigue:
        \begin{enumerate}
            \item $F_A(B)=F(A,B)$ para todo $B\in\Obj{\mathcal{B}}$.
            \item $F_A(f)=F=(1_A,f)$ para todo $f$ morfismo en la categoría $\mathcal{B}$.
        \end{enumerate}
        Similaremente, para cada $B\in\Obj{\mathcal{B}}$, existe un funtor $\cf{F^B}{\mathcal{A}}{\mathcal{C}}$, llamado \textbf{funtor asociado izquierdo respecto a $B$}, como sigue:
        \begin{enumerate}
            \item $F^B(A)=F(A,B)$ para todo $A\in\Obj{\mathcal{A}}$.
            \item $F^B(f)=F=(f,1_B)$ para todo $f$ morfismo en la categoría $\mathcal{A}$.
        \end{enumerate}
        Y, si $F$ es covariante (resp. contravariante), entonces $F_A$ y $F^B$ son covariantes (resp. contravariantes). 
    \end{propo}

    \begin{proof}
        Sea $A\in\Obj{\mathcal{A}}$. Se probará que el funtor asociado derecho respecto a $A\in\Obj{\mathcal{A}}$ es un funtor. Claramente está bien definido (manda objetos de $\mathcal{A}$ en objetos de $\mathcal{C}$ y lo mismo con morfismos). Veamos que se cumplen dos condiciones:
        \begin{enumerate}
            \item Sea $B\in\mathcal{B}$. Entonces,
            \begin{equation*}
                \begin{split}
                    F_A(1_B)&=F(1_A,1_B)\\
                    &=1_{F(A,B)}\\
                    &=1_{F_A(B)}\\
                \end{split}
            \end{equation*}
            \item Suponga que $F$ es covariante. Sean $A,B,C\in\Obj{\mathcal{B}}$, $f\in\Hom{\mathcal{B}}{A}{B}$ y $g\in\Hom{\mathcal{B}}{B}{C}$. Entonces:
            \begin{equation*}
                \begin{split}
                    F_A(g\circ f)&=F(1_A,g\circ f)\\
                    &=F(1_A\circ 1_A,g\circ f)\\
                    &=F((1_A,g)\circ(1_A f))\\
                    &=F(1_A,g)\circ F(1_A,f)\\
                    &=F_A(g)\circ F_A(f)\\
                \end{split}
            \end{equation*}
        \end{enumerate}
        de los incisos anteriores, se sigue que $F_A$ es un funtor covariante (los demás casos son análogos).
    \end{proof}

    \begin{exa}
        Sea $\mathcal{C}$ una categoría  y $C\in\Obj{\mathcal{C}}$. El funtor ${\HOM{\mathcal{C}}}_C=\HOM{\mathcal{C}}(C,-)$ es el funtor asociado derecho respecto a $C$ y ${\HOM{\mathcal{C}}}^C=\HOM{\mathcal{C}}(-,C)$ es el funtor asociado izquierdo respecto a $C$.
    \end{exa}

    \begin{exa}
        El bifuntor producto cartesiano $-\times -:\Cat{Set}\times\Cat{Set}\rightarrow\Cat{Set}$, $(-\times-)(A,B)\mapsto A\times B$ y tal que $(-\times-)(f,g)\mapsto f\times g$, donde si $(f,g)\in\Hom{\Cat{Set}}{A}{C}\times\Hom{\Cat{Set}}{B}{D}$, entonces
        \begin{equation*}
            \begin{split}
                f\times g:A\times B&\rightarrow C\times D\\
                (a,b)&\mapsto (f(a),g(b))\\ 
            \end{split}
        \end{equation*}
    \end{exa}

    \begin{exa}
        Para cualquier conjunto $X$, los funtores producto cartesiano $X\times -$ y $-\times X$ son los funtores derecho e izquierdo asociados con respecto a $X$ del bifuntor producto cartesiano, respectivamnete.
    \end{exa}

    \section{Isomorfismos entre categorías}

    \begin{mydef}
        Sea $\cf{F}{\mathcal{C}}{\mathcal{D}}$ un funtor entre dos categorías $\mathcal{C}$ y $\mathcal{D}$. Diremos que $F$ es \textbf{fiel} (respectivamente, \textbf{pleno}) si para cada par de objetos $A,B\in\Obj{\mathcal{C}}$, la función
        \begin{equation*}
            \begin{split}
                F_{A,B}:\Hom{\mathcal{C}}{A}{B}&\rightarrow\Hom{\mathcal{D}}{F(A)}{F(B)}\\
                f\mapsto F(f)
            \end{split}
        \end{equation*}
        es inyectiva (respectivamente, suprayectiva).

        Se dirá que $F$ es \textbf{plenamente fiel} cuando sea fiel y pleno.
    \end{mydef}

    \begin{exa}
        Si $\mathcal{C}$ es una subcategoría de una categoría $\mathcal{D}$, entonces el funtor inclusión $\cf{i}{\mathcal{C}}{\mathcal{D}}$ es un funtor fiel.

        Más aún, $i$ es plenamente fiel si y sólo si $\mathcal{C}$ es subcategoría llena de $\mathcal{D}$. Particularmente,
        \begin{equation*}
            i:\mathcal{D}^{grp}\hookrightarrow \mathcal{D}
        \end{equation*}
        y, será plenamente fiel si y sólo si $\mathcal{D}$ ya es un grupoide.
    \end{exa}

    \begin{exa}
        Sean $\mathcal{C}$ y $\sim$ una relación de congruencia en $\mathcal{C}$. Considere el mapeo proyección
        \begin{equation*}
            \begin{split}
                \pi:\mathcal{C}&\rightarrow\mathcal{C}/\sim\\
                A&\mapsto A\\
                f&\mapsto\left[f\right]=\overline{f}\\
            \end{split}
        \end{equation*}
        es un funtor pleno. En general no va a ser fiel. En particular, si tomamos
        \begin{equation*}
            \begin{split}
                \pi:\Cat{Set}&\rightarrow\Cat{HTop}\\
            \end{split}
        \end{equation*}
        resulta que $\pi_{(\mathbb{R},\mathbb{R})}$ no es inyectiva. Más aún, es una función constante (donde $(\mathbb{R},\mathbb{R})$ denota al conjunto de todas lsa funciones continuas de $\mathbb{R}$ en $\mathbb{R}$). En efecto, sean
        \begin{equation*}
            f,g\in\Hom{\Cat{Top}}{\mathbb{R}}{\mathbb{R}}
        \end{equation*}
        Definimos
        \begin{equation*}
            \begin{split}
                F:\mathbb{R}\times[0,1]&\rightarrow\mathbb{R}\\
                (x,t)&\mapsto (1-t)f(x)+tg(x)\\
            \end{split}
        \end{equation*}
        es claro que $F$ es continua y, además,
        \begin{equation*}
            F(x,0)=f(x),\quad\textup{y}\quad F(x,1)=g(x)
        \end{equation*}
        para todo $x\in\mathbb{R}$, es decir que $f$ y $g$ son homotópicas, luego todas están en la misma clase de equivalencia, así $\pi$ de funciones de $\mathbb{R}$ en $\mathbb{R}$ no es continuo.
    \end{exa}

    \begin{exa}
        Definimos un funtor
        \begin{equation*}
            U:\Cat{Ring}\rightarrow\Cat{Grp}
        \end{equation*}
        como sigue, para cada $R\in\Obj{\Cat{Ring}}$ (anillo conmutativo con identidad). $U(R)=R^*$. Además, si $R,S\in\Obj{\Cat{Ring}}$, para $f\in\Hom{\Cat{Ring}}{R}{S}$,
        \begin{equation*}
            U(f)=f\big|_{ R^*}:R^*\rightarrow S^*
        \end{equation*}
        (donde $S^*$ es el conjunto de todos los elementos de $R$ invertibles respecto al producto del anillo). Es claro que $U$ es un funtor, pero no es fiel ni pleno.
    \end{exa}

    \begin{proof}
        Probaremos que $U$ no es fiel. En efecto, consideremos el anillo
        $\mathbb{Z}_2[X]$ (siendo $\mathbb{Z}_2=\mathbb{Z}/2\mathbb{Z}$). Observemos que
        \begin{equation*}
            \mathbb{Z}_2[X]^*=\mathbb{Z}_2^*=\left\{[1]\right\}
        \end{equation*}
        (por ser $\mathbb{Z}_2$ campo) luego,
        \begin{equation*}
            U_{\mathbb{Z}_2[X],\mathbb{Z}_2[X]}:\Hom{\Cat{Ring}}{\mathbb{Z}_2[X]}{\mathbb{Z}_2[X]}\rightarrow\Hom{\Cat{Grp}}{\left\{[1]\right\}}{\left\{[1]\right\}}
        \end{equation*}
        Se afirma que $\abs{\Hom{\Cat{Ring}}{\mathbb{Z}_2[X]}{\mathbb{Z}_2[X]}}\geq 2$. En efecto, primeramente el morfismo identidad es un morfismo en la categoría $\Cat{Ring}$, pero también
        \begin{equation*}
            \begin{split}
                g:\mathbb{Z}_2[X]&\rightarrow\mathbb{Z}_2[x]\\
                p(x)&\mapsto p(x)^2\\
            \end{split}
        \end{equation*}
        luego este conjunto de morfismos tiene al menos dos elementos, así que $U$ no puede ser fiel.

        Veamos ahora que $U$ no es pleno. Observemos que
        \begin{enumerate}
            \item $\mathbb{Z}^*=\left\{-1,1 \right\}\cong\mathbb{Z}_2$.
            \item Si $p\in\mathbb{N}$ es un número primo, entonces
            \begin{equation*}
                \mathbb{Z}_p^*\cong\mathbb{Z}_{ p-1}
            \end{equation*}
            (visto como isomorfismo de grupos).
            \item Si $m,n\in\mathbb{N}$ con $m,n\geq 1$, entonces
            \begin{equation*}
                \abs{\Hom{\Cat{Grp}}{\mathbb{Z}_n}{\mathbb{Z}_m}}=(n,m)
            \end{equation*}
            donde $(n,m)$ es el máximo común divisor de $m$ y $n$ (basta con observar que $f(1)$ determina por completo a $f$, siendo $\cf{f}{\mathbb{Z}_m}{\mathbb{Z}_n}$ un homomorfismo).
        \end{enumerate}
        Volviendo al problema original. Sea $p\in\mathbb{N}$ tal que $p>2$. Se tiene que
        \begin{equation*}
            U_{ \mathbb{Z},\mathbb{Z}_p}:\Hom{\Cat{Ring}}{\mathbb{Z}}{\mathbb{Z}_p}\rightarrow\Hom{\Cat{Grp}}{\mathbb{Z}_2}{\mathbb{Z}_{ p-1}}
        \end{equation*}
        por (1), (2) y (3), $\abs{\Hom{\Cat{Grp}}{\mathbb{Z}_2}{\mathbb{Z}_{ p-1}}}=2$. Pero, como $\mathbb{Z}$ es un objeto inicial de la categoría $\Cat{Ring}$, por definición se tiene que
        \begin{equation*}
            \abs{\Hom{\Cat{Ring}}{\mathbb{Z}}{\mathbb{Z}_p}}=1
        \end{equation*}
        así, $U$ no puede ser pleno.
    \end{proof}

    \begin{mydef}
        Se dice que una categoría $\mathcal{C}$ es concreta cuando existe un funtor fiel $\cf{F}{\mathcal{C}}{\Cat{Set}}$ (es como si pudiéramos encajar la categoría $\mathcal{C}$ en $\Cat{Set}$).
    \end{mydef}

    \begin{propo}
        Toda categoría pequeña es concreta.
    \end{propo}

    \begin{proof}
        Sea $\mathcal{C}$ una categoría pequeña, es decir que su clase de objetos $\Obj{\mathcal{C}}$ es un conjunto. Definimos
        \begin{equation*}
            F:\mathcal{C}\rightarrow\Cat{Set}
        \end{equation*}
        como sigue
        \begin{enumerate}
            \item Para cada $A\in\Obj{\mathcal{C}}$,
            \begin{equation*}
                F(A)=\left\{(C,\alpha)\in\Obj{\mathcal{C}}\times\textup{Hom}(\mathcal{C}) \Big|\alpha\in\Hom{\mathcal{C}}{C}{A}  \right\}
            \end{equation*}
            en donde
            \begin{equation*}
                \textup{Hom}(\mathcal{C})=\bigcup_{ C,D\in\Obj{\mathcal{C}}}\Hom{\mathcal{C}}{C}{D}
            \end{equation*}
            (notemos que $F(A)$ es efectivamente un conjunto, ya que tanto $\Obj{\mathcal{C}}$ como $\textup{Hom}(\mathcal{C})$ son conjuntos).
            \item Para cada $A,B\in\Obj{\mathcal{C}}$ y $f\in\Hom{\mathcal{C}}{A}{B}$:
            \begin{equation*}
                \begin{split}
                    F(f):F(A)&\rightarrow F(B)\\
                    (C,\alpha)&\mapsto (C,f\circ \alpha)\\
                \end{split}
            \end{equation*}
        \end{enumerate}
        Se afirma que $F$ es un funtor fiel. En efecto,
        \renewcommand{\theenumi}{\roman{enumi}}
        \begin{enumerate}
            \item Fijando $A\in\Obj{\mathcal{C}}$, queremos ver que
            \begin{equation*}
                F(1_A)=1_{F(A)}
            \end{equation*}
            Tomamos $(C,\alpha)\in F(A)$, es decir, $C\in\Obj{\mathcal{C}}$ y $\alpha\in\Hom{\mathcal{C}}{C}{A}$. Entonces,
            \begin{equation*}
                \begin{split}
                    F(1_A)(C,\alpha)&=(C,1_A\circ\alpha)\\
                    &=(C,\alpha)\\
                    &=1_{ F(A)}(C,\alpha)\\
                \end{split}
            \end{equation*}
            \item Sean $f\in\Hom{\mathcal{C}}{A}{B}$ y $g\in\Hom{\mathcal{C}}{B}{D}$. Para cada $(C,\alpha)\in F(A)$ se verifica que
            \begin{equation*}
                \begin{split}
                    F(g\circ f)(C,\alpha)&=(C,g\circ f\circ \alpha)\\
                    &=F(g)(C,f\circ\alpha)\\
                    &=F(g)(F(f)(C,\alpha))\\
                    &=(F(g)\circ F(f))(C,\alpha)\\
                    \Rightarrow F(g\circ f)&=F(g)\circ F(f)\\
                \end{split}
            \end{equation*}
        \end{enumerate}
        por los dos incisos anteriores se sigue que $F$ es un funtor covariante.
        Veamos que es inyectivo. En efecto, sean $A,B\in\Obj{\mathcal{C}}$. Hay que probar que
        \begin{equation*}
            F_{A,B}:\Hom{\mathcal{C}}{A}{B}\rightarrow\Hom{\Cat{Set}}{F(A)}{F(B)}
        \end{equation*}
        es inyectivo. En efecto, sean $f,g\in\Hom{\mathcal{C}}{A}{B}$ y supongamos que
        \begin{equation*}
            F(f)=F(g)
        \end{equation*}
        entonces,
        \begin{equation*}
            \begin{split}
                (A,f)&=F(f)(A,1_A)\\
                &=F(g)(A,1_A)\\
                &=(A,1_A\circ g)\\
                &=(A,g)\\
                \Rightarrow f&=g\\
            \end{split}
        \end{equation*}
        así pues, $f$ es inyectiva, luego $F$ es fiel.
    \end{proof}

    \renewcommand{\theenumi}{\arabic{enumi}}
    
    \begin{mydef}
        Sean $\mathcal{C}$ y $\mathcal{D}$ dos categorías. Un funtor covariante (respectivamente, contravariante) $\cf{F}{\mathcal{C}}{\mathcal{D}}$ se denomina \textbf{isomorfismo} (respectivamente, \textbf{anti-isomorfismo}) de categorías si existe un funtor covariante (respectivamente, contravariante) $\cf{G}{\mathcal{D}}{\mathcal{C}}$ tal que
        \begin{equation*}
            G\circ F=1_{\mathcal{C}}\quad\textup{y}\quad F\circ G=1_{\mathcal{D}}
        \end{equation*}
    \end{mydef}

    \begin{propo}
        Todo isomorfismo de categorías es plenamente fiel.
    \end{propo}

    \begin{proof}
        Sea $\cf{F}{\mathcal{C}}{\mathcal{D}}$ un isomorfismo de categorías y $\cf{G}{\mathcal{D}}{\mathcal{C}}$ el respectivo funtor tal que
        \begin{equation*}
            G\circ F=1_{\mathcal{C}}\quad\textup{y}\quad F\circ G=1_{\mathcal{D}}
        \end{equation*}
        Tomando $A,B\in\Obj{\mathcal{C}}$ arbitrarios, se tiene que $G(F(A))=A$ y $G(F(B))=B$. Consideramos las funciones:
        \begin{equation*}
            \cf{F_{A,B}}{\Hom{\mathcal{C}}{A}{B}}{\Hom{\mathcal{D}}{F(A)}{F(B)}}
        \end{equation*}
        y,
        \begin{equation*}
            \cf{G_{F(A),F(B)}}{\Hom{\mathcal{D}}{F(A)}{F(B)}}{\Hom{\mathcal{C}}{A}{B}}
        \end{equation*}
        Para cada $f\in \Hom{\mathcal{C}}{A}{B}$,
        \begin{equation*}
            G_{F(A),F(B)}(F_{A,B}(f))=G(F(f))=1_{\mathcal{C}}(f)=f
        \end{equation*}
        por tanto, $G_{F(A),F(B)}\circ F_{A,B}=1_{\Hom{\mathcal{C}}{A}{B}}$ y, $F_{A,B}\circ G_{F(A),F(B)}=1_{\Hom{\mathcal{D}}{F(A)}{F(B)}}$. Por tanto, $F_{A,B}$ es invertible, luego biyectiva. Por tanto, $F$ es plenamente fiel.
    \end{proof}

    \begin{cor}
        Se tiene lo siguiente:
        \begin{enumerate}
            \item Los isomorfismos entre categorías mapean objetos finales (resp. iniciales) en objetos finales (resp. iniciales).
            \item Los anti-isomorfismos mapean objetos iniciales (resp. finales) en objetos finales (resp. iniciales).
        \end{enumerate}
    \end{cor}

    \begin{proof}
        Sean $\cf{F}{\mathcal{C}}{\mathcal{D}}$ un anti-isomorfismo y $A\in\Obj{\mathcal{C}}$ un objeto inicial. Sea $D\in\Obj{\mathcal{D}}$ y $B\in\Obj{\mathcal{C}}$ tal que $F(B)=D$. Se tiene que
        \begin{equation*}
            \abs{\Hom{\mathcal{C}}{A}{B}}=1
        \end{equation*}
        Luego,
        \begin{equation*}
            \abs{\Hom{\mathcal{D}}{D}{F(A)}}=\abs{\Hom{\mathcal{D}}{F(B)}{F(A)}}=1
        \end{equation*}
        (pues existe una biyección entre estos dos conjuntos de morfismos). Así, $F(A)$ es un objeto final de $\mathcal{D}$.
    \end{proof}

    \begin{mydef}
        Sea $(R,+_R,\cdot_R)$ un anillo. Definimos el \textbf{anillo opuesto} $(R^{op},+_{ R^{op}},\cdot_{ R^{op}})$ de $R$ tal que satisface lo siguiente:
        \begin{enumerate}
            \item $(R,+_R)=(R^{op},+_{R^{op}})$.
            \item $\cf{\cdot_{R^{op}}}{R^{op}\times R^{op}}{R^{op}}$, $(s,t)\mapsto s\cdot_{R^{op}}t=t\cdot_{R^{op}}s$.
        \end{enumerate}
    \end{mydef}

    \begin{mydef}
        Sea $(R,+_R,\cdot_R)$ un anillo. Decimos que $(A,+_A,\cdot)_R$ es un \textbf{$R$-módulo izquierdo} (resp. \textbf{derecho}) si se cumple que
        \begin{enumerate}
            \item $(A,+_A)$ es un grupo abeliano.
            \item $\cf{\cdot}{R\times A}{A}$, $(\alpha,x)\mapsto \alpha\cdot x$ (resp. $\cf{\cdot}{A\times R}{A}$, $(x,\alpha)\mapsto x\cdot\alpha$) satisface lo siguiente:
            \begin{enumerate}
                \item $\alpha\cdot(x+_A y)=\alpha\cdot x+_A \alpha\cdot y$ (resp.$(x+_A y)\cdot\alpha=x+_A\cdot\alpha+_A y\cdot\alpha$).
                \item $(\alpha+_A\beta)\cdot x=\alpha\cdot x+_A\beta\cdot x$ (resp. $x\cdot(\alpha+_A\beta)=x\cdot\alpha+_A x\cdot\beta$).
                \item $(\alpha\cdot_R\beta)\cdot x=\alpha\cdot(\beta\cdot x)$ (resp. $x\cdot(\alpha\cdot_R\beta)=(x\cdot\alpha)\cdot\beta$).
            \end{enumerate}
            para todo $\alpha,\beta\in R$ y para todo $x,y\in A$. 
        \end{enumerate}
        Si $R$ es un anillo con identidad, entonces $\cdot$ satisface que
        \renewcommand{\theenumi}{\roman{enumi}}
        \begin{enumerate}
            \setcounter{enumi}{3}
            \item $1_R\cdot x=x$, para todo $x\in A$.
        \end{enumerate}
        y diremos que es un \textbf{módulo unitario}.
    \end{mydef}

    \begin{lema}
        Sea $(R,+_R,\cdot_R)$ un anillo y $R^{op}$ su anillo opuesto. Si $(A,+_A,\cdot_A)$ es un $R$-módulo ixquierdo (resp. derecho), entonces $(A,+_A,\cdot_{op})_{R^{op}}$ es un $R^{op}$-módulo derecho (resp. izquierdo), donde
        \begin{equation*}
            \begin{split}
                \cdot_{op}:A\times R&\rightarrow A\\
                (x,\alpha)&\mapsto x\cdot_{op}\alpha=\alpha\cdot x 
            \end{split}
        \end{equation*}
    \end{lema}

    \begin{proof}
        Es inmediato de la definición de $R$-módulo.
    \end{proof}

    \begin{exa}
        Consideremos las categorías $_R\mathcal{M}$ y $\mathcal{M}_{R^{op}}$. Definimos un funtor $\cf{F}{_R\mathcal{M}}{\mathcal{M}_{R^{op}}}$ tal que
        \begin{equation*}
            \begin{split}
                F:\Obj{_R\mathcal{M}}&\rightarrow\Obj{\mathcal{M}_{R^{op}}}\\
                (A,+_A,\cdot)_R&\mapsto(A,+_A,\cdot_{op})_{R^{op}}\\
            \end{split}
        \end{equation*}
        y todo morfismo $f$ lo envía a sí mismo (esto es, $F(f)=f$). De manera similar, se define el morfismo $\cf{G}{\mathcal{M}_{R^{op}}}{_R\mathcal{M}}$ que lo haga al revés, de esta forma estos módulos son tales que
        \begin{equation*}
            G\circ F=1_{_R\mathcal{M}}\quad\textup{y}\quad F\circ G=1_{\mathcal{M}_{R^{op}}}
        \end{equation*}
    \end{exa}

    \begin{exa}
        Consideremos las categorías $\Cat{Ab}$ y $_{\mathbb{Z}}\mathcal{M}$ consideremos el funtor olvidadizo $\cf{F}{_{\mathbb{Z}}\mathcal{M}}{\Cat{Ab}}$ (este funtor olvida al producto).

        Definamos $\cf{\cdot_{\mathbb{Z}}}{\mathbb{Z}\times G}{G}$ tal que
        \begin{equation*}
            m\cdot_{\mathbb{Z}}x=\left\{
                \begin{array}{lcr}
                    \underbrace{x+_G\cdots+_Gx}_{n\textup{-veces}} & \textup{ si } & m>0\\
                    e_G & \textup{ si } & m=0\\
                    \underbrace{(-x)+_G\cdots+_G(-x)}_{n\textup{-veces}} & \textup{ si } & m<0\\
                \end{array}
            \right.
        \end{equation*}
        Se verifica que $(G,+_G,\cdot_{\mathbb{Z}})$ es un $\mathbb{Z}$-módulo.
    \end{exa}

    \begin{exa}
        $\Cat{Set}$ y $\Cat{Set}^{op}$ no son categorías isomorfas pues, en caso contrario, se tendría que el objeto inical $\emptyset$ sería un objeto final de $\Cat{Set}^{op}$, cosa que no puede suceder.
    \end{exa}

    \begin{exa}
        Recordemos que $\Cat{Mat}(\mathbb{K})$ con $\mathbb{K}$ un campo es la categoría de matrices con entradas en $\mathbb{K}$, donde
        \begin{enumerate}
            \item $\Obj{\Cat{Mat}(\mathbb{K})}=\mathbb{N}$.
            \item Para todo $n,m\in\mathbb{N}$, $\Hom{\Cat{Mat}(\mathbb{K})}{n}{m}=\mathbb{M}_{ n\times m}(\mathbb{K})$.
            \item Para todo $n,m,t\in\mathbb{N}$,
            \begin{equation*}
                \begin{split}
                    \Hom{\Cat{Mat}(\mathbb{K})}{n}{m}\times\Hom{\Cat{Mat}(\mathbb{K})}{m}{t}&\rightarrow\Hom{\Cat{Mat}(\mathbb{K})}{n}{t}\\
                    (A,B)&\mapsto AB
                \end{split}
            \end{equation*}
            \item Para todo $n\in\mathbb{N}$, $1_n=I_{n\times n}$.
        \end{enumerate}
        se tiene que esta categoría no es isomorfa a la categoría de espacios vectoriales de dimensión finita.
    \end{exa}

    \begin{mydef}
        Se define lo siguiente:
        \begin{enumerate}
            \item Sea $\cf{F}{\mathcal{C}}{\mathcal{D}}$ un funtor, decimos que $F$ \textbf{ preserva una propiedad $P$ de los morfismos} si cuando $F$ la tiene, también $F(f)$ la tiene.
            \item Se $\cf{F}{\mathcal{C}}{\mathcal{D}}$ un funtor. Decimos que $F$ \textbf{refleja una propiedad $P$ de los morfismos} si cuando $F(f)$ la tiene, también $f$ la tiene.
        \end{enumerate}
    \end{mydef}

    \begin{propo}
        Se tiene lo siguiente:
        \begin{enumerate}
            \item Cualquier funtor preserva isomorfismos.
            \item Cualquier funtor plenamente fiel refleja isomorfismos.
            \item Cualquier funtor plenamente fiel refleja objetos finales e iniciales.
            \item Cualquier funtor fiel refleja monomorfismos y epimorfismos.
        \end{enumerate}
    \end{propo}

    \begin{proof}
        Sean $\mathcal{C}$ y $\mathcal{D}$ dos categorías y $\cf{F}{\mathcal{C}}{\mathcal{D}}$ un funtor covariante.

        De (1): Sean $A,B\in\Obj{\mathcal{C}}$ y consideremos $f\in\Hom{\mathcal{C}}{A}{B}$ tal que $f$ es isomorfismo. Entonces, existe $g\in\Hom{\mathcal{C}}{B}{A}$ tal que
        \begin{equation*}
            f\circ g=1_B\quad\textup{y}\quad g\circ f=1_A
        \end{equation*}
        Entonces $F(f)\circ F(g)=F(f\circ g)=F(1_B)=1_{F(B)}$ y $F(g)\circ F(f)=F(g\circ f)=F(1_A)=1_{F(A)}$.
        Por tanto, $F(f)$ es un isomorfismo.

        De (2): Ejercicio.

        De (3): La prueba para objetos iniciales está en el libro. Sea $A\in\Obj{\mathcal{C}}$ tal que $F(A)\in\Obj{\mathcal{D}}$ es un objeto final. Tenemos que
        \begin{equation*}
            \abs{\Hom{\mathcal{D}}{D}{F(A)}}=1,\quad\forall D\in\Obj{\mathcal{D}}
        \end{equation*}
        Entonces,
        \begin{equation*}
            \abs{\Hom{\mathcal{D}}{F(B)}{F(A)}}=1,\quad\forall B\in\Obj{\mathcal{C}}
        \end{equation*}
        Como $F$ es plenamente fiel, entonces $\cf{F_{A,B}}{\Hom{\mathcal{C}}{B}{A}}{\Hom{\mathcal{D}}{F(B)}{F(A)}}$ es una biyección, luego
        \begin{equation*}
            \abs{\Hom{\mathcal{D}}{B}{A}}=1,\quad\forall B\in\Obj{\mathcal{C}}
        \end{equation*}
        Por tanto, $A$ es objeto final de $\mathcal{C}$.

        De (4): La prueba para monomorfismos está en el libro. Sean $A,B\in\Obj{\mathcal{C}}$, $f\in\Hom{\mathcal{C}}{A}{B}$ tla que $F(f)\in\Hom{\mathcal{D}}{F(A)}{F(B)}$ es un epimorfismo, Sea $C\in\Obj{\mathcal{C}}$ y consideremos $g_1,g_2\in\Hom{\mathcal{C}}{B}{C}$. Si
        \begin{equation*}
            \begin{split}
                g_1\circ f=g_2\circ f&\Rightarrow
            \end{split}
        \end{equation*}
        %TODO
    \end{proof}

    \chapter{Trasnformaciones Naturales}

    \newcommand{\Nat}[2]{\ensuremath{\textup{Nat}\left(#1,#2\right)}}

    \begin{mydef}
        Sean $\mathcal{C},\mathcal{D}$ categorías y $\cf{F,G}{\mathcal{C}}{\mathcal{D}}$ dos funtores. Una \textbf{transformación nautral} consiste en una familia de morfismos $\cf{\alpha_C}{F(C)}{G(C)}$ con $C\in\Obj{\mathcal{C}}$ en $\mathcal{D}$ tales que para todo $f\in\Hom{\mathcal{C}}{C}{C'}$, el siguiente diagrama
        \begin{equation*}
            \begin{array}{rcccl}
                &F(C) & \overset{\longrightarrow}{\alpha_C} & G(C) &\\
                F(f) & \downarrow & & \downarrow & G(f) \\
                &F(C') & \overset{\longrightarrow}{\alpha_{ C'}} & G(C') &\\
            \end{array}
        \end{equation*}
        Denotaremos por $\Nat{f}{g}$. Una transformación natural $\alpha$ se denotará por $\cf{\alpha}{F}{G}$.
    \end{mydef}

    \begin{obs}
        Si todas las componentes de la transformación natural son isomorfismos, llamaremos a nuestra transformación natural un \textbf{isomorfismo natural} y lo denotaremos como $F\cong G$.
    \end{obs}

    \begin{mydef}
        Sean $W,V$ un categorías. Se define para $\cf{u}{W}{V}$ el funtor del \textbf{espacio dual} $\cf{u^*}{W^*}{V}$ tal que $w\mapsto w\circ u$ para todo $w\in W^*$, donde el asterisco es la categoría dual.
    \end{mydef}

    \begin{mydef}
        Considere la categoría $_R\mathcal{M}$ de espacios vectoriales. De define el \textbf{funtor del espacio doble dual} como
        \begin{equation*}
            (-)^{ **}:_K\mathcal{M}\rightarrow _K\mathcal{M}
        \end{equation*}
        tal que si $\cf{u}{U}{V}$, siendo $U,V$ espacios vectoriales, es una transformación lineal, entonces
        \begin{equation*}
            u^{**}:U^{ **}\rightarrow V^{ **}
        \end{equation*}
        tal que $\phi\mapsto \phi\circ u^*$ para todo $\cf{\phi}{U^*}{K}$.
    \end{mydef}

    \begin{exa}
        Definimso una transformación (aún no natural) entre el funtor identidad de la categoría $_R\mathcal{M}$ y el funtor doble dual. Definimos la transformación como sigue:
        \begin{equation*}
            \eta:L_{_K\mathcal{M}}\rightarrow (-)^{ **}
        \end{equation*}
        cuyas componentes son
        \begin{equation*}
            \eta_V:V\rightarrow V^{**}
        \end{equation*}
        con $V$ espacio vectorial, definida como sigue:
        \begin{equation*}
            \eta_V(v)(f)=f(v),\quad\forall v\in V\textup{ y }\forall f\in V^*
        \end{equation*}
        Probaremos que el diagrama de abajo es conmutativo.
    \end{exa}

    \begin{proof}
        En efecto, sea $u\in\Hom{_k\mathcal{M}}{V}{W}$, hay que ver que el diagrama
        \begin{equation*}
            \begin{array}{rcccl}
                &V & \overset{\longrightarrow}{\eta_V} & V^** &\\
                u & \downarrow & & \downarrow & u^{**} \\
                &W & \overset{\longrightarrow}{\eta_W} & W^{**} &\\
            \end{array}
        \end{equation*}
        En efecto, sea $v\in V$ y $f\in W^*$. Se tiene que
        \begin{equation*}
            \begin{split}
                [u^**\circ\eta_V(v)][f]&=[\eta_V(v)\circ u^*](f)\\
                &=\eta_V(v)(u^*(f))\\
                &=u^*(f)(v)\\
                &=f(u(v))\\
                &=\eta_W(u(v))(f)\\
                &=[(\eta_W\circ u)(v)](f)\\
            \end{split}
        \end{equation*}
        por tanto, $\cf{\eta}{1_{_k\mathcal{M}}}{(-)^{**}}$ es una transformación natural.
    \end{proof}

    \begin{exa}
        Para todo funtor $\cf{F}{\mathcal{C}}{\mathcal{D}}$ se define la \textbf{transformación natural identidad}, tal que
        \begin{equation*}
            \begin{split}
                1_F:F&\rightarrow F\\
                1_F&=\left(1_{ F(C)} \right)_{C\in\Obj{\mathcal{C}}}\\
            \end{split}
        \end{equation*}
    \end{exa}

    \begin{exa}
        Sean $\cf{F,G,H}{\mathcal{C}}{\mathcal{D}}$ tres funtores y $\cf{\alpha}{F}{G}$ y $\cf{\beta}{G}{H}$ transformaciones naturales. Definimos una transformación natural como sigue:
        \begin{equation*}
            \begin{split}
                (\beta\circ\alpha)_C&:F(C)\rightarrow H(C)\\
                (\beta\circ\alpha)_C&=\beta_c\circ\alpha_C,\quad\forall C\in\Obj{\mathcal{C}}\\
            \end{split}
        \end{equation*}
        a $\left((\beta\circ\alpha)_C \right)_{C\in\Obj{\mathcal{C}}}$ la conoceremos como \textbf{composición vertical de las transformaciones naturales $\alpha$ y $\beta$}.
    \end{exa}

    \begin{exa}
        Sean $\cf{F,G}{\mathcal{C}}{\mathcal{D}}$ funtores y $\cf{\alpha}{F}{G}$ un isomorfismo natural. Definimos la siguiente transformación
        \begin{equation*}
            \begin{split}
                \alpha^{-1}:G\mapsto F
            \end{split}
        \end{equation*}
        tal que $(\alpha^{-1})_C=(\alpha_C)^{-1}$ para todo $C\in\Obj{\mathcal{C}}$. Esta transformación natural es llamada \textbf{transformación natural inversa}.
    \end{exa}

    \begin{exa}
        Sean $\cf{F,G}{\mathcal{C}}{\mathcal{D}}$ y $\cf{H}{\mathcal{D}}{\mathcal{E}}$ tres funtores y $\cf{\alpha}{F}{G}$ una transformación natural. Definimos una transformación nautral
        \begin{equation*}
            \cf{H_\alpha}{H\circ F}{H\circ G}
        \end{equation*}
        tal que $(H_\alpha)_C=H(\alpha_C)$ para todo $C\in\Obj{\mathcal{C}}$. La llamaremos \textbf{Whiskering de la transformación nautral por la derecha de $H$}.

        De forma análoga, si $\cf{K}{\mathcal{B}}{\mathcal{C}}$ es un funtor, se define la transformación natural
        \begin{equation*}
            _\alpha K:F\circ K\rightarrow G\circ K
        \end{equation*}
        tal que $(_\alpha K)_B=\alpha_{ K(B)}$ para todo $B\in\Obj{\mathcal{B}}$. Llamaremos a esta tranformación nautral \textbf{Whiskering de la transformación natural por la izquierda de $K$}.
    \end{exa}

    \begin{obs}
        En el ejemplo anterior, si $\alpha$ es un isomorfismo natural, entonces $H_\alpha$ y $_\alpha K$ también serán isomorfismos naturales.
    \end{obs}

    \begin{propo}
        Sean $\cf{F,G}{\mathcal{C}}{\mathcal{D}}$ y $\cf{H,K}{\mathcal{D}}{\mathcal{E}}$ funtores, $\cf{\alpha}{F}{G}$ y $\cf{\beta}{H}{K}$ transformaciones naturales. Entonces, $\cf{\beta*\alpha}{H\circ F}{K\circ G}$ definida como sigue:
        \begin{equation*}
            (\beta*\alpha)_C=\beta_{G(C)} \circ H_{\alpha(C)},\quad \forall C\in\Obj{\mathcal{C}}
        \end{equation*}
        es una transformación nautral.
    \end{propo}

    \begin{proof}
        Aplicamos que $\beta$ es una transformación nautral al morfismo $\cf{\alpha_C}{F(C)}{G(C)}$. Queremos que el diagrama
        \begin{equation*}
            \begin{array}{ccc}
                H\circ F(C) & \longrightarrow & K\circ F(C)\\
                \downarrow & & \downarrow\\
                H\circ G(C) &\longrightarrow &K\circ G(C)\\
            \end{array}
        \end{equation*}
    \end{proof}

\end{document}