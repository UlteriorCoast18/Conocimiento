\documentclass[12pt]{report}
\usepackage[spanish]{babel}
\usepackage[utf8]{inputenc}
\usepackage{amsmath}
\usepackage{amssymb}
\usepackage{amsthm}
\usepackage{graphics}
\usepackage{subfigure}
\usepackage{lipsum}
\usepackage{array}
\usepackage{multicol}
\usepackage{enumerate}
\usepackage[framemethod=TikZ]{mdframed}
\usepackage[a4paper, margin = 1.5cm]{geometry}

%En esta parte se hacen redefiniciones de algunos comandos para que resulte agradable el verlos%

\renewcommand{\theenumii}{\roman{enumii}}

\def\proof{\paragraph{Demostración:\\}}
\def\endproof{\hfill$\blacksquare$}

\def\sol{\paragraph{Solución:\\}}
\def\endsol{\hfill$\square$}

%En esta parte se definen los comandos a usar dentro del documento para enlistar%

\newtheoremstyle{largebreak}
  {}% use the default space above
  {}% use the default space below
  {\normalfont}% body font
  {}% indent (0pt)
  {\bfseries}% header font
  {}% punctuation
  {\newline}% break after header
  {}% header spec

\theoremstyle{largebreak}

\newmdtheoremenv[
    leftmargin=0em,
    rightmargin=0em,
    innertopmargin=-2pt,
    innerbottommargin=8pt,
    hidealllines = true,
    roundcorner = 5pt,
    backgroundcolor = gray!60!red!30
]{exa}{Ejemplo}[section]

\newmdtheoremenv[
    leftmargin=0em,
    rightmargin=0em,
    innertopmargin=-2pt,
    innerbottommargin=8pt,
    hidealllines = true,
    roundcorner = 5pt,
    backgroundcolor = gray!50!blue!30
]{obs}{Observación}[section]

\newmdtheoremenv[
    leftmargin=0em,
    rightmargin=0em,
    innertopmargin=-2pt,
    innerbottommargin=8pt,
    rightline = false,
    leftline = false
]{theor}{Teorema}[section]

\newmdtheoremenv[
    leftmargin=0em,
    rightmargin=0em,
    innertopmargin=-2pt,
    innerbottommargin=8pt,
    rightline = false,
    leftline = false
]{propo}{Proposición}[section]

\newmdtheoremenv[
    leftmargin=0em,
    rightmargin=0em,
    innertopmargin=-2pt,
    innerbottommargin=8pt,
    rightline = false,
    leftline = false
]{cor}{Corolario}[section]

\newmdtheoremenv[
    leftmargin=0em,
    rightmargin=0em,
    innertopmargin=-2pt,
    innerbottommargin=8pt,
    rightline = false,
    leftline = false
]{lema}{Lema}[section]

\newmdtheoremenv[
    leftmargin=0em,
    rightmargin=0em,
    innertopmargin=-2pt,
    innerbottommargin=8pt,
    roundcorner=5pt,
    backgroundcolor = gray!30,
    hidealllines = true
]{mydef}{Definición}[section]

\newmdtheoremenv[
    leftmargin=0em,
    rightmargin=0em,
    innertopmargin=-2pt,
    innerbottommargin=8pt,
    roundcorner=5pt
]{excer}{Ejercicio}[section]

%En esta parte se colocan comandos que definen la forma en la que se van a escribir ciertas funciones%

\newcommand\abs[1]{\ensuremath{\big|#1\big|}}
\newcommand\divides{\ensuremath{\bigm|}}
\newcommand\cf[3]{\ensuremath{#1:#2\rightarrow#3}}
\newcommand\contradiction{\ensuremath{\#_c}}
\newcommand{\Obj}[1]{\ensuremath{\textup{Obj}\left(#1\right)}}
\newcommand{\Hom}[3]{\ensuremath{\textup{Hom}_{#1}\left(#2,#3\right)}}

\newcommand{\Cat}[1]{\ensuremath{\textup{\textbf{#1}}}}
\newcommand{\Iso}[3]{\ensuremath{\textup{I}_{\textup{SO}_{#1}}\left(#2,#3\right)}}
\newcommand{\SO}[1]{\ensuremath{\textup{SO}\left(#1\right)}}
\newcommand{\Quo}[1]{\ensuremath{\textup{Quo}\left(#1 \right)}}

%recuerda usar \clearpage para hacer un salto de página

\begin{document}
    \setlength{\parskip}{5pt} % Añade 5 puntos de espacio entre párrafos
    \setlength{\parindent}{12pt} % Pone la sangría como me gusta
    \title{Notas de Álgebra Moderna IV.
    
    Una introducción a la teoría de categorías.}
    \author{Cristo Daniel Alvarado}
    \maketitle

    \tableofcontents %Con este comando se genera el índice general del libro%

    \setcounter{chapter}{2} %En esta parte lo que se hace es cambiar la enumeración del capítulo%
    
    \chapter{Funtores}

    \section{Conceptos Fundamentales}
    
    \begin{mydef}
        Sean $\mathcal{C}$ y $\mathcal{D}$ dos categorías. Un \textbf{funtor covariante} (respectivamente, \textbf{funtor contravariante}), denotado por $\cf{F}{\mathcal{C}}{\mathcal{D}}$, consta de
        \begin{enumerate}
            \item Un mapeo $\cf{F}{\Obj{\mathcal{C}}}{\mathcal{D}}$ tal que $A\mapsto F(A)$.
            \item Para cualesquier dos pares de objetos $A,B\in\Obj{\mathcal{C}}$, un mapeo $\cf{F}{\Hom{\mathcal{C}}{A}{B}}{\Hom{\mathcal{D}}{F(A)}{F(B)}}$ (resp. $\cf{F}{\Hom{\mathcal{C}}{A}{B}}{\Hom{\mathcal{D}}{F(B)}{F(A)}}$) tal que $f\mapsto F(f)$, que cumple las condiciones siguientes:
            \begin{enumerate}
                \item Para cada $A\in\Obj{\mathcal{C}}$, $F(1_A)=1_{F(A)}$.
                \item Para cada $f\in\Hom{\mathcal{C}}{A}{B}$ y $g\in\Hom{\mathcal{C}}{B}{C}$, se tiene que
                \begin{equation*}
                    F(g\circ f)=F(g)\circ F(f)
                \end{equation*}
                (resp. $F(g\circ f)=F(f)\circ F(g)$).
            \end{enumerate}
        \end{enumerate}
    \end{mydef}

    \begin{mydef}
        la \textbf{imagen de un funtor $F$ entre las categorías $\mathcal{C}$ y $\mathcal{D}$}, consta de una clase $\left\{F(C)\Big|C\in\Obj{\mathcal{C}} \right\}$ junto con todos los conjuntos $\left\{F(f)\Big|f\in\Hom{\mathcal{C}}{A}{B}\textup{ con }A,B\in\Obj{\mathcal{C}} \right\}$.
    \end{mydef}

    \begin{obs}
        La imagen de un funtor no necesariamente es una categoría.
    \end{obs}

    \begin{proof}
        En efecto, sean $\mathcal{C}$ y $\mathcal{D}$ dos categorías. Para cualesquiera $C_1,C_2,C_3,C_4\in\Obj{\mathcal{C}}$ y $D_1,D_2,D_3\in\Obj{\mathcal{D}}$, $f\in\Hom{\mathcal{C}}{C_1}{C_2}$ y $g\in\Hom{\mathcal{C}}{C_3}{C_4}$, $h\in\Hom{\mathcal{D}}{D_1}{D_2}$ y $k\in\Hom{\mathcal{D}}{D_2}{D_3}$. Se tiene lo siguiente:
        \begin{equation*}
            \begin{split}
                C_1\longrightarrow C_2\textup{ y }C_3\longrightarrow C_4\\
                D_1\longrightarrow D_2\longrightarrow D_3\\
            \end{split}
        \end{equation*}
        la imagen de $\cf{F}{\mathcal{C}}{\mathcal{D}}$ noes una categoría, pues si hacemos que
        \begin{equation*}
            F(C_1)=D_1, F(C_2)=F(C_3)=D_2\textup{ y }F(C_4)=D_4
        \end{equation*}
        haciendo
        \begin{equation*}
            F(f)=h,F(g)=k
        \end{equation*}
        además,
        \begin{equation*}
            F(1_{C_1})=1_{D_1}\quad F(1_{C_2})=F(1_{C_3})=1_{D_2}\quad F(1_{C_4})=1_{D_3} 
        \end{equation*}
        pues, $h$ y $k$ paretenecen a la imagen de $F$, pero su composición no lo está.
    \end{proof}

    \begin{obs}
        Si $F$ es inyectiva, entonces la imagen de $F$ será una categoría.
    \end{obs}

    \begin{propo}
        Si $\cf{F}{\mathcal{C}}{\mathcal{D}}$ es un funtor covariante (resp. contravariante), entonces $\cf{F'}{\mathcal{C}}{\mathcal{D}}$ es un funtor contravariante (resp. covariante).
    \end{propo}

    \begin{proof}
        
    \end{proof}

\end{document}