\documentclass[12pt]{report}
\usepackage[spanish]{babel}
\usepackage[utf8]{inputenc}
\usepackage{amsmath}
\usepackage{amssymb}
\usepackage{amsthm}
\usepackage{graphics}
\usepackage{subfigure}
\usepackage{lipsum}
\usepackage{array}
\usepackage{multicol}
\usepackage{enumerate}
\usepackage[framemethod=TikZ]{mdframed}
\usepackage[a4paper, margin = 1.5cm]{geometry}
\usepackage{enumitem}
\usepackage{tikz}
\usepackage{pgffor}
\usepackage{ifthen}

\usetikzlibrary{shapes.multipart}

\newcounter{it}
\newcommand*\watermarktext[1]{\begin{tabular}{c}
    \setcounter{it}{1}%
    \whiledo{\theit<100}{%
    \foreach \col in {0,...,15}{#1\ \ } \\ \\ \\
    \stepcounter{it}%
    }
    \end{tabular}
    }

\AddToHook{shipout/foreground}{
    \begin{tikzpicture}[remember picture,overlay, every text node part/.style={align=center}]
        \node[rectangle,black,rotate=30,scale=2,opacity=0.08] at (current page.center) {\watermarktext{Cristo Daniel Alvarado ESFM\quad}}; 
  \end{tikzpicture}
}
%En esta parte se hacen redefiniciones de algunos comandos para que resulte agradable el verlos%

\renewcommand{\theenumii}{\roman{enumii}}

\def\proof{\paragraph{Demostración:\\}}
\def\endproof{\hfill$\blacksquare$}

\def\sol{\paragraph{Solución:\\}}
\def\endsol{\hfill$\square$}

%En esta parte se definen los comandos a usar dentro del documento para enlistar%

\newtheoremstyle{largebreak}
  {}% use the default space above
  {}% use the default space below
  {\normalfont}% body font
  {}% indent (0pt)
  {\bfseries}% header font
  {}% punctuation
  {\newline}% break after header
  {}% header spec

\theoremstyle{largebreak}

\newmdtheoremenv[
    leftmargin=0em,
    rightmargin=0em,
    innertopmargin=0pt,
    innerbottommargin=5pt,
    hidealllines = true,
    roundcorner = 5pt,
    backgroundcolor = gray!60!red!30
]{exa}{Ejemplo}[section]

\newmdtheoremenv[
    leftmargin=0em,
    rightmargin=0em,
    innertopmargin=0pt,
    innerbottommargin=5pt,
    hidealllines = true,
    roundcorner = 5pt,
    backgroundcolor = gray!50!blue!30
]{obs}{Observación}[section]

\newmdtheoremenv[
    leftmargin=0em,
    rightmargin=0em,
    innertopmargin=0pt,
    innerbottommargin=5pt,
    rightline = false,
    leftline = false
]{theor}{Teorema}[section]

\newmdtheoremenv[
    leftmargin=0em,
    rightmargin=0em,
    innertopmargin=0pt,
    innerbottommargin=5pt,
    rightline = false,
    leftline = false
]{propo}{Proposición}[section]

\newmdtheoremenv[
    leftmargin=0em,
    rightmargin=0em,
    innertopmargin=0pt,
    innerbottommargin=5pt,
    rightline = false,
    leftline = false
]{cor}{Corolario}[section]

\newmdtheoremenv[
    leftmargin=0em,
    rightmargin=0em,
    innertopmargin=0pt,
    innerbottommargin=5pt,
    rightline = false,
    leftline = false
]{lema}{Lema}[section]

\newmdtheoremenv[
    leftmargin=0em,
    rightmargin=0em,
    innertopmargin=0pt,
    innerbottommargin=5pt,
    roundcorner=5pt,
    backgroundcolor = gray!30,
    hidealllines = true
]{mydef}{Definición}[section]

\newmdtheoremenv[
    leftmargin=0em,
    rightmargin=0em,
    innertopmargin=0pt,
    innerbottommargin=5pt,
    roundcorner=5pt
]{excer}{Ejercicio}[section]

%En esta parte se colocan comandos que definen la forma en la que se van a escribir ciertas funciones%

\newcommand\abs[1]{\ensuremath{\big|#1\big|}}
\newcommand\divides{\ensuremath{\bigm|}}
\newcommand\cf[3]{\ensuremath{#1:#2\rightarrow#3}}
\newcommand\contradiction{\ensuremath{\#_c}}
\newcommand{\Obj}[1]{\ensuremath{\textup{Obj}\left(#1\right)}}
\newcommand{\Hom}[3]{\ensuremath{\textup{Hom}_{#1}\left(#2,#3\right)}}

\newcommand{\Cat}[1]{\ensuremath{\textup{\textbf{#1}}}}
\newcommand{\Iso}[3]{\ensuremath{\textup{I}_{\textup{SO}_{#1}}\left(#2,#3\right)}}
\newcommand{\SO}[1]{\ensuremath{\textup{SO}\left(#1\right)}}
\newcommand{\Quo}[1]{\ensuremath{\textup{Quo}\left(#1 \right)}}

%recuerda usar \clearpage para hacer un salto de página

\begin{document}
    \setlength{\parskip}{5pt} % Añade 5 puntos de espacio entre párrafos
    \setlength{\parindent}{12pt} % Pone la sangría como me gusta
    \title{Notas de Álgebra Moderna IV.
    
    Módulos.}
    \author{Cristo Daniel Alvarado}
    \maketitle

    \tableofcontents %Con este comando se genera el índice general del libro%

    %\setcounter{chapter}{3} %En esta parte lo que se hace es cambiar la enumeración del capítulo%

    \chapter{Moduos, Homomorfismos y Secuencias exactas}

    Los módulos son una generalización de los grupos abelianos y lo enteros (los cuales son módulos sobre $\mathbb{Z}$).

    \begin{mydef}
        Sea $R$ un anillo no trivial. Decimos que $R$ es un \textbf{anillo de división}, si $R$ es unitario y para cada $a\in A$ existe $a^{-1}\in A$.

        Si $R$ es conmutativo, entonces $R$ es un \textbf{campo}.
    \end{mydef}

    \begin{mydef}
        Sea $R$ un anillo, un \textbf{$R$-módulo (izquierdo)} es un grupo abeilano $A$ junto con una función $\cf{\cdot}{R\times A}{A}$ (denotada simplemente por $(r,a)\mapsto ra$) tal que para todo $r,s\in R$ y para todo $a\in A$:
        \begin{enumerate}[label=\textit{(\arabic*)}]
            \item $r(a+b)=ra+rb$.
            \item $(r+s)a=ra+sa$.
            \item $r(sa)=(rs)a$.
        \end{enumerate}
        si $R$ además tiene elemento identidad $1_R$ y se cumple que
        \begin{enumerate}[label=\textit{(\arabic*)}]
            \setcounter{enumi}{3}
            \item $1_Ra=a$, para todo $a\in A$.
        \end{enumerate}
        entonces decimos que $A$ es un \textbf{$R$-módulo unitario (izquierdo)}. En caso de que $R$ sea un anillo de división, el módulo unitario $A$ será llamado \textbf{espacio vectorial (izquierdo)}.
    \end{mydef}

    De forma análoga podemos definir los $R$-módulos derechos, cambiando el orden en el que se hacen las operaciones. Sin embargo, a lo largo del texto solo trabajaremos con módulos izquierdos y todos los resultados que se prueben para esto, también se cumplirán para los derechos.
    
    \begin{excer}
        Sea $A$ un $R$-módulo izquierdo. Si $R$ es conmutativo, podemos hacer de $A$ un $R$-módulo derecho definiendo:
        \begin{equation*}
            ar = ra,\quad\forall a\in A\textup{ y }\forall r\in R
        \end{equation*}
    \end{excer}

    \begin{proof}
        Considere la función de $\cf{\cdot}{A\times R}{A}$ dada por:
        \begin{equation*}
            (a,r)\mapsto ar=ra,\quad\forall (a,r)\in A\times R
        \end{equation*}
        Afirmamos que esta función hace de $A$ un $R$-módulo derecho. En efecto, debemos verificar tres condiciones, sean $r,s\in R$ y $a,b\in A$:
        \begin{enumerate}[label=\textit{(\arabic*)}]
            \item Se tiene que:
            \begin{equation*}
                \begin{split}
                    (a+b)r&=r(a+b)\\
                    &=ra+rb\\
                    &=ar+br\\
                \end{split}
            \end{equation*}
            \item Se tiene que:
            \begin{equation*}
                \begin{split}
                    a(r+s)&=(r+s)a\\
                    &=ra+sa\\
                    &=ar+as\\
                \end{split}
            \end{equation*}
            \item Se tiene que:
            \begin{equation*}
                \begin{split}
                    (as)r&=r(as)\\
                    &=r(sa)\\
                    &=(rs)a\textup{, como $R$ es conmutativo:} \\
                    &=(sr)a\\
                    &=a(sr)\\
                \end{split}
            \end{equation*}
        \end{enumerate}
        por los tres incisos anteriores se sigue que $A$ es un $R$-módulo derecho.
    \end{proof}

    \begin{obs}
        A menos que se especifique lo contrario, todo $R$-módulo $A$ sobre un anillo conmutativo $R$ será izquierdo y derecho haciendo:
        \begin{equation*}
            ra=ar,\quad\forall a\in A\textup{ y }\forall r\in R
        \end{equation*}
    \end{obs}

    \begin{obs}
        Denotaremos al elemento identidad de un $R$-módulo $A$ por $0_A$, y al elemento neutro de $R$ por $0_R$.
    \end{obs}

    \begin{propo}
        Sea $A$ un $R$-módulo, entonces:
        \begin{equation*}
            r0_A=0_A\quad\textup{y}\quad 0_Ra=0_A
        \end{equation*}
        para todo $r\in R$ y para todo $a\in A$.
    \end{propo}

    \begin{proof}
        Sea $r\in R$, se tiene que:
        \begin{equation*}
            r0_A=r(0_A+0_A)=r0_A+r0_A\Rightarrow r0_A=0_A
        \end{equation*}
        y, para todo $a\in A$:
        \begin{equation*}
            0_Ra=(0_R+0_R)a=0_Ra+0_Ra\Rightarrow 0_Ra=0_A
        \end{equation*}
    \end{proof}

    Por lo que, en lo que sigue del texto se denotará por 0 a $0_A,0_R$, $0\in\mathbb{Z}$ y al módulo trivial $\left\{0\right\}$.

    %TODO

    \newpage

    \section{Referencias}

    \begin{itemize}
        \item \textit{Algebra} de Thomas Hungerford, ed. Springer.
    \end{itemize}

    \chapter{Módulos Libres y Espacios Vectoriales}

    \section{Conceptos Fundamentales}

    No queda de otra más que asumir este resultado de categorías:

    \begin{theor}[\textbf{Hungerford, Theorem I.7.8}]
        Si $\mathcal{C}$ es una categoría concreta, $F$ y $F'$ son objetos en $C$ tales que $F$ es libre en el conjunto $X$ y $F'$ lo es en $X'$ siendo estos conjuntos tales que $\abs{X}=\abs{X'}$, entonces $F$ es equivalente a $F'$.
    \end{theor}

    En particular, la categoría de $R$-módulos unitarios es una categoría concreta, donde la equivalencia entre dos objetos de la categoría es un isomorfismo entre ambos $R$-módulos.

    \begin{theor}
        Sea $R$ un anillo conmutativo con identidad. Las siguientes condiciones son equivalentes en un $R$-módulo unitario $F$:
        \renewcommand{\theenumi}{\roman{enumi}}
        \begin{enumerate}
            \item $F$ tiene base no vacía.
            \item $F$ es la suma interna directa de una familia cíclica de $R$-módulos, cada uno de los cuales es isomorfo a $R$ como un $R$-módulo.
            \item $F$ es un $R$-módulo isomorfo a la suma directa de copias del $R$-módulo izquierdo $R$.
            \item Existe un conjunto no vacío $X$ y una función $\cf{i}{X}{F}$ con la siguiente propiedad: dado un $R$-módulo, $A$ y una función $\cf{f}{X}{A}$ existe un único homomorfismo de $R$-módulos $\cf{\overline{f}}{F}{A}$ tal que
            \begin{equation*}
                \overline{f}\circ i=f
            \end{equation*}
            En otras palabras, $F$ es un objeto libre en la categoría de $R$-módulos uniatrios.
        \end{enumerate}
    \end{theor}

    \begin{proof}
        $(i)\Rightarrow(iv)$: Sea $X$ una base no vacía de $F$ y sea $\cf{i}{X}{F}$ el mapeo inclusión. Sea $A$ un $R$-módulo y $\cf{f}{X}{A}$ una función.

        Si $u\in F$, entonces existen $n\in\mathbb{N}\cup\left\{0\right\}$, $r_i\in R$ y $x_i\in X$, para todo $i\in\left\{1,...,n\right\}$ tales que
        \begin{equation*}
            u=\sum_{ i=1}^n r_i x_i
        \end{equation*}
        Definimos la función $\cf{\overline{f}}{F}{A}$ dada por:
        \begin{equation*}
            \overline{f}(u)=\sum_{ i=1}^n r_if(x_i)
        \end{equation*}
        Esta función está bien definida, pues $F$ tiene como base a $X$ (por ende, todo elemento se representa de forma única como combinación lineal finita de elementos de $X$). Además,
        \begin{equation*}
            \begin{split}
                \overline{f}\circ i(x_i)&=\overline{f}(x_i)\\
                &=1_R\cdot f(x_i) \\
                &=f(x_i),\quad\forall x_i\in X \\
            \end{split}
        \end{equation*}
        por ende, $\overline{f}\circ i=f$.

        Veamos que es homomorfismo de $R$-módulos (no sé como se verifica eso, chécalo porfa Roque).

        Ahora, si $\cf{g}{F}{A}$ es otro homomorfismo de $R$-módulos tal que
        \begin{equation*}
            g\circ i=f
        \end{equation*}
        se tiene que
        \begin{equation*}
            \overline{f}\circ i=g\circ i\Rightarrow \overline{f}\big|_{X}=g\big|_{X}
        \end{equation*}
        Como $X$ genera $F$ y todo homomorfismo de $R$-módulos que vaya de $F$ en algún $R$-módulo, $B$ queda únicamente determinado por $X$, basta ver que $\overline{f}=g$ en $X$, lo cual sucede por la igualdad anterior. Por tanto, $\overline{f}$ es único.

        $(iv)\Rightarrow(iii)$: Asumiendo $(iv)$, sean $X\subseteq F$ no vacío y una función $\cf{i}{X}{F}$ que cumplan esta propiedad. Considere el $R$-módulo
        \begin{equation*}
            A=\sum_{ x\in X}R
        \end{equation*}
        (es decir, es la suma directa de $\abs{X}$-veces el $R$-módulo izquierdo $R$). Sea
        \begin{equation*}
            Y=\left\{\theta_x\Big| x\in X \right\}
        \end{equation*}
        donde
        \begin{equation*}
            \theta_x(y)=\left\{
                \begin{array}{lcr}
                    1_R & \textup{ si } & y=x\\
                    0_R & \textup{ si } & y\neq x\\
                \end{array}
            \right.,\quad\forall y\in Y
        \end{equation*}
        Como $X$ es no vacío, entonces $Y$ es no vacío. Por la parte $(iii)\Rightarrow(i)$, se sabe que $Y$ es una base del $R$-módulo unitario $A$. En particular, como $(iii)\Rightarrow(iv)$, se tiene que $A$ es un $R$-módulo libre en la categoría de $R$-módulos unitarios.
        
        En particular, $F$ y $A$ son $R$-módulos libres en la categoría de $R$-módulos unitarios y son tales que $\abs{X}=\abs{Y}$ (por la forma en que se construyó $Y$), luego por el Teorema anterior son equivalentes en esta categoría, es decir que existe un isomorfismo $\cf{f}{F}{A}$. Así que
        \begin{equation*}
            F\cong\sum_{ x\in X}R
        \end{equation*}
        lo que prueba el resultado.
    \end{proof}

\end{document}