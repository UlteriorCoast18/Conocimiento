\documentclass[12pt]{report}
\usepackage[spanish]{babel}
\usepackage[utf8]{inputenc}
\usepackage{amsmath}
\usepackage{amssymb}
\usepackage{amsthm}
\usepackage{graphics}
\usepackage{subfigure}
\usepackage{lipsum}
\usepackage{array}
\usepackage{multicol}
\usepackage{enumerate}
\usepackage[framemethod=TikZ]{mdframed}
\usepackage[a4paper, margin = 1.5cm]{geometry}
\usepackage{bbm}
\usepackage{tikz}
\usepackage{pgffor}
\usepackage{ifthen}

\usetikzlibrary{shapes.multipart}

\newcounter{it}
\newcommand*\watermarktext[1]{\begin{tabular}{c}
    \setcounter{it}{1}%
    \whiledo{\theit<100}{%
    \foreach \col in {0,...,15}{#1\ \ } \\ \\ \\
    \stepcounter{it}%
    }
    \end{tabular}
    }

\AddToHook{shipout/foreground}{
    \begin{tikzpicture}[remember picture,overlay, every text node part/.style={align=center}]
        \node[rectangle,black,rotate=30,scale=2,opacity=0.08] at (current page.center) {\watermarktext{Cristo Daniel Alvarado ESFM\quad}}; 
  \end{tikzpicture}
}
%En esta parte se hacen redefiniciones de algunos comandos para que resulte agradable el verlos%

\renewcommand{\theenumii}{\roman{enumii}}

\def\proof{\paragraph{Demostración:\\}}
\def\endproof{\hfill$\blacksquare$}

\def\sol{\paragraph{Solución:\\}}
\def\endsol{\hfill$\square$}

%En esta parte se definen los comandos a usar dentro del documento para enlistar%

\newtheoremstyle{largebreak}
  {}% use the default space above
  {}% use the default space below
  {\normalfont}% body font
  {}% indent (0pt)
  {\bfseries}% header font
  {}% punctuation
  {\newline}% break after header
  {}% header spec

\theoremstyle{largebreak}

\newmdtheoremenv[
    leftmargin=0em,
    rightmargin=0em,
    innertopmargin=-2pt,
    innerbottommargin=8pt,
    hidealllines = true,
    roundcorner = 5pt,
    backgroundcolor = gray!60!red!30
]{exa}{Ejemplo}[section]

\newmdtheoremenv[
    leftmargin=0em,
    rightmargin=0em,
    innertopmargin=-2pt,
    innerbottommargin=8pt,
    hidealllines = true,
    roundcorner = 5pt,
    backgroundcolor = gray!50!blue!30
]{obs}{Observación}[section]

\newmdtheoremenv[
    leftmargin=0em,
    rightmargin=0em,
    innertopmargin=-2pt,
    innerbottommargin=8pt,
    rightline = false,
    leftline = false
]{theor}{Teorema}[section]

\newmdtheoremenv[
    leftmargin=0em,
    rightmargin=0em,
    innertopmargin=-2pt,
    innerbottommargin=8pt,
    rightline = false,
    leftline = false
]{propo}{Proposición}[section]

\newmdtheoremenv[
    leftmargin=0em,
    rightmargin=0em,
    innertopmargin=-2pt,
    innerbottommargin=8pt,
    rightline = false,
    leftline = false
]{cor}{Corolario}[section]

\newmdtheoremenv[
    leftmargin=0em,
    rightmargin=0em,
    innertopmargin=-2pt,
    innerbottommargin=8pt,
    rightline = false,
    leftline = false
]{lema}{Lema}[section]

\newmdtheoremenv[
    leftmargin=0em,
    rightmargin=0em,
    innertopmargin=-2pt,
    innerbottommargin=8pt,
    roundcorner=5pt,
    backgroundcolor = gray!30,
    hidealllines = true
]{mydef}{Definición}[section]

\newmdtheoremenv[
    leftmargin=0em,
    rightmargin=0em,
    innertopmargin=-2pt,
    innerbottommargin=8pt,
    roundcorner=5pt
]{excer}{Ejercicio}[section]

%En esta parte se colocan comandos que definen la forma en la que se van a escribir ciertas funciones%

\newcommand\abs[1]{\ensuremath{\left|#1\right|}}
\newcommand\divides{\ensuremath{\bigm|}}
\newcommand\cf[3]{\ensuremath{#1:#2\rightarrow#3}}
\newcommand\natint[1]{\ensuremath{\left[\!\left[ #1\right]\!\right]}}
\newcommand{\afa}{\:
    \begin{tikzpicture}
        \draw [line width = 0.17 mm, black] (0,0) -- (-0.115,0.29);
        \draw [line width = 0.17 mm, black] (0,0) -- (0.115,0.29);
        \draw [line width = 0.17 mm, black] (-0.12,0) arc (190:-10:0.12cm);
    \end{tikzpicture}
    \:
}
\newcommand{\bbm}[1]{\ensuremath{\mathbbm{#1}}}
\newcommand\contradiction{\ensuremath{\#_c}}
\newcommand{\V}[1]{\ensuremath{\mathcal{V}(#1)}}
\newcommand{\Int}[1]{\ensuremath{\mathring{#1}}}
\newcommand{\Cls}[1]{\ensuremath{\overline{#1}}}
\newcommand{\Fr}[1]{\ensuremath{\textup{Fr}(#1)}}
\newcommand{\floor}[1]{\ensuremath{\lfloor#1\rfloor}}
\newcommand{\Card}[1]{\ensuremath{\textup{Card}\left(#1\right)}}
\newcommand{\Pot}[1]{\ensuremath{\mathcal{P}\left(#1\right)}}
\newcommand{\id}[1]{\ensuremath{\textup{id}_{#1}}}
%Este símvolo es para casi todo salvo una cantidad finita

%recuerda usar \clearpage para hacer un salto de página

\begin{document}
    \setlength{\parskip}{5pt} % Añade 5 puntos de espacio entre párrafos
    \setlength{\parindent}{12pt} % Pone la sangría como me gusta
    \title{Notas Curso Topología II}
    \author{Cristo Daniel Alvarado}
    \maketitle

    \tableofcontents %Con este comando se genera el índice general del libro%

    %\setcounter{chapter}{3} %En esta parte lo que se hace es cambiar la enumeración del capítulo%
    
    \chapter{Metrizabilidad}
    
    \section{Introducción}
    
    ¿Cuándo un espacio topológico es metrizable? Supongamos que tenemos un espacio topológico $(X,\tau)$, queremos una métrica $\cf{d}{X\times X}{\mathbb{R}}$ tal que $\tau_d=\tau$.

    La respuesta a esta pregunta es que no siempre será posible encontrar tal métrica. Por ejemplo, tome cualquier espacio topológico que no sea $T_1$.

    \begin{itemize}
        \item Pável Urysohn 1898-1924. El Lema de Urysohn fue publicado en 1924 póstumo a la muerte de su autor.
        \item Primera guerra mundial 28 de julio de 1914 a 11 de noviembre de 1918, inició con el asesinato del Archiduque Franciso de Austria.
        \item Segunda guerra mundial 1939 a 1945, cuando Hitler invade Polonia.
        \item En 1950 Bing, Nagata y Morita resuelven el problema de metrizabilidad de espacios topológicos.
    \end{itemize}

    Lo que veremos a continuación tiene como base fundamental el siguiente lema:

    \begin{lema}[\textbf{Lema de Urysohn}]
        Sea $(X,\tau)$ espacio topológico. Entonces, $(X,\tau)$ es $T_4$ si y sólo si dados $A,B\subseteq X$ cerrados disjuntos existe una función continua $\cf{f}{X}{[0,1]}$ tal que
        \begin{equation*}
            f(A)=\left\{0\right\} \quad\textup{y}\quad f(B)=\left\{1\right\}
        \end{equation*}
    \end{lema}

    Este lema se probó en el curso pasado.

    \begin{propo}
        Sea $(X,\tau)$ un espacio topológico segundo numerable. Entonces
        \begin{enumerate}
            \item $(X,\tau)$ es primero numerable.
            \item $(X,\tau)$ es de Lindelöf.
            \item $(X,\tau)$ es separable.
        \end{enumerate}
    \end{propo}

    \begin{proof}
        Sea $\mathcal{B}=\left\{B_i \right\}_{i\in\mathbb{N}}$ una base numerable para $\tau$.
        
        De (1): Sea $x\in X$. Tomemos
        \begin{equation*}
            \mathcal{B}_x=\left\{B_n\in\mathcal{B}\Big|x\in B_n \right\}
        \end{equation*}
        este es un conjunto no vacío pues al ser $\mathcal{B}$ base, existe $B\in\mathcal{B}$ tal que $x\in B$. Además es a lo sumo numerable por ser subcolección de $\mathcal{B}$.

        Sea $U\subseteq X$ abierto tal que $x\in U$. Como $\mathcal{B}$ es base de $\tau$, existe $B\in\mathcal{B}$ tal que $x\in B\subseteq U$, luego $B\in\mathcal{B}_x$. Por tanto, $\mathcal{B}_x$ es un sistema fundamental de vecindades de $x$. Al ser el $x$ arbitrario, se sigue que $(X,\tau)$ es primero numerable.

        De (2): Sea $\mathcal{A}=\left\{A_\alpha\right\}_{ \alpha\in I}$ una cubierta abierta de $(X,\tau)$. Dado $x\in X$ existe $A_\alpha\in\mathcal{A}$ tal que $x\in A_\alpha$, como $A_\alpha\in\tau$, existe $B_x\in\mathcal{B}$ tal que
        \begin{equation*}
            x\in B_x\subseteq A_\alpha
        \end{equation*}
        Sea
        \begin{equation*}
            \mathcal{K}=\left\{n\in\mathbb{N}\Big|\exists A_\alpha\in\mathcal{A}\textup{ tal que }B_n\subseteq A_\alpha \right\}
        \end{equation*}
        por la observación anterior, esta colección es no vacía. Dado $k\in\mathcal{K}$ escogemos un único $A_{\alpha_k}$ tal que
        \begin{equation*}
            B_k\subseteq A_{\alpha_k}
        \end{equation*}
        Sea
        \begin{equation*}
            \mathcal{A}'=\left\{A_{\alpha_k} \right\}_{ k\in\mathcal{K}}
        \end{equation*}
        se tiene que $\mathcal{A}'\subseteq\mathcal{A}$ es numerable. Sea $x\in X$, Como $\mathcal{A}$ es cubierta, existe $A'\in\mathcal{A}$ tal que
        \begin{equation*}
            x\in A'\in\tau
        \end{equation*}
        luego, al ser $\mathcal{B}$ base existe $B_n\in\mathcal{B}$ tal que
        \begin{equation*}
            x\in B_n\subseteq A'
        \end{equation*}
        Se sigue pues que $x\in A_{ \alpha_n}$. Por ende, $x\in\bigcup_{ n\in\mathbb{N}}A_{\alpha_n}$. Así, $\mathcal{A}$ posee una subcubierta a lo sumo numerable. Se sigue que al ser la cubierta abierta arbitraria que el espacio $(X,\tau)$ es Lindelöf.

        De (3): Ejercicio.
    \end{proof}

    \begin{propo}
        Si $(X,\tau)$ es metrizable, entonces los coneptos de espacio de Lindelöf, espacio separable y espacio segundo numerable son equivalentes.
    \end{propo}

    \begin{proof}
        Probaremos que Lindelöf implica separabilidad que implica segunda numerabilidad.

        Suponga que $(X,\tau)$ es metrizable, entonces existe una métrica $\cf{d}{X\times X}{\mathbb{R}}$ tal que $\tau_d=\tau$.
        \begin{itemize}
            \item Suponga que $(X,\tau)$ es Lindelöf. Sea $n\in\mathbb{N}$ y tomemos
            \begin{equation*}
                \mathcal{U}_n=\left\{B_d\left(x,\frac{1}{n}\right) \Big|x\in X \right\}
            \end{equation*}
            $\mathcal{U}_n$ es una cubierta abierta de $(X,\tau)$. Como el espacio de Lindelöf, existe $\mathcal{V}_n$ a lo sumo numerable tal que
            \begin{equation*}
                \mathcal{V}_n=\left\{B_d\left(y,\frac{1}{n} \right)\Big|y\in Y_n \right\}
            \end{equation*}
            siendo $Y_n\subseteq X$ un conjunto a lo sumo numerable, de tal suerte que $\mathcal{V}_n$ es subcubierta de $\mathcal{U}_n$. Sea
            \begin{equation*}
                A=\bigcup_{n\in\mathbb{N}}Y_n
            \end{equation*}
            este es un conjunto a lo sumo numerable. Sea $U\in\tau$ con $U\neq\emptyset$. Como $U\neq\emptyset$, existe $x\in U$, así existe $\varepsilon>0$ tal que $B_d(x,\varepsilon)\subseteq U$. Sea $m\in\mathbb{N}$ tal que $\frac{1}{m}<\varepsilon$. Tenemos que $\mathcal{V}_m$ es una cubierta de $X$, luego existe $y\in Y_m$ tal que
            \begin{equation*}
                x\in B_d\left(y,\frac{1}{m}\right)
            \end{equation*}
            Por tanto, $y\in B_d\left(x,\frac{1}{m} \right)\subseteq B(x,\varepsilon)\subseteq U$, así $y\in U$. Pero como $y\in Y_m$ se tiene que $y\in A$. Por ende
            \begin{equation*}
                U\cap A\neq\emptyset
            \end{equation*}
            lo que prueba el resultado.

            \item Suponga que $(X,\tau)$ es separable, entonces existe $A\subseteq X$ subconjunto denso a lo sumo numerable. Sea
            \begin{equation*}
                \mathcal{B}=\left\{B_d\left(a,\frac{1}{n}\right) \Big|a\in A\textup{ y }n\in\mathbb{N} \right\}
            \end{equation*}
            Si probamos que $\mathcal{B}$ es base para $\tau$, se probará el resultado (pues $\mathcal{B}$ es a lo sumo numerable). Sea $x\in X$ y $\varepsilon>0$. Tomemos $m\in\mathbb{N}$ tal que
            \begin{equation*}
                \frac{2}{m}<\varepsilon
            \end{equation*}
            como $\overline{A}=X$, entonces existe $a\in A$ tal que $a\in B_d\left(x,\frac{1}{m}\right)$. Entonces
            \begin{equation*}
                x\in B_d\left(a,\frac{1}{m}\right)\subseteq B_d\left(x,\frac{2}{m} \right)\subseteq B_d\left(x,\varepsilon\right)
            \end{equation*}
            por tanto, $\mathcal{B}$ es una base para la topología $\tau$, luego el espacio $(X,\tau)$ es segundo numerable.
        \end{itemize}
    \end{proof}

    \begin{exa}
        Considere el espacio topológico $(\mathbb{R},\leq)$. Entonces el conjunto
        \begin{equation*}
            \mathcal{B}_l=\left\{[a,b)\Big|a,b\in\mathbb{R} \right\}
        \end{equation*}
        es una base para una topología sobre $\mathbb{R}$. La topología generada por esta base la denotamos por $\tau_l$ y se dice \textbf{la topología del límite inferior}.
    \end{exa}

    \begin{exa}
        El espacio $(\mathbb{R},\tau_l)$ es $T_2$. Dados $a,b\in\mathbb{R}$ se tiene que si $a<x<b$.
        \begin{equation*}
            (a,b)=\bigcup\left\{[x,b)\Big|a<x<b \right\}
        \end{equation*}
        por tanto, $\tau_u\subseteq\tau_l$, luego $(\mathbb{R},\tau_l)$ es $T_2$ pues con la topología usual lo es.

        Más aún, $(\mathbb{R},\tau_l)$ es primero numerable.
    \end{exa}

    \begin{proof}
        En efecto, sea $x\in\mathbb{R}$. Afirmamos que la colección
        \begin{equation*}
            \left\{[x,x+1/n)\Big|n\in\mathbb{N} \right\}
        \end{equation*}
        es un sistema fundamnetal de vecindades de $x$, por lo que este espacio es primero numerable.
    \end{proof}

    
    \begin{exa}
        El espacio $(\mathbb{R},\tau_l)$ no es segundo numerable.
    \end{exa}

    \begin{proof}
        Sea $\mathcal{B}$ una base para $\tau_l$. Para $x\in\mathbb{R}$ escogemos $B_x\in\mathcal{B}$ tal que
        \begin{equation*}
            x\in B_x\subseteq [x,x+1)
        \end{equation*}
        Se tiene que $x=\inf B_x$. Para $x,y\in\mathbb{R}$ se tiene que $B_x\neq B_y$ (pues si fueran iguales, tendrían el mismo ínfimo). Por tanto la colección $\mathcal{B}$ es no numerable.

        Así, el espacio $(\mathbb{R},\tau_l)$ no es segundo numerable.
    \end{proof}

    \begin{exa}
        El espacio $(\mathbb{R},\tau_l)$ es separable.
    \end{exa}

    \begin{proof}
        Tome $\mathbb{Q}\subseteq\mathbb{R}$.
    \end{proof}
    
    \begin{exa}
        $(\mathbb{R},\tau_l)$ es normal.
    \end{exa}

    \begin{proof}
        Sean $A,B\subseteq\mathbb{R}$ cerrados tales que $A\cap B=\emptyset$. Sea $a\in A$, entonces $a\notin B=\overline{B}$. Existe pues $x_a\in\mathbb{R}$ tal que
        \begin{equation*}
            [a,x_a)\subseteq \mathbb{R}-B
        \end{equation*}
        (por ser el conjunto de la derecha abierto). Entonces
        \begin{equation*}
            A\subseteq\bigcup_{ a\in A}[a,x_a)=U\in\tau_l
        \end{equation*}
        y
        \begin{equation*}
            B\subseteq\bigcup_{ b\in B}[b,x_b)=V\in\tau_l
        \end{equation*}
        Si $U\cap V\neq\emptyset$, entonces existe $a\in A$ y $b\in B$ tales que
        \begin{equation*}
            [a,x_a)\cap[b,x_b)\neq\emptyset
        \end{equation*}
        Si $a<b$ entonces $b\in [a,x_a)$, lo cual es una contradición. Por tanto, $U\cap V=\emptyset$. Así, el espacio $(\mathbb{R},\tau_l)$ es normal.
    \end{proof}

    \begin{propo}
        Si $(X,\tau)$ es metrizable, entonces $(X,\tau)$ es normal.
    \end{propo}

    \begin{proof}
        Sea $d$ una métrica definida sobre $X$ tal que $\tau_d=\tau$. Como $(X,\tau)$ es metrizable, entonces es $\mathbb{T}_2$ y por lo tanto es $T_1$. Veamos que $(X,\tau)$ es $T_4$.

        Sean $A,B\subseteq X$ cerrados disjuntos con $A\cap B\neq\emptyset$. Sea $a\in A$, entonces $a\in X-B\in\tau$. Entonces existe $\varepsilon_a>0$ tal que
        \begin{equation*}
            B_d(a,\varepsilon_a)\subseteq X-B
        \end{equation*}
        Sea
        \begin{equation*}
            U=\bigcup_{ a\in A}B_d\left(a,\frac{\varepsilon_a}{2}\right) \in\tau
        \end{equation*}
        es claro que $A\subseteq U$. De forma análoga se construye $V$:
        \begin{equation*}
            V=\bigcup_{ b\in B}B_d\left(b,\frac{\varepsilon_b}{2}\right) \in\tau
        \end{equation*}
        es tal que $B\subseteq V$. Suponga que $U\cap V\neq\emptyset$. Entonces existe $a\in A$ y $b\in B$ tales que
        \begin{equation*}
            B_d\left(a,\frac{\varepsilon_a}{2}\right)\cap B_d\left(b,\frac{\varepsilon_b}{2}\right)\neq\emptyset
        \end{equation*}
        se tiene que $d(a,b)<d(a,x)+d(x,b)<\frac{\varepsilon_a}{2}+\frac{\varepsilon_b}{2}<\max\left\{\varepsilon_a,\varepsilon_b \right\}$. Por tanto, $a\in B_d(b,\varepsilon_b)$ o $b\in B_d(a,\varepsilon_a)$, lo cual contradice la elección de estas bolas. Por tanto, $U\cap B=\emptyset$.

        Así, el espacio $(X,\tau)$ es $T_4$.
    \end{proof}

    \begin{cor}
        Si $(X,\tau)$ es metrizable, entonces es regular.
    \end{cor}

    \begin{proof}
        Inmediato del hecho que normalidad implica regularidad.
    \end{proof}

    \begin{propo}
        Si $(X,\tau)$ es metrizable, entonces $(X,\tau)$ es primero numerable.
    \end{propo}

    \begin{proof}
        Sea $d$ una métrica definida sobre $X$ tal que $\tau=\tau_d$. Sea $x\in X$, considere
        \begin{equation*}
            \mathcal{V}=\left\{B_d\left(x,\frac{1}{n}\right)\Big|n\in\mathbb{N} \right\}
        \end{equation*}
        entonces $\mathcal{V}$ es una colección numerable de vecindades de $X$ y es fundamental (por construcción). Por tanto, $(X,\tau)$ es primero numerable.
    \end{proof}

    \begin{propo}
        Sea $(X,\tau)$ un espacio $T_3$ y de Lindelöf, entonces $(X,\tau)$ es $T_4$
    \end{propo}

    \begin{proof}
        Sean $A,B\subseteq X$ cerrados disjuntos. Sea $a\in A\subseteq X-B\in\tau$. Como $(X,\tau)$ es $T_3$, existe $U_a\in\tau$ tal que
        \begin{equation*}
            a\in U_a\subseteq\overline{U}_a\subseteq X-B
        \end{equation*}
        Por ser $(X,\tau)$ de Lindelöf y ser $A\subseteq X$ cerrado, tenemos que $(A,\tau_A)$ es de Lindelöf. Se tiene que
        \begin{equation*}
            A\subseteq\bigcup_{ a\in A}U_a
        \end{equation*}
        donde $U_a\in\tau$ y $\overline{U}_a\cap B\neq\emptyset$. Existe pues $\left\{U_{ a_n} \right\}_{ n\in\mathbb{N}}$ tales que
        \begin{equation*}
            A\subseteq\bigcup_{ n\in\mathbb{N}}U_{ a_n}U_{ a_n}
        \end{equation*}
        y cumplen que
        \begin{equation*}
            \overline{U}_{ a_n}\cap B=\emptyset,\quad\forall n\in\mathbb{N}
        \end{equation*}
        De forma análoga podemos encontrar una familia $\left\{V_{ b_n} \right\}_{ n\in\mathbb{N}}$ de abiertos tales que
        \begin{equation*}
            V\subseteq\bigcup_{ n\in\mathbb{N}}U_{ b_n}V_{ b_n}
        \end{equation*}
        y que cumplan:
        \begin{equation*}
            \overline{V}_{ b_n}\cap A=\emptyset,\quad\forall n\in\mathbb{N}
        \end{equation*}
        Sea $m\in\mathbb{N}$. Se define
        \begin{equation*}
            U_m=U_{ a_m}-\bigcup_{ l=1}^m\overline{V}_{b_l}\in\tau
        \end{equation*}
        y $V_m$ se define de forma similar:
    \end{proof}

    \begin{obs}
        Por el ejemplo de $(\mathbb{R},\tau_l)$, se sigue que el recíproco de esta proposición anterior no es cierta.
    \end{obs}

    \begin{obs}
        Del ejemplo anterior se deduce de forma inmediata que el recíproco del teorema anterior no es cierto.
    \end{obs}

    El objetivo de los siguientes resultados va a ser el de probar estos siguientes dos teoremas:

    \begin{theor}[\textbf{Teorema de Urysohn}]
        Si $(X,\tau)$ es un espacio normal y segundo numerable, entonces es metrizable.
    \end{theor}

    \begin{theor}[\textbf{Teorema de Tychonoff}]
        Si $(X,\tau)$ es un espacio regular y segundo numerable, entonces es metrziable.
    \end{theor}

    los cuales caracterizan en su totalidad a los espacios metrizables.

    Notemos antes que se cumple lo siguiente (dados los resultados probados anteriormente):
    \begin{center}
        Metrizabilidad $\Rightarrow$ Normalidad $\Rightarrow$ Regularidad
    \end{center}
    pero, más adelante se verá que
    \begin{center}
        Metrizabilidad $\nRightarrow$ Segunda numerabilidad
    \end{center}
    y,
    \begin{center}
        Normalidad y primero numerabilidad $\nRightarrow$ Metrizabilidad
    \end{center}

    \begin{mydef}
        Para todo $n\in\mathbb{N}\cup\left\{0\right\}$ se define:
        \begin{equation*}
            \mathcal{D}_n=\left\{0,\frac{1}{2^n},\frac{2}{2^{n}}...,\frac{2^n-1}{2^n},1\right\}
        \end{equation*}
        y con ello, se construye el subconjunto de $\mathbb{Q}$:
        \begin{equation*}
            \mathcal{D}=\bigcup_{ n=0}^\infty\mathcal{D}_n
        \end{equation*}
    \end{mydef}

    \begin{propo}
        Sea $[0,1]$ como subespacio de $(\mathbb{R},\tau_u)$, entonces $\mathcal{D}$ es denso en $([0,1],{\tau_u}_{[0,1]})$.
    \end{propo}

    \begin{proof}
        Es inmediata.
    \end{proof}

    \section{Lema de Urysohn e implicaciones}

    \begin{lema}[\textbf{Lema de Urysohn}]
        Sea $(X,\tau)$ un espacio topológico. Entonces, $(X,\tau)$ es $T_4$ si y sólo si para todos $A,B\subseteq X$ cerrados disjuntos, existe una función continua $\cf{f}{(X,\tau)}{([0,1],\tau_u)}$ tal que $f(A)=\left\{1\right\}$ y $f(B)=\left\{0\right\}$.
    \end{lema}

    \begin{proof}
        $\Rightarrow):$ Para probar el resultado, debemos hacer varias cosas antes:
        \begin{enumerate}
            \item Sea
            \begin{equation*}
                P=\mathbb{Q}\cap[0,1]
            \end{equation*}
            Nuestro objetivo es que para cada $p\in P$ le asignemos un conjunto abierto $U_p\subseteq X$ tal que si $p,q\in P$ son tales que
            \begin{equation*}
                p<q\Rightarrow \Cls{U}_p \subseteq U_q
            \end{equation*}
            de esta forma, la familia $\left\{U_p\Big|p\in P \right\}$ estará simplemente ordenada de la misma forma en la que sus subíndices lo están en $P$. Como el conjunto $P$ es numerable, podemos usar inducción para definir cada uno de los $U_p$. Ordenemos los elementos de $P$ en una sucesión de tal forma que los números $0$ y $1$ son los primeros de la sucesión (denotada de ahora en adelante por $\left\{p_n \right\}_{n=1}^\infty$).

            Definiremos ahora los conjuntos $U_p$ como sigue: defina
            \begin{equation*}
                U_1=X-B
            \end{equation*}
            Como $A$ es un cerrado contenido en $U_1$, por ser $(X,\tau)$ $T_4$, se tiene que existe un conjunto abierto $U_0\subseteq X$ tal que
            \begin{equation*}
                A\subseteq U_0\quad\textup{y}\quad\Cls{U}_0\subseteq U_1
            \end{equation*}
            En general, sea $P_n$ el conjunto de los primeros $n$ números racionales en la sucesión de los elementos de $P$. Suponga que $U_p$ está definido para cada $p\in P_n$ y, satisface la condición:
            \begin{equation*}
                p,q\in P_n\textup{ tal que }p<q\Rightarrow \Cls{U}_p\subseteq U_q
            \end{equation*}
            Sea $r$ el siguiente número racional en la sucesión $\left\{p_n \right\}_{n=1}^\infty$, esto es $r=p_{ n+1}$. Definiremos $U_r$.

            Considere el conjunto
            \begin{equation*}
                P_{ n+1}=P_n\cup\left\{r \right\}
            \end{equation*}
            Este es un subconjunto finito del intervalo $[0,1]$ y, tiene un orden simple derivado del orden simple $<$ de $[0,1]$.

            En un conjunto finito simplemente ordenado, todo elemento tiene un predecesor inmediato y un sucesor inmediato. El número $0$ es el elemento más pequeño y, $1$ es el elemento más grande de $P_{n+1}$ y, $r$ no es $0$ o $1$. Por tanto, $r$ tiene un sucesor y un predecesor inmediato, denotados respectivamente por $q$ y $p$. Los conjuntos $U_p$ y $U_q$ están definidos y son tales que
            \begin{equation*}
                \Cls{U}_p\subseteq U_q
            \end{equation*}
            por hipótesis de inducción. Como $(X,\tau)$ es $T_4$, entonces existe un conjunto abierto $U_r\subseteq X$ tal que
            \begin{equation*}
                \Cls{U}_p\subseteq U_r\quad\textup{y}\quad\Cls{U}_r\subseteq U_q
            \end{equation*}
            Es claro (pues los conjuntos $U_p$ con $p\in P_n$ están ordenados por la contención), que
            \begin{equation*}
                p,q\in P_{ n+1}\textup{ tal que }p<q\Rightarrow \Cls{U}_p\subseteq U_q
            \end{equation*}
            Usando inducción, tenemos definidos los conjuntos $U_p$, para todo $p\in P$.

            \item Ahora que se tiene definido $U_p$ para todo número en $\mathbb{Q}\cap[0,1]$, extenderemos esta definición a todo $\mathbb{Q}$, haciendo
            \begin{equation*}
                \begin{split}
                    U_p&=\emptyset,\quad p<0\\
                    U_p&=X,\quad 1<p\\
                \end{split}
            \end{equation*}
            para todo $p\in\mathbb{Q}$. Se sigue cumpliendo que para todo $p,q\in\mathbb{Q}$
            \begin{equation*}
                p<q\Rightarrow \Cls{U}_p\subseteq U_q
            \end{equation*}
            \item Dado un punto $p\in X$, definamos el conjunto $\mathbb{Q}(x)$ como el conjunto de todos los números racionales $p\in\mathbb{Q}$ tales que los correspondientes $U_p$ contengan a $x$, es decir:
            \begin{equation*}
                \mathbb{Q}(x)=\left\{p\in\mathbb{Q}\Big|x\in U_p \right\}
            \end{equation*}
            Este conjunto no contiene a ningún número menor que $0$ ya que $x\notin U_p$ para todo $p\in\mathbb{Q}^-$, además, contiene a todo número mayor que $1$, pues $x\in U_p$ para todo $p\in\mathbb{Q}$, $p>1$. Por tanto, $\mathbb{Q}(x)$ es acotado inferiormente y no vacío, luego tiene ínfimo en el intervalo $[0,1]$. Defina
            \begin{equation*}
                f(x)=\inf\mathbb{Q}(x)=\inf\left\{p\in\mathbb{Q} \Big| x\in U_p \right\}
            \end{equation*}
            \item Afirmamos que $f$ es la función deseada. Si $x\in A$, entonces $x\in U_p$ para todo $p\in\mathbb{Q}_{\geq0}$, luego
            \begin{equation*}
                f(x)=\inf\mathbb{Q}(x)=0
            \end{equation*}
            Similarmente, si $x\in B$, entonces $x\notin U_p$ para todo $p\in\mathbb{Q}$ con $p\leq 1$. Luego, $\mathbb{Q}(x)$ consiste de todos los números racionales mayores a $1$ y, por ende, $f(x)=1$.

            Probaremos que $f$ es continua. Para ello, probaremos que se cumplen dos cosas:
            \begin{enumerate}
                \item $x\in\Cls{U}_r$ implica que $f(x)\leq r$.
                \item $x\notin U_r$ implica que $f(x)\geq r$.
            \end{enumerate}
            Para probar (1), notemos que si $x\in\Cls{U}_r$, entonces $x\in U_s$, para todo $s>r$. Entonces, $\mathbb{Q}(x)$ contiene a todos los números racionales mayores que $r$, así que, por definición tenemos que
            \begin{equation*}
                f(x)=\inf\mathbb{Q}(x)\leq r
            \end{equation*}

            Para probar (2), notemos que si $x\notin U_r$, entonces $x$ no está en $U_s$ para todo $s<r$. Por tanto, $\mathbb{Q}(x)$ no contiene números racionales menores que $r$, por lo cual
            \begin{equation*}
                f(x)=\inf\mathbb{Q}(x)\geq r
            \end{equation*}
            Ahora probaremos la continuidad de $f$. Sea $x_0\in X$ y un intervalo abierto $(c,d)$ en $\mathbb{R}$ tal que
            \begin{equation*}
                c<f(x_0)<d
            \end{equation*}
            podemos encontrar números racionales $p,q\in\mathbb{Q}$ tales que
            \begin{equation*}
                c<p<f(x_0)<q<d
            \end{equation*}
            Afirmamos que el conjunto
            \begin{equation*}
                U=U_q-\Cls{U}_p
            \end{equation*}
            es un abierto que cumple que $f(U)\subseteq(c,d)$ y es tal que $x_0\in U$. En efecto, notemos que $x_0\in U_q$ pues $f(x_0)<q$ implica por (2) que $f(x_0)\in U_q$ y, como $p<f(x_0)$, implica por (1) que $f(x_0)\notin \Cls{U}_p$. Por tanto, $f(x_0)\in U$.

            Sea $x\in U$, entonces $x\in U_q\subseteq \Cls{U}_q$, por lo cual de (1), $f(x)\leq q$ y, $x\notin \Cls{U}_p$ implica que $x\notin\Cls{U}_p$ por lo cual de (2) se sigue que $p\leq f(x)$. Por tanto, $f(x)\in[p,q]\subseteq (c,d)$.

            Luego, $f(U)\subseteq(c,d)$. Así, $f$ es continua en $x_0\in X$. Como el punto fue arbitrario, se sigue que $f$ es continua en $X$.
        \end{enumerate}
        Por los 4 incisos anteriores, se sigue el resultado.

        $\Leftarrow):$ Sean $A,B\subseteq X$ cerrados disjuntos. Por hipótesis existe una función continua $\cf{f}{(X,\tau)}{([0,1],\tau_u)}$ tal que $f(A)=1$ y $f(B)=0$. Los conjuntos $U=f^{-1}((r,1])$ $V=f^{-1}([0,r))$, donde $r\in(0,1)$, son dos abiertos (ya que $f$ es continua y $[0,r),(r,1],\in\tau_u$) tales que:
        \begin{equation*}
            A\subseteq U\quad B\subseteq V
        \end{equation*}
        y, $U\cap V=\emptyset$.
    \end{proof}

    \begin{exa}
        Sea $(X,\tau)$ un espacio topológico $T_4$ y $A,B\subseteq X$ cerrados disjuntos y considere al espacio $(A\cup B,\tau_{A\cup B})$. Sea $\cf{g}{(A\cup B,\tau_{A\cup B})}{([0,1],\tau_u)}$ la función definida como
        \begin{equation*}
            g(x)=\left\{ 
                \begin{array}{lcr}
                    0 &\textup{ si } & x\in A\\
                    1 &\textup{ si } & x\in B\\
                \end{array}
            \right.,\quad\forall x\in A\cup B
        \end{equation*}
        esta función es continua. Como $(X,\tau)$ es $T_4$, por el Lema de Urysohn existe $\cf{G}{(X,\tau)}{([0,1],\tau_u)}$ función continua tal que $G(A)=\left\{0 \right\}$ y $G(B)=\left\{1 \right\}$. Se tiene pues que $G$ es una extensión continua de la función $g$.
    \end{exa}

    \begin{excer}
        Pruebe que
        \begin{equation*}
            \frac{1}{3}\sum_{ i=1}^\infty\left(\frac{2}{3}\right)^{ i-1}=1
        \end{equation*}
        y,
        \begin{equation*}
            \frac{1}{3}\sum_{ i=n+1}^\infty\left(\frac{2}{3}\right)^{ i-1}=\left(\frac{2}{3}\right)^n
        \end{equation*}
        para todo $n\in\mathbb{N}$.
    \end{excer}

    \begin{proof}
        
    \end{proof}

    \begin{propo}
        Sea $(X,\tau)$ un espacio topológico $T_4$ y sea $A\subseteq X$ cerrado. Tomemos $r\in\mathbb{R}^+$ y considere $[-r,r]$ dotado de la topología usual. Si $\cf{f}{(A,\tau_A)}{([-r,r],\tau_u)}$ es una función continua, entonces existe una función continua $\cf{g}{(X,\tau)}{(\mathbb{R},\tau_u)}$ tal que
        \renewcommand{\theenumi}{\roman{enumi}}
        \begin{enumerate}
            \item $\forall x\in X$, $\abs{g(x)}\leq\frac{r}{3}$.
            \item $\forall a\in A$, $\abs{f(a)-g(a)}\leq\frac{2r}{3}$.
        \end{enumerate}
    \end{propo}

    \begin{proof}
        Definimos $I_1=[-r,-r/3]$, $I_2=[-r/3,r/3]$ e $I_3=[r/3,r]$. Hacemos $B=f^{-1}(I_1)$ y $C=f^{-1}(I_3)$. Como $f$ es continua entones $B$ y $C$ son dos cerrados en $(A,\tau_A)$, al ser $A$ cerrado en $X$, se sigue que $B$ y $C$ son cerrados en $X$ y además son disjuntos. Por el Lema de Urysohn existe una función continua $\cf{g}{(X,\tau)}{([-r/3,r/3],\tau_u)}$ tal que
        \begin{equation*}
            g(B)=\left\{ -r/3\right\}\textup{ y }g(C)=\left\{r/3\right\}
        \end{equation*}
        además, para todo $x\in X$ se cumple que $\abs{g(x)}\leq\frac{r}{3}$.

        Tenemos lo siguiente: sea $a\in A$, entonces:
        \begin{itemize}
            \item Si $a\in B$ se tiene que $f(a)\in I_1$ y $g(a)=-r/3$, por lo cual $f(a),g(a)\in I_1$, lo cual implica que
            \begin{equation*}
                \abs{f(a)-g(a)}\leq\frac{2r}{3}
            \end{equation*}
            \item Si $a\in C$ se tiene que $f(a)\in I_3$ y $g(a)=r/3$, por lo cual $f(a),g(a)\in I_3$, lo cual implica que
            \begin{equation*}
                \abs{f(a)-g(a)}\leq\frac{2r}{3}
            \end{equation*}
            \item $a\notin B\cup C$, entonces $f(a)\in I_2$ y ya se sabe que $g(a)\in I_2$, lo cual implica que
            \begin{equation*}
                \abs{f(a)-g(a)}\leq\frac{2r}{3}
            \end{equation*}
        \end{itemize}
        viendo a $g$ como una función de $(X,\tau)$ en $(\mathbb{R},\tau_u)$ se tiene el resultado.
    \end{proof}
    
    \begin{theor}[\textbf{Teorema de extensión de Tietze}]
        Sea $(X,\tau)$ un espacio topológico $T_4$. Considere $[a,b]$ como subespacio de $(\mathbb{R},\tau_u)$ con $a,b\in\mathbb{R}$ tales que $a<b$. Si $\cf{f}{(A,\tau_A)}{([a,b],\tau_u)}$ es una función continua, entonces existe una función continua $\cf{F}{(X,\tau)}{([a,b],\tau_u)}$ tal que para todo $a\in A$
        \begin{equation*}
            F(a)=f(a)
        \end{equation*}
    \end{theor}

    \begin{proof}
        La función $\cf{h}{([a,b],\tau_u)}{([-1,1],\tau_u)}$ definida como
        \begin{equation*}
            h(x)=\frac{2x-a-b}{b-a},\quad\forall x\in[a,b]
        \end{equation*}
        es un homeomorfismo.
        
        Si para la función continua $\cf{h\circ f}{(A,\tau_A)}{([-1,1],\tau_u)}$ existe una función continua $\cf{H}{(X,\tau)}{([-1,1],\tau_u)}$ tal que para todo $a\in A$, $H(a)=(h\circ f)(a)$, entonces $\cf{h^{-1}\circ H}{(X,\tau)}{([a,b],\tau_u)}$ es una función continua tal que
        \begin{equation*}
            (h^{-1}\circ H)(a)=f(a),\quad\forall a\in A
        \end{equation*}
        Tomando $F=h^{-1}\circ H$ se tiene el resultado. Por tanto, podemos suponer sin pérdida de generalidad que $\cf{f}{(A,\tau_A)}{([-1,1],\tau_u)}$. Podemos usar el resultado anterior para $r=1$.

        \begin{enumerate}
            \item Tenemos por la proposición anterior que existe una función continua $\cf{g_1}{(X,\tau)}{(\mathbb{R},\tau_u)}$ tal que
            \begin{itemize}
                \item $\forall x\in X$, $\abs{g_1(x)}\leq\frac{1}{3}$.
                \item $\forall a\in A$, $\abs{f(a)-g_1(a)}\leq\frac{2}{3}$.
            \end{itemize}
            \item Consideremos la función continua $\cf{f-g_1}{(A,\tau_A)}{([-2/3,2/3],\tau_u)}$ (esto por (1)). Para $r=\frac{2}{3}$ existe una función continua $\cf{g_2}{(X,\tau)}{(\mathbb{R},\tau_u)}$ tal que
            \begin{itemize}
                \item $\forall x\in X$, $\abs{g_2(x)}\leq\frac{1}{3}\cdot\frac{2}{3}$.
                \item $\forall a\in A$, $\abs{(f-g_1)(a)-g_2(a)}=\abs{(f-g_1-g_2)(a)}\leq\frac{2}{3}\cdot\frac{2}{3}$.
            \end{itemize}
            \item Suponga construidas funciones continuas $\cf{g_1,...,g_n}{(X,\tau)}{(\mathbb{R},\tau_u)}$ con $n\in\mathbb{N}$ tal que $n\geq 2$ tales que
            \begin{itemize}
                \item $\forall x\in X$, $\abs{g_i(x)}\leq\frac{1}{3}\cdot\left(\frac{2}{3}\right)^{i-1}$, para todo $i\in\natint{1,n}$.
                \item $\forall a\in A$, $\abs{f(a)-\sum_{ i=1}^n g_i(a)}\leq\left(\frac{2}{3}\right)^n$.
            \end{itemize}
            Entonces, por la proposición anterior para la función $\cf{f-\sum_{ i=1}^n g_i}{(A,\tau_A)}{\left(\left[-\left(\frac{2}{3}\right)^n,\left(\frac{2}{3}\right)^n\right],\tau_u\right) }$ existe una función continua $\cf{g_{ n+1}}{(X,\tau)}{(\mathbb{R},\tau_u)}$ tal que
            \begin{itemize}
                \item $\forall x\in X$, $\abs{g_{ n+1}(x)}\leq\frac{1}{3}\cdot\left(\frac{2}{3}\right)^n$.
                \item $\forall a\in A$, $\abs{(f-\sum_{ i=1}^n g_i)(a)-g_{ n+1}(a)}\leq\frac{2}{3}\cdot\left(\frac{2}{3}\right)^n=\left(\frac{2}{3}\right)^{n+1}$.
            \end{itemize}
        \end{enumerate}
        por inducción tenemos definida una sucesión de funciones $\left\{\cf{g_i}{(X,\tau)}{(\mathbb{R},\tau_u)} \right\}_{ i=1}^\infty$ que cumple las condiciones anteriores para todo $i\in\mathbb{N}$. Defina
        \begin{equation*}
            G_k=\sum_{ i=1}^n g_i
        \end{equation*}

        sean $m,k\in\mathbb{N}$, $x\in X$. Se tiene que
        \begin{equation*}
            \begin{split}
                \abs{G_{ n+k}(x)-G_n(x)}&=\abs{\sum_{ i=n+1}^{ n+k}g_i(x)}\\
                &\leq\sum_{ i=n+1}^{ n+k}\abs{g_i(x)}\\
                &\leq\frac{1}{3}\sum_{ i=n+1}^{ n+k}\left(\frac{2}{3}\right)^{ i-1}\\
                &<\frac{1}{3}\sum_{ i=n+1}^{\infty}\left(\frac{2}{3}\right)^{ i-1}\\
                &=\left(\frac{2}{3}\right)^n\\
            \end{split}
        \end{equation*}
        por ende, dado $\varepsilon>0$, existe $N\in\mathbb{N}$ tal que si $n\geq N$ se cumple que
        \begin{equation*}
            \abs{\sum_{ i=n+1}^{ n+k}g_i(x)}<\varepsilon,\quad\forall x\in X
        \end{equation*}
        y para todo $k\in\mathbb{N}$. Luego, dado $x\in X$ la serie $\sum_{ i=1}^\infty g_i(x)$ es convergente y así podemos definir una función $\cf{g}{(X,\tau)}{(\mathbb{R},\tau_u)}$ tal que
        \begin{equation*}
            g(x)=\sum_{ i=1}^\infty g_i(x)
        \end{equation*}
        Veamos que la función $\cf{g}{(X,\tau)}{(\mathbb{R},\tau_u)}$ es continua. Para todo $m\in\mathbb{N}$, $G_m$ es continua y para $x\in X$, $k,n\in\mathbb{N}$ con $n<k$ se tiene
        \begin{equation*}
            \begin{split}
                \abs{G_{n+k}(X)-G_n(x)}&=\abs{\sum_{ i=n+1}^{ n+k} g_i(x)}\\
                &<\left(\frac{2}{3}\right)^n\\
            \end{split}
        \end{equation*}
        para $n$ fijo y $k\rightarrow\infty$ tenemos que para todo $x\in X$:
        \begin{equation*}
            \abs{g(x)-G_n(x)}\leq\left(\frac{2}{3}\right)^n
        \end{equation*}
        por tanto, $G_n$ converge uniformemente a $g$, por ende $g$ es continua.

        Además, para todo $a\in A$
        \begin{equation*}
            \begin{split}
                \abs{f(a)-G_n(a)}&=\abs{f(a)-\sum_{ i=1}^n g_i(a)}\\
                &\leq\left(\frac{2}{3}\right)^n\\
            \end{split}
        \end{equation*}
        por tanto, para todo $a\in A$, $f(a)=g(a)$. Tomando $F=g$ se tiene el resultado.
    \end{proof}

    \begin{obs}
        Considere $\left(-\frac{\pi}{2},\frac{\pi}{2}\right)$ como subespacio de $(\mathbb{R},\tau_u)$. Sea $\cf{h}{(\mathbb{R},\tau_u)}{\left(\left(-\frac{\pi}{2},\frac{\pi}{2}\right),\tau_u\right)}$ la función definida como:
        \begin{equation*}
            h(x)=\arctan(x),\quad\forall x\in\left(-\frac{\pi}{2},\frac{\pi}{2}\right)
        \end{equation*}
        el cual es un homeomorfismo. Como a su vez el intervalo $\left(-\frac{\pi}{2},\frac{\pi}{2}\right)$ es homeomorfo a $(-1,1)$ como subespacios de $(\mathbb{R},\tau_u)$, tenemos que
        \begin{equation*}
            (\mathbb{R},\tau_u)\cong((-1,1),\tau_u)
        \end{equation*}
    \end{obs}

    \begin{obs}
        Además, si $(X,\tau)$ es un espacio topológico y $\cf{f}{(X,\tau)}{((-1,1),\tau_u)}$ es una función continua, entonces la función $\cf{F}{(X,\tau)}{([-1,1],\tau_u)}$ definida por
        \begin{equation*}
            F(x)=f(x),\quad\forall x\in X
        \end{equation*}
        es una función continua.
    \end{obs}

    \begin{propo}
        Sea $(X,\tau)$ un espacio $T_4$ y tomemos $A\subseteq X$ cerrado. Si $\cf{f}{(A,\tau_A)}{(\mathbb{R},\tau_u)}$ es una función continua, entonces existe una función continua $\cf{F}{(X,\tau)}{(\mathbb{R},\tau_u)}$ tal que
        \begin{equation*}
            F(a)=f(a),\quad\forall a\in A
        \end{equation*}
    \end{propo}

    \begin{proof}
        Podemos considerar a la función $\cf{f}{(A,\tau_A)}{([-1,1],\tau_u)}$ con $f(A)\subseteq(-1,1)$ (esto por las observaciones anteriores). Por el Teorema de extensión de Tietze existe una funcíón continua $\cf{\widetilde{F}}{(X,\tau)}{([-1,1],\tau_u)}$ tal que
        \begin{equation*}
            \widetilde{F}(a)=f(a),\quad\forall a\in A
        \end{equation*}
        Definimos
        \begin{equation*}
            D=\widetilde{F}^{-1}(\left\{-1\right\})\cup \widetilde{F}^{-1}(\left\{1\right\})
        \end{equation*}
        este es un conjunto cerrado (pues $\widetilde{F}$ es continua). Se tiene que $A\subseteq X$ es un cerrado tal que $f(A)\subseteq(-1,1)$, luego entonces
        \begin{equation*}
            A\cap D=\emptyset
        \end{equation*}
        Por el Lema de Urysohn existe una función continua $\cf{g}{(X,\tau)}{([0,1],\tau_u)}$ tal que
        \begin{equation*}
            g(D)=\left\{0 \right\}\quad\textup{y}\quad g(A)=\left\{1 \right\}
        \end{equation*}
        definimos $F=g\cdot \widetilde{F}$, esta es una función con dominio $(X,\tau)$ y contradominio $(\mathbb{R},\tau_u)$. Por ser producto de funciones continuas, esta es una función continua. Además, como para todo $x\in X$, $g(x)\in[0,1]$, se tiene que
        \begin{equation*}
            \abs{F(x)}=\abs{g(x)\cdot \widetilde{F}(x)}\leq \abs{\widetilde{F}(x)}\leq 1,\quad\forall x\in X
        \end{equation*}
        \begin{itemize}
            \item Si $x\in D$, entonces $g(x)=0$, esto es que $F(x)=0$.
            \item Si $x\notin D$, entonces $\widetilde{F}(x)\in(-1,1)$, luego $\abs{F(x)}<1$, se sigue que $F(x)\in(-1,1)$.
        \end{itemize}
        Luego, $\cf{F}{(X,\tau)}{((-1,1),\tau_u)}$ es una función continua tal que
        \begin{equation*}
            F(a)=g(a)\cdot \widetilde{F}(a)=1\cdot \widetilde{F}(a)=f(a),\quad\forall a\in A
        \end{equation*}
        Luego $F$ es la función continua buscada.
    \end{proof}

    \section{Espacios $T_{3.5}$ y Completamente Regulares}

    \begin{mydef}
        Sea $(X,\tau)$ un espacio topológico. Decimos que $(X,\tau)$ es \textbf{un espacio $T_{ 3.5}$} si dados $A\subseteq X$ cerrado no vacío y $x\notin A$, existe una función $\cf{f}{(X,\tau)}{([0,1],\tau_u)}$ tal que $f$ es continua y
        \begin{equation*}
            f(A)=\left\{1\right\}\quad\textup{y}\quad f(x)=0
        \end{equation*}
    \end{mydef}

    \begin{mydef}
        Un espacio topológico $(X,\tau)$ que es $T_{3.5}$ y $T_1$ se llama \textbf{espacio completamente regular} (también llamado \textbf{espacio de Tychonoff}).
    \end{mydef}

    \begin{exa}
        Sea $(X=\left\{0,1\right\},\tau_I=\left\{X,\emptyset \right\})$. Este espacio es $T_{3.5}$ por vacuidad y no es $T_1$.
    \end{exa}

    \begin{propo}
        La propiedad de ser un espacio $T_{3.5}$ se hereda.
    \end{propo}

    \begin{proof}
        Sea $(X,\tau)$ un espacio $T_{3.5}$ y $Y\subseteq X$. Sea $A\subseteq Y$ un conjunto cerrado no vacío de $(Y,\tau_Y)$. Sea $y\in Y-A$. Recordemos que
        \begin{equation*}
            \overline{A}^Y=A=\overline{A}\cap Y
        \end{equation*}
        (siendo $\overline{A}^Y$ la cerradura de $A$ en $Y$ y, $\overline{A}$ la cerradura de $A$ en $X$) donde en particular $y\in X-\overline{A}$. Tenemos pues que existe una función continua $\cf{f}{(X,\tau)}{([0,1],\tau_u)}$ tal que
        \begin{equation*}
            f\left(\overline{A}\right)=\left\{1\right\}\quad\textup{y}\quad f(y)=0
        \end{equation*}
        Entonces, la función $\cf{f\big|_{Y}}{(Y,\tau_Y)}{([0,1],\tau_u)}$ es una función continua tal que
        \begin{equation*}
            f\big|_{Y}(a)=f(a)=1,\quad\forall a\in A
        \end{equation*}
        pues $A\subseteq\overline{A}$, y $f\big|_{Y}(y)=f(y)=0$. Se sigue entonces que $(Y,\tau_Y)$ es un espacio $T_{3.5}$.
    \end{proof}

    \begin{cor}
        La propiedad de ser completamente regular se hereda.
    \end{cor}

    \begin{proof}
        Inmediata del hecho de que las propiedades $T_{3.5}$ y $T_1$ son hereditarias.
    \end{proof}

    \begin{excer}
        La propiedad de ser $T_{3.5}$ es topológica.
    \end{excer}

    \begin{proof}
        
    \end{proof}

    \begin{cor}
        La propiedad de ser completamente regular es topológica.
    \end{cor}

    \begin{proof}
        Es inmediata del ejercicio anterior y de que la propiedad de ser $T_1$ es topológica.
    \end{proof}

    \begin{propo}
        Si $(X,\tau)$ es un espacio topológico $T_{3.5}$, entonces $(X,\tau)$ es un espacio $T_3$.
    \end{propo}

    \begin{proof}
        Sea $A\subseteq X$ cerrado y $x\in X-A$ con $A\neq\emptyset$. Al ser $(X,\tau)$ espacio $T_{3.5}$, existe pues una función continua $\cf{f}{(X,\tau)}{([0,1],\tau_u)}$ tal que
        \begin{equation*}
            f(A)=\left\{1 \right\}\quad\textup{y}\quad f(x)=0
        \end{equation*}
        Sean $U=f^{-1}([0,1/2))$ y $V=f^{-1}((1/2,1])$, al ser $f$ función continua se tiene que $U,V\in\tau$ son disjuntos para los que se cumple que
        \begin{equation*}
            x\in U\quad\textup{y}\quad A\subseteq V
        \end{equation*}
        por tanto, $(X,\tau)$ es $T_3$.
    \end{proof}

    \begin{propo}
        Si $(X,\tau)$ es un espacio normal, entonces es completamente regular.
    \end{propo}

    \begin{proof}
        Suponga que $(X,\tau)$ es $T_4$ y $T_1$. Sean $A\subseteq X$ cerrado no vacío y $x\in X-A$. Como $(X,\tau)$ es $T_1$, el conjunto $B=\left\{x \right\}$ es cerrado para el que se cumple que $A\cap B=\emptyset$ siendo ambos conjuntos cerrados. Luego, por el Lema de Urysohn al ser $(X,\tau)$ un espacio $T_4$ existe una función continua $\cf{f}{(X,\tau)}{([0,1],\tau_u)}$ tal que
        \begin{equation*}
            f(A)=\left\{1 \right\}\quad\textup{y}\quad f(B)=\left\{0 \right\}
        \end{equation*}
        la segunda condición es equivalente a que $f(x)=0$.

        Por tanto, $(X,\tau)$ es $T_{3.5}$.
    \end{proof}

    Nos preguntamos ahora que sucede con el producto de espacios regulares.

    \begin{propo}
        Sea $\left\{(X_\alpha,\tau_\alpha) \right\}_{ \alpha\in I}$ una familia de esapcios topológicos y tomemos $X=\prod_{ \alpha\in I}X_\alpha$. Entonces, $(X,\tau_p)$ es $T_{3.5}$ si y sólo si $(X_\alpha,\tau_\alpha)$ lo es, para todo $\alpha\in I$.
    \end{propo}

    \begin{proof}
        $\Rightarrow$): Suponga que $(X,\tau_p)$ es $T_{3.5}$, como la propiedad de ser $T_{3.5}$ es hereditaria y topológica, se sigue de forma inmediata que $(X_\alpha,\tau_\alpha)$ es $T_{3.5}$, para todo $\alpha\in I$.

        $\Leftarrow$): Suponga que para todo $\alpha\in I$ se tiene que $(X_\alpha,\tau_\alpha)$ es $T_{3.5}$. Sea $A\subseteq X$ cerrado no vacío y $x\in X-A$. Tenemos que $X-A\in\tau_p$, por lo cual existe un básico $U\in\tau_p$ tal que
        \begin{equation*}
            U\subseteq X-A
        \end{equation*}
        siendo
        \begin{equation*}
            U=\prod_{\alpha\in I}U_\alpha
        \end{equation*}
        con $U_\alpha\in\tau_\alpha$ para todo $\alpha\in I$ y tal que $U_\alpha=X_\alpha\afa\alpha\in I$. Digamos que $J=\left\{\alpha_1,...,\alpha_m \right\}\subseteq I$ es tal que
        \begin{equation*}
            U_{\alpha_i}\neq X_\alpha,\quad\forall i\in\natint{1,m}
        \end{equation*}
        Se tiene que $x=\left(x_\alpha\right)_{\alpha\in I}\in U$, luego se cumple en particular para todo $i\in\natint{1,m}$, $x_{\alpha_i}\notin X_{\alpha_i}-U_{\alpha_i}$.

        Como $X_{\alpha_i}-U_{\alpha_i}$ es un cerrado no vacío que no contiene a $x_{\alpha_i}$, para todo $i\in\natint{1,m}$, al tenerse que $(X_{\alpha_i},\tau_{\alpha_i})$ es un espacio $T_{3.5}$, existe una función continua $\cf{f_{\alpha_i}}{(X_{\alpha_i},\tau_{\alpha_i})}{([0,1],\tau_u)}$ tal que:
        \begin{equation}
            f_{\alpha_i}\left(X_{\alpha_i}-U_{\alpha_i}\right)=\left\{0\right\}\quad\textup{y}\quad f_{\alpha_i}(x_{\alpha_i})=\left\{1\right\}
        \end{equation}

        Ahora, para $i\in\natint{1,m}$ consideremos la función proyección $\cf{p_{\alpha_i}}{(X,\tau_p)}{(X_{\alpha_i},\tau_{\alpha_i})}$. Definimos:
        \begin{equation*}
            g_{\alpha_i}=f_{\alpha_i}\circ p_{\alpha_i}
        \end{equation*}
        se tiene que $\cf{g_{\alpha_i}}{(X,\tau)}{([0,1],\tau_u)}$ es una función continua, para todo $i\in\natint{1,m}$. Definimos la función $\cf{f}{(X,\tau)}{([0,1],\tau_u)}$ dada por:
        \begin{equation*}
            f(x')=g_{\alpha_1}\cdot g_{\alpha_2}\cdots g_{\alpha_m}(x')
        \end{equation*}
        para todo $x'\in X$.

        Se tiene que la función $f$ es continua. Además, cumple que:
        \begin{equation*}
            \begin{split}
                f(x)&=g_{\alpha_1}\cdot g_{\alpha_2}\cdots g_{\alpha_m}(x)\\
                &=1\cdot 1\cdots 1\\
                &=1\\
            \end{split}
        \end{equation*}
        ahora, sea $a\in A$, se tiene que $a\notin U$, luego existe $i\in\natint{1,m}$ tal que $x_{\alpha_i}\in X_{\alpha_i}-U_{\alpha_i}$, por lo que $g_{\alpha_i}(x_{\alpha_i})=0$, esto es que $f(x)=0$.

        Así, $(X,\tau_p)$ es $T_{3.5}$.
    \end{proof}

    \begin{cor}
        Sea $\left\{(X_\alpha,\tau_\alpha) \right\}_{ \alpha\in I}$ una familia de esapcios topológicos y tomemos $X=\prod_{ \alpha\in I}X_\alpha$. Entonces, $(X,\tau_p)$ es completamente regular si y sólo si $(X_\alpha,\tau_\alpha)$ lo es, para todo $\alpha\in I$.
    \end{cor}

    \begin{proof}
        Inmediata del teorema anterior.
    \end{proof}

    \begin{obs}
        Se sabe que si $(X,\tau)$ es un espacio compacto y Hausdorff, entonces $(X,\tau)$ es normal y por ende, completamente regular.
    \end{obs}

    \begin{propo}
        Sea $(X,\tau)$ un espacio localmente compacto que no es compacto y además, es de Hausdorff, entonces $(X,\tau)$ es completamente regular.
    \end{propo}

    \begin{proof}
        Considere $(\hat{X},\hat{\tau})$ (la compactificación unipuntual de $(X,\tau)$). Sabemos que $(X,\tau)$ es un espacio compacto y Hausdorff, lo cual implica inmediatamente por la observación anterior que es normal (esto se probó el semestre pasado) y, en consecuencia, $(\hat{X},\hat{\tau})$ es completamente regular. En particular, como $(X,\tau)$ es subespacio de $(\hat{X},\hat{\tau})$ y la propiedad de ser completamente regular es hereditaria, entonces $(X,\tau)$ es completamente regular.
    \end{proof}

    \section{El Teorema de Metrizabilidad de Urysohn}

    \begin{mydef}
        Considere el espacio de sucesiones $l_2(\mathbb{R})$ de sucesiones dos convergentes, es decir que $\left\{x_n \right\}_{ n=1}^\infty\in l_2(\mathbb{R})$ si y sólo si
        \begin{equation*}
            \sum_{ n=1}^\infty\abs{x_n}^2<\infty
        \end{equation*}
        Se define una métrica $\rho$ sobre $l_2(\mathbb{R})$ dada por:
        \begin{equation*}
            \rho(x,y)=\left[\sum_{ n=1}^\infty(x_n-y_n)^2\right]^{1/2}
        \end{equation*}
        Se define el \textbf{Cubo de Hilbert} como el subespacio métrico de $l_2(\mathbb{R})$ dado por:
        \begin{equation*}
            \mathcal{H}=\left\{\left\{x_n \right\}_{ n=1}^\infty\Big|\abs{x_n}\leq\frac{1}{n}\textup{ para todo }n\in\mathbb{N} \right\}
        \end{equation*}
    \end{mydef}

    \begin{theor}[\textbf{Teorema de metrización de Urysohn}]
        Todo espacio regular $(X,\tau)$ y segundo numerable es metrzable.
    \end{theor}

    \begin{proof}
        Como $(X,\tau)$ es regular, entonces es $T_3$ y $T_1$. Al ser segundo numerable, se tiene que es Lindelöf. Por la proposición 1.1.5 se tiene que $(X,\tau)$ es $T_4$.

        Sea $\mathcal{B}=\left\{B_n \right\}_{ n\in\mathbb{N}}$ una base numerable para $\tau$. Por ser base se cumple:
        \begin{itemize}
            \item[$(*)$:] Dados $x\in X$, $U\in\tau$ con $x\in U$ existe $k\in\mathbb{N}$ tal que $x\in B_k\subseteq U$.
            
            Ahora, por ser $(X,\tau)$ un espacio $T_3$, existe $V\in\tau$ tal que $x\in V\subseteq\overline{V}\subseteq B_k$. Nuevamente, como $V\in\tau$ entonces existe $j\in\mathbb{N}$ tal que $x\in B_j\subseteq V$. Por ende:
            \begin{equation*}
                x\in B_j\subseteq\overline{B_j}\subseteq B_k
            \end{equation*}
        \end{itemize}

        Definimos
        \begin{equation*}
            \mathcal{L}=\left\{(B_j,B_k)\Big|B_j,B_k\in\mathcal{B}\textup{ son tales que }\overline{B}_j\subseteq B_k \right\}
        \end{equation*}
        Por $(*)$ se tiene que $\mathcal{L}\neq\emptyset$. Como $\mathcal{L}\subseteq\mathcal{B}\times\mathcal{B}$, entonces $\mathcal{L}$ es numerable, por lo que podemos escribir a $\mathcal{L}$ como:
        \begin{equation*}
            \mathcal{L}=\left\{(B_{ m_i},B_{ n_i}) \right\}_{ i\in\mathbb{N}}
        \end{equation*}
        Sea $i\in\mathbb{N}$, se tiene que para $(B_{ m_i},B_{ n_i})\in\mathcal{L}$ se cumple:
        \begin{equation*}
            \overline{B}_{ m_i}\subseteq B_{ n_i}
        \end{equation*}
        luego, los conjuntos cerrados:
        \begin{equation*}
            \overline{B}_{ m_i}\textup{ y }X-B_{ n_i}
        \end{equation*}
        son ambos cerrados disjuntos.

        Ahora, como el espacio $(X,\tau)$ es $T_4$, para cada $i\in\mathbb{N}$ existe una función $\cf{h_i}{(X,\tau)}{(\left[0,\frac{1}{i}\right],\tau_u)}$ continua tal que
        \begin{equation*}
            h_i\left(\overline{B}_{ m_i}\right)=\left\{0\right\}\textup{ y }h_i\left(X-B_{ n_i}\right)=\frac{1}{i}
        \end{equation*}
        
        Sea $x\in X$, tenemos que la suma
        \begin{equation*}
            \sum_{ i=1}^\infty h_i(x)^2\leq\sum_{ i=1}^\infty\frac{1}{i^2} <\infty
        \end{equation*}
        es convergente para todo $x\in X$. Por tanto, podemos definir una función $\cf{h}{X}{\mathcal{H}}$ dada por:
        \begin{equation*}
            h(x)=(h_1(x),...,h_n(x),...),\quad\forall x\in X
        \end{equation*}
        Veamos que
        \begin{enumerate}
            \item $h$ es inyectiva. Sean $x,y\in X$ puntos distintos. Se tiene que $x\in X-\left\{y\right\}$, como el espacio es $T_1$ este conjunto es abierto. Por la observación anterior existen $(B_{ j_i},B_{ k_i})\in\mathcal{L}$ tales que
            \begin{equation*}
                x\in B_{ j_i}\subseteq\overline{B}_{ j_i}\subseteq B_{ k_i}\subseteq X-\left\{y\right\}
            \end{equation*}
            por ende, $x\in\overline{B}_{ j_i}$ y $y\in X-B_{ k_i}$. Por ende
            \begin{equation*}
                h_i(x)=0\quad\textup{y}\quad h_i(y)=\frac{1}{i},\quad\forall i\in\mathbb{N}
            \end{equation*}
            por ende, $h_i(x)\neq h_i(y)$, se sigue que $h(x)\neq h(y)$.
            \item $\cf{h}{(X,\tau)}{(\mathcal{H},\tau_{\rho})}$ es continua. Sea $x_0\in X$ y $\varepsilon>0$, encontremos $U\in\tau$, con $x_0\in U$ tla que $h(U)\subseteq B_\rho(h(x_0),\varepsilon)$, es decir que para todo $x\in U$,
            \begin{equation*}
                \rho(h(x_0),h(x))<\varepsilon
            \end{equation*}
            Sea $N\in\mathbb{N}$ tal que dado $x\in X$:
            \begin{equation*}
                \sum_{ n=N}^\infty\abs{ h_n(x_0)-h(x)}^2<\frac{\varepsilon^2}{2}
            \end{equation*}
            Además, para todo $m\in\left\{1,...,N \right\}$, la función $\cf{h_m}{(X,\tau)}{([0,1/m],\tau_u)}$ es continua, podemos encontrar $W_m\in\tau$ tal que $x_0\in W_m$ y además, para todo $x\in W_m$,
            \begin{equation*}
                \abs{h_m(x)-h_m(x_0)}^2<\frac{\varepsilon^2}{2N}
            \end{equation*}
            En particular se tiene que
            \begin{equation*}
                x_o\in\bigcap_{ m=1}^N W_m=W\in\tau
            \end{equation*}
            Sea $x\in W$:
            \begin{equation*}
                \begin{split}
                    \rho(h(x),h(x_0))&=\abs{\sum_{ n=1}^N\abs{h_n(x)-h_n(x_0)}^2+\sum_{ n=N+1}^\infty\abs{h_n(x)-h_n(x_0)}^2}^{1/2}\\
                    &\leq\abs{\sum_{ n=1}^N\frac{\varepsilon^2}{2N}+\frac{\varepsilon^2}{2}}^{1/2}\\
                    &=\left[\frac{\varepsilon^2}{2}+\frac{\varepsilon^2}{2} \right]^{ 1/2}\\
                    &=\varepsilon\\
                \end{split}
            \end{equation*}
            lo que prueba la continuidad de $h$ en $x_0$, que al ser arbitrario, se sigue que $h$ es continua en $X$.
            \item $\cf{h}{(X,\tau)}{(h(X),\tau_\rho)}$ es abierta. Sea $A\in\tau$. Tomemos $x\in h(A)$, sea $a\in A$ tal que
            \begin{equation*}
                h(a)=x
            \end{equation*}
            por la observación anterior existe $(B_{ m_i},B_{ n_i})\in\mathcal{L}$ con
            \begin{equation*}
                a\in B_{ m_i}\subseteq\overline{B}_{ m_i}\subseteq B_{ n_i}\subseteq A
            \end{equation*}
            por tanto, tenemos que $h_i(a)=0$ y $h_i(X-A)=\left\{\frac{1}{i}\right\}$. Por tanto,
            \begin{equation*}
                y\in h(X-A)=h(X)-h(A)
            \end{equation*}
            se cumple que
            \begin{equation*}
                \rho(x,y)\geq\frac{1}{i}
            \end{equation*}
            luego, $y\notin B_{\rho}(x,1/i)$. Se sigue que
            \begin{equation*}
                B_\rho(x,1/i)\subseteq h(A)
            \end{equation*}
            se sigue entonces que $\cf{h}{(X,\tau)}{(h(X),\tau_\rho)}$.
        \end{enumerate}
        De los tres incisos anteriores, se sigue que $\cf{h}{(X,\tau)}{(h(X),\tau_\rho)}$ es un homeomorfismo y como $(h(X),\tau_\rho)$ es metrizable, entonces $(X,\tau)$ es metrizable.
    \end{proof}

    \begin{excer}
        Sea $(X,\tau)$ un espacio topológico segundo numerable, entonces las siguientes proposiciones son equivalentes:
        \begin{enumerate}
            \item $(X,\tau)$ es completamente regular.
            \item $(X,\tau)$ es normal.
            \item $(X,\tau)$ es metrizable.
        \end{enumerate}
    \end{excer}

    \begin{proof}
        %TODO
    \end{proof}

    \section{Espacios Paracompactos}

    \begin{obs}
        El concepto de espacio paracompacto fue definido por Dieudonné en 1944.
    \end{obs}

    \begin{mydef}
        Sea $(X,\tau)$ espacio topológico, $x\in X$ y $\mathcal{U}=\left\{U_\alpha \right\}_{ \alpha\in I}$ una familia de subconjuntos de $X$.
        \begin{enumerate}
            \item La familia $\mathcal{U}$ es \textbf{punto finita en $x\in X$}, si existe un subconjunto finito $K$ de $I$ tal que para todo $\alpha\notin K$, $x\notin U_\alpha$.
            \item La familia $\mathcal{U}$ es \textbf{localmente finita en $x\in X$} si existe una vecindad $V$ de $x$ y un subconjunto finito $J\subseteq I$ tales que para todo $\alpha\notin J$,
            \begin{equation*}
                V\cap U_\alpha=\emptyset
            \end{equation*}
            \item La familia $\mathcal{U}$ es \textbf{punto (resp. localmente) finita en el espacio $(X,\tau)$} si $\mathcal{U}$ es punto (resp. localmente) finita en cada uno de los pntos de $X$.
            \item La familia $\mathcal{U}$ es \textbf{$\sigma$-localmente finita} si $\mathcal{U}$ se puede escribir como una unión numerable de colecciones localmente finitas.
        \end{enumerate}
    \end{mydef}

    \begin{obs}
        Se tiene lo siguiente:
        \begin{enumerate}
            \item Si $\mathcal{U}$ es localmente finita, entonces $\mathcal{U}$ es punto-finita.
            \item Si $\mathcal{U}$ es una colección finita, entonces $\mathcal{U}$ es localmente finita.
        \end{enumerate}
    \end{obs}

    En los siguientes ejemplos, considere $(\mathbb{R},\tau_u)$.
    
    \begin{exa}
        Sea $\left\{\left(\frac{1}{n},\frac{1}{n}\right) \right\}_{ n\in\mathbb{N}}$. Esta colección no es punto finta en $0$, luego tampoco es localmente finita en $0$.
    \end{exa}

    \begin{exa}
        Considere la colección de intervalos $\left\{\left(n,n+2\right) \right\}_{ n\in\mathbb{N}}$. Esta es una colección localmente finita en $(\mathbb{R},\tau)$. Sea $r\in\mathbb{R}$, existe $m\in\mathbb{Z}$ tal que:
        \begin{equation*}
            m\leq r<m+1
        \end{equation*}
        Considere el entero $n=m-1$, se tiene que
        \begin{equation*}
            n<r<n+2
        \end{equation*}
        %TODO
    \end{exa}

    \begin{exa}
        Dado $n\in\mathbb{N}$, sea $A_n=(0,1/n)$. Se tiene que $\left\{A_n \right\}_{ n\in\mathbb{N}}$ es punto finita, pero no es localmente finita.

        Sea $r\in\mathbb{R}$. Podemos suponer que $r\in(0,1)$ (ya que la colección solo contiene puntos de este conjunto) (es el único caso que genera problemas), luego existe $N_r\in\mathbb{N}$ tal que
        \begin{equation*}
            \frac{1}{m}\leq r,\quad\forall m\geq N_r
        \end{equation*}
        Sea $J_r=\left\{1,...,N_r \right\}$. Se tiene que
        \begin{equation*}
            r\notin A_m\quad\forall m\notin J_r
        \end{equation*}
        así que $\left\{A_n \right\}_{ n\in\mathbb{N}}$ es punto finita, pero claramente no es localmente finita (observe que sucede con cualquier vecindad de $0$).
    \end{exa}

    \begin{exa}[\textbf{**}]
        Sea $n\in\mathbb{N}$, definimos $A_n=\left(-\frac{1}{n},\frac{1}{n}\right)$ y $B_n=\mathbb{R}-A_n=(-\infty,-1/n]\cup[1/n,\infty)$. Tenemos que $B_n=\overline{B}_n$.

        Sea $m\in\mathbb{N}$ con $m\geq 2$, esto es que $\frac{1}{m}\leq\frac{1}{2}$. Se tiene que $\frac{1}{2}\in B_m$. Por tanto,
        \begin{equation*}
            \left\{B_n\right\}_{n\in\mathbb{N}}
        \end{equation*}
        no es punto finita en $\frac{1}{2}$.

        Por otro lado, se tiene que $0\notin B_n$ para todo $n\in\mathbb{N}$, por lo cual
        \begin{equation*}
            \bigcup_{ n\in\mathbb{N}}B_n\subsetneqq \mathbb{R}
        \end{equation*}
        sin embargo,
        \begin{equation*}
            \overline{\bigcup_{ n\in\mathbb{N}}B_n}=\overline{\bigcup_{ n\in\mathbb{N}}(\mathbb{R}-A_n)}=\overline{\mathbb{R}-\bigcap_{ n\in\mathbb{N}}A_n}=\overline{\mathbb{R}-\left\{ 0\right\}}=\mathbb{R}
        \end{equation*}
        por ende,
        \begin{equation*}
            \overline{\bigcup_{ n\in\mathbb{N}}B_n}\subsetneqq \bigcup_{ n\in\mathbb{N}}B_n=\bigcup_{ n\in\mathbb{N}}\overline{B}_n
        \end{equation*}
    \end{exa}

    \begin{obs}
        Sea $\mathcal{A}=\left\{A_\alpha \right\}_{\alpha\in I}\subseteq\mathcal{P}(X)$. Escribimos $\overline{\mathcal{A}}=\left\{\overline{A}_\alpha \right\}_{\alpha\in I}$.
    \end{obs}

    \begin{propo}
        Sea $(X,\tau)$ un espacio topológico. Tomemos $\mathcal{A}=\left\{A_\alpha \right\}_{\alpha\in I}$ una colección de subconjuntos de $X$ que es localmente finita en $(X,\tau)$. Entonces, se cumple lo siguiente:
        \begin{enumerate}
            \item Sea $J\subseteq I$ y sea $\mathcal{B}=\left\{B_\alpha \right\}_{\alpha\in J}\subseteq\mathcal{P}(X)$ tal que para todo $\alpha\in J$, $B_\alpha\subseteq A_\alpha$. Entonces, $\mathcal{B}$ es localmente finita.
            \item Sea $\overline{\mathcal{A}}=\left\{\overline{A}_\alpha\Big|\alpha\in I \right\}$, entonces $\overline{\mathcal{A}}$ es localmente finita.
            \item $\overline{\bigcup_{\alpha\in I}A_\alpha}=\bigcup_{ \alpha\in I}\overline{A}_\alpha$.
        \end{enumerate}
    \end{propo}

    \begin{proof}
        %TODO
    \end{proof}

    \begin{cor}
        Sea $(X,\tau)$ un espacio topológico.
        \begin{enumerate}
            \item Sean $\mathcal{L},\mathcal{B}\subseteq\mathcal{P}(X)$ tales que $\mathcal{L}\subseteq\mathcal{B}$ y $\mathcal{B}$ es localmente finita, entonces $\mathcal{L}$ es localmente finita.
            \item Si $\mathcal{A}$ es una colecciónde conjuntos cerrados de $(X,\tau)$ y $\mathcal{A}$ es localmente finita, entonces $\bigcup\mathcal{A}=\bigcup_{A\in\mathcal{A}}A$ es un conjunto cerrado.
        \end{enumerate}
    \end{cor}

    \begin{proof}
        %TODO
    \end{proof}

    \begin{mydef}
        Sea $X$ un conjunto y sean $\mathcal{V},\mathcal{U}$ dos colecciones de subconjuntos de $X$. Decimos que $\mathcal{U}$ es un \textbf{refinamiento} de $\mathcal{V}$, o que $\mathcal{U}$ refina a $\mathcal{V}$ y se escribe por $\mathcal{U}<\mathcal{V}$, si para cada $U\in\mathcal{U}$ existe $V\in\mathcal{V}$ tal que $U\subseteq V$.
    \end{mydef}

    \begin{obs}
        Sean $\mathcal{U},\mathcal{V},\mathcal{W}$ tres colecciones de subconjuntos de un conjunto $X$. Entonces:
        \begin{enumerate}
            \item $\mathcal{U}<\mathcal{U}$.
            \item $\mathcal{U}<\mathcal{V}$ y $\mathcal{V}<\mathcal{W}$ implican que $\mathcal{U}<\mathcal{W}$.
        \end{enumerate}
    \end{obs}

    \begin{exa}
        Considere $\mathcal{U}=\left\{(-1/n,1/n) \right\}_{ n\in\mathbb{N}}$ y $\mathcal{V}=\left\{(-1,1) \right\}_{ n\in\mathbb{N}}$. Entonces $\mathcal{U}<\mathcal{V}$ y $\mathcal{V}<\mathcal{U}$.
    \end{exa}

    El ejemplo anterior muestra que la relación $<$ no es antisimétrica.

    \begin{mydef}
        Un espacio topológico $(X,\tau)$ es \textbf{paracompacto} si para cada cubierta abierta $\mathcal{U}$ de $X$, existe una cubierta abierta $\mathcal{V}$ de $X$ tal que $\mathcal{V}$ es localmente finita y $\mathcal{V}<\mathcal{U}$. 
    \end{mydef}

    \begin{exa}
        Considere el espacio topológico $(\mathbb{R},\tau_D)$ y la familia $\mathcal{M}=\left\{\left\{r\right\} \right\}_{ r\in \mathbb{R}}$. Se tiene que $\mathcal{M}$ es una cubierta abierta localmente finita de $(\mathbb{R},\tau_D)$. Ahora, si $\mathcal{U}=\left\{U_\alpha \right\}_{\alpha\in I}$ es una cubierta abierta de $(\mathbb{R},\tau_D)$ se tiene de forma inmediata que:
        \begin{equation*}
            \mathcal{M}<\mathcal{U}
        \end{equation*}
        Por tanto, $(\mathbb{R},\tau_D)$ es paracompacto.
    \end{exa}

    El ejemplo anterior muestra que no todo espacio paracompacto es compacto. Sin embargo, veremos que el converso es cierto.

    \begin{excer}
        Sea $(X,\tau)$ un espacio topológico. Entonces, $(X,\tau)$ es compacto si y sólo si dada una cubierta abierta $\mathcal{U}$ de $(X,\tau)$ existe una cubierta abierta finita $\mathcal{V}$ de $(X,\tau)$ tal que $\mathcal{V}<\mathcal{U}$.
    \end{excer}

    \begin{proof}
        Es inmediata de la definición de compacidad.
    \end{proof}

    \begin{propo}
        Sea $(X,\tau)$ un espacio topológico y $A\subseteq X$. Entonces, $(A,\tau_A)$ es paracompacto si y sólo si dada $\mathcal{U}\subseteq\tau$ tal que
        \begin{equation*}
            A\subseteq\bigcup_{ U\in\mathcal{U}}U
        \end{equation*}
        existe $V\subseteq\tau$ tal que $A\subseteq\bigcup_{ V\in\mathcal{V}}V$, $\mathcal{V}<\mathcal{U}$ y $\mathcal{V}$ es localmente finita en $(A,\tau_A)$.
    \end{propo}

    \begin{proof}
        
    \end{proof}

    \begin{propo}
        Sea $(X,\tau)$ un espacio paracompacto y $A\subseteq X$ cerrado, entonces $(A,\tau_A)$ es paracompacto.
    \end{propo}

    \begin{proof}
        
    \end{proof}

    \begin{propo}
        La propiedad de ser paracompacto es topológica.
    \end{propo}

    \begin{proof}
        
    \end{proof}

    \begin{propo}
        Sea $(X,\tau)$ un espacio topológico. Se cumple:
        \begin{enumerate}
            \item Si $(X,\tau)$ es compacto, entonces $(X,\tau)$ es paracompacto.
            \item Si $(X,\tau)$ es paracompacto y $T_3$, entonces es $T_4$.
            \item Si $(X,\tau)$ es paracompacto y Hausdorff, entonces es regular.
            \item Si $(X,\tau)$ es paracompacto y Hausdorff, entonces $(X,\tau)$ es normal.
        \end{enumerate}
    \end{propo}

    \begin{proof}
        
        De (1): Ya se tiene.
        
        De (2): 

    \end{proof}



\end{document}