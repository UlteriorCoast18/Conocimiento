\documentclass[12pt]{report}
\usepackage[spanish]{babel}
\usepackage[utf8]{inputenc}
\usepackage{amsmath}
\usepackage{amssymb}
\usepackage{amsthm}
\usepackage{graphics}
\usepackage{subfigure}
\usepackage{lipsum}
\usepackage{array}
\usepackage{multicol}
\usepackage{enumerate}
\usepackage[framemethod=TikZ]{mdframed}
\usepackage[a4paper, margin = 1.5cm]{geometry}
\usepackage{bbm}

%En esta parte se hacen redefiniciones de algunos comandos para que resulte agradable el verlos%

\renewcommand{\theenumii}{\roman{enumii}}

\def\proof{\paragraph{Demostración:\\}}
\def\endproof{\hfill$\blacksquare$}

\def\sol{\paragraph{Solución:\\}}
\def\endsol{\hfill$\square$}

%En esta parte se definen los comandos a usar dentro del documento para enlistar%

\newtheoremstyle{largebreak}
  {}% use the default space above
  {}% use the default space below
  {\normalfont}% body font
  {}% indent (0pt)
  {\bfseries}% header font
  {}% punctuation
  {\newline}% break after header
  {}% header spec

\theoremstyle{largebreak}

\newmdtheoremenv[
    leftmargin=0em,
    rightmargin=0em,
    innertopmargin=-2pt,
    innerbottommargin=8pt,
    hidealllines = true,
    roundcorner = 5pt,
    backgroundcolor = gray!60!red!30
]{exa}{Ejemplo}[section]

\newmdtheoremenv[
    leftmargin=0em,
    rightmargin=0em,
    innertopmargin=-2pt,
    innerbottommargin=8pt,
    hidealllines = true,
    roundcorner = 5pt,
    backgroundcolor = gray!50!blue!30
]{obs}{Observación}[section]

\newmdtheoremenv[
    leftmargin=0em,
    rightmargin=0em,
    innertopmargin=-2pt,
    innerbottommargin=8pt,
    rightline = false,
    leftline = false
]{theor}{Teorema}[section]

\newmdtheoremenv[
    leftmargin=0em,
    rightmargin=0em,
    innertopmargin=-2pt,
    innerbottommargin=8pt,
    rightline = false,
    leftline = false
]{propo}{Proposición}[section]

\newmdtheoremenv[
    leftmargin=0em,
    rightmargin=0em,
    innertopmargin=-2pt,
    innerbottommargin=8pt,
    rightline = false,
    leftline = false
]{cor}{Corolario}[section]

\newmdtheoremenv[
    leftmargin=0em,
    rightmargin=0em,
    innertopmargin=-2pt,
    innerbottommargin=8pt,
    rightline = false,
    leftline = false
]{lema}{Lema}[section]

\newmdtheoremenv[
    leftmargin=0em,
    rightmargin=0em,
    innertopmargin=-2pt,
    innerbottommargin=8pt,
    roundcorner=5pt,
    backgroundcolor = gray!30,
    hidealllines = true
]{mydef}{Definición}[section]

\newmdtheoremenv[
    leftmargin=0em,
    rightmargin=0em,
    innertopmargin=-2pt,
    innerbottommargin=8pt,
    roundcorner=5pt
]{excer}{Ejercicio}[section]

%En esta parte se colocan comandos que definen la forma en la que se van a escribir ciertas funciones%

\newcommand\abs[1]{\ensuremath{\left|#1\right|}}
\newcommand\divides{\ensuremath{\bigm|}}
\newcommand\cf[3]{\ensuremath{#1:#2\rightarrow#3}}
\newcommand\natint[1]{\ensuremath{\left[\!\left[ #1\right]\!\right]}}
\newcommand{\afa}{\:
    \begin{tikzpicture}
        \draw [line width = 0.17 mm, black] (0,0) -- (-0.115,0.29);
        \draw [line width = 0.17 mm, black] (0,0) -- (0.115,0.29);
        \draw [line width = 0.17 mm, black] (-0.12,0) arc (190:-10:0.12cm);
    \end{tikzpicture}
    \:
}
\newcommand{\bbm}[1]{\ensuremath{\mathbbm{#1}}}
%Este símvolo es para casi todo salvo una cantidad finita

%recuerda usar \clearpage para hacer un salto de página

\begin{document}
    \setlength{\parskip}{5pt} % Añade 5 puntos de espacio entre párrafos
    \setlength{\parindent}{12pt} % Pone la sangría como me gusta
    \title{Notas Curso Topología II}
    \author{Cristo Daniel Alvarado}
    \maketitle

    \tableofcontents %Con este comando se genera el índice general del libro%

    %\setcounter{chapter}{3} %En esta parte lo que se hace es cambiar la enumeración del capítulo%
    
    \chapter{Metrizabilidad}
    
    \section{Conceptos Fundamentales}
    
    ¿Cuándo un espacio topológico es metrizable? Supongamos que tenemos un espacio topológico $(X,\tau)$, queremos una métrica $\cf{d}{X\times X}{\mathbb{R}}$ tal que $\tau_d=\tau$.

    La respuesta a esta pregunta es que no siempre será posible encontrar tal métrica. Por ejemplo, tome cualquier espacio topológico que no sea $T_1$.

    \begin{itemize}
        \item Pável Urysohn 1898-1924. El Lema de Urysohn fue publicado en 1924 póstumo a la muerte de su autor.
        \item Primera guerra mundial 28 de julio de 1914 a 11 de noviembre de 1918, inició con el asesinato del Archiduque Franciso de Austria.
        \item Segunda guerra mundial 1939 a 1945, cuando Hitler invade Polonia.
        \item En 1950 Bing, Nagata y Morita resuelven el problema de metrizabilidad de espacios topológicos.
    \end{itemize}

    Lo que veremos a continuación tiene como base fundamental el siguiente lema:

    \begin{lema}[\textbf{Lema de Urysohn}]
        Sea $(X,\tau)$ espacio topológico. Entonces, $(X,\tau)$ es $T_4$ si y sólo si dados $A,B\subseteq X$ cerrados disjuntos existe una función continua $\cf{f}{X}{[0,1]}$ tal que
        \begin{equation*}
            f(A)=\left\{0\right\} \quad\textup{y}\quad f(B)=\left\{1\right\}
        \end{equation*}
    \end{lema}

    Este lema se probó en el curso pasado.

    \begin{propo}
        Sea $(X,\tau)$ un espacio topológico segundo numerable. Entonces
        \begin{enumerate}
            \item $(X,\tau)$ es primero numerable.
            \item $(X,\tau)$ es de Lindelöf.
            \item $(X,\tau)$ es separable.
        \end{enumerate}
    \end{propo}

    \begin{proof}
        Sea $\mathcal{B}=\left\{B_i \right\}_{i\in\mathbb{N}}$ una base numerable para $\tau$.
        
        De (1): Sea $x\in X$. Tomemos
        \begin{equation*}
            \mathcal{B}_x=\left\{B_n\in\mathcal{B}\Big|x\in B_n \right\}
        \end{equation*}
        este es un conjunto no vacío pues al ser $\mathcal{B}$ base, existe $B\in\mathcal{B}$ tal que $x\in B$. Además es a lo sumo numerable por ser subcolección de $\mathcal{B}$.

        Sea $U\subseteq X$ abierto tal que $x\in U$. Como $\mathcal{B}$ es base de $\tau$, existe $B\in\mathcal{B}$ tal que $x\in B\subseteq U$, luego $B\in\mathcal{B}_x$. Por tanto, $\mathcal{B}_x$ es un sistema fundamental de vecindades de $x$. Al ser el $x$ arbitrario, se sigue que $(X,\tau)$ es primero numerable.

        De (2): Sea $\mathcal{A}=\left\{A_\alpha\right\}_{ \alpha\in I}$ una cubierta abierta de $(X,\tau)$. Dado $x\in X$ existe $A_\alpha\in\mathcal{A}$ tal que $x\in A_\alpha$, como $A_\alpha\in\tau$, existe $B_x\in\mathcal{B}$ tal que
        \begin{equation*}
            x\in B_x\subseteq A_\alpha
        \end{equation*}
        Sea
        \begin{equation*}
            \mathcal{K}=\left\{n\in\mathbb{N}\Big|\exists A_\alpha\in\mathcal{A}\textup{ tal que }B_n\subseteq A_\alpha \right\}
        \end{equation*}
        por la observación anterior, esta colección es no vacía. Dado $k\in\mathcal{K}$ escogemos un único $A_{\alpha_k}$ tal que
        \begin{equation*}
            B_k\subseteq A_{\alpha_k}
        \end{equation*}
        Sea
        \begin{equation*}
            \mathcal{A}'=\left\{A_{\alpha_k} \right\}_{ k\in\mathcal{K}}
        \end{equation*}
        se tiene que $\mathcal{A}'\subseteq\mathcal{A}$ es numerable. Sea $x\in X$, Como $\mathcal{A}$ es cubierta, existe $A'\in\mathcal{A}$ tal que
        \begin{equation*}
            x\in A'\in\tau
        \end{equation*}
        luego, al ser $\mathcal{B}$ base existe $B_n\in\mathcal{B}$ tal que
        \begin{equation*}
            x\in B_n\subseteq A'
        \end{equation*}
        Se sigue pues que $x\in A_{ \alpha_n}$. Por ende, $x\in\bigcup_{ n\in\mathbb{N}}A_{\alpha_n}$. Así, $\mathcal{A}$ posee una subcubierta a lo sumo numerable. Se sigue que al ser la cubierta abierta arbitraria que el espacio $(X,\tau)$ es Lindelöf.

        De (3): Ejercicio.
    \end{proof}

    \begin{propo}
        Si $(X,\tau)$ es metrizable, entonces los coneptos de espacio de Lindelöf, espacio separable y espacio segundo numerable son equivalentes.
    \end{propo}

    \begin{proof}
        Probaremos que Lindelöf implica separabilidad que implica segunda numerabilidad.

        Suponga que $(X,\tau)$ es metrizable, entonces existe una métrica $\cf{d}{X\times X}{\mathbb{R}}$ tal que $\tau_d=\tau$.
        \begin{itemize}
            \item Suponga que $(X,\tau)$ es Lindelöf. Sea $n\in\mathbb{N}$ y tomemos
            \begin{equation*}
                \mathcal{U}_n=\left\{B_d\left(x,\frac{1}{n}\right) \Big|x\in X \right\}
            \end{equation*}
            $\mathcal{U}_n$ es una cubierta abierta de $(X,\tau)$. Como el espacio de Lindelöf, existe $\mathcal{V}_n$ a lo sumo numerable tal que
            \begin{equation*}
                \mathcal{V}_n=\left\{B_d\left(y,\frac{1}{n} \right)\Big|y\in Y_n \right\}
            \end{equation*}
            siendo $Y_n\subseteq X$ un conjunto a lo sumo numerable, de tal suerte que $\mathcal{V}_n$ es subcubierta de $\mathcal{U}_n$. Sea
            \begin{equation*}
                A=\bigcup_{n\in\mathbb{N}}Y_n
            \end{equation*}
            este es un conjunto a lo sumo numerable. Sea $U\in\tau$ con $U\neq\emptyset$. Como $U\neq\emptyset$, existe $x\in U$, así existe $\varepsilon>0$ tal que $B_d(x,\varepsilon)\subseteq U$. Sea $m\in\mathbb{N}$ tal que $\frac{1}{m}<\varepsilon$. Tenemos que $\mathcal{V}_m$ es una cubierta de $X$, luego existe $y\in Y_m$ tal que
            \begin{equation*}
                x\in B_d\left(y,\frac{1}{m}\right)
            \end{equation*}
            Por tanto, $y\in B_d\left(x,\frac{1}{m} \right)\subseteq B(x,\varepsilon)\subseteq U$, así $y\in U$. Pero como $y\in Y_m$ se tiene que $y\in A$. Por ende
            \begin{equation*}
                U\cap A\neq\emptyset
            \end{equation*}
            lo que prueba el resultado.

            \item
        \end{itemize}
    \end{proof}

\end{document}