\documentclass[12pt]{report}
\usepackage[spanish]{babel}
\usepackage[utf8]{inputenc}
\usepackage{amsmath}
\usepackage{amssymb}
\usepackage{amsthm}
\usepackage{graphics}
\usepackage{subfigure}
\usepackage{lipsum}
\usepackage{array}
\usepackage{multicol}
\usepackage{enumerate}
\usepackage[framemethod=TikZ]{mdframed}
\usepackage[a4paper, margin = 1.5cm]{geometry}
\usepackage[mathscr]{euscript}

%En esta parte se hacen redefiniciones de algunos comandos para que resulte agradable el verlos%

\renewcommand{\theenumii}{\roman{enumii}}

\def\proof{\paragraph{Demostración:\\}}
\def\endproof{\hfill$\blacksquare$}

\def\sol{\paragraph{Solución:\\}}
\def\endsol{\hfill$\square$}

%En esta parte se definen los comandos a usar dentro del documento para enlistar%

\newtheoremstyle{largebreak}
  {}% use the default space above
  {}% use the default space below
  {\normalfont}% body font
  {}% indent (0pt)
  {\bfseries}% header font
  {}% punctuation
  {\newline}% break after header
  {}% header spec

\theoremstyle{largebreak}

\newmdtheoremenv[
    leftmargin=0em,
    rightmargin=0em,
    innertopmargin=0pt,
    innerbottommargin=5pt,
    hidealllines = true,
    roundcorner = 5pt,
    backgroundcolor = gray!60!red!30
]{exa}{Ejemplo}[section]

\newmdtheoremenv[
    leftmargin=0em,
    rightmargin=0em,
    innertopmargin=0pt,
    innerbottommargin=5pt,
    hidealllines = true,
    roundcorner = 5pt,
    backgroundcolor = gray!50!blue!30
]{obs}{Observación}[section]

\newmdtheoremenv[
    leftmargin=0em,
    rightmargin=0em,
    innertopmargin=0pt,
    innerbottommargin=5pt,
    rightline = false,
    leftline = false
]{theor}{Teorema}[section]

\newmdtheoremenv[
    leftmargin=0em,
    rightmargin=0em,
    innertopmargin=0pt,
    innerbottommargin=5pt,
    rightline = false,
    leftline = false
]{propo}{Proposición}[section]

\newmdtheoremenv[
    leftmargin=0em,
    rightmargin=0em,
    innertopmargin=0pt,
    innerbottommargin=5pt,
    rightline = false,
    leftline = false
]{cor}{Corolario}[section]

\newmdtheoremenv[
    leftmargin=0em,
    rightmargin=0em,
    innertopmargin=0pt,
    innerbottommargin=5pt,
    rightline = false,
    leftline = false
]{lema}{Lema}[section]

\newmdtheoremenv[
    leftmargin=0em,
    rightmargin=0em,
    innertopmargin=0pt,
    innerbottommargin=5pt,
    roundcorner=5pt,
    backgroundcolor = gray!30,
    hidealllines = true
]{mydef}{Definición}[section]

\newmdtheoremenv[
    leftmargin=0em,
    rightmargin=0em,
    innertopmargin=0pt,
    innerbottommargin=5pt,
    roundcorner=5pt
]{excer}{Ejercicio}[section]

%En esta parte se colocan comandos que definen la forma en la que se van a escribir ciertas funciones%

\newcommand\abs[1]{\ensuremath{\lvert#1\rvert}}
\newcommand\divides{\ensuremath{\bigm|}}
\newcommand{\cf}[3]{\ensuremath{#1:#2\rightarrow#3}}
\newcommand{\N}[1]{\ensuremath{\mathscr{N}(#1)}}
\newcommand{\Ns}[1]{\ensuremath{\mathscr{N}^*(#1)}}
\newcommand{\piz}[1]{\ensuremath{\mathscr{#1}}}
\newcommand{\eul}[1]{\ensuremath{\mathbb{#1}}}
\newcommand{\natint}[1]{\ensuremath{[\!\left[#1\right]\!]}}
\newcommand{\contradiction}{\ensuremath{\#_c}}
\newcommand{\Cls}[1]{\ensuremath{\overline{#1}}}
\newcommand{\id}[1]{\ensuremath{\textup{id}_{#1}}}

%recuerda usar \clearpage para hacer un salto de página

\begin{document}
    \setlength{\parskip}{5pt} % Añade 5 puntos de espacio entre párrafos
    \setlength{\parindent}{12pt} % Pone la sangría como me gusta
    \title{Grupos Topológicos}
    \author{Cristo Daniel Alvarado}
    \maketitle

    \tableofcontents %Con este comando se genera el índice general del libro%

    %\setcounter{chapter}{3} %En esta parte lo que se hace es cambiar la enumeración del capítulo%
    
    \chapter{Elementos de la teoría de grupos topológicos}
    
    \section{Preliminares}

    \begin{mydef}
        Sea $G$ un conjunto no vacío dotado de una operación binaria (denotada por $\cdot$) y una familia $\tau$ de subconjuntos de $G$. $G$ es llamado \textbf{grupo topológico} si
        \begin{enumerate}
            \item $(G,\cdot)$ es un grupo.
            \item $(G,\tau)$ es un espacio topológico.
            \item Las funciones $\cf{g_1}{(G,\tau)\times (G,\tau)}{(G,\tau)}$ y $\cf{g_2}{(G,\tau)}{(G,\tau)}$ dadas por $(x,y)\mapsto x\cdot y$ y $x\mapsto x^{-1}$, respectivamente, son continuas, siendo $x^{-1}$ el inverso de $x$ en $G$.
        \end{enumerate}
    \end{mydef}

    Se denotará a la operación $\cdot$ por yuxtaposición, es decir $x\cdot y = xy$.

    \begin{obs}
        Una equivalencia de la condición (3) de la proposición anterior es la siguiente:

        Sea $G$ un grupo topológico. Denotamos por $\mathscr{N}(x)$ a \textbf{ la familia de todas las vecindades de $x\in G$}. 3) es equivalente a
        \begin{enumerate}
            \setcounter{enumi}{3}
            \item Si $x,y\in G$, entonces para cada $U\in \mathscr{N}(xy)$ existen vecindades $V\in \mathscr{N}(x)$ y $W\in \mathscr{N}(y)$ tales que $V\cdot W\subseteq U$, donde
            \begin{equation*}
                V\cdot W = \left\{vw \big| v\in V \textup{ \& } w\in W \right\}
            \end{equation*}
            y, para cada $U\in\N{x^{-1}}$ existe $V\in\N{x}$ tal que $V^{-1}\subseteq U$, siendo
            \begin{equation*}
                V^{-1}=\left\{v^{-1}\big| v\in V \right\}
            \end{equation*}
        \end{enumerate}

        esta equivalencia es inmediada de la definición de continuidad de una función en un espacio topológico.
    \end{obs}

    \begin{obs}
        El símbolo $e_G$ denotará siempre a la identidad de un grupo $G$.

        Con frecuencia se referirá al grupo topológico $G$, con operación binaria $\cdot$ y topología $\tau$ como la terna $(G,\cdot,\tau)$. Si no hay ambiguedad, se denotará simplemente por $G$.
    \end{obs}

    \begin{lema}
        Sean $(G,\cdot)$ un grupo, y $\tau$ una topología en $G$. Entonces, $(G,\cdot,\tau)$ es un grupo topológico si y sólo si la función
        \begin{equation*}
            \begin{split}
                g_3:(G,\tau)\times (G,\tau)&\rightarrow(G,\tau)\\
                (x,y)&\mapsto xy^{-1}\\
            \end{split}
        \end{equation*}
        es continua.
    \end{lema}

    \begin{proof}
        $\Rightarrow):$ Suponga que $G$ es un grupo topológico, entonces las funciones $g_1$ y $g_2$ son continuas (por la condición 3) de la definición anterior). Notemos que
        \begin{equation*}
            g_3=g_1(x,g_2(y)),\quad\forall x,y\in G
        \end{equation*}
        por ende, $g_3$ es continua.

        $\Leftarrow):$ Suponga que la función $g_3$ es continua. Notemos que
        \begin{equation*}
            g_2(x)=g_3(x,e_G),\quad\forall x\in G
        \end{equation*}
        por ser $g_3$ continua, se sigue que $g_2$ también lo es. Además
        \begin{equation*}
            g_1(x,y)=g_3(x,g_2(y)),\quad\forall x,y\in G
        \end{equation*}
        por lo cual, $g_1$ también es continua. Por tanto, $G$ es grupo topológico.
    \end{proof}

    Una de las primeras ventajas que surgen en el estudio de los grupos topológicos es que, ciertas propiedades locales se vuelven globales desde el punto de vista de la topología.

    \begin{theor}
        Sea $G$ un grupo topológico. Si $g\in G$ es un elemento fijo arbitrario, entonces las funciones $\varphi_g(x)=xg$ y $\sigma_g(x)=gx$, para todo $x\in G$, de $G$ en $G$, son homeomorfismos. La inversión $\cf{f}{G}{G}$, definida por $f(y)=y^{-1}$, también es un homeomorfismos. Las funciones $\varphi_g$ y $\sigma_g$ son llamadas \textbf{traslaciones por la derecha e izquierda}, respectivamente.
    \end{theor}

    \begin{proof}
        Por la definición de grupo topológico, las funciones $\varphi_g$, $\sigma_g$ y $f$ son continuas. Veamos que son homeomorfismos de $G$ en $G$.
        \begin{enumerate}
            \item Veamos que $\varphi_g$ es inyectiva. Si $a,b\in G$ son tales que $\varphi_g(a)=\varphi_g(b)$, entonces $ag = bg\Rightarrow a = b$, con lo que se tiene el resultado.
            
            Además es suprayectiva, pues para cada $b\in G$ existe $g^{-1}b\in G$ tal que $\varphi_g(bg^{-1})=b$.

            Luego, $\varphi$ es homeomorfismo de $G$, con inversa $\varphi_{g^{-1}}$. Además es homomorfismo.
            \item Para $\sigma_g$ el caso es similar a $\varphi_g$.
            \item Para $f$ el resultado es inmediato, pues es biyectiva, homomorfismo y su inversa es ella misma.
        \end{enumerate}
    \end{proof}

    Los resultados siguientes nos perimitirán estudiar las propiedades topológicas locales de un grupo topológico $G$ en un solo punto, que por simplificar siempre tomaremos como la identidad $e_G$ del grupo.

    \begin{cor}
        Todo grupo topológico $G$ es un espacio homogéneo.
    \end{cor}

    \begin{proof}
        Debemos probar que dados dos elementos arbitrarios del grupo topológico $G$, digamos $g,h\in G$, existe un homeomorfismo de $G$ sobre sí mismo tal que manda un elemento en el otro. Por el teorema anterior, tomando como homeomorfismo a $\varphi_{g^{-1}h}$ se tiene el resultado, pues $\varphi_{g^{-1}h}(g)=h$.
    \end{proof}

    Como en grupos y espacios topológicos, nos interesan las funciones que preservan las propiedades entre éstos. Por lo cual se estudiarán los siguientes tipos de funciones:

    \begin{mydef}
        Decimos que una función biyectiva $\cf{f}{G}{G'}$ entre dos grupos topológicos $G$ y $G'$ es un \textbf{isomorfismo topológico} si $f$ y $f^{-1}$ son homomorfismos continuos.

        Si $G=G'$, el isomorfismo $f$ se llama \textbf{automorfismo topológico}. dos grupos topológicos son \textbf{topológicamente isomorfos} si existe un isomorfismo topológico de uno al otro. Utilizaremos el símbolo $G\cong H$ para indicar que los grupos $G$ y $H$ son topológicamente isomorfos.
    \end{mydef}

    El objetivo del siguiente teorema es ver que un grupo topológico no abeliano admite muchos automorfismos.

    \begin{theor}
        Si $G$ es un grupo topológico y $a\in G$ está fijo, entonces la función $g(x)=axa^{-1}$ es un automorfismo topológico.
    \end{theor}

    \begin{proof}
        Observemos que $g(x)=\sigma_a(\varphi_{a}^{-1}(x))$, donde las dos funciones de la composición definidas como en el teorema anterior son homeomofismos, y por ende $g$ lo es. Además $g$ es homomorfismo ya que
        \begin{equation*}
            g(xy)=axya^{-1}=\left(axa^{-1}\right)\left(aya^{-1}\right)=g(x)g(y)
        \end{equation*}
        el cual es invertible, con inversa $f(x)=a^{-1}xa$.
    \end{proof}

    \begin{obs}
        En el caso de que el grupo $G$ sea abeliano, el automorfismo topológico $G$ definido en el teorema anterior, es trivial ya que coincide con la identidad.
    \end{obs}

    El siguiente resultado tiene como objetivo el describir la topología del grupo, que en este caso resulta más sencillo que describir la topología de un espacio topológico. Para ello, basta describir una base local para la identidad del grupo $e_G$.

    \begin{lema}
        Sea $G$ un grupo topológico, y sea $\N{e_G}$ una base local para la identidad del grupo $e_G$. Entonces las familias $\left\{xU\right\}$ y $\left\{Ux\right\}$, donde $x$ toma los valores en los elemntos de $G$ y $U$ varía sobre todos los elementos de $\N{e_G}$, son bases para la topología de $G$.
    \end{lema}

    \begin{proof}
        Sea $W$ un abieto no vacío de $G$ y $a\in G$ un elemento de $W$. Probaremos que existe un elemento $\hat{U}$ de alguna las familias descritas anteriormente tal que
        \begin{equation*}
            a\in\hat{U}\subseteq W
        \end{equation*}
        Considere la función $\cf{f}{G}{G}$, $x\mapsto a^{-1}x$. Esta función es un homeomorfismo, el cual transforma a $W$ en $a^{-1}W$. Notemos que $e_G\in a^{-1}W$, pues el elemento $e_G=a^{-1}a\in a^{-1}W$. Como $\N{e_G}$ es una base local de $e_G$, entonces existe $U\in\N{e_G}$ tal que
        \begin{equation*}
            e_G\in U \subseteq a^{-1}W
        \end{equation*}
        Por lo cual
        \begin{equation*}
            a\in aU \subseteq aa^{-1}W=W
        \end{equation*}
        Por tanto, tomando $\hat{U}=aU$ se tiene el resultado para la primera familia. Para la segunda se procede de forma análoga cambiando el orden del producto en la función $f$.
    \end{proof}

    El siguiente lema nos proporciona una base local para la identidad formada por vecindades tales que $V^{-1}=V$. Estas vecindades reciben el nombre de \textbf{simétricas}.

    \begin{lema}
        Sea $G$ un grupo topológico y $U\in\N{e_G}$, entonces existe $V\in\N{e_G}$ tal que $V^{-1}=V\subseteq U$. Por lo tanto, las vecindades simétricas de la identidad constituyen una base local de $e_G$.
    \end{lema}

    \begin{proof}
        Sean $U\in\N{e_G}$ y $\cf{f}{G}{G}$, $x\mapsto x^{-1}$. Como $f$ es un homeomorfismo de $G$ sobre $G$, entonces $f(U)=U^{-1}$ es abierto y $e_G\in U^{-1}$. Por lo cual $V=U\cap U^{-1}$ es abierto y $V^{-1}=V$ es tal que $e_G\in V\subseteq U$.
    \end{proof}

    En lo sucesivo denotaremos por $\Ns{e_G}$ a la base local de vecindades de $e_G$ que son abiertas y simétricas en un grupo topológico $G$.

    Otra propiedad importante de la identidad del grupo topológico $G$, es que admite una base local formada por subconjuntos cerrados.

    \begin{lema}
        Sea $G$ grupo topológico.
        \begin{enumerate}
            \item Si $U\in\N{e_G}$, entonces para cada $n\in\mathbb{N}^+$ existe $V\in\N{e_G}$ con $V^{n}\subseteq U$, donde
            \begin{equation*}
                V^n=\underbrace{V\cdots V}_{n\text{-veces}}
            \end{equation*}
            \item Si $U\in\N{e_G}$, entonces existe $V\in\N{e_G}$ con $\overline{V}\subseteq U$. En particular, las vecindades cerradas de $e_G$ constituyen una base local de la identidad $e_G$ cuyos elementos son subconjuntos cerrados.
        \end{enumerate}
    \end{lema}

    \begin{proof}
        De 1): Procederemos por inducción sobre $n$. Para $n=1$ el resultado es inmediato, pues tomando $V=U$ se sigue el resultado.

        Suponga el resultado cierto para algún $n\in\mathbb{N}^+$, entonces para $U$ existe $W\in\N{e_G}$ tal que $W^n\subseteq U$. Como la multiplicación es continua $g_1(x,y)=xy$, y $g_1(e_G,e_G)=e_G$, entonces para $W$ existen vecindades $V_1,V_2\in\N{e_G}$ tales que $f(V_1\times V_2)=V_1\cdot V_2\subseteq W$. Tomemos $V=V_1\cap V_2$, claro que $e_G\in V$, por lo cual $V\in\N{e_g}$ y, además:
        \begin{equation*}
            V^{n+1}= V\cdot V\cdot V^{n-1}\subseteq V_1\cdot V_2\cdot W^{n-1}\subseteq W\cdot W^{n-1}=W^n\subseteq U
        \end{equation*}
        Aplicando inducción se sigue el resultado.

        De 2): Por 1) y por el hecho de que $\Ns{e_G}$ es una base local de $e_G$, existe $V\in\Ns{e_G}$ tal que $V^2\subseteq U$. Si $x\in\overline{V}$, entonces como $xV$ es una vecindad de $x$, la intersección $xV\cap V\neq\emptyset$ (pues $x$ está en la adherencia de $V$), es decir, existen $v_1,v_2\in V$ tales que
        \begin{equation*}
            xv_1=v_2\Rightarrow x=v_2v_1^{-1}\in V\cdot V^{-1}=V^{2}\subseteq U
        \end{equation*}
        Por ende, $\overline{V}\subseteq U$.
    \end{proof}

    \begin{theor}
        Sea $G$ un grupo topológico, $a\in G$ y $A, B, O, M$ subconjuntos de $G$. Entonces
        \begin{enumerate}
            \item Si $O$ es abierto, entonces los conjuntos $aO$, $Oa$, $O^{-1}$, $MO$ y $OM$ son abiertos.
            \item Si $A$ es cerrado, entonces $aA, Aa, A^{-1}$ son conjuntos cerrados.
            \item Si $A$ y $B$ son compactos, también lo son $AB$ y $A^{-1}$.
            \item Se cumple que
            \begin{equation*}
                \overline{A}=\bigcap_{W\in\N{e_G}}AW=\bigcap_{W\in\N{e_G}}WA
            \end{equation*}
        \end{enumerate}
    \end{theor}

    \begin{proof}
        De 1): Por el teorema 1.1.1, $\varphi_a$, $\sigma_a$ y $f(x)=x^{-1}$ son homeomorfismos, para cualquier $a\in G$ fijo. Por lo tanto, si $O$ es abierto, entonces la imagen directa de $O$ bajo estas funciones (es decir, los conjuntos $aO, Oa, O^{-1}$) son abiertos. Para los dos últimos conjuntos, basta ver que
        \begin{equation*}
            \begin{split}
                MO =& \bigcup\left\{mO\big|m\in\mathbb{M} \right\}\\
                OM =& \bigcup\left\{Om\big|m\in\mathbb{M} \right\}\\
            \end{split}
        \end{equation*}
        por ser uniones arbitrarias de abiertos, los conjuntos $MO$ y $OM$ son abiertos.

        De 2): Es análogo a 1), usando el hecho de que los homomorfismos son aplicaciones cerradas.

        De 3): Notemos que $A\times B$ es compacto en el espacio topológico producto $G\times G$, por lo cual al ser $\cf{g_1}{G\times G}{G}$, $(x,y)\mapsto xy$ una función continua, se sigue que la imagen de este compacto $f(A\times B) = AB$ es compacto. De forma similar con $A^{-1}$ con la función $f(x)=x^{-1}$ se obtiene que $A^{-1}$ es compacto.

        De 4): Nuestro objetivo será intentar caracterizar a $AW$ y $WA$ (donde $W\in\N{e_G}$) antes de ver los elementos de la intersección. Sea $W\in\N{e_G}$, entonces existe un abierto $V\in\Ns{e_G}$ tal que $V\subseteq W$. Por 1) el producto $AV$ es abierto y $A\subseteq AV$ (pues $e_G\in V$).

        Además, $\overline{A}\subseteq AW$, pues si $x\in\overline{A}$, entonces $xV$ es una vecindad de $x$ y por lo tanto $xV\cap A\neq\emptyset$, así existen $v\in V$ y $a\in A$ tales que $xv=a\Rightarrow x=av^{-1}\in AV^{-1}=AV\subseteq AW$. Como el $W$ fue arbitrario, se sigue que
        \begin{equation*}
            \overline{A}\subseteq\bigcap_{W\in\N{e_G}}AW
        \end{equation*} 
        (de forma análoga con $\bigcap_{W\in\N{e_G}}WA$). Ahora, sean $x\in\bigcap_{W\in\N{e_G}}AW$ y $V\in\N{x}$. Debemos probar que $V\cap A\neq\emptyset$ (con ello, se tendría que $x\in\overline{A}$). Se tiene que $x^{-1}V\in\N{e_G}$ y, por ende $V^{-1}x\in\N{e_G}$.

        Por tanto, como $x\in\bigcap_{W\in\N{e_G}}AW$ en particular $x\in AV^{-1}x$, así existen $a\in A$ y $v\in V$ tales que $x=av^{-1}x$, es decir $a=v\in V$ y $a\in A$, por lo cual $a\in A\cap V$. Por tanto, $A\cap V\neq\emptyset$.
    \end{proof}

    Como ya se sabe que todo grupo topológico es un espacio homogéneo, para verificar propiedades locales del grupo (tales como la conexidad local, compacidad local, carácter numerable, etc...), basta con verificar la propiedad en la identidad del grupo. Una de éstas propiedades es la $T_3$.

    \begin{lema}
        Todo grupo topológico $G$ cumple las propiedades siguientes:
        \begin{enumerate}
            \item $G$ es un espacio $T_3$.
            \item Si $A\subseteq G$ es compacto y $B\subseteq G$ cerrado, entonces $AB$ y $BA$ son cerrados.
        \end{enumerate}
    \end{lema}

    \begin{proof}
        De 1): Se debe probar que $G$ es un espacio regular, es decir, hay que probar que $G$ es $T_1$ y que para todo $x\in G$ y toda vecindad $V$ de $x$ existe una vecindad $U$ de $x$ tal que $\overline{U}\subseteq V$. Esto es inmediato del lema 1.1.4 2).

        De 2). Probaremos que $BA$ es cerrado. Para ello, se probará que $G\backslash BA$ es abierto. Sea $a\in G\backslash BA$. Para cada $x\in A$, el conjunto $Bx$ es cerrado (por ser $B$ cerrado), así que existen vecindades $U_x,V_x\in\Ns{e_G}$ con $aU_x\cap Bx=\emptyset$ y $V_x^2\subseteq U_x$. Por ello, $aV_x\cap BxV_x=\emptyset$.

        Ahora, como ${xV_x}_{x\in A}$ es una cubierta abierta de $A$, al ser $A$ compacto existen $x_1,...,x_n\in A$ tales que
        \begin{equation*}
            A\subseteq\bigcup_{i=1}^n x_iV{x_i}
        \end{equation*}
        Sea
        \begin{equation*}
            W=\bigcap_{i=1}^n x_iV{x_i}
        \end{equation*}
        Este conjunto es abierto y simétrico, y además $aW\cap BxV_{x_i}=\emptyset$ para todo $i\in\natint{1,n}$. Por tanto, $aW\cap BA=\emptyset$. Así que $aW$ es una vecindad de $a$ ajena a $BA$. De forma análoga se prueba que $BA$ es cerrado.
    \end{proof}

    \begin{exa}
        Sea $G$ un grupo no trivial dotado de la topología indscreta. Entonces $G$ es un grupo topológico que no es ni $T_0$ ni $T_1$ (en esta topología solo hay dos conjuntos: $\emptyset$ y $G$).
    
        Ahora, si tenemos un grupo topológico que es $T_0$ esto es equivalente a que sea $T_1$. En efecto, supongamos que es $T_0$ y sean $x_1,x_2\in G$ elementos distingos. Como es $T_0$ existe $U\subseteq G$ abierto que contiene a $e_G$ ó $x_1x_2^{-1}$. Si $e_G\in U$, entonces existe $V\in\Ns{e_G}$ tal que $V\subseteq U$, en particular $e_G\in V$ y $x_1x_2^{-1}\notin V$, por lo cual $x_2,x_1^{-1}\notin V$, luego $Vx_2$ es un abierto que contiene a $x_2$ y no a $x_1$, y $Vx_1$ es un abierto $x_1$ que contiene a $x_1$ pero no a $x_2$.

        Si $x_1x_2^{-1}\in U$, entonces $W=Ux_2x_1^{-1}\in\N{e_G}$, y $x_1x_2\notin W$. Haciendo lo análogo a lo anterior, se llega al resultado. Por tanto, la propiedad de ser $T_0$ y $T_1$ en un grupo topológico $G$ son equivalentes.
        
        De esta forma, todo grupo que sea $T_0$ es en automático un espacio regular, y más aún, es Hausdorff.
    \end{exa}

    De ahora en adelante sólo se considerarán grupos topológicos $T_0$ (en automático, esto serán espacios regulares). Más adelante se probará que todo grupo $T_0$ es Tikhonov.

    \begin{obs}
        En todo grupo topológico que sea un espacio $T_0$ se tiene que el conjunto $e_G$ es cerrado.
    \end{obs}

    \begin{proof}
        En efecto, sea $G$ un grupo topológico con las propiedades anteriores. Considere:
        \begin{equation*}
            A=\bigcap_{U\in\N{e_G}}\overline{U}
        \end{equation*}
        Claro que $e_G\in A$. Suponga que existe $a\in A$ tal que no es la identidad del grupo topológico, como $G$ es $T_1$ existe $V$ abierto tal que $e_G\in V$ pero $a\notin V$. Como la cerradura de los elementos de $\N{e_G}$ forman una base local para $e_G$, existe $U_0\in\N{e_G}$ tal que $\overline{U_0}\subseteq V$. Por tanto, $a\notin \bigcap_{U\in\N{e_G}}\overline{U}$ pues $a\notin \overline{U_0}$\contradiction. Luego $A=\left\{e_G\right\}$, pero $A$ es cerrado por ser intersección arbitraria de cerrados. Por tanto, el conjunto unipuntual $\left\{e_G\right\}$ es cerrado (más aún, el conjunto $\left\{x\right\}$ es cerrado, para todo $x\in G$).
    \end{proof}

    En los grupos topológicos, los subespacios compactos tienen propieadedes similares a las de los puntos en relación con las condiciones de separación:

    \begin{theor}
        Sea $G$ un grupo topológico, $K\subseteq U\subseteq G$, $U$ abierto y $K$ compacto. Entonces, existe $W\in\N{e_G}$ con la siguiente propiedad:
        \begin{equation*}
            K\subseteq KW\subseteq U
        \end{equation*}
    \end{theor}

    \begin{proof}
        Para cada $x\in K$ existe $V_x\in\N{e_G}$ con $xV_x\subseteq U$. Además, existe $W_x\in\N{e_G}$ tal que $W_x^2\subseteq V_x$.

        Como $K$ es compacto, y $K\subseteq\bigcup_{x\in K}xW_x$, entonces existen $x_1,...,x_n\in K$ tales que
        \begin{equation*}
            K\subseteq\bigcup_{i=1}^nx_iW_{x_i}
        \end{equation*}
        Sea $W=\bigcap_{i=1}^nW_{x_i}$. El conjunto $W$ es una vecindad abierta de $e_G$ y, por ende $K\subseteq KW$.

        Si $x\in K$, entonces por la contención anterior se sigue que existe $i\in\natint{1,n}$ tal que $x\in x_iW_{x_i}$. Así:
        \begin{equation*}
            xW\subseteq x_iW_{x_i}W\subseteq x_iW_{x_i}W_{x_i}\subseteq x_iV_{x_i}\subseteq U
        \end{equation*}
        es decir, $KW\subseteq U$.
    \end{proof}

    El siguiente teorema tiene como objetivo el resumir varias de las propiedades obtenidas anteriormente para la familia $\N{e_G}$; de hecho esta familia se caracteriza completamente. Esta propiedad es un de las que distinguen a los grupos topológicos de los espacios topológicos arbitrarios. Además, dicho teorema nos proporciona un método para definir topologías de grupos topológicos.

    \begin{theor}
        Sea $G$ un grupo topológico de Hausdorff. Existe una base local $\mathcal{V}$ para $e_G$ tal que cumple las siguientes condiciones:
        \begin{enumerate}
            \item $\bigcap\mathcal{V}=\left\{e_G\right\}$.
            \item Si $U,V$ son dos elementos arbitrarios de $\mathcal{V}$, entonces existe $W\in\mathcal{V}$ tal que $W\subseteq U\cap V$.
            \item Para cada $U\in\mathcal{V}$ existe $V\in\mathcal{V}$ tal que $VV^{-1}\subseteq U$.
            \item Para cada $U\in\mathcal{V}$ y para cada $x\in U$ existe $V\in \mathcal{V}$ con $xV\subseteq U$.
            \item Para cada $U\in \mathcal{V}$ y $a\in G$ existe $W\in\mathcal{V}$ con $aWa^{-1}\subseteq U$.
        \end{enumerate}

        Recíprocamente, si tenemos un grupo $G$ y una familia $\mathcal{V}$ no vacía de subconjuntos de $G$ que contienen a $e_G$, tales que satisfacen las condiciones de (1) a (5) para $\mathcal{V}$, entonces cada una de las familias $\left\{xU\big|U\in\mathcal{V}, x\in G \right\}$ y $\left\{Ux\big|U\in\mathcal{V}, x\in G \right\}$ es base para una topología de grupo $\tau$ para $G$. Además, $\mathcal{V}$ es una base local para $e_G$ en $(G,\tau)$.
    \end{theor}

    \begin{proof}
        $\Rightarrow$): Sea $G$ un grupo topológico \textbf{Hasdorff} (de forma inmediata es un espacio $T_0$ y, por ende es un espacio regular). Considere la familia:
        \begin{equation*}
            \mathcal{V}=\left\{V\cap V^{-1}\big|V\in\N{e_G} \right\}
        \end{equation*}
        es inmedaito que $\mathcal{V}$ cumple las condiciones (1) y (2) (por la observación 1.1.4 y la otra por ser $\mathcal{V}$ una base local de $e_G$). Para probar (3), sea $U\in\mathcal{V}$. Por un lema anterior existe $V\in\N{e_G}$ tal que $V^2\subseteq U$; entonces $W=V\cap V^{-1}$ pertenece a $\mathcal{V}$, y se tiene que $W^{-1}=W$ y $WW^{-1}=W^2\subseteq V^2\subseteq U$.

        Para (4), sean $U\in\mathcal{V}$ y $x\in U$. Como la multiplicación es una operación continua y $xe_G=x$, existen vecindades abiertas $V_x$ y $W$ de $x$ y $e_G$ respectivamente, tales que $V_xW\subseteq U$. El conjunto $V=W\cap W^{-1}$ pertenece a $\mathcal{V}$ y se cumple que $xV\subseteq V_xW\subseteq U$.

        Para (5), si $a\in G$ y $U\in\mathcal{V}$, como $aa^{-1}=ae_Ga$ y por ser continua la multiplicación, existen vecindades abiertas $W_a,V,W_{a^{-1}}$ de $a,e_G$ y $a$, respectivamente tales que
        \begin{equation*}
            W_aVW_{a^{-1}}\subseteq U
        \end{equation*}
        entonces, $W=V\cap V^{-1}$ pertenece a $\mathcal{V}$, y $aWa^{-1}\subseteq W_aVW_{a^{-1}}\subseteq U$.

        $\Leftarrow$): Sea $\mathcal{V}$ una familia de subconjuntos de $G$ que satisfacen las condiciones (1) a (5) del teorema. Debemos probar que
        \begin{equation*}
            \mathcal{B}=\left\{xU\big|U\in\mathcal{V}, x\in G \right\}
        \end{equation*}
        es una base para una topología de grupo $\tau$ en $G$. Sea $\tau$ la familia de subconjuntos de $G$ que son uniones arbitrarias de subconjuntos de $\mathcal{B}$, es decir $U\in\tau$ si y sólo si $U=\bigcup\mathcal{A}$, donde $\mathcal{A}$ es una subfamilia de $\mathcal{B}$.

        Se tienen que verificar dos condiciones:

        \begin{enumerate}
            \item Sean $x_1,x_2\in G$ y $U_1,U_2\in\mathcal{V}$. Si $x_3\in x_1U_1\cap x_2U_2$, entonces $x_1^{-1}x_3\in U_1$ y $x_2^{-1}x_3\in U_2$, por (4) existen $V_1,V_2\in\mathcal{V}$ tales que
            \begin{equation*}
                \begin{split}
                    x_1^{-1}x_3V_1&\subseteq U_1\\
                    x_2^{-1}x_3V_2&\subseteq U_2\\
                \end{split}
            \end{equation*}
            por ende,
            \begin{equation*}
                \begin{split}
                    x_3V_1&\subseteq x_1U_1\\
                    x_3V_2&\subseteq x_2U_2\\
                \end{split}
            \end{equation*}
            Tomemos $U_3\in\mathcal{V}$ tal que $U_3\subseteq V_1\cap V_2\in\mathcal{V}$ (lo cual se puede por la condición (2)). Luego, $x_3U_3\subseteq x_3V_1\cap x_3V_2\subseteq x_1U_1\cap x_2U_2$.

            \item Sea $x\in G$. Si $U\in\mathcal{V}$ entonces $x\in xU$, por lo cual:
            \begin{equation*}
                G=\bigcup_{x\in G}xU
            \end{equation*}
        \end{enumerate}
        por las dos condiciones anteriores, se sigue que $\mathcal{B}$ es una base para la topología $\tau$ definida anteriormente.

        Ahora, probemos que $\mathcal{V}$ es base local para $e_G$. Sea $U\in\tau$ tal que $e_G\in U$. Como $U\in\tau$ entonces podemos escribir:
        \begin{equation*}
            U=\bigcup\mathcal{A}
        \end{equation*}
        donde $\mathcal{A}$ es una subfamilia de $\mathcal{B}$. Entonces por estar en la unión, existe $x\in G$ y $V\in\mathcal{V}$ tal que $e_G\in xV\subseteq U$. En particular, $x^{-1}\in V$, luego por (4) podemos encontrar $W\in\mathcal{V}$ tal que $x^{-1}W\subseteq V$, es decir $W\subseteq xV$. Pero $e_G\in W$, por ende:
        \begin{equation*}
            e_G\in W\subseteq xV\subseteq U
        \end{equation*}
        
        Luego, $\mathcal{V}$ es una base local de $e_G$.

        Ahora probaremos que $\tau$ es una topología del grupo $G$. Para ello hay que ver que la función $(a,b)\mapsto ab^{-1}$ es continua. Sean $a,b\in G$ y $U$ una vecindad de $ab^{-1}$. De la condición (4) existe $V\in\mathcal{V}$ tal que $ab^{-1}V\subseteq U$, y por (5) y (3) existen $W_1,W_2\in\mathcal{V}$ tales que $bW_1b^{-1}\subseteq V$ y $W_2W_2^{-1}\subseteq W_1$.

        Entonces, $aW_2$ y $bW_2$ son vecindades de los puntos $a$ y $b$ para las cuales se tiene que
        \begin{equation*}
            \begin{split}
                (aW_2)(bW_2)^{-1}&=aW_2W_2^{-1}b^{-1}\\
                &\subseteq aW_1b^{-1}\\
                &\subseteq ab^{-1}(bW_1b^{-1})\\
                &\subseteq ab^{-1}(bW_1b^{-1})\\
                &\subseteq ab^{-1}V\\
                &\subseteq U\\
            \end{split}
        \end{equation*}
        lo cual prueba que la operación considerada es continua.

        Para terminar la demostració, hay que ver que la familia:
        \begin{equation*}
            \left\{Ux\Big|x\in G, U\in\mathcal{V} \right\}
        \end{equation*}
        es también una base para la topología $\tau$. Sean $a\in G$ y $U\in\tau$ tales que $a\in U$. Por (4) existe $V\in\mathcal{V}$ tal que $aV\subseteq U$, y por (5) existe $W\in\mathcal{V}$ tal que $a^{-1}Wa\subseteq V$. Entonces:
        \begin{equation*}
            a\in Wa\subseteq aV\subseteq U
        \end{equation*}
        lo cual prueba que la familia es una base para la topología $\tau$, terminando así la prueba.

    \end{proof}

    De un lema anterior sabemos que si $A,B\subseteq G$ con $G$ grupo topológico, $A$ compacto y $B$ cerrado, entonces $AB$ y $BA$ son cerrados. La hipótesis de que $A$ sea compacto no es imprescindible.

    \begin{exa}
        Sea $G$ un grupo arbitrario con la topología discreta, es decir, aquella formada por todos los subconjuntos de $G$; entonces $G$ forma un grupo topológico llamado \textbf{grupo discreto}.
    \end{exa}

    \begin{exa}
        Cualquier grupo $G$ con la topología indiscreta, es decir, aquella que consiste únicamente en el conjunto vacío y $G$ mismo, es un grupo topológico. Éste no es un grupo topológico $T_0$ si $G$ contiene más de un elemento. 
    \end{exa}

    \begin{exa}
        El conjunto de los números reales $\mathbb{R}$ con su topología y operación de suma usuales es un grupo topológico.
    \end{exa}

    \begin{exa}
        En el grupo aditivo de los números enteros, $(\mathbb{Z},+)$, definiremos varias topologías de grupo:
        \begin{enumerate}
            \item Sea $p\in\mathbb{Z}$ un número primo fijo y para cada $k\in\mathbb{N}$ sea $U_k=p^k\mathbb{Z}$; entonces la familia $\mathcal{V}=\left\{U_k\big|k\in\mathbb{N} \right\}$ satisface las condiciones del teorema anterior: todos los miembros de $\mathcal{V}$ contienen al cero y se prueba que su intersección es el cero. La condición (3) se deduce de la relación $U_k=-U_k$.
            Las propiedades (2) y (4) se obtienen a partir de la contención $2(U_k)\subseteq U_k$ y de la definición de $U_k$. Por último, (5) se cumple por ser $(\mathbb{Z},+)$ abeliano.

            Esta topología de $G$ recibe el nombre de \textbf{$p$-ádica}. Para números primos distintos $p$ y $q$, las topologías obtenidas de esta manera son distintas porque el conjunto $M=\left\{p,p^2,...,p^n,...\right\}$ tiene al $0\in\mathbb{Z}$ como punto de acumulación en la $p$-ádica. Por el contrario, el $0$ no es punto de acumulación de $M$ en la $q$-ádica.
        \end{enumerate}
    \end{exa}

    \begin{exa}
        El \textbf{grupo lineal general de orden $n$ sobre $\mathbb{R}$}. Considere el grupo $GL(n,\mathbb{R})$ de las matrices no sigulares (invertibles) de orden $n$ con elementos en el campo $\mathbb{R}$ y como operación de grupo la multiplicación de matricecs.

        En $GL(n,\mathbb{R})$ considere la topología heredada por ser un subesapcio del espacio euclideano real de dimensión $n^2$, es decir, con la topología generada por la métrica:
        \begin{equation*}
            d(A,B)=\sqrt{\sum_{i,j=1}^n\abs{A_{i,j}-B_{i,j}}^2},
        \end{equation*}
        para cualesquier $A=\left(A_{i,j}\right)$, $B=\left(B_{i,j}\right)$. Observe que la función $(A,B)\mapsto AB^{-1}$ es continua pues los elementos de la matriz producto son sumas de productos de los elementos de $A$ y $B$.
    \end{exa}

    \subsection{Grupos Ordenados}

    Se presentarán en esta parte dos ejemplos de grupos topológicos cuya construcción es interesante. Estos ejemplos (aunque no se analicen a profundidad más adelante en el texto) se exponen con el propósito de ayudar al lector a familiarizarse con la noción de grupo topológico. Se estudiará la estructura de grupo ordenado.

    Sea $G$ un grupo con más de un elemento que está ordenado linelamente por una relación $<$, es decir, $<$ cumple las condiciones siguientes:
    \begin{itemize}
        \item $<$ es irreflexiva (para toda $x\in G$, $x\nless x$).
        \item $<$ es antisimétrica (para todo $x,y\in G$ se tiene que $x<y$ ó $y<x$).
        \item Cualesquiera dos elementos de $G$ son comparables (ley de tricotomía).
    \end{itemize}
    Falta establecer una conexión entre este orden lineal y las operaciones del grupo, para lo cual se supone además lo siguiente:
    \begin{itemize}
        \item Si $x,y\in G$ son tales que $x<y$, entonces para todo $a\in G$ se tiene que $ax<ay$ y $xa<ya$.
    \end{itemize}
    Los grupos con esta estructura se conocen como \textbf{grupos linealmente ordenados}.

    Hay varias propiedades que tienen estos grupos, las cuales se probarán a continuación:

    \begin{propo}
        Sea $G$ un grupo linealmente ordenado por $<$. Entonces,
        $G$ no tiene elementos máximo y mínimo (esto implica que $G$ es infinito si $G$ tiene más de un elemento).
    \end{propo}

    \begin{proof}
        Observemos que $e_G$ no puede ser máximo o mínimo, ya que si $a<e_G$ entonces $e_G<a^{-1}$, para todo $a\in G$. Por otro lado si $x$ fuera elemento mínimo de $G$, en particular $x<e_G$ (ya que $x\neq e_G$) y, por ende $x^2<x$\contradiction. Por tanto $G$ no tiene elementos máximo o mínimo.
    \end{proof}

    Definamos con lo anterior una topología en $G$ como sigue: si $a,b\in G$ y $a<b$, sea $(a,b)=\left\{x\in G\big|a<x<b \right\}$; la familia
    \begin{equation*}
        \mathcal{B}=\left\{(a,b)\big| a,b\in G, a<b \right\}
    \end{equation*}
    forma una base de una topología en $G$. En efecto, veamos que se cumplen las dos condiciones:
    \begin{enumerate}
        \item Si $(a_1,b_1),(a_2,b_2)\in\mathcal{B}$, donde $a_1<b_1$ y $a_2<b_2$ entonces si $x\in(a_1,b_1)\cap(a_2,b_2)$, se tiene que $a_2<b_1$, ya que en otro caso $x$ no podría estar en la intersección, luego el conjunto $(a_2,b_1)\in\mathcal{B}$ y se tiene que $x\in(a_2,b_1)$.
        \item Sea $x\in G$. Como $G$ no admite elementos máximos ni mínimos, existen $a,b\in G$ tales que $a<x<b$. Por ende, $x\in(a,b)\in\mathcal{B}$.
    \end{enumerate}
    Por las dos condiciones anteriores, se tiene que al cumplirlas $\mathcal{B}$ existe una única topología $\tau$ para la cual $\mathcal{B}$ es una base. Consideremos de ahora en adelante tal topología. Veamos que las funciones
    \begin{equation*}
        \begin{split}
            h:G&\rightarrow G\\
            a&\mapsto a^{-1}\\
        \end{split}
    \end{equation*}
    y
    \begin{equation*}
        \begin{split}
            g:G\times G&\rightarrow G\\
            (a,b)&\mapsto ab\\
        \end{split}
    \end{equation*}
    son funciones continuas en $G$. En efecto, sea $(a,b)\in\mathcal{B}$, donde $a,b\in G$. Entonces:
    \begin{equation*}
        \begin{split}
            h^{-1}((a,b))=&\left\{x\in G\big| a<h(x)<b \right\}\\
            =&\left\{x\in G\big| a<x^{-1}<b \right\}\\
        \end{split}
    \end{equation*}
    pero, si $a<x^{-1}\Rightarrow xa<e_G\Rightarrow x<a^{-1}$. Por ende:
    \begin{equation*}
        \begin{split}
            h^{-1}((a,b))=&\left\{x\in G\big| b^{-1}<x<a^{-1} \right\}\\
            =&(b^{-1},a^{-1})\\
        \end{split}
    \end{equation*}
    es decir, que imágenes inversas de vecindades abiertos son abiertas. Por tanto, $h$ es continua.

    Ahora para $g$

    \subsection{Grupos Booleanos}

    \section{Homomorfismos e isomorfismos}

    \begin{mydef}
        Decimos que el homomorfismo $\cf{f}{G}{G'}$ es \textbf{homomorfismo abierto} si $f$ es una función abierta (es decir, que manda abiertos en abiertos).
    \end{mydef}

    Este concepto es importante pues permite establecer el concepto de grupso topológicos equivalentes. A continación se enunciarán y demostrarán propiedades elementales importantes de los homomorfismos continuos.

    \begin{lema}
        Sea $\cf{\varphi}{G}{H}$ un homomorfismo entre grupos topológicos. El homomorfismo $\varphi$ es continuo (respectivamente, abierto) si lo es en la identidad $e_G$, es decir, si $\varphi$ satisface la condición (1) (respectivamente (2)) siguiente:
        \begin{enumerate}
            \item Para toda $W$ vecindad de $e_H$ en $H$, existe $U$ vecindad de $e_G$ en $G$ tal que $\varphi(U)\subseteq W$.
            \item Para toda vecindad $U$ de $e_G$ en $G$, existe $W$ vecindad de $e_H$ tal que $W\subseteq\varphi(U)$.
        \end{enumerate}
    \end{lema}

    \begin{proof}
        Supongamos que se cumple la condición (1), debemos probar que $\varphi$ es continua en todo punto de $G$. Para ello, basta con ver que si $g\in G$ es arbitrario y $W$ es una vecindad de $\varphi(g)$ en $H$, entonces existe una vecindad $U$ de $G$ tal que $\varphi(U)\subseteq W$.

        Sean $g\in G$ y $W$ es una vecindad de $h=\varphi(g)$ en $H$. Se puede expresar a $W=hW'$ $W'$ es una vecindad de $e_H$. Por (1) existe una vecindad $U'$ de $e_G$ tal que $\varphi(U')\subseteq W'$. Entonces $U=gU'$ es una vecindad de $g$ y,
        \begin{equation*}
            \varphi(U)=\varphi(gU')=\varphi(g)\varphi(U')=h\varphi(U')\subseteq hW'=W
        \end{equation*}
        por tanto, $\varphi$ es continua en $g$.

        Para (2), debemos probar que dado un abierto $O$ en $G$, su imagen respecto a $\varphi$ es abierta en $H$.

        Sea entonces $O$ abierto en $G$ y $h\in\varphi(O)$; entonces $h=\varphi(g)$ para alguna $g\in G$. Por lo anteiror, $g^{-1}O$ es una vecindad de $e_G$, aunado con la condición (2) se sigue que existe una vecindad $W$ de $e_H$ tal que $W\subseteq \varphi^{-1}(g^{-1}O)=\varphi^{-1}(g)\varphi(O)$, por lo cual $hW=\varphi(g)W\subseteq\varphi(O)$, siendo $hW$ vecindad de $h$.
    \end{proof}

    ¿Existen homomorfismos entre grupos topológicos que no sean abiertos? La respuesta a esta pregunta es que sí.

    \begin{exa}
        Sea $(G,\tau)$ un grupo topológico no discreto. Considere los grupos $(G,\tau)$ y $(G,\tau_D)$, donde $\tau_D$ es la topología discreta sobre $G$. Entonces la función identidad $\cf{id_G}{G}{G}$ es un homomorfismo de $G$ sobre sí mismo el cual es continuo, pero no es un mapeo abierto entre los grupos topológicos $(G,\tau_D)$ y $(G,\tau)$.
    \end{exa}

    \section{Subgrupos topológicos y grupo cociente}
    
    Una nueva forma de producir grupos topológicos es mediante los subgrupos de un grupo topológico y, mediante el grupo cociente (dotado de la toplogía cociente). Se pretende estudiar estas estructuras con el afán de producir más grupos de este tipo.

    \begin{mydef}
        Sea $G$ un grupo topológico; un subconjunto no vacío $H$ de $G$ se llama  \textbf{subgrupo topológico} de $G$ si
        \begin{enumerate}
            \item $H$ es subgrupo de $G$.
            \item $H$ es un subespacio con la topología inducida por $G$.
        \end{enumerate}
    \end{mydef}

    El siguiente resultado justifica la definción de subgrupo topológico en el sentido de que este último es por sí mismo un grupo topológico.
    
    \begin{propo}
        Sean $G$ un grupo topológico y $H$ un subgrupo topológico de $G$. Entonces, $H$ es un grupo topológico con la topología que hereda de $G$.
    \end{propo}

    \begin{proof}
        Es inmediata del hecho de que la función $(x,y)\mapsto xy^{-1}$ es continua reestringiada a $H\times H$, y va en $H$.
    \end{proof}

    Se presentan ahora propiedades generales de cerradura dentro de un grupo topológico.

    \begin{propo}
        Si $A$ y $B$ son subconjuntos de un grupo topológico $G$, entonces:
        \begin{enumerate}
            \item $\Cls{A}\cdot\Cls{B}\subseteq \Cls{AB}$.
            \item $(\Cls{A})^{-1}=\Cls{A^{-1}}$
            \item $x\Cls{A}y=\Cls{xAy}$, para cualesquier $x,y\in G$.
            \item Si $ab=ba$ para toda $a\in A$ y $b\in B$, entonces $ab=ba$ para toda $a\in\Cls{A}$ y $b\in\Cls{B}$.
        \end{enumerate}
    \end{propo}

    \begin{proof}
        De (1): Sean $x\in \Cls{A}$ y $y\in \Cls{B}$ y $W$ un abierto tal que $xy\in W$. Debemos probar que $W\cap (AB)\neq\emptyset$. En efecto, como $(x,y)\mapsto x\cdot y$ es continua, existen $V_1$ y $V_2$ vecindades de $x$ y $y$, respectivamente, tales que $V_1\cdot V_2\subseteq W$. Por como se eligieron $x$ y $y$, existen $a\in V_1$ y $b\in V_2$ tales que $a\in A$ y $b\in B$, luego $ab\in (AB)\cap W$, pues $ab\in V_1\cdot V_2=W$. Así, $xy\in \Cls{AB}$.

        De (2) y (3): Son inmediatas del hecho de que para cualquier homeomorfismo $\cf{f}{G}{G}$ se tiene que $f(\Cls{A})=\Cls{f(A)}$, siendo $A\subseteq G$ arbitrario, en particular tomamos como $f$ a las funciones $z\mapsto z^{-1}$ y $z\mapsto xzy$, siendo $x,y\in G$ arbitrarios fijos.

        De (4): Considere $\cf{h}{G\times G}{G}$ tal que $(a,b)\mapsto aba^{-1}b^{-1}$. Esta función es contiua, por lo cual el conjunto:
        \begin{equation*}
            \begin{split}
                H&= h^{-1}(e_G)\\
                &=\left\{(a,b)\in G\times\Big| h(a,b)=e_G \right\}\\
                &=\left\{(a,b)\in G\times\Big| aba^{-1}b^{-1}=e_G \right\}\\
            \end{split}
        \end{equation*}
        es cerrado. Además, $A\times B\subseteq H$, por lo cual $\Cls{A\times B} \subseteq H$. Además, ya se sabe que $\Cls{A\times B}=\Cls{A}\times \Cls{B}$, por tanto $\Cls{A}\times\Cls{B}\subseteq H$, es decir que para todo $a\in\Cls{A}$ y $b\in\Cls{B}$, $ab=ba$.
    \end{proof}

    \begin{mydef}
        Sea $G$ grupo y $N$ un subgrupo de $H$. Decimos que \textbf{$N$ es subgrupo normal de $G$}, si $a^{-1}Na\subseteq N$, para todo $a\in G$.

        Esto es equivalente a que $a^{-1}Na=N$, para todo $a\in G$.
    \end{mydef}

    \begin{propo}
        Sean $G$ grupo topológico y $H,N$ subgrupos de $G$. Entonces
        \begin{enumerate}
            \item $\Cls{H}$ es subgrupo de $G$.
            \item Si $N$ es subgrupo normal de $G$, entonces $\Cls{N}$ también es subgrupo normal de $G$.
            \item $H$ es abierto si y sólo si su interior es no vacío.
            \item Si $H$ es abierto, entonces $\Cls{H}=H$.
        \end{enumerate}
    \end{propo}

    \begin{proof}
        De (1): Veamos que $\Cls{H}$ es un subgrupo de $G$. En efecto, como $H$ es un subgrupo de $G$, entonces $H^2\subseteq H$ y, por (1) de la proposición anterior, $\left(\Cls{H}\right)^2\subseteq\Cls{H^2}\subseteq \Cls{H}$ (es decir que $\Cls{H}$ es cerrado bajo el producto). Además, $H^{-1}\subseteq H$ y, por (2) de la proposición anterior, $\Cls{H}^{-1}=\Cls{H^{-1}}\subseteq \Cls{H}$ (es decir que $\Cls{H}$ es cerrado bajo inversos). Luego, $\Cls{H}$ es subgrupo de $G$ (por ser $H$ subgrupo, $\Cls{H}$ es no vacío).

        De (2): Como $N$ es subgrupo normal de $G$, en particular es subgrupo de $G$, luego $\Cls{N}$ es subgrupo de $G$ (por (1)). Para ver que es subgrupo normal de $G$ hay que verificar la contención:
        \begin{equation*}
            a\Cls{N}a^{-1}\subseteq\Cls{N},\quad\forall a\in G
        \end{equation*}
        pero, por hipótesis y usando (3) de la proposición anterior, se tiene que:
        \begin{equation*}
            \begin{split}
                aNa^{-1}&\subseteq N,\quad\forall a\in G\\
                \Rightarrow \Cls{aNa^{-1}}&\subseteq \Cls{N},\quad\forall a\in G\\
                \Rightarrow a\Cls{N}a^{-1}&\subseteq \Cls{N},\quad\forall a\in G\\
            \end{split}
        \end{equation*}
        por tanto, $\Cls{N}$ es subgrupo normal de $G$.

        De (3): Proabremos la doble implicación:

        $\Rightarrow)$: Suponga que $H$ es abierto, entonces es claro que su interior es no vacío ya que todos los puntos de $H$ son puntos interiores del mismo.

        $\Leftarrow)$: Suponga que $H$ tiene interior no vacío, sea $x\in H$ un punto interior, entonce existe $U$ vecindad de $e_G$ tal que $xU\subseteq H$. Ahora, para todo $y\in H$ se tiene que:
        \begin{equation*}
            yU= (yx^{-1})xU\subseteq yx^{-1}H=H
        \end{equation*}
        por tanto, $yU$ es una vecindad de $y$ contenida en $H$, luego $y$ es un punto interior. Por lo cual, $H$ es abierto ya que está contenido en su interior.

        De (4): Suponga que $H$ es abierto. Se tiene que el conjunto:
        \begin{equation*}
            G-H=\bigcup_{x\in G-H}Hx
        \end{equation*}
        es abierto, pues $Hx$ es abierto para cada $x\in G-H$. Luego $H$ es cerrado, es decir que $H=\Cls{H}$.

    \end{proof}

    El próximo resultado tiene como objetivo generar subgrupos abiertos a partir de vecindades de la identidad.

    \begin{theor}
        Sean $G$ un grupo topológico y $U$ cualquier vecindad abierta de $e_G$. Entonces, el conjunto:
        \begin{equation*}
            L=\bigcup_{n=1}^\infty U^n
        \end{equation*}
        es un subgrupo abierto y cerrado de $G$.
    \end{theor}

    \begin{proof}
        Es claro que $L\neq\emptyset$ ya que $e_G\in L$. Sean ahora $x,y\in G$, entonces existen $k,l\in\mathbb{N}$ tales que $x\in U^k$ y $y\in U^l$ luego, $xy\in U^{k+l}$ y $x^{-1}\in \left(U^{k}\right)^{-1}=(U^{-1})^k= U^k$, es decir que $xy,x^{-1}\in L$. Por tanto, $L$ es subgrupo de $G$.

        Además, es abierto por ser unión arbitraria de abiertos, y cerrado por el inciso (4) de la proposición anterior.
    \end{proof}

    \begin{obs}
        Un subespacio topológico de un espacio topológico es \textbf{discreto} si todos los puntos del subespacio son aislados, que es lo mismo que decir que está dotado de la topología discreta.
    \end{obs}

    \begin{propo}
        Un subgrupo $H$ de un grupo topológico $G$ es discreto si y sólo si tiene un punto aislado.
    \end{propo}

    \begin{proof}
        $\Rightarrow)$: Se tiene que $H$ es un subgrupo discreto, es decir que está dotado de la topología discreta luego, el conjunto $\left\{e_G\right\}$ es una vecindad abierta de $e_G$ para la cual se cumple que $\left\{e_G\right\}\cap H=\left\{e_G\right\}$, es decir que es un punto aislado de $H$.

        $\Leftarrow)$: Sea $x\in H$ un punto aislado de $H$; es decir que existe una vecindad abierta $U$ de $e_G$ tal que $xU\cap H=\left\{x\right\}$. Si ahora $y\in H$, se tiene entonces que:
        \begin{equation*}
            yU\cap H=yU\cap yx^{-1}H=yx^{-1}(xU\cap H)=yx^{-1}\left\{x\right\}=\left\{y\right\}
        \end{equation*}
        es decir, que todos los puntos de $H$ son aislados. Por tanto, $H$ es discreto.
    \end{proof}

    El siguiente lema proporciona una condición suficiente para probar que un subgrupo topológico es cerrado.

    \begin{lema}
        Sean $G$ un grupo topológico y $H$ un subgrupo de $G$ tales que $\Cls{U}\cap H$ es cerrado en $G$ para alguna vecindad abierta $U$ de $e_G$. Entonces, $H$ es cerrado.
    \end{lema}

    \begin{proof}
        Sea $U$ una vecindad abierta de $e_G$ tal que $\Cls{U}\cap H$ es cerrado, y sea $V$ una vecindad abierta de $e_G$ tal que $V^{2}\subseteq U$. Probaremos que $H=\Cls{H}$.

        En efecto, sea $x\in \Cls{H}$, entonces el conjunto $xV\cap H\neq\emptyset$, sea $y\in xV\cap H$. Afirmamos que $xy\in \Cls{U}\cap H$. En efecto, suponga que esto no es cierto, entonces al ser este un conjunto cerrado, existiría una vecindad abierta $W$ de $e_G$ tal que
        \begin{equation*}
            Wxy\cap \Cls{U}\cap H=\emptyset
        \end{equation*}
        Ahora, es claro que $x\in (W\cap V)x$ y, como $x\in\Cls{H}$, entonces existiría $z\in (W\cap V)x\cap H$. Este elemento cumple que:
        \begin{itemize}
            \item $zy\in HH=H$.
            \item $zy\in Vxx^{-1}V=V^{2}\subseteq U\subseteq\Cls{U}$.
            \item $zy\in (Wx)y=Wxy$
        \end{itemize}
        por ende, $zy\in Wxy\cap \Cls{U}\cap H$\contradiction, pues hemos dicho que tal conjunto es vacío. Luego $xy\in\Cls{U}\cap H$, en particular $xy\in H$. Como $y^{-1}\in H^{-1}\subseteq H$ se sigue que $x=(xy)y^{-1}\in HH=H$, lo cual termina la demostración.
    \end{proof}

    Este siguiente teorema presenta una propiedad exclusiva de los grupos topológicos, pues no todo subespacio discreto de un espacio topológico es cerrado.

    \begin{propo}
        Todo subgrupo discreto $H$ de un grupo topológico $G$ es cerrado.
    \end{propo}

    \begin{proof}
        Sea $U$ una vecindad abierta de $e_G$ tal que $U\cap H=\left\{e_G\right\}$. Por un lema anterior existe $V$ vecindad de $e_G$ tal que $\Cls{V}\subseteq U$. Por tanto:
        \begin{equation*}
            \Cls{V}\cap H=\left\{e_G\right\}
        \end{equation*}
        donde, por ser $G$ un espacio de Hausdorff se sigue que $\left\{e_G\right\}$ es cerrado en $G$. Por tanto, del lema anterior se sigue que $H$ es cerrado.
    \end{proof}

    \section{Grupos cocientes}

    Se estudiarán ahora el grupo cociente de un grupo topológico. 

    Recordemos que si $G$ es un grupo y $H$ un subgrupo de $H$, entonces se define una relación de equivalencia en $G$ como sigue:
    \begin{equation*}
        a\sim b\textup{ si y sólo si }ab^{-1}\in H
    \end{equation*}
    Las clases de equivalencia de esta relación reciben el nombre de \textbf{clases laterales derechas de $H$}. Denotaremos por $G/_DH$ al conjunto formado formado por las clases laterales derechas, es decir:
    \begin{equation*}
        G/_DH=\left\{Ha\Big|a\in G \right\}
    \end{equation*}
    de manera similar se definen las clases laterales laterales izquierdas de $H$. Se denota por $G/_I H$ al conjunto:
    \begin{equation*}
        G/_IH=\left\{aH\Big|a\in G \right\}
    \end{equation*}
    En caso de que el grupo sea normal, se tiene que $G/_IH=G/_DH=G/H$. Los siguientes resultados se establecen para clases laterales derechas y de forma análoga se cumplen para clases laterales izquierdas.

    Introducimos ahora en $G/H$ una topología de la siguiente manera. Sea $\mathcal{B}$ una base del grupo topológico $G$ y $H$ un subgrupo de $G$. Para cada $U\in\mathcal{B}$ definamos:
    \begin{equation*}
        U^*=\left\{ Hx\Big|x\in U \right\}
    \end{equation*}
    y:
    \begin{equation*}
        \mathcal{B}^*=\left\{U^*\Big|U\in\mathcal{B} \right\}
    \end{equation*}

    \begin{propo}
        Para todo subgrupo $H$ de $G$, $\mathcal{B}^*$ es una base de una topología sobre $G/_DH$. Si $H$ es cerrado, entonces esta topología sobre $G/_DH$ es $T_1$.
    \end{propo}

    \begin{proof}
        Para la primera parte se deben verificar dos condiciones:
        \begin{enumerate}
            \item Sean $U^*,V^*\in\mathcal{B}^*$. Si $Ha\in U^*\cap V^*$ debemos encontrar $W^*\in\mathcal{B}^*$ tal que $Ha\in W^*\subseteq U^*\cap V^*$.
            
            Como $U^*,V^*\in\mathcal{B}^*$, entonces existe $u\in U$ y $v\in V$ tales que:
            \begin{equation*}
                Ha=Hu=Hv
            \end{equation*}
            por lo cual, $Ha\subseteq HU,HV$, es decir que $Ha\subseteq HU\cap HV$. Además, $HU\cap HV$ es un conjunto abierto (por ser intersección de dos abiertos). Como $u\in HU\cap HV$, entonces existe $W\in\mathcal{B}$ tal que:
            \begin{equation*}
                u\in W\subseteq HU\cap HV
            \end{equation*}
            Claramente se tiene que $Ha=Hu\in W^*$. Veremos que $W^*\subseteq U^*\cap V^*$. En efecto, sea $Hw\in W^*$, es decir que $w\in W$. Por la contención anterior, $w\in HU\cap HV$, es decir que existen $h_1,h_2\in H$, $u_1\in U$ y $v_1\in V$ tales que:
            \begin{equation*}
                w=h_1u_1=h_2v_1
            \end{equation*}
            Luego:
            \begin{equation*}
                Hw=Hh_1u_1=Hu_1\in U^*\quad\textup{y}\quad Hw=Hh_2v_1=Hv_1\in V^*
            \end{equation*}
            Por lo cual, $Hw\in U^*\cap V^*$. Finalmente, se sigue que $W^*\subseteq U^*\cap V^*$.

            \item Sea $Hx\in G/_DH$. Como $\mathcal{B}$ es base para $G$, entonces existe $U\in\mathcal{B}$ tal que $x\in U$, luego $Hx\in U^*$. Así que
            \begin{equation*}
                G/_DH=\bigcup\mathcal{B}^*
            \end{equation*}
        \end{enumerate}
        por ambos incisos, se sigue que $\mathcal{B}^*$ es base de una topología sobre $G/_DH$.

        Para la otra parte, suponga que $H$ es cerrado. Sean $Ha\neq Hb$ dos elementos de $G/_DH$. Como $Ha$ es cerrado (pues $H$ es cerrado) y $b\notin Ha$, entonces existe $U\in\mathcal{B}$ tal que $b\in U$ y $U\cap Ha=\emptyset$ (pues, $G$ es un espacio $T_3$). Entonces, $Hb\in U^*$ y $Ha\notin U^*$. Por tanto, $G/_DH$ es un espacio $T_1$.
    \end{proof}

    \begin{obs}
        El espacio topológico $G/H$ así construido recibe el nombre de \textbf{espacio cociente de $G$ entre $H$}, o \textbf{grupo cociente de $G$ entre $H$}, si $H$ es cerrado y normal en $G$.
    \end{obs}

    Ahora se estudian las propiedades de la función canónica $\cf{\pi}{G}{G/_DH}$, tal que $x\mapsto Hx$.

    \begin{propo}
        Sean $G$ un grupo topológico, $H$ un subgrupo cerrado de $G$ y $\cf{\pi}{G}{G/_DH}$ la función canónica tal que $x\mapsto Hx$. Entonces, $\pi$ es continua y abierta.
    \end{propo}

    \begin{proof}
        Sea $x\in G$ y $U\subseteq G$ un abierto tal que $\pi(x)=Hx\in U^*$. Para probar que $\pi$ es continua, hay que encontrar un abierto $V$ en $G$ tal que
        \begin{equation*}
            \pi(V)\subseteq U^*
        \end{equation*}
        tomemos $V=HU$, el cual es abierto por ser $U$ abierto en $G$, además $x\in V$ y:
        \begin{equation*}
            \pi(V)=\pi(HU)=\pi(U)=U^*
        \end{equation*}
        por tanto, $\pi$ es continua.

        Veamos ahora que es un mapeo abierto. Sea $U\subseteq G$ abierto y tomemos $\mathcal{B}$ una base de la topología de $G$; entonces:
        \begin{equation*}
            U=\bigcup_{ j\in J}B_j
        \end{equation*}
        donde $\left\{B_j \right\}_{j\in J}\subseteq \mathcal{B}$. Tenemos que:
        \begin{equation*}
            \pi(U)=\pi(\bigcup_{ j\in J}B_j)=\bigcup_{ j\in J}\pi(B_j)=\bigcup_{ j\in J}B_j^*
        \end{equation*}
        donde $B_j^*\in\mathcal{B}^*$, para todo $j\in J$. Luego, $\pi(U)$ es un mapeo abierto.
    \end{proof}

    En general la función canónica $\cf{\pi}{G}{G/_DH}$ no es cerrada como lo muestra el siguiente ejemplo. Sin embargo, si $H$ es compacto entonces la función sí es cerrada (esto se verá en el siguiente capítulo).

    \begin{exa}
        Considere $G=\mathbb{R}$ como grupo aditivo de los reales dotado de la topología usual, y $H=\mathbb{Z}$ el subgrupo de los números enteros. Entonces, definimos la función $\cf{\psi}{\mathbb{R}/\mathbb{Z}}{[0,1[}$
        dada por:
        \begin{equation*}
            \psi(\mathbb{Z}+x)=x-[x]
        \end{equation*}
        donde $[x]$ denota a la parte entera de $x$. Esta función es biyectiva (más aún, es un isomorfismo de grupos, dotando a $[0,1[$ con cierta operación binaria que lo haga grupo) y es un homeomofismos si se le dota a este conjunto de la topolgía usual de $[0,1[$.

        Se tiene además, que el conjunto $A=\left\{2+\frac{1}{2},...,n+\frac{1}{n},... \right\}$ es cerrado en $\mathbb{R}$, pero su imagen bajo la función canónica es $\pi(A)=\left\{\mathbb{Z}+\frac{1}{2},...,\mathbb{Z}+\frac{1}{n},... \right\}$, el cual es homeomorfo a $\left\{\frac{1}{2},...,\frac{1}{n},... \right\}$ no es cerrado en $[0,1[$, por lo cual no el conjunto $\pi(A)$ no es cerrado en $\mathbb{R}/\mathbb{Z}$.
    \end{exa}

    El siguiente resultado muestra que las propiedades topológicas locales del espacio cociente de $G$ en $H$ también se pueden estudiar en un solo punto.

    \begin{propo}
        Sean $G$ un grupo topológico y $H$ un subgrupo cerrado de $G$. Entonces, $G/_DH$ es un espacio homogéneo.
    \end{propo}

    \begin{proof}
        Hay que probar que para todo para de puntos $Ha,Hb\in G/_DH$ existe un homeomorfismo $f$ de $G_D/H$ en sí mismo tal que $f(Ha)=Hb$.
        
        Sea $\cf{\psi_a}{G_D/H}{G_D/H}$ dada por: $Hx\mapsto H(xa)$. Es claro que esta función está bien definida y, afirmamos que es un homeomorfismo. 
        
        Veamos que es inyectiva, en efecto, si $\psi_a(Hx)=\psi_a(Hy)$, entonces $H(xa)=H(ya)$, luego $(xa)(ya)^1=xaa^{-1}y^{-1}=xy^{-1}\in H$, luego $Hx=Hy$.

        Y, es suprayectiva, pues para todo $x\in G$, $\psi_a(H(xa^{-1}))=Hx$.

        Además, tiene como función inversa a $(\psi_a)^{-1}=\psi_{a^{-1}}$. Por lo cual, para ver que es un homeomorfismo, basta con ver que ella y su inversa son mapeos inversos, pero, por como está dada la función inversa, es suficiente con ver que $\psi_a$ es abierta (ya que, el $a\in G$ fue arbitrario).

        Considere $\cf{\pi}{G}{G/_DH}$ la función canónica, y sea $U\subseteq G$ abierto,entonces $U^=\pi(U)*$ es abierto en $G/_DH$. Entonces, 
        \begin{equation*}
            \begin{split}
                \psi_a(U^*)&=\left\{\psi_a(Hx)\Big| Hx\in U^* \right\}\\
                &=\left\{H(xa) \Big| x\in U \right\}\\
                &=\left\{Hy \Big| y\in Ua \right\}\\
                &=\pi(Ua)\\
            \end{split}
        \end{equation*}
        donde, como $U$ es abierto en $G$, entonces $Ua$ también lo es y, al ser $\pi$ un mapeo abierto, se sigue que $\psi_a(U^*)$ es abierta. 

        Por tanto, $\psi_a$ es homeomorfismo. Obsevemos ahora que si $Ha,Hb\in G/_DH$, entonces:
        \begin{equation*}
            \psi_{a^{-1}b}(Ha)=H(aa^{-1}b)=Hb
        \end{equation*}
        con lo que se tiene que $G/_DH$ es un espaico homogéneo.
    \end{proof}

    El siguiente resultado es un lema auxiliar que nos permitirá establecer dos propiedades topológicas de $G/_DH$.

    \begin{lema}
        Sean $G$ un grupo topológico y $H$ un subgrupo cerrado de $G$, $U,V$ vecindades abiertas de $e_G$ en $G$ tales que $VV^{-1}\subseteq U$; entonces, si $\cf{\pi}{G}{G/_DH}$ es la función canónica, se cumple que
        \begin{equation*}
            \Cls{\pi(V)}\subseteq\pi(U)
        \end{equation*}
    \end{lema}

    \begin{proof}
        Sea $Hx\in\Cls{\pi(V)}$ donde $x\in G$; entonces, $\pi(xV)$ es una vecindad abierta que contiene a $Hx$ y, por ende, contiene puntos de $\pi(V)$. Esto es que la intersección:
        \begin{equation*}
            \pi(xV)\cap\pi(V)
        \end{equation*}
        es no vacía, luego existen, $v_1,v_2\in V$ tales que
        \begin{equation*}
            Hxv_1=Hv_2
        \end{equation*}
        esto es, que $Hx=Hv_2v_1^{-1}\in H(VV^{-1})\subseteq H(U)$. Por tanto se sigue la contención deseada.
    \end{proof}

    \begin{theor}
        Sean $G$ un grupo topológico y $H$ un subgrupo cerrado de $G$. Entonces,
        \begin{enumerate}
            \item $G/_DH$ es un espacio regular y por tanto de Hausdorff.
            \item $G/_DH$ es un espacio discreto si y sólo si $H$ es abierto en $G$.
        \end{enumerate}
    \end{theor}

    \begin{proof}
        De (1): Por hipótesis, $H$ es cerrado en $G$, luego el conjunto $Ha$ es cerrado en $G$, para todo $a\in G$. Sea $a\in G$, se tiene que el conjunto:
        \begin{equation*}
            G/_DH\backslash\left\{Ha \right\}=\left\{Hx\in G_D/H\Big|Hx\neq Ha \right\}=\pi(G\backslash Ha)
        \end{equation*}
        es abierto, pues es la imagen de un abierto y el mapeo $\pi$ es abierto, luego cada punto $\left\{Ha \right\}$ es un conjunto cerrado en $G/_DH$, es decir que $G/_DH$ es $T_2$, en particular es $T_1$. Para probar que es regular (es decir, $T_3$).

        Para ello, es suficiente probar que para todo abierto $U^*$ en $G/_DH$ tal que $H\in U^*$ (basta hacerlo aquí pues, el espacio es homogéneo) existe una vecindad $V^*$ de $H$ tal que $\Cls{V^*}\subseteq U^*$. En efecto, si $U^*$ es un abierto tal que $H\in U^*$, entonces existe $U\subseteq G$ abierto tal que $\pi(U)=U^*$.

        En particular, se tiene que $e_G\in U$, luego por un lema anterior existe $V\subseteq G$ vecindad abierta de $e_G$ tal que:
        \begin{equation*}
            VV^{-1}\subseteq U
        \end{equation*}
        por el lema anterior, se sigue que $\Cls{\pi(V)}\subseteq \pi(U)$, es decir que $\Cls{V^*}\subseteq U^*$, donde $V^*$ es un abierto que contiene a $H$.

        De (2): Se probará la doble implicación.

        $\Rightarrow)$: Si $G/_DH$ es discreto, entonces el conjunto $\left\{H\right\}=\left\{He_G\right\}$ es abierto, luego su imagen inversa bajo $\pi$, $\pi^{-1}(He_G)=H$ es abierta en por ser $\pi$ una función continua, es decir que $H$ es abierto en $G$.

        $\Leftarrow)$: Si $H$ es abierto, como $\pi$ es un mapeo abierto, se sigue que $\pi(H)=\left\{H\right\}$ es abierto en $G/_DH$, esto es, es un punto aislado de $G/_DH$, luego el espacio $G/_DH$ es discreto ya que es homogéneo.
    \end{proof}

    En el caso particular de que $N$ sea un subgrupo normal de $G$, de modo que $G/N$ sea grupo, se tiene el siguiente resultado:
    
    \begin{theor}
        Sean $G$ un grupo topológico y $N$ un subgrupo cerrado normal de $G$; entonces
        \begin{enumerate}
            \item $G/N$ con la topología cociente es un grupo topológico.
            \item La función continua $\cf{\pi}{G}{G/N}$ es un homomorfismo abierto y continuo.
            \item El grupo $G/N$ es un espacio $T_1$ y por tanto, regular.
            \item El grupo $G/N$ es discreto si y sólo si $N$ es abierto.
        \end{enumerate}
    \end{theor}

    \begin{proof}
        De (1): Hay que probar que la función $(Na,Nb)\mapsto Nab^{-1}$ es continua. En efecto, sean $a,b\in G$ y $W^*=\pi(W)\subseteq G/N$ un abierto (siendo $W\subseteq G$ abierto) tal que $Nab^{-1}\in W^*$ (se sigue que $ab^{-1}\in W$).

        Como la función $(a,b)\mapsto ab^{-1}$ es continua, entonces existen $U,V\subseteq G$ abiertos tales que $a\in U$ ,$b\in V$ y
        \begin{equation*}
            UV^{-1}\subseteq W
        \end{equation*}
        por tanto, si $u\in U$ y $v\in V$, se tiene que $Nu\in U^*$, $Nv\in V^*$, con:
        \begin{equation*}
            (Nu)(Nv)^{-1}=Nuv^{-1}\in W^*
        \end{equation*}
        pues, $uv^{-1}\in W$. Por ende, $U^*(V^*)^{-1}\subseteq W^*$. Por tanto, la función establecida originalmente es continua.
        
        De (2): Ya se sabe que la función canónica $\pi$ es continua y abierta. Además, es homomorfismo por ser $N$ subgrupo normal de $G$.

        De (3) Es inmediato del teorema anterior.

        De (4) Es inmediato del teorema anterior.
    \end{proof}

    Resulta que, podemos establecer el primer teorema de isomorfismos en la parte de grupos topológicos, haciendo ciertas limitaciones correspondientes para que todo funcione adecuadamente.

    \begin{theor}
        Sean $G$ y $G'$ dos grupos topológicos y sea $\cf{f}{G}{G'}$ un epimorfismo continuo y abierto con kernel $N=\ker(f)$. Entonces, $N$ es un subgrupo normal y cerrado de $G$ y, el isomorfismo $h$ de $G/N$ en $G'$ dado por:
        \begin{equation*}
            h(Nx)=f(x)
        \end{equation*}
        para todo $Nx\in G/N$, es un isomorfismo topológico entre $G/N$ y $G'$.
    \end{theor}

    \begin{proof}
        El conjunto $N$ es un subgrupo normal de $G$ pues es el núcleo del homomorfismo $f$. Además, $N$ es cerrado ya que es la imágen inversa del cerrado $\left\{e'\right\}\subseteq G'$.

        Sea $\cf{\pi}{G}{G/N}$ el homomorfismo canónico. Conviene notar que $h$ se ha definido de tal manera que $f=h\circ\pi$. Veamos que $h$ es una biyección continua y abierta (con lo cual, se sigue que $h$ es isomorfismo topológico ya que es homomorfismo y homeomorfismo). En efecto, probar que $h$ es biyección no es complicado, al igua que probar que es homomorfismo (ver la demostración del primer teorema de isomorfismo de grupos). Veamos que es abierta y continua:
        \begin{itemize}
            \item \textbf{$h$ es continua}. En efecto, sea $V\subseteq G'$ abierto. Afirmamos que:
            \begin{equation}
                h^{-1}(V)=\pi(f^{-1}(V))
            \end{equation}
            en efecto, se tiene que:
            \begin{equation*}
                f^{-1}(V)=(h\circ \pi)^{-1}(V)=\pi^{-1}(h^{-1}(V))
            \end{equation*}
            Como $\pi$ es suprayectiva, se cumple entonces que:
            \begin{equation*}
                \pi(f^{-1}(V))=\pi(\pi^{-1}(h^{-1}(V)))=h^{-1}(V)
            \end{equation*}
            con lo que se tiene la afirmación. Como $f$ es continua, entonces $f^{-1}(V)$ es abierto en $G$ luego, al ser $\pi$ un mapeo abierto, se tiene que $\pi(f^{-1}(V))=h^{1}(V)$ es abierto. Por ende, $h$ es continua.
            \item \textbf{$h$ es abierta}. En efecto, sea $U\subseteq G/N$ abierto, entonces existe $V\subseteq G$ abierto tal que $\pi(V)=U$. Por ende:
            \begin{equation*}
                h(U)=h(\pi(V))=f(V)
            \end{equation*}
            donde el miembro de la izquierda es abierto ya que $f$ es función abierta. Por tanto, $h$ es abierto.
        \end{itemize}
        Luego, $h$ es isomorfismo topológico.
    \end{proof}

    %disertación para estblecer relación entre subgrupos normales cerrados de un homomorfismo.

    \begin{mydef}
        El \textbf{grupo del círculo} se define como:
        \begin{equation*}
            \mathbb{T}=\mathbb{R}/\mathbb{Z}
        \end{equation*}
    \end{mydef}

    \begin{obs}
        Como $\mathbb{Z}$ es un subgrupo cerrado y normal de $\mathbb{R}$, entonces $\mathbb{T}$ es grupo topológico. Un conjunto completo de representantes es $R=[0,1[$, es decir:
        \begin{equation*}
            \mathbb{T}=\left\{r+\mathbb{Z}\Big|r\in R \right\}
        \end{equation*} 
        donde $r+\mathbb{Z}=\left[r\right]$ es la clase de equivalencia con representante $r$.
        Considere ahora a la circunferencia unitaria
        \begin{equation*}
            S^1=\left\{e^{2\pi ix}\in\mathbb{C}\Big|x\in[0,1[ \right\}
        \end{equation*}
        Sea $\cf{f}{\mathbb{R}}{S^1}$ tal que $x\mapsto e^{2\pi ix}$. Es claro que $f$ es un epimorfismo continuo y abierto (no es difícil de probar). Por tanto, como $\ker\left(f\right)=\mathbb{Z}$, se sigue que $\mathbb{T}$ y $S^1$ son topológicamente isomorfos.

        Además como $\cf{\pi}{\mathbb{R}}{\mathbb{T}}$ es continua, como $[0,1]$ es compacto, se sigue que $\pi([0,1])=\mathbb{T}$ también es compacto.
    \end{obs}

    \section{Productos Directos}

    Sea $\left\{G_i \right\}_{i\in I}$ una familia de grupos topológicos. Damos una estructura de grupo la conjunto $G=\prod_{i\in I}G_i$, definiendo:
    \begin{equation*}
        \left(x_i\right)_{ i\in I}\cdot\left(y_i\right)_{ i\in I}=\left(x_iy_i\right)_{ i\in I}
    \end{equation*}
    Si para cada $i\in I$, $e_i\in G_i$ es el elemento identidad, entonces $e=\left(e_i \right)_{ i\in I}$ es el elemento identidad de $G$, y 
    \begin{equation*}
        \left(x_i\right)_{ i\in I}^{-1}=\left(x_i^{-1}\right)_{ i\in I}
    \end{equation*}
    para cada $\left(x_i\right)_{ i\in I}\in G$. La topología producto es compatible con esta estructura de grupo pues, la función $\cf{h}{G\times G}{G}$ dada por:
    \begin{equation*}
        \left(\left(x_i\right)_{ i\in I},\left(y_i\right)_{ i\in I} \right)\mapsto \left(x_iy_i^{-1}\right)_{ i\in I}
    \end{equation*}
    es la composición de las funciones $\left((x_i,y_i)\right)_{i\in I}\mapsto \left(x_iy_i^{-1}\right)_{ i\in I}$ de $\prod_{ i\in I}\left(G_i\times G_i \right)$ en $G$ y, la proyección $\left(\left(x_i\right)_{ i\in I},\left(y_i\right)_{ i\in I}\right)\mapsto\left(\left(x_i,y_i\right) \right)_{i\in I}$ de $G\times G$ en $\prod_{ i\in I}\left(G_i\times G_i \right)$ son continuas (esto en la topología producto).

    \begin{mydef}
        Sea $\left\{G_i \right\}_{i\in I}$ una familia de grupos topológicos. El \textbf{producto directo} de los grupos topológicos $\left\{G_i\Big|i\in I \right\}$ se obtiene al dar al producto
        \begin{equation*}
            G=\prod_{i\in I}G_i
        \end{equation*}
        la topología producto.
    \end{mydef}

    \begin{propo}[\textbf{Asociatividad del producto directo de grupos}]
        Sea $\left\{G_i \right\}_{i\in I}$ una familia de grupos topológicos. Si $\left\{I_x \right\}_{x\in K}$ es una partición de $I$ entonces, $G$ es isomorfo al producto de los grupos topológicos $\prod_{i\in I_x}G_i$, es decir:
        \begin{equation*}
            G=\prod_{i\in I}G_i\equiv\prod_{x\in L}\prod_{i\in I_x}G_i
        \end{equation*}
    \end{propo}

    \begin{proof}
        La prueba no se hace en el libro, pero se deja pendiente.
    \end{proof}

    \begin{obs}
        Para cada $j\in I$, la función \textbf{proyección natural}, $\cf{\pi_j}{G}{G_j}$ tal que $\pi_j(x)=x_j$, para todo $x=\left(x_i\right)_{i\in I}$ es un homomorfismo continuo. Este último hecho se deduce de la construcción de la topología producto sobre $G$.

        Más aún, la función $\cf{\phi_j}{G_j}{G}$ definida por $\phi_j(x)=\left(y_i \right)_{ i\in I}$, donde:
        \begin{equation*}
            y_i=\left\{
                \begin{array}{lcr}
                    e_i & \textup{ si } & i \neq j.\\
                    x & \textup{ si } & i = j.\\
                \end{array}
            \right.\quad\forall y\in I
        \end{equation*}
        es un isomorfismo topológico entre $G_j$ y $N_j=\phi_j(G_j)$. Es decir, $\phi_j$ es una inmersión de $G_j$ en $G$.
    \end{obs}

    \begin{theor}
        Sea $\left\{G_i \right\}_{ i\in I}$ una familia de grupos topológicos y sea $H$ el subconjunto de $G=\prod_{ i\in I}G_i$ que consta de todos los $x=\left(x_i \right)_{ i\in I}$ tales que $x_i$ es el elemento identidad de $G_i$ para todo $i\in I$ salvo un número finito de índices; $H$ es entonces un subgrupo normal denso en $G$.
    \end{theor}

    \begin{proof}
        
    \end{proof}

    El objetivo de este teorema es el de determinar bajo que condiciones se puede agregar la parte topológica al concpeto de descomposición algebraica de un grupo en subgrupos. Sea $\left\{G_i \right\}_{i=1}^n$ una familia finita de grupos topológicos y $G$ su producto directo. Es fácil ver que para vecindades arbitrarias $U_1,...,U_n$ de $e_1,...,e_n$ relativas a $G_1,...,G_n$, respectivamnete, el producto $U_1\times\cdots\times U_n$ es una vecindad de $e_G$ en la topología de grupo $G$. Por todo esto, podemos plantear una definición de producto interno de grupos topológicos. Por todo esto, podemos plantear una definición de producto interno de grupos topológicos.

    \begin{mydef}
        Sea $G$ un grupo topológico y $N_1,...,N_n$ subgrupos cerrados normales de $G$. Diremos que el grupo topológico $G$ se \textbf{descompone topológicamente en el producto directo de los subgrupos $N_1,...,N_n$} si $G$ se descompone (en el sentido algebraico) el en producto directo de estos subgrupos y, además, para cualquier colección de vecindades abiertas $U_1,...,U_n$ de $e$, relativas a $N_1,...,N_n$, existe una vecindad de $e$ relativa a todo el grupo $G$ tal que
        \begin{equation*}
            U\subseteq U_1\cdot...\cdot U_n
        \end{equation*}
    \end{mydef}

    \begin{propo}
        Supongamos que el grupo topológico $G$ se descompone topológicamente en el producto directo de los subgrupos $N_1,...,N_n$, y sea $H$ el producto directo de estos subgrupos. A cada elemento $x=(x_1,...,x_n)\in H$ le asociamos el elemento $\psi(x)=x_1\cdot...\cdot x_n\in G$. Entonces, $\psi$ es un isomorfismo topológico entre $G$ y $H$. Además, $\psi\circ\phi_j=\id{N_j}$, donde las $\phi_j$ están dadas por $\phi_j(x)=\left(y_i \right)_{i=1}^n$, siendo
        \begin{equation*}
            y_i=\left\{
                \begin{array}{lcr}
                    e_i & \textup{ si } & i\neq j\\
                    x & \textup{ si } & i=j\\
                \end{array}
             \right\},\quad\forall i\in\natint{1,n}
        \end{equation*}
    \end{propo}

    \begin{proof}
        Ya se sabe de la parte algebraica que $\psi$ es un isomorfismo de grupos entre $G$ y $H$. Observemos que
        \begin{equation*}
            \begin{split}
                \psi\circ\phi_j(x)&=\psi(\phi_j(n))\\
                &=\psi\left(\left(y_i \right)_{ i=1}^n\right)\\
                &=y_1\cdot...\cdot y_n \\
                &=e\cdot...\cdot e\cdot x\cdot e\cdot...\cdot e\\
                &= x\\
                &= \id{N_j}(x) \\
            \end{split}
        \end{equation*}
        para todo $x\in N_j$, $j\in\natint{1,n}$, siendo $x$ el $j$-ésimo elemento del producto en la tercera linea de la igualdad. Por tanto, $\psi\circ\phi_j=\id{N_j}$ para todo $j\in\natint{1,n}$. Para terminar de probar el resultado, basta con probar que $\psi$ es un homeomorfismo. Como es biyectivo, solo hay que probar que es una función continua y abierta.
        \begin{enumerate}
            \item \textup{$\psi$ es continua}. Sea $U\subseteq G$ una vecindad arbitraria de $e$ en $G$ y $V\subseteq G$ otra vecindad de $e$ tal que $V^n\subseteq U$. Definimos
            \begin{equation*}
                V_i=N_i\cap V,\quad\forall i\in\natint{1,n}
            \end{equation*}
            Es claro que $V_i$ es una vecindad de $e$ en $N_i$, para todo $i\in\natint{1,n}$. Tomemos
            \begin{equation*}
                V'=\left\{(x_1,...,x_n)\in H\Big|x_i\in V_i,\textup{ para todo }i\in\natint{1,n} \right\}
            \end{equation*}
            Este conjunto cumple que es una vecindad de la identidad de $H$ (pues, $V'=V_1\times...\times V_n$ es un elemento básico de la topología producto). Además,
            \begin{equation*}
                \begin{split}
                    \psi(V')&=\psi(V_1\times...\times V_n)\\
                    &=V_1\cdot...\cdot V_n \\
                    &\subseteq \underbrace{V\cdot...\cdot V}_{n\textup{-veces}}\\
                    &=V^n\\
                    &\subseteq U\\
                    \Rightarrow \psi(V')&\subseteq U\\
                \end{split}
            \end{equation*}
            por tanto, $\psi$ es continua.
            \item \textbf{$\psi$ es abierta}. Sea $W$ una vecindad abierta de la identidad, entonces contiene un conjunto de la forma $V_1\times...\times V_n$, donde cada $V_i$ es un conjunto abierto, vecindad de la identidad en $N_i$. Por hipótesis, existe una vecindad $U$ abierta de la identidad tal que
            \begin{equation*}
                U\subseteq V_1\cdot... \cdot V_n=\psi(V)
            \end{equation*}
            por tanto, $\psi$ es abierta ¿?
            %TODO no entendí pq es abierta.
        \end{enumerate}
    \end{proof}

    \section{Cardinales invariantes elementales}

    Ahora se estudiarán algunas propieades de las funciones cardinales definidas en grupos.

    La primera propiedad, respecto a las funciones cardinales, que encuentra una expresión muy especial en el caso de grupos topolǵocis se describe en el corolario que sigue a estos dos resultados.

    \begin{lema}
        Sea $G$ un grupo topológico. Suponga que $D$ es un subconjunto denso en $G$ y que $U$ es una vecindad de la identidad en $e_G$; entonces, $G=DU$.
    \end{lema}

    \begin{proof}
        Es claro que $DU\subseteq G$. Probaremos que $G\subseteq DU$. Sea $g\in G$. Dado que $D$ es denso y $gU^{-1}$ es un conjunto abierto no vacío que contiene a $g$, existe un elemento $x\in D\cap gU^{-1}$. Por tanto, $g\in xU\subseteq DU$. Por tanto, $G\subseteq DU$.

        Por ambas contenciones, se sigue que $G=DU$.
    \end{proof}

    \begin{theor}
        Sea $G$ un grupo topológico y $\mathcal{B}$ una base local para $e_G$. Suponga que para cada $B\in\mathcal{B}$ existe $D_B\subseteq G$ tal que $G=D_BB$. Entonces el conjunto
        \begin{equation*}
            \left\{xB\Big|x\in D_b\textup{ con }B\in\mathcal{B}\right\}
        \end{equation*}
        es una base para $G$.
    \end{theor}

    \begin{proof}
        Sea $g\in G$ y $U$ un abierto en $G$ que contiene a $g$. Por un teorema existe una vecindad $V$ de la identidad tal que
        \begin{equation*}
            gV\subseteq U
        \end{equation*}
        Considere una vecindad de $e_G$, $W$ tal que
        \begin{equation*}
            W^{-1}W\subseteq V
        \end{equation*}
        (esta existe por un teorema), existe pues un elemento $B\in\mathcal{B}$ tal que $B\subseteq W$. Como $G=D_BB$, existe $x\in D_B$ tal que $g\in xB$. Por tanto,
        \begin{equation*}
            g\in xB\subseteq gB^{-1}B\subseteq gW^{-1}W\subseteq gV\subseteq U
            \Rightarrow g\in xB\subseteq U
        \end{equation*}
        Así, el conjunto anterior es base para la topología definida sobre $G$.
    \end{proof}

    \begin{cor}
        Si $G$ es un grupo topológico, entonces $w(G)=d(G)\cdot\chi(G)$.
    \end{cor}

    \begin{proof}
        Ya se sabe que
        \begin{equation*}
            \left\{
                \begin{array}{rl}
                    d(G)&\leq w(G)\\
                    \chi(G)&\leq w(G)\\
                \end{array}
            \right.
        \end{equation*}
        (la primera se da por definición y, la segunda por el hecho de que estmaos considerando grupos topológicos que son $T_0$, en particular, son Hausdorff, luego se tiene la desigualdad) por tanto $d(G)\cdot\chi(G)\leq w(G)$. Para la otra desigualdad, considere un conjunto $D$ denso en $G$ de cardinalidad $d(G)$. Sea $\mathcal{B}$ una base local para $e_G$ tal que
        \begin{equation*}
            \abs{\mathcal{B}}=\chi(G)
        \end{equation*}
        Por el lema anterior se sabe que $G=BD$, para todo $B\in\mathcal{B}$ y, del teorema anterior se deduce que la familia
        \begin{equation*}
            \mathcal{V}=\left\{xB\Big|x\in D\textup{ y }B\in\mathcal{B} \right\}
        \end{equation*}
        es una base para $G$, con cardinalidad no mayor a $d(G)\cdot\chi(G)$. Por ende, $w(G)\leq d(G)\cdot\chi(G)$.

        De ambas desigualdades se sigue que $w(G)=d(G)\cdot\chi(G)$.
    \end{proof}

    \begin{theor}
        Sea $G$ un grupo topológico. Entonces
        \begin{enumerate}
            \item $\pi\chi(G)=\chi(G)$.
            \item $\pi w(G)=w(G)$.
        \end{enumerate}      
    \end{theor}

    \begin{proof}
        Para evitar situaciones obvias, supondremos que $G$ no es discreto.
        
        De (1): Como toda base local es una $\pi$-base local, se sigue que
        \begin{equation*}
            \pi\chi(x,G)\leq\chi(x,G),\quad\forall x\in G
        \end{equation*}
        (pues estamos sacando mínimos) por tanto, sacando supremo en $G$ se sigue que
        \begin{equation*}
            \pi\chi(G)\leq\chi(G)
        \end{equation*}
        Para la otra desigualdad, basta con probar que si $\mathcal{B}$ es una $\pi$-base local en $e_G$, entonces el conjunto
        \begin{equation*}
            \left\{BB^{-1}\Big|B\in\mathcal{B} \right\}
        \end{equation*}
        es una base local en $e_G$. En efecto, sea $U$ una vecindad de $e_G$, entonces existe otra vecindad $V$ de $e_G$ tal que
        \begin{equation*}
            V\subseteq VV^{-1}\subseteq U
        \end{equation*}
        por ser $\mathcal{B}$ una $\pi$-base local, existe $B\in\mathcal{B}$ tal que
        \begin{equation*}
            B\subseteq V
        \end{equation*}
        luego, $e_G\in BB^{-1}\subseteq VV^{-1}\subseteq U$ siendo $BB^{-1}$ abierto. Por tanto, esta familia es una base local en $e_G$.

        Así, toda $\pi$-base local induce una base local de misma cardinalidad. Se sigue entonces que
        \begin{equation*}
            \chi(G)\leq\pi\chi(G)
        \end{equation*}

        De (2): De (1) y del corolario anterior se tiene que
        \begin{equation*}
            \begin{split}
                w(G)&\leq d(G)\cdot\chi(G)\\
                &\leq\pi w(G)\cdot \pi\chi(G)\\
                &\leq\pi w(G)\cdot\pi w(G)\\
                &=\pi w(G)\\
                \Rightarrow w(G)&\leq \pi w(G)\\
            \end{split}
        \end{equation*}
        y, como en todo espacio topológico se tiene que $w\pi(G)\leq w(G)$, se sigue entonces que
        \begin{equation*}
            w\pi(G)= w(G) 
        \end{equation*}
    \end{proof}

    Se presenta ahora una desigualdad cardinal válida en cualquier grupo topológico.
    
    \begin{propo}
        Sean $G$ un grupo topológico y $H$ un subgrupo normal cerrado de $G$. Entonces, $w(H)\leq w(G)$ y $w(G/H)\leq w(G)$.

        Además, $\chi(G/H)\leq\chi(G)$.
    \end{propo}

    \begin{proof}
        De la definición de peso es inmediato que
        \begin{equation*}
            w(H)\leq w(G/H)
        \end{equation*}
        para la otra desigualdad, note que si $U\subseteq G$ es abierto, entonces su correspondiente conjunto en $G/H$, $U^*=\pi(U)$, es abierto en $G/H$ (siendo $\cf{\pi}{G}{G/H}$) el mapeo cociente). Además, si $\mathcal{B}$ es una base para $G$, el conjunto
        \begin{equation*}
            \pi(\mathcal{B})=\left\{\pi(U)\Big|U\in\mathcal{B} \right\}
        \end{equation*}
        es una base en $G/H$. Luego, $w(G/H)\leq w(G)$. La desigualdad $\chi(G/H)\leq\chi(G)$ se prueba de forma análoga tomando una base local para la identidad.
    \end{proof}

    Ahora demostraremos un resultado un poco sorprendente, el cual es válido sólo en grupos topológicos.

    \begin{theor}
        Sea $G$ un grupo topológico tal que $d(G)<\abs{G}$. Entonces, $\Delta(G)=\abs{G}$.
    \end{theor}

    \begin{proof}
        Para el caso, recordemos que estamos trabajando en grupos infinitos. Sea $D'\subseteq G$ un conjunto denso en $G$ de cardinalidad $d(G)$. Considere
        \begin{equation*}
            D=\langle D'\rangle
        \end{equation*}
        $D$ es un subgrupo de $G$ tal que
        \begin{equation*}
            \begin{split}
                \abs{D}&=\abs{\langle D'\rangle}\\
                &=\aleph_0\cdot\abs{D'}\\
                &=\abs{D'}\\
                &=d(G)\\
            \end{split}
        \end{equation*}
        (pues al menos $D'$ es de cardinalidad $\aleph_0$). Como $d(G)<\abs{G}$, entonces el conjunto $G\backslash D\neq\emptyset$. Sea $g_1\in G\backslash D$ y, tomemos $D_0=D$.

        Definamos con ello $D_1=g_1D_0$. $D_1$ es un conjunto denso ajeno a $D_0$ en $G$ y de cardinalidad $d(G)$. Supongamos de esta forma que hemos construido conjuntos densos $D_\alpha$ para toda $\alpha<\beta$, con $\beta<\abs{G}$. Sea $g_\beta\in G\backslash\left(\bigcup_{ \alpha<\beta}D_\alpha \right)$. Podemos encontrar tal $g_\beta$, pues
        \begin{equation*}
            \begin{split}
                \abs{\bigcup_{\alpha<\beta}D_\alpha}&\leq\sum_{\alpha<\beta}\abs{D_\alpha}\\
                &\leq\beta\abs{D}\\
                &<\abs{G}\\
            \end{split}
        \end{equation*}
        siempre que $\beta<\abs{G}$. Continuando este proceso logramos $\abs{G}$ conjuntos densos ajenos entre sí. Cada denso debe intersectar a todo abierto no vacío y, por lo tanto, debe tener cardinalidad de $\abs{G}$; en consecuencia, $\Delta(G)=\abs{G}$.
    \end{proof}

    \section{Metrizabilidad}

    \begin{mydef}
        Sea $G$ un grupo. Una \textbf{pseudonorma en $G$} es una función $\cf{N}{G}{\mathbb{R}}$ que cumple las siguientes condiciones:
        \renewcommand{\theenumi}{\roman{enumi}}
        \begin{enumerate}
            \item $N(x)\geq0$, para todo $x\in G$.
            \item $N(e)=0$, con $e\in G$ la identidad del grupo.
            \item $N(x\cdot y^{-1})\leq N(x)+N(y)$, para todo $x,y\in G$.
        \end{enumerate}
        (iii) puede ser reformulado de la siguiente forma:
        \begin{equation*}
            N(x\cdot y)\leq N(x)+N(y)\quad\textup{y}\quad N(x)=N(x^{-1})
        \end{equation*}
        para todo $x,y\in G$.
    \end{mydef}

    \begin{propo}
        Sea $G$ un grupo y $N$ una pseudonorma. Entonces:
        \renewcommand{\theenumi}{\arabic{enumi}}
        \begin{enumerate}
            \item $r\cdot N$ es una pseudonorma en $G$ para todo $r\geq0$.
            \item Para cada $g\in G$, la función $N_g$ en $G$ dada por:
            \begin{equation*}
                N_g(x)=N(gxg^{-1}),\quad\forall x\in G
            \end{equation*}
            es una pseudonorma.
            \item La suma de dos pseudnormas en $G$ es una pseudonorma en $G$.
            \item Para cualquier función $\cf{f}{G}{\mathbb{R}}$ acotada, la función $\cf{N_f}{G}{\mathbb{R}}$ dada por:
            \begin{equation*}
                N_f(x)=\sup_{ y\in G}\abs{f(y\cdot x)-f(y)},\quad\forall x\in G
            \end{equation*}
            es una pseudonorma en $G$.
        \end{enumerate}
    \end{propo}

    \begin{proof}
        De (4): Sea $\cf{f}{G}{\mathbb{R}}$ una función acotada. Veamos que se cumplen las tres condiciones:
        \renewcommand{\theenumi}{\roman{enumi}}
        \begin{enumerate}
            \item Sea $x\in G$, entonces
            \begin{equation*}
                \begin{split}
                    N_f(x)&=\sup_{ y\in G}\abs{f(y\cdot x)-f(y)}\\
                    &\geq0\\
                \end{split}
            \end{equation*}
            \item Sea $e\in G$ la identidad del grupo. Entonces:
            \begin{equation*}
                \begin{split}
                    N_f(e)&=\sup_{ y\in G}\abs{f(y\cdot e)-f(y)}\\
                    &=\sup_{ y\in G}\abs{f(y)-f(y)}\\
                    &=\sup_{ y\in G}\abs{0}\\
                    &=0
                \end{split}
            \end{equation*}
            \item Sean $x_1,x_2\in G$. Se tiene que
            \begin{equation*}
                \begin{split}
                    N_f(x_1\cdot x_2^{-1})&=\sup_{ y\in G}\abs{f(y x_1 x_2^{-1})-f(y)}\\
                    &=\sup_{ y\in G}\abs{f(y x_1 x_2^{-1})-f(y)+f(x_1y)-f(x_1y)}\\
                    &=\sup_{ y\in G}\abs{[f(x_1y)-f(y)]+[f(y x_1 x_2^{-1})-f(x_1y)]}\\
                    &\leq\sup_{ y\in G}\abs{f(x_1y)-f(y)}+\sup_{ y\in G}\abs{f(y x_1 x_2^{-1})-f(x_1y)}\\
                    &\leq\sup_{ y\in G}\abs{f(x_1y)-f(y)}+\sup_{ z\in G}\abs{f(zx_2^{-1})-f(z)}\\
                    &= N(x_1)+N(x_2^{-1})\\
                \end{split}
            \end{equation*}
        \end{enumerate}
        Por (i)-(iii) se sigue que $N_f$ es una pseudonorma en $G$.
    \end{proof}

    Trabajando con grupos topológicos, uno está interesado en las pseudonormas continuas. Debido a que todo grupo topológico es un espacio homogéneo, se verifica la siguiente proposición de forma inmediata.

    \begin{propo}
        Sea $G$ un grupo topológico y $N$ una pseudonorma. Entonces $N$ es continua si y sólo si es continua en la identidad del grupo.
    \end{propo}



    \begin{theor}
        Sea $G$ un grupo topológico, $\left\{U_n\right\}_{ n=1}^\infty$ una sucesión decreciente de vecindades abiertas de $e$ tales que
        \begin{equation*}
            U_n^{-1}=U_n\quad\forall n\in\mathbb{N}
        \end{equation*}
        (es decir que son vecindades simétricas), y
        \begin{equation*}
            U_{ n+1}^2\subseteq U_n,\quad\forall n\in\mathbb{N}
        \end{equation*}
        Entonces, existe una pseudonorma continua $N$ en $G$ tal que
        \begin{equation*}
            \left\{x\in G\Big|N(x)<\frac{1}{2^n} \right\}\subseteq U_n\subseteq\left\{x\in G\Big|N(x)\leq\frac{1}{2^{n-1}} \right\}
        \end{equation*}
        para todo $n\in\mathbb{N}\cup\left\{0\right\}$.
    \end{theor}

    \begin{proof}
        Procederemos por inducción sobre $n$. Para $n=0$ haga $U(1)=U_0$ vecindad abierta simétrica de $e$.

        Suponga que para $n\in\mathbb{N}\cup\left\{0\right\}$ se han definido vecindades abiertas simétricas de $e$, $U(m/2^n)$, donde $m=1,2,...,2^n$. Definimos recursivamente
        \begin{equation*}
            U(1/2^{ n+1})=U_{ n+1}\quad\textup{y}\quad U((2m+1)/2^{ n+1})=U(m/2^n)\cdot U_{ n+1}
        \end{equation*}
        para todo $m=1,2,...,2^n-1$.
        %TODO
    \end{proof}

    \begin{cor}
        Sea $G$ grupo topológico. Para cada vecindad $U$ de $e\in G$ existe una pseudonorma continua $N$ en $G$ tal que $\left\{x\in G\Big|N(x)<1 \right\}\subseteq U$. 
    \end{cor}

    \begin{proof}
        Sin pérdida de generalidad, podemos suponer que $U$ es abierto. Sea
        \begin{equation*}
            U_0=U\cap U^{ -1} 
        \end{equation*}
        Por un teorema anterior existe $U_1$ vecindad abierta simétrica de $e$ tal que
        \begin{equation*}
            U_1^2\subseteq U_0
        \end{equation*}
        Suponga elegidos $U_1,...,U_n$ vecindades abiertas de la identidad tales que
        \begin{equation*}
            U_{ i+1}^2\subseteq U_i,\quad\forall i=0,1,...,n
        \end{equation*}
        Por un teorema anterior existe $U_{ n+1}$ vecindad abierta simétrica de $e$ tal que
        \begin{equation*}
            U_{ n+1}^2\subseteq U_n
        \end{equation*}
        por ende, se cumple que
        \begin{equation*}
            U_{ i+1}^2\subseteq U_i,\quad\forall i=0,1,...,n+1
        \end{equation*}
        Por inducción se construye una sucesión $\left\{U_n \right\}_{ i=n}^\infty$ de vecindades abiertas simétricas de la identidad tales que
        \begin{equation*}
            U_{ n+1}^2\subseteq U_n,\quad\forall n\in\mathbb{N}
        \end{equation*}
        Por el teorema anterior existe una pseudonorma continua $N$ en $G$ que satisface:
        \begin{equation*}
            \left\{x\in G\Big|N(x)<1/2^n \right\}\subseteq U_i\subseteq\left\{x\in G\Big| N(x)\leq 1/2^{ n-1} \right\},\quad\forall n\in\mathbb{N}\cup\left\{0\right\}
        \end{equation*}
        En particular, para $n=0$ se tiene que
        \begin{equation*}
            \left\{x\in G\Big| N(x)<1 \right\}\subseteq U_0=U
        \end{equation*}
        lo que prueba el resultado.
    \end{proof}

    \begin{theor}
        Un grupo topológico es metrizable si y sólo si posee una base numerable en la identidad.
    \end{theor}

    \begin{proof}
        
    \end{proof}

    \begin{cor}
        Un grupo topológico es metrizable si y sólo si es primero numerable.
    \end{cor}

    \begin{proof}
        Es inmediata del teorema anterior y del hecho que todo grupo topológico es un espacio homogéneo.
    \end{proof}


\end{document}