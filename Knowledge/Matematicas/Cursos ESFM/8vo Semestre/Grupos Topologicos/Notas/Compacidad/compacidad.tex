\documentclass[12pt]{report}
\usepackage[spanish]{babel}
\usepackage[utf8]{inputenc}
\usepackage{amsmath}
\usepackage{amssymb}
\usepackage{amsthm}
\usepackage{graphics}
\usepackage{subfigure}
\usepackage{lipsum}
\usepackage{array}
\usepackage{multicol}
\usepackage{enumerate}
\usepackage[framemethod=TikZ]{mdframed}
\usepackage[a4paper, margin = 1.5cm]{geometry}

%En esta parte se hacen redefiniciones de algunos comandos para que resulte agradable el verlos%

\renewcommand{\theenumii}{\roman{enumii}}

\def\proof{\paragraph{Demostración:\\}}
\def\endproof{\hfill$\blacksquare$}

\def\sol{\paragraph{Solución:\\}}
\def\endsol{\hfill$\square$}

%En esta parte se definen los comandos a usar dentro del documento para enlistar%

\newtheoremstyle{largebreak}
  {}% use the default space above
  {}% use the default space below
  {\normalfont}% body font
  {}% indent (0pt)
  {\bfseries}% header font
  {}% punctuation
  {\newline}% break after header
  {}% header spec

\theoremstyle{largebreak}

\newmdtheoremenv[
    leftmargin=0em,
    rightmargin=0em,
    innertopmargin=-2pt,
    innerbottommargin=8pt,
    hidealllines = true,
    roundcorner = 5pt,
    backgroundcolor = gray!60!red!30
]{exa}{Ejemplo}[section]

\newmdtheoremenv[
    leftmargin=0em,
    rightmargin=0em,
    innertopmargin=-2pt,
    innerbottommargin=8pt,
    hidealllines = true,
    roundcorner = 5pt,
    backgroundcolor = gray!50!blue!30
]{obs}{Observación}[section]

\newmdtheoremenv[
    leftmargin=0em,
    rightmargin=0em,
    innertopmargin=-2pt,
    innerbottommargin=8pt,
    rightline = false,
    leftline = false
]{theor}{Teorema}[section]

\newmdtheoremenv[
    leftmargin=0em,
    rightmargin=0em,
    innertopmargin=-2pt,
    innerbottommargin=8pt,
    rightline = false,
    leftline = false
]{propo}{Proposición}[section]

\newmdtheoremenv[
    leftmargin=0em,
    rightmargin=0em,
    innertopmargin=-2pt,
    innerbottommargin=8pt,
    rightline = false,
    leftline = false
]{cor}{Corolario}[section]

\newmdtheoremenv[
    leftmargin=0em,
    rightmargin=0em,
    innertopmargin=-2pt,
    innerbottommargin=8pt,
    rightline = false,
    leftline = false
]{lema}{Lema}[section]

\newmdtheoremenv[
    leftmargin=0em,
    rightmargin=0em,
    innertopmargin=-2pt,
    innerbottommargin=8pt,
    roundcorner=5pt,
    backgroundcolor = gray!30,
    hidealllines = true
]{mydef}{Definición}[section]

\newmdtheoremenv[
    leftmargin=0em,
    rightmargin=0em,
    innertopmargin=-2pt,
    innerbottommargin=8pt,
    roundcorner=5pt
]{excer}{Ejercicio}[section]

%En esta parte se colocan comandos que definen la forma en la que se van a escribir ciertas funciones%

\newcommand\abs[1]{\ensuremath{\big|#1\big|}}
\newcommand\divides{\ensuremath{\bigm|}}
\newcommand\cf[3]{\ensuremath{#1:#2\rightarrow#3}}
\newcommand{\natint}[1]{\left[\!\left[#1\right]\!\right]}
\newcommand{\contradiction}{\ensuremath{\#_{c}}}
\newcommand{\Cls}[1]{\ensuremath{\overline{#1}}}

%recuerda usar \clearpage para hacer un salto de página

\begin{document}
    \setlength{\parskip}{5pt} % Añade 5 puntos de espacio entre párrafos
    \setlength{\parindent}{12pt} % Pone la sangría como me gusta
    \title{Notas Grupos Topológicos}
    \author{Cristo Daniel Alvarado}
    \maketitle

    \tableofcontents %Con este comando se genera el índice general del libro%

    \setcounter{chapter}{1} %En esta parte lo que se hace es cambiar la enumeración del capítulo%
    
    \chapter{Compacidad}

    En este capítulo se estudiarán algunas de las propiedades más investigadas en grupos topológicos.

    \section{Grupos compactos y localmente compactos}

    Este primer teorema es una generalización de un teorema de la sección anterior.

    \begin{theor}
        Sean $G$ un grupo topológico, $U$ una vecindad de $e_G$ y $F$ un subconjunto compacto de $G$. Entonces, existe una vecindad $V$ de $e_G$ tal que
        \begin{equation*}
            xVx^{-1}\subseteq U\quad\forall x\in F
        \end{equation*}
    \end{theor}

    \begin{proof}
        Sea $W\subseteq G$ una vecindad simétrica de $e_G$, esto es que $W^3\subseteq U$. Como
        \begin{equation*}
            F\subseteq\bigcup_{ x\in F}Wx
        \end{equation*}
        al tenerse que $F$ es compacto, existen $x_1,..,x_k\in F$ tales que
        \begin{equation*}
            F\subseteq \bigcup_{ i=1}^k Wx_i
        \end{equation*}
        Sea $V=\bigcap_{ i=1}^kx_i^{-1}Wx_i$. Es claro que $V$ es una vecindad de $e_G$ y que $V\subseteq x_i^{-1}Wx_i$, para todo $i\in\natint{1,k}$. Si $x\in F$ entonces existe $i\in\natint{1,k}$ tal que
        \begin{equation*}
            x\in Wx_i
        \end{equation*}
        es decir $x=wx_i$ para algún $w\in W$. Luego,
        \begin{equation*}
            \begin{split}
                xVx^{-1}&=wx_iVx_i^{-1}w^{-1}\\
                &\subseteq wWw^{-1}\\
                &\subseteq W^3\\
                &\subseteq U\\
            \end{split}
        \end{equation*}
    \end{proof}

    Se sabe de la sección anterior que si $G$ es un grupo topológico y $H$ es un subgrupo cerrado de $G$, entonces la función canónica $\cf{\pi}{G}{G/H}$ es continua y abierta. En el caso en que $H$ sea compacto, este resultado puede mejorarse:

    \begin{theor}
        Sea $G$ un grupo topológico y $H$ un subgrupo compacto de $G$. Entonces, la función canónica $\cf{\pi}{G}{G/H}$ es una función cerrada.
    \end{theor}

    \begin{proof}
        Sea $A\subseteq G$ cerrado. Para probar el resultado bastará con probar que el complemento de $\pi(A)$ es abierto en $G/H$. Sea $x\in G$ tal que $\pi(x)\notin\pi(A)$. Notemos que
        \begin{equation*}
            \pi(A)=AH
        \end{equation*}
        donde el conjunto $AH\subseteq G$ es cerrado en $G$ (ya que $A$ es cerrado y $H$ es compacto), luego como $x\notin AH$ existe un abierto $U\subseteq G$ tal que $x\in U$ y
        \begin{equation*}
            U\cap AH=\emptyset
        \end{equation*}
        Afirmamos que $U^*=\pi(U)$ es un abierto que contiene a $\pi(x)$ ajeno a $\pi(A)$. En efecto, es claro que contiene a $\pi(x)$. Suponga que existe $z\in G$ tal que $\pi(z)=zH\in \pi(U)\cap\pi(A)$, luego existen $u\in U$ y $a\in A$ tales que:
        \begin{equation*}
            zH=uH=aH
        \end{equation*}
        luego $a^{-1}u\in H$ de donde se sigue que $u\in AH$, es decir que $U\cap AH\neq\emptyset$\contradiction. Por tanto, $\pi(U)\cap \pi(A)=\emptyset$. Así, el conjunto $G/H\backslash\pi(A)$ es abierto, luego $\pi(A)$ es cerrado.
    \end{proof}

    \begin{obs}
        Note que la condición de que $H$ sea compacto es necesaria para que el conjunto $AH$ sea cerrado.
    \end{obs}

    Las propiedades de compacidad y compacidad local se heredan a espacios cocientes de $G$ entre subgrupos cerrados $H$.

    \begin{theor}
        Sean $G$ un grupo topológico y $H$ un subgrupo cerrado de $G$. Si $G$ es compacto, entonces $H$ y $G/H$ también lo son. Si $G$ es localmente compacto, entonces $H$ y $G/H$ también lo son.
    \end{theor}

    \begin{proof}
        Es claro que la compacidad y la compacidad local se hereda a subgrupos cerrados de $G$.

        Dado que la función canónica $\cf{\pi}{G}{G/H}$ es continua y sobre, la imagen $\pi(G)=G/H$ es un conjunto compacto en $G/H$, es decir que $G/H$ es compacto.

        Ahora probaremos que si $G$ es localmente compacto, entonces $G/H$ también lo es. En efecto, suponga que $G$ es localmente compacto. Sea $\pi(a)\in G/H$, debemos encontrar una vecindad compacta de $\pi(a)$. Como $a\in G$ existe una vecindad $U$ de $a$ tal que $a\in U\subseteq\Cls{U}$, siendo $\Cls{U}$ compacto en $G$.

        Como $\pi$ es continua, se tiene que $\pi(\Cls{U})$ es compacto en $G/H$, en particular, cerrado. Además, como $U\subseteq\Cls{U}$, entonces
        \begin{equation*}
            \pi(U)\subseteq\pi(\Cls{U})\Rightarrow\Cls{\pi(U)}\subseteq\pi(\Cls{U})
        \end{equation*}
        es decir, que $\pi(U)$ es una vecindad de $\pi(a)$ tal que $\Cls{\pi(U)}$ es compacta (por ser un cerrado contenido en un compacto). Luego, $G/H$ es localmente compacto.
    \end{proof}

    Hemos usado el hecho de que si $\cf{\pi}{G}{G/H}$ es la función canónica y $K\subseteq G$ es compacto, entonces $\pi(K)$ es compacto en $G/H$. El recíproco de este resultado también es cierto con una hipótesis adicional: la imagen inversa de un compacto en el espacio cociente $G/H$ es compacta si $H$ es compacto. Antes de probar este resultado necesitamos de algunos hechos auxiliares.

    \begin{mydef}
        Sean $X$ y $Y$ espacios topológicos y $\cf{f}{X}{Y}$ una función continua. Decimos que $f$ es \textbf{perfecta} si $f$ es cerrada y todas las fibras $f^{-1}(y)\subseteq X$ son compactas para todo $y\in Y$.
    \end{mydef}

    El teorema 2.1.2 se puede generalizar al afirmar que la función canónica $\cf{\pi}{G}{G/H}$ es perfecta si el subgrupo $H$ de $G$ es compacto.

    Se harán a continuación la prueba de algunos resultados necesarios para un teorema posterior.

    \begin{propo}
        Si $\cf{f}{X}{Y}$ es una función cerrada, entonces para cualquier subespacio $L\subseteq Y$ la reestricción $\cf{f_L=f\big|_{ f^{-1}(L)}}{f^{-1}(L)}{L}$ es cerrada.
    \end{propo}

    \begin{proof}
        Sea $A\subseteq X$ un conjunto cerrado en $X$, entonces
        \begin{equation*}
            \begin{split}
                f_L(A\cap f^{-1}(L))&=f(A\cap f^{-1}(L))\\
                &=f(A)\cap L\\
            \end{split}
        \end{equation*}
        pues, $f(f^{-1}(L))=L$. Por ende, como $f$ es cerrada se sigue que $f(A)\cap L$ es cerrado en el subespacio de $L$ de $Y$. Así, $f_L$ es cerrada.
    \end{proof}

    \begin{propo}
        Sean $X,Y$ espacios topológicos y $\cf{f}{X}{Y}$ una función perfecta, entonces para cualquier cerrado $A\subseteq X$ y cualquier subespacio $B\subseteq Y$ las restricciones $\cf{f\big|_A}{A}{Y}$ y $\cf{f_B=f\big|_{ f^{-1}(B)}}{f^{-1}(B)}{B}$ son perfectas.
    \end{propo}

    \begin{proof}
        
    \end{proof}

    \begin{theor}
        Sean $X,Y$ espacios topológicos. Si $\cf{f}{X}{Y}$ es una función perfecta, entonces para todo subconjunto compacto $Z\subseteq Y$, su imagen inversa $f^{-1}(Z)$ es compacta en $X$. En particular, si $Y$ es compacto, $X$ también lo es.
    \end{theor}

    \begin{proof}
        
    \end{proof}

    Suponga que tenemos un grupo $G$, un subgrupo $H$ de $G$ y una propiedad topológica $\mathcal{P}$. Un problema muy conocido en la teoría de grupos topológicos es: si $G/H$ y $H$ tienen la propiedad $\mathcal{P}$, ¿también $G$ posee la propiedad $\mathcal{P}$? En este libro se responderá afirmativamente esta pregunta para varias propiedades $\mathcal{P}$, por ejemplo, conexidad, compacidad, etc... Comencemos con la compacidad.

    \begin{theor}
        
    \end{theor}

    \begin{proof}
        
    \end{proof}

\end{document}