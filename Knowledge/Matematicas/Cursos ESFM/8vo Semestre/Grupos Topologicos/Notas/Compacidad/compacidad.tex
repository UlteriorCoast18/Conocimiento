\documentclass[12pt]{report}
\usepackage[spanish]{babel}
\usepackage[utf8]{inputenc}
\usepackage{amsmath}
\usepackage{amssymb}
\usepackage{amsthm}
\usepackage{graphics}
\usepackage{subfigure}
\usepackage{lipsum}
\usepackage{array}
\usepackage{multicol}
\usepackage{enumerate}
\usepackage[framemethod=TikZ]{mdframed}
\usepackage[a4paper, margin = 1.5cm]{geometry}

%En esta parte se hacen redefiniciones de algunos comandos para que resulte agradable el verlos%

\renewcommand{\theenumii}{\roman{enumii}}

\def\proof{\paragraph{Demostración:\\}}
\def\endproof{\hfill$\blacksquare$}

\def\sol{\paragraph{Solución:\\}}
\def\endsol{\hfill$\square$}

%En esta parte se definen los comandos a usar dentro del documento para enlistar%

\newtheoremstyle{largebreak}
  {}% use the default space above
  {}% use the default space below
  {\normalfont}% body font
  {}% indent (0pt)
  {\bfseries}% header font
  {}% punctuation
  {\newline}% break after header
  {}% header spec

\theoremstyle{largebreak}

\newmdtheoremenv[
    leftmargin=0em,
    rightmargin=0em,
    innertopmargin=-2pt,
    innerbottommargin=8pt,
    hidealllines = true,
    roundcorner = 5pt,
    backgroundcolor = gray!60!red!30
]{exa}{Ejemplo}[section]

\newmdtheoremenv[
    leftmargin=0em,
    rightmargin=0em,
    innertopmargin=-2pt,
    innerbottommargin=8pt,
    hidealllines = true,
    roundcorner = 5pt,
    backgroundcolor = gray!50!blue!30
]{obs}{Observación}[section]

\newmdtheoremenv[
    leftmargin=0em,
    rightmargin=0em,
    innertopmargin=-2pt,
    innerbottommargin=8pt,
    rightline = false,
    leftline = false
]{theor}{Teorema}[section]

\newmdtheoremenv[
    leftmargin=0em,
    rightmargin=0em,
    innertopmargin=-2pt,
    innerbottommargin=8pt,
    rightline = false,
    leftline = false
]{propo}{Proposición}[section]

\newmdtheoremenv[
    leftmargin=0em,
    rightmargin=0em,
    innertopmargin=-2pt,
    innerbottommargin=8pt,
    rightline = false,
    leftline = false
]{cor}{Corolario}[section]

\newmdtheoremenv[
    leftmargin=0em,
    rightmargin=0em,
    innertopmargin=-2pt,
    innerbottommargin=8pt,
    rightline = false,
    leftline = false
]{lema}{Lema}[section]

\newmdtheoremenv[
    leftmargin=0em,
    rightmargin=0em,
    innertopmargin=-2pt,
    innerbottommargin=8pt,
    roundcorner=5pt,
    backgroundcolor = gray!30,
    hidealllines = true
]{mydef}{Definición}[section]

\newmdtheoremenv[
    leftmargin=0em,
    rightmargin=0em,
    innertopmargin=-2pt,
    innerbottommargin=8pt,
    roundcorner=5pt
]{excer}{Ejercicio}[section]

%En esta parte se colocan comandos que definen la forma en la que se van a escribir ciertas funciones%

\newcommand\abs[1]{\ensuremath{\left|#1\right|}}
\newcommand\divides{\ensuremath{\bigm|}}
\newcommand\cf[3]{\ensuremath{#1:#2\rightarrow#3}}
\newcommand{\natint}[1]{\left[\!\left[#1\right]\!\right]}
\newcommand{\contradiction}{\ensuremath{\#_{c}}}
\newcommand{\Cls}[1]{\ensuremath{\overline{#1}}}
\newcommand{\supp}[1]{\ensuremath{\textup{supp}\left(#1\right)}}

%recuerda usar \clearpage para hacer un salto de página

\begin{document}
    \setlength{\parskip}{5pt} % Añade 5 puntos de espacio entre párrafos
    \setlength{\parindent}{12pt} % Pone la sangría como me gusta
    \title{Notas Grupos Topológicos}
    \author{Cristo Daniel Alvarado}
    \maketitle

    \tableofcontents %Con este comando se genera el índice general del libro%

    \setcounter{chapter}{1} %En esta parte lo que se hace es cambiar la enumeración del capítulo%
    
    \chapter{Compacidad}

    En este capítulo se estudiarán algunas de las propiedades más investigadas en grupos topológicos.

    \section{Grupos compactos y localmente compactos}

    Este primer teorema es una generalización de un teorema de la sección anterior.

    \begin{theor}
        Sean $G$ un grupo topológico, $U$ una vecindad de $e_G$ y $F$ un subconjunto compacto de $G$. Entonces, existe una vecindad $V$ de $e_G$ tal que
        \begin{equation*}
            xVx^{-1}\subseteq U\quad\forall x\in F
        \end{equation*}
    \end{theor}

    \begin{proof}
        Sea $W\subseteq G$ una vecindad simétrica de $e_G$, esto es que $W^3\subseteq U$. Como
        \begin{equation*}
            F\subseteq\bigcup_{ x\in F}Wx
        \end{equation*}
        al tenerse que $F$ es compacto, existen $x_1,..,x_k\in F$ tales que
        \begin{equation*}
            F\subseteq \bigcup_{ i=1}^k Wx_i
        \end{equation*}
        Sea $V=\bigcap_{ i=1}^kx_i^{-1}Wx_i$. Es claro que $V$ es una vecindad de $e_G$ y que $V\subseteq x_i^{-1}Wx_i$, para todo $i\in\natint{1,k}$. Si $x\in F$ entonces existe $i\in\natint{1,k}$ tal que
        \begin{equation*}
            x\in Wx_i
        \end{equation*}
        es decir $x=wx_i$ para algún $w\in W$. Luego,
        \begin{equation*}
            \begin{split}
                xVx^{-1}&=wx_iVx_i^{-1}w^{-1}\\
                &\subseteq wWw^{-1}\\
                &\subseteq W^3\\
                &\subseteq U\\
            \end{split}
        \end{equation*}
    \end{proof}

    Se sabe de la sección anterior que si $G$ es un grupo topológico y $H$ es un subgrupo cerrado de $G$, entonces la función canónica $\cf{\pi}{G}{G/H}$ es continua y abierta. En el caso en que $H$ sea compacto, este resultado puede mejorarse:

    \begin{theor}
        Sea $G$ un grupo topológico y $H$ un subgrupo compacto de $G$. Entonces, la función canónica $\cf{\pi}{G}{G/H}$ es una función cerrada.
    \end{theor}

    \begin{proof}
        Sea $A\subseteq G$ cerrado. Para probar el resultado bastará con probar que el complemento de $\pi(A)$ es abierto en $G/H$. Sea $x\in G$ tal que $\pi(x)\notin\pi(A)$. Notemos que
        \begin{equation*}
            \pi(A)=AH
        \end{equation*}
        donde el conjunto $AH\subseteq G$ es cerrado en $G$ (ya que $A$ es cerrado y $H$ es compacto), luego como $x\notin AH$ existe un abierto $U\subseteq G$ tal que $x\in U$ y
        \begin{equation*}
            U\cap AH=\emptyset
        \end{equation*}
        Afirmamos que $U^*=\pi(U)$ es un abierto que contiene a $\pi(x)$ ajeno a $\pi(A)$. En efecto, es claro que contiene a $\pi(x)$. Suponga que existe $z\in G$ tal que $\pi(z)=zH\in \pi(U)\cap\pi(A)$, luego existen $u\in U$ y $a\in A$ tales que:
        \begin{equation*}
            zH=uH=aH
        \end{equation*}
        luego $a^{-1}u\in H$ de donde se sigue que $u\in AH$, es decir que $U\cap AH\neq\emptyset$\contradiction. Por tanto, $\pi(U)\cap \pi(A)=\emptyset$. Así, el conjunto $G/H\backslash\pi(A)$ es abierto, luego $\pi(A)$ es cerrado.
    \end{proof}

    \begin{obs}
        Note que la condición de que $H$ sea compacto es necesaria para que el conjunto $AH$ sea cerrado.
    \end{obs}

    Las propiedades de compacidad y compacidad local se heredan a espacios cocientes de $G$ entre subgrupos cerrados $H$.

    \begin{theor}
        Sean $G$ un grupo topológico y $H$ un subgrupo cerrado de $G$. Si $G$ es compacto, entonces $H$ y $G/H$ también lo son. Si $G$ es localmente compacto, entonces $H$ y $G/H$ también lo son.
    \end{theor}

    \begin{proof}
        Es claro que la compacidad y la compacidad local se hereda a subgrupos cerrados de $G$.

        Dado que la función canónica $\cf{\pi}{G}{G/H}$ es continua y sobre, la imagen $\pi(G)=G/H$ es un conjunto compacto en $G/H$, es decir que $G/H$ es compacto.

        Ahora probaremos que si $G$ es localmente compacto, entonces $G/H$ también lo es. En efecto, suponga que $G$ es localmente compacto. Sea $\pi(a)\in G/H$, debemos encontrar una vecindad compacta de $\pi(a)$. Como $a\in G$ existe una vecindad $U$ de $a$ tal que $a\in U\subseteq\Cls{U}$, siendo $\Cls{U}$ compacto en $G$.

        Como $\pi$ es continua, se tiene que $\pi(\Cls{U})$ es compacto en $G/H$, en particular, cerrado. Además, como $U\subseteq\Cls{U}$, entonces
        \begin{equation*}
            \pi(U)\subseteq\pi(\Cls{U})\Rightarrow\Cls{\pi(U)}\subseteq\pi(\Cls{U})
        \end{equation*}
        es decir, que $\pi(U)$ es una vecindad de $\pi(a)$ tal que $\Cls{\pi(U)}$ es compacta (por ser un cerrado contenido en un compacto). Luego, $G/H$ es localmente compacto.
    \end{proof}

    Hemos usado el hecho de que si $\cf{\pi}{G}{G/H}$ es la función canónica y $K\subseteq G$ es compacto, entonces $\pi(K)$ es compacto en $G/H$. El recíproco de este resultado también es cierto con una hipótesis adicional: la imagen inversa de un compacto en el espacio cociente $G/H$ es compacta si $H$ es compacto. Antes de probar este resultado necesitamos de algunos hechos auxiliares.

    \begin{mydef}
        Sean $X$ y $Y$ espacios topológicos y $\cf{f}{X}{Y}$ una función continua. Decimos que $f$ es \textbf{perfecta} si $f$ es cerrada y todas las fibras $f^{-1}(y)\subseteq X$ son compactas para todo $y\in Y$.
    \end{mydef}

    El teorema 2.1.2 se puede generalizar al afirmar que la función canónica $\cf{\pi}{G}{G/H}$ es perfecta si el subgrupo $H$ de $G$ es compacto.

    Se harán a continuación la prueba de algunos resultados necesarios para un teorema posterior.

    \begin{propo}
        Si $\cf{f}{X}{Y}$ es una función cerrada, entonces para cualquier subespacio $L\subseteq Y$ la reestricción $\cf{f_L=f\big|_{ f^{-1}(L)}}{f^{-1}(L)}{L}$ es cerrada.
    \end{propo}

    \begin{proof}
        Sea $A\subseteq X$ un conjunto cerrado en $X$, entonces
        \begin{equation*}
            \begin{split}
                f_L(A\cap f^{-1}(L))&=f(A\cap f^{-1}(L))\\
                &=f(A)\cap L\\
            \end{split}
        \end{equation*}
        pues, $f(f^{-1}(L))=L$. Por ende, como $f$ es cerrada se sigue que $f(A)\cap L$ es cerrado en el subespacio de $L$ de $Y$. Así, $f_L$ es cerrada.
    \end{proof}

    \begin{propo}
        Sean $X,Y$ espacios topológicos y $\cf{f}{X}{Y}$ una función perfecta, entonces para cualquier cerrado $A\subseteq X$ y cualquier subespacio $B\subseteq Y$ las restricciones $\cf{f\big|_A}{A}{Y}$ y $\cf{f_B=f\big|_{ f^{-1}(B)}}{f^{-1}(B)}{B}$ son perfectas.
    \end{propo}

    \begin{proof}
        Es claro que la reestricción $f\big|_A$ y $f_B$ son cerradas (siendo la última por la proposición anterior). Sea ahora $y\in Y$, se tiene:
        \begin{equation*}
            f\big|_{A}^{-1}(y)=A\cap f^{-1}(y)
        \end{equation*}
        donde el conjunto de la derecha es un cerrado contenido en el compacto $f^{-1}(y)$, luego compacto. Así, $f\big|_A$ es perfecta.

        Para $f_B$, veamos que si $y\in B$:
        \begin{equation*}
            \begin{split}
                f_B^{-1}(y)=f^{-1}(y)
            \end{split}
        \end{equation*}
        donde el conjunto de la derecha es compacto. Por tanto, $f_B$ es perfecta.
    \end{proof}

    \begin{theor}
        Sean $X,Y$ espacios topológicos. Si $\cf{f}{X}{Y}$ es una función perfecta, entonces para todo subconjunto compacto $Z\subseteq Y$, su imagen inversa $f^{-1}(Z)$ es compacta en $X$. En particular, si $Y$ es compacto, $X$ también lo es.
    \end{theor}

    \begin{proof}
        Primero notemos que si $y\in Y$ y $U\subseteq X$ es una vecindad abierta de $f^{-1}(y)$, entonces existe una vecindad $W\subseteq Y$ de $y$ tal que
        \begin{equation*}
            f^{-1}(W)\subseteq U
        \end{equation*}
        En efecto, tomemos $W=Y\backslash f(X\backslash U)$. Claramente $W$ es cerrada pues $f$ es perfecta (en particular, $f$ es cerrada) y es tal que $y\in W$. Además,
        \begin{equation*}
            \begin{split}
                x\in f^{-1}(Y\backslash f(X\backslash U))&\iff f(x)\in Y\backslash f(X\backslash U)\\
                &\iff f(x)\in Y\backslash f(X\backslash U)\\
                &\iff f(x)\in Y\textup{ y }f(x)\notin f(X\backslash U)\\
                &\Rightarrow x\notin X\backslash U\\
                &\iff x\in U\\
            \end{split}
        \end{equation*}
        por tanto ,$f^{-1}(W)\subseteq W$.

        Ahora, por el teorema 2.1.3 es suficiente con probar que si $Y$ es compacto, entonces $X$ también lo es (esto pues podemos tomar la reestricción de $f$ a $U$ y sería una función $\cf{f\big|_U}{U}{Y}$ y el resultado se cumpliría para $U$). Sea $\mathcal{U}$ una cubierta abierta de $X$.
        
        Sin péridida de generalidad podemos suponer que $\mathcal{U}$ es cerrada bajo uniones finitas (en caso de que no lo sea, podemos crear una cubierta más grande formada por todos los elementos de $\mathcal{U}$, al extraer la subcubierta abierta finita de esta cubierta más grande tendríamos a su vez una cubierta abierta finita del conjunto formada por una cantidad finita de elementos de $\mathcal{U}$). Sabemos que para toda $y\in Y$, el conjunto $f^{-1}(y)$ es compacto en $X$. Por ende, existe un abierto $U_y\in\mathcal{U}$ tal que
        \begin{equation*}
            f^{-1}(y)\subseteq U_y
        \end{equation*}
        por lo probado anteriormente se tiene que existe un abierto $V_y\subseteq Y$ tal que $y\in V_y$ y:
        \begin{equation*}
            f^{-1}(V_y)\subseteq U_y
        \end{equation*}
        como $Y$ es compacto y la familia $\left\{V_y\Big|y\in Y \right\}$ forma una cubierta abierta de $Y$, entonces existen $y_1,...,y_n\in Y$ tales que
        \begin{equation*}
            Y\subseteq \bigcup_{ i=1}^n V_{ y_i} 
        \end{equation*}
        Se sigue entonces que
        \begin{equation*}
            \begin{split}
                X&=f^{-1}(Y)\\
                &\subseteq f^{-1}\left(\bigcup_{ i=1}^n V_{ y_i} \right)\\
                &\subseteq\bigcup_{ i=1}^n f^{-1}(V_{ y_i})\\
                &\subseteq\bigcup_{ i=1}^n U_{ y_i}
            \end{split}
        \end{equation*}
        luego, $X$ es compacto.
    \end{proof}

    Suponga que tenemos un grupo $G$, un subgrupo $H$ de $G$ y una propiedad topológica $\mathcal{P}$. Un problema muy conocido en la teoría de grupos topológicos es: si $G/H$ y $H$ tienen la propiedad $\mathcal{P}$, ¿también $G$ posee la propiedad $\mathcal{P}$? En este libro se responderá afirmativamente esta pregunta para varias propiedades $\mathcal{P}$, por ejemplo, conexidad, compacidad, etc... Comencemos con la compacidad.

    \begin{theor}
        Sean $G$ un grupo topológico y $H$ un subgrupo compacto de $G$. Si $Q\subseteq G/H$ es compacto, entonces $P=\pi^{-1}(Q)$ es compacto, donde $\cf{\pi}{G}{G/H}$ es la función canónica. En particular, si $G/H$ es compacto, entonces $G$ también lo es.
    \end{theor}

    \begin{proof}
        Por el teorema anterior, basta con probar que $\pi$ es perfecta. De un teorema anterior se tiene que $\pi$ es cerrada.

        Sea $x\in G$, entonces el conjunto $\pi^{-1}(\pi(x))=xH$ que es compacto. Así, $\pi$ es perfecta.
    \end{proof}

    \begin{cor}
        Sea $G$ grupo topológico y $H$ un subgrupo cerrado de $G$. Entonces, $\cf{\pi}{G}{G/H}$ es perfecta si y sólo si $H$ es compacto.
    \end{cor}

    \begin{proof}
        Es inmediata de la prueba del teorema anterior, pues:
        \begin{equation*}
            \pi^{-1}(\pi(x))=xH
        \end{equation*}
    \end{proof}

    Existen grupos numerables sin puntos aislados, por ejemplo, el grupo $\mathbb{Z}$ con la $p$-topología. Demostraremos que tales grupos no pueden ser localmente compactos.

    \begin{theor}
        Todo grupo topológico localmente compacto $G$ tal que $\abs{G}<\mathfrak{c}$ es discreto.
    \end{theor}

    \begin{proof}
        Suponga que $G$ es localmente compacto y no es discreto. Entonces $G$ no puede tener puntos aislados (pues es homogéneo). Sea $U$ una vecindad abierta de $e_G$ tal que $K=\Cls{U}$ es compacto.
    \end{proof}

    Como todo espacio compacto es localmente compacto, se sigue que todo grupo topológico compacto no discreto tiene cardinalidad no menor que $\mathfrak{c}$ (por el teorema anterio). Lo mismo se cumple para los grupos numerablemente compactos (como se verá en un ejercicio posterior).

    El siguiente resultado acerca de subconjuntos compactos de un grupo nos será de utliidad posteriormente.

    \begin{theor}
        Sean $G$ un grupo topológico, $F$ un subconjunto compacto de $G$, $U$ un subconjunto abierto de $G$ tal que $F\subseteq U$. Entonces existe una vecindad $V$ de $e_G$ tal que
        \begin{equation*}
            (FV)\cup(VF)\subseteq U
        \end{equation*}
        Si $G$ es localmente compacto, entonces $V$ se puede elegir de tal forma que ambos conjuntos $\Cls{FV}$ y $\Cls{VF}$ sean compactos.
    \end{theor}

    \begin{proof}
        Como $W$ es abierto, entonces para cada $x\in F$ existe $W_x\subseteq G$ vecindad abierta de $e_G$ que
        \begin{equation*}
            xW_x\subseteq U
        \end{equation*}
        además, existe una vecindad $V_x$ de $e_G$ tal que
        \begin{equation*}
            V_x^2\subseteq W_x
        \end{equation*}
        Notemos que $\left\{xV_x \right\}_{x\in F}$ es una cubierta abierta de $F$ luego, $F$ por ser compacto existen $x_1,...,x_n\in F$ tales que
        \begin{equation*}
            F\subseteq \bigcup_{ i=1}^n x_iV_{ x_i}
        \end{equation*}
        Tomemos:
        \begin{equation*}
            V_1=\bigcap_{ i=1}^n V_{ x_i}
        \end{equation*}
        la cual es una vecindad abierta de $e_G$. Se tiene que:
        \begin{equation*}
            \begin{split}
                FV_1&\subseteq \left( \bigcup_{ i=1}^n x_iV_{ x_i}\right)V_1\\
                &\subseteq\bigcup_{ i=1}^n x_iV_{ x_i}V_{ x_i}\\
                &=\bigcup_{ i=1}^n x_iV_{ x_i}^2\\
                &=\bigcup_{ i=1}^n x_iW_{ x_i}\\
                &\subseteq U\\
                \Rightarrow FV_1&\subseteq U\\
            \end{split}
        \end{equation*}
        de forma similar obtenemos una vecindad abierta $V_2$ de $e_G$ tal que $V_2F\subseteq U$. Sea $V=V_1\cap V_2$, se obtiene con ello que
        \begin{equation*}
            (FV)\cup(VF)\subseteq U
        \end{equation*}
        Supongamos ahora que $G$ es localmente compacto. Entonces, en lo hecho anteriormente podemos elegir a $V$ tal que $\Cls{V}$ es un conjunto compacto en $G$. Luego, por un teorema se sigue que el conjunto $F\Cls{V}$ es compacto.
        
        Como $FV\subseteq F\Cls{V}$ y $F\Cls{V}$ es cerrado, entonces $\Cls{FV}\subseteq F\Cls{V}$, luego $\Cls{FV}$ es compacto. De forma análoga se deduce que $\Cls{VF}$ es compacto.
    \end{proof}

    Ahora se analizarán algunos resultados sobre compacidad numerable.

    \begin{obs}
        Recordemos antes que todo espacio es compacto si y sólo si es numerablemente compacto y Lindelöf.
    \end{obs}

    \begin{theor}
        Si $G$ es un grupo topológico que contiene un subgrupo compacto $H$ tal que el espacio cociente $G/H$ es numerablemente compacto, entonces $G$ es numerablemente compacto.
    \end{theor}

    \begin{proof}
        Por un resultado del apéndice, basta con probar que cada subconjunto infinito de $G$ tiene un punto de acumulación (en el caso en que $\abs{G}$ sea infinito, si es finito el resultado es inmediato). Sea $X\subseteq G$ tal que
        \begin{equation*}
            \abs{X}=\aleph_0
        \end{equation*}
        (basta con este cardinal ya que para cardinales más grande siempre se puede extraer un subconjunto de cardinalidad al menos $\aleph_0$). Se tienen dos casos:
        \begin{itemize}
            \item Existe $g\in G$ tal que $\abs{X\cap gH}=\aleph_0$. En este caso el subconjunto $X\cap gH$ del subespacio $gH$ de $G$ debe tener un punto de acumulación (pues, el subespacio $gH$ es compacto, en particular numerablemente compacto), luego $X$ tiene un punto de acumulación en $G$.
            \item Para toda $x\in G$, $\abs{X\cap gH}<\aleph_0$. Sea $\cf{\pi}{G}{G/H}$ la función canónica. De esta suposición se desprende que
            \begin{equation*}
                \abs{\pi(X)}=\aleph_0
            \end{equation*}
            (ya que en caso contrario, una clase lateral debería contener una cantidad numerable de elementos de $X$, cosa que no puede suceder), luego al ser $G/H$ numerablemente compacto se sigue que $\pi(X)$ tiene un punto de acumulación en $G/H$, es decir que existe $a\in G$ tal que para toda $U$ vecindad de $a$ en $G$ se cumple que
            \begin{equation*}
                \abs{\pi(X)\cap U^*}=\aleph_0
            \end{equation*}
            (donde $U^*=\pi(U)$). Probaremos que para alguna $p\in aH$, $p$ es punto de acumulación de $X$. Suponga lo contrario, es decir que para todo $p\in aH$ existe una vecindad abierta $V_p$ en $G$ de $p$ tal que
            \begin{equation*}
                \abs{X\cap V_p}<\aleph_0
            \end{equation*}
            Se tiene entonces que el conjunto $\left\{V_p \right\}_{ p\in aH}$ forma una cubierta abierta de $aH$. Por ser compacto, existen $p_1,...,p_n\in aH$ tales que
            \begin{equation*}
                aH\subseteq \bigcup_{ i=1}^n V_{ p_i}=V
            \end{equation*}
            se tiene que $V$ es una vecindad abierta de $aH$ para la cual
            \begin{equation*}
                \abs{X\cap V}\leq\sum_{ i=1}^n\abs{X\cap V_{ p_i}}<\aleph_0
            \end{equation*}
            por ser $\pi$ perfecta (por ser $H$ compacto), existe una vecindad abierta $U^*$ de $aH$ en $G/H$ tal que
            \begin{equation*}
                \pi^{-1}(U^*)\subseteq V
                %TODO PORQUE?
            \end{equation*}
            luego,
            \begin{equation*}
                \abs{U^*\cap \pi(X)}\leq \abs{\pi^{-1}(U^*)\cap X}<\aleph_0
            \end{equation*}
            lo cual es una contradicción. Por tanto, $X$ tiene un punto de acumulación $p$ tal que $p\in aH$.
        \end{itemize}
    \end{proof}

    En este teorema anteiror no se puede debilitar la hipótesis de que $H$ a numerablemente compacto sin perder el resultado, pues existen grupos topológicos numerablemente compactos $H$ y $K$ tales que $G=H\times K$ no es numerablemente compacto. Este tipo de construcciones usan axiomas adicionales de ZFE, por ejemplo, la hipótesis del continuo o el axioma de Martin.

    ¿Existen grupos topológicos numerablemente compactos que no sean compactos? La respuesta es que si, en el que se usa el $\Sigma$-producto.

    \begin{mydef}
        Sea $\left\{X_i\right\}_{ i\in I}$ un conjunto de espacios topológicos. En el producto $X=\prod_{ i\in I}X_i$ considere un punto $p\in X$, es decir $p=\left(p_i \right)_{ i\in I}$. Definimos el \textbf{soporte de un punto $x\in X$ respecto a $p$} como:
        \begin{equation*}
            \supp{x}=\left\{i\in I\Big|x_i\neq p_i \right\}
        \end{equation*}
        El \textbf{$\Sigma$-producto} denotado por $\Sigma(p)$ en $X$ se define como:
        \begin{equation*}
            \Sigma(p)=\left\{x\in X\Big|\abs{\supp{x}}\leq\aleph_0 \right\}
        \end{equation*}
    \end{mydef}

    \begin{exa}
        Sea $\left\{G_i \right\}_{ i\in I}$ una familia de grupos topológicos compactos con $\abs{G_i}>0$ para todo $i\in I$, donde $\abs{I}>\aleph_0$. En el producto
        \begin{equation*}
            G=\prod_{ i\in I}G_i
        \end{equation*}
        considere el $\Sigma$-producto $G^*$ con la identidad $e_G$ como punto base (básicamente se toman todos los elementos de $G$ tales que el elemento tiene a lo sumo una cantidad numerable de entradas diferentes de la identidad). Se tiene que $G^*$ es un subgrupo propio denso de $G$ que es numerablemente compacto que no es compacto.
    \end{exa}

    \begin{proof}
        Ya se tiene que $G^*$ es un subgrupo propio de $G$. Veamos que es denso. En efecto, el hecho de que sea denso se sigue de que $H<G$ definido como todos los elementos de $G$ que tienen una cantidad finita de entradas diferentes de la identidad es denso en $G$, luego $H<G^*<G$. Veamos que $G^*$ es numerablemente compacto.

        En efecto, considere un subconjunto infinito numerable $F\subseteq G^*$. Debemos probar que $F$ tiene un punto de acumulación en $G^*$. Sea
        \begin{equation*}
            H=\bigcup\left\{\supp{x}\Big|x\in F \right\}
        \end{equation*}
        Por definición, $\supp{x}$ es a lo sumo numerable para todo $x\in F$, luego $\abs{H}\leq\aleph_0$. Observe que por el teorema de Tikhonov, el espacio
        \begin{equation*}
            G'=\left[\prod_{i\in H}G_i\right]\times\left[\prod_{ j\in I\backslash H}\left\{e_j\right\} \right]
        \end{equation*}
        es compacto y es tal que $F\subseteq G'$, por lo cual existe $z\in G'$ tal que es punto de acumulación de $F$ en $G'$. Solo resta notar que $z\in G'\subseteq G^*$.
    \end{proof}

    \begin{mydef}
        Un subgrupo $H$ de un grupo topológico $G$ es \textbf{totalmente denso} en $G$ si para todo subgrupo cerrado $N$ de $G$ el subgrupo $H\cap N$ es denso en $N$. El subgrupo $H$ es \textbf{débilmente denso} en $G$ si para todo subgrupo normal $N$ de $G$, el conjunto $H\cap N$ es denso en $N$.
    \end{mydef}

    \begin{mydef}
        Un espacio $X$ se llama \textbf{precompacto} si la cerradura de cualquier subconjunto numerable $Y\subseteq X$ es compacta.
    \end{mydef}

    \begin{obs}
        Note que todo espacio precompacto es numerablemente compacto.
    \end{obs}

    \begin{propo}
        Sean $G$ un grupo compacto, $\cf{\varphi}{G}{G'}$ un homomorfismo continuo de $G$ sobre $G'$, $H'$ un subgrupo de $G'$ y $H=\varphi^{-1}(H')$. Entonces,
        \begin{enumerate}
            \item Si $H'$ es denso en $G'$, entonces $H$ es denso en $G$.
            \item Si $H'$ es totalmente denso en $G'$, entonces $H$ también es totalmente denso en $G$.
            \item Si $H'$ es precompacto, entonces $H$ también lo es.
        \end{enumerate}
    \end{propo}

    \begin{proof}
        %TODO
        Antes note que todo homomorfismo $\cf{\varphi}{G}{G'}$ continuo es abierto. En partcular, $\varphi\big|_H$ es abierto: luego, si $U$ es un abierto en $H$ y $V$ es un abierto en $G$ tal que
        \begin{equation*}
            U=H\cap V
        \end{equation*}
        entonces, $\varphi(U)=\varphi(V)\cap H'$. Por tanto, $\varphi(U)$ es abierto en $H'$. (recuerde que toda función continua de un compacto en un Hausdorff es perfecta).

        \begin{enumerate}
            \item Queremos probar que $H=\varphi^{-1}(H')$ es denso en $G$. Sea $U$ un abierto en $G$, entonces $\varphi(U)$ es un abierto en $G'$. Como $H'$ es denso en $G'$ se tiene que
            \begin{equation*}
                \varphi(U)\cap H'\neq\emptyset\Rightarrow U\cap\varphi^{-1}(H')\neq\emptyset
            \end{equation*}
            así, $H$ es denso en $G$.
            \item Sea $K$ un subgrupo cerrado de $G$. Se tiene que $H'\cap\varphi(K)$ es denso en $G$. Sea $U$ un abierto en $G$, entonces $\varphi(U)$ es abierto en $G'$ y por consiguiente:
            \begin{equation*}
                s
            \end{equation*}
        \end{enumerate}
        
        De (1): 
    \end{proof}

\end{document}