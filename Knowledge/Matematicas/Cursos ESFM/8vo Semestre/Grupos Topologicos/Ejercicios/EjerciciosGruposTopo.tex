\documentclass[12pt]{report}
\usepackage[spanish]{babel}
\usepackage[utf8]{inputenc}
\usepackage{amsmath}
\usepackage{amssymb}
\usepackage{amsthm}
\usepackage{graphics}
\usepackage{subfigure}
\usepackage{lipsum}
\usepackage{array}
\usepackage{multicol}
\usepackage{enumerate}
\usepackage[framemethod=TikZ]{mdframed}
\usepackage[a4paper, margin = 1.5cm]{geometry}

%En esta parte se hacen redefiniciones de algunos comandos para que resulte agradable el verlos%

\def\proof{\paragraph{Demostración:\\}}
\def\endproof{\hfill$\blacksquare$}

\def\sol{\paragraph{Solución:\\}}
\def\endsol{\hfill$\square$}

%En esta parte se definen los comandos a usar dentro del documento para enlistar%

\newtheoremstyle{largebreak}
  {}% use the default space above
  {}% use the default space below
  {\normalfont}% body font
  {}% indent (0pt)
  {\bfseries}% header font
  {}% punctuation
  {\newline}% break after header
  {}% header spec

\theoremstyle{largebreak}

\newmdtheoremenv[
    leftmargin=0em,
    rightmargin=0em,
    innertopmargin=0pt,
    innerbottommargin=5pt,
    hidealllines = true,
    roundcorner = 5pt,
    backgroundcolor = gray!60!red!30
]{exa}{Ejemplo}[section]

\newmdtheoremenv[
    leftmargin=0em,
    rightmargin=0em,
    innertopmargin=0pt,
    innerbottommargin=5pt,
    hidealllines = true,
    roundcorner = 5pt,
    backgroundcolor = gray!50!blue!30
]{obs}{Observación}[section]

\newmdtheoremenv[
    leftmargin=0em,
    rightmargin=0em,
    innertopmargin=0pt,
    innerbottommargin=5pt,
    rightline = false,
    leftline = false
]{theor}{Teorema}[section]

\newmdtheoremenv[
    leftmargin=0em,
    rightmargin=0em,
    innertopmargin=0pt,
    innerbottommargin=5pt,
    rightline = false,
    leftline = false
]{propo}{Proposición}[section]

\newmdtheoremenv[
    leftmargin=0em,
    rightmargin=0em,
    innertopmargin=0pt,
    innerbottommargin=5pt,
    rightline = false,
    leftline = false
]{cor}{Corolario}[section]

\newmdtheoremenv[
    leftmargin=0em,
    rightmargin=0em,
    innertopmargin=0pt,
    innerbottommargin=5pt,
    rightline = false,
    leftline = false
]{lema}{Lema}[section]

\newmdtheoremenv[
    leftmargin=0em,
    rightmargin=0em,
    innertopmargin=0pt,
    innerbottommargin=5pt,
    roundcorner=5pt,
    backgroundcolor = gray!30,
    hidealllines = true
]{mydef}{Definición}[section]

\newmdtheoremenv[
    leftmargin=0em,
    rightmargin=0em,
    innertopmargin=0pt,
    innerbottommargin=5pt,
    roundcorner=5pt
]{excer}{Ejercicio}[section]

%En esta parte se colocan comandos que definen la forma en la que se van a escribir ciertas funciones%

\newcommand\abs[1]{\ensuremath{\lvert#1\rvert}}
\newcommand\divides{\ensuremath{\bigm|}}
\newcommand{\cf}[3]{\ensuremath{#1:#2\rightarrow#3}}
\newcommand{\N}[1]{\ensuremath{\mathscr{N}(#1)}}
\newcommand{\Ns}[1]{\ensuremath{\mathscr{N}^*(#1)}}
\newcommand{\piz}[1]{\ensuremath{\mathscr{#1}}}
\newcommand{\eul}[1]{\ensuremath{\mathbb{#1}}}
\newcommand{\natint}[1]{\ensuremath{\left[\big|#1\big|\right]}}
\newcommand{\contradiction}{\ensuremath{\#_c}}
\newcommand{\Cls}[1]{\ensuremath{\overline{#1}}}

\begin{document}
    \setlength{\parskip}{5pt} % Añade 5 puntos de espacio entre párrafos
    \setlength{\parindent}{12pt} % Pone la sangría como me gusta
    \title{Ejercicios Grupos Topológicos}
    \author{Cristo Daniel Alvarado}
    \maketitle

    \tableofcontents %Con este comando se genera el índice general del libro%

    \chapter{Ejercicios Capítulo 1}

    \begin{excer}
        Sea $H$ un subgrupo denso abeliano de un grupo topológico $G$. Entonces, $G$ es abeliano.
    \end{excer}

    \begin{proof}
        Por la proposición 1.3.2 (4), como $ab=ba$ para todo $a,b\in H$, entonces se sigue que $ab=ba$ para todo $a,b\in\Cls{H}$. Como $H$ es denso en $G$ es tiene entonces que $\Cls{H}=G$, es decir:
        \begin{equation*}
            ab=ba,\quad\forall a,b\in G
        \end{equation*}
        por tanto, $G$ es abeliano.
    \end{proof}

    \begin{excer}
        Suponga que $H$ es un subgrupo denso de un grupo topológico $G$ y $n\in\mathbb{N}$. Pruebe que si $x^n=e_G$ para todo $x\in H$, entonces los elementos del grupo $G$ satisfacen la misma ecuación.
    \end{excer}

    \begin{proof}
        Sea $\cf{f}{G}{G}$ tal que $x\mapsto x^n$. Esta es una función continua para la que cual se tiene que el conjunto
        \begin{equation*}
            \begin{split}
                A=&f^{-1}(e_G)\\
                =&\left\{x\in G\Big|f(x)=e_G \right\} \\
                =&\left\{x\in G\Big|x^n=e_G \right\} \\
            \end{split}
        \end{equation*}
        es cerrado, pero $H\subseteq A$, luego $G=\Cls{H}\subseteq A$, es decir que
        \begin{equation*}
            x^n=e_g,\quad\forall x\in G
        \end{equation*}
    \end{proof}

    \begin{mydef}
        Sea $G$ un grupo. Decimos que $G$ es \textbf{grupo de torsión} si para todo $g\in G$ existe $n_g\in G$ tal que $g^{n_g}=e_G$.
    \end{mydef}

    \newpage

    \begin{excer}
        Sean $G$ un grupo topológico y $H$ un subgrupo denso de $G$ tal que todo elemento $h\in H$ es de orden finito. ¿Es $G$ de torsión?
    \end{excer}

    \begin{sol}
        Considere el grupo $(\mathbb{S}^1,\cdot)$ donde:
        \begin{equation*}
            \mathbb{S}^1=\left\{e^{ix}\in\mathbb{C} \Big|x\in\mathbb{R} \right\}
        \end{equation*}
        donde el producto $\cdot$ es el producto usual de $\mathbb{C}$, dado por:
        \begin{equation*}
            e^{ix}\cdot e^{iy}=e^{i(x+y)}
        \end{equation*}
        dotado de la topología $\tau_{\mathbb{S}^1}$
        \begin{equation*}
            \tau_{\mathbb{S}^1}=\left\{U\cap\mathbb{S}^1\Big| U\textup{ es abierto en }\mathbb{C} \right\}
        \end{equation*}
        es claro que las funciones $\cf{f}{\mathbb{S}^1 \times \mathbb{S}^1}{\mathbb{S}^1}$ tales que $(e^{ix},e^{iy})\mapsto e^{i(x+y)}$ y, $\cf{g}{\mathbb{S}^1}{\mathbb{S}^1}$ tales que $e^{ ix}\mapsto e^{-ix}$ son continuas ya que son reestricciones de funciones continuas de $\mathbb{C}\backslash\left\{ 0\right\}$ a $\mathbb{C}\backslash\left\{ 0\right\}$. Es claro que el conjunto:
        \begin{equation*}
            \mathbb{H}^1=\left\{e^{2\pi ir}\in\mathbb{S}^1\Big|r\in\mathbb{Q} \right\}
        \end{equation*}
        es subgrupo de $(\mathbb{S}^1,\cdot)$, el cual es denso en $\mathbb{S}^1$, para el que se cumple que todo elemento es de orden finito, pues si $r=\frac{p}{q}\in\mathbb{Q}$:
        \begin{equation*}
            (e^{2\pi ir})^q=e^{2\pi ip}=1
        \end{equation*}
        donde $1$ es la identidad de $(\mathbb{S}^1,\cdot)$. Por ende, todo elemento del subgrupo denso $\mathbb{H}^1$ es de orden finito, pero $G$ no es de torsión, ya que el elemento
        \begin{equation*}
            e^{i}
        \end{equation*}
        no es de orden finito.
    \end{sol}

    \begin{excer}
        Demuestre que si $S$ es denso en un grupo topológico $G$ y $O$ es abierto no vacío en $G$, entonces $O\cdot S=S\cdot O= G$.
    \end{excer}

    \begin{proof}
        
    \end{proof}

    %página 58 del pdf
    
    \begin{excer}
        Sea $G$ un grupo topológico. ¿Es $G'=\left\{xyx^{-1}y^{-1}\in G\Big|x,y\in G \right\}$ un subgrupo de $G$? ¿Es $G'$ cerrado en $G$?
    \end{excer}

    \begin{sol}
        Afirmamos que $G'$ no es subgrupo de $G$. En efecto, es claro que $e\in G'$, pero... (hay algo con el producto que falla)
        %TODO

        Es cerrado, ya que si $\cf{f}{G\times G}{G}$ es tal que $(x,y)\mapsto xyx^{-1}y^{-1}$, se tiene que $f$ es una función continua para la cual
        \begin{equation*}
            \begin{split}
                G'&=\left\{xyx^{-1}y^{-1}\in G\Big|x,y\in G \right\}\\
                &=\left\{f(x,y)\in G\Big|(x,y)\in G\times G \right\}\\
                &=f^{-1}(G)
            \end{split}
        \end{equation*}
        es decir, que $G'$ es la imagen inversa de un cerrado (el conjunto $G$) y, por ende es cerrado.
    \end{sol}

    \begin{excer}
        Pruebe que si $G$ es un grupo topológico, entonces el conjunto
        \begin{equation*}
            H=\left\{g\in G\Big|gx=xg,\forall x\in G \right\}
        \end{equation*}
        es un subgrupo cerrado normal de $G$.
    \end{excer}

    \begin{proof}
        Veamos que es subgrupo. En efecto, es claro que $e_G\in H$. Sean ahora $g,h\in G$, entonces se tiene que $g^{-1}\in G$, pues:
        \begin{equation*}
            \begin{split}
                gx&=xg\\
                \Rightarrow gxg^{-1}&=x\\
                \Rightarrow xg^{-1}&=g^{-1}x\\
                \Rightarrow g^{-1}x&=xg^{-1}\\\
            \end{split}
        \end{equation*}
        $\forall x\in G$ y, además:
        \begin{equation*}
            (gh)x=g(hx)=g(xh)=(gx)h=x(gh),\quad\forall x\in G
        \end{equation*}
        por tanto, $gh\in H$. Se sigue entonces que $H$ es subgrupo de $G$.

        Veamos que es normal. Sea $g\in G$ y $h\in G$, hay que ver que $ghg^{-1}\in H$. En efecto, veamos que:
        \begin{equation*}
            (ghg^{-1})x=(gg^{-1})hx=(e_G)xh=x(he_G)=x(hgg^{-1})=x(ghg^{-1}),\quad\forall x\in G
        \end{equation*}
        por tanto, $ghg^{-1}\in H$. Luego, $H$ es normal en $G$.

        Ahora, como $H$ es subgrupo, entonces $\Cls{H}$ también lo es...
        %TODO
    \end{proof}

    \begin{excer}
        Sea $G$ un grupo tal que todos sus elementos son de orden $2$. Demuestre que $G$ tiene que ser abeliano. Pruebe que si $G$ es infinito, entonces admite una topología de Hausdorff no discreta.
    \end{excer}

    \begin{proof}
        Veamos que $G$ es abeliano. En efecto, sean $a,b\in G$, se tiene entonces que:
        \begin{equation*}
            (ab)^2=(ab)(ab)=e_G
        \end{equation*}
        es decir, que $ab=(ab^{-1})=b^{-1}a^{-1}$, pero $a^{-1}=a$ y $b^{-1}=b$. Por ende, $ab=ba$ luego, $G$ es abeliano.

        Suponga que $G$ es infinito. (no sé).
    \end{proof}

    \begin{excer}
        Dé un ejemplo de grupo que admite al menos dos topologías de Hausdorff de grupo distintas. 
    \end{excer}

    \begin{sol}
        
    \end{sol}

    \begin{excer}
        Sea $G=\mathbb{R}\backslash\left\{0\right\}$ el grupo multiplicativo de los números reales con la topología usual, y sean $G'=\left\{-1,1\right\}$ y $G''=\left\{x\in \mathbb{R}\Big|x>0 \right\}$.
        \begin{enumerate}
            \item Pruebe que $G'$ y $G''$ son subgrupos de $G$.
            \item Pruebe que existe un isomorfismo topológico entre $G/G'$ y $G''$.
            \item Pruebe que $G$ y $G'\oplus G''$ son topológicamente isomorfos.
            \item Pruebe que $G'\cong \mathbb{Z}_2$, $G''\cong \mathbb{R}$ y, deduzca que $G\cong \mathbb{Z}_2\oplus \mathbb{R}$.
        \end{enumerate}
    \end{excer}

    \begin{proof}
        De (1): Es claro que son subgrupos de $G$.

        De (2): Notemos que:
        \begin{equation*}
            \begin{split}
                G/G'&=\left\{ G'a\Big|a\in G \right\}\\
                &=\left\{ \left\{-1,1\right\}a\Big|a\in G \right\}\\
                &=\left\{ \left\{-a,a\right\}\Big|a\in G \right\}\\
            \end{split}
        \end{equation*}
        Defina $\cf{f}{G''}{G/G'}$ tal que $a\mapsto \left\{-a,a\right\}$. Afirmamos que esta función es continua. En efecto, 
    \end{proof}

\end{document}