\documentclass[12pt]{report}
\usepackage[spanish]{babel}
\usepackage[utf8]{inputenc}
\usepackage{amsmath}
\usepackage{amssymb}
\usepackage{amsthm}
\usepackage{graphics}
\usepackage{subfigure}
\usepackage{lipsum}
\usepackage{array}
\usepackage{multicol}
\usepackage{enumerate}
\usepackage[framemethod=TikZ]{mdframed}
\usepackage[a4paper, margin = 1.5cm]{geometry}

%En esta parte se hacen redefiniciones de algunos comandos para que resulte agradable el verlos%

\def\proof{\paragraph{Demostración:\\}}
\def\endproof{\hfill$\blacksquare$}

\def\sol{\paragraph{Solución:\\}}
\def\endsol{\hfill$\square$}

%En esta parte se definen los comandos a usar dentro del documento para enlistar%

\newtheoremstyle{largebreak}
  {}% use the default space above
  {}% use the default space below
  {\normalfont}% body font
  {}% indent (0pt)
  {\bfseries}% header font
  {}% punctuation
  {\newline}% break after header
  {}% header spec

\theoremstyle{largebreak}

\newmdtheoremenv[
    leftmargin=0em,
    rightmargin=0em,
    innertopmargin=0pt,
    innerbottommargin=5pt,
    hidealllines = true,
    roundcorner = 5pt,
    backgroundcolor = gray!60!red!30
]{exa}{Ejemplo}[section]

\newmdtheoremenv[
    leftmargin=0em,
    rightmargin=0em,
    innertopmargin=0pt,
    innerbottommargin=5pt,
    hidealllines = true,
    roundcorner = 5pt,
    backgroundcolor = gray!50!blue!30
]{obs}{Observación}[section]

\newmdtheoremenv[
    leftmargin=0em,
    rightmargin=0em,
    innertopmargin=0pt,
    innerbottommargin=5pt,
    rightline = false,
    leftline = false
]{theor}{Teorema}[section]

\newmdtheoremenv[
    leftmargin=0em,
    rightmargin=0em,
    innertopmargin=0pt,
    innerbottommargin=5pt,
    rightline = false,
    leftline = false
]{propo}{Proposición}[section]

\newmdtheoremenv[
    leftmargin=0em,
    rightmargin=0em,
    innertopmargin=0pt,
    innerbottommargin=5pt,
    rightline = false,
    leftline = false
]{cor}{Corolario}[section]

\newmdtheoremenv[
    leftmargin=0em,
    rightmargin=0em,
    innertopmargin=0pt,
    innerbottommargin=5pt,
    rightline = false,
    leftline = false
]{lema}{Lema}[section]

\newmdtheoremenv[
    leftmargin=0em,
    rightmargin=0em,
    innertopmargin=0pt,
    innerbottommargin=5pt,
    roundcorner=5pt,
    backgroundcolor = gray!30,
    hidealllines = true
]{mydef}{Definición}[section]

\newmdtheoremenv[
    leftmargin=0em,
    rightmargin=0em,
    innertopmargin=0pt,
    innerbottommargin=5pt,
    roundcorner=5pt
]{excer}{Ejercicio}[section]

%En esta parte se colocan comandos que definen la forma en la que se van a escribir ciertas funciones%

\newcommand\abs[1]{\ensuremath{\lvert#1\rvert}}
\newcommand\divides{\ensuremath{\bigm|}}
\newcommand{\cf}[3]{\ensuremath{#1:#2\rightarrow#3}}
\newcommand{\N}[1]{\ensuremath{\mathscr{N}(#1)}}
\newcommand{\Ns}[1]{\ensuremath{\mathscr{N}^*(#1)}}
\newcommand{\piz}[1]{\ensuremath{\mathscr{#1}}}
\newcommand{\eul}[1]{\ensuremath{\mathbb{#1}}}
\newcommand{\natint}[1]{\ensuremath{\left[\big|#1\big|\right]}}
\newcommand{\contradiction}{\ensuremath{\#_c}}
\newcommand{\Cls}[1]{\ensuremath{\overline{#1}}}

\begin{document}
    \setlength{\parskip}{5pt} % Añade 5 puntos de espacio entre párrafos
    \setlength{\parindent}{12pt} % Pone la sangría como me gusta
    \title{Ejercicios Grupos Topológicos}
    \author{Cristo Daniel Alvarado}
    \maketitle

    \tableofcontents %Con este comando se genera el índice general del libro%

    \chapter{Ejercicios Capítulo 1}

    \begin{excer}
        Sea $H$ un subgrupo denso abeliano de un grupo topológico $G$. Entonces, $G$ es abeliano.
    \end{excer}

    \begin{proof}
        Por la proposición 1.3.2 (4), como $ab=ba$ para todo $a,b\in H$, entonces se sigue que $ab=ba$ para todo $a,b\in\Cls{H}$. Como $H$ es denso en $G$ es tiene entonces que $\Cls{H}=G$, es decir:
        \begin{equation*}
            ab=ba,\quad\forall a,b\in G
        \end{equation*}
        por tanto, $G$ es abeliano.
    \end{proof}

    \begin{excer}
        Suponga que $H$ es un subgrupo denso de un grupo topológico $G$ y $n\in\mathbb{N}$. Pruebe que si $x^n=e_G$ para todo $x\in H$, entonces los elementos del grupo $G$ satisfacen la misma ecuación.
    \end{excer}

    \begin{proof}
        Sea $\cf{f}{G}{G}$ tal que $x\mapsto x^n$. Esta es una función continua para la que cual se tiene que el conjunto
        \begin{equation*}
            \begin{split}
                A=&f^{-1}(e_G)\\
                =&\left\{x\in G\Big|f(x)=e_G \right\} \\
                =&\left\{x\in G\Big|x^n=e_G \right\} \\
            \end{split}
        \end{equation*}
        es cerrado, pero $H\subseteq A$, luego $G=\Cls{H}\subseteq A$, es decir que
        \begin{equation*}
            x^n=e_g,\quad\forall x\in G
        \end{equation*}
    \end{proof}

    \begin{mydef}
        Sea $G$ un grupo. Decimos que $G$ es \textbf{grupo de torsión} si para todo $g\in G$ existe $n_g\in G$ tal que $g^{n_g}=e_G$.
    \end{mydef}

    \newpage

    \begin{excer}
        Sean $G$ un grupo topológico y $H$ un subgrupo denso de $G$ tal que todo elemento $h\in H$ es de orden finito. ¿Es $G$ de torsión?
    \end{excer}

    \begin{sol}
        No creo, pa. Ahí para la otra te contesto, padre.
    \end{sol}

    \begin{excer}
        Demuestre que si $S$ es denso en un grupo topológico $G$ y $O$ es abierto no vacío en $G$, entonces $O\cdot S=S\cdot O= G$.
    \end{excer}

    \begin{proof}
        
    \end{proof}

    %página 58 del pdf
    

\end{document}