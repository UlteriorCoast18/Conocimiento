\documentclass[12pt]{report}
\usepackage[spanish]{babel}
\usepackage[utf8]{inputenc}
\usepackage{amsmath}
\usepackage{amssymb}
\usepackage{amsthm}
\usepackage{graphics}
\usepackage{subfigure}
\usepackage{lipsum}
\usepackage{array}
\usepackage{multicol}
\usepackage{enumerate}
\usepackage[framemethod=TikZ]{mdframed}
\usepackage[a4paper, margin = 1.5cm]{geometry}

%En esta parte se hacen redefiniciones de algunos comandos para que resulte agradable el verlos%

\def\proof{\paragraph{Demostración:\\}}
\def\endproof{\hfill$\blacksquare$}

\def\sol{\paragraph{Solución:\\}}
\def\endsol{\hfill$\square$}

%En esta parte se definen los comandos a usar dentro del documento para enlistar%

\newtheoremstyle{largebreak}
  {}% use the default space above
  {}% use the default space below
  {\normalfont}% body font
  {}% indent (0pt)
  {\bfseries}% header font
  {}% punctuation
  {\newline}% break after header
  {}% header spec

\theoremstyle{largebreak}

\newmdtheoremenv[
    leftmargin=0em,
    rightmargin=0em,
    innertopmargin=0pt,
    innerbottommargin=5pt,
    hidealllines = true,
    roundcorner = 5pt,
    backgroundcolor = gray!60!red!30
]{exa}{Ejemplo}[section]

\newmdtheoremenv[
    leftmargin=0em,
    rightmargin=0em,
    innertopmargin=0pt,
    innerbottommargin=5pt,
    hidealllines = true,
    roundcorner = 5pt,
    backgroundcolor = gray!50!blue!30
]{obs}{Observación}[section]

\newmdtheoremenv[
    leftmargin=0em,
    rightmargin=0em,
    innertopmargin=0pt,
    innerbottommargin=5pt,
    rightline = false,
    leftline = false
]{theor}{Teorema}[section]

\newmdtheoremenv[
    leftmargin=0em,
    rightmargin=0em,
    innertopmargin=0pt,
    innerbottommargin=5pt,
    rightline = false,
    leftline = false
]{propo}{Proposición}[section]

\newmdtheoremenv[
    leftmargin=0em,
    rightmargin=0em,
    innertopmargin=0pt,
    innerbottommargin=5pt,
    rightline = false,
    leftline = false
]{cor}{Corolario}[section]

\newmdtheoremenv[
    leftmargin=0em,
    rightmargin=0em,
    innertopmargin=0pt,
    innerbottommargin=5pt,
    rightline = false,
    leftline = false
]{lema}{Lema}[section]

\newmdtheoremenv[
    leftmargin=0em,
    rightmargin=0em,
    innertopmargin=0pt,
    innerbottommargin=5pt,
    roundcorner=5pt,
    backgroundcolor = gray!30,
    hidealllines = true
]{mydef}{Definición}[section]

\newmdtheoremenv[
    leftmargin=0em,
    rightmargin=0em,
    innertopmargin=0pt,
    innerbottommargin=5pt,
    roundcorner=5pt
]{excer}{Ejercicio}[section]

%En esta parte se colocan comandos que definen la forma en la que se van a escribir ciertas funciones%

\newcommand\abs[1]{\ensuremath{\lvert#1\rvert}}
\newcommand\divides{\ensuremath{\bigm|}}
\newcommand{\cf}[3]{\ensuremath{#1:#2\rightarrow#3}}
\newcommand{\N}[1]{\ensuremath{\mathscr{N}(#1)}}
\newcommand{\Ns}[1]{\ensuremath{\mathscr{N}^*(#1)}}
\newcommand{\piz}[1]{\ensuremath{\mathscr{#1}}}
\newcommand{\eul}[1]{\ensuremath{\mathbb{#1}}}
\newcommand{\natint}[1]{\ensuremath{\left[\big|#1\big|\right]}}
\newcommand{\contradiction}{\ensuremath{\#_c}}
\newcommand{\Cls}[1]{\ensuremath{\overline{#1}}}

\begin{document}
    \setlength{\parskip}{5pt} % Añade 5 puntos de espacio entre párrafos
    \setlength{\parindent}{12pt} % Pone la sangría como me gusta
    \title{Ejercicios Grupos Topológicos}
    \author{Cristo Daniel Alvarado}
    \maketitle

    \tableofcontents %Con este comando se genera el índice general del libro%

    \chapter{Ejercicios Capítulo 1}

    \begin{excer}
        Sea $H$ un subgrupo denso abeliano de un grupo topológico $G$. Entonces, $G$ es abeliano.
    \end{excer}

    \begin{proof}
        Por la proposición 1.3.2 (4), como $ab=ba$ para todo $a,b\in H$, entonces se sigue que $ab=ba$ para todo $a,b\in\Cls{H}$. Como $H$ es denso en $G$ es tiene entonces que $\Cls{H}=G$, es decir:
        \begin{equation*}
            ab=ba,\quad\forall a,b\in G
        \end{equation*}
        por tanto, $G$ es abeliano.
    \end{proof}

    \begin{excer}
        Suponga que $H$ es un subgrupo denso de un grupo topológico $G$ y $n\in\mathbb{N}$. Pruebe que si $x^n=e_G$ para todo $x\in H$, entonces los elementos del grupo $G$ satisfacen la misma ecuación.
    \end{excer}

    \begin{proof}
        Sea $\cf{f}{G}{G}$ tal que $x\mapsto x^n$. Esta es una función continua para la que cual se tiene que el conjunto
        \begin{equation*}
            \begin{split}
                A=&f^{-1}(e_G)\\
                =&\left\{x\in G\Big|f(x)=e_G \right\} \\
                =&\left\{x\in G\Big|x^n=e_G \right\} \\
            \end{split}
        \end{equation*}
        es cerrado, pero $H\subseteq A$, luego $G=\Cls{H}\subseteq A$, es decir que
        \begin{equation*}
            x^n=e_g,\quad\forall x\in G
        \end{equation*}
    \end{proof}

    \begin{mydef}
        Sea $G$ un grupo. Decimos que $G$ es \textbf{grupo de torsión} si para todo $g\in G$ existe $n_g\in G$ tal que $g^{n_g}=e_G$.
    \end{mydef}

    \newpage

    \begin{excer}
        Sean $G$ un grupo topológico y $H$ un subgrupo denso de $G$ tal que todo elemento $h\in H$ es de orden finito. ¿Es $G$ de torsión?
    \end{excer}

    \begin{sol}
        Considere el grupo $(\mathbb{S}^1,\cdot)$ donde:
        \begin{equation*}
            \mathbb{S}^1=\left\{e^{ix}\in\mathbb{C} \Big|x\in\mathbb{R} \right\}
        \end{equation*}
        donde el producto $\cdot$ es el producto usual de $\mathbb{C}$, dado por:
        \begin{equation*}
            e^{ix}\cdot e^{iy}=e^{i(x+y)}
        \end{equation*}
        dotado de la topología $\tau_{\mathbb{S}^1}$
        \begin{equation*}
            \tau_{\mathbb{S}^1}=\left\{U\cap\mathbb{S}^1\Big| U\textup{ es abierto en }\mathbb{C} \right\}
        \end{equation*}
        es claro que las funciones $\cf{f}{\mathbb{S}^1 \times \mathbb{S}^1}{\mathbb{S}^1}$ tales que $(e^{ix},e^{iy})\mapsto e^{i(x+y)}$ y, $\cf{g}{\mathbb{S}^1}{\mathbb{S}^1}$ tales que $e^{ ix}\mapsto e^{-ix}$ son continuas ya que son reestricciones de funciones continuas de $\mathbb{C}\backslash\left\{ 0\right\}$ a $\mathbb{C}\backslash\left\{ 0\right\}$. Es claro que el conjunto:
        \begin{equation*}
            \mathbb{H}^1=\left\{e^{2\pi ir}\in\mathbb{S}^1\Big|r\in\mathbb{Q} \right\}
        \end{equation*}
        es subgrupo de $(\mathbb{S}^1,\cdot)$, el cual es denso en $\mathbb{S}^1$, para el que se cumple que todo elemento es de orden finito, pues si $r=\frac{p}{q}\in\mathbb{Q}$:
        \begin{equation*}
            (e^{2\pi ir})^q=e^{2\pi ip}=1
        \end{equation*}
        donde $1$ es la identidad de $(\mathbb{S}^1,\cdot)$. Por ende, todo elemento del subgrupo denso $\mathbb{H}^1$ es de orden finito, pero $G$ no es de torsión, ya que el elemento
        \begin{equation*}
            e^{i}
        \end{equation*}
        no es de orden finito.
    \end{sol}

    \begin{excer}
        Demuestre que si $S$ es denso en un grupo topológico $G$ y $O$ es abierto no vacío en $G$, entonces $O\cdot S=S\cdot O= G$.
    \end{excer}

    \begin{proof}
        
    \end{proof}

    %página 58 del pdf
    
    \begin{excer}
        Sea $G$ un grupo topológico. ¿Es $G'=\left\{xyx^{-1}y^{-1}\in G\Big|x,y\in G \right\}$ un subgrupo de $G$? ¿Es $G'$ cerrado en $G$?
    \end{excer}

    \begin{sol}
        Afirmamos que $G'$ no es subgrupo de $G$. En efecto, es claro que $e\in G'$, pero... (hay algo con el producto que falla)
        %TODO

        Es cerrado, ya que si $\cf{f}{G\times G}{G}$ es tal que $(x,y)\mapsto xyx^{-1}y^{-1}$, se tiene que $f$ es una función continua para la cual
        \begin{equation*}
            \begin{split}
                G'&=\left\{xyx^{-1}y^{-1}\in G\Big|x,y\in G \right\}\\
                &=\left\{f(x,y)\in G\Big|(x,y)\in G\times G \right\}\\
                &=f^{-1}(G)
            \end{split}
        \end{equation*}
        es decir, que $G'$ es la imagen inversa de un cerrado (el conjunto $G$) y, por ende es cerrado.
    \end{sol}

    \begin{excer}
        Pruebe que si $G$ es un grupo topológico, entonces el conjunto
        \begin{equation*}
            H=\left\{g\in G\Big|gx=xg,\forall x\in G \right\}
        \end{equation*}
        es un subgrupo cerrado normal de $G$.
    \end{excer}

    \begin{proof}
        Veamos que es subgrupo. En efecto, es claro que $e_G\in H$. Sean ahora $g,h\in G$, entonces se tiene que $g^{-1}\in G$, pues:
        \begin{equation*}
            \begin{split}
                gx&=xg\\
                \Rightarrow gxg^{-1}&=x\\
                \Rightarrow xg^{-1}&=g^{-1}x\\
                \Rightarrow g^{-1}x&=xg^{-1}\\\
            \end{split}
        \end{equation*}
        $\forall x\in G$ y, además:
        \begin{equation*}
            (gh)x=g(hx)=g(xh)=(gx)h=x(gh),\quad\forall x\in G
        \end{equation*}
        por tanto, $gh\in H$. Se sigue entonces que $H$ es subgrupo de $G$.

        Veamos que es normal. Sea $g\in G$ y $h\in G$, hay que ver que $ghg^{-1}\in H$. En efecto, veamos que:
        \begin{equation*}
            (ghg^{-1})x=(gg^{-1})hx=(e_G)xh=x(he_G)=x(hgg^{-1})=x(ghg^{-1}),\quad\forall x\in G
        \end{equation*}
        por tanto, $ghg^{-1}\in H$. Luego, $H$ es normal en $G$.

        Ahora, como $H$ es subgrupo, entonces $\Cls{H}$ también lo es...
        %TODO
    \end{proof}

    \begin{excer}
        Sea $G$ un grupo tal que todos sus elementos son de orden $2$. Demuestre que $G$ tiene que ser abeliano. Pruebe que si $G$ es infinito, entonces admite una topología de Hausdorff no discreta.
    \end{excer}

    \begin{proof}
        Veamos que $G$ es abeliano. En efecto, sean $a,b\in G$, se tiene entonces que:
        \begin{equation*}
            (ab)^2=(ab)(ab)=e_G
        \end{equation*}
        es decir, que $ab=(ab^{-1})=b^{-1}a^{-1}$, pero $a^{-1}=a$ y $b^{-1}=b$. Por ende, $ab=ba$ luego, $G$ es abeliano.

        Suponga que $G$ es infinito. (no sé).
    \end{proof}

    \begin{excer}
        Dé un ejemplo de grupo que admite al menos dos topologías de Hausdorff de grupo distintas. 
    \end{excer}

    \begin{sol}
        
    \end{sol}

    \begin{excer}
        Sea $G=\mathbb{R}\backslash\left\{0\right\}$ el grupo multiplicativo de los números reales con la topología usual, y sean $G'=\left\{-1,1\right\}$ y $G''=\left\{x\in \mathbb{R}\Big|x>0 \right\}$.
        \begin{enumerate}
            \item Pruebe que $G'$ y $G''$ son subgrupos de $G$.
            \item Pruebe que existe un isomorfismo topológico entre $G/G'$ y $G''$.
            \item Pruebe que $G$ y $G'\oplus G''$ son topológicamente isomorfos.
            \item Pruebe que $G'\cong \mathbb{Z}_2$, $G''\cong \mathbb{R}$ y, deduzca que $G\cong \mathbb{Z}_2\oplus \mathbb{R}$.
        \end{enumerate}
    \end{excer}

    \begin{proof}
        De (1): Es claro que son subgrupos de $G$.

        De (2): Notemos que:
        \begin{equation*}
            \begin{split}
                G/G'&=\left\{ G'a\Big|a\in G \right\}\\
                &=\left\{ \left\{-1,1\right\}a\Big|a\in G \right\}\\
                &=\left\{ \left\{-a,a\right\}\Big|a\in G \right\}\\
            \end{split}
        \end{equation*}
        Defina $\cf{f}{G''}{G/G'}$ tal que $a\mapsto \left\{-a,a\right\}$. Afirmamos que esta función es continua. En efecto, 
    \end{proof}

    \begin{excer}
        Sea $GL(n,\mathbb{R})$ el grupo lineal general con la topología definida en un ejemplo anterior. Introduzcamos los siguientes subconjuntos de $GL(n,\mathbb{R})$; el conjunto $SL(n,\mathbb{R})$ de las matrices con determinante igual a $1$; el conjunto $TL(n,\mathbb{R})$ de las matrices triangulares superiores con los elementos de la diagonal principal iguales a $1$; el conjunto $O(n,\mathbb{R})$ de las matrices ortogonales. Pruebe lo siguiente:
        \begin{enumerate}
            \item Cada uno de los conjuntos $SL(n,\mathbb{R})$, $TL(n,\mathbb{R})$, $O(n,\mathbb{R})$ es un subgrupo cerrado de $GL(n,\mathbb{R})$.
            \item $SL(n,\mathbb{R})$ es un subgrupo normal de $GL(n,\mathbb{R})$, pero $TL(n,\mathbb{R})$ y $O(n,\mathbb{R})$ no lo son si $n\geq 2$.
        \end{enumerate}
    \end{excer}

    \begin{proof}
        De (1): Primero, ya se sabe que $GL(n,\mathbb{R})$ es grupo con el producto usual de matrices. Veamos que es grupo topológico con la topología dotada por la métrica:
        \begin{equation*}
            d(A,B)=\sqrt{\sum_{i,j=1}^n\abs{A_{i,j}-B_{i,j}}^2}
        \end{equation*}
        ya que, la función $(A,B)\mapsto AB^{-1}$ es continua (podemos verla como una función de $\mathbb{R}^{ n^2}$ a $\mathbb{R}^n$ donde solo se involucran sumas, productos, cuadrados y diferencias de elementos de $\mathbb{R}$, por ende, es continua). Luego, es grupo topológico.

        Ya se sabe que $SL(n,\mathbb{R})$, $TL(n,\mathbb{R})$, $O(n,\mathbb{R})$ son subgrupos de $GL(n,\mathbb{R})$. Veamos que son cerrados.
        \begin{enumerate}
            \item $SL(n,\mathbb{R})$ es cerrado. En efecto, la función determinante $\cf{\det}{GL(n,\mathbb{R})}{\mathbb{R}}$ es continua (vista como función de $\mathbb{R}^{ n^2}$ a $\mathbb{R}$ lo es), además:
            \begin{equation*}
                \begin{split}
                    SL(n,\mathbb{R})&=\left\{A\in GL(n,\mathbb{R})\Big|\det(A)=1 \right\}\\
                    &=\textup{det}^{-1}(1) \\
                \end{split}
            \end{equation*}
            donde $\left\{1\right\}\subseteq\mathbb{R}$ es cerrado, luego $SL(n,\mathbb{R})$ es cerrado.

            \item $TL(n,\mathbb{R})$ es cerrado. En efecto, la función s
        \end{enumerate}
    \end{proof}

    \begin{excer}
        Sea $G$ un grupo topológico abeliano. Pruebe que para todo $n\in\mathbb{N}$, $G_n=\left\{g\in G\Big|g^n=e_G \right\}$ es un subgrupo cerrado de $G$. ¿Es válida la conclusión si el grupo $G$ no es abeliano?

        \textit{Sugerencia}. Considere el grupo $G=O(2,\mathbb{R})$.
    \end{excer}

    \begin{proof}
        Veamos que es subgrupo. Sea $n\in\mathbb{N}$, es claro que $e_G\in G_n$. Además, si $x,y\in G_n$, entonces:
        \begin{equation*}
            \left(xy^{-1}\right)^n=x^ny^{-n}=e_G
        \end{equation*}
        pues, como $y^n=e_G$, entonces $y^{-n}=e_G$. Luego, $xy^{-1}\in G_n$. Por ende, $G_n$ es subgrupo de $G$.

        Veamos ahora que es cerrado. En efecto, notemos que la función $\cf{f}{G}{G}$ tal que $x\mapsto x^n$ es una función continua, y
        \begin{equation*}
            \begin{split}
                G_n&=\left\{g\in G\Big| x^n=e_G \right\}\\
                &=\left\{g\in G\Big| f(x)=e_G \right\}\\
                &=\left\{g\in G\Big| f(x)\in\left\{e_G\right\} \right\}\\
                &=f^{-1}(e_G)\\
            \end{split}
        \end{equation*}
        donde el conjunto $\left\{e_G\right\}$ es cerrado, luego $G_n$ es cerrado.

        Para la otra parte, considere $G=O(2,\mathbb{R})$, se tiene entonces que:
        \begin{equation*}
            asd
        \end{equation*} 
    \end{proof}

    \begin{excer}
        Sea $S(X)$ el grupo de todas las permutaciones de un conjunto dado $X$, es decir, $S(X)$ consta de todas las funciones biyectivas de $X$ en $X$. Si $n\in\mathbb{N}$ y $x_1,...,x_n\in X$, y $y_1,...,y_n\in X$, denotemos:
        \begin{equation*}
            U(x_1,...,x_n,y_1,...,y_n)=\left\{f\in S(X)\Big| f(x_i)=y_i\textup{ para todo }i\in\natint{1,n} \right\}
        \end{equation*}
        Demuestre que la familia de todos los conjuntos $U(x_1,...,x_n,y_1,...,y_n)$ forma base de una topología de grupo Hausdorff $\mathcal{P}$ en $S(X)$. La topología $\mathcal{P}$ se llama \textbf{topología de la convergencia puntual en $S(X)$}.
    \end{excer}

    \begin{proof}
        Denotemos por $\mathcal{U}$ a la familia de todos estos conjuntos. Si el conjunto es vacío, esta familia es vacía, por lo que no tiene sentido analizar este caso particular, suponga entonces que $X\neq\emptyset$.

        Hay que verificar que se cumplen dos condiciones:
        \begin{enumerate}
            \item Sean $m,n\in\mathbb{N}$, y $x_1,...,x_n,y_1,...,y_n,z_1,...,z_m,u_1,...,u_m\in X$. Queremos ver que el conjunto:
            \begin{equation*}
                U(x_1,...,x_n,y_1,...,y_n)\cap U(z_1,...,z_m,u_1,...,u_m)
            \end{equation*}
            se expresa como unión de elementos de $\mathcal{U}$. En efecto, si $f$ está en la intersección si y sólo si
            \begin{equation*}
                f(x_i)=y_i\quad\textup{y}\quad f(z_j)=u_j
            \end{equation*}
            $\forall i\in\natint{1,n},j\in\natint{1,m}$, es decir que
            \begin{equation*}
                f\in U(x_1,...,x_n,z_1,...,z_m,y_1,...,y_n,u_1,...,u_m)
            \end{equation*}
            por tanto, se tiene que
            \begin{equation*}
                U(x_1,...,x_n,y_1,...,y_n)\cap U(z_1,...,z_m,u_1,...,u_m)=U(x_1,...,x_n,z_1,...,z_m,y_1,...,y_n,u_1,...,u_m)
            \end{equation*}
            (podemos renombrar los $x_i$ y $z_j$ como algún $\alpha_k$ y de manera análoga con los otros elementos de $X$, pero no es muy relevante a la demostración tal procedimiento).
            
            \item $X$ es unión de elementos de esta familia. En efecto, como $X$ es no vacío, existe $x_0\in X$, sea $\mathcal{X}_{x_0}=\left\{U(x_0,y)\Big|y\in X \right\}$. Se tiene entonces que:
            \begin{equation*}
                \bigcup_{U\in \mathcal{X}_{x_0}}U=X
            \end{equation*}
            en efecto, una contención es inmediata. Sea $f\in S(X)$, entonces $f(x_0)\in X$, luego $f\in U(x_0,f(x_0))\subseteq \bigcup_{U\in \mathcal{X}}U$.
        \end{enumerate}
        Luego, $\mathcal{U}$ es base de una topología sobre $S(X)$.

        Además, es Hausdorff. En efecto, sean $f,g\in S(X)$ tales que $f\neq g$, entonces existe $x\in X$ tal que $f(x)\neq g(x)$. Tenemos que:
        \begin{equation*}
            f\in U(x,f(x))\quad\textup{y}\quad g\in U(x,g(x))
        \end{equation*}
        se tiene que $U(x,f(x))\cap U(x,g(x))=\emptyset$ ya que los elementos de $S(X)$ son funciones. Por ende, estos son dos abiertos disjuntos que contienen a $f$ y $g$. Por tanto, el espacio es Hausdorff.
    \end{proof}

    \begin{excer}
        Sea $S_f(X)$ el subgrupo de $S(X)$ que consiste en todas las permutaciones de $X$ que mueven a lo más un número finito de puntos. Pruebe que $S_f(X)$ es denso en $S(X)$.
    \end{excer}

    \begin{proof}
        Hay que probar que todo abierto no vacío en $S(X)$ dotado de la topología $\mathcal{P}$ del inciso anterior, contiene puntos de $S_f(X)$.
        
        En efecto, sea $U\subseteq S(X)$ abierto no vacío y $f\in U$. Por el ejercicio anterior, como $\mathcal{U}$ es base de la topología $\mathcal{P}$ sobre $S(X)$, existen $n\in\mathbb{N}$ y $x_1,...,x_n,y_1,...,y_m\in X$ tales que
        \begin{equation*}
            f\in U(x_1,...,x_n,y_1,...,y_m)\subseteq U
        \end{equation*}
        es decir, que $f(x_i)=y_i$ para todo $i\in\natint{1,n}$. Se tiene entones que la función:
        \begin{equation*}
            i_n(x)=\left\{
                \begin{array}{lcr}
                    y_i & \textup{ si } & x=x_i\textup{ para algún }i\in\natint{1,n} \\
                    x & \textup{ si } & x\neq x_i\textup{ para todo }i\in\natint{1,n}\\
                \end{array}
            \right.
        \end{equation*}
        está en $S_f(X)$ y, más aún, en $U(x_1,...,x_n,y_1,...,y_m)$, luego $f\in U$. Por tanto, $U\cap S_f(X)\neq\emptyset$.

        Finalmente, se sigue que $S_f(X)$ es denso en $S(X)$.

    \end{proof}

    \begin{excer}[*]
        Sea $G$ grupo topológico que tiene una base en la identidad consistente de subgrupos de $G$. Demuestre que $G$ se encaja en $S(X)$ para algún conjunto $X$ como subgrupo topológico.
    \end{excer}

    \begin{excer}
        Sean $G$ cualquier grupo y $\mathcal{V}$ una familia de subgrupos normales de $G$ cerrada bajo intersecciones finitas. Muestre que la familia de todos los conjuntos de la forma $gN$, con $g$ recorriendo todo $G$ y $N$ recorriendo todo $\mathcal{V}$, es base para una topología de grupo para $G$.
    \end{excer}

    \begin{proof}
        
    \end{proof}

    \begin{excer}
        Sea $\mathcal{F}$ la familia de todos los subgrupos de un grupo dado $G$ que tienen índice finito en $G$. Pruebe que la familia $\mathcal{F}$ es base en $e_G$ para una topología de grupo en $G$.
    \end{excer}

    \begin{proof}
        
    \end{proof}

    \begin{excer}
        Sea $G$ un grupo topológico.
        \begin{enumerate}
            \item Verifique que $G^*=G/\Cls{\left\{e_G \right\}}$ es un grupo topológico Hausdorff. Muestre que si $H$ es cualquier grupo Hausdorff y, $\cf{f}{G}{H}$ es un homomorfismo continuo, entonces existe un homomorfismo continuo $\cf{g}{G^*}{H}$ tal que $g\circ\pi=f$, donde $\cf{\pi}{G}{G^*}$ es el homomorfismo canónico. Comente este resultado.
            \item Sea $G_i$ el grupo $G$ con la topología indiscreta, y sea $\cf{i}{G}{G_i}$ la función identidad. Verifique que la función $\cf{\pi\Delta i}{G}{G^*\times G:i}$, dada por:
            \begin{equation*}
                \pi\Delta i(g)=\left(\pi(g),i(g) \right)
            \end{equation*}
            es un isomorfismo topológico entre $G$ y su imagen $\pi\Delta i(G)$.
        \end{enumerate}
    \end{excer}

    \begin{proof}
        
    \end{proof}

    \begin{excer}
        Sea $H$ un subgrupo abierto y divisible de un grupo topológico abeliano $G$. Demuestre que $G$ es topológicamente isomorfo a $H\times G/H$ (note que $G/H$ es un grupo discreto).
    \end{excer}

    \begin{proof}
        
    \end{proof}

    \begin{excer}
        Sea $G$ un grupo abeliano libre de torsión. Muestre que si $g$ y $g$ son elementos distintos de $G$ entonces, existe un homomorfismo $\phi$ de $G$ en $\mathbb{R}$ tal que $\phi(h)\neq\phi(g)$. Use este homomorfismo para definir una topología en $G$ que sea Hausdorff.
    \end{excer}

    \begin{proof}
        Vamos a probar que, en general, el resultado no es correcto. Considere al grupo
        \begin{equation*}
            G=\prod_{ i\in I}\mathbb{Z}
        \end{equation*}
        donde $I=\mathcal{P}\left(\mathbb{R}\right)$. Se tiene que $\abs{I}=\aleph_2$, por ende:
        \begin{equation*}
            \abs{G}=\abs{\prod_{ i\in I}\mathbb{Z}}=\abs{I}\cdot\abs{\mathbb{Z}}=\abs{I}\cdot\aleph_0=\abs{I}
        \end{equation*}
        Luego, $\abs{G}=\aleph_2$. Es claro que $G$ es abeliano. Veamos que es libre de torsión. En efecto, si $x=\left\{x_i \right\}_{ i\in I}\neq0$ es tal que $x_i\in\mathbb{Z}$, entonces como
        \begin{equation*}
            x_{\left\{0 \right\}}^m
        \end{equation*}
    \end{proof}

\end{document}