\documentclass[12pt]{report}
\usepackage[spanish]{babel}
\usepackage[utf8]{inputenc}
\usepackage{amsmath}
\usepackage{amssymb}
\usepackage{amsthm}
\usepackage{graphics}
\usepackage{subfigure}
\usepackage{lipsum}
\usepackage{array}
\usepackage{multicol}
\usepackage{enumerate}
\usepackage[framemethod=TikZ]{mdframed}
\usepackage[a4paper, margin = 1.5cm]{geometry}
\usepackage[mathscr]{euscript}

%En esta parte se hacen redefiniciones de algunos comandos para que resulte agradable el verlos%

\def\proof{\paragraph{Demostración:\\}}
\def\endproof{\hfill$\square$}
\renewcommand{\theenumii}{\roman{enumii}}
\renewcommand{\thechapter}{\Alph{chapter}}

%En esta parte se definen los comandos a usar dentro del documento para enlistar%

\newtheoremstyle{largebreak}
  {}% use the default space above
  {}% use the default space below
  {\normalfont}% body font
  {}% indent (0pt)
  {\bfseries}% header font
  {}% punctuation
  {\newline}% break after header
  {}% header spec

\theoremstyle{largebreak}

\newmdtheoremenv[
    leftmargin=0em,
    rightmargin=0em,
    innertopmargin=-2pt,
    innerbottommargin=8pt,
    hidealllines = true,
    roundcorner = 5pt,
    backgroundcolor = gray!60!red!30
]{exa}{Ejemplo}[section]

\newmdtheoremenv[
    leftmargin=0em,
    rightmargin=0em,
    innertopmargin=-2pt,
    innerbottommargin=8pt,
    hidealllines = true,
    roundcorner = 5pt,
    backgroundcolor = gray!50!blue!30
]{obs}{Observación}[section]

\newmdtheoremenv[
    leftmargin=0em,
    rightmargin=0em,
    innertopmargin=-2pt,
    innerbottommargin=8pt,
    rightline = false,
    leftline = false
]{theor}{Teorema}[section]

\newmdtheoremenv[
    leftmargin=0em,
    rightmargin=0em,
    innertopmargin=-2pt,
    innerbottommargin=8pt,
    rightline = false,
    leftline = false
]{propo}{Proposición}[section]

\newmdtheoremenv[
    leftmargin=0em,
    rightmargin=0em,
    innertopmargin=-2pt,
    innerbottommargin=8pt,
    rightline = false,
    leftline = false
]{cor}{Corolario}[section]

\newmdtheoremenv[
    leftmargin=0em,
    rightmargin=0em,
    innertopmargin=-2pt,
    innerbottommargin=8pt,
    rightline = false,
    leftline = false
]{lema}{Lema}[section]

\newmdtheoremenv[
    leftmargin=0em,
    rightmargin=0em,
    innertopmargin=-2pt,
    innerbottommargin=8pt,
    roundcorner=5pt,
    backgroundcolor = gray!30,
    hidealllines = true
]{mydef}{Definición}[section]

\newmdtheoremenv[
    leftmargin=0em,
    rightmargin=0em,
    innertopmargin=-2pt,
    innerbottommargin=8pt,
    roundcorner=5pt
]{excer}{Ejercicio}[section]

%En esta parte se colocan comandos que definen la forma en la que se van a escribir ciertas funciones%

\newcommand\abs[1]{\ensuremath{\lvert#1\rvert}}
\newcommand\divides{\ensuremath{\bigm|}}
\newcommand{\eul}[1]{\ensuremath{\mathscr{#1}}}
\renewcommand{\theenumi}{\arabic{enumi})}
\newcommand{\Int}[1]{\text{Int}\ensuremath{#1}}
\newcommand{\cf}[3]{\ensuremath{#1:#2\rightarrow#3}}
\newcommand{\natint}[1]{\ensuremath{\left[\big|#1\big|\right]}}
\newcommand{\coord}[1]{\ensuremath{\textup{coord}\left(#1\right)}}

%recuerda usar \clearpage para hacer un salto de página

\begin{document}
    \title{Grupos Topológicos}
    \author{Cristo Daniel Alvarado}
    \maketitle

    \tableofcontents %Con este comando se genera el índice general del libro%

    %\setcounter{chapter}{3} %En esta parte lo que se hace es cambiar la enumeración del capítulo%
    
    \setcounter{chapter}{1}

    \chapter{Topología}
    
    \section{Espacios Topológicos}

    En esta parte se hará un breve recordatorio de los resultados más relevantes de la parte de espacios topológicos.

    \begin{mydef}
        Un \textbf{espacio topológico} es una pareja $(X,\tau)$ que consiste en un conjunto $X$ y una familia $\tau$ de subconjuntos de $X$ con las siguientes propiedades:
        \begin{enumerate}
            \item $\emptyset, X\in \tau$.
            \item Si $U_1,U_2\in \tau$, entonces $U_1\cap U_2\in \tau$.
            \item Si $\eul{F}\subseteq \tau$, entonces
            \begin{equation*}
                \bigcup_{F\in \eul{F}}F\in\tau
            \end{equation*}
        \end{enumerate}
    \end{mydef}

    A los miembros de $\tau$ se les conoce como \textbf{conjuntos abiertos} en $X$. La familia $\tau$ es una \textbf{topología} en $X$.

    \begin{mydef}
        Sea $X$ un espacio topológico y $x\in X$. Si $U$ es un subconjunto abierto de $X$ tal que $x\in U$, diremos que \textbf{$U$ es una vecindad de $x$}.
    \end{mydef}

    Como resultado de lo anterior, se tiene que un subconjunto $V\subseteq X$ es abierto si para todo $x\in V$ existe una vecindad $U_x$ contenida en $V$.

    \begin{mydef}
        Sea $X$ un espacio topológico. Una \textbf{base} del espacio topológico $X$ es una familia $\eul{B}\subseteq\tau$ tal que todo subconjunto abierto no vacío de $X$ es unión de elementos de $\eul{B}$.
    \end{mydef}

    \begin{propo}
        Sea $X$ un espacio topológico. Una familia $\eul{B}\subseteq\tau$ es una base del espacio si y sólo si para todo punto $x\in X$ y para cualquier vecindad $V$ de $x$ existe $U\in\eul{B}$ tal que $x\in U\subseteq V$.
    \end{propo}

    El objetivo de la base de un espacio topológico es la de disminuir el número de elementos de la familia $\tau$, y de que esta familia más pequeña cumple propiedaes más generales que, resultan útiles para resultados posteriores.

    \begin{propo}
        Sea $X$ un espacio topológico. Una base $\eul{B}$ de $X$ tiene las propiedades siguientes:
        \renewcommand{\theenumi}{B\arabic{enumi})}
        \begin{enumerate}
            \item Para cualesquier $U_1, U_2\in\eul{B}$ y todo punto $x\in U_1\cap U_2$ existe un $U\in\eul{B}$ tal que $x\in U\subseteq U_1\cap U_2$.
            \item Para todo $x\in X$ existe $U\in\eul{B}$ tal que $x\in U$, es decir $X=\bigcup_{B\in\eul{B}}B$.
        \end{enumerate}
        \renewcommand{\theenumi}{\arabic{enumi})}
        Además, si una familia $\eul{B}$ de subconjuntos de $X$ cumple B1) y B2), entonces existe una única topología $\tau$ en $X$ para la cual $\eul{B}$ es una base.
    \end{propo}

    \begin{mydef}
        Si $(X,\tau)$ es un espacio topológico que posee una base numerable $\eul{B}$, se dice que $X$ es \textbf{segundo numerable}.
    \end{mydef}

    Una familia $\eul{P}\subseteq\tau$ es una \textbf{sub-base} de un espacio topológico $(X,\tau)$ si la familia de todas las intersecciones finitas $U_1\cap U_2\cap\cdots\cap U_k$, donde $U_i\in\eul{P}$ para $i=1,\dots,k$, es una base de $(X,\tau)$.\

    \begin{mydef}
        Una familia $\eul{B}(x)$ de vecindades de $x$ es una \textbf{base local} en $x\in X$ en el espacio topológico $(X,\tau)$, si para toda vecindad $V$ de $x$ existe $U\in\eul{B}(x)$ tal que $x\in U\subseteq V$.
    \end{mydef}

    Observe que si $\eul{B}$ es una base de $(X,\tau)$, la familia $\eul{B}(x)$ consistente en todos los elementos de $\eul{B}$ que contienen a $x$ es una base local para $x$ en $(X,\tau)$. Por otro lado, si para todo $x\in X$ contamos con una base local $\eul{B}(x)$ para $x$, enotnces $\eul{B}=\bigcup_{x\in X}\eul{B}(x)$ es un base de $(X,\tau)$.

    \begin{mydef}
        Sea $(X,\tau)$ un espacio topológico y supongamos que para todo $x\in X$ tenemos una base local $\eul{B}(x)$ en $x$; la familia
        \begin{equation*}
            \left\{\eul{B}(x) \big| x\in X \right\}
        \end{equation*}
        es un \textbf{sistema de vecindades} para el espacio topológico $(X,\tau)$.
    \end{mydef}

    \begin{propo}
        Sea $X$ un espacio topológico. Entonces, cualquier sistema de vecindades para el espacio $X$ tiene las siguientes propiedades:
        \renewcommand{\theenumi}{BP\arabic{enumi})}
        \begin{enumerate}
            \item Para toda $x\in X$, $\eul{B}(x)\neq\emptyset$ y para toda $U\in\eul{B}(x)$, $x\in U$.
            \item Si $U_1\in\eul{B}(x)$, $U_2\in\eul{B}(y)$ y $z\in U_1\cap U_2$, existe un $U\in \eul{B}(z)$ tal que $U\subseteq U_1\cap U_2$.
        \end{enumerate}
        \renewcommand{\theenumi}{B\arabic{enumi})}
    \end{propo}

    \begin{mydef}
        Si $(X,\tau)$ es un espacio topológico tal que todo punto $x\in X$ posee una base local en $x$ numerable, decimos que $X$ es un espacio \textbf{primero numerable}. 
    \end{mydef}

    \begin{mydef}
        Sea $(X,\tau)$ un espacio topológico. Decimos que un subconjunto $F\subseteq X$ es \textbf{cerrado}, si $X\backslash F\in \tau$, es decir, si su complemento relativo a $X$ es abierto. De forma inmediata se deducen las propiedades siguientes:
        \renewcommand{\theenumi}{C\arabic{enumi})}
        \begin{enumerate}
            \item El conjunto $X$ es cerrado, lo mismo con $\emptyset$.
            \item La unión de dos conjuntos cerrados es cerrada.
            \item La intersección de cualquier familia de conjuntos cerrados es cerrada.
        \end{enumerate}
        \renewcommand{\theenumi}{B\arabic{enumi})}
    \end{mydef}

    De ahora en adelante, cada que se mencione al conjunto $X$, se entenderá que es el espacio topológico $(X,\tau)$. Si no hay ambiguedad, no se mencionará la topología $\tau$.

    Ahora se procederá a definir dos conjuntos importantes para todo subconjunto $A\subseteq X$, con el objetivo de relacionar a éste con algún elemento de la topología $\tau$.

    \begin{mydef}
        Sea $A\subseteq X$. Considere la familia $\mathcal{C}_A$ de todos los conjuntos cerrados que contienen a $A$. La intersección
        \begin{equation*}
            \overline{A}=\bigcap_{E\in\mathcal{C}_A}E
        \end{equation*}
        es la \textbf{cerradora} o \textbf{clausura de $A$}. Es claro que $\overline{A}$ es un conjunto cerrado.
    \end{mydef}

    \begin{propo}
        Sea $X$ un espacio topológico; enotnces
        \renewcommand{\theenumi}{\arabic{enumi})}
        \begin{enumerate}
            \item $A\subseteq \overline{A}$.
            \item Si $A\subseteq B$, entonces $\overline{A}\subseteq\overline{B}$.
            \item Si $x\in\overline{A}$, entonces para toda vecindad $U$ de $x$ se cumple que
            \item $U\cap A\neq\emptyset$.
            \item $\overline{\emptyset}=\emptyset$.
            \item $\overline{A\cup B}=\overline{A}\cup\overline{B}$.
            \item $\overline{\overline{A}}=\overline{A}$.
        \end{enumerate}
    \end{propo}

    \begin{mydef}
        El \textbf{interior} de un subconjunto $A\subseteq X$ de un espacio topológico $X$ es la unión de todos los subconjuntos abiertos contenidos en $A$, o en forma equivalente, el abierto más grande contenido en $A$. El interior de $A$ se denota como $\Int{A}$ y es claramente un conjunto abierto.
    \end{mydef}

    Algunas propiedades del interior de un conjunto son las siguientes:

    \begin{propo}
        Sea $X$ un espacio topológico, enotnces
        \begin{enumerate}
            \item Para todo $A\subseteq X$ se cumple que $\Int{A}=X\backslash\overline{X\backslash A}$.
            \item $\Int{X}=X$.
            \item $\Int{A}\subseteq A$.
            \item $\Int{A\cap B}=\Int{A}\cap\Int{B}$.
            \item $\Int{\Int{A}}=\Int{A}$.
        \end{enumerate}
    \end{propo}

    Bajo esta perspectiva, podemos considerar a la cerradura e interior de un conjunto como operadores que actúan sobre los subconjuntos de un espacio topológico. Ahora definiremos un operador más en un espacio topológico:

    \begin{mydef}
        Un punto $x\in X$ de un espacio topológico $X$ es un \textbf{punto de acumulación} de un conjunto $A\subseteq X$, si $x\in\overline{A\backslash\left\{x\right\}}$; el conjunto de todos los puntos de acumulación de $A$ es el \textbf{conjunto derivado} de $A$ y se denota por $A^d$.

        A los puntos de $A^d$ se les conoce como \textbf{puntos no aislados} del conjunto $A$. Un punto $x$ es \textbf{aislado} en $X$ si y sólo si el conjunto $\left\{x\right\}$ es abierto.
    \end{mydef}

    Algunas de las propiedades del conjunto derivado son las siguientes:

    \begin{propo}
        Sean $X$ un espacio topológico, $x\in X$ y $A\subseteq X$. Entonces,
        \renewcommand{\theenumi}{D\arabic{enumi})}
        \begin{enumerate}
            \item El punto $x$ pertenece a $A^d$ si y sólo si toda vecindad de $x$ contiene al menos un punto de $A$ distinto de $x$.
            \item $\overline{A}=A\cup A^d$.
            \item Si $A\subseteq B$, entonces $A^d\subseteq B^d$.
            \item $(A\cup B)^d=A^d\cup B^d$.
            \item $\bigcup_{i\in I} A_i^d\subseteq \left(\bigcup_{i\in I}A_i\right)^d$.
        \end{enumerate}
    \end{propo}

    Ahora se definirán varios conceptos importantes en la topología general.

    \begin{mydef}
        Sea $X$ un espacio topológico.
        \renewcommand{\theenumi}{\arabic{enumi})}
        \begin{enumerate}
            \item Un conjunto $A\subseteq X$ es \textbf{denso} en $X$ si $\overline{A}=X$.
            \item Un conjunto $A\subseteq X$ es \textbf{denso en ninguna parte} en $X$, si $X\backslash \overline{A}$ es denso en $X$.
            \item Un conjunto $A\subseteq X$ es \textbf{denso en sí mismo} si $A=A^d$.
        \end{enumerate}
    \end{mydef}

    Entre las propiedades de subconjuntos de espacios relativas a la definición anterior se cuentan las siguientes:

    \begin{propo}
        Sea $(X,\tau)$ un espacio topológico. Entonces, se cumplen los 3 incisos siguientes:
        \begin{enumerate}
            \item Un conjunto $A$ es denso en $X$ si y sólo si todo subconjunto abierto no vacío de $X$ interseca a $A$.
            \item El conjunto $A$ es denso en ningun a parte en $X$ si y sólo si todo abierto no vacío de $X$ contiene un abuerto no vacío ajeno a $A$.
            \item Si $A$ es denso en $X$, entonces para todo abierto $U\subseteq X$ tenemos que $\overline{U}=\overline{U\cap A}$.
        \end{enumerate}
    \end{propo}

    \begin{mydef}
        Decimos que un espacio $X$ es \textbf{separable} si contiene un conjunto denso numerable.
    \end{mydef}

    \section{Funciones y homeomorfismos}

    Ahora, definiremos dos conceptos de gran transendencia en la topología general: la continuidad de funciones y el homeomorfismo.

    \begin{mydef}
        Sean $(x,\tau)$ y $(Y,\sigma)$ espacios topológicos; una función de $X$ en $Y$ es \textbf{continua} si $f^{-1}(U)\in\tau$, para todo $U\in\sigma$, es decir que la imagen inversa de todo abierto en $Y$ es abierta en $X$.
    \end{mydef}

    A continuación se presentan varios criterios de continuidad:

    \begin{propo}
        Para una función $\cf{f}{X}{Y}$ de un espacio $X$ en $Y$, las siguientes afirmaciones son equivalentes:
        \begin{enumerate}
            \item La función $f$ es continua.
            \item Las imágenes inversas de miembros de una sub-base $\eul{P}$ en $Y$ son abiertas en $X$.
            \item Las imágenes inversas de miembros de una base $\eul{B}$ en $Y$ son abiertas en $X$.
            \item Imágenes inverssa de conjuntos cerrados en $Y$ son cerados en $Y$.
            \item Para todo $A\subseteq X$ tenemos que $f(\overline{A})\subseteq\overline{f(A)}$.
            \item Para todo $B\subseteq Y$ tenemos que $f^{-1}(\Int{B})\subseteq\Int{f^{-1}(B)}$.
            \item Para todo $B\subseteq Y$ tenemos que $\overline{f^{-1}(B)}\subseteq f^{-1}(\overline{B})$.
        \end{enumerate}
    \end{propo}

    Una función $f$ del espacoi $X$ al espacio $Y$ es \textbf{continua en el punto $x\in X$} si para toda vecindad $V$ de $f(x)$ existe una vecindad $U$ de $x$ tal que $f(U)\subseteq V$. Claramente, una función $f$ de $X$ a $Y$ es continua si y sólo si es continua en cada punto $x\in X$.

    \begin{mydef}
        Un subconjunto de $G$ de un espacio $X$ es un \textbf{conjunto $G_\delta$} si $G$ es la intersección de una familia numerable de conjuntos abiertos en $X$.

        Un subconjunto $F$ de un espacio $X$ es un conjunto \textbf{conjunto $F_\sigma$} si $F$ es la unión de una familia numerable de cerrados en $X$.
    \end{mydef}

    \begin{obs}
        Observe que si $f$ es una función continua de $X$ a $Y$, entonces para cualquier conjunto $B\subseteq Y$ que sea $G_\delta$ (respectivamente, $F_\sigma$), la imagen inversa $f^{-1}(B)$ es un $G_\delta$ (respectivamente, $F_\sigma$) en $X$.
    \end{obs}

    \begin{mydef}
        Una función continua $\cf{f}{X}{Y}$ es \textbf{cerrada} (respectivamente, \textbf{abierta}) si para todo subconjunto cerrado (respectivamente, abierto) $A\subseteq X$, su imagen directa $f(A$) es cerrada (respectivamente, cerrada) en $Y$.
    \end{mydef}

    Ahora analizaremos la clase de funciones continuas que son homeomorfismos.
    
    \begin{mydef}
        Una función continua $\cf{f}{X}{Y}$ es un \textbf{homeomorfismo} si $f$ es una biyección y su inversa $\cf{f^{-1}}{Y}{X}$ es continua.
    \end{mydef}

    Si existe un homeomorfismo entre los espacios $X$ y $Y$, diremos que $X$ es \textbf{homeomorfo} a $Y$. La presencia de tal homeomorfismo hace topológicamente indistinguibles a los espacios $X$ y $Y$. Note que la relación de ser homeomorfos es una relación de equivalencia en la clase de todos los espacios topológicos.

    Otra clase importante de funciones la conforman las llamadas funciones cociente, que se definen a continuación.

    \begin{mydef}
        Una función $f$ de un espacio topológico $(X,\tau)$ sobre otro espacio topológico $(Y,\sigma)$ es una \textbf{función cociente} si un conjunto $V$ es abuerto en $Y$ si y sólo si su imagen inversa $f^{-1}(V)$ es abierta en $X$.
    \end{mydef}

    \begin{obs}
        Es inmediato que toda función cociente biyectiva es un homeomorfismo.
    \end{obs}

    \begin{propo}
        Sean $X$ y $Y$ espacios topológicos. Entonces
        \begin{enumerate}
            \item Toda función continua y abierta de $X$ sobre $Y$ es cociente.
            \item Toda función cerrada de $X$ sobre $Y$ es cociente.
        \end{enumerate}
    \end{propo}

    \section{Axiomas de separación}

    Por si sola, la definición de espacio topológico es muy general y pocos son los resultados interesantes que se pueden probar en todo espacio topológico. Para solventar este problema, se requiere de imponer ciertas reestricciones para que se puedan desarrollar resultados más interesantes. Básicamente lo que sigue es para establecer formas de separar puntos y conjuntos cerrados en espacios. 

    \begin{mydef}
        Espacios $T_i$:
        \begin{enumerate}
            \item Un espacio topológico $X$ es un \textbf{Espacio $T_0$} si para toda pareja de puntos distintos $x_1,x_2\in X$ existe un abierto que contiene uno de los puntos pero no el otro.
            \item Un espacio $X$ es un \textbf{Espacio $T_1$} si para toda pareja de puntos distintos $x_1,x_2\in X$ existe un abierto $U$ tal que $x_1\in U$ y $x_2\notin U$. Obseve que también existe un abierto $V$ que contiene a $x_2$ pero no a $x_1$.
            
            La diferencia entre un espacio $T_0$ y $T_1$ radica en que, en los $T_0$ no puedes elegir cual punto cumple la condición de estar en el abierto, y en el $T_1$ esto es posible.
            \item Un espacio $X$ es un \textbf{espacio $T_2$} o \textbf{Haussdorff} si para toda pareja de puntos distintos $x_1,x_2\in X$ existen conjuntos abiertos ajenos $U_1$ y $U_2$ en $X$ tales que $x_1\in U_1$ y $x_2\in U_2$.
        \end{enumerate}
    \end{mydef}

    \begin{lema}
        Para cualquier pareja de funciones continuas $f,g$ de un espacio $X$ en un espacio Hausdorff $Y$, el conjunto
        \begin{equation*}
            \left\{x\in X\big|f(x)=g(x) \right\}
        \end{equation*}
        es cerrado en $X$.
    \end{lema}

    \begin{mydef}
        Un espacio $X$ es $T_3$ o \textbf{regular} si es $T_1$ y para toda $x\in X$ y todo cerrado $F\subseteq X$ tal que $x\notin F$ existen abiertos $U_1,U_2$ en $X$ tales que $x\in U_1$, $F\subseteq U_2$ y $U_1\cap U_2=\emptyset$.
    \end{mydef}

    Claramente todo espacio regular es Hausdorff. No obstante, los espacios regulares tienen una propiedad aún más importante.\

    \begin{propo}
        Si $X$ es un espacio $T_1$, entonces $X$ es regular si y sólo si para todo punto $x\in X$ y toda vecindad $V$ de $x$ existe una vecindad $U$ de $x$ tal que $\overline{U}\subseteq V$.
    \end{propo}

    Una clase aún más particular de espacios topológicos son los espacios Tikhonov.

    \begin{mydef}
        Un espacio $X$ es un espacio $T_{3\frac{1}{2}}$ o \textbf{completamente regular} si para todo $x\in X$ y para todo cerrado $F\subseteq X$ tal que $x\notin F$ existe una función continua $\cf{f}{X}{[0,1]}$ con la propiedad de que $f(x)=0$ y $f(y)=1$ para todo $y\in F$. Si además el espacio $X$ es $T_1$, entonces $X$ es \textbf{completamente regular} o \textbf{Tikhonov}.
    \end{mydef}

    \begin{theor}
        Todo espacio Tikhonov es regular.
    \end{theor}

    \begin{mydef}
        Un espacio topológico $X$ es un espacio $T_4$ o \textbf{normal} si $X$ es $T_1$ y para toda pareja de cerrados ajenos $A,B\subseteq X$ existen abiertos ajenos $U_A$ y $U_B$ tales que $A\subseteq U_A$ y $B\subseteq U_B$.
    \end{mydef}

    \begin{obs}
        Si un espacio $X$ es $T_1$, entonces es normal si y sólo si para todo conjunto cerrado $F\subseteq X$ y todo abierto $V\subseteq X$ que contiene a $F$ existe un abierto $U\subseteq X$ tal que $F\subseteq U\subseteq \overline{U}\subseteq V$.Claramente todo espacio normal es regular. El hecho de que todo espacio normal es Tikhonov se deduce del siguiente teorema.
    \end{obs}

    \begin{theor}[Lema de Urysohn]
        Para cada pareja de subconjuntos cerrados ajenos $A,B$ de un espacio normal $X$ existe una función continua $\cf{f}{X}{[0,1]}$ tal que $f(x)=0$ para todo $x\in A$ y $f(x)=1$ para todo $x\in B$.
    \end{theor}
    
    \begin{obs}
        Todo espacio segundo numerable y regular es normal. Lo mismo ocurre con todo espacio regular y numerable.
    \end{obs}

    \begin{cor}
        Un subconjunto $A$ de un espacio normal $X$ es un cerrado $G_\delta$ si y sólo si existe una función continua $\cf{f}{X}{[0,1]}$ tal que $A=f^{-1}(0)$.
    \end{cor}

    \begin{mydef}
        \begin{itemize}
            \item Dos subconjuntos $A,B$ de un espacio topológico $X$ están \textbf{completamente separados} si existe una función continua $\cf{f}{X}{[0,1]}$ tal que $f(x)=0$ para $x\in A$ y $f(x)=1$ para $x\in B$.
            \item Un subconjunto $A$ de un espacio $X$ es \textbf{funcionalmente cerrado} (también llamado \textbf{conjunto nulo}) si $A=f^{-1}(0)$ para alguna función continua $\cf{f}{X}{[0,1]}$. El complemento de un conjunto funcionalmente cerrad oes un conjunto \textbf{funcionalmente abierto} (también llamado \textbf{conjunto cocero}).
            \item Un espacio topológico $X$ es \textbf{perfectamente normal} si $X$ es normal y todo subconjunto cerrado de $X$ es un $G_\delta$. Claramente un espacio normal $X$ es perfectamente normal si y sólo si todo subconjunto abierto de $X$ es un $F_\sigma$.
        \end{itemize}
    \end{mydef}

    \section{Convergencia}

    El concepto de sucesión convergente es de los más antiguos en topología general. Sin embargo, tiene carencias el enunciar el concepto de un espacio topológico simplemente en estos términos, pues no resulta la definición general. Sin embargo, este concepto puede definirse en cualquier espacio topológico. A continuación se verá la definición del mismo.

    \begin{mydef}
        Una sucesión $\left\{x_n\right\}_{n=1}^{\infty}$ de puntos de un espacio $X$ \textbf{converge a un punto $x\in X$} si para cada vecindad $V$ de $x$ existe $m\in\mathbb{N}$ tal que $x_k\in V$, para todo $k\geq m$.
    \end{mydef}

    Una forma de generalizar el concepto de convergencia de sucesiones es por medio de filtros.

    \begin{mydef}
        Sea $A$ un conjunto y $\mathcal{F}$ una familia no vacía de subconjuntos de $A$. La familia $\mathcal{F}$ es dice que es un \textbf{filtro} cuando:
        \begin{enumerate}
            \item $\emptyset\notin\mathcal{F}$.
            \item Si $A_1,A_2\in\mathcal{F}$, entonces $A_1\cap A_2\in\mathcal{F}$.
            \item Si $A\in\mathcal{F}$ y $A\subseteq B$, entonces $B\in\mathcal{F}$.
        \end{enumerate}
        Un filtro $\mathcal{F}$ en $A$ es un \textbf{filtro maximal} o un \textbf{ultrafiltro} si para todo filtro $\mathcal{F}'$ en $A$ tal que $\mathcal{F}\subseteq\mathcal{F}'$ se cumple que $\mathcal{F}=\mathcal{F}'$.

        Una familia no vacía de subconjuntos $\mathcal{B}$ de un conjunto $A$ es una \textbf{base de filtro} si $\emptyset\notin\mathcal{B}$ y para toda pareja $A,B\in\mathcal{B}$ existe $C\in\mathcal{B}$ tal que $C\subseteq A\cap B$.
    \end{mydef}

    \begin{mydef}
        Sea $\mathcal{F}$ un filtro de subconjuntos de un espacio topológico $X$. Un punto $x\in X$ es un \textbf{punto límite} de $\mathcal{F}$ si toda vecindad de $x$ pertenece a $\mathcal{F}$. En tal caso, decimos que $\mathcal{F}$ \textbf{converge a $x$}. La convergencia de una base filtro $\mathcal{B}$ a un punto $x\in X$ se define de forma similar.

        Un punto $x\in X$ es un \textbf{punto de adherencia} de un filtro $\mathcal{F}$ (de una base de filtro $\mathcal{B}$) si toda vecindad de $x$ interseca todos los elementos de $\mathcal{F}$ (todos los elementos de $\mathcal{B}$).
    \end{mydef}

    En la siguiente propsición usaremos como $\lim\mathcal{F}$ al conjunto formado por todos los puntos límite de un filtro $\mathcal{F}$.

    \begin{propo}
        El punto $x\in X$ pertenece a la cerradura $\overline{A}$, donde $A$ es un subconjunto de un espacio $X$ si y sólo si existe una base filtro consistente de subconjuntos de $A$ que converge a $X$

        Una función $f$ de un espacio topológico $X$ a un espacio topológico $Y$ es continua si y sólo si para cada base filtro $\mathcal{G}$ en el espacio $X$ y la base filtro
         \begin{equation*}
            f(\mathcal{G})=\left\{f(A)|A\in\mathcal{G} \right\}
         \end{equation*}
         en el espacio $Y$ se cumple que
         \begin{equation*}
            f(\lim\mathcal{G})=\lim f(\mathcal{G})
         \end{equation*}
         
         Un espacio topológico $X$ es Hausdorff si y spolo si todo filtro en $X$ tiene a lo más un punto límite.
    \end{propo}

    El problema de la definición de espacio topológico en términos de sucesiones es que, no todo espacio topológico puede describirse de esta manera. Por lo cual se definirán dos tipos de espacios que sí dependen, en su definición, de cierto tipo de convergencia.

    \begin{mydef}
        Un espacio topológico $X$ es \textbf{secuencial} si un conjunto $A\subseteq X$ es cerrado si y sólo si junto con cualquier sucesión $A$ contiene sus puntos límite.

        Un espacio $X$ es \textbf{Fréchet-Urysohn} si para todo $A\subseteq  X$ y todo $x\in\overline{A}$ existe una sucesión $\left\{x_n\right\}_{n=1}^{\infty}$ de puntos de $A$ que converge a $x$.
    \end{mydef}

    Es fácil ver que todo espacio primero numerable es Fréchet-Urysohn, y estos últimos son secuenciales.

    \section{Subespacios}

    Todo espacio tooplógico da lugar a numerosos espacios nuevos: cualquier subconjunto del espacio se puede ver como un espacio con la topología definida a continuación.

    \begin{mydef}
        Sean $(X,\tau)$ un espacio topológico y $M\subseteq X$. En $M$ consideramos la familia $\left\{U\cap M|U\in\tau \right\}$, la cual define una topología en $M$. Visto con esta topología, $M$ se convierte en un espacio topológico y decimos que $M$ es un \textbf{subespacio} de $X$.
    \end{mydef}

    El concepto de subespacio da lugar a que ciertas propiedades topológicas se vuelvan relativas. Por ejemplo, si $X$ es un espacio y $Y$ un subespacio, un subconjunto $A\subseteq Y$ puede ser cerrado respecto a $Y$, pero no serlo resepcto a $X$. De hecho, $Y$ siempre es cerrado respecto a $Y$, pero puede no serlo respecto a $X$.

    Es fácil probar que si $X$ es un espacio y $Y$ es un subespacio de $X$, entonces la cerradura de un subconjunto $A\subseteq Y$ respecto a $Y$ es $\overline{A}^{Y}=\overline{A}^X\cap Y$.

    \begin{mydef}
        Una función $\cf{f}{X}{Y}$ es un \textbf{encaje homeomórfico} si $\cf{f}{X}{f(X)}$ es un homeomorfismo. Si para un espacio $X$ existe un encaje homeomórfico $\cf{f}{X}{Y}$, decimos que \textbf{$X$ se encaja en $Y$}.
    \end{mydef}

    A continuación se presenta un teorema de extensión de funciones continuas en espacios normales.

    \begin{theor}[Teorema de Tietze-Urysohn]
        Toda función continua de un subespacio cerrado $M$ de un espacio normal $X$ a $[0,1]$ (respectivamente, $\mathbb{R}$) se puede extender a una función continua ed $X$ a $[0,1]$ (respectivamente, $\mathbb{R}$).
    \end{theor}

    \section{Espacios Producto}

    Suponga que se tiene una familia de espacios $\left\{X_i|i\in I \right\}$; considere el producto cartesiano de la familia dado por:
    \begin{equation*}
        X=\prod_{i\in I}X_i
    \end{equation*}
    y una familia $\tau$ de subconjuntos $U\subseteq X$ tales que $U=\prod_{i\in I}U_i$, donde $U_i= X_i$, con excepción de a lo sumo un número finito de índices $i_1,...,i_n\in I$ y $U_{i_k}$ es abierto en $U_{i_k}$ para $k\in\natint{1,n}$. Se verifica que la familia $\tau$ es una base para una topología en $X$. Esta topología es llamada \textbf{topología de Tikhonov para el producto cartesiano $X$}. Los elementos de esta base se conocen como abiertos \textbf{básicos} y la base misma es la \textbf{base canónica} en $X$.

    Si $U=\prod_{i=1}^{\infty}U_i$ es un abierto básico $U\subseteq X$, definimos el conjunto $\coord{U}$ com el conjunto de índices $i\in I$ tales que $U_i\neq X$.

    Los elementos del producto $X$ son puntos de la forma $\left\{x_i|i\in I \right\}$, donde $x_i\in X_i$, para todo $i\in I$.

    Para $i\in I$.

    %Me quedé en la página 183

    \chapter{Funciones Cardinales}

    \section{nose}

\end{document}