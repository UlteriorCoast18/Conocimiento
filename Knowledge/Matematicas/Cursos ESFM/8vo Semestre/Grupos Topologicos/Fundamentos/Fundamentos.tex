\documentclass[12pt]{report}
\usepackage[spanish]{babel}
\usepackage[utf8]{inputenc}
\usepackage{amsmath}
\usepackage{amssymb}
\usepackage{amsthm}
\usepackage{graphics}
\usepackage{subfigure}
\usepackage{lipsum}
\usepackage{array}
\usepackage{multicol}
\usepackage{enumerate}
\usepackage[framemethod=TikZ]{mdframed}
\usepackage[a4paper, margin = 1.5cm]{geometry}
\usepackage[mathscr]{euscript}

%En esta parte se hacen redefiniciones de algunos comandos para que resulte agradable el verlos%

\def\proof{\paragraph{Demostración:\\}}
\def\endproof{\hfill$\square$}
\renewcommand{\theenumi}{\arabic{enumi})}
\renewcommand{\theenumii}{\roman{enumii}}

%En esta parte se definen los comandos a usar dentro del documento para enlistar%

\newtheoremstyle{largebreak}
  {}% use the default space above
  {}% use the default space below
  {\normalfont}% body font
  {}% indent (0pt)
  {\bfseries}% header font
  {}% punctuation
  {\newline}% break after header
  {}% header spec

\theoremstyle{largebreak}

\newmdtheoremenv[
    leftmargin=0em,
    rightmargin=0em,
    innertopmargin=-2pt,
    innerbottommargin=8pt,
    hidealllines = true,
    roundcorner = 5pt,
    backgroundcolor = gray!60!red!30
]{exa}{Ejemplo}[section]

\newmdtheoremenv[
    leftmargin=0em,
    rightmargin=0em,
    innertopmargin=-2pt,
    innerbottommargin=8pt,
    hidealllines = true,
    roundcorner = 5pt,
    backgroundcolor = gray!50!blue!30
]{obs}{Observación}[section]

\newmdtheoremenv[
    leftmargin=0em,
    rightmargin=0em,
    innertopmargin=-2pt,
    innerbottommargin=8pt,
    rightline = false,
    leftline = false
]{theor}{Teorema}[section]

\newmdtheoremenv[
    leftmargin=0em,
    rightmargin=0em,
    innertopmargin=-2pt,
    innerbottommargin=8pt,
    rightline = false,
    leftline = false
]{propo}{Proposición}[section]

\newmdtheoremenv[
    leftmargin=0em,
    rightmargin=0em,
    innertopmargin=-2pt,
    innerbottommargin=8pt,
    rightline = false,
    leftline = false
]{cor}{Corolario}[section]

\newmdtheoremenv[
    leftmargin=0em,
    rightmargin=0em,
    innertopmargin=-2pt,
    innerbottommargin=8pt,
    rightline = false,
    leftline = false
]{lema}{Lema}[section]

\newmdtheoremenv[
    leftmargin=0em,
    rightmargin=0em,
    innertopmargin=-2pt,
    innerbottommargin=8pt,
    roundcorner=5pt,
    backgroundcolor = gray!30,
    hidealllines = true
]{mydef}{Definición}[section]

\newmdtheoremenv[
    leftmargin=0em,
    rightmargin=0em,
    innertopmargin=-2pt,
    innerbottommargin=8pt,
    roundcorner=5pt
]{excer}{Ejercicio}[section]

%En esta parte se colocan comandos que definen la forma en la que se van a escribir ciertas funciones%

\newcommand\abs[1]{\ensuremath{\lvert#1\rvert}}
\newcommand\divides{\ensuremath{\bigm|}}
\newcommand{\cf}[3]{\ensuremath{#1:#2\rightarrow#3}}
\newcommand{\N}[1]{\ensuremath{\mathscr{N}(#1)}}
\newcommand{\Ns}[1]{\ensuremath{\mathscr{N}*(#1)}}
\newcommand{\piz}[1]{\ensuremath{\mathscr{#1}}}
\newcommand{\eul}[1]{\ensuremath{\mathbb{#1}}}

%recuerda usar \clearpage para hacer un salto de página

\begin{document}
    \title{Grupos Topológicos}
    \author{Cristo Daniel Alvarado}
    \maketitle

    \tableofcontents %Con este comando se genera el índice general del libro%

    %\setcounter{chapter}{3} %En esta parte lo que se hace es cambiar la enumeración del capítulo%
    
    \chapter{Elementos de la teoría de grupos topológicos}
    
    \section{Preliminares}

    \begin{mydef}
        Sea $G$ un conjunto no vacío dotado de una operación binaria (denotada por $\cdot$) y una familia $\tau$ de subconjuntos de $G$. $G$ es llamado \textbf{grupo topológico} si
        \begin{enumerate}
            \item $(G,\cdot)$ es un grupo.
            \item $(G,\tau)$ es un espacio topológico.
            \item Las funciones $\cf{g_1}{(G,\tau)\times (G,\tau)}{(G,\tau)}$ y $\cf{g_2}{(G,\tau)}{(G,\tau)}$ dadas por $(x,y)\mapsto x\cdot y$ y $x\mapsto x^{-1}$, respectivamente, son continuas, siendo $x^{-1}$ el inverso de $x$ en $G$.
        \end{enumerate}
    \end{mydef}

    Se denotará a la operación $\cdot$ por yuxtaposición, es decir $x\cdot y = xy$.

    \begin{obs}
        Una equivalencia de la condición (3) de la proposición anterior es la siguiente:

        Sea $G$ un grupo topológico. Denotamos por $\mathscr{N}(x)$ a \textbf{ la familia de todas las vecindades de $x\in G$}. 3) es equivalente a
        \begin{enumerate}
            \setcounter{enumi}{3}
            \item Si $x,y\in G$, entonces para cada $U\in \mathscr{N}(xy)$ existen vecindades $V\in \mathscr{N}(x)$ y $W\in \mathscr{N}(y)$ tales que $V\cdot W\subseteq U$, donde
            \begin{equation*}
                V\cdot W = \left\{vw \big| v\in V \textup{ \& } w\in W \right\}
            \end{equation*}
            y, para cada $U\in\N{x^{-1}}$ existe $V\in\N{x}$ tal que $V^{-1}\subseteq U$, siendo
            \begin{equation*}
                V^{-1}=\left\{v^{-1}\big| v\in V \right\}
            \end{equation*}
        \end{enumerate}

        esta equivalencia es inmediada de la definición de continuidad de una función en un espacio topológico.
    \end{obs}

    \begin{obs}
        El símbolo $e_G$ denotará siempre a la identidad de un grupo $G$.

        Con frecuencia se referirá al grupo topológico $G$, con operación binaria $\cdot$ y topología $\tau$ como la terna $(G,\cdot,\tau)$. Si no hay ambiguedad, se denotará simplemente por $G$.
    \end{obs}

    \begin{lema}
        Sean $(G,\cdot)$ un grupo, y $\tau$ una topología en $G$. Entonces, $(G,\cdot,\tau)$ es un grupo topológico si y sólo si la función
        \begin{equation*}
            \begin{split}
                g_3:(G,\tau)\times (G,\tau)&\rightarrow(G,\tau)\\
                (x,y)&\mapsto xy^{-1}\\
            \end{split}
        \end{equation*}
        es continua.
    \end{lema}

    \begin{proof}
        $\Rightarrow):$ Suponga que $G$ es un grupo topológico, entonces las funciones $g_1$ y $g_2$ son continuas (por la condición 3) de la definición anterior). Notemos que
        \begin{equation*}
            g_3=g_1(x,g_2(y)),\quad\forall x,y\in G
        \end{equation*}
        por ende, $g_3$ es continua.

        $\Leftarrow):$ Suponga que la función $g_3$ es continua. Notemos que
        \begin{equation*}
            g_2(x)=g_3(x,e_G),\quad\forall x\in G
        \end{equation*}
        por ser $g_3$ continua, se sigue que $g_2$ también lo es. Además
        \begin{equation*}
            g_1(x,y)=g_3(x,g_2(y)),\quad\forall x,y\in G
        \end{equation*}
        por lo cual, $g_1$ también es continua. Por tanto, $G$ es grupo topológico.
    \end{proof}

    Una de las primeras ventajas que surgen en el estudio de los grupos topológicos es que, ciertas propiedades locales se vuelven globales desde el punto de vista de la topología.

    \begin{theor}
        Sea $G$ un grupo topológico. Si $g\in G$ es un elemento fijo arbitrario, entonces las funciones $\varphi_g(x)=xg$ y $\sigma_g(x)=gx$, para todo $x\in G$, de $G$ en $G$, son homeomorfismos. La inversión $\cf{f}{G}{G}$, definida por $f(y)=y^{-1}$, también es un homeomorfismos. Las funciones $\varphi_g$ y $\sigma_g$ son llamadas \textbf{traslaciones por la derecha e izquierda}, respectivamente.
    \end{theor}

    \begin{proof}
        Por la definición de grupo topológico, las funciones $\varphi_g$, $\sigma_g$ y $f$ son continuas. Veamos que son homeomorfismos de $G$ en $G$.
        \begin{enumerate}
            \item Veamos que $\varphi_g$ es inyectiva. Si $a,b\in G$ son tales que $\varphi_g(a)=\varphi_g(b)$, entonces $ag = bg\Rightarrow a = b$, con lo que se tiene el resultado.
            
            Además es suprayectiva, pues para cada $b\in G$ existe $g^{-1}b\in G$ tal que $\varphi_g(bg^{-1})=b$.

            Luego, $\varphi$ es homeomorfismo de $G$, con inversa $\varphi_{g^{-1}}$. Además es homomorfismo.
            \item Para $\sigma_g$ el caso es similar a $\varphi_g$.
            \item Para $f$ el resultado es inmediato, pues es biyectiva, homomorfismo y su inversa es ella misma.
        \end{enumerate}
    \end{proof}

    Los resultados siguientes nos perimitirán estudiar las propiedades topológicas locales de un grupo topológico $G$ en un solo punto, que por simplificar siempre tomaremos como la identidad $e_G$ del grupo.

    \begin{cor}
        Todo grupo topológico $G$ es un espacio homogéneo.
    \end{cor}

    \begin{proof}
        Debemos probar que dados dos elementos arbitrarios del grupo topológico $G$, digamos $g,h\in G$, existe un homeomorfismo de $G$ sobre sí mismo tal que manda un elemento en el otro. Por el teorema anterior, tomando como homeomorfismo a $\varphi_{g^{-1}h}$ se tiene el resultado, pues $\varphi_{g^{-1}h}(g)=h$.
    \end{proof}

    Como en grupos y espacios topológicos, nos interesan las funciones que preservan las propiedades entre éstos. Por lo cual se estudiarán los siguientes tipos de funciones:

    \begin{mydef}
        Decimos que una función biyectiva $\cf{f}{G}{G'}$ entre dos grupos topológicos $G$ y $G'$ es un \textbf{isomorfismo topológico} si $f$ y $f^{-1}$ son homomorfismos continuos.

        Si $G=G'$, el isomorfismo $f$ se llama \textbf{automorfismo topológico}. dos grupos topológicos son \textbf{topológicamente isomorfos} si existe un isomorfismo topológico de uno al otro. Utilizaremos el símbolo $G\cong H$ para indicar que los grupos $G$ y $H$ son topológicamente isomorfos.
    \end{mydef}

    El objetivo del siguiente teorema es ver que un grupo topológico no abeliano admite muchos automorfismos.

    \begin{theor}
        Si $G$ es un grupo topológico y $a\in G$ está fijo, entonces la función $g(x)=axa^{-1}$ es un automorfismo topológico.
    \end{theor}

    \begin{proof}
        Observemos que $g(x)=\sigma_a(\varphi_{a}^{-1}(x))$, donde las dos funciones de la composición definidas como en el teorema anterior son homeomofismos, y por ende $g$ lo es. Además $g$ es homomorfismo ya que
        \begin{equation}
            g(xy)=axya^{-1}=\left(axa^{-1}\right)\left(aya^{-1}\right)=g(x)g(y)
        \end{equation}
        el cual es invertible, con inversa $f(x)=a^{-1}xa$.
    \end{proof}

    \begin{obs}
        En el caso de que el grupo $G$ sea abeliano, el automorfismo topológico $G$ definido en el teorema anterior, es trivial ya que coincide con la identidad.
    \end{obs}

    El siguiente resultado tiene como objetivo el describir la topología del grupo, que en este caso resulta más sencillo que describir la topología de un espacio topológico. Para ello, basta describir una base local para la identidad del grupo $e_G$.

    \begin{lema}
        Sea $G$ un grupo topológico, y sea $\N{e_G}$ una base local para la identidad del grupo $e_G$. Entonces las familias $\left\{xU\right\}$ y $\left\{Ux\right\}$, donde $x$ toma los valores en los elemntos de $G$ y $U$ varía sobre todos los elementos de $\N{e_G}$, son bases para la topología de $G$.
    \end{lema}

    \begin{proof}
        Sea $W$ un abieto no vacío de $G$ y $a\in G$ un elemento de $W$. Probaremos que existe un elemento $\hat{U}$ de alguna las familias descritas anteriormente tal que
        \begin{equation*}
            a\in\hat{U}\subseteq W
        \end{equation*}
        Considere la función $\cf{f}{G}{G}$, $x\mapsto a^{-1}x$. Esta función es un homeomorfismo, el cual transforma a $W$ en $a^{-1}W$. Notemos que $e_G\in a^{-1}W$, pues el elemento $e_G=a^{-1}a\in a^{-1}W$. Como $\N{e_G}$ es una base local de $e_G$, entonces existe $U\in\N{e_G}$ tal que
        \begin{equation*}
            e_G\in U \subseteq a^{-1}W
        \end{equation*}
        Por lo cual
        \begin{equation*}
            a\in aU \subseteq aa^{-1}W=W
        \end{equation*}
        Por tanto, tomando $\hat{U}=aU$ se tiene el resultado para la primera familia. Para la segunda se procede de forma análoga cambiando el orden del producto en la función $f$.
    \end{proof}

    El siguiente lema nos proporciona una base local para la identidad formada por vecindades tales que $V^{-1}=V$. Estas vecindades reciben el nombre de \textbf{simétricas}.

    \begin{lema}
        Sea $G$ un grupo topológico y $U\in\N{e_G}$, entonces existe $V\in\N{e_G}$ tal que $V^{-1}=V\subseteq U$. Por lo tanto, las vecindades simétricas de la identidad constituyen una base local de $e_G$.
    \end{lema}

    \begin{proof}
        Sean $U\in\N{e_G}$ y $\cf{f}{G}{G}$, $x\mapsto x^{-1}$. Como $f$ es un homeomorfismo de $G$ sobre $G$, entonces $f(U)=U^{-1}$ es abierto y $e_G\in U^{-1}$. Por lo cual $V=U\cap U^{-1}$ es abierto y $V^{-1}=V$ es tal que $e_G\in V\subseteq U$.
    \end{proof}

    En lo sucesivo denotaremos por $\Ns{e_G}$ a la base local de vecindades de $e_G$ que son abiertas y simétricas en un grupo topológico $G$.

    Otra propiedad importante de la identidad del grupo topológico $G$, es que admite una base local formada por subconjuntos cerrados.

    \begin{lema}
        Sea $G$ grupo topológico.
        \begin{enumerate}
            \item Si $U\in\N{e_G}$, entonces para cada $n\in\mathbb{N}^+$ existe $V\in\N{e_G}$ con $V^{n}\subseteq U$, donde
            \begin{equation*}
                V^n=\underbrace{V\cdots V}_{n\text{-veces}}
            \end{equation*}
            \item Si $U\in\N{e_G}$, entonces existe $V\in\N{e_G}$ con $\overline{V}\subseteq U$. En particular, las vecindades cerradas de $e_G$ constituyen una base local de la identidad $e_G$ cuyos elementos son subconjuntos cerrados.
        \end{enumerate}
    \end{lema}

    \begin{proof}
        De 1): Procederemos por inducción sobre $n$. Para $n=1$ el resultado es inmediato, pues tomando $V=U$ se sigue el resultado.

        Suponga el resultado cierto para algún $n\in\mathbb{N}^+$, entonces para $U$ existe $W\in\N{e_G}$ tal que $W^n\subseteq U$. Como la multiplicación es continua $g_1(x,y)=xy$, y $g_1(e_G,e_G)=e_G$, entonces para $W$ existen vecindades $V_1,V_2\in\N{e_G}$ tales que $f(V_1\times V_2)=V_1\cdot V_2\subseteq W$. Tomemos $V=V_1\cap V_2$, claro que $e_G\in V$, por lo cual $V\in\N{e_g}$ y, además:
        \begin{equation*}
            V^{n+1}= V\cdot V\cdot V^{n-1}\subseteq V_1\cdot V_2\cdot W^{n-1}\subseteq W\cdot W^{n-1}=W^n\subseteq U
        \end{equation*}
        Aplicando inducción se sigue el resultado.

        De 2): Por 1) y por el hecho de que $\Ns{e_G}$ es una base local de $e_G$, existe $V\in\Ns{e_G}$ tal que $V^2\subseteq U$. Si $x\in\overline{V}$, entonces como $xV$ es una vecindad de $x$, la intersección $xV\cap V\neq\emptyset$ (pues $x$ está en la adherencia de $V$), es decir, existen $v_1,v_2\in V$ tales que
        \begin{equation*}
            xv_1=v_2\Rightarrow x=v_2v_1^{-1}\in V\cdot V^{-1}=V^{2}\subseteq U
        \end{equation*}
        Por ende, $\overline{V}\subseteq U$.
    \end{proof}

    \begin{theor}
        Sea $G$ un grupo topológico, $a\in G$ y $A, B, O, M$ subconjuntos de $G$. Entonces
        \begin{enumerate}
            \item Si $O$ es abierto, entonces los conjuntos $aO$, $Oa$, $O^{-1}$, $MO$ y $OM$ son abiertos.
            \item Si $A$ es cerrado, entonces $aA, Aa, A^{-1}$ son conjuntos cerrados.
            \item Si $A$ y $B$ son compactos, también lo son $AB$ y $A^{-1}$.
            \item Se cumple que
            \begin{equation*}
                \overline{A}=\bigcap_{W\in\N{e_G}}AW=\bigcap_{W\in\N{e_G}}WA
            \end{equation*}
        \end{enumerate}
    \end{theor}

    \begin{proof}
        De 1): Por el teorema 1.1.1, $\varphi_a$, $\sigma_a$ y $f(x)=x^{-1}$ son homeomorfismos, para cualquier $a\in G$ fijo. Por lo tanto, si $O$ es abierto, entonces la imagen directa de $O$ bajo estas funciones (es decir, los conjuntos $aO, Oa, O^{-1}$) son abiertos. Para los dos últimos conjuntos, basta ver que
        \begin{equation*}
            \begin{split}
                MO =& \bigcup\left\{mO\big|m\in\mathbb{M} \right\}\\
                OM =& \bigcup\left\{Om\big|m\in\mathbb{M} \right\}\\
            \end{split}
        \end{equation*}
        por ser uniones arbitrarias de abiertos, los conjuntos $MO$ y $OM$ son abiertos.

        De 2): Es análogo a 1), usando el hecho de que los homomorfismos son aplicaciones cerradas.

        De 3): Notemos que $A\times B$ es compacto en el espacio topológico producto $G\times G$, por lo cual al ser $\cf{g_1}{G\times G}{G}$, $(x,y)\mapsto xy$ una función continua, se sigue que la imagen de este compacto $f(A\times B) = AB$ es compacto. De forma similar con $A^{-1}$ con la función $f(x)=x^{-1}$ se obtiene que $A^{-1}$ es compacto.

        De 4): Nuestro objetivo será intentar caracterizar a $AW$ y $WA$ (donde $W\in\N{e_G}$) antes de ver los elementos de la intersección. Sea $W\in\N{e_G}$, entonces existe un abierto $V\in\Ns{e_G}$ tal que $V\subseteq W$. Por 1) el producto $AV$ es abierto y $A\subseteq AV$ (pues $e_G\in V$).

        Además, $\overline{A}\subseteq AW$, pues si $x\in\overline{A}$, entonces $xV$ es una vecindad de $x$ y por lo tanto $xV\cap A\neq\emptyset$, así existen $v\in V$ y $a\in A$ tales que $xv=a\Rightarrow x=av^{-1}\in AV^{-1}=AV\subseteq AW$. Como el $W$ fue arbitrario, se sigue que
        \begin{equation*}
            \overline{A}\subseteq\bigcap_{W\in\N{e_G}}AW
        \end{equation*} 
        (de forma análoga con $\bigcap_{W\in\N{e_G}}WA$). Ahora, 
    \end{proof}

\end{document}