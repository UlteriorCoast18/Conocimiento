\documentclass[12pt]{report}
\usepackage[spanish]{babel}
\usepackage[utf8]{inputenc}
\usepackage{amsmath}
\usepackage{amssymb}
\usepackage{amsthm}
\usepackage{graphics}
\usepackage{subfigure}
\usepackage{lipsum}
\usepackage{array}
\usepackage{multicol}
\usepackage{enumerate}
\usepackage[framemethod=TikZ]{mdframed}
\usepackage[a4paper, margin = 1.5cm]{geometry}
\usepackage[mathscr]{euscript}

%En esta parte se hacen redefiniciones de algunos comandos para que resulte agradable el verlos%

\def\proof{\paragraph{Demostración:\\}}
\def\endproof{\hfill$\square$}
\renewcommand{\theenumi}{\arabic{enumi})}
\renewcommand{\theenumii}{\roman{enumii}}

%En esta parte se definen los comandos a usar dentro del documento para enlistar%

\newtheoremstyle{largebreak}
  {}% use the default space above
  {}% use the default space below
  {\normalfont}% body font
  {}% indent (0pt)
  {\bfseries}% header font
  {}% punctuation
  {\newline}% break after header
  {}% header spec

\theoremstyle{largebreak}

\newmdtheoremenv[
    leftmargin=0em,
    rightmargin=0em,
    innertopmargin=-2pt,
    innerbottommargin=8pt,
    hidealllines = true,
    roundcorner = 5pt,
    backgroundcolor = gray!60!red!30
]{exa}{Ejemplo}[section]

\newmdtheoremenv[
    leftmargin=0em,
    rightmargin=0em,
    innertopmargin=-2pt,
    innerbottommargin=8pt,
    hidealllines = true,
    roundcorner = 5pt,
    backgroundcolor = gray!50!blue!30
]{obs}{Observación}[section]

\newmdtheoremenv[
    leftmargin=0em,
    rightmargin=0em,
    innertopmargin=-2pt,
    innerbottommargin=8pt,
    rightline = false,
    leftline = false
]{theor}{Teorema}[section]

\newmdtheoremenv[
    leftmargin=0em,
    rightmargin=0em,
    innertopmargin=-2pt,
    innerbottommargin=8pt,
    rightline = false,
    leftline = false
]{propo}{Proposición}[section]

\newmdtheoremenv[
    leftmargin=0em,
    rightmargin=0em,
    innertopmargin=-2pt,
    innerbottommargin=8pt,
    rightline = false,
    leftline = false
]{cor}{Corolario}[section]

\newmdtheoremenv[
    leftmargin=0em,
    rightmargin=0em,
    innertopmargin=-2pt,
    innerbottommargin=8pt,
    rightline = false,
    leftline = false
]{lema}{Lema}[section]

\newmdtheoremenv[
    leftmargin=0em,
    rightmargin=0em,
    innertopmargin=-2pt,
    innerbottommargin=8pt,
    roundcorner=5pt,
    backgroundcolor = gray!30,
    hidealllines = true
]{mydef}{Definición}[section]

\newmdtheoremenv[
    leftmargin=0em,
    rightmargin=0em,
    innertopmargin=-2pt,
    innerbottommargin=8pt,
    roundcorner=5pt
]{excer}{Ejercicio}[section]

%En esta parte se colocan comandos que definen la forma en la que se van a escribir ciertas funciones%

\newcommand\abs[1]{\ensuremath{\lvert#1\rvert}}
\newcommand\divides{\ensuremath{\bigm|}}
\newcommand{\cf}[3]{\ensuremath{#1:#2\rightarrow#3}}
\newcommand{\N}[1]{\ensuremath{\mathscr{N}(#1)}}
\newcommand{\piz}[1]{\ensuremath{\mathscr{#1}}}
\newcommand{\eul}[1]{\ensuremath{\mathbb{#1}}}

%recuerda usar \clearpage para hacer un salto de página

\begin{document}
    \title{Grupos Topológicos}
    \author{Cristo Daniel Alvarado}
    \maketitle

    \tableofcontents %Con este comando se genera el índice general del libro%

    %\setcounter{chapter}{3} %En esta parte lo que se hace es cambiar la enumeración del capítulo%
    
    \chapter{Elementos de la teoría de grupos topológicos}
    
    \section{Preliminares}

    \begin{mydef}
        Sea $G$ un conjunto no vacío dotado de una operación binaria (denotada por $\cdot$) y una familia $\tau$ de subconjuntos de $G$. $G$ es llamado \textbf{grupo topológico} si
        \begin{enumerate}
            \item $(G,\cdot)$ es un grupo.
            \item $(G,\tau)$ es un espacio topológico.
            \item Las funciones $\cf{g_1}{(G,\tau)\times (G,\tau)}{(G,\tau)}$ y $\cf{g_2}{(G,\tau)}{(G,\tau)}$ dadas por $(x,y)\mapsto x\cdot y$ y $x\mapsto x^{-1}$, respectivamente, son continuas, siendo $x^{-1}$ el inverso de $x$ en $G$.
        \end{enumerate}
    \end{mydef}

    Se denotará a la operación $\cdot$ por yuxtaposición, es decir $x\cdot y = xy$.

    \begin{obs}
        Una equivalencia de la condición (3) de la proposición anterior es la siguiente:

        Sea $G$ un grupo topológico. Denotamos por $\mathscr{N}(x)$ a \textbf{ la familia de todas las vecindades de $x\in G$}. 3) es equivalente a
        \begin{enumerate}
            \setcounter{enumi}{3}
            \item Si $x,y\in G$, entonces para cada $U\in \mathscr{N}(xy)$ existen vecindades $V\in \mathscr{N}(x)$ y $W\in \mathscr{N}(y)$ tales que $V\cdot W\subseteq U$, donde
            \begin{equation*}
                V\cdot W = \left\{vw \big| v\in V \textup{ \& } w\in W \right\}
            \end{equation*}
            y, para cada $U\in\N{x^{-1}}$ existe $V\in\N{x}$ tal que $V^{-1}\subseteq U$, siendo
            \begin{equation*}
                V^{-1}=\left\{v^{-1}\big| v\in V \right\}
            \end{equation*}
        \end{enumerate}

        esta equivalencia es inmediada de la definición de continuidad de una función en un espacio topológico.
    \end{obs}

    \begin{obs}
        El símbolo $e_G$ denotará siempre a la identidad de un grupo $G$.

        Con frecuencia se referirá al grupo topológico $G$, con operación binaria $\cdot$ y topología $\tau$ como la terna $(G,\cdot,\tau)$. Si no hay ambiguedad, se denotará simplemente por $G$.
    \end{obs}

    \begin{lema}
        Sean $(G,\cdot)$ un grupo, y $\tau$ una topología en $G$. Entonces, $(G,\cdot,\tau)$ es un grupo topológico si y sólo si la función
        \begin{equation*}
            \begin{split}
                g_3:(G,\tau)\times (G,\tau)&\rightarrow(G,\tau)\\
                (x,y)&\mapsto xy^{-1}\\
            \end{split}
        \end{equation*}
        es continua.
    \end{lema}

    \begin{proof}
        $\Rightarrow):$ Suponga que $G$ es un grupo topológico, entonces las funciones $g_1$ y $g_2$ son continuas (por la condición 3) de la definición anterior). Notemos que
        \begin{equation*}
            g_3=g_1(x,g_2(y)),\quad\forall x,y\in G
        \end{equation*}
        por ende, $g_3$ es continua.

        $\Leftarrow):$ Suponga que la función $g_3$ es continua. Notemos que
        \begin{equation*}
            g_2(x)=g_3(x,e_G),\quad\forall x\in G
        \end{equation*}
        por ser $g_3$ continua, se sigue que $g_2$ también lo es. Además
        \begin{equation*}
            g_1(x,y)=g_3(x,g_2(y)),\quad\forall x,y\in G
        \end{equation*}
        por lo cual, $g_1$ también es continua. Por tanto, $G$ es grupo topológico.
    \end{proof}

    

\end{document}