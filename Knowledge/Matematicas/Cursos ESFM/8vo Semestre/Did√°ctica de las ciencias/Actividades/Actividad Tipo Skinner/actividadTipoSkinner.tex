\documentclass[12pt]{report}
\usepackage[spanish]{babel}
\usepackage[utf8]{inputenc}
\usepackage{amsmath}
\usepackage{amssymb}
\usepackage{amsthm}
\usepackage{graphics}
\usepackage{subfigure}
\usepackage{lipsum}
\usepackage{array}
\usepackage{multicol}
\usepackage{enumerate}
\usepackage[framemethod=TikZ]{mdframed}
\usepackage[a4paper, margin = 1.5cm]{geometry}

%En esta parte se hacen redefiniciones de algunos comandos para que resulte agradable el verlos%

\renewcommand{\theenumii}{\roman{enumii}}

\def\proof{\paragraph{Demostración:\\}}
\def\endproof{\hfill$\blacksquare$}

\def\sol{\paragraph{Solución:\\}}
\def\endsol{\hfill$\square$}

%En esta parte se definen los comandos a usar dentro del documento para enlistar%

\newtheoremstyle{largebreak}
  {}% use the default space above
  {}% use the default space below
  {\normalfont}% body font
  {}% indent (0pt)
  {\bfseries}% header font
  {}% punctuation
  {\newline}% break after header
  {}% header spec

\theoremstyle{largebreak}

\newmdtheoremenv[
    leftmargin=0em,
    rightmargin=0em,
    innertopmargin=-2pt,
    innerbottommargin=8pt,
    hidealllines = true,
    roundcorner = 5pt,
    backgroundcolor = gray!60!red!30
]{exa}{Ejemplo}[section]

\newmdtheoremenv[
    leftmargin=0em,
    rightmargin=0em,
    innertopmargin=-2pt,
    innerbottommargin=8pt,
    hidealllines = true,
    roundcorner = 5pt,
    backgroundcolor = gray!50!blue!30
]{obs}{Observación}[section]

\newmdtheoremenv[
    leftmargin=0em,
    rightmargin=0em,
    innertopmargin=-2pt,
    innerbottommargin=8pt,
    rightline = false,
    leftline = false
]{theor}{Teorema}[section]

\newmdtheoremenv[
    leftmargin=0em,
    rightmargin=0em,
    innertopmargin=-2pt,
    innerbottommargin=8pt,
    rightline = false,
    leftline = false
]{propo}{Proposición}[section]

\newmdtheoremenv[
    leftmargin=0em,
    rightmargin=0em,
    innertopmargin=-2pt,
    innerbottommargin=8pt,
    rightline = false,
    leftline = false
]{cor}{Corolario}[section]

\newmdtheoremenv[
    leftmargin=0em,
    rightmargin=0em,
    innertopmargin=-2pt,
    innerbottommargin=8pt,
    rightline = false,
    leftline = false
]{lema}{Lema}[section]

\newmdtheoremenv[
    leftmargin=0em,
    rightmargin=0em,
    innertopmargin=-2pt,
    innerbottommargin=8pt,
    roundcorner=5pt,
    backgroundcolor = gray!30,
    hidealllines = true
]{mydef}{Definición}[section]

\newmdtheoremenv[
    leftmargin=0em,
    rightmargin=0em,
    innertopmargin=-2pt,
    innerbottommargin=8pt,
    roundcorner=5pt
]{excer}{Ejercicio}[section]

%En esta parte se colocan comandos que definen la forma en la que se van a escribir ciertas funciones%

\newcommand\abs[1]{\ensuremath{\left|#1\right|}}
\newcommand\divides{\ensuremath{\bigm|}}
\newcommand\cf[3]{\ensuremath{#1:#2\rightarrow#3}}
\newcommand\natint[1]{\ensuremath{\left[\!\left[ #1\right]\!\right]}}
\newcommand{\afa}{\:
    \begin{tikzpicture}
        \draw [line width = 0.17 mm, black] (0,0) -- (-0.115,0.29);
        \draw [line width = 0.17 mm, black] (0,0) -- (0.115,0.29);
        \draw [line width = 0.17 mm, black] (-0.12,0) arc (190:-10:0.12cm);
    \end{tikzpicture}
    \:
}
%Este símvolo es para casi todo salvo una cantidad finita

%recuerda usar \clearpage para hacer un salto de página

\begin{document}
    \setlength{\parskip}{5pt} % Añade 5 puntos de espacio entre párrafos
    \setlength{\parindent}{12pt} % Pone la sangría como me gusta
    \title{Actividad Tipo Skinner}
    \author{Cristo Daniel Alvarado}
    \maketitle

    %\setcounter{chapter}{3} %En esta parte lo que se hace es cambiar la enumeración del capítulo%
    
    \setcounter{chapter}{1}

    \section{Sistemas de dos Ecuaciones Lineales}

    Un \textbf{sistema de dos ecuaciones lineales} son dos ecuaciones lineales de la forma
    \begin{equation}
        \left\{
            \begin{array}{ccccc}
                2x & + & y & = & 10 \\
                5x & + & 9y & = & 14 \\
            \end{array}
        \right.
    \end{equation}
    $x$ e $y$ son llamadas las \textbf{incógnitas} del sistema. Este sistema es llamado \textbf{lineal}, pues las incógnitas $x$ e $y$ aparecen con exponente 1 y no hay más funciones involucradas que contengan a $x$ e/o $y$. En el sistema (1.1), los números $2$ y $5$ son llamados \textbf{coeficientes de la incógnita $x$}, y los 1 y 9 son los de la incógnita $y$.

    \begin{obs}
        Recuerde que $y=1y$ y que $0=0y$.
    \end{obs}

    \begin{excer}
        Dado el sistema:
        \begin{equation*}
            \left\{
                \begin{array}{ccccc}
                    10x & + & 4y & = & 10 \\
                    5x & + & 2y & = & 5 \\
                \end{array}
            \right.
        \end{equation*}
        indique los coeficientes de las incógnitas $x$ e $y$, respectivamente.
    \end{excer}

    \begin{excer}
        Dado el sistema:
        \begin{equation*}
            \left\{
                \begin{array}{ccccc}
                    (1+1+1)x &  &  & = & 10 \\
                    15x & + & \sqrt{19}y & = & 5 \\
                \end{array}
            \right.
        \end{equation*}
        indique los coeficientes de las incógnitas $x$ e $y$, respectivamente.
    \end{excer}

    \begin{excer}
        Dado el sistema:
        \begin{equation*}
            \left\{
                \begin{array}{ccccc}
                    \sqrt{2}x &  &  & = & 10 \\
                     &  & \frac{\pi}{3}y & = & 5 \\
                \end{array}
            \right.
        \end{equation*}
        indique los coeficientes de las incógnitas $x$ e $y$, respectivamente.
    \end{excer}

    \section{Soluciones}

    \begin{itemize}
        \item Ejericicio 1.1.1: Coeficientes de $x$: 10 y 5. Coeficientes de $y$: 4 y 2.
        \item Ejericicio 1.1.2: Coeficientes de $x$: 3 y 15. Coeficientes de $y$: 0 y $\sqrt{19}$.
        \item Ejericicio 1.1.3: Coeficientes de $x$: $\sqrt{2}$ y 0. Coeficientes de $y$: 0 y $\frac{\pi}{3}$.
    \end{itemize}

\end{document}