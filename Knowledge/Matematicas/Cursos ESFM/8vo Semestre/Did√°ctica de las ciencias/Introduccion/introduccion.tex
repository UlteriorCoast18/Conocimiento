\documentclass[12pt]{report}
\usepackage[spanish]{babel}
\usepackage[utf8]{inputenc}
\usepackage{amsmath}
\usepackage{amssymb}
\usepackage{amsthm}
\usepackage{graphics}
\usepackage{subfigure}
\usepackage{lipsum}
\usepackage{array}
\usepackage{multicol}
\usepackage{enumerate}
\usepackage[framemethod=TikZ]{mdframed}
\usepackage[a4paper, margin = 1.5cm]{geometry}

%En esta parte se hacen redefiniciones de algunos comandos para que resulte agradable el verlos%

\renewcommand{\theenumii}{\roman{enumii}}

\def\proof{\paragraph{Demostración:\\}}
\def\endproof{\hfill$\blacksquare$}

\def\sol{\paragraph{Solución:\\}}
\def\endsol{\hfill$\square$}

%En esta parte se definen los comandos a usar dentro del documento para enlistar%

\newtheoremstyle{largebreak}
  {}% use the default space above
  {}% use the default space below
  {\normalfont}% body font
  {}% indent (0pt)
  {\bfseries}% header font
  {}% punctuation
  {\newline}% break after header
  {}% header spec

\theoremstyle{largebreak}

\newmdtheoremenv[
    leftmargin=0em,
    rightmargin=0em,
    innertopmargin=-2pt,
    innerbottommargin=8pt,
    hidealllines = true,
    roundcorner = 5pt,
    backgroundcolor = gray!60!red!30
]{exa}{Ejemplo}[section]

\newmdtheoremenv[
    leftmargin=0em,
    rightmargin=0em,
    innertopmargin=-2pt,
    innerbottommargin=8pt,
    hidealllines = true,
    roundcorner = 5pt,
    backgroundcolor = gray!50!blue!30
]{obs}{Observación}[section]

\newmdtheoremenv[
    leftmargin=0em,
    rightmargin=0em,
    innertopmargin=-2pt,
    innerbottommargin=8pt,
    rightline = false,
    leftline = false
]{theor}{Teorema}[section]

\newmdtheoremenv[
    leftmargin=0em,
    rightmargin=0em,
    innertopmargin=-2pt,
    innerbottommargin=8pt,
    rightline = false,
    leftline = false
]{propo}{Proposición}[section]

\newmdtheoremenv[
    leftmargin=0em,
    rightmargin=0em,
    innertopmargin=-2pt,
    innerbottommargin=8pt,
    rightline = false,
    leftline = false
]{cor}{Corolario}[section]

\newmdtheoremenv[
    leftmargin=0em,
    rightmargin=0em,
    innertopmargin=-2pt,
    innerbottommargin=8pt,
    rightline = false,
    leftline = false
]{lema}{Lema}[section]

\newmdtheoremenv[
    leftmargin=0em,
    rightmargin=0em,
    innertopmargin=-2pt,
    innerbottommargin=8pt,
    roundcorner=5pt,
    backgroundcolor = gray!30,
    hidealllines = true
]{mydef}{Definición}[section]

\newmdtheoremenv[
    leftmargin=0em,
    rightmargin=0em,
    innertopmargin=-2pt,
    innerbottommargin=8pt,
    roundcorner=5pt
]{excer}{Ejercicio}[section]

%En esta parte se colocan comandos que definen la forma en la que se van a escribir ciertas funciones%

\newcommand\abs[1]{\ensuremath{\left|#1\right|}}
\newcommand\divides{\ensuremath{\bigm|}}
\newcommand\cf[3]{\ensuremath{#1:#2\rightarrow#3}}
\newcommand\natint[1]{\ensuremath{\left[\!\left[ #1\right]\!\right]}}
\newcommand{\afa}{\:
    \begin{tikzpicture}
        \draw [line width = 0.17 mm, black] (0,0) -- (-0.115,0.29);
        \draw [line width = 0.17 mm, black] (0,0) -- (0.115,0.29);
        \draw [line width = 0.17 mm, black] (-0.12,0) arc (190:-10:0.12cm);
    \end{tikzpicture}
    \:
}
%Este símvolo es para casi todo salvo una cantidad finita

%recuerda usar \clearpage para hacer un salto de página

\begin{document}
    \setlength{\parskip}{5pt} % Añade 5 puntos de espacio entre párrafos
    \setlength{\parindent}{12pt} % Pone la sangría como me gusta
    \title{Notas Didáctica de las Ciencias}
    \author{Cristo Daniel Alvarado}
    \maketitle

    \tableofcontents %Con este comando se genera el índice general del libro%

    %\setcounter{chapter}{3} %En esta parte lo que se hace es cambiar la enumeración del capítulo%
    
    \chapter{Nombre del capitulo}
    
    \section*{Introducción}

    El objetivo del curso va a ser crear un cuadernillo donde se pongan actividades para que el alumno realice (no refiriéndose a la persona que escribe el texto).

    %Esto es una actividad

    \section{Un curso ideal}

    \textit{¿Qué es lo que yo considero un curso ideal de Matemáticas (independientemente del tema que se aborde)?}
    
    Primeramente considero que un curso ideal de matemáticas debe de tener las siguientes características:

    \begin{itemize}
        \item Exposición clara y conscisa del objetivo al que se va a llegar en la clase y la sesión.
        \item Exposición del tema de parte del profesor siendo ésta tal que quede claro el tema.
        \item En medio de la exposición de información de parte del mismo, que se deje al alumno digerir adecuadamente la información y se pueda hacer al alumno cuestionarse sobre el tema que se está abordando.
        \item Dentro de la exposición que se planteen interrogantes y las mismas (en caso de que sea posible) se resuelvan y se discutan de forma grupal, éstas mismas pueden ser ejercicios, problemas o lecturas.
        \item Fomentar un ambiente sano de discusión (entre pares o grupal) de los temas que se revisan. De igual forma fomentar la discusión de ideas.
        \item Actividades relacionadas al tema, ya sean lecturas para después de la clase, problemas o ejercicios.
    \end{itemize}

    El profesor que dé la clase debe tener un dominio del tema lo suficientemente adecuado como para poder transmitir sus conocimientos adecuadamente.

    %Esto es el fin de la actividad.

    En el canal de teams que está dedicado a mí, se debe subir la actividad que se hizo anteriormente. En caso de algún problema, aquí está el correo de la profesora:

    Profesora María Gonzales: lmgonzaleza@ipn.mx

    La actividad anterior se llama: Diagnóstico.

    El medio de comunicación va a ser por el chat para cuando se solicite la revisión de las actividades que se harán en el curso.

    %Esto es otra actividad%

    Yo he visto o he escuchado los siguientes problemas:

    \textit{Con el profesor:}
    
    \begin{itemize}
        \item No hay mucho apoyo de parte del profesor para poder comprender adecuadamente el tema.
        \item Una mala gestión del tiempo de la clase, que hace de parte del profesor que termine siendo muy pesada en los últimos minutos.
        \item No se aprovechan las herramientas que ofrece la escuela para las actividades que se llevan a cabo.
        \item Falta de empatía.
        \item Actitud muy negativa y desagradable con los alumnos.
        \item Acoso.
        \item No se llega a terminar el temario establecido al inicio del curso.
    \end{itemize}

    \textit{Con los compañeros:}

    \begin{itemize}
        \item Muchas veces hay compañeros que toman una actitud muy pedante hacia los demás y no fomentan un ambiente de discusión sana.
        \item Algunos compañeros acostumbran comer en el salón, cosa que en ocasiones resulta molesta para poder concentrarse.
        \item Compañeros que dificultan el trabajo por intentar hacerse sentir más que los demás.
        \item Falta de interés.
    \end{itemize}

    \textit{Con los materiales de apoyo:}

    \begin{itemize}
        \item Hay materiales de apoyo pero la escuela no fomenta su uso. Por ejemplo Wolfram Alpha ofrece a la escuela acceso a la herramienta pro de forma gratuita, pero casi nadie lo sabe. La única forma de saberlo es investigando en la página del IPN.
        \item Muchas veces se usan materiales muy viejos, que resultan complicados de leer tanto en su lenguaje como en su notación matemática.
    \end{itemize}

    \newpage

    \begin{center}
        \textbf{Actividad de equipo:}
    \end{center}

    \begin{table}[ht]
        \begin{center}
            \begin{tabular}{p{0.4\linewidth} | p{0.6\linewidth}}
                \hline
                \hline
                Problema & Compromiso del profesor \\
                \hline
                \hline
                Mala gestión de tiempo destinado a cada tema. & Se gestione adecuadamente el tiempo. Con ello que también avance conforme a las capcidades y aptitudes de los alumnos \\
                \hline
                Poco éticos y apáticos. & Ser más comprensivo con los alumnos. \\
                \hline
                Falta de capacitación pedagógica. & Compromiso del profesor para informarse de técnicas pedagógicas actuales con las herramientas disponibles. \\
                \hline
                Poca retroalimentación de actividades. & Que haya coherencia y constante comunicación con los alumnos. \\
                \hline 
                Ambigua asignación de calificaciones & Sea claro en cada punto que se está evaluando. \\
                \hline
            \end{tabular}
            \caption{Problemáticas con el profesor.}
        \end{center}
    \end{table}

    \begin{table}[ht]
        \begin{center}
            \begin{tabular}{p{0.4\linewidth} | p{0.6\linewidth}}
                \hline
                \hline
                Problema & Compromiso del profesor \\
                \hline
                \hline
                Falta de empatía de los compañeros. & Fomentar un ambiente de convivencia y trabajo en equipo. \\
                \hline
                Falta de compromiso en trabajos en equipo. & Asignación de actividades y sanciones a los alumnos que no deseen trabajar.\\ 
                \hline
                Actitud de desinterés y falta de respeto. & Comprometerse a generar un ambiente de participación activa. \\
                \hline
                Indisposición a realizar actividades complicadas. & Fomentar ejercicios y/o actividades de todos los niveles. \\ 
                \hline
            \end{tabular}
            \caption{Problemáticas con los compañeros.}
        \end{center}
    \end{table}

    \begin{table}[h]
        \begin{center}
            \begin{tabular}{p{0.2\linewidth} | p{0.375\linewidth} | p{0.4\linewidth}}
                \hline
                \hline
                Problema & Compromiso del profesor & Compromisos de los alumnos \\
                \hline
                \hline
                Difícil acceso. & Facilitar el acceso rápido. & Compromiso de los alumnos a compartir y difundir el material al que tienen acceso. \\
                Materiales y temas poco actualizados. & Compromenterse a que los materiales usados se integren temas nuevos. & Comprometerse a la investigación y búsqueda de los materiales.\\
            \end{tabular}
            \caption{Problemáticas con los materiales de apoyo.}
        \end{center}
    \end{table}

    Participantes:

    \begin{itemize}
        \item Cristo Daniel Alvarado.
        \item Lozano Vite Iris Paola.
        \item Daniel Guzmán Vargas.
        \item Ramírez Reyes Mariene.
        \item Belem Torres López.
    \end{itemize}

    %Fin de la actividad%

    \section{Pedagogía del contrato.}

    En objetivo de esta actividad anterior es de llegar a acuerdos comunes entre ambas partes (alumnos y padres de familia con los profesores).

    La explicación es que los reglamentos no funcionan adecuadamente. Para solventar este problema se discuten los acuerdos en grupo. El objetivo es que todas las partes estén enteradas sobre la forma en la que se va a desarrollar la dinámica de la clase y las actividades dentro de la misma.

    \begin{obs}
        En caso de incumplir un contrato didáctico, se tiene que discutir sobre el incumplimiento y solventar la situación.
    \end{obs}

    Se puede además añadir una sección de correctivos.

    \newpage

    \section{Objetivo del curso}

    El objetivo es ocupar toda la didáctica para hacer un plan de trabajo de clase. No se va a llegar a un método general, sino a una metodología de investigación-acción.

    \section{Apertura de las actividades}

    Hay diferentes grados de apertura de una actividad. Cada una tiene diferentes etapas y se colocan en la siguiente tabla:

    \begin{table}[ht]
        \begin{center}
            \begin{tabular}{ p{0.2\linewidth} | p{0.1\linewidth} | p{0.1\linewidth} | p{0.1\linewidth} | p{0.1\linewidth} | p{0.1\linewidth} | p{0.1\linewidth} }
                \hline
                \hline
                Etapas & 1 & 2 & 3 & 4 & 5 & 6 \\
                \hline
                Área de interés & P & P & P & P & P & A \\
                Establecimiento del problema & P & P & P & P & A & A \\
                Planificación & P & P/A & A & A & A & A \\
                Determinación de la estrategia & P & P/A & A & A & A & A \\
                Realización & A & A & A & A & A & A \\
                Interpretación de resultados & P/A & P/A & P/A & A & A & A \\
                \hline
                \hline

            \end{tabular}
            \caption{Grados de apertura de actividades, A: Alumno, P: Profesor, P/A: Profesor y Alumno.}
        \end{center}
    \end{table}



\end{document}