\documentclass[12pt]{report}
\usepackage[spanish]{babel}
\usepackage[utf8]{inputenc}
\usepackage{amsmath}
\usepackage{amssymb}
\usepackage{amsthm}
\usepackage{graphics}
\usepackage{subfigure}
\usepackage{lipsum}
\usepackage{array}
\usepackage{multicol}
\usepackage{enumerate}
\usepackage[framemethod=TikZ]{mdframed}
\usepackage[a4paper, margin = 1.5cm]{geometry}

%En esta parte se hacen redefiniciones de algunos comandos para que resulte agradable el verlos%

\renewcommand{\theenumii}{\roman{enumii}}

\def\proof{\paragraph{Demostración:\\}}
\def\endproof{\hfill$\blacksquare$}

\def\sol{\paragraph{Solución:\\}}
\def\endsol{\hfill$\square$}

%En esta parte se definen los comandos a usar dentro del documento para enlistar%

\newtheoremstyle{largebreak}
  {}% use the default space above
  {}% use the default space below
  {\normalfont}% body font
  {}% indent (0pt)
  {\bfseries}% header font
  {}% punctuation
  {\newline}% break after header
  {}% header spec

\theoremstyle{largebreak}

\newmdtheoremenv[
    leftmargin=0em,
    rightmargin=0em,
    innertopmargin=-2pt,
    innerbottommargin=8pt,
    hidealllines = true,
    roundcorner = 5pt,
    backgroundcolor = gray!60!red!30
]{exa}{Ejemplo}[section]

\newmdtheoremenv[
    leftmargin=0em,
    rightmargin=0em,
    innertopmargin=-2pt,
    innerbottommargin=8pt,
    hidealllines = true,
    roundcorner = 5pt,
    backgroundcolor = gray!50!blue!30
]{obs}{Observación}[section]

\newmdtheoremenv[
    leftmargin=0em,
    rightmargin=0em,
    innertopmargin=-2pt,
    innerbottommargin=8pt,
    rightline = false,
    leftline = false
]{theor}{Teorema}[section]

\newmdtheoremenv[
    leftmargin=0em,
    rightmargin=0em,
    innertopmargin=-2pt,
    innerbottommargin=8pt,
    rightline = false,
    leftline = false
]{propo}{Proposición}[section]

\newmdtheoremenv[
    leftmargin=0em,
    rightmargin=0em,
    innertopmargin=-2pt,
    innerbottommargin=8pt,
    rightline = false,
    leftline = false
]{cor}{Corolario}[section]

\newmdtheoremenv[
    leftmargin=0em,
    rightmargin=0em,
    innertopmargin=-2pt,
    innerbottommargin=8pt,
    rightline = false,
    leftline = false
]{lema}{Lema}[section]

\newmdtheoremenv[
    leftmargin=0em,
    rightmargin=0em,
    innertopmargin=-2pt,
    innerbottommargin=8pt,
    roundcorner=5pt,
    backgroundcolor = gray!30,
    hidealllines = true
]{mydef}{Definición}[section]

\newmdtheoremenv[
    leftmargin=0em,
    rightmargin=0em,
    innertopmargin=-2pt,
    innerbottommargin=8pt,
    roundcorner=5pt
]{excer}{Ejercicio}[section]

%En esta parte se colocan comandos que definen la forma en la que se van a escribir ciertas funciones%

\newcommand\abs[1]{\ensuremath{\left|#1\right|}}
\newcommand\divides{\ensuremath{\bigm|}}
\newcommand\cf[3]{\ensuremath{#1:#2\rightarrow#3}}
\newcommand\natint[1]{\ensuremath{\left[\!\left[ #1\right]\!\right]}}
\newcommand{\afa}{\:
    \begin{tikzpicture}
        \draw [line width = 0.17 mm, black] (0,0) -- (-0.115,0.29);
        \draw [line width = 0.17 mm, black] (0,0) -- (0.115,0.29);
        \draw [line width = 0.17 mm, black] (-0.12,0) arc (190:-10:0.12cm);
    \end{tikzpicture}
    \:
}
%Este símvolo es para casi todo salvo una cantidad finita

%recuerda usar \clearpage para hacer un salto de página

\begin{document}
    \setlength{\parskip}{5pt} % Añade 5 puntos de espacio entre párrafos
    \setlength{\parindent}{12pt} % Pone la sangría como me gusta
    \title{Notas Didáctica de las Ciencias}
    \author{Cristo Daniel Alvarado}
    \maketitle

    \tableofcontents %Con este comando se genera el índice general del libro%

    %\setcounter{chapter}{3} %En esta parte lo que se hace es cambiar la enumeración del capítulo%
    
    \chapter{Nombre del capitulo}
    
    \section*{Introducción}

    El objetivo del curso va a ser crear un cuadernillo donde se pongan actividades para que el alumno realice (no refiriéndose a la persona que escribe el texto).

    %Esto es una actividad

    \section{Un curso ideal}

    \textit{¿Qué es lo que yo considero un curso ideal de Matemáticas (independientemente del tema que se aborde)?}
    
    Primeramente considero que un curso ideal de matemáticas debe de tener las siguientes características:

    \begin{itemize}
        \item Exposición clara y conscisa del objetivo al que se va a llegar en la clase y la sesión.
        \item Exposición del tema de parte del profesor siendo ésta tal que quede claro el tema.
        \item En medio de la exposición de información de parte del mismo, que se deje al alumno digerir adecuadamente la información y se pueda hacer al alumno cuestionarse sobre el tema que se está abordando.
        \item Dentro de la exposición que se planteen interrogantes y las mismas (en caso de que sea posible) se resuelvan y se discutan de forma grupal, éstas mismas pueden ser ejercicios, problemas o lecturas.
        \item Fomentar un ambiente sano de discusión (entre pares o grupal) de los temas que se revisan. De igual forma fomentar la discusión de ideas.
        \item Actividades relacionadas al tema, ya sean lecturas para después de la clase, problemas o ejercicios.
    \end{itemize}

    El profesor que dé la clase debe tener un dominio del tema lo suficientemente adecuado como para poder transmitir sus conocimientos adecuadamente.

    %Esto es el fin de la actividad.

    En el canal de teams que está dedicado a mí, se debe subir la actividad que se hizo anteriormente. En caso de algún problema, aquí está el correo de la profesora:

    Profesora María Gonzales: lmgonzaleza@ipn.mx

    La actividad anterior se llama: Diagnóstico.

    El medio de comunicación va a ser por el chat para cuando se solicite la revisión de las actividades que se harán en el curso.

    %Esto es otra actividad%

    Yo he visto o he escuchado los siguientes problemas:

    \textit{Con el profesor:}
    
    \begin{itemize}
        \item No hay mucho apoyo de parte del profesor para poder comprender adecuadamente el tema.
        \item Una mala gestión del tiempo de la clase, que hace de parte del profesor que termine siendo muy pesada en los últimos minutos.
        \item No se aprovechan las herramientas que ofrece la escuela para las actividades que se llevan a cabo.
        \item Falta de empatía.
        \item Actitud muy negativa y desagradable con los alumnos.
        \item Acoso.
        \item No se llega a terminar el temario establecido al inicio del curso.
    \end{itemize}

    \textit{Con los compañeros:}

    \begin{itemize}
        \item Muchas veces hay compañeros que toman una actitud muy pedante hacia los demás y no fomentan un ambiente de discusión sana.
        \item Algunos compañeros acostumbran comer en el salón, cosa que en ocasiones resulta molesta para poder concentrarse.
        \item Compañeros que dificultan el trabajo por intentar hacerse sentir más que los demás.
        \item Falta de interés.
    \end{itemize}

    \textit{Con los materiales de apoyo:}

    \begin{itemize}
        \item Hay materiales de apoyo pero la escuela no fomenta su uso. Por ejemplo Wolfram Alpha ofrece a la escuela acceso a la herramienta pro de forma gratuita, pero casi nadie lo sabe. La única forma de saberlo es investigando en la página del IPN.
        \item Muchas veces se usan materiales muy viejos, que resultan complicados de leer tanto en su lenguaje como en su notación matemática.
    \end{itemize}

    \newpage

    \begin{center}
        \textbf{Actividad de equipo:}
    \end{center}

    \begin{table}[ht]
        \begin{center}
            \begin{tabular}{p{0.4\linewidth} | p{0.6\linewidth}}
                \hline
                \hline
                Problema & Compromiso del profesor \\
                \hline
                \hline
                Mala gestión de tiempo destinado a cada tema. & Se gestione adecuadamente el tiempo. Con ello que también avance conforme a las capcidades y aptitudes de los alumnos \\
                \hline
                Poco éticos y apáticos. & Ser más comprensivo con los alumnos. \\
                \hline
                Falta de capacitación pedagógica. & Compromiso del profesor para informarse de técnicas pedagógicas actuales con las herramientas disponibles. \\
                \hline
                Poca retroalimentación de actividades. & Que haya coherencia y constante comunicación con los alumnos. \\
                \hline 
                Ambigua asignación de calificaciones & Sea claro en cada punto que se está evaluando. \\
                \hline
            \end{tabular}
            \caption{Problemáticas con el profesor.}
        \end{center}
    \end{table}

    \begin{table}[ht]
        \begin{center}
            \begin{tabular}{p{0.4\linewidth} | p{0.6\linewidth}}
                \hline
                \hline
                Problema & Compromiso del profesor \\
                \hline
                \hline
                Falta de empatía de los compañeros. & Fomentar un ambiente de convivencia y trabajo en equipo. \\
                \hline
                Falta de compromiso en trabajos en equipo. & Asignación de actividades y sanciones a los alumnos que no deseen trabajar.\\ 
                \hline
                Actitud de desinterés y falta de respeto. & Comprometerse a generar un ambiente de participación activa. \\
                \hline
                Indisposición a realizar actividades complicadas. & Fomentar ejercicios y/o actividades de todos los niveles. \\ 
                \hline
            \end{tabular}
            \caption{Problemáticas con los compañeros.}
        \end{center}
    \end{table}

    \begin{table}[h]
        \begin{center}
            \begin{tabular}{p{0.2\linewidth} | p{0.375\linewidth} | p{0.4\linewidth}}
                \hline
                \hline
                Problema & Compromiso del profesor & Compromisos de los alumnos \\
                \hline
                \hline
                Difícil acceso. & Facilitar el acceso rápido. & Compromiso de los alumnos a compartir y difundir el material al que tienen acceso. \\
                Materiales y temas poco actualizados. & Compromenterse a que los materiales usados se integren temas nuevos. & Comprometerse a la investigación y búsqueda de los materiales.\\
            \end{tabular}
            \caption{Problemáticas con los materiales de apoyo.}
        \end{center}
    \end{table}

    Participantes:

    \begin{itemize}
        \item Cristo Daniel Alvarado.
        \item Lozano Vite Iris Paola.
        \item Daniel Guzmán Vargas.
        \item Ramírez Reyes Mariene.
        \item Belem Torres López.
    \end{itemize}

    %Fin de la actividad%

    \section{Pedagogía del contrato.}

    En objetivo de esta actividad anterior es de llegar a acuerdos comunes entre ambas partes (alumnos y padres de familia con los profesores).

    La explicación es que los reglamentos no funcionan adecuadamente. Para solventar este problema se discuten los acuerdos en grupo. El objetivo es que todas las partes estén enteradas sobre la forma en la que se va a desarrollar la dinámica de la clase y las actividades dentro de la misma.

    \begin{obs}
        En caso de incumplir un contrato didáctico, se tiene que discutir sobre el incumplimiento y solventar la situación.
    \end{obs}

    Se puede además añadir una sección de correctivos.

    \newpage

    \section{Objetivo del curso}

    El objetivo es ocupar toda la didáctica para hacer un plan de trabajo de clase. No se va a llegar a un método general, sino a una metodología de investigación-acción.

    \section{Apertura de las actividades}

    Hay diferentes grados de apertura de una actividad. Cada una tiene diferentes etapas y se colocan en la siguiente tabla:

    \begin{table}[ht]
        \begin{center}
            \begin{tabular}{ p{0.2\linewidth} | p{0.1\linewidth} | p{0.1\linewidth} | p{0.1\linewidth} | p{0.1\linewidth} | p{0.1\linewidth} | p{0.1\linewidth} }
                \hline
                \hline
                Etapas & 1 & 2 & 3 & 4 & 5 & 6 \\
                \hline
                Área de interés & P & P & P & P & P & A \\
                Establecimiento del problema & P & P & P & P & A & A \\
                Planificación & P & P/A & A & A & A & A \\
                Determinación de la estrategia & P & P/A & A & A & A & A \\
                Realización & A & A & A & A & A & A \\
                Interpretación de resultados & P/A & P/A & P/A & A & A & A \\
                \hline
                \hline

            \end{tabular}
            \caption{Grados de apertura de actividades, A: Alumno, P: Profesor, P/A: Profesor y Alumno.}
        \end{center}
    \end{table}

    \section{Actividad: Línea de tiempo de la didáctica de las ciencias}

    \begin{itemize}
        \item 1916: Aparición de la revista \textit{Science Eduaction}.
        \item 1927 a inicios de 1980: publicación de investigaciones en torno a la enseñanza y aprendizaje de las ciencias en inglés.
        \begin{itemize}
            \item 1960-1966: Surgimiento del movimiento \textit{aprendizaje por descubrimiento}: Iniciar un proceso de innovación y de investigaciones sistemáticas.
            \item 1963: Aparición de \textit{Journal of Research in Science Teaching}.
        \item 1972: Publicación de \textit{Studies in Science}.
        \end{itemize}
        \item A inicios de los 80 se ignoraba en gran parte del mundo hispano los problemas educativos como temas de investigación o elaboración de tesis doctorales. Sin embargo, en el mundo angloparlante surgieron entre otros las siguientes revistas científicas enfocadas en el estudio de la enseñanza de las ciencias:
        \begin{itemize}
            \item Aparición de \textit{European Journal of Science Education}.
            \item Aparición de \textit{Enseñanza de las Ciencias}.
            \item Aparición de \textit{ÁSTER}.
            \item Aparición de \textit{Science and Technological Education}.
            \item Aparición de \textit{The Australian Journal of Science Education}.
            \item Aparición de \textit{La Enseñanza de la Física}.
            \item Aparición de \textit{O Ensenio de la Física}.
            \item Aparición de \textit{Investigación en la Escuela}.
            \item Aparición de \textit{Didaskalia}.
            \item Aparición de \textit{Alambique}.
            \item 1989: Aparición de \textit{Aliage}.
        \end{itemize}
        \item A finales de los 80 apoyados en la obra \textit{La comprensión humana}, investigadores apuntaron a un rápido despliegue de investigación enfocada en los problemas de la enseñanza y aprendizaje de las ciencias.
        \item A inicios de los 90 se comienza un surgimiento del análisis del aprenziaje de la didáctica de las ciencias no sólo en inglés, sino también en el mundo hispanohablante.
        \begin{itemize}
            \item Aparición de \textit{Science and Education}.
            \item Aparición de los \textit{Handbooks}.
            \item 1996: Aparición de los \textit{National Science Education Standards}.
        \end{itemize}
    \end{itemize}

    \section{Actividad 10 de Septiembre}

    \begin{obs}
        Se vió el vídeo del aprendizaje de chimpances de Kohler.
    \end{obs}

    \textit{Me llamó la atención:}

    \begin{itemize}
        \item El uso de herramientas para resolver problemas.
        \item Trabajo colaborativo.
        \item Los chimpances denotan inteligencia espacial.
        \item Aprendimiento rápido para resolver problemas.
    \end{itemize}
        
    \textit{Me pregunto:}

    \begin{itemize}
        \item ¿Cuál fue el objetivo del experimento?
        \item ¿Qué se dedujo a partir de este experimento?
        \item ¿Qué experimentos en personas surgieron a partir del mismo?
    \end{itemize}

    \begin{obs}
        Nota: indagar sobre Kohler y hacer documento en Teams para el debate Kohler vs Skinner.
    \end{obs}

    \section{Jerome Bruner}

    Fue una de las personas fundamntales que participaron en las reformas educacionales en Estados Unidos.

    Bruner identifica 4 puntos principales

    \begin{enumerate}
        \item Identifica diferentes maneras de aprender en base a nuestras experiencias. Hay tres modos de aprender: enactivo, icónico y simbólico:
        \begin{itemize}
            \item \textbf{El modelo enactivo} se basa en la acción física, actuando sobre las cosas. Es una interacción con el entorno donde se usan los sentidos como base para la representación actuante. Este aprendizaje se realiza a través de la \textit{imitación}, la manipulación de objetos, el baile y la actuación. Esto es llamado \textbf{representación actuante}. Sobre todo es el que se ve en los primeros años de desarrollo humano.
            
            En esencia es un modelo de aprender actuando sobre la realidad.
            \item \textbf{El modelo icónico} se basa en el uso de dibujos e imágenes en general que puedan servir para aportar imformación sobre algo más alla de ellas mismas. Se pretende reprsentar un aparte de la realidad y aporta información \textit{más allá} de ellos mismos. No se quedan limitados solo en su cuerpo.
            
            Por ejemplo, dibujando es cuando aprenden en esta etapa.

            \item \textbf{El modelo símbolico} se basa en el uso del lenguaje, ya sea hablado o escrito. Ya no es la representación a través del cuerpo o imágenes, sino a través del lenguaje hablado y escrito.
            
            Este modelo de apredizaje es el que le permite acceder a los contenidos y proceso relacionados con lo abstracto.

            El lenguaje es fundamnetal en la parte del aprendizaje a través de forma oral o escrita. Se límita a conceptos muy concretos o muy abstractos, lo cual diferencia totalmente con las partes anteroires, todo surge como una repesentación mediante palabras orales o escritas.
        \end{itemize}
        \item Teoría del aprendizaje por descubirmineto, la teoría principal de Bruner.
        
        Para Bruner el descubrimiento es la acción guiáda por los niños para que descruban algo nuevo.

        Uno de los elementos principales a la hora de conocer es la \textit{participación activa del sujeto que aprende}. Para que el sujeto aprenda, forzosamente el sujeto debe estar actuando, no pued estar solo como especatador. Debe hacer algo/elaborar/descubir para poder llegar a un aprendizaje.

        Para que el niño actúe, la fuente del aprendizaje debe venir de una motivación \textit{intrínseca}, puede ser curiosidad, establecer metaz, en general es aquello que genere interés en el aprendiz. Esto debe llevar al sujeto a descubrir, aprender y elaborar aprendizajes significativos.

        Esto es, hay que motivar al sujeto a que aprenda. Motivación intrínseca.

        En caso de que no exista este interés, se debe motivar al niño para que surga el interés.

        El objeto primordial de la eduacción es fomentar la solución inteligente de todo tipo de problemas con los que se encuentran las personas. Debe poder enfocar su atención a tratar de resolver ese problema. Este es un punto fundamental.

        Por tanto, se deben crear sitaucioones de enseñanza-aprendizaje, que logren motivar al niño de modo que logren en el educando \textit{comprensión} y \textit{reflexión} acerca de o que se le presneta como conetnido del aprendizaje (lo que se aprende).

        \begin{itemize}
            \item \textit{Comprensión}: Singifica que educando llega a ver como usar con provecho de fomra que le interesa, ideas generales y hechos que los confirman. Que identifique ideas, conceptos, que se base en la realidad.
            \item \textit{Reflexión}: Siginfica hcer un examen crítica de una idea o conocimiento a la luz de la evidencia comprobable que lo sostiende y de las ulteroires concecuencias hacia las que señala.
        \end{itemize}
        Una bonita forma de ver lo anterior es imaginar que se trabaja con un investigador, el observa, analiza, busca evidencia empírica (porqué sucede eso que sucede) y con ello, llegar a una conclusión que nos permitirá la posibilidad de generalización.
        %TEoría cognocitiva.
    \end{enumerate}

    Como resumen, encaminar al sujeto a descubrir, cuestionarse, plantearse problemas, buscar información empírica, usar lo aprendido y aplicarlo a otras áreas de conocimiento.

    \begin{itemize}
        \item \textbf{Aprendizaje}: El aprendizaje es un proceso dinámico, es el cambio que se produce en los conocimientos o estructuras mentales mediante la experiencia interactiva de los mismos y de lo que llega de afuera del individuo.
        \item \textbf{Conocimiento o estructuras cognocitivas}: son respuetas a preguntas como: ¿de qué está hecho algo?, ¿qué es lo que hace a uno pertenecer a ?, ¿de qué calidad es una cosa o acción?, ¿qué debería hacer?
    \end{itemize}

    Se usa el conocimiento previo y se reestructura para genera nuevo conocimiento.

    Las estructuras cognocsitivas nos llevan a entender que hablan. Nos da ideas de como podemos actuar en la realidad. Estas estrcuturas sonlas que tiene el sujeto y que intereactúan con el sujeto para generar el aprendizaje.

    En resumen, para Bruner el aprender es un \textit{proceso de desarrollo de la estructura cognoscitiva} o de los conocimientos.

    \subsection{Andiamaje}

    El proceso de enseñanza es ayudar al estudiante a poseer la estructura de conocimiento.

    Bruner sostiene que no se aprende de manera individual, sino dentro del un contexto social, eso es que no hay aprendizaje sin la ayuda de otros, ya sean maestros, padres, amigos con más experiencia, etc\dots Bruner retoma esta idea de Vigotsky. El niño siempre aprende con ayuda de las personas a su alrededor.

    El papel de estos amigos/maestros/padres es de \textit{facilitadores}, ayudan a actuar paraw que se realice un descubrimiento guiado cuyo motor es la curiosidad de los aprendices.

    Los facilitadores deben poner en juego todos los medio spara que el aprendiz pueda desarrollar sus intereses y obtener prática y conociminetos a cambio. Se debe crear una situación didáctica que ayude al niño a identificar un problema, motivar al alumno a que resuelva el proceso y que los facilitadores vayan facilitando el proceso de aprendizaje.

    El andamiaje son estos factores anteriores que se han mencionado.

    \subsection{La escuela}

    Las escuelas deben ser lugares que den salida a la curiosidad natural de los estudiantes, ofreciéndoles maneras de parender mediante la indagación y la posibilidad de desarrollar sus intereses gracisa a la participación de tereceros que guían y actúan como referentes.

    Bruner propone un currículo educativo en espiral, en el que los contenidos sean abordados de forma periódica para que cada vez se vayan consolidando los contenidos ya aprendidos a la luz de la nueva información de la que se dispone (se va subiendo el nivel de dificultad de cada información).

    Bruner se basó en tres personajes:

    \begin{itemize}
        \item Piaget. Etapas del desarrollo.
        \item Ausubel. Aprendizaje significativo. Es todo lo contario de lo aprendizaje mecánico.
        
        Este tipo de aprendizaje se construye mediante \textit{recepción} y mediante \textit{descubrimiento} o \textit{construcción}.
        \item Vigotsky. El aprendizaje es colectivo (requiere de otros).
        \item Bruner. Propone 3 modos de aprender (que son también etapas de aprendizaje):
        \begin{itemize}
            \item Enactiva.
            \item Icónica.
            \item Simbólica.
        \end{itemize}
        éstas tres anteriores son llamadas \textit{formas de representación del conocimiento}.
        \item El conductismo generaba aprendizaje mecánico.
    \end{itemize}

    Tanto Bruner como Ausbel no apoyaban el aprendizaje mecánico. A diferencia de los conductistas, Ausbel usaba los conocimientos previos del sujeto.

    También con Vigotsky se contrasta el conductismo, ya que en este último el aprendizaje de prodce de forma individual. En cambio, Vigotsky lo propone de forma colectiva con las personas que conocen más que el sujeto.

    Bruner con los tres autores mencionados anteriormente propone el aprendizaje por descubrimiento.

    Bruner constrasta con Ausubel, ya que niega el aprendizaje mediante recepción. Para Ausbel se aprenden conceptos y para Bruner, se aprende mediante estructuras cognoscitivas a parte de aprender conceptos se aprenden:
    \begin{itemize}
        \item Conceptos.
        \item Procedimientos.
        \item Formas de actuar.
    \end{itemize}

    \begin{excer}
        Con el tema que escogí, diseñar una actividad de aprendizaje para logar aprendizaje mecánico. En ese mismo tema, usar una actividad que use las aportaciones de Bruner y los demás autores. No necesariamente que se llegue a un insight, pero que se haga al aprendizaj de conceptos, procedimientos, formas de actuar, etc...

        Esto para el próximo martes.
    \end{excer}

    \section{Ecuaciones Lineales}

    \begin{equation*}
        y=mx+b
    \end{equation*}

    Para calcular la pendiente de una recta dados dos puntos $(x_1,y_1),(x_2,y_2)$, se obtiene la pendiente con la fórmula:
    \begin{equation*}
        m=\frac{y_1-y_2}{x_1-x_2}
    \end{equation*}
    \begin{excer}
        Calcule la pendiente de la recta que pasa por $(0,0)$ y $(1,10)$.
    \end{excer}

    Ponemos puntos en una gráfica $(0,0),(10, 20),(20,40)$

    \section{Actividad 24 de Septiembre}

    Para los conductistas, el maestro aporta el conocimiento. Los conductistas buscaban que se realizara un aprendizaje mecánico.

    Para Bruner, el descubrimiento debe ser por parte del estudiante. Mediante el descubrimiento es cuando se llega a un aprendizaje verdadero.

    Para Ausbel, el aprendizaje verdadero lo llama \textit{aprendizaje significativo}. La mayoría de los autores concurren en que el aprendizaje verdadero se realiza por descubrimiento y por trabajo de los estudiantes. Ausbel no está totalmente de acuerdo con esta forma de trabajo, dice que también el estudiante puede obtener conocimiento siendo que éste venga de parte del profesor.

    Ausebl propone dos tipos de aprendizaje:
    \begin{itemize}
        \item Aprendizaje por recepción (el maestro aporta conocimiento).
        \item Aprendizaje por descubrimiento.
    \end{itemize}
    ambos son aprendizajes significativos. El maestro aporta conocimiento pero no como lo hacen los conductistas (siendo éste tipo de aprendizaje muy imponente). Deben de cumplirse ciertas condiciones para que el aprendizaje sea potencialmente significativo.

    \begin{excer}
        Extraer del artículo las características para que el aprendizaje de los estudiantes sea significativo.
    \end{excer}

    \section{Aprendizaje Significativo}

    El \textit{aprendizaje significativo} es una teoría que se ocupa del proceso de construcción de significados por parte de quién aprende, que se constituye como eje esencial de la enseñanza, dando cuenta de todo aquello de lo que el docente debe de contemplar en su tarea de enseñar, si lo que pretende es la significatividad de lo que su alumnado aprende.
    
    La finalidad es aportar todo aquello que garantice la adquisición, la asimilación y la retención del contenido que la escuela ofrece a los estudiantes, de manera que estos puedan atribuirle significado a esos contenidos.

    Éste es el constructo esencial de la teoría que Ausubel postuló; según él, los
    estudiantes no comienzan su aprendizaje de cero, esto es, como mentes en blanco,
    sino que aportan a ese proceso de dotación de significados sus experiencias y
    conocimientos, de tal manera que éstos condicionan aquello que aprenden y, si son
    explicitados y manipulados adecuadamente, pueden ser aprovechados para mejorar el
    proceso mismo de aprendizaje y para hacerlo significativo. 

    La atribución de significados sólo es posible
    por medio de un aprendizaje significativo, de modo que éste no sólo es el producto
    final, sino también el proceso que conduce al mismo, que se caracteriza y define por la
    interacción.

    Se produce así una interacción entre esos nuevos
    contenidos y elementos relevantes presentes en la estructura cognitiva que reciben el
    nombre de subsumidores. No se trata de una interacción cualquiera, de suerte que la
    presencia de ideas, conceptos o proposiciones inclusivas, claras y disponibles en la
    mente del aprendiz es lo que dota de significado a ese nuevo contenido en esa
    interacción, de la que resulta también la transformación de los subsumidores en la
    estructura cognitiva, que van quedando así progresivamente más diferenciados,
    elaborados y estables

    La consecución de un aprendizaje
    significativo supone y reclama dos condiciones esenciales: 
    \begin{itemize}
        \item Actitud potencialmente significativa de aprendizaje de quien aprende, es decir, que haya predisposición para aprender de manera significativa.
        \item Presentación de un material potencialmente significativo. Esto requiere:
        \begin{itemize}
            \item que el material tenga significado lógico, esto es, que sea potencialmente relacionable con la estructura cognitiva del que aprende, de manera no arbitraria y sustantiva.
            \item que existan ideas de anclaje o subsumidores adecuados en el sujeto que permitan la interacción con el material nuevo que se presenta.
        \end{itemize}
    \end{itemize}

    \begin{center}
        \textit{De todos los factores que influyen en el aprendizaje, el más importante consiste en lo que el alumno ya sabe}
    \end{center}

    \begin{center}
        \textit{Aprendizaje significativo no es lo mismo que aprendizaje (que puede ser mecánico) de material lógicamente significativo}
    \end{center}

    El proceso de aprendizaje significativo no se debe confundir con las herramientas que pueden facilitarlo o potenciarlo.

    No hay aprendizaje significativo si no se captan los significados. El conocimiento tiene carácter social, siendo solo posible a través de la mediación semiótica.

    \begin{mydef}
        La \textbf{semiótica} es la ciencia que esttudia los sistemas de comunicación y los signos de la comunicación humana.
    \end{mydef}

    El aprendizaje significativo no se da cuando el alumno se divierte aprendiendo. La finalidad del docente no es de entretener al alumnado.

    El aprendizaje significativo no se produce cuando los contenidos se adaptan a los intereses del alumno. El docente debe interesar a los estudiantes en aquello que deberían aprender significativamente y debe, también, generar las condiciones para que eso ocurra.

    No es suficiente con que el estudiante quiera aprender para que haya aprendizaje significativo. Si el docente no presenta material potencialmente significativo que, a su vez, supone significado lógico del material y la presencia de los subsumidores relevantes en la estructura cognitiva del que aprende.

    El aprendizaje significativo no supone que el educando descubre por sí solo lo que aprende, pues recuerde que
    \begin{center}
        \textit{aprendizaje significativo no es aprendizaje por descubrimiento}
    \end{center}

    Érroneo es también equiparar aprendizaje significativo con aplicación de lo aprendido, pues ésta puede ser mecánica, repetitiva o reproductiva simplemente. El aprendizaje significativo debe ser transferible a nuevas situaciones y contextos, pero de forma autónoma y productiva por parte de quien aprende.

    El aprendizaje significativo también requiere estudio, ejercicios, prácticas, pero siempre con significado, con negociación de significados, con la búsqueda de los significados.

    Únicamente la negociación, el intercambio, la
    contrastación con significado contribuyen a la consolidación y, en ese proceso, posiblemente a la diferenciación progresiva y la reconciliación integradora. Por eso el discurso del aula debe ser una comunicación dialógica bakhtiniana, en la que las voces del alumnado juegan un papel esencial.

    El aprendizaje significativo es un constructo muy manido, pero poco conocido.

    \begin{center}
        \textbf{¿Qué deben hacer los profesores para que haya aprendizaje signifcativo?}
    \end{center}

    La programación lineal en los libros de texto y clases es radicalmente contraria a la esencia misma de un aprendizaje significativo.

    Para servir de ayuda al profesorado con el fin de facilitar un aprendizaje significativo, Ausbel postuló cuatro principios programáticos:
    \begin{itemize}
        \item \textit{Diferenciación progresiva}: proceso característico del aprendizaje verbal
        significativo subordinado, que se produce cuando disponemos de un subsumidor que
        engloba el nuevo concepto o contenido, que lo subsume, por ser más abarcador e
        inclusivo; por tanto, en términos pedagógicos, deberíamos usarlo para estos fines con
        el mismo sentido. Esto supone planificar la docencia desde lo más general a lo más
        específico, desde lo global hasta lo particular.
        \item \textit{Reconciliación integradora}: Cuando se trata de ideas que resultan nuevas para los estudiantes, el proceso de
        discriminación con respecto a las ya existentes resulta más complejo. En este caso, el
        proceso mental que deben seguir los obliga a establecer reconciliaciones integradoras
        características de los aprendizajes superordenado(que se produce cuando se
        incorpora un concepto o una idea que es capaz de subordinar a otras ya existentes en
        la mente del individuo, porque tiene un mayor grado de abstracción y generalidad,
        resultando más inclusiva) y combinatorio (en el queno se dan relaciones de
        subordinación ni de superordenación, sino que se establecen conexiones con
        contenidos disponibles en la estructura cognitiva, pero sólo de modo general).
        \item \textit{Organización secuencial}: Según ésta, es necesario respetar las relaciones naturales de
        dependencia del contenido. Así, el material estudiado y aprendido en primer lugar o
        presentado previamente ejerce el papel de soporte ideacional u organizador del que se
        presentará a continuación; de este modo, actúa como facilitador, justificando así la
        importancia que tiene una organización curricular en secuencia.
        \item \textit{Consolidación}: No se refiere al
        dominio mecánico como prerrequisito, sino que destaca la necesidad de la reiteración
        y de la realización de tareas en contextos y momentos diferentes, para que se
        produzca la generalización y la interiorización efectiva y significativa de lo
        aprendido.
    \end{itemize}
    Los dos primeros son principios definitorios del aprendizaje significativo aplicados a las tareas de organización y planificación; los otros dos son derivaciones naturales de los
    mismos.

    Díaz Barriga y Hernández (2002) sugieren como principios para la instrucción
    derivados de la teoría del aprendizaje significativo los siguientes:
    \begin{enumerate}
        \item El aprendizaje se facilita cuando los contenidos se le presentanal alumno
        organizados de manera conveniente y siguen una secuencia lógica y psicológica
        apropiada.
        \item Es conveniente delimitar intencionalidades y contenidos deaprendizaje en una
        progresión continua que respete niveles deinclusividad, abstracción y generalidad.
        Esto
        implica
        determinar
        las
        relaciones
        de
        superordinación-
        subordinación,antecedentes-consecuentes que guardan los núcleos deinformación
        entre sí.
        \item Los contenidos escolares deben presentarse en forma desistemas conceptuales
        (esquemas de conocimiento) organizados, interrelacionados y jerarquizados, y no
        comodatos aislados y sin orden.
        \item La activación de los conocimientos y experiencias previos queposee el aprendiz
        en su estructura cognitiva facilitará los procesos de aprendizaje significativo de nuevos
        materialesde estudio.
        \item El establecimiento de “puentes cognitivos” (conceptos e ideasgenerales que
        permiten enlazar la estructura cognitiva con el material que se va a aprender) pueden
        orientar al alumno a detectar lasideas fundamentales, a organizarlas e
        interpretarlassignificativamente.
        \item Los contenidos aprendidos significativamente (por recepción opor
        descubrimiento) serán más estables, menos vulnerables alolvido y permitirán la
        transferencia de lo aprendido, sobretodo si se trata de conceptos generales e
        integrados.
        \item Puesto que el estudiante en su proceso de aprendizaje, ymediante ciertos
        mecanismos autorreguladores, puede llegar acontrolar eficazmente el ritmo, secuencia
        y profundidad de susconductas y procesos de estudio, una de las tareas principalesdel
        docente es estimular la motivación y participación activadel sujeto a aumentar la
        significación potencial de los materialesacadémicos.
    \end{enumerate}

    Moreira (2000, 2005, 2010), establece como principios
    que definen un aprendizaje de esta naturaleza los siguientes:
    \begin{itemize}
        \item Aprender que aprendemos a partir de lo que ya sabemos. (Principio del
        conocimiento previo).
        \item Aprender/enseñar preguntas en lugar de respuestas. (Principio de la
        interacción social y del cuestionamiento).
        \item Aprender a partir de distintos materiales educativos. (Principio de la no
        centralidad del libro de texto).
        \item Aprender que somos perceptores y representadores del mundo. (Principio
        del aprendiz como perceptor/representador).
        \item Aprender que el lenguaje está totalmente involucrado en todos los intentos
        humanos de percibir la realidad. (Principio del conocimiento como
        lenguaje).
        \item Aprender que el significado está en las personas, no en las palabras.
        (Principio de la conciencia semántica).
        \item Aprender que el ser humano aprende corrigiendo sus errores. (Principio del
        aprendizaje por el error).
        \item Aprender a desaprender, a no usar los conceptos y las estrategias
        irrelevantes para la sobrevivencia. (Principio del desaprendizaje).
        \item Aprender que las preguntas son instrumentos de percepción y que las
        definiciones y las metáforas son instrumentos para pensar. (Principio de
        laincertidumbredelconocimiento).
        \item Aprender a partir de diferentes estrategias de enseñanza. (Principio de la
        no utilización de la pizarra).
        \item Aprender que simplemente repetir la narrativa de otra persona no estimula
        la comprensión. (Principio del abandono de la narrativa).
    \end{itemize}

    El profesor debe hacer accesible el conocimiento, no facilitarlo.

\end{document}