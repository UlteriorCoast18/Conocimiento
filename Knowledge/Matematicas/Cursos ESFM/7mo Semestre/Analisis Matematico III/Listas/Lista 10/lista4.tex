\documentclass[12pt]{report}
\usepackage[spanish]{babel}
\usepackage[utf8]{inputenc}
\usepackage{amsmath}
\usepackage{amssymb}
\usepackage{amsthm}
\usepackage{graphics}
\usepackage{subfigure}
\usepackage{lipsum}
\usepackage{array}
\usepackage{multicol}
\usepackage{enumerate}
\usepackage{mathrsfs}
\usepackage[framemethod=TikZ]{mdframed}
\usepackage[a4paper, margin = 1.5cm]{geometry}

%En esta parte se hacen redefiniciones de algunos comandos para que resulte agradable el verlos%

\def\proof{\paragraph{Solución:\\}}
\def\endproof{\hfill$\square$\\}
\renewcommand{\theenumii}{\roman{enumii}}

%En esta parte se definen los comandos a usar dentro del documento para enlistar%

\newtheoremstyle{largebreak}
  {}% use the default space above
  {}% use the default space below
  {\normalfont}% body font
  {}% indent (0pt)
  {\bfseries}% header font
  {}% punctuation
  {\newline}% break after header
  {}% header spec

\theoremstyle{largebreak}

\newmdtheoremenv[
    leftmargin=0em,
    rightmargin=0em,
    innertopmargin=-2pt,
    innerbottommargin=8pt,
    roundcorner=5pt
]{excer}{Ejercicio}[section]

\newmdtheoremenv[
    leftmargin=0em,
    rightmargin=0em,
    innertopmargin=-2pt,
    innerbottommargin=8pt,
    rightline = false,
    leftline = false
]{theor}{Teorema}[section]

\newmdtheoremenv[
    leftmargin=0em,
    rightmargin=0em,
    innertopmargin=-2pt,
    innerbottommargin=8pt,
    rightline = false,
    leftline = false
]{propo}{Proposición}[section]

\newmdtheoremenv[
    leftmargin=0em,
    rightmargin=0em,
    innertopmargin=-2pt,
    innerbottommargin=8pt,
    rightline = false,
    leftline = false
]{cor}{Corolario}[section]

\newmdtheoremenv[
    leftmargin=0em,
    rightmargin=0em,
    innertopmargin=-2pt,
    innerbottommargin=8pt,
    rightline = false,
    leftline = false
]{lema}{Lema}[section]

\newmdtheoremenv[
    leftmargin=0em,
    rightmargin=0em,
    innertopmargin=-2pt,
    innerbottommargin=8pt,
    roundcorner=5pt,
    backgroundcolor = gray!30,
    hidealllines = true
]{mydef}{Definición}[section]

\newmdtheoremenv[
    leftmargin=0em,
    rightmargin=0em,
    innertopmargin=-2pt,
    innerbottommargin=8pt,
    hidealllines = true,
    roundcorner = 5pt,
    backgroundcolor = gray!60!red!30
]{exa}{Ejemplo}[section]

\newmdtheoremenv[
    leftmargin=0em,
    rightmargin=0em,
    innertopmargin=-2pt,
    innerbottommargin=8pt,
    hidealllines = true,
    roundcorner = 5pt,
    backgroundcolor = gray!50!blue!30
]{obs}{Observación}[section]

%En esta parte se colocan comandos que definen la forma en la que se van a escribir ciertas funciones%

\newcommand\abs[1]{\ensuremath{\lvert#1\rvert}}
\newcommand\divides{\ensuremath{\bigm|}}
\newcommand\cf[3]{\ensuremath{#1:#2\rightarrow#3}}
\renewcommand{\theenumi}{\Roman{enumi}}

%recuerda usar \clearpage para hacer un salto de página

\begin{document}
\title{Lista 4 AnM III}
\author{Cristo Daniel Alvarado}
\date{\today}
\maketitle

\tableofcontents
\chapter{Lista 4}

\section{Ejercicios}
    \begin{excer}
        \begin{enumerate}
            \item Sea $f:[0,1]\rightarrow \mathbb{R}$ la función
            \begin{equation*}
                f(x)=\int_{0}^{x+e^{x^2}}\log\left(\frac{t+1}{t^2+1}\right)dt,\quad\forall x\geq0.
            \end{equation*}
            \textbf{Verifique} que $f$ está bien definida y que es derivable en $[0,\infty[$. \textbf{Calcule} $f'(x)$, $\forall x\geq 0$.
            \item Sea $f:[0,\infty[\rightarrow\mathbb{R}$ la función
            \begin{equation*}
                f(x)=\left(\int_{0}^{\sqrtsign{x+1}\log\left(x^2+1\right)}e^{-t^2+1}dt\right)^2, \quad \forall x\geq0.
            \end{equation*}
            \textbf{Demuestre} que $f$ está bien definida y que es derivable en $[0,\infty[$. \textbf{Calcule} $f'(x)$, $\forall x\geq 0$.
        \end{enumerate}
    \end{excer}
    
    \begin{proof}
        De (I): Veamos que la función está bien definida. Para ello, notemos que la función $t\mapsto\log\left(\frac{t+1}{t^2+1}\right)$ es integrable en todo subintervalo compacto de $[0,\infty[$ (por ser continua). Luego $f$ está bien definida. Notemos que podemos reescribir a $f$ como
        \begin{equation}
            f(x)=G\circ h(x)
        \end{equation}
        donde $G(x)=\int_{0}^{x}\log\left(\frac{t+1}{t^2+1}\right)dt$ y $h(x)=x+e^{x^2}$. Siendo que $t\mapsto\log\left(\frac{t+1}{t^2+1}\right)$ es integrable en todo subintervalo compacto, la función $G$ es diferenciable c.t.p. en $[0,\infty[$, y
        \begin{equation*}
            G'(x)=\log\left(\frac{x+1}{x^2+1}\right)
        \end{equation*}
        c.t.p. en $[0,\infty[$. Para ver que la igualdad es en todo el dominio de la función, basta ver que la función $G'$ es continua en $[0,\infty[$. En efecto, notemos que
    \end{proof}

    \begin{excer}
        Sea $\cf{f}{[-\pi, \pi]}{\mathbb{C}}$ una función integrable en $[-\pi, \pi]$. Sea $\cf{F}{[-\pi, \pi]}{\mathbb{C}}$ la integral indefinida de $f$, dada por
        \begin{equation*}
            F(x)=\int_{-\pi}^{x}\left(f(t)-c\right)dt,\quad\forall x\in[-\pi, \pi]
        \end{equation*}
        donde $c$ es constante. Defina
        \begin{equation*}
            c_k'=\int_{-\pi}^{\pi}f(x)e^{-ikx}dx\quad\textup{y}\quad c_k=\int_{-\pi}^{\pi}F(x)e^{-ikx}dx
        \end{equation*}
        donde $k\in\mathbb{Z}$. \textbf{Determine} la relación entre $c_k$ y $c_k'$.
    \end{excer}

    \begin{proof}
        Determinemos la relación entre las variables $c_k$ y $c_k'$. Para ello, notemos que podemos escribir
        \begin{equation*}
            \begin{split}
                c_k = & \int_{-\pi}^{\pi}F(x)e^{-ikx}dx\\
                = & \int_{-\pi}^{\pi}\left[\int_{-\pi}^{x}\left(f(t)-c\right)dt\right]e^{-ikx}dx\\
            \end{split}
        \end{equation*}        
    \end{proof}

    \begin{excer}
        Integrando por partes, \textbf{calcule} las integrales siguientes
        \begin{equation*}
            \int_{a}^{b}e^{\alpha x}\sin(\beta x)dx\quad\textup{y}\quad\int_{a}^{b}e^{\alpha x}\cos(\beta x)dx
        \end{equation*}
        donde $\alpha$ y $\beta$ son constantes.
    \end{excer}

    \begin{proof}
        
    \end{proof}

    \begin{excer}
        Sean $0<a<1$ y $-\pi<\lambda<\pi$ fijos. \textbf{Verifique} que existen las integrales siguientes y mediante una integración por partes \textbf{pruebe} la identidad:
        \begin{equation*}
            \int_{0}^{\infty}\frac{-e^{i\lambda}x^a}{(e^{i\lambda}x+1)^2}dx=\int_{0}^{\infty}\frac{ax^{a-1}}{e^{i\lambda}x+1}dx
        \end{equation*}
    \end{excer}

    \begin{proof}
        
    \end{proof}

    \begin{excer}
        \textbf{Determine} cuáles de las siguientes funciones son de clase $C^1$, de variación acotada y/o absolutamente continuas en $[0,1]$.
        \begin{enumerate}
            \item \begin{equation*}
                f(x) = 
                \left \{
                    \begin{aligned}
                    x^{4/3}\sin\left(\frac{1}{\sqrt{x}}\right) &,\ \text{si} \ x \in ]0,1]\\
                    0 &,\ \text{si} \ x=0
                    \end{aligned}
                \right .
            \end{equation*}
            \item \begin{equation*}
                f(x) = 
                \left \{
                    \begin{aligned}
                    x^2\sin\left(\frac{1}{x^2}\right) &,\ \text{si} \ x \in ]0,1]\\
                    0 &,\ \text{si} \ x=0
                    \end{aligned}
                \right .
            \end{equation*}
        \end{enumerate}
    \end{excer}

    \begin{proof}
        
    \end{proof}

    \begin{excer}
        \textbf{Proporcione} un contraejemplo de una función que sea absolutamente continua en todo intervalo $[a,1]$, $0< a\leq 1$, y continua en cero, pero que no sea absolutamente continua en $[0,1]$.
    \end{excer}

    \begin{proof}
        
    \end{proof}

    \begin{excer}
        \textbf{Demuestre} que si una función $f$ es absolutamente continua en todo intervalo $[a,1]$, $0< a\leq 1$, continua en cero y de variación acotada en $[0,1]$, entonces $f$ es absolutamente continua en $[0,1]$.
        
        \textit{Sugerencia.} Verifique que $f'$ es integrable en $[0,1]$ y utilice la función $G(x)=\int_{1}^{x}f'$, $\forall x\in]0,1]$.
    \end{excer}

    \begin{proof}
        
    \end{proof}

    \begin{excer}
        \textbf{Determine} si la función $f(x)=x^{1/2}$ es o no de clase $C^1$, de variación acotada y/o absolutamente continua en $[0,1]$.
    \end{excer}

    \begin{proof}
        
    \end{proof}

    \begin{excer}
        \textbf{Determine} si la función
        \begin{equation*}
            f(x) = 
                \left \{
                    \begin{aligned}
                    x^{3/2}\sin\left(\frac{1}{x^2}\right) &,\ \textup{si} \ x \in ]0,1]\\
                    0 &,\ \textup{si} \ x=0
                    \end{aligned}
                \right .
        \end{equation*}
        es o no de clase $C^1$, absolutamente continua y/o de variación acotada en $[0,1]$.
    \end{excer}

    \begin{proof}
        
    \end{proof}

    \begin{excer}
        Se dice que una función $f:[a,b]\rightarrow \mathbb{K}$ satisface la \textbf{condición de Lipschitz} en $[a,b]$ si existe una constante $M\geq 0$ tal que
        \begin{equation*}
            \abs{f(x)-f(y)}\leq M\abs{x-y},\quad\forall x,y\in [a,b].
        \end{equation*}
        \begin{enumerate}
            \item \textbf{Pruebe} que si $f$ satisface la condición de Lipschitz en $[a,b]$, entonces $f$ es absolutamente continua en $[a,b]$.
            \item \textbf{Demuestre} que si $f$ es absolutamente continua, entonces $f$ satisface la condición de Lipschitz en $[a,b]$ si y sólo si $\abs{f'}$ es acotada en $[a,b]$.
            \item Si $f$ es continua en $[a,b]$ y una de sus derivadas (digamos $D^+$) es acotada en $[a,b]$, \textbf{muestre} que $f$ satisface la condición de Lipschitz en $[a,b]$.
        \end{enumerate}
    \end{excer}

    \begin{proof}
        De (I): Suponga que $f$ satisface la condición de Lipschitz, es decir existe una constante $M\geq0$ tal que
        \begin{equation*}
            \abs{f(x)-f(y)}\leq M\abs{x-y}<(M+1)\abs{x-y}
        \end{equation*}
        Lo haremos por la definición. Sea $\varepsilon>0$, tomemos $\delta=\frac{\varepsilon}{M+1}>0$. Si $\left\{]x_k,x_k'[\right\}_{k=1}^{m}$ es una familia de intervalos abiertos disjuntos contenidos en $[a,b]$ tales que
        \begin{equation*}
            \sum_{k=1}^{m}\abs{x_k'-x_k}=\sum_{k=1}^{m}(x_k'-x_k)\leq\delta =\frac{\varepsilon}{M+1}
        \end{equation*}
        entonces, por la condición de Lipschitz se tiene
        \begin{equation*}
            \sum_{k=1}^{m}\abs{f(x_k')-f(x_k)}\leq\sum_{k=1}^{m}M\abs{x_k'-x_k}<(M+1)\cdot\frac{\varepsilon}{M+1}=\varepsilon
        \end{equation*}
        Luego, $f$ es absolutamente continua en $[a,b]$.

        De (II): Suponga que $f$ es absoultamente continua.

        $\Rightarrow$): Suponga que $f$ satisface la condición de Lipschitz, entonces existe $M\geq0$ tal que
        \begin{equation*}
            \abs{f(x)-f(y)}\leq M\abs{x-y}<(M+1)\abs{x-y}
        \end{equation*}
        Como $f$ es de variación acotada, es diferenciable c.t.p. en $[a,b]$. Sea $A$ el conjunto de puntos en los que $f$ es diferenciable. Si $x\in A$ y $y\in[a,b]/\left\{x\right\}$ se tiene que
        \begin{equation*}
            \begin{split}
                \abs{f(x)-f(y)}\leq &M\abs{x-y}\\
                \Rightarrow \abs{\frac{f(x)-f(y)}{x-y}}\leq&M\\
            \end{split}
        \end{equation*}
        siendo $y\in[a,b]/\left\{x\right\}$ arbitrario. Tomando límites con respecto a $x$ se tiene que
        \begin{equation*}
            \abs{f'(x)}\leq M
        \end{equation*}
        para todo $x\in A$. Es decir, la función $f'$ definida c.t.p. en $A$ es acotada (en particular lo es en $[a,b]$).

        $\Leftarrow$): Suponga que $f'$ es acotada en $[a,b]$. 

        %Te quedaste acá%

    \end{proof}

    \begin{excer}
        \textbf{Determine} si la función
        \begin{equation*}
            f(x) = 
                \left \{
                    \begin{aligned}
                    x^2\sin\left(\frac{1}{x}\right) &,\ \text{si} \ x \in ]0,1] \\
                    0 &,\ \text{si} \ x=0
                    \end{aligned}
                \right .
        \end{equation*}
        es o no de clase $C^1$, absoultamente continua y/o de variación acotada en $[0,1]$.
    \end{excer}
    \begin{proof}
        
    \end{proof}

    \begin{excer}
        Considere la función $\cf{f}{[-1,1]}{\mathbb{R}}$,
        \begin{equation*}
            f(x) = 
                \left \{
                    \begin{aligned}
                    x\sin\left(\frac{1}{x}\right) &,\ \text{si} \ x \in [-1,1]/\left\{0\right\} \\
                    0 &,\ \text{si} \ x=0
                    \end{aligned}
                \right .
        \end{equation*}
        \textbf{Calcule} las cuatro derivadas de $f$ en cero. ¿Es $f$ de variación acotada en $[-1,1]$?
    \end{excer}

    \begin{proof}
        
    \end{proof}

    \begin{excer}
        Si $\cf{f}{[a,b]}{\mathbb{R}}$ es continua en $[a,b]$ y alcanza un máximo local en un punto $c\in]a,b[$, \textbf{pruebe} que
        \begin{equation*}
            D_+f(x)\leq D^+f(x)\leq D_-f(c)\leq D^-f(c)
        \end{equation*}
        ¿Qué relaciones se darían si $f$ alcanzara un máximo local en $a$ o $b$?
    \end{excer}
    
    \begin{proof}
        
    \end{proof}

    \begin{excer}
        Si $\cf{f}{[a,b]}{\mathbb{R}}$ es continua y alguna de sus derivadas (digamos $D^+$) es no negativa en todo punto de $[a,b]$, \textbf{demuestre} que $f(b)\geq f(a)$.
    \end{excer}

    \begin{proof}
        
    \end{proof}

    \begin{excer}
        Sean $\cf{f,g}{[a,b]}{\mathbb{R}}$ dos funciones
        \begin{enumerate}
            \item \textbf{Muestre} que $D^+(f+g)\leq D^+f+D^+g$ en todo punto de $[a,b[$. \textbf{Establezca} desigualdades similares para las otras derivadas.
            \item \textbf{Pruebe} que si $f$ y $g$ son no negativas y continuas en un punto $c\in[a,b[$, entonces
            \begin{equation*}
                D^+(fg)(c)\leq f(c)D^+g(c)+g(c)D^+f(c)
            \end{equation*}
        \end{enumerate}
    \end{excer}

    \begin{proof}
        
    \end{proof}

    \begin{excer}
        \textbf{Proporcione} un ejemplo de una función monótona en $[0,1]$ que sea discontinua en cada número racional.
    \end{excer}

    \begin{proof}
        
    \end{proof}

    \begin{excer}
        \textbf{Demuestre} las proposiciones 4.18, 4.21 y 4.36.
    \end{excer}

    \begin{proof}
        
    \end{proof}

    \begin{excer}
        Sea $\left\{f_\nu\right\}_{\nu=1}^{\infty}$ una sucesión de funciones en $\mathscr{V}([a,b],\mathbb{K})$ que converge puntualmente en $[a,b]$ a una función $f$. \textbf{Pruebe} que
        \begin{equation*}
            V_f([a,b])\leq \liminf_{\nu\rightarrow\infty}V_{f_\nu}([a,b])
        \end{equation*}
    \end{excer}

    \begin{proof}
        
    \end{proof}

    \begin{excer}
        \textbf{Pruebe} que si $\cf{f}{[a,b]}{\mathbb{R}}$ es una función de variación acotada en $[a,b]$, entonces
        \begin{equation*}
            \int_{a}^{b}\abs{f'}\leq V_f([a,b]);
        \end{equation*}
        y si $f$ es, además, absolutamente continua en $[a,b]$, \textbf{muestre} que
        \begin{equation*}
            \int_{a}^{b}\abs{f'}=V_f([a,b]).
        \end{equation*}
    \end{excer}

    \begin{proof}
        
    \end{proof}

    \begin{excer}
        \textbf{Demuestre} que si $\cf{f}{[0,1]}{\mathbb{R}}$ es una función monótona, absolutamente continua en $[0,1]$ y $A$ es un conjunto despreciable contenido en $[0,1]$, entonces $f(A)$ es un conjunto despreciable.
    \end{excer}

    \begin{proof}
        Sea $A$ un conjunto despreciable, es decir $m\left(a\right)=0$.
    \end{proof}

    \begin{excer}
        Haga lo siguiente
        \begin{enumerate}
            \item \textbf{Proporcione} un ejemplo de una función $f$ estrictamente creciente y absolutamente continua en $[0,1]$ tal lque $f'=0$ en algún conjunto con medida positiva.
            
            \textit{Sugerencia.} Sea $A$ el complemento de un conjunto de Cantor generalizado $\mathscr{C}_\alpha$ con medida $1-\alpha>0$. Tome como $f$ la integral indefinida de $\chi_A$.
            \item \textbf{Pruebe} que existe un conjunto despreciable $E$ contenido en $[0,1]$ tal que $f^{-1}(E)$ no es medible.
        \end{enumerate}
    \end{excer}

    \begin{proof}
        
    \end{proof}

    \begin{excer}[\textbf{Teorema de Cambio de Variable}]
        Sea $\cf{\phi}{[a,b]}{[c,d]}$ una función creciente absolutamente continua en $[a,b]$ tal que $\phi(a)=c$ y $\phi(b)=d$. Observe que $\phi$ \textbf{NO} necesariamente debe ser un isomorfismo $C^1$ de $[a,b]$ en $[c,d]$.
        \begin{enumerate}
            \item \textbf{Muestre} que para cualquier conjunto abierto $W$ contenido en $[c,d]$ se cumple
            \begin{equation*}
                m(W)=\int_{\phi^{-1}(W)}\phi'(x)dx.
            \end{equation*}
            \item Sea $N=\left\{x\in[a,b]|\phi'(x)\neq0\right\}$. Si $C$ es un subconjunto despreciable de $[c,d]$, \textbf{demuestre} que $\phi^{-1}(C)\cap N$ es un subconjunto despreciable de $[a,b]$.
            \item Si $D$ es un subconjunto medible de $[c,d]$, \textbf{pruebe} que $B=\phi^{-1}(D)\cap N$ es un subconjunto medible de $[a,b]$ y
            \begin{equation*}
                m(D)=\int_{B}\phi'=\int_{c}^{d}\chi_D(\phi(x))\phi'(x)dx.
            \end{equation*}
            \item Si $\cf{f}{[c,d]}{\mathbb{R}}$ es una función medible no negativa, \textbf{muestre} que la función $\cf{(f\circ\phi)\cdot\phi'}{[a,b]}{\mathbb{R}}$ es medible no negativa y
            \begin{equation*}
                \int_{c}^{d}f(y)dy=\int_{a}^{b}f(\phi(x))\phi'(x)dx.
            \end{equation*}
        \end{enumerate}
    \end{excer}

    \begin{proof}
        Esta cosa es una prueba.
    \end{proof}
    
\end{document}