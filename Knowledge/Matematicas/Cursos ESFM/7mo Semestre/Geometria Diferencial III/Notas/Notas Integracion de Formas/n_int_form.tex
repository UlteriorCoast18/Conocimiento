\documentclass[12pt]{report}
\usepackage[spanish]{babel}
\usepackage[utf8]{inputenc}
\usepackage{amsmath}
\usepackage{amssymb}
\usepackage{amsthm}
\usepackage{graphics}
\usepackage{subfigure}
\usepackage{lipsum}
\usepackage{array}
\usepackage{multicol}
\usepackage{enumerate}
\usepackage[framemethod=TikZ]{mdframed}
\usepackage[a4paper, margin = 1.5cm]{geometry}

%En esta parte se hacen redefiniciones de algunos comandos para que resulte agradable el verlos%

\def\proof{\paragraph{Demostración:\\}}
\def\endproof{\hfill$\square$}
\renewcommand{\theenumii}{\roman{enumii}}

%En esta parte se definen los comandos a usar dentro del documento para enlistar%

\newtheoremstyle{largebreak}
  {}% use the default space above
  {}% use the default space below
  {\normalfont}% body font
  {}% indent (0pt)
  {\bfseries}% header font
  {}% punctuation
  {\newline}% break after header
  {}% header spec

\theoremstyle{largebreak}

\newmdtheoremenv[
    leftmargin=0em,
    rightmargin=0em,
    innertopmargin=-2pt,
    innerbottommargin=8pt,
    hidealllines = true,
    roundcorner = 5pt,
    backgroundcolor = gray!60!red!30
]{exa}{Ejemplo}[section]

\newmdtheoremenv[
    leftmargin=0em,
    rightmargin=0em,
    innertopmargin=-2pt,
    innerbottommargin=8pt,
    hidealllines = true,
    roundcorner = 5pt,
    backgroundcolor = gray!50!blue!30
]{obs}{Observación}[section]

\newmdtheoremenv[
    leftmargin=0em,
    rightmargin=0em,
    innertopmargin=-2pt,
    innerbottommargin=8pt,
    rightline = false,
    leftline = false
]{theor}{Teorema}[section]

\newmdtheoremenv[
    leftmargin=0em,
    rightmargin=0em,
    innertopmargin=-2pt,
    innerbottommargin=8pt,
    rightline = false,
    leftline = false
]{propo}{Proposición}[section]

\newmdtheoremenv[
    leftmargin=0em,
    rightmargin=0em,
    innertopmargin=-2pt,
    innerbottommargin=8pt,
    rightline = false,
    leftline = false
]{cor}{Corolario}[section]

\newmdtheoremenv[
    leftmargin=0em,
    rightmargin=0em,
    innertopmargin=-2pt,
    innerbottommargin=8pt,
    rightline = false,
    leftline = false
]{lema}{Lema}[section]

\newmdtheoremenv[
    leftmargin=0em,
    rightmargin=0em,
    innertopmargin=-2pt,
    innerbottommargin=8pt,
    roundcorner=5pt,
    backgroundcolor = gray!30,
    hidealllines = true
]{mydef}{Definición}[section]

\newmdtheoremenv[
    leftmargin=0em,
    rightmargin=0em,
    innertopmargin=-2pt,
    innerbottommargin=8pt,
    roundcorner=5pt
]{excer}{Ejercicio}[section]

%En esta parte se colocan comandos para poner en específico teoremas anidados al capítulo%

\newcommand{\Det}{\textup{Det}}

\begin{document}
    \title{Notas Integracion de Formas GD III}
    \author{Cristo Daniel Alvarado}
    \date{Diciembre de 2023}
    \maketitle

    \tableofcontents %Con este comando se genera el índice general del libro%

    \setcounter{chapter}{2} %En esta parte lo que se hace es cambiar la enumeración del capítulo%
    \chapter{Integración de formas}
    
    \section{Introducción}
    El objetivo 
    \section{Lema de Poincaré para formas con soporte compacto en rectángulos acotados}
    
    \begin{mydef}
        Sea $\nu$ una \textit{k}-forma en $\mathbb{R}^{n}$. Definimos el \textbf{suporte de $\nu$} como
        \begin{equation*}
            \textup{supp}\left(\nu\right)=\bar{\left\{x\in\mathbb{R}^n|\nu_x\neq0\right\}}
        \end{equation*}
        y decimos que $\nu$ tiene \textbf{soporte compacto} si supp$(\nu)$ es compacto.
    \end{mydef}

    \begin{theor}[\textbf{Lema de Poincaré para rectángulos}]
        Sea $\omega$ una n-forma con soporte compacto tal que $\textup{supp}(\omega)\subseteq\textup{int}(Q)$. Entonces los siguientes son equivalentes:
        \begin{enumerate}
            \item $\int \omega = 0$.
            \item Existe una (n-1)-forma $\mu$ con soporte compacto tal que $\textup{supp}(\mu)\subseteq\textup{int}(Q)$ que satisface $d\mu=\omega$.
        \end{enumerate}
    \end{theor}

    \begin{proof}
        $(1)\Rightarrow(2)$: Sea
        \begin{equation}
            \mu=\sum_{i=1}^{n}f_i dx_1\wedge\cdots\wedge\hat{dx_i}\wedge\cdots\wedge dx_n
        \end{equation}
    \end{proof}

    \begin{theor}
        Sea $U\subseteq\mathbb{R}^{n-1}$ un abierto y $A\subseteq\mathbb{R}$ un intervalo abierto. Entonces si $U$ tiene la propiedad $P$, entonces $U\times A$ también la tiene.
    \end{theor}

    \begin{obs}
        Es muy sencillo ver que el intervalo abierto $A$ por si mismo tiene la propiedad $P$
    \end{obs}

    \section{Lema de Poincaré para formas con soporte compacto en abiertos conexos en $\mathbb{R}^n$}

    \begin{theor}
        Sea $U\subseteq\mathbb{R}^n$ un abierto conexo y $\omega$ una n-forma con soporte compacto tal que $\textup{supp}(\omega)\subseteq U$. Entonces las siguientes son equivalentes:
        \begin{enumerate}
            \item $\int_{\mathbb{R}^n}\omega = 0$.
            \item Existe una (n-1)-forma con soporte compacto $\mu$ tal que $\textup{supp}(\mu)\subseteq U$ y $d\mu = \omega$.
        \end{enumerate}
    \end{theor}
    
    \begin{proof}
        $(2)\Rightarrow(1)$: Es inmediata de la parte anterior.

        $(1)\Rightarrow(2)$: Supongase que
        \begin{equation*}
            \int_{\mathbb{R}^n}\omega = 0.
        \end{equation*}
        para probar el resultado se probará un resultado anterior
        \begin{theor}
            Si $\omega$ es una n-forma con soporte compacto tal que $\textup{supp}(\omega)\subseteq U$ y $c=\int_{\mathbb{R}^n}\omega$ entonces $\omega \sim c\omega_0$.
        \end{theor}
        \begin{proof}
            Para demostrar el teorema, sea $\left\{Q_i\right\}_{i\in\mathbb{N}}$ una familia de rectángulos acotados contenidos en $U$ tales que $\cup_{i\in\mathbb{N}}\textup{int}(Q_i)=U$ y sea $\left\{\phi_i\right\}_{i\in\mathbb{N}}$ una partición de la unidad tal que $\textup{supp}(\phi_i)\subseteq \textup{int}(Q_i)$, es decir, se tiene que
            \begin{enumerate}
                \item $\sum_{i\in\mathbb{N}}\phi_i=1$.
                \item $\textup{supp}(\phi_i)\subseteq \textup{int}(Q_i),\forall i\in\mathbb{N}$.
            \end{enumerate}
            Para la demostración del teorema, asumiremos la condición de finitud local, es decir\dots
            Dado a que el soporte de $\omega$ está contenido en el abierto $U$, entonces se debe tener que existe un $m\in\mathbb{N}$ tal que
            \begin{equation*}
                \textup{supp}(\omega)\subseteq\cup_{i=1}^{m}Q_i
            \end{equation*}
            De esta forma, podemos escribir a $\omega$ como:
            \begin{equation*}
                \omega = \sum_{i=1}^{m}\phi_i \omega
            \end{equation*}
            Donde, para cada $i=1,\dots,m$ se tiene que $\textup{supp}(\phi_i \omega)\subseteq \textup{int}(Q_i)$. Con esto se tiene que
            \begin{equation*}
                c=\int_{\mathbb{R}^n}\omega = \sum_{i=1}^{m}\int_{\mathbb{R}^n}\phi_i\omega=\sum_{i=1}^{m}c_i, \quad \textup{donde }c_i=\int_{\mathbb{R}^n}\phi_i\omega 
            \end{equation*}
            Para continuar con la demostración, se asumirá el siguiente resultado:
            \begin{lema}
                Para cada $j=1\cdots,m$ existe una sucesión de rectangulos acotados $\left\{R_{i,j}\right\}_{i=0}^{N+1}$ tales que $R_{0,j}=Q_0$ y $R_{N+1,j}=Q_j$
            \end{lema}
            Con esto en mente, podemos elegir para cada $j=1\cdots m$, $N+2$ n-formas $\nu_{k,j}$ (con $k=1,\cdots N+2$) tales que
            \begin{equation*}
                \begin{split}
                    \textup{supp}(\nu_{k,j})\subseteq& \textup{int}(R_{k,j})\cap\textup{int}(R_{k,j})\\
                    \int_{\mathbb{R}^n}\nu_{k,j}=&1
                \end{split}
            \end{equation*}
            En particular, se cumple que 
            \begin{equation*}
                \textup{supp}(\nu_{k,j}-\nu_{k+1,j})\subseteq\textup{int}(Q_{k+1,j})
            \end{equation*}
            Por lo cual, integrando se debe tener que
            \begin{equation*}
                \int_{\mathbb{R}^n}\nu_{k,j}-\nu_{k+1,j}=0
            \end{equation*}
            Por el lema de Poincaré para rectángulos, se tiene que la n-forma que estamos integrando es exacta, por lo cual se puede ver como
            \begin{equation*}
                \nu_{k,j}-\nu_{k+1,j}=d\mu_{k+1,j}
            \end{equation*}
            Cumpliendo esta n-forma que
            \begin{equation*}
                \textup{supp}(\mu_{k+1,j})\subseteq \textup{int}(Q_{k+1,j})
            \end{equation*}
            Para cada $k=0,\cdots, N+1$. Por otro lado $\omega_0 \sim \nu_{0,j}$, pues la integral de su diferencia es 0 y el soporte de su diferencia está contenido en el interior del $Q_0$. De esta forma escribimos
            \begin{equation*}
                \omega_0-\nu_{0,j}=d\mu_{0,j},\quad \textup{con supp}(\mu_{0,j})\subseteq\textup{int}(Q_0)
            \end{equation*}
            También, se afirm que $c_i\mu_{N+1,j}\sim\phi_i\omega$. En efecto, pues como en lo anterior se tiene que la diferencia de sus integrales es cero y el soporte de cada una está contenido en un rectángulo, aplicando el lema de Poincaré para rectángulos se sigue el resultado. De esta forma se puede escribir
            \begin{equation*}
                c_j\mu_{N+1,j}-\phi_j\omega =dc_j\mu_{N+2,j}
            \end{equation*}
            en resumen
            \begin{equation*}
                \begin{split}
                    c_j\omega_0 -c_j\omega_{0,j}=&d(c_j\omega_{0,j})\\
                    c_j\omega_0 -c_j\omega_{1,j}=&d(c_j\omega_{1,j})\\
                    c_j\omega_1 -c_j\omega_{2,j}=&d(c_j\omega_{2,j})\\
                    c_j\omega_N -c_j\omega_{N+1,j}=&d(c_j\omega_{N+1,j})\\
                    c_j\omega_{N+1} -\phi_j\omega=&d(c_j\omega_{N+2,j})\\
                \end{split}
            \end{equation*}
            Sumando todas las ecuaciones anteriores, se sigue que
            \begin{equation*}
                c_j\omega_0-\phi_j\omega = d\left(c_j\sum_{k=0}^{N+2}\mu_{k,j}\right)=-d\lambda_j
            \end{equation*}
            donde $\lambda_j$ es una (n-1)-forma que se define de forma natural como la suma. Así:
            \begin{equation*}
                \phi_j\omega = c_j\omega_0+d\lambda_j, \quad j=1,\cdots,m
            \end{equation*}
            Finalmente
            \begin{equation*}
                \begin{split}
                    \omega =&\sum_{j=1}^{m}\phi_i\omega\\
                    =&\sum_{j=1}^{m}\left(c_j\omega_0+d\lambda_j\right)\\
                    =&\omega_0\sum_{j=1}^{m}\left(c_j\right)+d\left(\sum_{j=1}^{m}\lambda_j\right)\\
                \Rightarrow \omega - c\omega_0 =& d\Lambda\\
                \end{split}
            \end{equation*}
            Donde $\Lambda$ es una (n-1)-forma con soporte en U. Por tanto $\omega \sim c\omega_0$, lo que prueba al resultado.
        \end{proof}
        
        Por este teorema, como c = 0, se sigue que, usando la notación del teorema anterior, podemos escribir $w=d\Lambda$.
    \end{proof}

    \section{El grado de un mapeo diferencial}

    \begin{mydef}
        Sean $U$ y $V$ subconjuntos abiertos en $\mathbb{R}^n$ y $\mathbb{R}^m$. Un mapeo continuo $f:U\rightarrow V$, es llamado \textbf{propio} si para cada subconjunto compacto $K\subseteq V$, la imagen inversa $f^{-1}(K)$ es compacto en $U$.
    \end{mydef}

    Este tipo de mapeos tienen (entre muchas otras), las siguientes propiedades:

    \begin{obs}
        Sea $f:U\rightarrow V$ un mapeo propio $C^{\infty}$ y $\omega$ una k-forma en $\mathbb{R}^m$ con soporte compacto tal que $\textup{supp}(\omega)\subseteq V$, entonces $f*\omega$ es una k-forma en $\mathbb{R}^n$ tal que $\textup{supp}(f*\omega)\subseteq U$.
    \end{obs}

    En particular, si $U,V\subseteq \mathbb{R}^n$ son abiertos conexos y $f:U\rightarrow V$ es un mapeo propio $C^{\infty}$ entonces existe un invariante topológico de $f$, llamado el \textbf{grado de f} (denotado por $\textup{deg}(f))$ tal que
    \begin{equation*}
        \int_{U}f^*\omega=\textup{deg}(f)\int_{V}\omega
    \end{equation*}
    para toda $\omega\in\Omega_c^n(V)$

    La parte de que sea un invariante topológico es extremandamente complicado de probar (ahora mismo), pero se probará la identidad de la ecuación anterior. Veamos a que se refiere la ecuación anterior. Si
    \begin{equation*}
        \omega = \phi(y)dy_1\wedge\cdots\wedge dy_n
    \end{equation*}
    (para esto, estamos en el acuerdo de que $x=\left(x_1,\dots,x_n\right)\in U$ y $y=\left(y_1,\dots, y_n\right)\in V$) Con $\phi:V\rightarrow \mathbb{R}$ es una función $C_0^{\infty}(V)$. Entonces en $x\in U$:
    \begin{equation*}
        f^*\omega = \left(\phi\circ f\right)(x)\det\left(Df(x)\right)dx_1\wedge\cdots\wedge dx_n.
    \end{equation*}
    Por lo cual, la igualdad anterior lo que nos dice es que
    \begin{equation*}
        \int_{U}\left(\phi\circ f\right)(x)\det(Df(x))dx=\deg\left(f\right)\int_{V}\phi(y)dy
    \end{equation*}
    Probaremos el resultado
    \begin{proof}
        Sea $\omega_0\in\Omega_c^n(V)$ tal que
        \begin{equation*}
            \begin{split}
                \textup{supp}(\omega_0)\subseteq& V\\
                \int_{V}\omega_0 =1
            \end{split}
        \end{equation*}
        Si $\deg(f)=\int_{U}f^*\omega$, el resultado se sigue de forma inmediata (usando la expresión obtendida anteriormente).
        
        Para el caso general, por un teorema anterior, se tiene que
        \begin{equation*}
            \omega -c\omega_0 = d\mu
        \end{equation*}
        donde $\mu$ es una (n-1)-forma tal que $\textup{supp}(\mu)\subseteq V$ y $c=\int_{V}\omega$. Como el pullback conmuta con la diferencial exterior, se tiene entonces de la ecuación anterior que
        \begin{equation*}
            \begin{split}
                \int_{U}f^*\omega-c\int_{U}f^*\omega_0 =& \int_{U}d(f^*\mu)\\
                =& 0\\
            \end{split}
        \end{equation*}
        ya que $f^*\mu$ es una (n-1)-forma en el abierto $V$, el cual es conexto, luego por el Lema de Poincaré, su integral es cero y así:
        \begin{equation*}
            \begin{split}
                \int_{U}f^*\omega=&c\int_{U}f^*\omega_0\\
            \end{split}
        \end{equation*}
        pero $\int_{U}f^*\omega_0=\deg(f)$ y $c=\int_{V}\omega$. Por tanto:
        \begin{equation}
            \int_{U}f^*\omega=\deg(f)\int_{V}\omega
        \end{equation}
    \end{proof}
    \begin{propo}
        Sean $U,V,W\subseteq \mathbb{R}^n$ conjuntos abiertos conexos y $f:U\rightarrow V$, $g:V \rightarrow W$ mapeos propios $C^\infty$. Entonces
        \begin{equation}
            \deg(g\circ f)=\deg(f)\deg(g)\label{equation:mult_p_deg}
        \end{equation}
    \end{propo}

    \begin{proof}
        
    \end{proof}

    De la ecuación (\ref{equation:mult_p_deg}) se deduce de forma inmediata el siguiente resultado

    \begin{theor}
        Sea $A$ una matriz no singular $n\times n$ y $f_A:\mathbb{R}^n\rightarrow \mathbb{R}^n$ el mapeo asociado a $A$. Entonces $\deg(f_A)=1$ si $\Det(A)>0$ y $\deg(f_A)=-1$ si $\Det(A)<0$.
    \end{theor}

    \begin{proof}
        
    \end{proof}

    \section{La fórmula de cambio de variable}

\end{document}