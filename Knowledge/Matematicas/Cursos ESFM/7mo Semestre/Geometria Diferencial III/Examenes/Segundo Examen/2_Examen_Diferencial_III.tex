\documentclass[12pt]{report}
\usepackage[spanish]{babel}
\usepackage[utf8]{inputenc}
\usepackage{amsmath}
\usepackage{amssymb}
\usepackage{amsthm}
\usepackage{graphics}
\usepackage{subfigure}
\usepackage{lipsum}
\usepackage{array}
\usepackage{multicol}
\usepackage{enumerate}
\usepackage[framemethod=TikZ]{mdframed}
\usepackage[a4paper, margin = 1.5cm]{geometry}

%En esta parte se hacen redefiniciones de algunos comandos para que resulte agradable el verlos%

\def\proof{\paragraph{Demostración:\\}}
\def\endproof{\hfill$\square$}
\renewcommand{\theenumii}{\roman{enumii}}

%En esta parte se definen los comandos a usar dentro del documento para enlistar%

\newtheoremstyle{largebreak}
  {}% use the default space above
  {}% use the default space below
  {\normalfont}% body font
  {}% indent (0pt)
  {\bfseries}% header font
  {}% punctuation
  {\newline}% break after header
  {}% header spec

\theoremstyle{largebreak}

\newmdtheoremenv[
    leftmargin=0em,
    rightmargin=0em,
    innertopmargin=-2pt,
    innerbottommargin=8pt,
    hidealllines = true,
    roundcorner = 5pt,
    backgroundcolor = gray!60!red!30
]{exa}{Ejemplo}[section]

\newmdtheoremenv[
    leftmargin=0em,
    rightmargin=0em,
    innertopmargin=-2pt,
    innerbottommargin=8pt,
    hidealllines = true,
    roundcorner = 5pt,
    backgroundcolor = gray!50!blue!30
]{obs}{Observación}[section]

\newmdtheoremenv[
    leftmargin=0em,
    rightmargin=0em,
    innertopmargin=-2pt,
    innerbottommargin=8pt,
    rightline = false,
    leftline = false
]{theor}{Teorema}[section]

\newmdtheoremenv[
    leftmargin=0em,
    rightmargin=0em,
    innertopmargin=-2pt,
    innerbottommargin=8pt,
    rightline = false,
    leftline = false
]{propo}{Proposición}[section]

\newmdtheoremenv[
    leftmargin=0em,
    rightmargin=0em,
    innertopmargin=-2pt,
    innerbottommargin=8pt,
    rightline = false,
    leftline = false
]{cor}{Corolario}[section]

\newmdtheoremenv[
    leftmargin=0em,
    rightmargin=0em,
    innertopmargin=-2pt,
    innerbottommargin=8pt,
    rightline = false,
    leftline = false
]{lema}{Lema}[section]

\newmdtheoremenv[
    leftmargin=0em,
    rightmargin=0em,
    innertopmargin=-2pt,
    innerbottommargin=8pt,
    roundcorner=5pt,
    backgroundcolor = gray!30,
    hidealllines = true
]{mydef}{Definición}[section]

\newmdtheoremenv[
    leftmargin=0em,
    rightmargin=0em,
    innertopmargin=-2pt,
    innerbottommargin=8pt,
    roundcorner=5pt
]{excer}{Ejercicio}[section]

%En esta parte se colocan comandos que definen la forma en la que se van a escribir ciertas funciones%

\newcommand\abs[1]{\ensuremath{\lvert#1\rvert}}
\newcommand\divides{\ensuremath{\bigm|}}
\newcommand\cf[3]{\ensuremath{#1:#2\rightarrow#3}}
\newcommand\supp[1]{\ensuremath{\textup{supp}\left(#1\right)}}

%recuerda usar \clearpage para hacer un salto de página

\begin{document}
    \title{Segundo Examen Geometría Diferencial III}
    \author{Cristo Daniel Alvarado}
    \date{\today}
    \maketitle

    \setcounter{chapter}{1} %En esta parte lo que se hace es cambiar la enumeración del capítulo%
        
    \section{Demostración del teorema de punto fijo de Brower}
    
En este documento se pretende dar una prueba, lo más entendible posible, del teorema del punto fijo de Brower, desde el punto de vista de las formas diferenciales. 

    \begin{theor}[Punto Fijo de Brower]
        Denotemos por $B^n$ a la bola unitaria en $\mathbb{R}^n$, es decir, se denota al conjunto
        \begin{equation*}
            B^n=\left\{x\in\mathbb{R}^n|\|x\|\leq 1\right\}.
        \end{equation*}
        Sea $\cf{f}{B^n}{B^n}$ una función continua. Entonces $f$ tiene un punto fijo, es decir existe $x_0\in B^n$ tal que $f(x_0)=x_0$.    \end{theor}

Para la demostración del teorema, se requieren de unos resultados anteriores. Entre ellos se tiene el Teorema 3.5.11 y un lema (casi inmediato) del mismo:

\begin{theor}
    Sea $V\subseteq\mathbb{R}^n$ abierto. Si $\cf{\phi}{\mathbb{R}^n}{\mathbb{R}}$ es una función continua tal que $\supp{\phi}\subseteq V$. Entonces para cada $\varepsilon>0$ existe una función $\cf{\psi}{\mathbb{R}^n}{\mathbb{R}}$ con soporte compacto tal que $\supp{\psi}\subseteq V$ y
    \begin{equation*}
        \sup\abs{\phi-\psi}<\varepsilon
    \end{equation*}
\end{theor}

\begin{proof}
    Denotemos por $A=\supp{\phi}$, y definamos
    \begin{equation*}
        d=\inf\left\{\|x-y\|_{\infty}|x\in A \textup{ y }y\in V^c\right\}
    \end{equation*}
    Recordando que 
    \begin{equation*}
        \|x\|_{\infty}=\max\left\{\abs{x_i}|i=1,\dots,n\right\}\textup{, donde }x=\left(x_1,\dots,x_n\right)\in\mathbb{R}^n
    \end{equation*}
    Afirmamos que $d>0$. Como $A\subseteq V$, entonces se tiene que $A\cap V^c=\emptyset$. Siendo que $V$ es abierto, se sigue que $V^c$ es cerrado. Defina $\cf{f}{A}{\mathbb{R}}$ dada como sigue:
    \begin{equation*}
        \begin{split}
            f(x)=&\inf\left\{\|x-z\|_{\infty}|z\in V^c\right\}\\
            =&d_\infty(x,V^c)\\
        \end{split}
    \end{equation*}
    para todo $x\in A$. Afirmamos que la función $f$ es continua en $A$. En efecto, sean $x\in A$ $\left\{x_n\right\}_{n=1}^{\infty}$ una sucesión de puntos de $A$ que convergen a $x$ (con la norma infinito). Se tiene
    \begin{equation}
        \begin{split}
            \abs{f(x)-f(x_n)}=&\abs{d_\infty(x,V^c)-d_\infty(x_n,V^c)}\\
        \end{split}
    \end{equation}
    Por la desigualdad del triángulo, se tiene que si $z\in V^c$:
    \begin{equation*}
        d_{\infty}(x,z)\leq d_{\infty}(x,x_n)+d_{\infty}(x_n,z)
    \end{equation*}
    Pero $d_{\infty}(x,V^c)\leq d_{\infty}(x,z)$. Por lo cual:
    \begin{equation*}
        \begin{split}
            d_{\infty}(x,V^c)\leq&d_{\infty}(x,x_n)+d_{\infty}(x_n,z)\\
            \Rightarrow d_{\infty}(x,V^c)-d_{\infty}(x,x_n)\leq&d_{\infty}(x_n,z)\\
        \end{split}
    \end{equation*}
    para todo $z\in V^c$. De esta forma, $d_{\infty}(x,V^c)-d_{\infty}(x,x_n)$ es cota inferior de $\left\{d_{\infty}(x_n,z)|z\in V^c\right\}$. Así
    \begin{equation*}
        \begin{split}
            d_{\infty}(x,V^c)-d_{\infty}(x,x_n)\leq&d_{\infty}(x_n,V^c)\\
            \Rightarrow d_{\infty}(x,V^c)-d_{\infty}(x_n,V^c)\leq&d_{\infty}(x,x_n)\\
        \end{split}
    \end{equation*}
    de forma análoga se prueba también que
    \begin{equation*}
        -\left(d_{\infty}(x,V^c)-d_{\infty}(x_n,V^c)\right)\leq d_{\infty}(x,x_n)
    \end{equation*}
    para todo $n\in\mathbb{N}$. Por lo cual
    \begin{equation*}
        \abs{d_{\infty}(x,A)-d_{\infty}(x_n,A)}\leq d_{\infty}(x, x_n)
    \end{equation*}
    para todo $n\in\mathbb{N}$. Retomando la ecuación (1.1) se sigue que
    \begin{equation*}
        0\leq\abs{f(x_n)-f(x)}\leq d_{\infty}(x,x_n)
    \end{equation*}
    Tomando ambos límites de ambos lados
    \begin{equation*}
        \lim_{n\rightarrow\infty}\abs{f(x_n)-f(x)}=0
    \end{equation*}
    pues $\lim_{n\rightarrow\infty}d_{\infty}(x,x_n)=0$. Luego $f$ es continua en $x$. Por ser $x\in A$ arbitrario, se sigue que $f$ es continua en $A$.

    Pero $A$ es un conjunto compacto y $f$ es una función que toma valores reales, por lo cual alcanza su máximo y su mínimo, digamos lo alcanza en $a\in A$, es decir
    \begin{equation*}
        f(a)\leq f(x),\quad\forall x\in A
    \end{equation*}
    donde $a\notin V^c$, pues en caso contrario se tendría que $A\cap V^c\neq \emptyset$. Se tiene que $f(a)=d_{\infty}(a,V^c)>0$, ya que en caso contrario, por propiedades del ínfimo existiría una sucesión en $V^c$ que converge a $a$, cosa que no puede suceder, pues $V^c$ es cerrado y $a\notin V^c$ (esto implicaría que $V^c$ no tiene a todos sus puntos de acumulación). Así
    \begin{equation*}
        \begin{split}
            d=&\inf\left\{\|x-y\|_{\infty}|x\in A \textup{ y }y\in V^c\right\}\\
            =&\inf\left\{d_{\infty}(x,V^c)|x\in A\right\}\\
            =&\inf\left\{f(x)|x\in A\right\}\\
            =&\min\left\{f(x)|x\in A\right\}\\
            =&f(a)\\
            >&0\\
        \end{split}
    \end{equation*}
    por lo cual $d>0$.
    
    Como $\phi$ es continua en el compacto $A$, es uniformemente continua en $A$. Pero también es uniformemente continua fuera de $A$, pues toma el valor constante $0$, así $\phi$ es continua en todo su dominio.
    Por tanto, para $\varepsilon>0$ existe $\delta'>0$ tal que si $x,y\in\mathbb{R}^n$ son tales que $\|x-y\|_{\infty}<\delta'$, entonces $\abs{\phi(x)-\phi(y)}<\varepsilon$. Sea $\delta =\min\left\{\delta',\frac{d}{2}\right\}>0$.

    Definamos
    
    \begin{equation*}
        Q=\left\{x\in\mathbb{R}^n|\|x\|_{\infty}\leq\delta\right\}
    \end{equation*}
    
    y sea $\cf{\rho}{\mathbb{R}^n}{\mathbb{R}}$ una función $C^{\infty}$ no negativa con soporte compacto tal que $\supp{\rho}\subseteq Q$ y
    
    \begin{equation*}
        \int_{\mathbb{R}^n}\rho(y)dy=1
    \end{equation*}

    \end{proof}

\begin{lema}
    Sea $U\subseteq\mathbb{R}^n$ abierto, $C\subseteq U$ compacto y $\cf{\phi}{U}{\mathbb{R}}$ una función continua que es clase $C^{\infty}$ en el complemento de C. Entonces, para todo $\varepsilon > 0$ existe una función $\cf{\psi}{U}{\mathbb{R}}$ $C^{\infty}$ en $U$, tal que $\phi - \psi$ tiene soporte compacto y $\abs{\phi - \psi}<\varepsilon$.
\end{lema}

\begin{theor}
    
\end{theor}

    \begin{propo}[Nombre]
        Proposición
    \end{propo}

    \begin{cor}[Nombre]
        Corolario
    \end{cor}

    \begin{lema}[Nombre]
        Lema
    \end{lema}

    \begin{mydef}[Nombre]
        Definición
    \end{mydef}

    \begin{obs}[Nombre]
        Observación
    \end{obs}

    \begin{exa}[Nombre]
        Ejemplo
    \end{exa}

    \begin{excer}[Nombre]
        Ejercicio
    \end{excer}

\end{document}