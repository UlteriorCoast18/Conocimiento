\documentclass[12pt]{report}
\usepackage[spanish]{babel}
\usepackage[utf8]{inputenc}
\usepackage{amsmath}
\usepackage{amssymb}
\usepackage{amsthm}
\usepackage{graphics}
\usepackage{subfigure}
\usepackage{lipsum}
\usepackage{array}
\usepackage{multicol}
\usepackage{enumerate}
\usepackage[framemethod=TikZ]{mdframed}
\usepackage[a4paper, margin = 1.5cm]{geometry}

%En esta parte se hacen redefiniciones de algunos comandos para que resulte agradable el verlos%

\def\proof{\paragraph{Demostración:\\}}
\def\endproof{\hfill$\square$}
\renewcommand{\theenumii}{\roman{enumii}}

%En esta parte se definen los comandos a usar dentro del documento para enlistar%

\newtheoremstyle{largebreak}
  {}% use the default space above
  {}% use the default space below
  {\normalfont}% body font
  {}% indent (0pt)
  {\bfseries}% header font
  {}% punctuation
  {\newline}% break after header
  {}% header spec

\theoremstyle{largebreak}

\newmdtheoremenv[
    leftmargin=0em,
    rightmargin=0em,
    innertopmargin=-2pt,
    innerbottommargin=8pt,
    hidealllines = true,
    roundcorner = 5pt,
    backgroundcolor = gray!60!red!30
]{exa}{Ejemplo}[section]

\newmdtheoremenv[
    leftmargin=0em,
    rightmargin=0em,
    innertopmargin=-2pt,
    innerbottommargin=8pt,
    hidealllines = true,
    roundcorner = 5pt,
    backgroundcolor = gray!50!blue!30
]{obs}{Observación}[section]

\newmdtheoremenv[
    leftmargin=0em,
    rightmargin=0em,
    innertopmargin=-2pt,
    innerbottommargin=8pt,
    rightline = false,
    leftline = false
]{theor}{Teorema}[section]

\newmdtheoremenv[
    leftmargin=0em,
    rightmargin=0em,
    innertopmargin=-2pt,
    innerbottommargin=8pt,
    rightline = false,
    leftline = false
]{propo}{Proposición}[section]

\newmdtheoremenv[
    leftmargin=0em,
    rightmargin=0em,
    innertopmargin=-2pt,
    innerbottommargin=8pt,
    rightline = false,
    leftline = false
]{cor}{Corolario}[section]

\newmdtheoremenv[
    leftmargin=0em,
    rightmargin=0em,
    innertopmargin=-2pt,
    innerbottommargin=8pt,
    rightline = false,
    leftline = false
]{lema}{Lema}[section]

\newmdtheoremenv[
    leftmargin=0em,
    rightmargin=0em,
    innertopmargin=-2pt,
    innerbottommargin=8pt,
    roundcorner=5pt,
    backgroundcolor = gray!30,
    hidealllines = true
]{mydef}{Definición}[section]

\newmdtheoremenv[
    leftmargin=0em,
    rightmargin=0em,
    innertopmargin=-2pt,
    innerbottommargin=8pt,
    roundcorner=5pt
]{excer}{Ejercicio}[section]

%En esta parte se colocan comandos que definen la forma en la que se van a escribir ciertas funciones%

\newcommand\abs[1]{\ensuremath{\lvert#1\rvert}}
\newcommand\divides{\ensuremath{\bigm|}}
\newcommand\cf[3]{\ensuremath{#1:#2\rightarrow#3}}
\newcommand\supp[1]{\ensuremath{\textup{supp}\left(#1\right)}}

%recuerda usar \clearpage para hacer un salto de página

\begin{document}
    \title{Segundo Examen Geometría Diferencial III}
    \author{Cristo Daniel Alvarado}
    \date{\today}
    \maketitle

    \setcounter{chapter}{3} %En esta parte lo que se hace es cambiar la enumeración del capítulo%
    \setcounter{section}{5}
    \section{Demostración del Teorema de punto fijo de Brower}
    
En este documento se pretende dar una prueba, lo más entendible posible, del teorema del punto fijo de Brower, desde el punto de vista de las formas diferenciales. 

Para la demostración del teorema se requieren de unos resultados preliminares.

\setcounter{section}{4}
\setcounter{theor}{6}

\begin{theor}
    Sean $U$ y $V$ subconjuntos abiertos de $\mathbb{R}^n$ y $\mathbb{R}^m$, respectivamente y sea $\cf{f}{U}{V}$ un mapeo propio continuo. Si $B\subseteq V$ es compacto y $A=f^{-1}(B)$, entonces para cada conjunto abierto $U_0$ tal que $A\subseteq U_0\subseteq U$, existe un conjunto abierto $V_0$ tal que $B\subseteq V_0\subseteq V$ y $f^{-1}(V_0)\subseteq U_0$.
\end{theor}

\begin{proof}
    Sea $U_0\subseteq U$ abierto tal que $A\subseteq U_0$. Tomemos $C\subseteq V$ compacto tal que $B\subseteq \textup{int}(C)$ y $W=f^{-1}(C)\backslash U_0=f^{-1}(C)\cap U_0^c$, el cual es compacto (por ser la intersección del compacto $f^{-1}(C)$ ya que $f$ es propio, con el cerrado $U_0^c$). Como $f$ es continua, entonces $f(W)$ es compacto.
    
    Veamos que $f(W)$ y $B$ son disjuntos. En efecto, suponga que tienen intersección no vacía. Sea $x\in f(W)\cap B$, entonces existe $y\in W$ tal que $x=f(y)$ donde $y\in f^{-1}(C)\backslash U_0$, es decir $x=f(y)\in C$ y en $B$ pero no en $U_0$.

    Pero $f^{-1}(B)\subseteq U_0$, por lo cual $B\subseteq f(U_0)$, así como $x\in B$, también $x\in U_0$, que es una contradicción. Por tanto, $f(W)\cap B=\emptyset$.
    
    Sea
    \begin{equation*}
        V_0=\textup{int}(C)\backslash f(W)
    \end{equation*}
    Claramenet $V_0$ es abierto (por ser la intersección de dos abiertos), además como $f(W)\cap B=\emptyset$ se sigue que $B\subseteq V_0$ (pues, $\textup{int}(C)$ contiene a $B$), y es subconjunto de $V$. Además
    \begin{equation*}
        \begin{split}
            x\in f^{-1}(V_0)&\Rightarrow f(x)\in V_0\\
            &\Rightarrow f(x)\in \textup{int}(C)\backslash f(W)\\
            &\Rightarrow f(x)\in \textup{int}(C)\textup{ y } f(x)\notin f(W)\\
            &\Rightarrow f(x)\in \textup{int}(C)\textup{ y } x\notin W\\
            &\Rightarrow f(x)\in \textup{int}(C)\textup{ y } x\notin f^{-1}(C)\backslash U_0\\
            &\Rightarrow f(x)\in C\textup{ y } x\in f^{-1}(C)^c\cup U_0\\
            &\Rightarrow f(x)\in C\textup{ y } x\in f^{-1}(C^c)\cup U_0\\
            &\Rightarrow f(x)\in C\textup{ y, } f(x)\in C^c\textup{ o }x\in U_0\\
            &\Rightarrow x\in U_0\\
        \end{split}
    \end{equation*}
    Por tanto, $f^{-1}(V_0)\subseteq U_0$.
\end{proof}

\setcounter{section}{5}
\setcounter{theor}{10}

\begin{theor}
    Sea $V\subseteq\mathbb{R}^n$ abierto. Si $\cf{\phi}{\mathbb{R}^n}{\mathbb{R}}$ es una función continua tal que $\supp{\phi}\subseteq V$. Entonces para cada $\varepsilon>0$ existe una función $\cf{\psi}{\mathbb{R}^n}{\mathbb{R}}$ con soporte compacto tal que $\supp{\psi}\subseteq V$ y
    \begin{equation*}
        \sup_{x\in\mathbb{R}^n}\abs{\phi(x)-\psi(x)}<\varepsilon
    \end{equation*}
\end{theor}

\begin{proof}
    Denotemos por $A=\supp{\phi}$, y definamos
    \begin{equation*}
        d=\inf\left\{\|x-y\|_{\infty}|x\in A \textup{ y }y\in V^c\right\}
    \end{equation*}
    Recordando que 
    \begin{equation*}
        \|x\|_{\infty}=\max\left\{\abs{x_i}|i=1,\dots,n\right\}\textup{, donde }x=\left(x_1,\dots,x_n\right)\in\mathbb{R}^n
    \end{equation*}
    Afirmamos que $d>0$. Como $A\subseteq V$, entonces se tiene que $A\cap V^c=\emptyset$. Siendo que $V$ es abierto, se sigue que $V^c$ es cerrado. Defina $\cf{f}{A}{\mathbb{R}}$ dada como sigue:
    \begin{equation*}
        \begin{split}
            f(x)=&\inf\left\{\|x-z\|_{\infty}|z\in V^c\right\}\\
            =&d_\infty(x,V^c)\\
        \end{split}
    \end{equation*}
    para todo $x\in A$. Afirmamos que la función $f$ es continua en $A$. En efecto, sean $x\in A$ $\left\{x_n\right\}_{n=1}^{\infty}$ una sucesión de puntos de $A$ que convergen a $x$ (con la norma infinito). Se tiene
    \begin{equation}
        \begin{split}
            \abs{f(x)-f(x_n)}=&\abs{d_\infty(x,V^c)-d_\infty(x_n,V^c)}\\
        \end{split}
    \end{equation}
    Por la desigualdad del triángulo, se tiene que si $z\in V^c$:
    \begin{equation*}
        d_{\infty}(x,z)\leq d_{\infty}(x,x_n)+d_{\infty}(x_n,z)
    \end{equation*}
    Pero $d_{\infty}(x,V^c)\leq d_{\infty}(x,z)$. Por lo cual:
    \begin{equation*}
        \begin{split}
            d_{\infty}(x,V^c)\leq&d_{\infty}(x,x_n)+d_{\infty}(x_n,z)\\
            \Rightarrow d_{\infty}(x,V^c)-d_{\infty}(x,x_n)\leq&d_{\infty}(x_n,z)\\
        \end{split}
    \end{equation*}
    para todo $z\in V^c$. De esta forma, $d_{\infty}(x,V^c)-d_{\infty}(x,x_n)$ es cota inferior de $\left\{d_{\infty}(x_n,z)|z\in V^c\right\}$. Así
    \begin{equation*}
        \begin{split}
            d_{\infty}(x,V^c)-d_{\infty}(x,x_n)\leq&d_{\infty}(x_n,V^c)\\
            \Rightarrow d_{\infty}(x,V^c)-d_{\infty}(x_n,V^c)\leq&d_{\infty}(x,x_n)\\
        \end{split}
    \end{equation*}
    de forma análoga se prueba también que
    \begin{equation*}
        -\left(d_{\infty}(x,V^c)-d_{\infty}(x_n,V^c)\right)\leq d_{\infty}(x,x_n)
    \end{equation*}
    para todo $n\in\mathbb{N}$. Por lo cual
    \begin{equation*}
        \abs{d_{\infty}(x,A)-d_{\infty}(x_n,A)}\leq d_{\infty}(x, x_n)
    \end{equation*}
    para todo $n\in\mathbb{N}$. Retomando la ecuación (1.1) se sigue que
    \begin{equation*}
        0\leq\abs{f(x_n)-f(x)}\leq d_{\infty}(x,x_n)
    \end{equation*}
    Tomando ambos límites de ambos lados
    \begin{equation*}
        \lim_{n\rightarrow\infty}\abs{f(x_n)-f(x)}=0
    \end{equation*}
    pues $\lim_{n\rightarrow\infty}d_{\infty}(x,x_n)=0$. Luego $f$ es continua en $x$. Por ser $x\in A$ arbitrario, se sigue que $f$ es continua en $A$.

    Pero $A$ es un conjunto compacto y $f$ es una función que toma valores reales, por lo cual alcanza su máximo y su mínimo, digamos lo alcanza en $a\in A$, es decir
    \begin{equation*}
        f(a)\leq f(x),\quad\forall x\in A
    \end{equation*}
    donde $a\notin V^c$, pues en caso contrario se tendría que $A\cap V^c\neq \emptyset$. Se tiene que $f(a)=d_{\infty}(a,V^c)>0$, ya que en caso contrario, por propiedades del ínfimo existiría una sucesión en $V^c$ que converge a $a$, cosa que no puede suceder, pues $V^c$ es cerrado y $a\notin V^c$ (esto implicaría que $V^c$ no tiene a todos sus puntos de acumulación). Así
    \begin{equation*}
        \begin{split}
            d=&\inf\left\{\|x-y\|_{\infty}|x\in A \textup{ y }y\in V^c\right\}\\
            =&\inf\left\{d_{\infty}(x,V^c)|x\in A\right\}\\
            =&\inf\left\{f(x)|x\in A\right\}\\
            =&\min\left\{f(x)|x\in A\right\}\\
            =&f(a)\\
            >&0\\
        \end{split}
    \end{equation*}
    por lo cual $d>0$.
    
    Como $\phi$ es continua en el compacto $A$, es uniformemente continua en $A$. Pero también es uniformemente continua fuera de $A$, pues toma el valor constante $0$, así $\phi$ es continua en todo su dominio.
    Por tanto, para $\varepsilon>0$ existe $\delta'>0$ tal que si $x,y\in\mathbb{R}^n$ son tales que $\|x-y\|_{\infty}<\delta'$, entonces $\abs{\phi(x)-\phi(y)}<\varepsilon$. Sea $\delta =\min\left\{\delta',\frac{d}{2}\right\}>0$.
    
    Definamos
    \begin{equation*}
        Q=\left\{x\in\mathbb{R}^n|\|x\|_{\infty}\leq\delta\right\}
    \end{equation*}
    y sea $\cf{\rho}{\mathbb{R}^n}{\mathbb{R}}$ una función $C^{\infty}$ no negativa con soporte compacto tal que $\supp{\rho}\subseteq Q$ y
    \begin{equation*}
        \int_{\mathbb{R}^n}\rho(y)dy=1
    \end{equation*}
    
    Defina $\cf{\psi}{\mathbb{R}^n}{\mathbb{R}}$ dada como:
    \begin{equation*}
        \begin{split}
            \psi(x)=&\int_{\mathbb{R}^n}\rho(y-x)\phi(y)dy\\
            =&\int_{A}\rho(y-x)\phi(y)dy\\
        \end{split}
    \end{equation*}
    (ya que $\phi$ tiene soporte en $A$). Como $\rho$ es de clase $C^{\infty}$, se sigue que $\psi$ también lo es.

    Sea
    \begin{equation*}
        A_{\delta}=\left\{x\in\mathbb{R}^n|d_{\infty}(x,A)\leq\delta\right\}
    \end{equation*}
    Notemos que si $x\notin A_\delta$ entonces para todo $y\in A$, $\|x-y\|_{\infty}>\delta$, como $\supp{\rho}\subseteq Q$ entonces $\rho(x-y)=0$. Por tanto se tendría que
    para $x\notin A_\delta$:
    \begin{equation*}
        \begin{split}
            \psi(x)=&\int_{A}\rho(y-x)\phi(y)dy\\
            =&0
        \end{split}
    \end{equation*}
    de esta forma el soporte de $\psi$ es un subconjunto de $A_\delta$. Siendo que $\delta\leq\frac{d}{2}$, como la distancia de $A$ a $V^c$ es $d$, entonces $A_\delta\subseteq V$, así $\supp{\psi}\subseteq V$.

    Ahora, por el teorema de cambio de variable se tiene que
    
    \begin{equation*}
        \int_{\mathbb{R}^n}\rho(y-x)dy=\int_{\mathbb{R}^n}\rho(y)dy=1,\quad\forall x\in\mathbb{R}^n
    \end{equation*}

    Por tanto,

    \begin{equation*}
        \phi(x)=\int_{\mathbb{R}^n}\phi(x)\rho(y-x)dy,\quad \forall x\in\mathbb{R}^n
    \end{equation*}

    Así,

    \begin{equation*}
        \begin{split}
            \phi(x)-\psi(x)=&\int_{\mathbb{R}^n}\left(\phi(x)-\phi(y)\right)\rho(y-x)dy\\
            \Rightarrow \abs{\phi(x)-\psi(x)}\leq&\int_{\mathbb{R}^n}\abs{\phi(x)-\phi(y)}\rho(y-x)dy,\quad \forall x\in\mathbb{R}^n\\
        \end{split}
    \end{equation*}
    
    Notemos varias cosas. Primero, si tenemos $x,y\in\mathbb{R}^n$ tales que $\|x-y\|_\infty>\delta$, entonces $\rho(x-y)=0$. Si por el contrario $\|x-y\|_\infty\leq\delta$, entonces $\abs{\phi(x)-\phi(y)}<\varepsilon$. Por lo cual
    \begin{equation*}
        \begin{split}
            \abs{\phi(x)-\psi(x)}\leq&\int_{\mathbb{R}^n}\abs{\phi(x)-\phi(y)}\rho(y-x)dy\\
            <&\int_{\mathbb{R}^n}\varepsilon\cdot\rho(y-x)dy\\
            =&\varepsilon,\quad\forall x\in\mathbb{R}^n\\
        \end{split}
    \end{equation*}
    
    Luego,
    
    \begin{equation*}
        \sup_{x\in\mathbb{R}^n}\abs{\phi(x)-\psi(x)}<\varepsilon
    \end{equation*}

\end{proof}

Con la prueba de este teorema hecha, se procederá a probar el siguiente lema:

\setcounter{section}{6}
\setcounter{lema}{14}

\begin{lema}
    Sea $U\subseteq\mathbb{R}^n$ abierto, $C\subseteq U$ compacto y $\cf{\phi}{U}{\mathbb{R}}$ una función continua que es clase $C^{\infty}$ en el complemento de C. Entonces, para todo $\varepsilon > 0$ existe una función $\cf{\psi}{U}{\mathbb{R}}$ $C^{\infty}$ en $U$, tal que $\phi - \psi$ tiene soporte compacto y $\abs{\phi - \psi}<\varepsilon$.
\end{lema}

\begin{proof}
    Sea $\varepsilon>0$ y $\cf{\rho}{U}{\mathbb{R}}$ una función $C_0^{\infty}(U)$ tal que toma el valor de $1$ en $C$. Por el Teorema 3.5.11 para $\rho\cdot\phi$ función continua y $\varepsilon>0$ existe una función $\psi_0\in C^{\infty}_{0}(U)$ tal que
    \begin{equation*}
        \abs{\rho\cdot\phi(x)-\psi_0(x)}<\varepsilon,\quad\forall x\in U
    \end{equation*}

    Tomemos $\psi=\left(1-\rho\right)\phi+\psi_0$. Ya se sabe que $\psi_0$ es $C^\infty$. Como $\rho$ toma el valor de 1 dentro de $C$, entonces $1-\rho$ toma el valor de $0$ en $C$. Por ser $\phi$ $C^\infty$ en el complemento de $C$, y tomar el valor de $0$ en $C$, se sigue que $\left(1-\rho\right)\phi$ es $C^\infty$ en $U$, luego la diferencia $\left(1-\rho\right)\phi+\psi_0=\psi$ lo es.

    Además, se cumple que
    \begin{equation*}
        \begin{split}
            \abs{\phi(x)-\psi(x)}=&\abs{\phi(x)-\left(\left(1-\rho\right)\phi+\psi_0\right)(x)}\\
            =&\abs{\phi(x)-\phi(x)+\rho(x)\phi(x)-\psi_0(x)}\\
            =&\abs{\rho(x)\phi(x)-\psi_0(x)}\\
            <&\varepsilon,\quad\forall x\in U\\
        \end{split}
    \end{equation*}
    
    donde la función $\rho\cdot\phi-\psi_0$ tiene soporte contenido en el compacto $C$ (por tenerlo $\rho$ y $\psi_0$). Por lo cual $\phi-\rho$ tienen soporte contenido en $C$, i.e. soporte compacto. 
\end{proof}

Asumiremos como cierto el siguiente resultado, el cual nos permitirá computar el grado de un mapeo.

\setcounter{section}{6}
\setcounter{theor}{1}

\begin{theor}[Sard]
    Sean $U$ y $V$ son subconjuntos abiertos de $\mathbb{R}^n$ y $\cf{f}{U}{V}$ un mapeo propio $C^\infty$. Entonces el conjunto de valores regulares de $f$ es un conjunto abierto denso de $V$.
\end{theor}

Con esto, se procederá con la prueba de algunos resultados.

\begin{theor}
    Sean $U$ y $V$ son subconjuntos abiertos de $\mathbb{R}^n$ y $\cf{f}{U}{V}$ un mapeo propio $C^\infty$. Sea $q\in U$ un valor regular de $f$. Entonces el conjunto $f^{-1}(q)$ es finito. Más aún, si $f^{-1}(1)=\left\{p_1,\dots,p_N\right\}$, existen vecindades abiertas conexas $U_i$ de $p_i$ para cada $i=1,\dots,N$ en $U$ y una vecindad abierta $W$ de $q$ contenida en $V$ tal que
    \begin{enumerate}
        \item Para $i\neq j$, los conjuntos $U_i$ y $U_j$ son disjuntos.
        \item $f^{-1}(W)=U_1\cup\cdots\cup U_N$.
        \item $f$ mapea a los $U_i$ difeomorfamente en $W$. 
    \end{enumerate}
\end{theor}

\begin{proof}
    Probaremos primero que $f^{-1}(q)$ es finito. Sea $p\in f^{-1}(q)$. Como $q$ es un valor regular de $f$ entonces $p\in C_f$, donde $C_f$ denota al conjunto de todos los puntos críticos de $f$. Por lo cual
    \begin{equation*}
        \det(Df(p))\neq0
    \end{equation*}
    luego, por el teorema de la función inversa existen vecindades $p\in U_p$ de $U$ y $q\in V_q$ de $V$ tales que $f|_{U_p}$ es difeomorfismo de $U_p$ en $V_q$.

    Considere la familia de estas vecindades dada por
    \begin{equation*}
        \left\{U_p|p\in f^{-1}(q)\right\}
    \end{equation*}
    Claramente esta familia de subconjuntos de $U$ forman una cubierta abierta de $f^{-1}(q)$. Como $f$ es propio, el conjunto $f^{-1}(q)$ es compacto, luego existen $p_1,\dots,p_N\in U$ tales que
    \begin{equation*}
        f^{-1}(q)\subseteq\bigcup_{i=1}^{N}U_{p_i}
    \end{equation*}
    Como las reestricciones de $f$ a los $U_{p_i}$ son biyectivas, $p_i$ es el único punto tal que $f$ lo mapea a $q$, para todo $i=1,\dots,N$. Por lo cual debido a la contención anterior se sigue que
    \begin{equation*}
        f^{-1}(q)=\left\{p_1,\dots,p_N\right\}
    \end{equation*}
    
    Sin pérdida de generalidad, podemos suponer que los $U_{p_i}$ son disjuntos a pares y conexos. En caso de que no lo sean, podemos tomar bolas abiertas conexas disjuntas a pares centradas en cada $p_i$ contenidas en cada $U_{p_i}$ (pues $\mathbb{R}^n$ es Hausdorff y las bolas abiertas son conexas). Por el Teorema 3.4.7 existe una vecindad $W'$ de $q$ en $V$ para la cual
    \begin{equation*}
        f^{-1}(W')\subseteq\bigcup_{i=1}^{N}U_{p_i}
    \end{equation*}
    si $W'$ no es conexa, tomando a $W\subseteq W'$ como una bola abierta centrada en $q$ contenida en $W'$, se sigue que
    \begin{equation*}
        f^{-1}(W)\subseteq\bigcup_{i=1}^{N}U_{p_i}
    \end{equation*}
    donde $W$ es conexa.

    Tomemos $U_i=f^{-1}(W)\cap U_{p_i}$, para cada $i=1,\dots,N$. Estos conjuntos abiertos (lo son, pues $f$ es continua y por ende $f^{-1}(W)$ es abierto) los podemos tomar como conexos (en caso de no serlo, haceos más pequeñas las vecindades $U_{p_i}$ del tal forma que la intersección con $f^{-1}(W)$ sea conexa). Se cumple que

    \begin{enumerate}
        \item Los $U_i$ son ajenos a pares, es decir $U_i\cap U_j=\emptyset$ para $i\neq j$.
        \item Claramente
        \begin{equation*}
            f^{-1}(W)=U_1\cup\dots\cup U_N
        \end{equation*}
        \item $f|_{U_i}$ es difeomorfismo, para cada $1,\dots,N$.
    \end{enumerate}
    
    Lo cual termina la demostración.
\end{proof}

\begin{theor}
    En las condiciones del Teorema 3.6.3, para cada $p_i\in f^{-1}(q)$, sea $\sigma_{p_i}=1$ si $\cf{f}{U_i}{W}$ preserva la orientación, y $-1$ si la invierte. Entonces
    \begin{equation*}
        \deg(f)=\sum_{i=1}^{N}\sigma_{p_i}
    \end{equation*}
\end{theor}

\begin{proof}
    Sea $\omega$ una $n$-forma con soporte compacto en $W$ tal que
    \begin{equation*}
        \int_{W}\omega = 1
    \end{equation*}
    Entonces
    \begin{equation*}
        \begin{split}
            \deg(f)&=\int_{U}f^*\omega\\
            &=\sum_{i=1}^{N}\int_{U_i}f^*\omega\\
        \end{split}
    \end{equation*}
    pues, $f^{-1}(W)=U_1\cup\dots\cup U_N$. Pero como $f|_{U_i}$ es difeomorfismo, para cada $i=1,\dots,N$, entonces se tiene que
    \begin{equation*}
        \int_{U_i}f^*\omega=\pm \int_{W}\omega=\pm 1
    \end{equation*}
    Siendo $+1$ si $f$ preserva la orientación y $-1$ en caso contrario. Por lo cual
    \begin{equation*}
        \deg(f)=\sum_{i=1}^{N}\sigma_{p_i}
    \end{equation*}
\end{proof}

\begin{theor}
    En las condiciones del Teorema 3.6.4, si $\cf{f}{U}{V}$ no es suprayectiva, entonces $\deg(f)=0$.
\end{theor}

\begin{proof}
    Afirmamos que $V\backslash f(U)$ es abierto. En efecto, sea $x\in V\backslash f(U)\neq \emptyset$ pues $f$ no es suprayectiva. Entonces como $x\notin f(U)$, se tiene que $f^{-1}(x)=\emptyset=A$. Como $f$ es un mapeo propio continuo, por el Teorema 3.4.7 para $U_0=\emptyset$ existe un abierto $V_0\subseteq V$ tal que $x\in V_0$ y $f^{-1}(V_0)\subseteq U_0=\emptyset$. Es decir, que $f^{-1}(V_0)=\emptyset$, por tanto $V_0\subseteq V\backslash f(U)$. Por ende $V\backslash f(U)$ es abierto.

    De esta forma, existe una $n$-forma $\omega$ con soporte compacto en $V\backslash f(U)$ tal que
    \begin{equation*}
        \int_{V}\omega = 1
    \end{equation*}
    Como $\omega$ toma el valor de cero en la imagen de $f$, se sigue que
    \begin{equation*}
        0=\int_{U}f^*\omega=\deg(f)\int_{V}\omega = \deg(f)
    \end{equation*}
\end{proof}

Con estos teoremas a la mano, se procederá con la prueba de la siguiente proposición.

\begin{propo}
    Sea $U\subseteq \mathbb{R}^n$ abierto conexo y $\cf{f}{U}{U}$ un mapeo $C^{\infty}$. Entonces si $f|_{C^c}=\textup{id}_{C^c}$ donde $C\subseteq\mathbb{R}^n$ es compacto, entonces $f$ es propio y $\deg(f)=1$.
\end{propo}

\begin{proof}
    Sea $A\subseteq U$. Veamos que $f^{-1}(A)\subseteq A\cup C$. Si $f^{-1}(A)=\emptyset$, el resultado es inmediato. Suponga que es no vacío, si $x\in f^{-1}(A)$, entonces $f(x)\in A$.

    Tenemos dos casos: si $x\notin C$ entonces como $f$ es la identidad en el complemento de $C$ se sigue que $x = f(x)\in A$. Por tanto, $x\in A\cup C$. Así $f^{-1}(A)\subseteq A\cup C$.

    Si $A$ es compacto, entonces $f^{-1}(A)\subseteq A\cup C$ es cerrado (pues la imagen inversa de cerrados es cerrado, por ser $f$ continua) y es acotado (ya que $A$ y $C$ son compactos, en particular su unión lo es, luego $f^{-1}(A)$ es acotado), luego compacto. Por tanto, $f$ es un mapeo propio.
    
    Para la otra parte, como $C\subseteq U$ es compacto, entonces $U\backslash f(C)$ es no vacío, pues $f(C)\subseteq U$ es compacto, entonces debe existir al menos un elemento $x\in U$ tal que $x\notin f(C)$. Se cumple para este elemento que $f^{-1}(x)=\left\{x\right\}$, pues $x\notin f(C)$. Además es valor regular de $f$ ya que
    \begin{equation*}
        Df(x)=\textup{id}_{\mathbb{R}^n}
    \end{equation*}
    
    Por el Teorema 3.6.3 existe una vecindad conexa $U_1$ de $x$ contenida en $U$ y una vecindad abierta conexa $W$ también de $x$ tal que
    \begin{equation*}
        f^{-1}(W)=U_1
    \end{equation*}
    siendo $f|_{U_1}$ difeomorfismo. En particular como es una vecindad de $x$ y $Df(x)=\textup{id}_{\mathbb{R}^n}$ entonces $f$ debe preservar la orientación, por lo cual
    \begin{equation*}
        \deg(f)=\sum_{i=1}^{1}1=1
    \end{equation*}
\end{proof}

Procedamos ahora con la prueba del teorema del punto fijo. Denotemos por $B^n$ a la bola unitaria en $\mathbb{R}^n$, es decir, se denota al conjunto
\begin{equation*}
    \begin{split}
        B^n=&\left\{x\in\mathbb{R}^n|\|x\|\leq 1\right\}.\\
        =&\left\{(x_1,\dots,x_n)\in\mathbb{R}^n|\sqrt{x_1^2+\cdots+x_n^2}\leq1\right\}
    \end{split}
\end{equation*}

\setcounter{section}{6}
\setcounter{theor}{12}

\begin{theor}[Punto Fijo de Brower]
    Sea $\cf{f}{B^n}{B^n}$ una función continua. Entonces $f$ tiene un punto fijo, es decir existe $x_0\in B^n$ tal que $f(x_0)=x_0$.
\end{theor}

\begin{proof}
    Procederemos por contradicción. Suponga que para todo $x\in B^n$ se tiene que $f(x)\neq x$. Sea $\cf{l_x}{[0,\infty[}{\mathbb{R^n}}$ la función dada por:
    \begin{equation*}
        s\mapsto f(x)+s\left(x-f(x)\right),\quad \forall s\in[0,\infty[
    \end{equation*}
    En esencia, $l_x$ es el rayo que une a $f(x)$ con $x$, y es prolongado en la dirección de $x$. Afirmamos que para $x\in B^n$ existe $s_0\in[0,\infty[$ tal que $\|l_x(s_0)\|=1$, es decir, que este rayo tiene un punto en $\partial B^n=S^{n-1}$, donde
    \begin{equation*}
        S^{n-1}=\left\{x\in\mathbb{R}^n|\|x\|=1\right\}
    \end{equation*}
    En efecto, sea $x\in B^n$. El mapeo $l_x$ es continuo en $[0,\infty[$ (por ser lineal). Como $\|x-f(x)\|>0$ ya que $x\neq f(x)$, se sigue que
    \begin{equation*}
        \begin{split}
            l_x(\frac{2}{\|x-f(x)\|})=&f(x)+\frac{2}{\|x-f(x)\|}\left(x-f(x)\right)\\
            \Rightarrow \|l_x(\frac{2}{\|x-f(x)\|})\|=&\|f(x)+\frac{2}{\|x-f(x)\|}\left(x-f(x)\right)\|\\
            \geq& 2\cdot\frac{\|x-f(x)\|}{\|x-f(x)\|}-\|f(x)\|\\
            =&2-\|f(x)\|\\
            \geq&1
        \end{split}
    \end{equation*}
    pues $0\leq \|f(x)\|\leq 1 \Rightarrow -1\leq \|f(x)\|\leq 0$. Tomando $s_1=\frac{2}{\|x-f(x)\|}$, se sigue que $\|l_x(s_1)\|\geq 1$ y $\|l_x(0)\|=\|f(x)\|\leq 1$. Por ser $l_x$ continua y $[0,\infty[$ conexo, entonces existir $s_0\in[0,s_1]$ tal que $\|l_x(s_0)\|=1$.

    \begin{obs}
        Tenemos dos casos, en caso de que haya dos de estos puntos, elegimos el máximo de ambos (este caso sucede cuando $x$ y $f(x)$ tienen ambos norma 1, para evitar el conflicto tomamos al mayor, es decir tal que $l_x(s_0)=x$, ya que el rayo parte desde $f(x)$ y luego pasa por $x$). Esto se hace para que en los puntos de la frontera, la función $\gamma$ definida más adelante tome sea la identidad en esta región. 

        De ahora en adelante $s_0(x)$ denotara al único elemento (y en caso de no ser único, al mayor) de $[0,\infty[$ tal que $l_x(s_0(x))=1$.
    \end{obs}

    Definamos $\cf{\gamma}{B^n}{S^{n-1}}$, $x\mapsto l_x(s_0)$. Por lo anterior esta función está bien definida. Veamos que es continua. Sea $x_0\in B^n$, $\varepsilon>0$ y $S=\max\left\{s_0(x),s_0(x_0)\right\}\geq0$. Como $f$ es continua en $x_0$ existe $\delta'>0$ si $x\in B^n$ con $\|x-x_0\|\leq\delta'$, entonces
    \begin{equation*}
        \|f(x)-f(x_0)\|<\frac{\varepsilon}{2(S+1)}
    \end{equation*}
    tomemos $\delta = \min\left\{\delta',\frac{\varepsilon}{2(S+1)}\right\}$. Si $x\in B^n$ es tal que $\|x-x_0\|\leq\delta$, entonces
    \begin{equation*}
        \begin{split}
            \|\gamma(x)-\gamma(x_0)\|=&\|l_x(s_0(x))-l_{x_0}(s_0(x_0))\|\\
            =&\|f(x)-s_0(x)\cdot\left(x-f(x)\right)-f(x_0)+s_0(x_0)\cdot\left(x_0-f(x_0)\right)\|\\
            \leq&\|f(x)-f(x_0)\|+S\cdot\|x-x_0-f(x)+f(x_0)\|\\
            \leq&\|f(x)-f(x_0)\|+S\cdot\|x-x_0\|+S\cdot\|f(x)-f(x_0)\|\\
            \leq&(S+1)\cdot\|f(x)-f(x_0)\|+S\cdot\|x-x_0\|\\
            \leq&(S+1)\cdot\|f(x)-f(x_0)\|+(S+1)\cdot\|x-x_0\|\\
            <&(S+1)\cdot\frac{\varepsilon}{2(S+1)}+(S+1)\cdot\frac{\varepsilon}{2(S+1)}\\
            =&\frac{\varepsilon}{2}+\frac{\varepsilon}{2}\\
            =&\varepsilon\\
        \end{split}
    \end{equation*}
    Por tanto, $\gamma$ es continua en $x_0$. Por ser el $x_0$ arbitrario se sigue que $\gamma$ es continua en $B^n$.

    Extendamos esta función a $\mathbb{R}^n$ haciendo que $\gamma(x)=x$, para todo $x\in\mathcal{C}$, donde
    \begin{equation*}
        \mathcal{C}=\left\{x\in\mathbb{R}^n|\|x\|\geq1\right\}
    \end{equation*}
    Como $\gamma$ coincide con la identidad en la frontera de $B^n=S^{n-1}$ y es continua en $B^n$ y $\mathcal{C}$ (pues en $\mathcal{C}$ es la identidad), por el teorema del pegado, $\gamma$ es continua en $\mathbb{R}^n$. Además esta función es $C^\infty$ en el complemento de $B^n$ (el cual es un conjunto compacto), luego las funciones componentes $\gamma_i$, $i=1,\dots,n$ también son $C^\infty$ en el complemento de $B^n$ y continuas en $\mathbb{R}^n$.
    Sea $\varepsilon>0$ y tomemos $\varepsilon'=\min\left\{\frac{1}{2},\varepsilon\right\}>0$, por el Lema 3.6.15, para cada $i=1,\dots,n$ existe una función $\cf{g_i}{\mathbb{R}^n}{\mathbb{R}}$ tal que $\gamma_i-g_i$ tiene soporte compacto y
    \begin{equation*}
        \abs{\gamma_i(x)-g_i(x)}<\varepsilon'\leq1,\quad\forall x\in\mathbb{R}^n
    \end{equation*}
    Sea $\cf{g}{\mathbb{R}^n}{\mathbb{R}^n}$ la función que tiene por funciones componentes las $g_i$. Esta función es $C^\infty$ y cumple que
    \begin{equation*}
        \begin{split}
            &\|\gamma(x)-g(x)\|_{\infty}< \varepsilon'\leq1\\
            \Rightarrow &\|\gamma(x)-g(x)\|< \varepsilon'\leq1,\quad\forall x\in\mathbb{R}^n\\
        \end{split}
    \end{equation*}
    (pues la bola euclideana está contenida en el cubo de mismo radio), donde la función $\gamma-g$ tiene soporte compacto (pues cada una de las funciones componentes lo tiene). Por tanto, existe un conjunto $\mathcal{D}\subseteq\mathbb{R}^n$ compacto en el cual $\gamma(x)=g(x)$ para todo $x\in\mathcal{D}^c$, sea $\mathcal{A}=B^n\cup\mathcal{D}$. $\mathcal{A}$ es compacto (por ser la unión de dos compactos) y $g$ coincide con la identidad en $\mathcal{A}^c$, entonces por la Proposición 3.6.1 $g$ es propio y $\deg(g)=1$.
    
    Por otra parte, si $x\in\mathbb{R}^n$, se tienen dos casos:
    \begin{enumerate}
        \item $x\in\mathcal{A}^c$, en este caso $g(x)=\gamma(x)=x$, por ende como $x\notin B^n$ se sigue que $\|g(x)\|\geq1>1-\varepsilon'$.
        \item $x\in\mathcal{A}$, en este caso como $\|\gamma(x)\|\geq1$ se sigue que
        \begin{equation*}
            \begin{split}
                \|\gamma(x)\|-\|g(x)\|&\leq\|g(x)-\gamma(x)\|\\
                &<\varepsilon'\\
                \Rightarrow \|\gamma(x)\|-\varepsilon'&<\|g(x)\|\\
                \Rightarrow 1-\varepsilon'&<\|g(x)\|\\
            \end{split}
        \end{equation*}
    \end{enumerate}
    Es decir, $\|g(x)\|>0$, para todo $x\in\mathbb{R}^n$, por lo cual $0\notin \textup{im}(g)$, es decir $g$ no es suprayectiva. Por tanto, por el Teorema 3.6.6 se tiene que $\deg(g)=0$, lo cual contradice el hecho de que $\deg(g)=1$.
    Así, existe al menos un $x\in B^n$ tal que $f(x_0)=x_0$.
\end{proof}

\end{document}