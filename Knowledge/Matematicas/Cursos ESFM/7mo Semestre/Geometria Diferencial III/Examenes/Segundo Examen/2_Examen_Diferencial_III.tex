\documentclass[12pt]{report}
\usepackage[spanish]{babel}
\usepackage[utf8]{inputenc}
\usepackage{amsmath}
\usepackage{amssymb}
\usepackage{amsthm}
\usepackage{graphics}
\usepackage{subfigure}
\usepackage{lipsum}
\usepackage{array}
\usepackage{multicol}
\usepackage{enumerate}
\usepackage[framemethod=TikZ]{mdframed}
\usepackage[a4paper, margin = 1.5cm]{geometry}

%En esta parte se hacen redefiniciones de algunos comandos para que resulte agradable el verlos%

\def\proof{\paragraph{Demostración:\\}}
\def\endproof{\hfill$\square$}
\renewcommand{\theenumii}{\roman{enumii}}

%En esta parte se definen los comandos a usar dentro del documento para enlistar%

\newtheoremstyle{largebreak}
  {}% use the default space above
  {}% use the default space below
  {\normalfont}% body font
  {}% indent (0pt)
  {\bfseries}% header font
  {}% punctuation
  {\newline}% break after header
  {}% header spec

\theoremstyle{largebreak}

\newmdtheoremenv[
    leftmargin=0em,
    rightmargin=0em,
    innertopmargin=-2pt,
    innerbottommargin=8pt,
    hidealllines = true,
    roundcorner = 5pt,
    backgroundcolor = gray!60!red!30
]{exa}{Ejemplo}[section]

\newmdtheoremenv[
    leftmargin=0em,
    rightmargin=0em,
    innertopmargin=-2pt,
    innerbottommargin=8pt,
    hidealllines = true,
    roundcorner = 5pt,
    backgroundcolor = gray!50!blue!30
]{obs}{Observación}[section]

\newmdtheoremenv[
    leftmargin=0em,
    rightmargin=0em,
    innertopmargin=-2pt,
    innerbottommargin=8pt,
    rightline = false,
    leftline = false
]{theor}{Teorema}[section]

\newmdtheoremenv[
    leftmargin=0em,
    rightmargin=0em,
    innertopmargin=-2pt,
    innerbottommargin=8pt,
    rightline = false,
    leftline = false
]{propo}{Proposición}[section]

\newmdtheoremenv[
    leftmargin=0em,
    rightmargin=0em,
    innertopmargin=-2pt,
    innerbottommargin=8pt,
    rightline = false,
    leftline = false
]{cor}{Corolario}[section]

\newmdtheoremenv[
    leftmargin=0em,
    rightmargin=0em,
    innertopmargin=-2pt,
    innerbottommargin=8pt,
    rightline = false,
    leftline = false
]{lema}{Lema}[section]

\newmdtheoremenv[
    leftmargin=0em,
    rightmargin=0em,
    innertopmargin=-2pt,
    innerbottommargin=8pt,
    roundcorner=5pt,
    backgroundcolor = gray!30,
    hidealllines = true
]{mydef}{Definición}[section]

\newmdtheoremenv[
    leftmargin=0em,
    rightmargin=0em,
    innertopmargin=-2pt,
    innerbottommargin=8pt,
    roundcorner=5pt
]{excer}{Ejercicio}[section]

%En esta parte se colocan comandos que definen la forma en la que se van a escribir ciertas funciones%

\newcommand\abs[1]{\ensuremath{\lvert#1\rvert}}
\newcommand\divides{\ensuremath{\bigm|}}
\newcommand\cf[3]{\ensuremath{#1:#2\rightarrow#3}}
\newcommand\supp[1]{\ensuremath{\textup{supp}\left(#1\right)}}

%recuerda usar \clearpage para hacer un salto de página

\begin{document}
    \title{Segundo Examen Geometría Diferencial III}
    \author{Cristo Daniel Alvarado}
    \date{\today}
    \maketitle

    \setcounter{chapter}{3} %En esta parte lo que se hace es cambiar la enumeración del capítulo%
    \setcounter{section}{5}
    \section{Demostración del teorema de punto fijo de Brower}
    
En este documento se pretende dar una prueba, lo más entendible posible, del teorema del punto fijo de Brower, desde el punto de vista de las formas diferenciales. 

Para la demostración del teorema, se requieren de unos resultados anteriores. Entre ellos se tiene el Teorema 3.5.11 y un lema (casi inmediato) del mismo:

\setcounter{section}{5}
\setcounter{theor}{10}

\begin{theor}
    Sea $V\subseteq\mathbb{R}^n$ abierto. Si $\cf{\phi}{\mathbb{R}^n}{\mathbb{R}}$ es una función continua tal que $\supp{\phi}\subseteq V$. Entonces para cada $\varepsilon>0$ existe una función $\cf{\psi}{\mathbb{R}^n}{\mathbb{R}}$ con soporte compacto tal que $\supp{\psi}\subseteq V$ y
    \begin{equation*}
        \sup_{x\in\mathbb{R}^n}\abs{\phi(x)-\psi(x)}<\varepsilon
    \end{equation*}
\end{theor}

\begin{proof}
    Denotemos por $A=\supp{\phi}$, y definamos
    \begin{equation*}
        d=\inf\left\{\|x-y\|_{\infty}|x\in A \textup{ y }y\in V^c\right\}
    \end{equation*}
    Recordando que 
    \begin{equation*}
        \|x\|_{\infty}=\max\left\{\abs{x_i}|i=1,\dots,n\right\}\textup{, donde }x=\left(x_1,\dots,x_n\right)\in\mathbb{R}^n
    \end{equation*}
    Afirmamos que $d>0$. Como $A\subseteq V$, entonces se tiene que $A\cap V^c=\emptyset$. Siendo que $V$ es abierto, se sigue que $V^c$ es cerrado. Defina $\cf{f}{A}{\mathbb{R}}$ dada como sigue:
    \begin{equation*}
        \begin{split}
            f(x)=&\inf\left\{\|x-z\|_{\infty}|z\in V^c\right\}\\
            =&d_\infty(x,V^c)\\
        \end{split}
    \end{equation*}
    para todo $x\in A$. Afirmamos que la función $f$ es continua en $A$. En efecto, sean $x\in A$ $\left\{x_n\right\}_{n=1}^{\infty}$ una sucesión de puntos de $A$ que convergen a $x$ (con la norma infinito). Se tiene
    \begin{equation}
        \begin{split}
            \abs{f(x)-f(x_n)}=&\abs{d_\infty(x,V^c)-d_\infty(x_n,V^c)}\\
        \end{split}
    \end{equation}
    Por la desigualdad del triángulo, se tiene que si $z\in V^c$:
    \begin{equation*}
        d_{\infty}(x,z)\leq d_{\infty}(x,x_n)+d_{\infty}(x_n,z)
    \end{equation*}
    Pero $d_{\infty}(x,V^c)\leq d_{\infty}(x,z)$. Por lo cual:
    \begin{equation*}
        \begin{split}
            d_{\infty}(x,V^c)\leq&d_{\infty}(x,x_n)+d_{\infty}(x_n,z)\\
            \Rightarrow d_{\infty}(x,V^c)-d_{\infty}(x,x_n)\leq&d_{\infty}(x_n,z)\\
        \end{split}
    \end{equation*}
    para todo $z\in V^c$. De esta forma, $d_{\infty}(x,V^c)-d_{\infty}(x,x_n)$ es cota inferior de $\left\{d_{\infty}(x_n,z)|z\in V^c\right\}$. Así
    \begin{equation*}
        \begin{split}
            d_{\infty}(x,V^c)-d_{\infty}(x,x_n)\leq&d_{\infty}(x_n,V^c)\\
            \Rightarrow d_{\infty}(x,V^c)-d_{\infty}(x_n,V^c)\leq&d_{\infty}(x,x_n)\\
        \end{split}
    \end{equation*}
    de forma análoga se prueba también que
    \begin{equation*}
        -\left(d_{\infty}(x,V^c)-d_{\infty}(x_n,V^c)\right)\leq d_{\infty}(x,x_n)
    \end{equation*}
    para todo $n\in\mathbb{N}$. Por lo cual
    \begin{equation*}
        \abs{d_{\infty}(x,A)-d_{\infty}(x_n,A)}\leq d_{\infty}(x, x_n)
    \end{equation*}
    para todo $n\in\mathbb{N}$. Retomando la ecuación (1.1) se sigue que
    \begin{equation*}
        0\leq\abs{f(x_n)-f(x)}\leq d_{\infty}(x,x_n)
    \end{equation*}
    Tomando ambos límites de ambos lados
    \begin{equation*}
        \lim_{n\rightarrow\infty}\abs{f(x_n)-f(x)}=0
    \end{equation*}
    pues $\lim_{n\rightarrow\infty}d_{\infty}(x,x_n)=0$. Luego $f$ es continua en $x$. Por ser $x\in A$ arbitrario, se sigue que $f$ es continua en $A$.

    Pero $A$ es un conjunto compacto y $f$ es una función que toma valores reales, por lo cual alcanza su máximo y su mínimo, digamos lo alcanza en $a\in A$, es decir
    \begin{equation*}
        f(a)\leq f(x),\quad\forall x\in A
    \end{equation*}
    donde $a\notin V^c$, pues en caso contrario se tendría que $A\cap V^c\neq \emptyset$. Se tiene que $f(a)=d_{\infty}(a,V^c)>0$, ya que en caso contrario, por propiedades del ínfimo existiría una sucesión en $V^c$ que converge a $a$, cosa que no puede suceder, pues $V^c$ es cerrado y $a\notin V^c$ (esto implicaría que $V^c$ no tiene a todos sus puntos de acumulación). Así
    \begin{equation*}
        \begin{split}
            d=&\inf\left\{\|x-y\|_{\infty}|x\in A \textup{ y }y\in V^c\right\}\\
            =&\inf\left\{d_{\infty}(x,V^c)|x\in A\right\}\\
            =&\inf\left\{f(x)|x\in A\right\}\\
            =&\min\left\{f(x)|x\in A\right\}\\
            =&f(a)\\
            >&0\\
        \end{split}
    \end{equation*}
    por lo cual $d>0$.
    
    Como $\phi$ es continua en el compacto $A$, es uniformemente continua en $A$. Pero también es uniformemente continua fuera de $A$, pues toma el valor constante $0$, así $\phi$ es continua en todo su dominio.
    Por tanto, para $\varepsilon>0$ existe $\delta'>0$ tal que si $x,y\in\mathbb{R}^n$ son tales que $\|x-y\|_{\infty}<\delta'$, entonces $\abs{\phi(x)-\phi(y)}<\varepsilon$. Sea $\delta =\min\left\{\delta',\frac{d}{2}\right\}>0$.
    
    Definamos
    \begin{equation*}
        Q=\left\{x\in\mathbb{R}^n|\|x\|_{\infty}\leq\delta\right\}
    \end{equation*}
    y sea $\cf{\rho}{\mathbb{R}^n}{\mathbb{R}}$ una función $C^{\infty}$ no negativa con soporte compacto tal que $\supp{\rho}\subseteq Q$ y
    \begin{equation*}
        \int_{\mathbb{R}^n}\rho(y)dy=1
    \end{equation*}
    
    Defina $\cf{\psi}{\mathbb{R}^n}{\mathbb{R}}$ dada como:
    \begin{equation*}
        \begin{split}
            \psi(x)=&\int_{\mathbb{R}^n}\rho(y-x)\phi(y)dy\\
            =&\int_{A}\rho(y-x)\phi(y)dy\\
        \end{split}
    \end{equation*}
    (ya que $\phi$ tiene soporte en $A$). Como $\rho$ es de clase $C^{\infty}$, se sigue que $\psi$ también lo es.

    Sea
    \begin{equation*}
        A_{\delta}=\left\{x\in\mathbb{R}^n|d_{\infty}(x,A)\leq\delta\right\}
    \end{equation*}
    Notemos que si $x\notin A_\delta$ entonces para todo $y\in A$, $\|x-y\|_{\infty}>\delta$, como $\supp{\rho}\subseteq Q$ entonces $\rho(x-y)=0$. Por tanto se tendría que
    para $x\notin A_\delta$:
    \begin{equation*}
        \begin{split}
            \psi(x)=&\int_{A}\rho(y-x)\phi(y)dy\\
            =&0
        \end{split}
    \end{equation*}
    de esta forma el soporte de $\psi$ es un subconjunto de $A_\delta$. Siendo que $\delta\leq\frac{d}{2}$, como la distancia de $A$ a $V^c$ es $d$, entonces $A_\delta\subseteq V$, así $\supp{\psi}\subseteq V$.

    Ahora, por el teorema de cambio de variable se tiene que
    
    \begin{equation*}
        \int_{\mathbb{R}^n}\rho(y-x)dy=\int_{\mathbb{R}^n}\rho(y)dy=1,\quad\forall x\in\mathbb{R}^n
    \end{equation*}

    Por tanto,

    \begin{equation*}
        \phi(x)=\int_{\mathbb{R}^n}\phi(x)\rho(y-x)dy,\quad \forall x\in\mathbb{R}^n
    \end{equation*}

    Así,

    \begin{equation*}
        \begin{split}
            \phi(x)-\psi(x)=&\int_{\mathbb{R}^n}\left(\phi(x)-\phi(y)\right)\rho(y-x)dy\\
            \Rightarrow \abs{\phi(x)-\psi(x)}\leq&\int_{\mathbb{R}^n}\abs{\phi(x)-\phi(y)}\rho(y-x)dy,\quad \forall x\in\mathbb{R}^n\\
        \end{split}
    \end{equation*}
    
    Notemos varias cosas. Primero, si tenemos $x,y\in\mathbb{R}^n$ tales que $\|x-y\|_\infty>\delta$, entonces $\rho(x-y)=0$. Si por el contrario $\|x-y\|_\infty\leq\delta$, entonces $\abs{\phi(x)-\phi(y)}<\varepsilon$. Por lo cual
    \begin{equation*}
        \begin{split}
            \abs{\phi(x)-\psi(x)}\leq&\int_{\mathbb{R}^n}\abs{\phi(x)-\phi(y)}\rho(y-x)dy\\
            <&\int_{\mathbb{R}^n}\varepsilon\cdot\rho(y-x)dy\\
            =&\varepsilon,\quad\forall x\in\mathbb{R}^n\\
        \end{split}
    \end{equation*}
    
    Luego,
    
    \begin{equation*}
        \sup_{x\in\mathbb{R}^n}\abs{\phi(x)-\psi(x)}<\varepsilon
    \end{equation*}

    \end{proof}

Con la prueba de este teorema hecha, se procederá a probar el siguiente lema:

\setcounter{section}{6}
\setcounter{lema}{14}

\begin{lema}
    Sea $U\subseteq\mathbb{R}^n$ abierto, $C\subseteq U$ compacto y $\cf{\phi}{U}{\mathbb{R}}$ una función continua que es clase $C^{\infty}$ en el complemento de C. Entonces, para todo $\varepsilon > 0$ existe una función $\cf{\psi}{U}{\mathbb{R}}$ $C^{\infty}$ en $U$, tal que $\phi - \psi$ tiene soporte compacto y $\abs{\phi - \psi}<\varepsilon$.
\end{lema}

\begin{proof}
    Sea $\varepsilon>0$ y $\cf{\rho}{U}{\mathbb{R}}$ una función $C_0^{\infty}(U)$ tal que toma el valor de $1$ en $C$. Por el Teorema 3.5.11 para $\rho\cdot\phi$ función continua y $\varepsilon>0$ existe una función $\psi_0\in C^{\infty}_{0}(U)$ tal que
    \begin{equation*}
        \abs{\rho\cdot\phi(x)-\psi_0(x)}<\varepsilon,\quad\forall x\in U
    \end{equation*}

    Tomemos $\psi=\left(1-\rho\right)\phi+\psi_0$. Ya se sabe que $\psi_0$ es $C^\infty$. Como $\rho$ toma el valor de 1 dentro de $C$, entonces $1-\rho$ toma el valor de $0$ en $C$. Por ser $\phi$ $C^\infty$ en el complemento de $C$, y tomar el valor de $0$ en $C$, se sigue que $\left(1-\rho\right)\phi$ es $C^\infty$ en $U$, luego la diferencia $\left(1-\rho\right)\phi+\psi_0=\psi$ lo es.

    Además, se cumple que
    \begin{equation*}
        \begin{split}
            \abs{\phi(x)-\psi(x)}=&\abs{\phi(x)-\left(\left(1-\rho\right)\phi+\psi_0\right)(x)}\\
            =&\abs{\phi(x)-\phi(x)+\rho(x)\phi(x)-\psi_0(x)}\\
            =&\abs{\rho(x)\phi(x)-\psi_0(x)}\\
            <&\varepsilon,\quad\forall x\in U\\
        \end{split}
    \end{equation*}
    
    donde la función $\rho\cdot\phi-\psi_0$ tiene soporte contenido en el compacto $C$ (por tenerlo $\rho$ y $\psi_0$). Por lo cual $\phi-\rho$ tienen soporte contenido en $C$, i.e. soporte compacto. 
\end{proof}

Asumiremos como cierto el siguiente resultado, el cual nos permitirá computar el grado de un mapeo.

\setcounter{section}{6}
\setcounter{theor}{1}

\begin{theor}[Sard]
    Sean $U$ y $V$ son subconjuntos abiertos de $\mathbb{R}^n$ y $\cf{f}{U}{V}$ un mapeo propio $C^\infty$. Entonces el conjunto de valores regulares de $f$ es un conjunto abierto denso de $V$.
\end{theor}

Con esto, se procederá con la prueba de algunos resultados.

\begin{theor}
    Sean $U$ y $V$ son subconjuntos abiertos de $\mathbb{R}^n$ y $\cf{f}{U}{V}$ un mapeo propio $C^\infty$. Sea $q\in U$ un valor regular de $f$. Entonces el conjunto $f^{-1}(q)$ es finito. Más aún, si $f^{-1}(1)=\left\{p_1,\dots,p_N\right\}$, existen vecindades abiertas conexas $U_i$ de $p_i$ para cada $i=1,\dots,N$ en $U$ y una vecindad abierta $W$ de $q$ contenida en $V$ tal que
    \begin{enumerate}
        \item Para $i\neq j$, los conjuntos $U_i$ y $U_j$ son disjuntos.
        \item $f^{-1}(W)=U_1\cup\cdots\cup U_N$.
        \item $f$ mapea a los $U_i$ difeomorfamente en $W$. 
    \end{enumerate}
\end{theor}

\begin{proof}
    
\end{proof}

\begin{theor}
    En las condiciones del Teorema 3.6.3, para cada $p_i\in f^{-1}(q)$, sea $\sigma_{p_i}=1$ si $\cf{f}{U_i}{W}$ preserva la orientación, y $-1$ si la invierte. Entonces
    \begin{equation*}
        \deg(f)=\sum_{i=1}^{N}\sigma_{p_i}
    \end{equation*}
\end{theor}

\begin{propo}
    Sea $U\subseteq \mathbb{R}^n$ abierto conexo y $\cf{f}{U}{U}$ un mapeo propio $C^{\infty}$. Entonces si $f|_{C^c}=\textup{id}_{C^c}$ donde $C\subseteq\mathbb{R}^n$ es compacto, entonces $f$ es propio y $\deg(f)=1$.
\end{propo}

\begin{proof}
    Claramente $f$ es un mapeo propio (pues por hipótesis se tiene que es mapeo propio $C^\infty$).
\end{proof}

Denotemos por $B^n$ a la bola unitaria en $\mathbb{R}^n$, es decir, se denota al conjunto
\begin{equation*}
    \begin{split}
        B^n=&\left\{x\in\mathbb{R}^n|\|x\|\leq 1\right\}.\\
        =&\left\{(x_1,\dots,x_n)\in\mathbb{R}^n|\sqrt{x_1^2+\cdots+x_n^2}\leq1\right\}
    \end{split}
\end{equation*}

\setcounter{section}{6}
\setcounter{theor}{12}

\begin{theor}[Punto Fijo de Brower]
    Sea $\cf{f}{B^n}{B^n}$ una función continua. Entonces $f$ tiene un punto fijo, es decir existe $x_0\in B^n$ tal que $f(x_0)=x_0$.
\end{theor}

\begin{proof}
    Procederemos por contradicción. Suponga que para todo $x\in B^n$ se tiene que $f(x)\neq x$. Sea $\cf{l_x}{[0,\infty[}{\mathbb{R^n}}$ la función dada por:
    \begin{equation*}
        s\mapsto f(x)+s\left(x-f(x)\right),\quad \forall s\in[0,\infty[
    \end{equation*}
    En esencia, $l_x$ es el rayo que une a $x$ con $f(x)$, y es prolongado en la dirección de $x$. Afirmamos que para $x\in B^n$ existe $s_0\in[0,\infty[$ tal que $\|l_x(s_0)\|=1$, es decir, que este rayo tiene un punto en $\partial B^n=S^{n-1}$, donde
    \begin{equation*}
        S^{n-1}=\left\{x\in\mathbb{R}^n|\|x\|=1\right\}
    \end{equation*}
    En efecto, sea $x\in B^n$. El mapeo $l_x$ es continuo en $[0,\infty[$ (por ser lineal). Como $\|x-f(x)\|>0$ ya que $x\neq f(x)$, se sigue que
    \begin{equation*}
        \begin{split}
            l_x(\frac{2}{\|x-f(x)\|})=&f(x)+\frac{2}{\|x-f(x)\|}\left(x-f(x)\right)\\
            \Rightarrow \|l_x(\frac{2}{\|x-f(x)\|})\|=&\|f(x)+\frac{2}{\|x-f(x)\|}\left(x-f(x)\right)\|\\
            \geq& 2\cdot\frac{\|x-f(x)\|}{\|x-f(x)\|}-\|f(x)\|\\
            =&2-\|f(x)\|\\
            \geq&1
        \end{split}
    \end{equation*}
    pues $0\leq \|f(x)\|\leq 1 \Rightarrow -1\leq \|f(x)\|\leq 0$. Tomando $s_1=\frac{2}{\|x-f(x)\|}$, se sigue que $\|l_x(s_1)\|\geq 1$ y $\|l_x(0)\|=\|f(x)\|\leq 1$. Por ser $l_x$ continua y $[0,\infty[$ conexo, entonces existir $s_0\in[0,s_1]$ tal que $\|l_x(s_0)\|=1$. Pero la función $l_x$ es inyectiva, por tanto este $s_0$ es único.

    \begin{obs}
    De ahora en adelanta $s_0(x)$ denotara al único elemento de $[0,\infty[$ tal que $l_x(s_0(x))=1$.
    \end{obs}

    Definamos $\cf{\gamma}{B^n}{S^{n-1}}$, $x\mapsto l_x(s_0)$. Por lo anterior esta función está bien definida. Veamos que es continua. Sea $x_0\in B^n$, $\varepsilon>0$ y $S=\max\left\{s_0(x),s_0(x_0)\right\}\geq0$. Como $f$ es continua en $x_0$ existe $\delta'>0$ si $x\in B^n$ con $\|x-x_0\|\leq\delta'$, entonces
    \begin{equation*}
        \|f(x)-f(x_0)\|<\frac{\varepsilon}{2(S+1)}
    \end{equation*}
    tomemos $\delta = \min\left\{\delta',\frac{\varepsilon}{2(S+1)}\right\}$. Si $x\in B^n$ es tal que $\|x-x_0\|\leq\delta$, entonces
    \begin{equation*}
        \begin{split}
            \|\gamma(x)-\gamma(x_0)\|=&\|l_x(s_0(x))-l_{x_0}(s_0(x_0))\|\\
            =&\|f(x)-s_0(x)\cdot\left(x-f(x)\right)-f(x_0)+s_0(x_0)\cdot\left(x_0-f(x_0)\right)\|\\
            \leq&\|f(x)-f(x_0)\|+S\cdot\|x-x_0-f(x)+f(x_0)\|\\
            \leq&\|f(x)-f(x_0)\|+S\cdot\|x-x_0\|+S\cdot\|f(x)-f(x_0)\|\\
            \leq&(S+1)\cdot\|f(x)-f(x_0)\|+S\cdot\|x-x_0\|\\
            \leq&(S+1)\cdot\|f(x)-f(x_0)\|+(S+1)\cdot\|x-x_0\|\\
            <&(S+1)\cdot\frac{\varepsilon}{2(S+1)}+(S+1)\cdot\frac{\varepsilon}{2(S+1)}\\
            =&\frac{\varepsilon}{2}+\frac{\varepsilon}{2}\\
            =&\varepsilon\\
        \end{split}
    \end{equation*}
    Por tanto, $\gamma$ es continua en $x_0$. Por ser el $x_0$ arbitrario se sigue que $\gamma$ es continua en $B^n$.

\end{proof}

\end{document}