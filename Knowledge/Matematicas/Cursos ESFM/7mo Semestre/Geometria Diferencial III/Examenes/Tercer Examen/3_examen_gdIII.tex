\documentclass[12pt]{report}
\usepackage[spanish]{babel}
\usepackage[utf8]{inputenc}
\usepackage{amsmath}
\usepackage{amssymb}
\usepackage{amsthm}
\usepackage{graphics}
\usepackage{subfigure}
\usepackage{lipsum}
\usepackage{array}
\usepackage{multicol}
\usepackage{enumerate}
\usepackage[framemethod=TikZ]{mdframed}
\usepackage[a4paper, margin = 1.5cm]{geometry}

%En esta parte se hacen redefiniciones de algunos comandos para que resulte agradable el verlos%

\def\proof{\paragraph{Demostración:\\}}
\def\endproof{\hfill$\square$}
\renewcommand{\theenumii}{\roman{enumii}}

%En esta parte se definen los comandos a usar dentro del documento para enlistar%

\newtheoremstyle{largebreak}
  {}% use the default space above
  {}% use the default space below
  {\normalfont}% body font
  {}% indent (0pt)
  {\bfseries}% header font
  {}% punctuation
  {\newline}% break after header
  {}% header spec

\theoremstyle{largebreak}

\newmdtheoremenv[
    leftmargin=0em,
    rightmargin=0em,
    innertopmargin=-2pt,
    innerbottommargin=8pt,
    hidealllines = true,
    roundcorner = 5pt,
    backgroundcolor = gray!60!red!30
]{exa}{Ejemplo}[section]

\newmdtheoremenv[
    leftmargin=0em,
    rightmargin=0em,
    innertopmargin=-2pt,
    innerbottommargin=8pt,
    hidealllines = true,
    roundcorner = 5pt,
    backgroundcolor = gray!50!blue!30
]{obs}{Observación}[section]

\newmdtheoremenv[
    leftmargin=0em,
    rightmargin=0em,
    innertopmargin=-2pt,
    innerbottommargin=8pt,
    rightline = false,
    leftline = false
]{theor}{Teorema}[section]

\newmdtheoremenv[
    leftmargin=0em,
    rightmargin=0em,
    innertopmargin=-2pt,
    innerbottommargin=8pt,
    rightline = false,
    leftline = false
]{propo}{Proposición}[section]

\newmdtheoremenv[
    leftmargin=0em,
    rightmargin=0em,
    innertopmargin=-2pt,
    innerbottommargin=8pt,
    rightline = false,
    leftline = false
]{cor}{Corolario}[section]

\newmdtheoremenv[
    leftmargin=0em,
    rightmargin=0em,
    innertopmargin=-2pt,
    innerbottommargin=8pt,
    rightline = false,
    leftline = false
]{lema}{Lema}[section]

\newmdtheoremenv[
    leftmargin=0em,
    rightmargin=0em,
    innertopmargin=-2pt,
    innerbottommargin=8pt,
    roundcorner=5pt,
    backgroundcolor = gray!30,
    hidealllines = true
]{mydef}{Definición}[section]

\newmdtheoremenv[
    leftmargin=0em,
    rightmargin=0em,
    innertopmargin=-2pt,
    innerbottommargin=8pt,
    roundcorner=5pt
]{excer}{Ejercicio}[section]

%En esta parte se colocan comandos que definen la forma en la que se van a escribir ciertas funciones%

\newcommand\abs[1]{\ensuremath{\lvert#1\rvert}}
\newcommand\divides{\ensuremath{\bigm|}}
\newcommand\cf[3]{\ensuremath{#1:#2\rightarrow#3}}

%recuerda usar \clearpage para hacer un salto de página

\begin{document}
    \title{Tercer Examen Geometría Diferencial III}
    \author{Cristo Daniel Alvarado}
    \maketitle

    \tableofcontents %Con este comando se genera el índice general del libro%

    %\setcounter{chapter}{3} %En esta parte lo que se hace es cambiar la enumeración del capítulo%
    
    \chapter{Examen}
    
    \renewcommand{\theenumi}{\roman{enumi}}
    \renewcommand{\labelenumi}{{(\theenumi)}}

    \section{Ejercicio 1}

    \begin{excer}[\textbf{Pullback de una forma diferencial}]
        Considere $U\subseteq ]0,\infty[\times]0,\pi[\times]0,2\pi[$ abierto en el espacio $(\rho, \phi, \theta)$ de $\mathbb{R}^3$. Defina $\cf{f}{F}{\mathbb{R}^3}$ dada como
        \begin{equation*}
            (x,y,z)=F(\rho,\phi,\theta)=(\rho\sin\phi\cos\theta,\rho\sin\phi\sin\theta,\rho\cos\phi)
        \end{equation*}
        pruebe que $F^*(dx\wedge dy\wedge dz)=\rho^2\sin\phi\:d\rho\wedge d\phi\wedge d\theta$.
    \end{excer}

    \begin{proof}
        Primeramente, veamos que
        \begin{equation*}
            F^*(dx\wedge dy\wedge dz)=F^*dx\wedge F^*dy\wedge F^*dz
        \end{equation*}
        pues $F$ es un mapeo $C^\infty$ entre las variedades $U$ y $\mathbb{R}^2$. Por la Proposición 18.11 se sigue la identidad de arriba.
        Pero, el producto wedge conmuta con la diferencial exterior, por lo cual
        \begin{equation}
            F^*dx=d(F^*x),\quad F^*dy=d(F^*y)\quad\textup{y }F^*dz=d(F^*z)
        \end{equation}
        siendo $\cf{x,y,z}{\mathbb{R}^3}{\mathbb{R}}$ las funciones coordenadas de $\mathbb{R}^3$. Con esto en mente, analicemos cada una de las diferenciales exteriores de (1.1).
        \begin{enumerate}
            \item Notemos que $F^*x=x\circ F$, por lo cual
            \begin{equation*}
                \begin{split}
                    d(F^*x)&=d(x\circ F)\\
                    &=d(F_1)\\
                \end{split}
            \end{equation*}
            donde $\cf{F_1}{U}{\mathbb{R}}$ es la función tal que $((\rho,\phi,\theta))\mapsto x\circ F((\rho,\phi,\theta))=\rho\sin\phi \cos\theta$, para todo $(\rho,\phi,\theta)\in U$. Por lo cual
            \begin{equation*}
                \begin{split}
                    d(F^*x)&=d(\rho\sin\phi\cos\theta)\\
                    &=\frac{\partial}{\partial\rho}(\rho\sin\phi\cos\theta)d\rho+\frac{\partial}{\partial\phi}(\rho\sin\phi\cos\theta)d\phi+\frac{\partial}{\partial\theta}(\rho\sin\phi\cos\theta)d\theta\\
                    &=\sin\phi\cos\theta d\rho+\rho\cos\phi\cos\theta d\phi-\rho\sin\phi\sin\theta d\theta\\
                    \Rightarrow d(F^*x)&=\sin\phi\cos\theta d\rho+\rho\cos\phi\cos\theta d\phi-\rho\sin\phi\sin\theta d\theta\\
                \end{split}
            \end{equation*}
            \item De forma similar al inciso (I), se tiene que
            \begin{equation*}
                \begin{split}
                    d(F^*y)&=d(y\circ F)\\
                    &=d(F_2)\\
                \end{split}
            \end{equation*}
            donde $\cf{F_2}{U}{\mathbb{R}}$ es la función tal que $((\rho,\phi,\theta))\mapsto y\circ F((\rho,\phi,\theta))=\rho\sin\phi \sin\theta$, para todo $(\rho,\phi,\theta)\in U$. Luego
            \begin{equation*}
                \begin{split}
                    d(F^*y)&=d(\rho\sin\phi\sin\theta)\\
                    &=\frac{\partial}{\partial\rho}(\rho\sin\phi\sin\theta)d\rho+\frac{\partial}{\partial\phi}(\rho\sin\phi\sin\theta)d\phi+\frac{\partial}{\partial\theta}(\rho\sin\phi\sin\theta)d\theta\\
                    &=\sin\phi\sin\theta d\rho+\rho\cos\phi\sin\theta d\phi+\rho\sin\phi\cos\theta d\theta\\
                    \Rightarrow d(F^*y)&=\sin\phi\sin\theta d\rho+\rho\cos\phi\sin\theta d\phi+\rho\sin\phi\cos\theta d\theta\\
                \end{split}
            \end{equation*}
            \item De forma similar al inciso (I), se tiene que
            \begin{equation*}
                \begin{split}
                    d(F^*z)&=d(z\circ F)\\
                    &=d(F_3)\\
                \end{split}
            \end{equation*}
            donde $\cf{F_3}{U}{\mathbb{R}}$ es la función tal que $((\rho,\phi,\theta))\mapsto z\circ F((\rho,\phi,\theta))=\rho\cos\phi$, para todo $(\rho,\phi,\theta)\in U$. Luego
            \begin{equation*}
                \begin{split}
                    d(F^*z)&=d(\rho\cos\phi)\\
                    &=\frac{\partial}{\partial\rho}(\rho\cos\phi)d\rho+\frac{\partial}{\partial\phi}(\rho\cos\phi)d\phi+\frac{\partial}{\partial\theta}(\rho\cos\phi)d\theta\\
                    &=\cos\phi d\rho-\rho\sin\phi d\phi\\
                    \Rightarrow d(F^*z)&=\cos\phi d\rho-\rho\sin\phi d\phi\\
                \end{split}
            \end{equation*}
        \end{enumerate}
        Por los tres incisos anteriores, se sigue que
        \begin{multline*}
            F^*(dx\wedge dy\wedge dz)=(\sin\phi\cos\theta d\rho+\rho\cos\phi\cos\theta d\phi-\rho\sin\phi\sin\theta d\theta)\\ \wedge(\sin\phi\sin\theta d\rho+\rho\cos\phi\sin\theta d\phi+\rho\sin\phi\cos\theta d\theta) \wedge(\cos\phi d\rho-\rho\sin\phi d\phi)
        \end{multline*}
        Computemos $(\sin\phi\sin\theta d\rho+\rho\cos\phi\sin\theta d\phi+\rho\sin\phi\cos\theta d\theta)\wedge(\cos\phi d\rho-\rho\sin\phi d\phi)$:
        \begin{multline*}
            (\sin\phi\sin\theta d\rho+\rho\cos\phi\sin\theta d\phi+\rho\sin\phi\cos\theta d\theta)\wedge(\cos\phi d\rho-\rho\sin\phi d\phi)\\
            =(\rho\cos^2\phi\sin\theta d\phi\wedge d\rho+\rho \sin\phi\cos\phi\cos\theta d\theta\wedge d\rho)+(-\rho \sin^2\sin\theta\phi d\rho\wedge d\phi-\rho^2\sin^2\phi\cos\theta d\theta\wedge d\phi)\\
            =(-\rho\cos^2\phi\sin\theta d\rho\wedge d\phi-\rho \sin\phi\cos\phi\cos\theta d\rho\wedge d\theta)+(-\rho \sin^2\phi\sin\theta d\rho\wedge d\phi+\rho^2\sin^2\phi\cos\theta d\phi\wedge d\theta)\\
            =-\rho\sin\theta d\rho\wedge d\phi-\rho \sin\phi\cos\phi\cos\theta d\rho\wedge d\theta+\rho^2\sin^2\phi\cos\theta d\phi\wedge d\theta
        \end{multline*}
        luego,
        \begin{multline*}
            \Rightarrow(\sin\phi\cos\theta d\rho+\rho\cos\phi\cos\theta d\phi-\rho\sin\phi\sin\theta d\theta)\\ \wedge(\sin\phi\sin\theta d\rho+\rho\cos\phi\sin\theta d\phi+\rho\sin\phi\cos\theta d\theta) \wedge(\cos\phi d\rho-\rho\sin\phi d\phi)\\
            =(\sin\phi\cos\theta d\rho+\rho\cos\phi\cos\theta d\phi-\rho\sin\phi\sin\theta d\theta)\\
            \wedge(-\rho\sin\theta d\rho\wedge d\phi-\rho \sin\phi\cos\phi\cos\theta d\rho\wedge d\theta+\rho^2\sin^2\phi\cos\theta d\phi\wedge d\theta)\\
            =\rho^2\sin\phi\sin^2\theta d\theta\wedge d\rho\wedge d\phi-\rho^2\sin\phi\cos^2\phi\cos^2\theta d\phi\wedge d\rho\wedge d\theta
            +\rho^2\sin^3\phi\cos^2\theta d\rho\wedge d\phi\wedge d\theta\\
            =\rho^2\sin\phi\sin^2\theta d\rho\wedge d\phi\wedge d\theta
            +\rho^2\sin\phi\cos^2\phi\cos^2\theta d\rho\wedge d\phi\wedge d\theta
            +\rho^2\sin^3\phi\cos^2\theta d\rho\wedge d\phi\wedge d\theta\\
            =\rho^2\sin\phi\sin^2\theta d\rho\wedge d\phi\wedge d\theta
            +\rho^2\sin\phi\cos^2\theta\left[\cos^2\phi+\sin^2\phi\right]d\rho\wedge d\phi\wedge d\theta\\
            =\rho^2\sin\phi\sin^2\theta d\rho\wedge d\phi\wedge d\theta
            +\rho^2\sin\phi\cos^2\theta d\rho\wedge d\phi\wedge d\theta\\
            =\rho^2\sin\phi\left[\sin^2\theta+\cos^2\theta\right] d\rho\wedge d\phi\wedge d\theta\\
            =\rho^2\sin\phi \:d\rho\wedge d\phi\wedge d\theta\\
        \end{multline*}
        por tanto
        \begin{equation*}
            F^*(dx\wedge dy\wedge dz)=\rho^2\sin\phi\:d\rho\wedge d\phi\wedge d\theta
        \end{equation*}
    \end{proof}

    \newpage

    \section{Ejercicio 2}

    \begin{excer}[\textbf{Pullback de una forma diferencial}]
        Sea $\cf{F}{\mathbb{R}^2}{\mathbb{R}^2}$ la función dada por
        \begin{equation*}
            F(x,y)=(u,v)=(x^2+y^2,xy)
        \end{equation*}
        Compute $F^*(u\:du+v\:dv)$.
    \end{excer}

    \begin{proof}
        Como el pullback es lineal, se tiene entonces que
        \begin{equation*}
            F^*(u\:du+v\:dv)=F^*(u\:du)+F^*(v\:dv)
        \end{equation*}
        Determinemos $F^*(u\:du)$ y $F^*(v\:dv)$.
        \begin{enumerate}
            \item Veamos que
            \begin{equation*}
                \begin{split}
                    F^*(u\:du)&=F^*(u\wedge du)\\
                    &=(F^*u)\wedge(F^*du)\\
                    &=(F^*u)\wedge d(F^*u)\\
                \end{split}
            \end{equation*}
            pues, podemos ver a la $1$-forma $u\:du$ como el producto wedge de una $0$-forma con una $1$-forma. De esta forma, por propiedades del pullback, se sigue la primera y segunda igualdad. Para la tercera, se cumple ya que el producto wedge conmuta con la diferencial exterior.
            
            Computemos $F^*u$:

            \begin{equation*}
                \begin{split}
                    F^*u&=u\circ F
                \end{split}
            \end{equation*}
            es decir, $F^*u$ es la función de $\mathbb{R}^2$ en $\mathbb{R}$ tal que $(x,y)\mapsto u\circ F(x,y)=u(x^2+y^2,xy)=x^2+y^2$. De esta forma, se tiene que
            \begin{equation*}
                \begin{split}
                    d(F^*u)&=d(x^2+y^2)\\
                    &=\frac{\partial }{\partial x}(x^2+y^2)\:dx+\frac{\partial }{\partial y}(x^2+y^2)\:dy\\
                    &=2x\:dx+2y\:dy\\
                    \Rightarrow d(F^*u)&=2x\:dx+2y\:dy\\
                \end{split}
            \end{equation*}
            \item Como en el inciso (I), se tiene que
            \item 
        \end{enumerate}
    \end{proof}

    \newpage

    \section{Ejercicio 3}

    \begin{excer}[\textbf{Pullback de una forma diferencial por una curva}]
        Sea $\omega$ la $1$-forma dada por
        \begin{equation*}
            \omega = \frac{-ydx+xdy}{x^2+y^2}
        \end{equation*}
        en $\mathbb{R}^2\backslash\left\{0\right\}$. Defina $\cf{c}{\mathbb{R}}{\mathbb{R}^2}$, $t\mapsto (\cos t,\sin t)$. Compute $c^*\omega$.
    \end{excer}

    \newpage

    \renewcommand{\theenumi}{\alph{enumi}}
    \renewcommand{\labelenumi}{{(\theenumi)}}

    \section{Ejercicio 4}

    \begin{excer}[\textbf{Una forma que no se desvanece sobre una hypersuperficie suave}]
        Resuelva los siguientes incisos.
        \begin{enumerate}
            \item Sea $f(x,y)$ una función $C^\infty$ en $\mathbb{R}^2$ y asuma que $0$ es valor regular de $f$. Por el teorema del valor regular, el conjunto cero $M$ de $f(x,y)$ es una subvariedad $1$-dimensional de $\mathbb{R}^2$. Construya una $1$-forma que no se anula en $M$.
            \item Sea $f(x,y,z)$ una función $C^\infty$ en $\mathbb{R}^3$ y asuma que $0$ es valor regular de $f$. Por el teorema del valor regular, el conjunto cero $M$ de $f(x,y,z)$ es una subvariedad $2$-dimensional de $\mathbb{R}^3$. Sean $f_x$, $f_y$ y $f_z$ las derivadas parciales de $f$ con respecto a $x$, $y$ y $z$, respectivamente. Muestre que las igualdades
            \begin{equation*}
                \frac{dx\wedge dy}{f_z}=\frac{dy\wedge dz}{f_x}=\frac{dz\wedge dx}{f_y}
            \end{equation*}
            son válidas en $M$ siempre y cuando estas tengan sentido, y por tanto juntas dan una $2$-forma sobre $M$ que no se desvanece en ninguna parte.
            \item Generalice este problema al conjunto de nivel regular de $f(x^1,\dots,x^{n+1})$ en $\mathbb{R}^{n+1}$.
        \end{enumerate}
    \end{excer}

    \newpage

    \begin{theor}[Nombre]
        Teorema
    \end{theor}

    \begin{propo}[Nombre]
        Proposición
    \end{propo}

    \begin{cor}[Nombre]
        Corolario
    \end{cor}

    \begin{lema}[Nombre]
        Lema
    \end{lema}

    \begin{mydef}[Nombre]
        Definición
    \end{mydef}

    \begin{obs}[Nombre]
        Observación
    \end{obs}

    \begin{exa}[Nombre]
        Ejemplo
    \end{exa}

    \begin{excer}[Nombre]
        Ejercicio
    \end{excer}

\end{document}