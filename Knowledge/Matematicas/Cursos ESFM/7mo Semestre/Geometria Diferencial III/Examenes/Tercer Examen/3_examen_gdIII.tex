\documentclass[12pt]{report}
\usepackage[spanish]{babel}
\usepackage[utf8]{inputenc}
\usepackage{amsmath}
\usepackage{amssymb}
\usepackage{amsthm}
\usepackage{graphics}
\usepackage{subfigure}
\usepackage{lipsum}
\usepackage{array}
\usepackage{multicol}
\usepackage{enumerate}
\usepackage[framemethod=TikZ]{mdframed}
\usepackage[a4paper, margin = 1.5cm]{geometry}

%En esta parte se hacen redefiniciones de algunos comandos para que resulte agradable el verlos%

\def\proof{\paragraph{Demostración:\\}}
\def\endproof{\hfill$\square$}
\renewcommand{\theenumii}{\roman{enumii}}

%En esta parte se definen los comandos a usar dentro del documento para enlistar%

\newtheoremstyle{largebreak}
  {}% use the default space above
  {}% use the default space below
  {\normalfont}% body font
  {}% indent (0pt)
  {\bfseries}% header font
  {}% punctuation
  {\newline}% break after header
  {}% header spec

\theoremstyle{largebreak}

\newmdtheoremenv[
    leftmargin=0em,
    rightmargin=0em,
    innertopmargin=-2pt,
    innerbottommargin=8pt,
    hidealllines = true,
    roundcorner = 5pt,
    backgroundcolor = gray!60!red!30
]{exa}{Ejemplo}[section]

\newmdtheoremenv[
    leftmargin=0em,
    rightmargin=0em,
    innertopmargin=-2pt,
    innerbottommargin=8pt,
    hidealllines = true,
    roundcorner = 5pt,
    backgroundcolor = gray!50!blue!30
]{obs}{Observación}[section]

\newmdtheoremenv[
    leftmargin=0em,
    rightmargin=0em,
    innertopmargin=-2pt,
    innerbottommargin=8pt,
    rightline = false,
    leftline = false
]{theor}{Teorema}[section]

\newmdtheoremenv[
    leftmargin=0em,
    rightmargin=0em,
    innertopmargin=-2pt,
    innerbottommargin=8pt,
    rightline = false,
    leftline = false
]{propo}{Proposición}[section]

\newmdtheoremenv[
    leftmargin=0em,
    rightmargin=0em,
    innertopmargin=-2pt,
    innerbottommargin=8pt,
    rightline = false,
    leftline = false
]{cor}{Corolario}[section]

\newmdtheoremenv[
    leftmargin=0em,
    rightmargin=0em,
    innertopmargin=-2pt,
    innerbottommargin=8pt,
    rightline = false,
    leftline = false
]{lema}{Lema}[section]

\newmdtheoremenv[
    leftmargin=0em,
    rightmargin=0em,
    innertopmargin=-2pt,
    innerbottommargin=8pt,
    roundcorner=5pt,
    backgroundcolor = gray!30,
    hidealllines = true
]{mydef}{Definición}[section]

\newmdtheoremenv[
    leftmargin=0em,
    rightmargin=0em,
    innertopmargin=-2pt,
    innerbottommargin=8pt,
    roundcorner=5pt
]{excer}{Ejercicio}[section]

%En esta parte se colocan comandos que definen la forma en la que se van a escribir ciertas funciones%

\newcommand\abs[1]{\ensuremath{\lvert#1\rvert}}
\newcommand\divides{\ensuremath{\bigm|}}
\newcommand\cf[3]{\ensuremath{#1:#2\rightarrow#3}}

%recuerda usar \clearpage para hacer un salto de página

\begin{document}
    \title{Tercer Examen Geometría Diferencial III}
    \author{Cristo Daniel Alvarado}
    \maketitle

    \tableofcontents %Con este comando se genera el índice general del libro%

    %\setcounter{chapter}{3} %En esta parte lo que se hace es cambiar la enumeración del capítulo%
    
    \chapter{Examen}
    
    \renewcommand{\theenumi}{\roman{enumi}}
    \renewcommand{\labelenumi}{{(\theenumi)}}

    \section{Ejercicio 1}

    \begin{excer}[\textbf{Pullback de una forma diferencial}]
        Considere $U\subseteq ]0,\infty[\times]0,\pi[\times]0,2\pi[$ abierto en el espacio $(\rho, \phi, \theta)$ de $\mathbb{R}^3$. Defina $\cf{f}{F}{\mathbb{R}^3}$ dada como
        \begin{equation*}
            (x,y,z)=F(\rho,\phi,\theta)=(\rho\sin\phi\cos\theta,\rho\sin\phi\sin\theta,\rho\cos\phi)
        \end{equation*}
        pruebe que $F^*(dx\wedge dy\wedge dz)=\rho^2\sin\phi\:d\rho\wedge d\phi\wedge d\theta$.
    \end{excer}

    \begin{proof}
        Primeramente, veamos que
        \begin{equation*}
            F^*(dx\wedge dy\wedge dz)=F^*dx\wedge F^*dy\wedge F^*dz
        \end{equation*}
        pues $F$ es un mapeo $C^\infty$ entre las variedades $U$ y $\mathbb{R}^2$. Por la Proposición 18.11 se sigue la identidad de arriba.
        Pero, el producto wedge conmuta con la diferencial exterior, por lo cual
        \begin{equation}
            F^*dx=d(F^*x),\quad F^*dy=d(F^*y)\quad\textup{y }F^*dz=d(F^*z)
        \end{equation}
        siendo $\cf{x,y,z}{\mathbb{R}^3}{\mathbb{R}}$ las funciones coordenadas de $\mathbb{R}^3$. Con esto en mente, analicemos cada una de las diferenciales exteriores de (1.1).
        \begin{enumerate}
            \item Notemos que $F^*x=x\circ F$, por lo cual
            \begin{equation*}
                \begin{split}
                    d(F^*x)&=d(x\circ F)\\
                    &=d(F_1)\\
                \end{split}
            \end{equation*}
            donde $\cf{F_1}{U}{\mathbb{R}}$ es la función tal que $((\rho,\phi,\theta))\mapsto x\circ F((\rho,\phi,\theta))=\rho\sin\phi \cos\theta$, para todo $(\rho,\phi,\theta)\in U$. Por lo cual
            \begin{equation*}
                \begin{split}
                    d(F^*x)&=d(\rho\sin\phi\cos\theta)\\
                    &=\frac{\partial}{\partial\rho}(\rho\sin\phi\cos\theta)d\rho+\frac{\partial}{\partial\phi}(\rho\sin\phi\cos\theta)d\phi+\frac{\partial}{\partial\theta}(\rho\sin\phi\cos\theta)d\theta\\
                    &=\sin\phi\cos\theta d\rho+\rho\cos\phi\cos\theta d\phi-\rho\sin\phi\sin\theta d\theta\\
                    \Rightarrow d(F^*x)&=\sin\phi\cos\theta d\rho+\rho\cos\phi\cos\theta d\phi-\rho\sin\phi\sin\theta d\theta\\
                \end{split}
            \end{equation*}
            \item De forma similar al inciso (I), se tiene que
            \begin{equation*}
                \begin{split}
                    d(F^*y)&=d(y\circ F)\\
                    &=d(F_2)\\
                \end{split}
            \end{equation*}
            donde $\cf{F_2}{U}{\mathbb{R}}$ es la función tal que $((\rho,\phi,\theta))\mapsto y\circ F((\rho,\phi,\theta))=\rho\sin\phi \sin\theta$, para todo $(\rho,\phi,\theta)\in U$. Luego
            \begin{equation*}
                \begin{split}
                    d(F^*y)&=d(\rho\sin\phi\sin\theta)\\
                    &=\frac{\partial}{\partial\rho}(\rho\sin\phi\sin\theta)d\rho+\frac{\partial}{\partial\phi}(\rho\sin\phi\sin\theta)d\phi+\frac{\partial}{\partial\theta}(\rho\sin\phi\sin\theta)d\theta\\
                    &=\sin\phi\sin\theta d\rho+\rho\cos\phi\sin\theta d\phi+\rho\sin\phi\cos\theta d\theta\\
                    \Rightarrow d(F^*y)&=\sin\phi\sin\theta d\rho+\rho\cos\phi\sin\theta d\phi+\rho\sin\phi\cos\theta d\theta\\
                \end{split}
            \end{equation*}
            \item De forma similar al inciso (I), se tiene que
            \begin{equation*}
                \begin{split}
                    d(F^*z)&=d(z\circ F)\\
                    &=d(F_3)\\
                \end{split}
            \end{equation*}
            donde $\cf{F_3}{U}{\mathbb{R}}$ es la función tal que $((\rho,\phi,\theta))\mapsto z\circ F((\rho,\phi,\theta))=\rho\cos\phi$, para todo $(\rho,\phi,\theta)\in U$. Luego
            \begin{equation*}
                \begin{split}
                    d(F^*z)&=d(\rho\cos\phi)\\
                    &=\frac{\partial}{\partial\rho}(\rho\cos\phi)d\rho+\frac{\partial}{\partial\phi}(\rho\cos\phi)d\phi+\frac{\partial}{\partial\theta}(\rho\cos\phi)d\theta\\
                    &=\cos\phi d\rho-\rho\sin\phi d\phi\\
                    \Rightarrow d(F^*z)&=\cos\phi d\rho-\rho\sin\phi d\phi\\
                \end{split}
            \end{equation*}
        \end{enumerate}
        Por los tres incisos anteriores, se sigue que
        \begin{multline*}
            F^*(dx\wedge dy\wedge dz)=(\sin\phi\cos\theta d\rho+\rho\cos\phi\cos\theta d\phi-\rho\sin\phi\sin\theta d\theta)\\ \wedge(\sin\phi\sin\theta d\rho+\rho\cos\phi\sin\theta d\phi+\rho\sin\phi\cos\theta d\theta) \wedge(\cos\phi d\rho-\rho\sin\phi d\phi)
        \end{multline*}
        Computemos $(\sin\phi\sin\theta d\rho+\rho\cos\phi\sin\theta d\phi+\rho\sin\phi\cos\theta d\theta)\wedge(\cos\phi d\rho-\rho\sin\phi d\phi)$:
        \begin{multline*}
            (\sin\phi\sin\theta d\rho+\rho\cos\phi\sin\theta d\phi+\rho\sin\phi\cos\theta d\theta)\wedge(\cos\phi d\rho-\rho\sin\phi d\phi)\\
            =(\rho\cos^2\phi\sin\theta d\phi\wedge d\rho+\rho \sin\phi\cos\phi\cos\theta d\theta\wedge d\rho)+(-\rho \sin^2\sin\theta\phi d\rho\wedge d\phi-\rho^2\sin^2\phi\cos\theta d\theta\wedge d\phi)\\
            =(-\rho\cos^2\phi\sin\theta d\rho\wedge d\phi-\rho \sin\phi\cos\phi\cos\theta d\rho\wedge d\theta)+(-\rho \sin^2\phi\sin\theta d\rho\wedge d\phi+\rho^2\sin^2\phi\cos\theta d\phi\wedge d\theta)\\
            =-\rho\sin\theta d\rho\wedge d\phi-\rho \sin\phi\cos\phi\cos\theta d\rho\wedge d\theta+\rho^2\sin^2\phi\cos\theta d\phi\wedge d\theta
        \end{multline*}
        luego,
        \begin{multline*}
            \Rightarrow(\sin\phi\cos\theta d\rho+\rho\cos\phi\cos\theta d\phi-\rho\sin\phi\sin\theta d\theta)\\ \wedge(\sin\phi\sin\theta d\rho+\rho\cos\phi\sin\theta d\phi+\rho\sin\phi\cos\theta d\theta) \wedge(\cos\phi d\rho-\rho\sin\phi d\phi)\\
            =(\sin\phi\cos\theta d\rho+\rho\cos\phi\cos\theta d\phi-\rho\sin\phi\sin\theta d\theta)\\
            \wedge(-\rho\sin\theta d\rho\wedge d\phi-\rho \sin\phi\cos\phi\cos\theta d\rho\wedge d\theta+\rho^2\sin^2\phi\cos\theta d\phi\wedge d\theta)\\
            =\rho^2\sin\phi\sin^2\theta d\theta\wedge d\rho\wedge d\phi-\rho^2\sin\phi\cos^2\phi\cos^2\theta d\phi\wedge d\rho\wedge d\theta
            +\rho^2\sin^3\phi\cos^2\theta d\rho\wedge d\phi\wedge d\theta\\
            =\rho^2\sin\phi\sin^2\theta d\rho\wedge d\phi\wedge d\theta
            +\rho^2\sin\phi\cos^2\phi\cos^2\theta d\rho\wedge d\phi\wedge d\theta
            +\rho^2\sin^3\phi\cos^2\theta d\rho\wedge d\phi\wedge d\theta\\
            =\rho^2\sin\phi\sin^2\theta d\rho\wedge d\phi\wedge d\theta
            +\rho^2\sin\phi\cos^2\theta\left[\cos^2\phi+\sin^2\phi\right]d\rho\wedge d\phi\wedge d\theta\\
            =\rho^2\sin\phi\sin^2\theta d\rho\wedge d\phi\wedge d\theta
            +\rho^2\sin\phi\cos^2\theta d\rho\wedge d\phi\wedge d\theta\\
            =\rho^2\sin\phi\left[\sin^2\theta+\cos^2\theta\right] d\rho\wedge d\phi\wedge d\theta\\
            =\rho^2\sin\phi \:d\rho\wedge d\phi\wedge d\theta\\
        \end{multline*}
        por tanto
        \begin{equation*}
            F^*(dx\wedge dy\wedge dz)=\rho^2\sin\phi\:d\rho\wedge d\phi\wedge d\theta
        \end{equation*}
    \end{proof}

    \newpage

    \section{Ejercicio 2}

    \begin{excer}[\textbf{Pullback de una forma diferencial}]
        Sea $\cf{F}{\mathbb{R}^2}{\mathbb{R}^2}$ la función dada por
        \begin{equation*}
            F(x,y)=(u,v)=(x^2+y^2,xy)
        \end{equation*}
        Compute $F^*(u\:du+v\:dv)$.
    \end{excer}

    \begin{proof}
        Como el pullback es lineal, se tiene entonces que
        \begin{equation*}
            F^*(u\:du+v\:dv)=F^*(u\:du)+F^*(v\:dv)
        \end{equation*}
        Determinemos $F^*(u\:du)$ y $F^*(v\:dv)$.
        \begin{enumerate}
            \item Veamos que
            \begin{equation*}
                \begin{split}
                    F^*(u\:du)&=F^*(u\wedge du)\\
                    &=(F^*u)\wedge(F^*du)\\
                    &=(F^*u)\wedge d(F^*u)\\
                \end{split}
            \end{equation*}
            pues, podemos ver a la $1$-forma $u\:du$ como el producto wedge de una $0$-forma con una $1$-forma. De esta forma, por propiedades del pullback, se sigue la primera y segunda igualdad. Para la tercera, se cumple ya que el producto wedge conmuta con la diferencial exterior.
            
            Computemos $F^*u$:

            \begin{equation*}
                \begin{split}
                    F^*u&=u\circ F
                \end{split}
            \end{equation*}
            es decir, $F^*u$ es la función de $\mathbb{R}^2$ en $\mathbb{R}$ tal que $(x,y)\mapsto u\circ F(x,y)=u(x^2+y^2,xy)=x^2+y^2$. De esta forma, se tiene que
            \begin{equation*}
                \begin{split}
                    d(F^*u)&=d(x^2+y^2)\\
                    &=\frac{\partial }{\partial x}(x^2+y^2)\:dx+\frac{\partial }{\partial y}(x^2+y^2)\:dy\\
                    &=2x\:dx+2y\:dy\\
                    \Rightarrow d(F^*u)&=2x\:dx+2y\:dy\\
                \end{split}
            \end{equation*}
            
            Por lo cual

            \begin{equation*}
                \begin{split}
                    F^*(u\:du)&=(F^*u)\wedge d(F^*u)\\
                    &=(x^2+y^2)\wedge(2x\:dx+2y\:dy)\\
                    &=(2x^3+2xy^2)\:dx+(2x^2y+2y^3)\:dy\
                \end{split}
            \end{equation*}
            
            \item Como en el inciso (I), se tiene que
            \begin{equation*}
                \begin{split}
                    F^*(v\: dv)&=F^*(v\wedge dv)\\
                    &=(F^*v)\wedge (F^*dv)\\
                    &=(F^*v)\wedge d(F^*v)\\
                \end{split}
            \end{equation*}
            siendo $F^*v=\cf{v\circ F}{\mathbb{R}^2}{\mathbb{R}}$ tal que $(x,y)\mapsto xy$. Por lo cual
            \begin{equation*}
                \begin{split}
                    d(F^*v)&=\frac{\partial}{\partial x}(xy)\:dx+\frac{\partial}{\partial y}(xy)\: dy\\
                    &=y\:dx+x\: dy\\
                \end{split}
            \end{equation*}
            entonces
            \begin{equation*}
                \begin{split}
                    \Rightarrow F^*(v\:dv)&=(F^*v)\wedge d(F^*v)\\
                    &=(xy)\wedge (y\:dx+x\: dy)\\
                    &=xy^2\:dx+x^2y\: dy)\\
                \end{split}
            \end{equation*}
        \end{enumerate}
        Por el inciso (I) y (II), se sigue que
        \begin{equation*}
            \begin{split}
                F^*(u\:du+v\:dv)=&F^*(u\:du)+F^*(v\:dv)\\
                &=\left[(2x^3+2xy^2)\:dx+(2x^2y+2y^3)\:dy\right]+\left[xy^2\:dx+x^2y\: dy\right]\\
                &=(2x^3+3xy^2)\:dx+(2y^3+3x^2y)\:dy\\
                &=(2x^2+3y^2)x\:dx+(2y^2+3x^2)y\:dy\\
            \end{split}
        \end{equation*}
        por lo tanto
        \begin{equation*}
            \therefore F^*(u\:du+v\:dv)=(2x^2+3y^2)x\:dx+(2y^2+3x^2)y\:dy
        \end{equation*}
    \end{proof}

    \newpage

    \section{Ejercicio 3}

    \begin{excer}[\textbf{Pullback de una forma diferencial por una curva}]
        Sea $\omega$ la $1$-forma dada por
        \begin{equation*}
            \omega = \frac{-ydx+xdy}{x^2+y^2}
        \end{equation*}
        en $\mathbb{R}^2\backslash\left\{0\right\}$. Defina $\cf{c}{\mathbb{R}}{\mathbb{R}^2}$, $t\mapsto (\cos t,\sin t)$. Compute $c^*\omega$.
    \end{excer}

    \begin{proof}
        Observemos que
        \begin{equation*}
            \begin{split}
                \omega&=\frac{-ydx+xdy}{x^2+y^2}\\
                &=-\frac{y}{x^2+y^2}\:dx+\frac{x}{x^2+y^2}\:dy\\
                &=-\frac{y}{x^2+y^2}\wedge\:dx+\frac{x}{x^2+y^2}\wedge\:dy\\
            \end{split}
        \end{equation*}
        (viendo a la $1$-forma $\omega$ como el producto wedge de una $0$-forma con una $1$-forma). Por tanto:
        \begin{equation*}
            \begin{split}
                c^*\omega=&c^*\left(-\frac{y}{x^2+y^2}\wedge\:dx+\frac{x}{x^2+y^2}\wedge\:dy\right) \\
                =&c^*\left(-\frac{y}{x^2+y^2}\wedge\:dx\right)+c^*\left(\frac{x}{x^2+y^2}\wedge\:dy\right) \\
                =&c^*\left(-\frac{y}{x^2+y^2}\wedge\:dx\right)+c^*\left(\frac{x}{x^2+y^2}\wedge\:dy\right) \\
                =&c^*\left(-\frac{y}{x^2+y^2}\right)\wedge c^*\left(dx\right)+c^*\left(\frac{x}{x^2+y^2}\right)\wedge c^*\left(dy\right) \\
                =&c^*\left(-\frac{y}{x^2+y^2}\right)\wedge d\left(c^*x\right)+c^*\left(\frac{x}{x^2+y^2}\right)\wedge d\left(c^*y\right) \\
            \end{split}
        \end{equation*}
        donde
        \begin{enumerate}
            \item Para el primer elemento se tiene que
            \begin{equation*}
                \begin{split}
                    c^*\left(-\frac{y}{x^2+y^2}\right)&=\left(-\frac{y}{x^2+y^2}\right)\circ c\\
                \end{split}
            \end{equation*}
            por lo cual,
            \begin{equation*}
                \begin{split}
                    c^*\left(-\frac{y}{x^2+y^2}\right)(t)&=\left(-\frac{y}{x^2+y^2}\right)\circ c(t)\\
                    &=\left(-\frac{y}{x^2+y^2}\right)(\cos t, \sin t)\\
                    &=-\frac{\sin t}{\cos^2t+\sin^2t}\\
                    &=-\sin t\\
                \end{split}
            \end{equation*}
            \item Para el segundo se tiene
            \begin{equation*}
                \begin{split}
                    d\left(c^*x\right)&=d\left(x\circ c\right)\\
                    &=d\left(x(\cos t,\sin t)\right)\\
                    &=d\left(\cos t\right)\\
                    &=-\sin t\:dt\\
                \end{split}
            \end{equation*}
            \item Para el tercero se tiene
            \begin{equation*}
                \begin{split}
                    c^*\left(\frac{x}{x^2+y^2}\right)&=\left(\frac{x}{x^2+y^2}\right)\circ c \\
                \end{split}
            \end{equation*}
            por lo cual
            \begin{equation*}
                \begin{split}
                    \left(\frac{x}{x^2+y^2}\right)\circ c(t)&=\left(\frac{x}{x^2+y^2}\right)(\cos t,\sin t)\\
                    &=\frac{\cos t}{\cos^2t+\sin^2t}\\
                    &=\cos t\\
                \end{split}
            \end{equation*}
            \item Para el cuarto se tiene
            \begin{equation*}
                \begin{split}
                    d\left(c^*y\right)&=d\left(y\circ c\right)\\
                    &=d\left(y(\cos t, \sin t)\right)\\
                    &=d\left(\sin t\right)\\
                    &=\cos t\:dt\\
                \end{split}
            \end{equation*}
        \end{enumerate}
        Por los incisos (I)-(IV), se sigue que
        \begin{equation*}
            \begin{split}
                c^*\omega=&c^*\left(-\frac{y}{x^2+y^2}\right)\wedge d\left(c^*x\right)+c^*\left(\frac{x}{x^2+y^2}\right)\wedge d\left(c^*y\right) \\
                &=(-\sin t)(-\sin t\:dt)+(\cos t)(\cos t\:dt)\\
                &=\sin^2t\:dt+\cos^2t\:dt\\
                &=dt\\
                \Rightarrow c^*\omega=&dt\\
            \end{split}
        \end{equation*}
    \end{proof}

    \newpage

    \renewcommand{\theenumi}{\alph{enumi}}
    \renewcommand{\labelenumi}{{(\theenumi)}}

    \section{Ejercicio 4}

    \begin{excer}[\textbf{Una forma que no se desvanece sobre una hípersuperficie suave}]
        Resuelva los siguientes incisos.
        \begin{enumerate}
            \item Sea $f(x,y)$ una función $C^\infty$ en $\mathbb{R}^2$ y asuma que $0$ es valor regular de $f$. Por el teorema del valor regular, el conjunto cero $M$ de $f(x,y)$ es una subvariedad $1$-dimensional de $\mathbb{R}^2$. Construya una $1$-forma que no se anula en $M$.
            \item Sea $f(x,y,z)$ una función $C^\infty$ en $\mathbb{R}^3$ y asuma que $0$ es valor regular de $f$. Por el teorema del valor regular, el conjunto cero $M$ de $f(x,y,z)$ es una subvariedad $2$-dimensional de $\mathbb{R}^3$. Sean $f_x$, $f_y$ y $f_z$ las derivadas parciales de $f$ con respecto a $x$, $y$ y $z$, respectivamente. Muestre que las igualdades
            \begin{equation*}
                \frac{dx\wedge dy}{f_z}=\frac{dy\wedge dz}{f_x}=\frac{dz\wedge dx}{f_y}
            \end{equation*}
            son válidas en $M$ siempre y cuando estas tengan sentido, y por tanto juntas dan una $2$-forma sobre $M$ que no se desvanece en ninguna parte.
            \item Generalice este problema al conjunto de nivel regular de $f(x^1,\dots,x^{n+1})$ en $\mathbb{R}^{n+1}$.
        \end{enumerate}
    \end{excer}

    \begin{proof}
        Recordemos una proposición y el teorema del valor regular

        \setcounter{section}{8}
        \setcounter{propo}{22}

        \begin{propo}
            Para una función real valuada $\cf{f}{M}{\mathbb{R}}$, donde $M$ es una variedad, se tiene que $p\in M$ es un punto crítico de $M$, si y sólo si relativo a alguna carta $(U,x^1,\dots,x^n)$ que contiene a $p$, todas la derivadas parciales de $f$ satisfacen
            \begin{equation*}
                \frac{\partial f}{\partial x^j}(p)=0,\quad\forall j=1,\dots,n.
            \end{equation*}
        \end{propo}

        \setcounter{section}{9}
        \setcounter{theor}{8}

        \begin{theor}[\textbf{Teorema del valor regular}]
            Sea $\cf{F}{N}{M}$ un mapeo $C^\infty$ entre variedades, donde $\dim N = n$ y $\dim M = m$. Entonces el conjunto de nivel no vacío $F^{-1}(c)$, donde $c\in M$, es una subvariedad regular de $N$ con dimensión $n-m$.
        \end{theor}

        De (I): En este caso, $\cf{f}{\mathbb{R}^2}{\mathbb{R}}$, donde $\mathbb{R}^2$ y $\mathbb{R}$ son variedades suaves, y
        \begin{equation*}
            \begin{split}
                M&=f^{-1}(0)\\
                &=\left\{(x,y)\in\mathbb{R}^2|f(x,y)=0\right\}
            \end{split}
        \end{equation*}
        siendo $M$ el conjunto cero de $f(x,y)$. Observemos que
        \begin{equation*}
            f(x,y)=0,\quad\forall (x,y)\in M
        \end{equation*}
        Por lo cual, tomando diferencial exterior de ambos lados, se sigue que
        \begin{equation*}
            \begin{split}
                df(x,y)&=0\\
                \Rightarrow f_x(x,y)\:dx+f_y(x,y)\:dy&=0,\quad\forall (x,y)\in M\\
            \end{split}
        \end{equation*}
        donde $f_x=\frac{\partial f}{\partial x}$ y $f_y=\frac{\partial f}{\partial y}$. Considere los conjuntos
        \begin{equation*}
            U_x=\left\{(x,y)\in M|f_x(x,y)\neq 0\right\}\textup{ y }U_y=\left\{(x,y)\in M|f_y(x,y)\neq 0\right\}
        \end{equation*}
        se tiene que en $U_x\cap U_y$:

        \begin{equation}
            \begin{split}
                f_x\:dx+f_y\:dy&=0\\
                \Rightarrow f_x\:dx&=-f_y\:dy\\
            \end{split}
        \end{equation}

        Defina así la $1$-forma $\omega$ como sigue

        \begin{equation}
            \omega(x,y)=\left\{
                \begin{aligned}
                    f_x(x,y)\:dx & \text{ si }(x,y)\in U_x\\
                    -f_y(x,y)\:dy & \text{ si }(x,y)\in U_y\\
                \end{aligned} 
            \right.
        \end{equation}
        para todo $(x,y)\in M$. Por (1.2), está $1$-forma está bien definida en $M$, ya que las $1$-formas $-f_y\:dy$ y $f_x\:dx$ coinciden en $U_x\cap U_y$. Veamos que $\omega$ es $C^\infty$ y no se anula en $M$. Para ello, consideremos las cartas de $\mathbb{R}^2$. Sean
        \begin{equation}
            \begin{split}
                U_x^+=\left\{(x,y),\in M|f_x(x,y)>0\right\}\\
                U_x^-=\left\{(x,y),\in M|f_x(x,y)<0\right\}\\
                U_y^+= \left\{(x,y),\in M|f_y(x,y)>0\right\}\\
                U_y^-=\left\{(x,y),\in M|f_y(x,y)<0\right\}\\
        \end{split}
        \end{equation}
        Es claro que $U_x=U_x^+\cup U_x^-$ y $U_y=U_y^+\cup U_y^-$. Notemos que, como $0$ es un valor regular de $f$, entonces todo punto de $f^{-1}(0)=M$ es un punto regular. Por tanto, de la Proposición (8.23) se tiene que para todo punto $p\in M$, alguna de las derivadas parciales
        \begin{equation*}
            \frac{\partial f}{\partial x}(p)=f_x(p),\quad\frac{\partial f}{\partial y}(p)=f_y(p)
        \end{equation*}
        no se anula. Es decir, que $p$ se encuentra en alguno de los conjuntos de (1.4). Luego, $M$ está totalmente contenida en la unión de estos conjuntos. Por tanto, $\omega$ no se anula en $M$.
        
        Y, claramente $\omega$ es $C^\infty$, pues las funciones $f_x$ y $f_y$ lo son. Así, la $1$-forma buscada es $\omega$.

        De (II): En este caso, $\cf{f}{\mathbb{R}^3}{\mathbb{R}}$, donde $\mathbb{R}^3$ y $\mathbb{R}$ son variedades suaves, y
        \begin{equation*}
            \begin{split}
                M&=f^{-1}(0)\\
                &=\left\{(x,y,z)\in\mathbb{R}^3|f(x,y,z)=0\right\}
            \end{split}
        \end{equation*}
        siendo $M$ el conjunto cero de $f(x,y,z)$. Observemos que

        \begin{equation}
            \begin{split}
                df(x,y,z)&=0\\
                \Rightarrow f_x(x,y,z)\:dx+f_y(x,y,z)\:dy+f_z(x,y,z)\:dz&=0,\quad\forall (x,y,z)\in M\\
            \end{split}
        \end{equation}
        por tanto, en $M$ se tienen las siguientes igualdades:
        \begin{equation*}
            \begin{split}
                f_y\:dy\wedge dx+f_z\:dz\wedge dx=0\\
                f_x\:dx\wedge dy+f_z\:dz\wedge dy=0\\
                f_x\:dx\wedge dz+f_y\:dy\wedge dz=0\\
            \end{split}
        \end{equation*}
        (haciendo el producto wedge de (1.5) con $dx$, $dy$ y $dz$). Las igualdades anteriores son equivalentes a
        \begin{equation*}
            \begin{split}
                -f_y\:dx\wedge dy+f_z\:dz\wedge dx=0\Rightarrow f_y\:dx\wedge dy=f_z\:dz\wedge dx \\
                f_x\:dx\wedge dy-f_z\:dy\wedge dz=0\Rightarrow f_x\:dx\wedge dy=f_z\:dy\wedge dz\\
                -f_x\:dz\wedge dx+f_y\:dy\wedge dz=0\Rightarrow f_x\:dz\wedge dx=f_y\:dy\wedge dz \\
            \end{split}
        \end{equation*}
        por tanto
        \begin{equation*}
            \frac{dx\wedge dy}{f_z}=\frac{dz\wedge dx}{f_y}\text{ y }\frac{dx\wedge dy}{f_z}=\frac{dy\wedge dz}{f_x}
        \end{equation*}
        (siempre que $f_x(x,y,z),f_y(x,y,z),f_z(x,y,z)\neq 0$, con $(x,y,z)\in M$). Luego
        \begin{equation}
            \frac{dx\wedge dy}{f_z}=\frac{dy\wedge dz}{f_x}=\frac{dz\wedge dx}{f_y}
        \end{equation}
        
        Sean
        \begin{equation}
            \begin{split}
                U_x=\left\{(x,y,z)\in M|f_x(x,y,z)\neq0\right\}\\
                U_y=\left\{(x,y,z)\in M|f_y(x,y,z)\neq0\right\}\\
                U_z=\left\{(x,y,z)\in M|f_z(x,y,z)\neq0\right\}\\
            \end{split}
        \end{equation}
        como en el inciso (I), es claro que $M$ está contenida en la unión de estos conjuntos (ya que al ser 0 valor regular de $f$, todos los puntos de $M$ son regulares y por ende, alguna de las derivadas parciales no se anula en $p\in M$). Definamos
        
        \begin{equation}
            \omega(x,y,z)=\left\{
                \begin{aligned}
                    \frac{dy\wedge dz}{f_x(x,y,z)} & \text{ si }(x,y,z)\in U_x\\
                    \frac{dz\wedge dx}{f_y(x,y,z)} & \text{ si }(x,y,z)\in U_y\\
                    \frac{dz\wedge dx}{f_y(x,y,z)} & \text{ si }(x,y,z)\in U_z\\
                \end{aligned}
            \right.
        \end{equation}

        Por (1.7), es claro que $\omega$ es una $2$-forma está bien definida en $M$, pues en la intersección de 2 conjuntos de (1.6), se cumplen las igualdades en (1.7). Y además es $C^\infty$, pues cada una de las funciones $f_x$, $f_y$ y $f_z$ lo es.

        De (III): Sea $f$ una función $C^\infty$ en $\mathbb{R}^{n+1}$ y suponga que $c\in\mathbb{R}$ es un valor regular de $f$. Entonces por el teorema del valor regular, el conjunto $M=f^{-1}(c)$ es una subvariedad $n$-dimensional de $\mathbb{R}^{n+1}$. Construya una $n$-forma que no se anula en $M$.

        Para ello, notemos que si $(x^1,\dots,x^{n+1})\in M$, donde
        \begin{equation*}
            \begin{split}
                M=&f^{-1}(c)\\
                =&\left\{(x^1,\dots,x^{n+1})\in \mathbb{R}^{n+1}|f(x^1,\dots,x^{n+1})=c\right\} \\
            \end{split}
        \end{equation*}
        entonces
        \begin{equation*}
            \begin{split}
                f(x^1,\dots,x^{n+1})&=c\\
                \Rightarrow df(x^1,\dots,x^{n+1})&=0\\
                \Rightarrow f_{x^1}(x^1,\dots,x^{n+1})dx_1+\cdots+f_{x^{n+1}}(x^1,\dots,x^{n+1})dx_{n+1}&=0\\
            \end{split}
        \end{equation*}
        Si $I=(i_1,i_2,\dots,i_{n-1})$ es un multi-índice creciente de $n-1$ de longitud $n-1$, se tiene que haciendo el producto wedge de la igualdad de arriba con $dx_{i_1}\wedge\cdots\wedge dx_{i_{n-1}}$:

        \begin{equation*}
            f_{x^{j_1}}(x^1,\dots,x^{n+1})dx_j\wedge dx_{i_1}\wedge\cdots\wedge dx_{i_{n-1}}+f_{x^{j_2}}(x^1,\dots,x^{n+1})dx_2\wedge dx_{i_1}\wedge\cdots\wedge dx_{i_{n-1}}=0
        \end{equation*}
        donde $j_1$ y $j_2$ son tales que no están en $I$. Por tanto

        \begin{equation}
            f_{x^{j_1}}(x^1,\dots,x^{n+1})dx_j\wedge dx_{i_1}\wedge\cdots\wedge dx_{i_{n-1}}=-f_{x^{j_2}}(x^1,\dots,x^{n+1})dx_2\wedge dx_{i_1}\wedge\cdots\wedge dx_{i_{n-1}}
        \end{equation}

        Con esto, de forma similar al inciso (II), se obtienen igualdades
        que relacionan a $f_{x^i}$ con $f_{x^j}$, siendo $i,j\in\left\{1,\dots,n+1 \right\}$ con $i<j$, dadas por
        \begin{equation}
            \begin{split}
                f_{x^i}(x^1,\dots,x^{n+1})\:dx_1\wedge\cdots\wedge\widehat{dx_j}\wedge\cdots\wedge dx_{n+1}&=f_{x^j}(x^1,\dots,x^{n+1})\:dx_1\wedge\cdots\wedge\widehat{dx_i}\wedge\cdots\wedge dx_{n+1}\\
                \Rightarrow \frac{dx_1\wedge\cdots\wedge\widehat{dx_j}\wedge\cdots\wedge dx_{n+1}}{f_{x^j}(x^1,\dots,x^{n+1})}&=\frac{dx_1\wedge\cdots\wedge\widehat{dx_i}\wedge\cdots\wedge dx_{n+1}}{f_{x^i}(x^1,\dots,x^{n+1})}\\
            \end{split}
        \end{equation}
        (siempre que las derivadas parciales no se anulen). Definimos así a la $n$-forma $\omega$, como
        \begin{equation}
            \omega(x^1,\dots,x^{n+1})=\frac{dx_1\wedge\cdots\wedge\widehat{dx_i}\wedge\cdots\wedge dx_{n+1}}{f_{x^i}(x^1,\dots,x^{n+1})}
        \end{equation}
        si $(x^1,\dots,x^{n+1})\in U_{x^i}$, donde
        \begin{equation*}
            U_{x^i}=\left\{(x^1,\dots,x^{n+1})\in M|f_{x_1}(x^1,\dots,x^{n+1})\neq0 \right\},\quad \forall i=1,\dots,n+1
        \end{equation*}
        Por (1.10) esta $n$-forma está bien definida, y por la proposición (8.23), como $c$ es valor regular de $f$, entonces $M\subseteq \bigcup_{i=1}^{n+1}U_{x^i}$ (si $p\in M$, entonces alguna de las derivadas parciales de $f$ no se anula en $p$). De esta forma, como las funciones $f_{x^i}$ son $C^\infty$, se sigue que $\omega$ es una $n$-forma $C^\infty$ que no se anula en $M$.

    \end{proof}

\end{document}