\documentclass[12pt]{report}
\usepackage[spanish]{babel}
\usepackage[utf8]{inputenc}
\usepackage{amsmath}
\usepackage{amssymb}
\usepackage{amsthm}
\usepackage{graphics}
\usepackage{subfigure}
\usepackage{lipsum}
\usepackage{array}
\usepackage{multicol}
\usepackage{enumerate}
\usepackage[framemethod=TikZ]{mdframed}
\usepackage[a4paper, margin = 1.5cm]{geometry}

%En esta parte se hacen redefiniciones de algunos comandos para que resulte agradable el verlos%

\def\proof{\paragraph{Demostración:\\}}
\def\endproof{\hfill$\blacksquare$}

\def\sol{\paragraph{Solución:\\}}
\def\endsol{\hfill$\square$}

%En esta parte se definen los comandos a usar dentro del documento para enlistar%

\newtheoremstyle{largebreak}
  {}% use the default space above
  {}% use the default space below
  {\normalfont}% body font
  {}% indent (0pt)
  {\bfseries}% header font
  {}% punctuation
  {\newline}% break after header
  {}% header spec

\theoremstyle{largebreak}

\newmdtheoremenv[
    leftmargin=0em,
    rightmargin=0em,
    innertopmargin=0pt,
    innerbottommargin=5pt,
    hidealllines = true,
    roundcorner = 5pt,
    backgroundcolor = gray!60!red!30
]{exa}{Ejemplo}[section]

\newmdtheoremenv[
    leftmargin=0em,
    rightmargin=0em,
    innertopmargin=0pt,
    innerbottommargin=5pt,
    hidealllines = true,
    roundcorner = 5pt,
    backgroundcolor = gray!50!blue!30
]{obs}{Observación}[section]

\newmdtheoremenv[
    leftmargin=0em,
    rightmargin=0em,
    innertopmargin=0pt,
    innerbottommargin=5pt,
    rightline = false,
    leftline = false
]{theor}{Teorema}[section]

\newmdtheoremenv[
    leftmargin=0em,
    rightmargin=0em,
    innertopmargin=0pt,
    innerbottommargin=5pt,
    rightline = false,
    leftline = false
]{propo}{Proposición}[section]

\newmdtheoremenv[
    leftmargin=0em,
    rightmargin=0em,
    innertopmargin=0pt,
    innerbottommargin=5pt,
    rightline = false,
    leftline = false
]{cor}{Corolario}[section]

\newmdtheoremenv[
    leftmargin=0em,
    rightmargin=0em,
    innertopmargin=0pt,
    innerbottommargin=5pt,
    rightline = false,
    leftline = false
]{lema}{Lema}[section]

\newmdtheoremenv[
    leftmargin=0em,
    rightmargin=0em,
    innertopmargin=0pt,
    innerbottommargin=5pt,
    roundcorner=5pt,
    backgroundcolor = gray!30,
    hidealllines = true
]{mydef}{Definición}[section]

\newmdtheoremenv[
    leftmargin=0em,
    rightmargin=0em,
    innertopmargin=0pt,
    innerbottommargin=5pt,
    roundcorner=5pt
]{excer}{Ejercicio}[section]

%En esta parte se colocan comandos que definen la forma en la que se van a escribir ciertas funciones%

\newcommand\abs[1]{\ensuremath{\left|#1\right|}}
\newcommand\divides{\ensuremath{\bigm|}}
\newcommand\cf[3]{\ensuremath{#1:#2\rightarrow#3}}
\newcommand\contradiction{\ensuremath{\#_c}}
\newcommand\natint[1]{\ensuremath{\left[\big|#1\big|\right]}}
\newcommand{\ord}[1]{\ensuremath{\textup{ord}\left(#1\right)}}

\begin{document}
    \setlength{\parskip}{5pt} % Añade 5 puntos de espacio entre párrafos
    \setlength{\parindent}{12pt} % Pone la sangría como me gusta
    \title{Notas de Álgebra Moderna III:
    
    Una Introducción a la Teoría de Galois Finita}
    \author{Cristo Daniel Alvarado}
    \maketitle

    \tableofcontents %Con este comando se genera el índice general del libro%

    %\setcounter{chapter}{3} %En esta parte lo que se hace es cambiar la enumeración del capítulo%

    \newpage

    \chapter{Anillo de Polinomios}

    \section{Series de Potencias}

    \begin{mydef}
        Sea $A$ un anillo. Denotemos por
        \begin{equation*}
            S_A=\left\{f\Big|\cf{f}{\mathbb{N}\cup\left\{0\right\}}{A} \right\}
        \end{equation*}
        es decir que $S_A$ es el \textbf{conjunto de sucesiones de $A$}. Si $f\in S_A$ escribimos a $f$ como:
        \begin{equation*}
            f=(a_0,a_1,...)
        \end{equation*}
        Sobre $S_A$ se definen dos operaciones, la \textbf{suma} y \textbf{producto}. A saber, si $f=(a_0,a_1,...)$ y $g=(b_0,b_1,...)$, entonces:
        \begin{equation*}
            f+g=(a_0+b_0,a_1+b_1,...,a_k+b_k,...)
        \end{equation*}
        y,
        \begin{equation*}
            fg = f\cdot g = (c_0,c_1,...,c_k,...)
        \end{equation*}
        donde
        \begin{equation*}
            \begin{split}
                c_k&=\sum_{ i=0}^k a_ib_{ k-i}\\
                &=a_0b_k+a_1b_{ k-1}+...+a_kb_0\\
                &=\sum_{ i=0}^k a_{k-i}b_{i}\\
                &=\sum_{ i+j=k}a_ib_j\\
            \end{split}
        \end{equation*}
    \end{mydef}

    \begin{obs}
        En la definición anterior, se tiene que $S_A$ es un anillo con cero el elemento $(0,0,...,0,...)$ e inverso $-f=(-a_0,-a_1,...,-a_k,...)$ para todo $f\in S_A$. Además, existe un monomorfismo de $A$ en $S_A$, a saber:
        \begin{equation*}
            A\hookrightarrow S_A, a\mapsto (a,0,...,0,...)
        \end{equation*}
        Por lo cual $A$ está encajado en $S_A$. Debido a esto, se denotará de ahora en adelante como
        \begin{equation*}
            a=(a,0,...,0,...),\quad\forall a\in A
        \end{equation*}
    \end{obs}

    \begin{mydef}
        Sean $A$ y $X$ un objeto tal que $X\notin A$. $X$ es llamado una \textbf{indeterminada para $A$}. Definimos para todo $n\in\mathbb{N}\cup\left\{0\right\}$ y para todo $a\in A$:
        \begin{equation*}
            aX^n=(\underbrace{0,0,0,...,0,0,0,a}_{{n+1}-\textbf{ésima entrada}},0,...)
        \end{equation*}
        Si $A$ tiene identidad, entonces
        \begin{equation*}
            1X^n=X^n=(\underbrace{0,0,0,...,0,0,0,1}_{{n+1}-\textbf{ésima entrada}},0,...)
        \end{equation*}
        En caso que $n=1$, $1X^1=X^1=x$ y si $n=0$, $1X^0=X^0=1$ (abusando en este caso de la notación). Se tiene entonces que
        \begin{equation*}
            X^n\in S_A,\quad\forall n\in\mathbb{N}\cup\left\{0\right\}
        \end{equation*}
        Ahora, independientemente de si $A$ tiene o no identidad, tenemos que:
        \begin{equation*}
            \begin{split}
                (a_0,a_1,a_2,...,a_k,...)&=(a_0,0,0,...)\\
                &+(0,a_1,0,...)\\
                &+(0,0,a_2,...)\\
                &+(0,0,...,a_k,...)\\
                &+...\\
                &=a_0X^0+a_1X+a_2X^2+-...\\
            \end{split}
        \end{equation*}
        Por lo tanto, si $f=(a_0,a_1,a_2,...,a_k,...)$, entonces:
        \begin{equation*}
            \begin{split}
                f&=\sum_{i=0}^\infty a_iX^i\\
                &=a_0+a_1X+a_2X^2+...+a_kX^k+...\\
            \end{split}
        \end{equation*}
        así pues, podemos decir que una serie es una expresión algebraica de la forma:
        \begin{equation}
            f=\sum_{i=0}^\infty a_iX^i=a_0+a_1X+a_2X^2+...+a_kX^k+...
        \end{equation}
        donde $a_0:=a_0X^0$.

        Si $f=\sum_{ n=0}^\infty a_nX^n$ y $g=\sum_{ n=0}^\infty b_nX^n$, entonces:
        \begin{equation*}
            \begin{split}
                f+g&=\sum_{n=0}^\infty(a_n+b_n)X^n\\
                f\cdot g&=\sum_{ n=0}^\infty c_nX^n\\
            \end{split}
        \end{equation*}
        siendo $c_n=\sum_{ k=0}^n a_kb_{ n-k}$ para todo $n\in\mathbb{Z}_{\geq0}$. Con estas operaciones $S_A$ es un anillo, cambiándose el símbolo por $A[[X]]$, en este caso $A[[X]]$ es llamado \textbf{anillo de series de potencias con coeficientes en $A$ en la indeterminada $X$}.

        Si $f=\sum_{ n=0}^\infty a_nX^n\in A[[X]]$, los elementos $a_0,a_1,...\in A$ son llamados \textbf{coeficientes o términos de la serie $f$}. En particular, $a_0$ es llamado \textbf{coeficiente constante} o \textbf{coeficiente lider de la serie $f$}. El cero de $A[[X]]$ es la serie \textbf{serie cero} dada por:
        \begin{equation*}
            0=\sum_{ n=0}^\infty a_nX^n,\quad a_n=0\quad\forall n\in\mathbb{Z}_{ \geq0}
        \end{equation*}
    \end{mydef}

    \begin{obs}
        Una serie de potencias $f\in A[[X]]$ cumple que $f=0$ si y sólo si $a_n=0$ para todo $a_n=0,\forall n\in\mathbb{Z}_{\geq0}$.

        En el caso de que $f$ tenga coeficientes cero, éstos no se expresan dentro de $f$, es decir que no se expresan dentro de la sumatoria de tipo $f=a_0+a_1X+...$.
    \end{obs}

    \begin{obs}
        En el caso de que el anillo $A$ tenga identidad, $A[[X]]$ también lo tiene y es la identidad de $A$, a saber:
        \begin{equation*}
            1=1+0X+0X^2+...
        \end{equation*}
        En tal caso, $1X^n=X^n$ para todo $n\in\mathbb{N}$.
    \end{obs}

    \begin{obs}
        Si $n,m\in\mathbb{Z}_{\geq0}$, entonces $X^n,X^m\in A[[X]]$ y,
        \begin{equation*}
            X^n\cdot X^m=X^{ n+m}\in A[[X]]
        \end{equation*}
    \end{obs}

    \begin{obs}
        Si $A$ es un anillo conmutativo, entonces $A[[X]]$ también lo es (se verifica de forma inmediata de la definición de producto en $A[[X]]$).
    \end{obs}

    \begin{mydef}
        Sea $A$ un anillo. Para cada serie $f=\sum_{ n=0}^\infty a_nX^n\neq0$ en $A[[X]]$ se define el \textbf{orden de $f$}, denotado por $\ord{f}$ como el mínimo entero no negativo $m$ tal que $a_{m}\neq0$. Así, si $f$ es una serie de potencias no cero, entonces:
        \begin{equation*}
            f=\sum_{ n=\ord{f}}^\infty a_nX^n
        \end{equation*}
    \end{mydef}

    \begin{excer}
        Sea $A$ un anillo con $1$. Si $f=\sum_{ n=0}^\infty a_nX^n\in A[[X]]$, entonces existe $g=\sum_{ n=0}^\infty b_nX^n$ tal que $\ord{g}=0$ y,
        \begin{equation*}
            f=x^{\ord{f}}g
        \end{equation*}
    \end{excer}

    \begin{proof}
        Ejercicio.
    \end{proof}

    \begin{propo}
        Sea $A$ anillo y $f=\sum_{ n=0}^\infty a_nX^n$ y $\sum_{ n=0}^\infty b_nX^n$ dos series de potencias no cero en $A[[X]]$. Entonces:
        \begin{enumerate}
            \item $f+g=0$ ó $\ord{f+g}\geq\min\left\{\ord{f},\ord{g}\right\}$.
            \item $fg=0$ ó $\ord{fg}\geq\ord{f}+\ord{g}$.
        \end{enumerate}
    \end{propo}

    \begin{proof}
        Ejercicio.
    \end{proof}

    \begin{propo}
        Sean $A$ anillo conmutativo con $1$ y $f=\sum_{ n=0}^\infty a_nX^n\in A[[X]]$ una serie de potencias no cero. Entonces $f$ es unidad de $A[[X]]$ (esto es, elemento invertible de $A[[X]]$) si y sólo si $a_0$ (el término constante de $f$) es unidad de $A$.
    \end{propo}

    \begin{proof}
        Pendiente.
    \end{proof}

    \begin{cor}
        Si $K$ es campo, entonces $f=\sum_{ n=0}^\infty a_nX^n$ en $K[[X]]\backslash\left\{0\right\}$ es unidad si y sólo si $a_0\neq0$.
    \end{cor}

    \begin{proof}
        Es inmediata de la proposición anterior y del hecho que las unidades de $K$ son $K\backslash\left\{0\right\}$.
    \end{proof}

    \begin{cor}
        Si $K$ es campo y $f=\sum_{ n=0}^\infty a_nX^n\in K[[X]]\backslash\left\{0\right\}$, entonce existe una única serie de potencias $g\in K[[X]]$ invertible tal que $f=x^{\ord{f}}g$.
    \end{cor}

    \begin{exa}
        Considere en $\mathbb{R}[[X]]$ la serie de potencias $f=1+X+X^2+...=\sum_{ n=0}^\infty X^n$. Se tiene que $f$ es invertible y su inversa es $1-x\in\mathbb{R}[[X]]$.
    \end{exa}

    \begin{proof}
        Ejercicio.
    \end{proof}

\end{document}