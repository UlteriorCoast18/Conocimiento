\documentclass[12pt]{report}
\usepackage[spanish]{babel}
\usepackage[utf8]{inputenc}
\usepackage{amsmath}
\usepackage{amssymb}
\usepackage{amsthm}
\usepackage{graphics}
\usepackage{subfigure}
\usepackage{lipsum}
\usepackage{array}
\usepackage{multicol}
\usepackage{enumerate}
\usepackage[framemethod=TikZ]{mdframed}
\usepackage[a4paper, margin = 1.5cm]{geometry}

%En esta parte se hacen redefiniciones de algunos comandos para que resulte agradable el verlos%

\def\proof{\paragraph{Demostración:\\}}
\def\endproof{\hfill$\blacksquare$}

\def\sol{\paragraph{Solución:\\}}
\def\endsol{\hfill$\square$}

%En esta parte se definen los comandos a usar dentro del documento para enlistar%

\newtheoremstyle{largebreak}
  {}% use the default space above
  {}% use the default space below
  {\normalfont}% body font
  {}% indent (0pt)
  {\bfseries}% header font
  {}% punctuation
  {\newline}% break after header
  {}% header spec

\theoremstyle{largebreak}

\newmdtheoremenv[
    leftmargin=0em,
    rightmargin=0em,
    innertopmargin=0pt,
    innerbottommargin=5pt,
    hidealllines = true,
    roundcorner = 5pt,
    backgroundcolor = gray!60!red!30
]{exa}{Ejemplo}[section]

\newmdtheoremenv[
    leftmargin=0em,
    rightmargin=0em,
    innertopmargin=0pt,
    innerbottommargin=5pt,
    hidealllines = true,
    roundcorner = 5pt,
    backgroundcolor = gray!50!blue!30
]{obs}{Observación}[section]

\newmdtheoremenv[
    leftmargin=0em,
    rightmargin=0em,
    innertopmargin=0pt,
    innerbottommargin=5pt,
    rightline = false,
    leftline = false
]{theor}{Teorema}[section]

\newmdtheoremenv[
    leftmargin=0em,
    rightmargin=0em,
    innertopmargin=0pt,
    innerbottommargin=5pt,
    rightline = false,
    leftline = false
]{propo}{Proposición}[section]

\newmdtheoremenv[
    leftmargin=0em,
    rightmargin=0em,
    innertopmargin=0pt,
    innerbottommargin=5pt,
    rightline = false,
    leftline = false
]{cor}{Corolario}[section]

\newmdtheoremenv[
    leftmargin=0em,
    rightmargin=0em,
    innertopmargin=0pt,
    innerbottommargin=5pt,
    rightline = false,
    leftline = false
]{lema}{Lema}[section]

\newmdtheoremenv[
    leftmargin=0em,
    rightmargin=0em,
    innertopmargin=0pt,
    innerbottommargin=5pt,
    roundcorner=5pt,
    backgroundcolor = gray!30,
    hidealllines = true
]{mydef}{Definición}[section]

\newmdtheoremenv[
    leftmargin=0em,
    rightmargin=0em,
    innertopmargin=0pt,
    innerbottommargin=5pt,
    roundcorner=5pt
]{excer}{Ejercicio}[section]

%En esta parte se colocan comandos que definen la forma en la que se van a escribir ciertas funciones%

\newcommand\abs[1]{\ensuremath{\left|#1\right|}}
\newcommand\divides{\ensuremath{\bigm|}}
\newcommand\cf[3]{\ensuremath{#1:#2\rightarrow#3}}
\newcommand\contradiction{\ensuremath{\#_c}}
\newcommand\natint[1]{\ensuremath{\left[\big|#1\big|\right]}}

\begin{document}
    \setlength{\parskip}{5pt} % Añade 5 puntos de espacio entre párrafos
    \setlength{\parindent}{12pt} % Pone la sangría como me gusta
    \title{Notas de Álgebra Moderna III:
    
    Una Introducción a la Teoría de Galois Finita}
    \author{Cristo Daniel Alvarado}
    \maketitle

    \tableofcontents %Con este comando se genera el índice general del libro%

    %\setcounter{chapter}{3} %En esta parte lo que se hace es cambiar la enumeración del capítulo%

    \newpage

    \chapter{Anillo de Polinomios}

    \section{Series de Potencias}

    \begin{obs}
        De ahora en adelante todos los anillos se considerarán como anillos connmutativos con identidad, a menos que se establezca lo contrario.
    \end{obs}

    \begin{mydef}
        Sea $A$ un anillo. Denotemos por
        \begin{equation*}
            S_A=\left\{f\Big|\cf{f}{\mathbb{N}\cup\left\{0\right\}}{A} \right\}
        \end{equation*}
        es decir que $S_A$ es el \textbf{conjunto de sucesiones de $A$}. Si $f\in S_A$ escribimos a $f$ como:
        \begin{equation*}
            f=(a_0,a_1,...)
        \end{equation*}
        Sobre $S_A$ se definen dos operaciones, la \textbf{suma} y \textbf{producto}. A saber, si $f=(a_0,a_1,...)$ y $g=(b_0,b_1,...)$, entonces:
        \begin{equation*}
            f+g=(a_0+b_0,a_1+b_1,...,a_k+b_k,...)
        \end{equation*}
        y,
        \begin{equation*}
            fg = f\cdot g = (c_0,c_1,...,c_k,...)
        \end{equation*}
        donde
        \begin{equation*}
            \begin{split}
                c_k&=\sum_{ i=0}^k a_ib_{ k-i}\\
                &=a_0b_k+a_1b_{ k-1}+...+a_kb_0\\
                &=\sum_{ i=0}^k a_{k-i}b_{i}\\
                &=\sum_{ i+j=k}a_ib_j\\
            \end{split}
        \end{equation*}
    \end{mydef}

    \begin{obs}
        En la definición anterior, se tiene que $S_A$ es un anillo con cero el elemento $(0,0,...,0,...)$ e inverso $-f=(-a_0,-a_1,...,-a_k,...)$ para todo $f\in S_A$. Además, existe un monomorfismo de $A$ en $S_A$, a saber:
        \begin{equation*}
            A\hookrightarrow S_A, a\mapsto (a,0,...,0,...)
        \end{equation*}
        Por lo cual $A$ está encajado en $S_A$. Debido a esto, se denotará de ahora en adelante como
        \begin{equation*}
            a=(a,0,...,0,...),\quad\forall a\in A
        \end{equation*}
    \end{obs}

    \begin{mydef}
        Sean $A$ y $x$ un objeto tal que $x\notin A$. $x$ es llamado una \textbf{indeterminada para $A$}. Definimos para todo $n\in\mathbb{N}\cup\left\{0\right\}$ y para todo $a\in A$:
        \begin{equation*}
            ax^n=(\underbrace{0,0,0,...,0,0,0,a}_{{n+1}-\textbf{ésima entrada}},0,...)
        \end{equation*}
        Si $A$ tiene identidad, entonces
        \begin{equation*}
            1x^n=x^n=(\underbrace{0,0,0,...,0,0,0,1}_{{n+1}-\textbf{ésima entrada}},0,...)
        \end{equation*}
        En caso que $n=1$, $1x^1=x^1=x$ y si $n=0$, $1x^0=x^0=1$ (abusando en este caso de la notación). Se tiene entonces que
        \begin{equation*}
            x^n\in S_A,\quad\forall n\in\mathbb{N}\cup\left\{0\right\}
        \end{equation*}
    \end{mydef}



\end{document}