\documentclass[12pt]{report}
\usepackage[spanish]{babel}
\usepackage[utf8]{inputenc}
\usepackage{amsmath}
\usepackage{amssymb}
\usepackage{amsthm}
\usepackage{graphics}
\usepackage{subfigure}
\usepackage{lipsum}
\usepackage{array}
\usepackage{multicol}
\usepackage{enumerate}
\usepackage[framemethod=TikZ]{mdframed}
\usepackage[a4paper, margin = 1.5cm]{geometry}
\usepackage{tikz}

%En esta parte se hacen redefiniciones de algunos comandos para que resulte agradable el verlos%

\renewcommand{\theenumii}{\roman{enumii}}

\def\proof{\paragraph{Demostración:\\}}
\def\endproof{\hfill$\blacksquare$}

\def\sol{\paragraph{Solución:\\}}
\def\endsol{\hfill$\square$}

%En esta parte se definen los comandos a usar dentro del documento para enlistar%

\newtheoremstyle{largebreak}
  {}% use the default space above
  {}% use the default space below
  {\normalfont}% body font
  {}% indent (0pt)
  {\bfseries}% header font
  {}% punctuation
  {\newline}% break after header
  {}% header spec

\theoremstyle{largebreak}

\newmdtheoremenv[
    leftmargin=0em,
    rightmargin=0em,
    innertopmargin=-2pt,
    innerbottommargin=8pt,
    hidealllines = true,
    roundcorner = 5pt,
    backgroundcolor = gray!60!red!30
]{exa}{Ejemplo}[section]

\newmdtheoremenv[
    leftmargin=0em,
    rightmargin=0em,
    innertopmargin=-2pt,
    innerbottommargin=8pt,
    hidealllines = true,
    roundcorner = 5pt,
    backgroundcolor = gray!50!blue!30
]{obs}{Observación}[section]

\newmdtheoremenv[
    leftmargin=0em,
    rightmargin=0em,
    innertopmargin=-2pt,
    innerbottommargin=8pt,
    rightline = false,
    leftline = false
]{theor}{Teorema}[section]

\newmdtheoremenv[
    leftmargin=0em,
    rightmargin=0em,
    innertopmargin=-2pt,
    innerbottommargin=8pt,
    rightline = false,
    leftline = false
]{propo}{Proposición}[section]

\newmdtheoremenv[
    leftmargin=0em,
    rightmargin=0em,
    innertopmargin=-2pt,
    innerbottommargin=8pt,
    rightline = false,
    leftline = false
]{cor}{Corolario}[section]

\newmdtheoremenv[
    leftmargin=0em,
    rightmargin=0em,
    innertopmargin=-2pt,
    innerbottommargin=8pt,
    rightline = false,
    leftline = false
]{lema}{Lema}[section]

\newmdtheoremenv[
    leftmargin=0em,
    rightmargin=0em,
    innertopmargin=-2pt,
    innerbottommargin=8pt,
    roundcorner=5pt,
    backgroundcolor = gray!30,
    hidealllines = true
]{mydef}{Definición}[section]

\newmdtheoremenv[
    leftmargin=0em,
    rightmargin=0em,
    innertopmargin=-2pt,
    innerbottommargin=8pt,
    roundcorner=5pt
]{excer}{Ejercicio}[section]

%En esta parte se colocan comandos que definen la forma en la que se van a escribir ciertas funciones%

\newcommand\abs[1]{\ensuremath{\left|#1\right|}}
\newcommand\divides{\ensuremath{\bigm|}}
\newcommand\cf[3]{\ensuremath{#1:#2\rightarrow#3}}
\newcommand\natint[1]{\ensuremath{\left[\!\left[ #1\right]\!\right]}}
\newcommand{\afa}{\:
    \begin{tikzpicture}
        \draw [line width = 0.17 mm, black] (0,0) -- (-0.115,0.29);
        \draw [line width = 0.17 mm, black] (0,0) -- (0.115,0.29);
        \draw [line width = 0.17 mm, black] (-0.12,0) arc (190:-10:0.12cm);
    \end{tikzpicture}
    \:
}
%Este símvolo es para casi todo salvo una cantidad finita

%recuerda usar \clearpage para hacer un salto de página

\begin{document}
    \setlength{\parskip}{5pt} % Añade 5 puntos de espacio entre párrafos
    \setlength{\parindent}{12pt} % Pone la sangría como me gusta
    \title{Notas Curso Álgebra Moderna III
    
    Una Introducción a la Teoría de Galois Finita}
    \author{Cristo Daniel Alvarado}
    \maketitle

    \tableofcontents %Con este comando se genera el índice general del libro%

    %\setcounter{chapter}{3} %En esta parte lo que se hace es cambiar la enumeración del capítulo%
    
    \chapter{Extensiones de Campos}
    
    \section{Fundamentos}

    \begin{obs}
        El símbolo $\afa$ significa para casi todo salvo una cantidad finita de elementos.
    \end{obs}

    \begin{mydef}
        Sean $E$ y $F$ campos. Decimos que $E/F$ es una \textbf{extensión de campos} si se cumple que $F\subseteq E$. Se denomina como \textbf{grado de la extensión $E/F$} a la dimensión de $E$ como espacio vectorial sobre $F$, denotado por $[E:F]$, esto es
        \begin{equation*}
            [E:F]=\dim_{ F}(E)
        \end{equation*}
    \end{mydef}
    
    \begin{mydef}
        Decimos que una extensión de campos $E/F$ es una \textbf{extensión finita}, si $[E:F]<\infty$. En caso contrario, decimos que es una \textbf{extensión infinita}, y lo escribimos como $[E:F]=\infty$.
    \end{mydef}

    \begin{theor}
        Sea $K\subseteq F\subseteq E$ una torre de campos (también llamada cadena de campos). Entonces,
        \begin{equation*}
            [E:K]=[E:F]\cdot[F:K]
        \end{equation*}
    \end{theor}

    \begin{proof}
        Sea $\left\{\alpha_i \right\}_{ i\in I}$ y $\left\{\beta_j \right\}_{ j\in J}$ bases de $F$ sobre $K$ y $E$ sobre $F$, respectivamente.

        \begin{equation*}
            \begin{split}
                E & \\
                | & \leftarrow \left\{\beta_j \right\}_{ j\in J} \\
                F & \\
                | & \leftarrow \left\{\alpha_i \right\}_{ i\in I}\\
                K & \\
            \end{split}
        \end{equation*}
        
        Afirmamos que $\left\{\alpha_i\beta_j \right\}_{ (i,j)\in I\times J}$ es base de $E$ sobre $K$. En efecto, claramente $\alpha_i\beta_j\in E$ para todo $(i,j)\in I\times J$. Notemos que necesariamente
        \begin{equation*}
            \abs{\left\{\alpha_i\Big|i\in I \right\}}=\abs{I}\quad\textup{y}\quad\abs{\left\{\beta_j\Big|j\in J \right\}}=\abs{J}
        \end{equation*}
        por ser ambas bases. Sea $u\in E$, entonces $u$ se expresa de forma única como combinación lineal de elementos de la base $\left\{\beta_j \right\}_{ j\in J}$ con coeficientes en $F$, digamos
        \begin{equation*}
            u=\sum_{j\in J}f_j\beta_j
        \end{equation*}
        donde $f_j\in F$ para todo $j\in J$ y $f_j=0\afa j\in J$. Por otro lado, cada $f_j\in F$ se expresa de forma única como una combinación lineal de elementos de la base $\left\{\alpha_i\right\}_{ i\in I}$ sobre $K$:
        \begin{equation*}
            f_j=\sum_{ i\in I}k_{ i,j}\alpha_i
        \end{equation*}
        donde $k_{ i,j}\in K$ para todo $i\in I$ siendo $k_{ i,j}=0\afa i\in I$, para cada $j\in J$. Luego,
        \begin{equation*}
            \begin{split}
                u&=\sum_{j\in J}f_j\beta_j\\
                &=\sum_{j\in J}\left( \sum_{ i\in I}k_{ i,j}\alpha_i \right) \beta_j\\
                &=\sum_{(i,j)\in I\times J}k_{ i,j}\alpha_i\beta_j\\
            \end{split}
        \end{equation*}
        donde $k_{ i,j}\in K$ y $k_{ i,j}=0\afa (i,j)\in I\times J$. Luego
        \begin{equation*}
            E=\mathcal{L}\left(\left\{\alpha_i\beta_j\Big|(i,j)\in I\times J \right\} \right)
        \end{equation*}
        Probemos que $\left\{\alpha_i\beta_j \right\}_{ (i,j)\in I\times J}$ es l.i. sobre $K$. En efecto, sean $a_{ i,j}\in F$ tales que
        \begin{equation*}
            \sum_{ (i,j)\in I\times J}a_{ i,j}\alpha_i\beta_j=0
        \end{equation*}
        entonces,
        \begin{equation*}
            \begin{split}
                \sum_{ (i,j)\in I\times J}a_{ i,j}\alpha_i\beta_j&=0\\
                \Rightarrow \sum_{j\in J}\left(\sum_{i\in I}a_{ i,j}\alpha_i\right)\beta_j&=0\\
            \end{split}
        \end{equation*}
        como $\left\{\beta_j \right\}_{ j\in J}$ es base de $E$ sobre $F$, entonces
        \begin{equation*}
            \sum_{i\in I}a_{ i,j}\alpha_i=0\quad\forall j\in J
        \end{equation*}
        y, como $\left\{\alpha_i \right\}_{ i\in I}$ es base de $F$ sobre $K$, se tiene que
        \begin{equation*}
            a_{ i,j}=0\quad\forall (i,j)\in I\times J
        \end{equation*}
        Así, $\left\{\alpha_i\beta_j \right\}_{ (i,j)\in I\times J}$ es l.i. sobre $K$, y
        \begin{equation*}
            \abs{\left\{\alpha_i\beta_j\Big|(i,j)\in I\times J\right\}}=\abs{I\times J}=\abs{I}\abs{J}
        \end{equation*}
        por ende, $[E:K]=[E:F][F:K]$.
    \end{proof}

    \begin{exa}
        Sean $p,q\in\mathbb{N}$ números primos diferentes. Podemos suponer que $p<q$. Definimos
        \begin{equation*}
            E=\mathbb{Q}(\sqrt{p})
        \end{equation*}
    \end{exa}

    \begin{propo}
        Sean $p_1,...,p_n\in\mathbb{N}$ primos distintos tales que $p_i<p_{ i+1}$ para todo $i\in\natint{1,n-1}$. Definimos por recursión $E_0=\mathbb{Q}$ y
        \begin{equation*}
            E_i=E_{ i-1}(\sqrt{p_i}),\quad\forall i\in\natint{1,n}
        \end{equation*}
        Entonces, $E_0\subseteq E_1\subseteq ... \subseteq E_n$ es una torre de campos tal que la extensión $E_i/E_{ i-1}$ tiene una base $\left\{1,\sqrt{p_i} \right\}$ para todo $i\in\natint{1,n}$. En particular,
        \begin{equation*}
            [E_n:E_0]=2^n
        \end{equation*}
    \end{propo}

    \begin{proof}
        
    \end{proof}

    \section{Construcciones}

    \begin{obs}
        Sea $E$ un campo y $S\subseteq E$ un subconjunto de $E$ no vacío. Denotamos por $\mathcal{F}$ al conjunto de subcampos (respectivamente, $\mathcal{A}$ al de subanillos) que contienen a $S$ no triviales (es decir, que al menos contienen al 0 y 1). Se sabe que
        \begin{equation*}
            \bigcap\mathcal{F}\quad\textup{y}\quad\bigcap\mathcal{A}
        \end{equation*}
        son un campo y un anillo, respectivamente, contenidos en $E$. Ambos los denotamos por $(S)$ y $[S]$, respectivamente.
    \end{obs}

    \begin{mydef}
        En la observación $(S)$ es el mínimo subcampo de $E$ que contiene a $S$, llamado \textbf{subcampo generado por $S$} y, $[S]$ es el mínimo subanillo de $E$ que contiene a $S$, llamado \textbf{subanillo generado por $S$}.

        Al conjunto $S$ se le llama \textbf{conjunto de generadores de $(S)$ ó $[S]$}.

        En el caso que $S=\emptyset$, denominamos como \textbf{campo primo de $E$} a $(S)=(\emptyset)$ denotado por $P_E$.
    \end{mydef}

    \begin{obs}
        En general, $S\neq\emptyset$. Si $S\subseteq E$ y $K$ es subcampo de $E$, entonces denotamos por $K(S)$ (resp. $K[S]$) a $(K\bigcup S)$ (resp. $[K\cup S]$). Si $K$ y $L$ son subcampos de $E$, entonces $(L\cup K)$ (resp. $[L\cup K]$) es denotado por $KL=K(L)=LK=L(K)$ (resp. $K[L]=L[K]$). $KL$ es llamado \textbf{el compuesto de $K$ y $L$}.

        En el caso que $S$ sea subconjunto finito de $E$ con $S=\left\{u_1,...,u_n \right\}$ escribimos:
        \begin{equation*}
            K(u_1,...,u_n)=K(\left\{u_1,...,u_n \right\})
        \end{equation*}
    \end{obs}

    \begin{mydef}
        Sea $E/K$ una extensión de campos. Decimos que $E/K$ es \textbf{finitamente generada} si existen $u_1,...,u_n\in E$ tales que $E=K(u_1,...,u_n)$.
    \end{mydef}




\end{document}