\documentclass[12pt]{report}
\usepackage[spanish]{babel}
\usepackage[utf8]{inputenc}
\usepackage{amsmath}
\usepackage{amssymb}
\usepackage{amsthm}
\usepackage{graphics}
\usepackage{subfigure}
\usepackage{lipsum}
\usepackage{array}
\usepackage{multicol}
\usepackage{enumerate}
\usepackage[framemethod=TikZ]{mdframed}
\usepackage[a4paper, margin = 1.5cm]{geometry}

%En esta parte se hacen redefiniciones de algunos comandos para que resulte agradable el verlos%

\renewcommand{\theenumii}{\roman{enumii}}

\def\proof{\paragraph{Demostración:\\}}
\def\endproof{\hfill$\blacksquare$}

\def\sol{\paragraph{Solución:\\}}
\def\endsol{\hfill$\square$}

%En esta parte se definen los comandos a usar dentro del documento para enlistar%

\newtheoremstyle{largebreak}
  {}% use the default space above
  {}% use the default space below
  {\normalfont}% body font
  {}% indent (0pt)
  {\bfseries}% header font
  {}% punctuation
  {\newline}% break after header
  {}% header spec

\theoremstyle{largebreak}

\newmdtheoremenv[
    leftmargin=0em,
    rightmargin=0em,
    innertopmargin=0pt,
    innerbottommargin=5pt,
    hidealllines = true,
    roundcorner = 5pt,
    backgroundcolor = gray!60!red!30
]{exa}{Ejemplo}[section]

\newmdtheoremenv[
    leftmargin=0em,
    rightmargin=0em,
    innertopmargin=0pt,
    innerbottommargin=5pt,
    hidealllines = true,
    roundcorner = 5pt,
    backgroundcolor = gray!50!blue!30
]{obs}{Observación}[section]

\newmdtheoremenv[
    leftmargin=0em,
    rightmargin=0em,
    innertopmargin=0pt,
    innerbottommargin=5pt,
    rightline = false,
    leftline = false
]{theor}{Teorema}[section]

\newmdtheoremenv[
    leftmargin=0em,
    rightmargin=0em,
    innertopmargin=0pt,
    innerbottommargin=5pt,
    rightline = false,
    leftline = false
]{propo}{Proposición}[section]

\newmdtheoremenv[
    leftmargin=0em,
    rightmargin=0em,
    innertopmargin=0pt,
    innerbottommargin=5pt,
    rightline = false,
    leftline = false
]{cor}{Corolario}[section]

\newmdtheoremenv[
    leftmargin=0em,
    rightmargin=0em,
    innertopmargin=0pt,
    innerbottommargin=5pt,
    rightline = false,
    leftline = false
]{lema}{Lema}[section]

\newmdtheoremenv[
    leftmargin=0em,
    rightmargin=0em,
    innertopmargin=0pt,
    innerbottommargin=5pt,
    roundcorner=5pt,
    backgroundcolor = gray!30,
    hidealllines = true
]{mydef}{Definición}[section]

\newmdtheoremenv[
    leftmargin=0em,
    rightmargin=0em,
    innertopmargin=0pt,
    innerbottommargin=5pt,
    roundcorner=5pt
]{excer}{Ejercicio}[section]

%En esta parte se colocan comandos que definen la forma en la que se van a escribir ciertas funciones%

\newcommand\abs[1]{\ensuremath{\biglvert#1\bigrvert}}
\newcommand\divides{\ensuremath{\bigm|}}
\newcommand\cf[3]{\ensuremath{#1:#2\rightarrow#3}}
\newcommand\contradiction{\ensuremath{\#_c}}
\newcommand\natint[1]{\ensuremath{\left[\!\left[ #1\right]\!\right]}}

\DeclareMathOperator{\car}{car}
\DeclareMathOperator{\irr}{irr}

\begin{document}
    \setlength{\parskip}{5pt} % Añade 5 puntos de espacio entre párrafos
    \setlength{\parindent}{12pt} % Pone la sangría como me gusta
    \title{Notas Extensiones Separables AM III}
    \author{Cristo Daniel Alvarado}
    \date{Diciembre de 2023}
    \maketitle

    \tableofcontents %Con este comando se genera el índice general del libro%

    \setcounter{chapter}{3} %En esta parte lo que se hace es cambiar la enumeración del capítulo%
    
    \chapter{Extensiones Separables}
    
    \section{Resultados preeliminares}
    
    Para enunciar lo que es una extensión separable, se necesitarán demostrar algunos resultados preeliminares para enunciarlo de forma adecuada.

    \begin{propo}
        Sea $F$ un campo y $f(X)\in F[X]$ un polinomio no constante. Si
        \begin{enumerate}
            \item $\car(F)=0$, entonces $f'(X)\neq0$. \label{F_1}
            \item $\car(F)=p>0$, entonces $f'(X)=0$  si y sólo si $\exists g(X)\in F[X]$ tal que $f(X)=g(X^{p})$. \label{F_2}
        \end{enumerate}
    \end{propo}

    \begin{proof}
        En ambos casos, para la demostración se requiere de usar el polinomio $f'(X)$. Expresamos
        \begin{equation}
            f(X)=a_0+a_1x+\cdots+a_nx^n,\quad n\geq1,\textup{ }a_n\neq0
        \end{equation}
        
        De (\ref{F_1}): Se tiene que
        \begin{equation*}
            f'(X)=\cdots+na_nx^{n-1}
        \end{equation*}
        donde $na_n\neq0$ ya que $\car(F)=0$. Por tanto, $f'(X)\neq0$. 

        De (\ref{F_2}): Se probará el si, sólo si.

        $\Leftarrow)$: Supongamos que $\exists$ $g(X)\in F[X]$ tal que $f(X)=g(X^p)$. Expresamos a $g(X)=b_0+b_1x+\cdots+b_mx^m$, donde $b_m\neq0$. Entonces
        \begin{equation*}
            \begin{split}
                f(X)=&g(X^p)\\
                =&b_0+b_1X^p+\cdots+b_mX^{pm}\\
                \Rightarrow f'(X)=&pb_1X^{p-1}+\cdots+pmb_mX^{pm-1}\\
                =&0\cdot X^{p-1}+\cdots+0\cdot X^{pm-1}\\
                =&0\\
            \end{split}
        \end{equation*}  

        $\Rightarrow)$: Supongamos que $f'(X)=0$, donde $f'(X)=\sum_{i=1}^{m}ia_ix^{i-1}$, entonces $ia_i=0$, para todo $i=1,\cdots,m$. Si $a_i\neq0$ para algún $i$, entonces debe suceder que $i\cdot1=i=0$, por lo cual $\car(F)=p\divides i$. Luego si $a_i\neq0$, existe $m_i\in\mathbb{N}$ tal que $i=pm_i$. Escribiendo a $f(X)$ con todos sus términos no cero, se tiene que
        \begin{equation*}
            \begin{split}
                f(X)=&a_0+a_{pm_1}X^{pm_1}+\cdots+a_{pm_n}X^{pm_n}\\
                =&a_0+a_{pm_1}(X^p)^{m_1}+\cdots+a_{pm_n}(X^p)^{m_n}\\
                =&g(X^p)
            \end{split}
        \end{equation*}
        donde $g(X)=a_0+a_{pm_1}X+\cdots+a_{pm_n}X^{m_n}$, siendo $a_{pm_n}\neq0$, pues $f(X)\neq0$
    \end{proof}

    De este teorema anterior y de un teorema del capítlo pasado, se deduce de forma inmediata el siguiente corolario:

    \begin{cor}
        Sea $F$ un campo y $f(X)\in F[X]$ un polinomio irreducible. Si
        \begin{enumerate}
            \item $\car(F)=0$, entonces todas las raíces de $f(X)$ son simples.
            \item $\car(F)=p>0$, entonces $f(X)$ tiene una raíz simple si y sólo si, $\exists g(X)\in F[X]$ tal que $f(X)=g(X^{p})$.
        \end{enumerate}
    \end{cor}

    

    \begin{proof}
        Es inmediata de la proposición anterior.
    \end{proof}

    El siguiente teorema tiene como objetivo caracterizar las extensiones separables, enunciando un resultado importante para su definición.
    
    \begin{theor}
        Sea $F$ un campo con $\car(F)=p>0$. Sea $f(X)\in F[X]$ un polinomio irreducible, y $e\in\mathbb{N}^{*}$ tal que $f(X)\in F[x^{p^e}]$, pero $f(X)\notin F[x^{p^{e+1}}]$. Sea $\Psi(X)\in F[X]$ el polinomio tal que $f(X)=\Psi(X^{p^e})$. Entonces
        \begin{enumerate}
            \item $\Psi(X)$ es un polinomio irredicible en $F[X]$.\label{T_1_1_1}
            \item Todas las raíces de $\Psi(X)$ son simples.\label{T_1_1_2}
            \item Todas las raíces de $f(X)$ tienen la misma multiplicidad, a saber, $p^e$.\label{T_1_1_3}
            \item Si $m=\deg(\Psi)$, entonces $\deg(f)=p^em$.\label{T_1_1_4}
        \end{enumerate}
    \end{theor}
    \begin{proof}
        De (1): Supongamos que $\Psi(X)$ es descomponible, entonces existen $g(X),h(X)\in F[X]$ con grados $\geq1$ tales que
        \begin{equation*}
            \begin{split}
                \Psi(X)=&g(X)h(X)\\
                \Rightarrow f(X)=&g(X^p)h(X^p)\\
                =&g_1(X)h_1(X)\\
            \end{split}
        \end{equation*}
        donde $g_1(X)=g(X^p)$ y $h_1(X)=h(X^p)$ con grados $\geq1$, lo cual implicaría que $f(X)$ es reducible. Luego $\Psi(X)$ tiene que ser irredicible.

        De (2): Supongamos que $\Psi(X)$ admite una raíz multiple, entonces $\exists$ $g(X)\in F[X]$ tal que $\Psi(X)=g(X^p)$. Así
        \begin{equation*}
            \begin{split}
                f(X)=&\Psi(X^{p^{e}})\\
                =&g(X^{p^{e+1}})\\
                \in&F[x^{p^{e+1}}]
            \end{split}
        \end{equation*}
        lo cual es una contradicción. Por lo tanto $\Psi(X)$ debe tenera todas sus raíces simples.

        De (3): Sea $m=\deg(\Psi)$. Sean $\beta_1,\cdots,\beta_m\in\bar{F}$ todas las raíces de $\Psi(X)$ en alguna cerradura algebraica de $F$. Se tiene entonces que
        \begin{equation*}
            \begin{split}
                \Psi(X)=&a\left(x-\beta_1\right)\cdots\left(x-\beta_m\right)\\
                \Rightarrow f(X)=&\Psi(X^{p^e})\\
                =&a\left(x^{p^e}-\beta_1\right)\cdots\left(x^{p^e}-\beta_m\right)\\
            \end{split}
        \end{equation*}
        Donde $a\in F$ es alguna constante. Ahora, para cada $i=1,\cdots,m$ sea $\alpha_i\in\bar{F}$ una raíz del polinomio $X^{p^e}-\beta_i=0$, esto es $\beta_i=\alpha^{p^e}$. Notemos que si $i\neq j$, debe suceder que $\alpha_i\neq\alpha_j$. Por tanto
        \begin{equation*}
            asd
        \end{equation*}

        De (4): Es inmediata.
    \end{proof}

    Se deduce de forma inmediata el siguiente corolario.

    \begin{cor}
        Sea $F$ campo y $f(X)\in F[X]$ un polinomio irredicible. Entonces todas las raíces de $f(X)$ tienen la misma multiplicidad. Si $\car(F)=0$, la multiplicidad de estas raíces es $1$, y si $\car(F)=p>0$, tienen multiplicidad $p^e$, para algún $e\in\mathbb{N}^{*}$ (este $e$ se obtiene del teorema anterior). 
    \end{cor}

    \section{Extensiones separables}

    Ahora estamos en las condiciones de enunciar la definición de separabilidad.

    \begin{mydef}
        De acuerdo con las notaciones del teorema anterior y de su demostracion, tenemos que el número $\deg(\Psi)$ es llamado \textbf{el grado de separabilidad de $f$}, y al entero no negativo $e$ es llamado \textbf{el grado de inseparabilidad de $f$}.
    \end{mydef}

    En otras palabras, podemos ver que el grado de separabilidad de $f$ es el número de raíces distintas de $f$.

    \begin{mydef}
        Sea $F$ un campo y $\bar{F}$ una cerradura algebraica de $F$. Si $\alpha\in \bar{F}$ y $f(X)=\irr(\alpha, F, X)$, entonces se define \textbf{el grado de separabilidad de $\alpha$}, como el grado de separabilidad de $f$, y al exponente $e$ de inseparabilidad de $f$, será el \textbf{exponente de inseparabilidad de $\alpha$}.
    \end{mydef}

    En el caso en que $\car(F)=0$, el exponente y grado de inseparabilidad de $f$ y $\alpha$ no tienen sentido en estar definidos, pues en ambos casos su valor siempre será de $1$.

    En cualquier caso, si $\alpha\in \bar{F}$ se denota al grado de separabilidad de $\alpha$ como
    \begin{equation}
        \left[F(\alpha):F\right]_s
    \end{equation}

    En el caso de que $\car(F)=0$, se tiene que
    \begin{equation}
        \left[F(\alpha):F\right]_s=\left[F(\alpha):F\right]=\deg(\irr(\alpha, F, X))
    \end{equation}
    y, si $\car(F)=p>0$, entonces
    \begin{equation}
        \left[F(\alpha):F\right]_s=\frac{\left[F(\alpha):F\right]}{p^e}
    \end{equation}

    \begin{propo}
        Sea $F$ un campo, $\bar{F}$ una cerradora algebraica de $F$ y $\alpha\in\bar{F}$. Entonces, $\left[F(\alpha):F\right]_s=N$, donde $N\in\mathbb{N}$ es el número de $F$-homomorfismos de $F(\alpha)$ en $\bar{F}$.
    \end{propo}
    
    \begin{proof}
        Sea $f(X)=\irr(\alpha,F, X)$. Tomemos $\alpha_1,\dots,\alpha_m\in \bar(F)$ las raíces distintas de $f(X)$. Se tiene por definición que
        \begin{equation*}
            m=\left[F(\alpha):F\right]_{S}
        \end{equation*}
        Sea $\cf{\phi}{F(\alpha)}{\bar(F)}$ un $F$-homomorfismo. Sabemos que $\phi$ está completamente detemrinada por su acción sobre $\alpha$, teniendo que $\phi(\alpha)$ es raíces de $f(X)$, esto es debe ser que $\phi(\alpha)=\alpha_{i}$, con $i\in\natint{1,m}$. luego, a lo más tenemos $m$ $F$-homomorfismos de $F(\alpha)$ en $\bar{F}$, con lo cual se tiene el resultado.
    \end{proof}

    \begin{mydef}
        Sea $E/F$ una extensión algebraica. Se define \textbf{el grado de separabilidad de $E$ sobre $F$} como la cardinalidad del conjunto de $F$-homomorfismos que van de $E$ en $\bar{F}$, donde $\bar{F}$ es una cerradura algebraica de $F$ que contiene a $E$. Tal cardinal es denotado por $\left[E:F\right]_s$.
    \end{mydef}

    De resultados de capítulo anterior, se deduce de forma inmediata el siguiente teorema.

    \begin{theor}
        Sea $E/F$ una extensión finita y $K$ un campo intermedio de la extensión $E/F$. Entonces
        \begin{equation}
            \left[E:F\right]_s=\left[E:K\right]_s \left[K:F\right]_s
        \end{equation}
    \end{theor}

    \begin{proof}
        Es inmediata de un teorema anterior.
    \end{proof}

    \begin{mydef}
        Sea $F$ un campo y $\alpha\in\bar{F}$. Decimos que \textbf{$\alpha$ es separable sobre $F$} si $\left[F(\alpha):F\right]_s=\left[F(\alpha):F\right]$. Si $E/F$ es una extensión algebraica, entonces se dice que \textbf{$E/F$ es separable} o \textbf{$E$ es separable sobre $F$}, si todo elemento de $E$ es separable sobre $F$.
    \end{mydef}

    Veremos ahora algunas caracterizaciones de las extensiones separables.

    \begin{obs}
        Sea $F$ camop y $\bar{F}$ cerradura algebraica de $F$.
        \begin{enumerate}
            \item Si $\alpha\in\bar{F}$, entonces $\alpha$ es separable sobre $F$ si y sólo si $f(X)=\irr(\alpha, F, X)$ es tal que todas sus raíces son simples. Cuando esto ocurra decimos que \textbf{$f(X)$ es separable sobre $F$}.
            \item Si $g(X)\in F[X]$, decimos que \textbf{$g(X)$ es separable sobre $F$} si todos sus factores irreducibles son separables sobre $F$.
        \end{enumerate}
    \end{obs}

    \begin{propo}
        Sea $E/F$ una extensión finita con $\car(F)=p>0$. Entonces existe un elemento $t\in\mathbb{N}^{*}$ tal que
        \begin{equation}
            \left[E:F\right]=p^t\left[E:F\right]_s
        \end{equation}
        En particular, si $p\nmid \left[E:F\right]$, enotnces $\left[E:F\right]=\left[E:F\right]_s$.
    \end{propo}

    \begin{proof}
        Sean $\alpha_1,\dots,\alpha_n\in E$ tales que $E=F(\alpha_1,\dots,\alpha_n)$. Consideraremos la torre de campos $F\subseteq F(\alpha_1)\subseteq\cdots\subseteq F(\alpha_1,\dots,\alpha_{n-1})\subseteq F(\alpha_1,\dots,\alpha_n)$.
        Sea $e^{i}$ es exponente de inseparabilidad de $\alpha_i$ sobre $F(\alpha_1,\dots,\alpha_{i-1})$, con $i\in\natint{2,n}$ y $e_1$ el grado de inseparabilidad de $\alpha_1$ sobre $F$. Entonces
        \begin{equation*}
            \begin{split}
                \left[E:F\right]_s=&\left[F(\alpha_1,\dots,\alpha_n):F(\alpha_1,\dots,\alpha_{n-1})\right]_s\cdot\dots\cdot[F(\alpha_1):F]_s\\
                =&\frac{1}{p^{e_n}} \left[F(\alpha_1,\dots,\alpha_n):F(\alpha_1,\dots,\alpha_{n-1})\right]\cdot\dots\cdot\frac{1}{p^{e_1}}[F(\alpha_1):F]\\
                \Rightarrow [E:F]=&p^{e_1+e_2+\cdots+e_n}\left[E:F\right]_s\\
            \end{split}
        \end{equation*}
        tomando $t=e_1+\cdots+e_n\in\mathbb{Z}_{\geq0}$ se sigue el resultado.
    \end{proof}

    \begin{obs}
        Si $E/F$ es una extensión finita y $\car(F)=0$, entonces $\left[E:F\right]=\left[E:F\right]_s$.
    \end{obs}

    \begin{propo}
        Sea $E/F$ una extension de campos con $\car(F)=p>0$ y $\alpha\in E$ algebraico sobre $F$. Sea $e$ el exponente de inseparabilidad de $\alpha$ sobre $F$. Entonces
        \begin{enumerate}
            \item $\alpha^{p^e}$ es separable sobre $F$.
            \item Las siguientes condiciones son equivalentes:
            \begin{enumerate}
                \item $\alpha$ es separable sobre $F$.
                \item $\left[F(\alpha):F\right]_s=\left[F(\alpha):F\right]$.
                \item $e=0$.
                \item $F(\alpha)=F(\alpha^p)$.
            \end{enumerate}
        \end{enumerate}
    \end{propo}

    \begin{proof}
        De (1): Sea $f(X)=\irr(\alpha,F,X)$ y $\psi(x)\in F[X]$ tal que $\psi(X^{p^e})=f(X)$, pero $f(X)\notin F[X^{p^{e+1}}]$. Sabemos que $\psi(X)$ es irreducible sobre $F$ y que todas sus raíces son simples, donde
        \begin{equation*}
            0=f(\alpha)=\psi(\alpha^{p^e})
        \end{equation*}
        esto es, $\alpha^{p^e}$ es raíz de $\psi(X)$, por lo cual $\psi(X)=\irr(\alpha^{p^e},F,X)$. Por tanto, $\alpha^{p^e}$ es separable sobre $F$.

        De (2): Es claro que I)$\iff$II)$\iff$III). Probaremos que I)$\iff$IV). Antes, notemos que
        \begin{equation*}
            F\subseteq F(\alpha^p)\subseteq F(\alpha)
        \end{equation*}
        
        I)$\Rightarrow$ IV): Sea $f(X)=\irr(\alpha, F, X)$. Tenemos que $g(X)=X^p-\alpha^p\in F(\alpha^p)[X]$ y $\alpha$ es raíz de $g(X)$. Por lo cual $\irr(\alpha,F(\alpha^p),X)\divides g(X)$ y $\irr(\alpha,F(\alpha^p),X)
        \divides f(X)$ en $F(\alpha^p)[X]$.

        Entonces, como todas las raíces de $f(X)$ son simples, las raíces de $h(X)=\irr(\alpha,F(\alpha^p),X)$ también lo son; además $h(X)\divides X^p-\alpha^p=(X-\alpha)^p\Rightarrow h(X)=(x-\alpha)\Rightarrow \alpha\in F(\alpha^p)$. Por tanto, $F(\alpha)=F(\alpha^p)$.

        IV)$\Rightarrow$ I): Recíprocamente, supongamos que $F(\alpha)=F(\alpha^p)$ pero $\alpha$ no es separable sobre $F$. Siendo $f(X)=\irr(\alpha,F,X)$, tenemos que $f(X)\in F[X^p]$, esto es, existe $g(X)\in F[X]$ tal que $f(X)=g(X^p)$ donde $\deg(f)=p\cdot\deg(g)>\deg(g)$.
        
        Notemos que $g(X)$ tiene por raíz a $\alpha^p$, pues $g(\alpha^p)=f(\alpha)=0$, de esta forma $\irr(\alpha^p,F,X)\divides g(X)\Rightarrow [F(\alpha^p):F]=\deg(\irr(\alpha^p,F,X))=\deg(g)<\deg(f)=[F(\alpha):F]$, luego $F(\alpha^p)\subsetneq F(\alpha)$, lo cual es una contradicción. Por tanto, $\alpha$ es separable sobre $F$.

    \end{proof}

    \begin{propo}
        Sea $E/F$ una extensión finita. Entonces $E/F$ es separable si y sólo si $\left[E:F\right]_s=\left[E:F\right]$.
    \end{propo}

    \begin{proof}
        $\Rightarrow):$ Suponga que $E/F$ es separable. Sean $\alpha_1,\dots,\alpha_n\in E$ tales que $F(\alpha_1,\dots,\alpha_n)=E$. Consideremos la torre de campos:
        
        \begin{equation*}
            F\subsetneq F(\alpha_1)\subsetneq\dots\subsetneq F(\alpha_1,\dots,\alpha_n)=E
        \end{equation*}
        para cada $i\in\natint{2,n}$, tenemos que $\alpha_i$ es separable y, por ende, lo es sobre $F(\alpha_1,\dots,\alpha_{i-1})$. Luego,
        \begin{equation*}
            \begin{split}
                [E:F]_s=&[F(\alpha_1,\dots,\alpha_n):F(\alpha_1,\dots,\alpha_{n-1})]_s\cdot\dots\cdot[F(\alpha_1):F]_s\\
                =&[F(\alpha_1,\dots,\alpha_n):F(\alpha_1,\dots,\alpha_{n-1})]\cdot\dots\cdot[F(\alpha_1):F]\\
                =&[E:F]
            \end{split}
        \end{equation*}

        $\Leftarrow):$ Sea $\alpha\in E$ arbitrario. Tenemos lo siguiente:
        \begin{equation*}
            \begin{split}
                [E:F(\alpha)]_s\cdot[F(\alpha):F]_s=&[E:F]_s\\
                =&[E:F]\\
                =&[E:F(\alpha)]\cdot[F(\alpha):F]\\
            \end{split}
        \end{equation*}
        donde $[E:F(\alpha)]_s\leq [E:F(\alpha)]$ y $[F(\alpha):F]_s\leq [F(\alpha):F]$. Por la igualdad anterior debe suceder que
        \begin{equation*}
            [F(\alpha):F]=[F(\alpha):F]
        \end{equation*}
        esto es, que $\alpha$ es separable sobre $F$. Como el $\alpha$ fue arbitrario, entonces se sigue que la extensión $E/F$ es una extensión separable.

    \end{proof}

    \begin{obs}
        Sea $F\subseteq K \subseteq E$ una torre de campos y $\alpha\in E$ separable sobre $F$. Entonces $\alpha$ es separable sobre $K$. Más generalmente, sean $E/F$ y $K/F$ extensiones de campos y $\alpha\in E$ separable sobre $F$. Si $\alpha$ es elemento de un campo $L$ extensión de $K$, entonces $\alpha$ es separable sobre $K$.
    \end{obs}

    \begin{propo}
        Sea $E/F$ una extensión de campos y $S\subseteq E$ tal que $E=F(S)$. Sea.
        \begin{equation}
            K = \left\{\alpha\in E | \alpha \textup{ es separable sobre }F\right\}\label{equation:k_sep_cer}
        \end{equation}
        Entonces
        \begin{enumerate}
            \item $K$ es un subcampo intermedio de la extensión $E/F$.
            \item $E/F$ es separable si y sólo si $\alpha$ es separable sobre $F$, para todo $\alpha\in S$.
        \end{enumerate}
    \end{propo}

    \begin{proof}
        De (1): Probaremos que $K$ es campo y que $F\subseteq K\subseteq E$. En efecto, sea $\alpha\in F$, se tiene que $\alpha$ es algebraico sobre $F$, con polinomio irreducible $f(X)=X-\alpha$, el cual tiene todas sus raíces distintas, por lo cual $\alpha$ es separable sobre $F$. Entonces $F\subseteq K\subseteq E$.
        Sean ahora $\alpha,\beta\in K\neq\emptyset$, pues $F\subseteq K$. Consideremos el campo intermedio de la extensión $E/F$, $F(\alpha,\beta)$. Se tiene entonces la torre de campos
        \begin{equation*}
            F\subseteq F(\alpha)\subseteq F(\alpha, \beta)\subseteq E
        \end{equation*}
        Como $\beta$ es separable sobre $F$, lo es sobre $F(\alpha)$, luego como el grado de separabilidad es multiplicativo, se tiene que
        \begin{equation*}
            \begin{split}
                \left[F(\alpha,\beta):F\right]_s=&\left[F(\alpha,\beta):F(\alpha)\right]_s\left[F(\alpha):F\right]_s\\
                =&\left[F(\alpha,\beta):F(\alpha)\right]\left[F(\alpha):F\right]\\
                =&\left[F(\alpha,\beta):F\right]\\
            \end{split}
        \end{equation*}
        por lo cual, la extensión $F(\alpha,\beta)/F$ es separable, luego los elementos $\alpha-\beta, \alpha\beta, \alpha^{-1}\in F(\alpha,\beta)$ son separables sobre $F$. Por tanto, $K$ es campo y por lo anterior, es subcampo intermedio de la extensión $E/F$.
        
        De (2): Veamos que

        $\Rightarrow$): Es inmediata, pues si $E/F$ es separable todo elemento de $E$ es separable sobre $F$. En particular todo elemento de $S$ es separable sobre $F$.

        $\Leftarrow$): Supongamos que $\alpha$ es separable sobre $F$, para todo $\alpha\in S$. Por (1) se tiene que $S\subseteq K$ y $F\subseteq K$, pero como $K$ es subcampo de $E$, se tiene que $F(S)\subseteq K$, por lo cual $F(S)=E=K$. Así, todos los elementos de $E$ son separables sobre $F$, es decir $E/F$ es una extensión separable. 
    \end{proof}

    \begin{mydef}
        El campo $K$ de la definición (\ref{equation:k_sep_cer}) es llamado \textbf{la cerradura separable} o de la extensión $E/F$ o simplemente de $E/F$, o de $F$ en $E$.

        Si consideramos la extensión $\bar{F}/F$, entonces la cerradura separable de $F$ en $\bar{F}$ simplemente se dice es la \textbf{cerradura separable de F}.
    \end{mydef}
    
    \begin{obs}
        Si $E/F$ es una extensión algebraica de tal manera que $E\subseteq \bar{F}$, entonces la cerradura separable de $F$ en $E$, $K$, es la intersección de la cerradura separable de $F$ con $E$.
    \end{obs}

    \begin{obs}
        En la literatura no existe notación establecida para referirse a la cerradura normal. En este momento nosotros acordaremos la siguiente. Sobre la extensión $E/F$, se denotará a la cerradura separable de $F$ en $E$ por:
        \begin{equation*}
            F_{S,E/F}\quad\textup{o}\quad F_{S,F}^{E}
        \end{equation*} 
        Cuando la extensión es $\bar{F}/F$ será
        \begin{equation*}
            F_S
        \end{equation*}
        y a veces a la cerradura algebraica se le denota por $\bar{F}=F^{a}$.
    \end{obs}

    \begin{propo}
        Sea $E/F$ una extensión normal \& $F_S$ la cerrradura separable de $E/F$. Entonces, la extensión $F_S/F$ es normal.
    \end{propo}

    \begin{proof}
        Sea $\alpha\in F_S$ con $f(X)=\irr(\alpha, F, X)$, y $\beta\in \bar{F}$ tal que $\alpha$ y $\beta$ son $F$-conjugados, es decir que ambos son raíces del polinomio $f(X)$.
        Como la extensión $E/F$ es normal, entonces $\beta\in E$, donde $\irr(\beta, F, X)=f(X)$ es separable sobre $F$, pues $\alpha$ es separable sobre $F$, es decir, $\beta$ es separable sobre $F$.
        Luego $\beta\in F_S$.
        Por tanto, la extensión $F_S/F$ es normal.
    \end{proof}

    \begin{obs}
        Si $F$ es campo, la extensión $\bar{F}/F$ es normal, por lo cual las extensiones $\bar{F}/F_S$ y $F_S/F$ son ambas normales (siendo $F_S$ la cerradura separable de $F$). 
    \end{obs}

    \begin{propo}
        Sea $E/F$ una extensión finita. y $F_S$ la cerradura separable de $F$ en $E$. Entonces,
        \begin{equation}
            \left[F_S:F\right]=\left[E:F\right]_s
        \end{equation}
    \end{propo}

    \begin{proof}
        Tenemos dos casos:
        \begin{itemize}
            \item Si $\car(F)=0$, entonces la extensión $E/F$ es separable y por tanto $F_S=E$. Por tanto
            \begin{equation*}
                \begin{split}
                    \left[F_S:F\right]=&\left[E:F\right]\\
                    =&\left[E:F\right]_s\\
                \end{split}
            \end{equation*}
            \item Si $\car(F)=p>0$. Tenemos que
            \begin{equation*}
                \begin{split}
                    \left[E:F\right]_S=&\left[E:F_S\right]_S\left[F_S:F\right]_S\\
                    =&\left[E:F_S\right]_S\left[F_S:F\right]\\
                \end{split}
            \end{equation*}
            Para probar el resultado, basta con probar que $\left[E:F_S\right]_S=1$. Recordemos que $\left[E:F_S\right]_S$ es el cardinal de $F_S$-homomorfismos de $E$ en $\bar{F}=\bar{F_S}$. Sea entonces $f:E\rightarrow \bar{F}$ un $F_s$-homomorfismo.
            Sea $\alpha\in E$. Si $\alpha\in F_S$m entonces $f(\alpha)=\alpha$. Si $\alpha\notin F_S$, se tiene por definción de $F_S$ que $\alpha$ no es separable sobre $F$. 
            
            Sea $f(X)=\irr(\alpha, F, X)$, y tomemos $e\in\mathbb{N}^{*}$ su exponente de inseparabilidad. 
            Por un resultado anterior sucede que $\alpha^{p^e}$ es separable sobre $F$, es decir $\alpha^{p^e}\in F_S$. Luego,
            \begin{equation*}
                \begin{split}
                    f(\alpha^{p^e})=&\alpha^{p^e}\\
                    \Rightarrow \left(\alpha-f(\alpha)\right)^{p^e}=&\alpha^{p^e}-f(\alpha)^{p^e}\\
                    =&0\\
                    \Rightarrow f(\alpha)-\alpha=&0\\
                    \Rightarrow f(\alpha)=&\alpha\\
                \end{split}
            \end{equation*}
            Es decir, $f=\textup{id}_E$. Por tanto $\left[E:F_S\right]=1$. Así por la ecuación anterior
            \begin{equation*}
                \left[E:F\right]_S\left[F_S:F\right]
            \end{equation*}
        \end{itemize}
    \end{proof}

    \begin{theor}
        La clase de extensiones separables es una clase distinguida.
    \end{theor}

    \begin{proof}
        De (a): Sea $F\subseteq K\subseteq E$ una torre de campos. Probaremos que $E/F$ es separable si, y sólo si $E/K$ y $K/F$ son separables.

        $\Rightarrow$): Supongamos que $E/F$ es separable. Sabemos ya que $E/K$ es separable. Pero, por otro lado, es claro que la extensión $K/F$ es separable.

        $\Leftarrow$): Supongamos que las extensiones $E/K$ y $K/F$ son separables. Sea $\alpha\in E$ arbitrario y tomemos $f(X)=\irr(\alpha, K, X)$, digamos
        \begin{equation*}
            f(X)=a_0+a_1X+\cdots+a_{m-1}X^{m-1}+X^m\in K[X]
        \end{equation*}
        Tenemos que $f(X)$ es separable sobre $K$, es decir todas las raíces de $f(X)$ son simples. Consideremos la torre de campos:
        \begin{equation*}
            F\subseteq F(a_0,a_1,\dots,a_{m-1})\subseteq F(a_0,a_1,\dots,a_{m-1},\alpha)
        \end{equation*}
        donde $F(a_0,a_1,\dots,a_{m-1})/F$ es finita y separable, al igual que $F(a_0,a_1,\dots,a_{m-1},\alpha)/F(a_0,a_1,\dots,a_{m-1})$. Notemos que $f(X)=\irr(\alpha, F(a_0,a_1,\dots,a_{m-1}), X)$. Entonces
        \begin{equation*}
            \begin{split}
                \left[F(a_0,a_1,\dots,a_{m-1},\alpha):F\right]=&\left[F(a_0,a_1,\dots,a_{m-1},\alpha):Fa_0,a_1,\dots,a_{m-1}\right]\left[F(a_0,a_1,\dots,a_{m-1}):F\right]\\
                =&\left[F(a_0,a_1,\dots,a_{m-1},\alpha):Fa_0,a_1,\dots,a_{m-1}\right]_s\left[F(a_0,a_1,\dots,a_{m-1}):F\right]_s\\
                =&\left[F(a_0,a_1,\dots,a_{m-1},\alpha):F\right]_s\\
            \end{split}
        \end{equation*}
        es decir, $F(a_0,a_1,\dots,a_{m-1},\alpha)/F$ es una extensión separable, en particular se tiene que $\alpha$ es separable sobre $F$. Por ser el $\alpha$ arbitrario en $E$, se sigue que $E/F$ es una extensión separable.

        De (b): Sean $E/F$ y $K/F$ extensiones separables, dónde $E/F$ es separable y, $E$ y $K$ subcampos de un campo común $L$. Como $K(E)=KE$, entonce basta ver que  los elementos de $E$ son separables sobre $K$, lo cual ya se tiene.

        Entonces, $KE/F$ es una extensión separable.
    \end{proof}

    \begin{cor}
        Sean $E/F$ y $K/F$ extensiones separables, con $E$ y $K$ subcamopos de un campo común $L$. Entonces, $KE/F$ es separable.
    \end{cor}

    \begin{proof}
        Es inmediato de la proposición teorema.
    \end{proof}

    \begin{mydef}
        Sea $F$ un campo. Se dice que \textbf{$F$ es perfecto} si toda extensión algebraica de $F$ es separable.
    \end{mydef}

    \begin{obs}
        Todo campo de caracterísitica $0$ es perfecto (ya que toda extensión algebraica de un campo con caracterísica $0$ sigue teniendo característica $0$, es decir que la extensión siempre va a ser separable).
    \end{obs}

    \begin{mydef}
        Sea $F$ campo de caracerística $p>0$. Sea $n\in\mathbb{N}$. Se define la función $\phi_n:F\rightarrow F$, $\alpha\mapsto\alpha^{p^n}$.

        Se tiene que $\phi_n$ es un homomorfismo, llamado \textbf{el homomorfismo de Fröbenius de grado $n$}. Para $n=1$ se dice simplemente que $\phi_1$ es el homomorfismo de Fröbenius, y se denota por $\phi$.
    \end{mydef}

    \begin{theor}
        Sea $F$ un campo de caracterísitca $p>0$. Las siguientes condiciones son equivalentes:
        \begin{enumerate}
            \item $F$ es perfecto.
            \item Toda extensión finita de $F$ es separable.
            \item Todo polinomio irreducible sobre $F$ es separable.
            \item Todo polinomio sobre $F$ es separable.
            \item El homomorfismo de Fröbenius es un automorfismo de $F$.
        \end{enumerate}
    \end{theor}
    
    \begin{proof}
        $(1)\Rightarrow(2)$: Es inmediato.
        
        $(2)\Rightarrow(3)$: Sea $f(X)\in F[X]\backslash F$ irreducible y sea $\alpha\in\bar{F}$ una raíz de $f(X)$. Por hipótesis, $F(\alpha)$ es una extensión separable de $F$, luego $\alpha$ es separable sobre $F$. Como $f(X)$ es asociado con $\irr(\alpha, F, X)$, entonces $f$ es separable sobre $F$.

        $(3)\iff(4)$: Es inmediato.

        $(4)\Rightarrow(5)$: Es claro que $\cf{\phi}{F}{F}$ es un monorfismo. Sea $\alpha\in F$ y considérese $f(X)=X^p-\alpha\in F[X]$. Sea $\beta\in\bar{F}$ una raíz de $f(X)$ y sea $g(X)=\irr(\beta,F,X)$, el cual es separable y divide a $f(X)$. Pero $f(X)=X^p-\alpha^p=(X-\alpha)^p$, así $\beta$ es la única raíz de $f(X)$, por lo que también lo es de $g(X)$. Por tanto, $g(X)=X-\beta\in F[X]$.

        Luego, $\beta\in F$. Así pues, $\phi$ es suprayectiva, luego es un automorfismo de $F$.

        $(5)\Rightarrow(1)$: Sea $E/F$ una extensión algebraica. Sean $\alpha\in E$ y $f(X)=\irr(\alpha, F, X)$, cuyas raíces tienen multiplicidad $p^e$, con $e\in\mathbb{Z}_{\geq0}$, siendo $e$ el exponente de inseparabilidad de $\alpha$. Suponiendo que $e\geq1$, entonces $\alpha$ es raíz múltiple de $f(X)$, por lo que existe $g(x)\in F[X]$ tal que $f(X)=g(X^p)$. Sea
        \begin{equation*}
            g(X)=b_0+b_1X+\cdots+b_mX^m
        \end{equation*}
        
        Por hipótesis, para todo $i\in\left\{0,\dots,m\right\}$ existe $c_i\in F$ tal que $b_i=c_i^p$. Pero, esto implica que
        \begin{equation*}
            f(X)=c_0^p+c_0^pX^p+\cdots c_m^pX^{mp}=\left(c_0+c_1X+\cdots+c_mX^m\right)^p
        \end{equation*}
        lo cual contradice el hecho de que $f(X)$ sea irredicible. Por tanto, $e=0$, luego $\alpha$ es separable sobre sobre $F$ y, en consecuencia, $E/F$ es una extensión separable.
    \end{proof}

    \begin{theor}
        Todo campo finito es perfecto.
    \end{theor}

    \begin{proof}
        Considerando el homomorfismo de Fröbenius $\cf{\phi}{F}{F}$, se tiene que $\phi$ es inyectivo, por lo cual $\big| \phi(F) \big| = \big| F\big|$. Pero, como $F$ es finito, entonces $\phi(F)=F$, luego $\phi$ es automorfismo de $F$. Así pues, $F$ es perfecto.
    \end{proof}

    \begin{mydef}
        Sea $E/F$ una extensión de campos y $\alpha$ inseparable sobre $F$. Entonces $f(X)=\irr(\alpha,F,X)$ es de la forma $f(X)=\left(X-\alpha_1\right)^{p^e}\cdot...\cdot\left(X-\alpha_m\right)^{p^e}$ con $e\geq1$, Se dice que $\alpha$ es \textbf{puramente inseparable sobre $F$} si y sólo si existe $t\in\mathbb{Z}_{\geq0}$ tal que $\alpha^{p^t}\in F$.
    \end{mydef}

    \section{Extesiones puramente inseparables}

    \begin{mydef}
        Sea $E/F$ una extensión de campos con $\car(F)=p>0$ y $\alpha\in E$. Decimos que $\alpha$ \textbf{es puramente inseparable} si existe $t\in \mathbb{Z}$, $t\geq 0$ tal que $\alpha^{p^t}\in F$.
        La extensión $E/F$ \textbf{es p.i.} si todo elemento de $E$ es p.i. sobre $F$.
    \end{mydef}

    \begin{obs}
        Si $E/F$ es una extensión de campos, entonces todos los elementos de $F$ son p.i. (separables) sobre $F$. 
    \end{obs}

    \begin{propo}
        Sea $E/F$ una extensión de campos con $\car(F)=p>0$. Sea
        \begin{equation*}
            K:=\left\{\alpha\in E|\alpha\textup{ es puramente inseparable sobre }F\right\}
        \end{equation*}
        (por la observación anterior, $K\neq \emptyset$). Entonces, $K$ es subcampo de $E$ que contiene a $F$.
    \end{propo}

    \begin{proof}
        Es claro que $K\neq \emptyset$ y $F\subseteq K \subseteq E$. Sean $\alpha, \beta\in K$, y $t_1,t_2\in \mathbb{Z}_{\geq0}$ tales que
        \begin{equation*}
            \alpha^{p^t_1},\beta^{p^t_2}\in F
        \end{equation*}
        Sea $t=\max\left\{t_1,t_2\right\}$. Por lo cual $\alpha^{p^t},\beta^{p^t}\in F$, así
        \begin{equation*}
            \begin{split}
                \left(\alpha-\beta\right)^{p^t}&=\alpha^{p^t}-\beta^{p^t}\in F\\
                \left(\alpha\beta\right)^{p^t}&=\alpha^{p^t}\beta^{p^t}\in F\\
                \left(\alpha^{-1}\right)^{p^t}&=\left(\alpha^{p^t}\right)^{-1}\in F\textup{ donde $\alpha\neq0$}
            \end{split}
        \end{equation*}
        por lo cual $K$ es campo intermedio de la extensión $E/F$.
    \end{proof}

    \begin{propo}
        Sea $E/F$ una extensión algebraica, con $\car(F)=p>0$. Sea $S\subseteq E$ tal que $E=F(S)$. Entonces, las siguientes condiciones son equivalentes:
        
        \begin{enumerate}
            \item $E/F$ es puramente inseparable.
            \item Todo elemento de $S$ es puramente inseparable sobre $F$.
            \item Los elementos de $E$ que son puramente inseparables y separables sobre $F$ son exactamente los de $F$.
            \item Si $\phi:E\rightarrow \Bar{F}$ es un $F$-homomorfismo, entonces $\phi(\alpha)=\alpha$, para todo $\alpha\in E$.
        \end{enumerate}
    \end{propo}

    \begin{proof}
        $(1)\Rightarrow(2)$: Es inmediato.

        $(2)\Rightarrow(3)$: Sea $\alpha\in E$ tal que es puramente inseparable sobre $F$ y separable sobre $F$, y $e\in\mathbb{Z}_{\geq0}$ el exponente de inseparablilidad de $\alpha$ sobre $F$.

        Tenemos que $\alpha^{p^e}$ es separable sobre $F$ (por una proposición anterior). Por otro lado, sea $t\in\mathbb{Z}_{\geq0}$ tal que $\alpha^{p^e}\in F$. Podemos suponer que $t\geq e$. Luego, $\alpha$ es raíz del polinomio $g(X)=X^{p^t}-\alpha^{p^t}=(X-\alpha)^{p^t}$, por lo cual $f(X)|g(X)$, donde $f(X)=\irr(\alpha,F,X)$.

        Así $f(x)=(X-\alpha)^{p^t}$. Como $\alpha$ es separable sobre $F$, se tiene que $e=0$, es decir que $f(X)=X-\alpha\in F[X]$, en particular, $\alpha\in F$.

        $(3)\Rightarrow(4)$: Sea $\phi:E\rightarrow\Bar{F}$ un $F$-homomorfismo arbitrario, y $\alpha\in E$, con $e\in\mathbb{Z}_{\geq0}$ su exponente de inseparabilidad. Sabemos que $\alpha^{p^e}$ es separable sobre $F$. Por hipótesis, $\alpha^{p^e}\in F$. Por lo cual
        \begin{equation*}
            \begin{split}
                \phi(\alpha^{p^e})&=\alpha^{p^e}\\
                \Rightarrow (\phi(\alpha)-\alpha)^{p^e}=\left(\phi\alpha^{p^e}\right)-\alpha^{p^e}&=0\\
                \Rightarrow \phi(\alpha)&=\alpha\\
            \end{split}
        \end{equation*} 

        $(4)\Rightarrow(1)$: sea $\alpha\in E$ arbitrario. Probaremos que existe $t\in\mathbb{Z}_{\geq0}$ tal que $\alpha^{p^t}\in F$. Sea $\beta\in\bar{F}$ un $F$-conjugado de $\alpha$. Sabemos que existe un $F$-isomorfismo $\psi:F(\alpha)\rightarrow F(\beta)$ tal que $\psi(\alpha)=\beta$. Extendemos $\psi$ a un $F$-homomorfismo $\phi:E\rightarrow \Bar{F}$. Por hipótesis, se tiene que $\phi(\gamma)=\gamma$, para todo $\gamma\in E$, en particular $\beta = \psi(\alpha)=\phi(\alpha)=\alpha$. Luego, si $e\in\mathbb{Z}_{\geq0}$ es el exponente de inseparabilidad de $\alpha$, entonces
        \begin{equation*}
            \begin{split}
                f(X)=&\irr(\alpha,F,X)\\
                =&(X-\alpha)^{p^e}\\
                =&X^{p^e}-\alpha^{p^e}\in F[X]\\
            \end{split}
        \end{equation*}
        por tanto $\alpha^{p^e}\in F$. Luego $\alpha$ es p.i. sobre $F$.
    \end{proof}

    \begin{mydef}
        Si $E/F$ es una extensión algebraica con $\car(F)=p>0$, entonces la \textbf{cerradura puramente inseparable} de la extensión $E/F$ o de $E$ en $F$, es el campo intermedio de todos los elementos $\alpha\in R$ tal que son puramente inseparables sobre $F$.
    \end{mydef}

    \begin{obs}
        Si $E/F$ es finita, entonces $E/F$ es p.i. $\iff$ $[E:F]_S=1$.
    \end{obs}

    \begin{obs}
        Sea $E/F$ una extensión algebraica la cual es p.i. y separable. Entonces, tenemos que $E=F$.
    \end{obs}

    \begin{theor}
        La clase de extensiones p.i. forman una clase distinguida.
    \end{theor}

    \begin{proof}
        (a): Sea $F\subseteq K\subseteq E$ una torre de campos con $\car(F)=p>0$. Supóngase que $E/F$ es puramente inseparable. Sea $\alpha\in E$, entonces existe $t\in\mathbb{R}_{\geq0}$ tal que $\alpha^{p^t}\in F\subseteq K$, por tanto $E/K$ es puramente inseparable.

        Por otro lado, es claro que todos los elementos de $K$ son p.i. sobre $F$, por lo cual $K/F$ es puramente inseparable.

        Recíprocamente, suponga que $E/K$ y $K/F$ son p.i. Sea $\alpha\in E$ ,entonces existe $r\in\mathbb{Z}_{\geq0}$ tal que $\alpha^{p^r}\in K$. Pero para este elemento existe $s\in\mathbb{Z}_{\geq0}$ tal que $(\alpha^{p^r})^{p^s}\in F$, es decir $\alpha^{p^{r+s}}\in F$. Por tanto, $E/F$ es puramente inseparable.

        (b): Sean $E/F$ y $K/F$ extensiones de campos con $\car(F)=p>0$, donde $E$ y $K$ son subcampos de un campo común $L$. Supóngase que la extensión $E/F$ es p.i. Probaremos que la extensión $EK/K$ es p.i.
        
        Tenemos que $EK=K(E)$. Si $\alpha\in E$, entonces existe $t\in\mathbb{Z}_{\geq0}$ tal que $\alpha^{p^t}\in F\subseteq K$. Por tanto, $EK/K$ es p.i.

        Por (a) y (b), se sigue que la clase de extensiones p.i. es una clase distinguida.
    \end{proof}

    \begin{cor}
        Sean $E/F$ y $K/F$ extensiones de campos tales que $\car(F)=p>0$, donde $K$ y $E$ son subcampos de un campo común $L$. Si $E/F$ y $K/F$ son p.i., entonces $EK/F$ es p.i.  
    \end{cor}

    \begin{proof}
        Es inmediato del teorema anterior.
    \end{proof}

    Sea $E/F$ una extensión algebraica con $\car(F)=p>0$, y sean $F_i$ y $F_s$ las cerraduras p.i. y separables, respectivamente. Entonces tenemos el siguiente diagrama:


    donde $F_i\cap F_s=F$.

    \begin{propo}
        En las condiciones de las notaciones anteriores, tenemos lo siguiente
        \begin{enumerate}
            \item $E/F_s$ es p.i.
            \item $E/F_i$ es separable si y sólo si $E=F_iF_s$.
        \end{enumerate}
    \end{propo}

    \begin{proof}
        De (1): Sea $\alpha\in E$, y $e\in\mathbb{Z}_{\geq0}$ su expontente de inseparabilidad sobre $F$. Sabemos que $\alpha^{p^e}$ es separable sobre $F$, por lo cual $\alpha\in F_s$. De esta forma, $E/F_s$ es puramente inseparable.

        De (2):

        $\Rightarrow)$: Supóngase que $E/F_i$ es separable, entonces $E/F_iF_s$ es separable y p.i., por lo cual $E=F_iF_s$.

        $\Leftarrow)$: Es inmediata.
    \end{proof}

    \begin{propo}
        Sea $F$ un campo cualquiera tal que $\car(F)=p>0$. Sea $\bar(F)$ su cerradura algebraica y $F_i$ la cerradura p.i. de $F$ en $\bar{F}$. Tenemos lo siguiente
        \begin{enumerate}
            \item El campo $F_i$ es perfecto.
            \item $F_i\cap F_s=F$ y $F_iF_s=\bar{F}$, dónde $F_s$ es la cerradura separable de $F$ en $\bar{F}$.
            \item Si $K$ es un campo perfecto tal que $F\subseteq K$, con $K\subseteq\bar{F}$, entonces $F_i\subseteq K$.
        \end{enumerate}
    \end{propo}

    \begin{proof}
        De (1): Probemos que $F_i^p=F_i$, donde
        \begin{equation*}
            F_i^p=\left\{\alpha^p|\alpha\in F_i\right\}
        \end{equation*}
        ya se tiene que $F_i^p\subseteq F_i$. Sea $\alpha\in F_i$, y $\beta\in \bar{F}$ tal que $\alpha=\beta^p$. Luego, existe $t\in\mathbb{Z}_{\geq0}$ tal que $\beta^{p^{t+1}}=\alpha^{p^t}\in F$, por lo cual $\beta\in F_i$. Asi $\alpha=\beta^p\in F_i$.

        Por tanto, $F_i=F_i^p$. Luego, $F_i$ es un campo perfecto.

        De (2): Ya sabemos que $F_i\cap F_s=F$. Para la otra igualdad, como $F_i$ es un campo perfecto, entonces la extensión $\bar{F}/F_i$ es separable, lo cual implica que $F_iF_s=\bar{F}$.

        De (3): Sea $K$ un campo intermedio de la extensión $\bar{F}/F$ el cual es perfecto. Probemos que $F_i\subseteq K$. Sea $\alpha\in F_i$. Consideremos la extensión $K(\alpha)/K$, esta extensión es separable; por otro lado, existe un elemento $t\in\mathbb{Z}_{\geq0}$ tal que $\alpha^{p^t}\in F\subseteq K$, luego la extensión $K(\alpha)/K$ es p.i., así $K(\alpha)=K$ lo cual implica que $\alpha\in K$.

        Por ende, $F_i\subseteq K$.
    \end{proof}

    \begin{cor}
        Sea $F$ un campo con $\car(F)=p>0$. Entonces, la intersección de cualquier familia de subcampos de $\bar{F}$ que contienen a $F$ es un campo perfecto.
    \end{cor}

    \begin{proof}
            Es inmediata.
    \end{proof}

    \begin{mydef}
        Sea $E/F$ una extensión finita arbitraria. Se define \textbf{el grado de inseparabilidad de la extensión $E/F$} como:
        \begin{equation*}
            [E:F]_i:=\frac{[E:F]}{[E:F]_s}
        \end{equation*}
        
        Notemos que si $\car(F)=0$, entonces $[E:F]_i=1$. Si $\car(F)=p>0$, entonces $[E:F]_i=p^t$, para algún $t\in\mathbb{Z}_{\geq0}$.
    \end{mydef}

    \begin{obs}
        Sea $E/F$ una extensión finita. Si $K$ es un campo intermedio de la extensión $E/F$, entonces
        \begin{equation*}
            [E:F]_i=[E:K]_i\cdot[K:F]_i
        \end{equation*}
        
        Si $E/F$ es una extensión finita con $\car(F)=p>0$, entonces $E/F$ es p.i. si y sólo si $[E:F]=[E:F]_i$.
    \end{obs}

    \begin{obs}
        Sea $E/F$ una extensión finita con $\car(F)=p>0$. Si $p\nmid [E:F]$ entonces $[E:F]_i=1$, es decir la extensión $E/F$ es separable.
    \end{obs}

    \begin{propo}
        Sea $F$ un campo, $\alpha_1,...,\alpha_n,\beta\in\bar{F}$ tales que $\alpha_1,...,\alpha_n$ son separables sobre $F$. Si $F$ es infinito entonces, existe $\theta\in F(\alpha_1,...,\alpha_n,\beta)$ tal que:
        \begin{equation*}
            F(\alpha_1,...,\alpha_n,\beta)=F(\theta)
        \end{equation*}
    \end{propo}

    \begin{proof}
        Procederemos por inducción sobre $n$. Para $n=1$, suponemos que tenemos la extensión $F(\alpha_1,\beta)/F$ donde $\alpha_1$ es separable sobre $F$ y $\beta$ simplemente es algebraico sobre $F$. Denotemos por $f(X)=\irr\left(\alpha_1,F,X \right)$ y $g(X)=\irr\left(\beta,F,X \right)$, con $m=\deg f$ y $k=\deg g$.

        Sean $\delta_1,...,\delta_m$ y $\beta_1,...,\beta_r$ las raíces distintas de $f(X)$ y $g(X)$, respectivamente, donde $r\leq k$.
        Consideremos las ecuaciones lineales siguientes:
        \begin{equation*}
            \delta_1X+\beta_1=\delta_iX+\beta_j
        \end{equation*}
        con $i\in\natint{2,m}$ y $j\in\natint{1,r}$. Si $\delta_1$ fuera la única raíz de $f(X)$, esto es $m=1$, entonces $f(X)=X-\delta_1\in F[X]$, luego $\alpha_1=\delta_1=\in F$. Por ende, $F(\alpha_1,\beta)=F(\beta)$. Así, basta tomar $\theta=\beta$.

        Supongamos que $\delta_1$ no es la única raíz de $f(X)$, es decir que $m\geq 2$. Hacemos $\delta_1=\alpha_1$ y $\beta_1=\beta$. Se tiene que las ecuaciones anteriores están bien determinadas.

        Elegimos un elemento $a\in F$ tal que
        \begin{equation*}
            \begin{split}
                a\delta_1+\beta_1&\neq a\delta_i+\beta_j\\
                \Rightarrow a\alpha_1+\beta&\neq a\delta_i+\beta_j\\
            \end{split}
        \end{equation*}
        para todo $i\in\natint{1,m}$ y para todo $j\in\natint{1,r}$. Tal elemento existe ya que $F$ es infinito. Definimos
        \begin{equation*}
            \theta=a\delta_1+\beta\in F(\alpha_1,\beta)
        \end{equation*}
        probemos que $F(\alpha_1,\beta)=F(\theta)$. Por lo de arriba se sigue que $F(\theta)\subseteq F(\alpha_1,\beta)$. Basta probar que $\alpha_1,\beta\in F(\theta)$. Para ello, consideremos el polinomio $h(X)=g(\theta-aX)\in F(\theta)[X]$.

        Notemos que $h(\alpha_1)=g(a\alpha_1+\beta_1-a\alpha_1)=g(\beta)=0$ y,
        \begin{equation*}
            \begin{split}
                h(\delta_i)&=g(\theta-a\delta_i)\\
                &=g((a\delta_1+\beta_1)-a\delta_i)\\
                &\neq 0,\quad\forall i\in\natint{2,m} \\
            \end{split}
        \end{equation*}
        pues, $(a\delta_1+\beta_1)-a\delta_\neq\beta_j$ para todo $j\in\natint{1,m}$, es decir que nunca puede ser alguna raíz de $g$. Así pues, $h$ y $f$ tienen solamente una raíz en común, a saber, $\alpha_1$, donde $h(X),f(X)\in F(\theta)[X]$.

        Sea $d(X)\in F(\theta)[X]$ el máximo común divisor de $h(X)$ y $f(X)$ (el cual existe pues este anillo es dominio euclideano), donde
        \begin{equation*}
            d(X)=l(X)h(X)+t(X)f(X)
        \end{equation*}
        siendo $l(X),t(X)\in F(\theta)[X]$. Notemos de la ecuación anterior que
        \begin{equation*}
            d(\alpha_1)=0
        \end{equation*}
        y, toda raíz de $d(X)$ es raíz de $f(X)$ y $h(X)$ (pues es el M.C.D.) pero, como $f(X)$ y $h(X)$ tienen a $\alpha_1$ como única raíz, entonces $d(X)$ solo tiene como raíz a $\alpha_1$. Por ende,
        \begin{equation*}
            d(X)=X-\alpha_1
        \end{equation*}
        (el coeficiente lider es $1$ ya que $f(X)$ es separable y $d(X)\divides f(X)$). Por tanto,
        \begin{equation*}
            X-\alpha_1=l(X)h(X)+t(X)f(X)\in F(\theta)[X]
        \end{equation*}
        por tanto, $\alpha_1\in F(\theta)$. En particular, como $a\in F$ entonces $a\alpha\in F(\theta)$, luego
        \begin{equation*}
            \beta=(a\alpha+\beta)-a\alpha=\theta-a\alpha\in F(\theta)
        \end{equation*}
        por tanto, $\alpha_1,\beta\in F(\theta)$. Finalmente se tiene que
        \begin{equation*}
            F(\theta)=F(\alpha_1,\beta)
        \end{equation*}

        De aquí que la proposición se cumple para $n=1$. Suponga que se cumple para algún $n\in\mathbb{N}$, probaremos que se cumple para $n+1$. En efecto, sean $\alpha_1,...,\alpha_{ n+1}\in\bar{F}$ separables sobre $F$ y $\beta\in\bar{F}$ algebraico.

        Por hipótesis de inducción, existe $\theta_1\in F(\alpha_1,...,\alpha_n,\beta)$ tal que
        \begin{equation*}
            F(\theta_1)=F(\alpha_1,...,\alpha_n,\beta)
        \end{equation*}
        y, por el caso $n=1$ existe $\theta\in F(\alpha_1,...,\alpha_{ n+1},\beta)$ tal que
        \begin{equation*}
            F(\theta)=F(\alpha_{n+1},\theta_1)
        \end{equation*}
        luego,
        \begin{equation*}
            \begin{split}
                F(\theta)&=F(\alpha_{n+1},\theta_1)\\
                &=F(\theta_1)(\alpha_{ n+1})\\
                &=F(\alpha_1,...,\alpha_n,\beta)(\alpha_{ n+1})\\
                &=F(\alpha_1,...,\alpha_{ n+1},\beta)\\
                \Rightarrow F(\theta)&=F(\alpha_1,...,\alpha_{ n+1},\beta)\\
            \end{split}
        \end{equation*}
        lo que prueba el caso $n+1$.
    \end{proof}

    \begin{cor}
        Sea $F$ un campo perfecto. Entonces, toda extensión $E/F$ finita es simple.
    \end{cor}

    \begin{proof}
        Es inmediata.
    \end{proof}

    \begin{cor}
        Si $F$ es un campo de característica cero, entonces toda extensión $E/F$ finita es simple.
    \end{cor}

    \begin{proof}
        Todo campo de característica cero es perfecto.
    \end{proof}

    \begin{exa}
        Toda extensión $E/\mathbb{Q}$ finita es simple. En particular, $\mathbb{Q}(\sqrt{2},\sqrt{3})=\mathbb{Q}(\sqrt{2}+\sqrt{3})$. En este caso, $\alpha_1=\sqrt{2}$ y $\beta=\sqrt{3}$ (en realidad da igual cual elijamos ya que cualquiera de estos dos elementos son separables sobre $\mathbb{Q}$ por ser este de característica cero). Así,
        \begin{equation*}
            \alpha_1=\delta_1=\sqrt{2}\quad\textup{y}\quad\delta_2=-\sqrt{2}
        \end{equation*}
        además,
        \begin{equation*}
            \beta=\beta_1=\sqrt{3}\quad\textup{y}\quad\beta_2=-\sqrt{3}
        \end{equation*}
        uno de los posibles $a\in\mathbb{Q}$ que nos sirven es $a=1$, ya que las ecuaciones que tenemos son:
        \begin{equation*}
            \left\{
                \begin{array}[]{rl}
                    \sqrt{2}X+\sqrt{3}&=-\sqrt{2}X+\sqrt{3}\\
                    \sqrt{2}X+\sqrt{3}&=-\sqrt{2}X-\sqrt{3}\\
                \end{array}
            \right.
        \end{equation*}
        siendo $X=a=1$ el que hace que no se cumpla la ecuación. Luego es por ello que tomamos $\theta=1\cdot\sqrt{2}+\sqrt{3}=\sqrt{2}+\sqrt{3}$.
    \end{exa}

    \begin{lema}
        Para $n\in\mathbb{N}$:
        \begin{equation*}
            n=\sum_{d| n\textup{ y }d\geq 1}\varphi(d)
        \end{equation*}
        donde $\varphi$ es la función de Euler.
    \end{lema}

    \begin{proof}
        Ejercicio.
    \end{proof}

    \begin{lema}
        Sea $G$ un grupo abeliano finito y multiplicativo tal que la ecuación $X^{m}=e$ tiene a lo más $m$ soluciones en $G$. Entonces, $G$ es grupo cíclico.
    \end{lema}

    \begin{proof}
        Ejercicio.
    \end{proof}

    \begin{propo}
        Si $F$ es un campo, entonces $F^*$ es un grupo multiplicativo y cada subgrupo finito de $F^*$ es cíclico.
    \end{propo}

    \begin{proof}
        Se sigue del lema anterior.
    \end{proof}

    \begin{theor}[\textbf{Teorema del elemento primitivo}]
        Toda extensión finita y separable de campos es simple.
    \end{theor}

    \begin{proof}
        Sea $E/F$ una extensión finita y separable. Si $F$ es un campo infinito, tenemos que $E/F$ es f.g. con elementos separables y $F$ finito. Por tanto, $E/F$ es simple.

        Si $F$ es finito, $E$ también es finito. Más aún,
        \begin{equation*}
            |E|=n|F|
        \end{equation*}
        entonces, $E^*$ es grupo multiplicativo abeliano y finito. Luego por una proposición anterior, es cíclico (visto como grup multiplicativo). Sea $\theta\in E^*$ tal que
        \begin{equation*}
            \begin{split}
                E^*&=\langle\theta \rangle\\
                &=\left\{\theta^t\Big|t\in\mathbb{N} \right\}
            \end{split}
        \end{equation*}
        luego, $E=F(\theta)$. Así, la extensión $E/F$ es simple.

    \end{proof}

    \begin{obs}
        El $\theta$ de la proposición anterior es llamado \textbf{elemento primitivo}.
    \end{obs}

    \chapter{Teoría de Galois Finita}

    \section{Conceptos Fundamentales}

    \begin{obs}
        Sea $E/F$ una extensión de campos, $\alpha\in E$, $f(X)\in E[X]$ tal que $f(\alpha)=0$ y $\cf{\varphi}{\overline{F}}{\overline{F}}$ es un $F$-homomorfismo. Entonces, $f(\varphi(\alpha))=0$
    \end{obs}

    \begin{proof}
        En efecto, notemos que
        \begin{equation*}
            \begin{split}
                0&=\varphi(0)\\
                &=\varphi(f(\alpha))\\
                &=f^{\varphi}(\varphi(\alpha))\\
                &=f(\alpha)\\
            \end{split}
        \end{equation*}
        por ser $\varphi$ un $F$-homomorfismo.
    \end{proof}

    De esta observación anterior se deduce que todo $F$-homomorfismo manda raíces en raíces.

    \begin{obs}
        Sea $F$ un campo, $f(X)\in F[X]\backslash F$. Supóngase que $\deg(f(X))=n\geq1$. Entonces, se tiene que en $\overline{F}$:
        \begin{equation*}
            f(X)=\lambda(X-\alpha_n)\cdot...\cdot(X-\alpha_n)
        \end{equation*}
        donde $\alpha_1,...,\alpha_n\in\overline{F}$ con $\lambda\in F$ el coefciente lider de $f(X)$. Luego, los coeficientes de $f(X)$ son los siguientes:
        \begin{equation*}
            \begin{split}
                a_n&=\lambda\\
                a_{ n-1}&=-\lambda\sum_{ i=1}^{n}\alpha_i\\
                a_{ n-2}&=\lambda\sum_{1\leq i_1<i_2\leq n}\alpha_{ i_1}\alpha_{ i_2}\\
                 \vdots &\\
                a_{ n-k}&=(-1)^k\lambda\sum_{1\leq i_1<i_2<\dots<i_k\leq n }\alpha_{ i_1}\alpha_{ i_2}\cdots \alpha_{i_k}\\
            \end{split}
        \end{equation*}
        con $k\in\natint{1,n}$, donde $f(X)=a_nX^n+a_{ n-1}X^{ n-1}+\cdots+a_0$. Luego, si $\cf{\varphi}{\overline{F}}{\overline{F}}$ es un $F$-automorfismo (basta con que sea $F$-homomorfismo, pues la extensión $\overline{F}/F$ es normal y en extensiones normales todo $F$-homomorfismo es un $F$-automorfismo), entonces
        \begin{equation*}
            \begin{split}
                f(X)&=f^{\varphi}(X)\\
                &=\lambda(X-\varphi(\alpha_1))\cdot...\cdot(X-\varphi(\alpha_n))\\
                &=\lambda(X-\alpha_1)\cdot...\cdot(X-\alpha_n)\\
                &=f(X)\\
            \end{split}
        \end{equation*}
        así que $\varphi$ lo que hace es permutar las raíces de $f(X)$. En particular:
        \begin{equation*}
            \left\{\varphi(\alpha_1),...,\varphi(\alpha_n) \right\}=\left\{\alpha_1,...,\alpha_n\right\}
        \end{equation*}
    \end{obs}

    \begin{obs}
        Sea $f(X)\in F[X]\backslash F$. $f(X)$ es separable si sus factores irreducibles son separables. Si $f(X)$ es irredicible, $f(X)$ es separable si y sólo si todas sus raíces son simples. 
    \end{obs}

    \begin{obs}
        Si $F$ es un campo tal que $\car(F)=0$, entonces todo polinomio en $f[X]$ es separable.
    \end{obs}

    \begin{obs}
        Si $F$ es campo tal que $\car(F)=p>0$ y si $f(X)\in F[X]$ es irredicible, entonces $f(X)=\lambda(X-\alpha_1)^{ p^e}\cdot...\cdot(X-\alpha_t)^{ p^e}$, donde $e\geq 0$ es el exponente de inseparabilidad de $f(X)$.
    \end{obs}

    \begin{obs}
        Una extensión $E/F$ es separable si y sólo si todo elemento de $\alpha\in E$ es separable sobre $F$ si y sólo si para todo $\alpha\in E$, $f(X)=\irr(\alpha,F,X)$ es separable.
    \end{obs}

    \begin{obs}
        La extensión $E/F$ es normal si y sólo si $E$ es el campo de descomposición de una familia de polinomios $\left\{f_i(X) \right\}_{i\in I}$ con coeficientes en $F$ (es decir, que $E=F(S)$ donde $S$ es la unión de los $S_i$ con $i\in I$, siendo $S_i$ el conjunto de raíces de $f_i(X)$ para todo $i\in I$).
    \end{obs}

\newcommand{\Aut}[2]{\ensuremath{\textup{Aut}_{#1}\left(#2\right)}}
\newcommand{\Gal}[1]{\ensuremath{\textup{Gal}\left(#1\right)}}


    \begin{mydef}
        Sea $F$ un campo. Se tiene que $\Aut{}{E}$ es grupo con la composición. Si $G<\Aut{}{F}$, se define el \textbf{campo fijo de $F$ por $G$} como:
        \begin{equation*}
            F^G=\left\{\alpha\in F\Big|\sigma(\alpha)=\alpha,\quad\forall \sigma\in G \right\}
        \end{equation*}
    \end{mydef}

    \begin{obs}
        En las condiciones de la definición anterior, notemos que $\emptyset\neq F^G\subseteq F$. Más aún $F^G$ es subcampo de $F$.
    \end{obs}

    \begin{proof}
        Sean $\alpha,\beta\in F^G$ y $\sigma\in G$, entonces:
        \begin{equation*}
            \sigma(\alpha-\beta)=\sigma(\alpha)-\sigma(\beta)=\alpha-\beta
        \end{equation*}
        Además, 
        \begin{equation*}
            \sigma(\alpha\cdot\beta)=\sigma(\alpha)\cdot\sigma(\beta)=\alpha\cdot\beta
        \end{equation*}
        y,
        \begin{equation*}
            1=\sigma(1)=\sigma(\alpha\cdot\alpha^{-1})=\sigma(\alpha)\cdot\sigma(\alpha^{-1})=\alpha\cdot\sigma(\alpha^{-1})
        \end{equation*}
        por ende, $\sigma(\alpha^{-1})=\alpha^{-1}$. Por ser $\sigma\in G$ arbitrario se sigue $\alpha-\beta,\alpha\cdot\beta,\alpha^{-1}\in F^G$ y, por ende, que $F^G$ es subcampo de $F$.
    \end{proof}

    \begin{mydef}
        Si $E/F$ es una extensión de campos, entonces denotamos por
        \begin{equation*}
            \Aut{F}{E}=\left\{\sigma\in\Aut{}{E}\Big|\sigma\textup{ deja fijo a }F \right\}
        \end{equation*}
        Se tiene que $\Aut{F}{E}<\Aut{}{E}$. Decimos que $E/F$ es de \textbf{Galois} si $E/F$ es normal y separable. Cuando esto ocurre, expresmos:
        \begin{equation*}
            \Gal{E/F}=\Aut{F}{E}
        \end{equation*}
        y es llamado el \textbf{grupo de Galois de $E/F$}.
    \end{mydef}

    \begin{propo}
        Sea $E/F$ una extensión de campos y $G=\Aut{F}{E}$. Entonces, las siguientes condiciones son equivalentes:
        \begin{enumerate}
            \item $E/F$ es de Galois.
            \item $E$ es el campo de descomposición sobre $F$ de una familia de polinomios separables (también sobre $F$).
            \item $E^G=F$.
        \end{enumerate}
    \end{propo}

    \begin{proof}
        Es claro que (1)$\iff$(2).

        $(1)\Rightarrow(3):$ Suponga que $E/F$ es de Galois. Notemos que se tiene la torre de campos
        \begin{equation*}
            F\subseteq E^G\subseteq E
        \end{equation*}
        Sea $\alpha\in E^G$ con $f(X)=\irr(\alpha, F,X)$. Sea $\beta\in E$ un $F$-conjugado de $\alpha$ (es decir que son raíces del mismo polinomio $f(X)$). Entonces, $\beta\in E$. Sea $\cf{\varphi}{F(\alpha)}{F(\beta)}$ un $F$-isomorfismo tal que $\varphi(\alpha)=\beta$. Extendemos a $\varphi$ a un $F$-homomorfismo $\cf{\sigma}{E}{\overline{F}}$ de $E$ en $\overline{F}$. Por normalidad, esta extensión es un $F$-automorfismo de $E$, luego
        \begin{equation*}
            \sigma\in G
        \end{equation*}
        así,
        \begin{equation*}
            \beta=\varphi(\alpha)=\sigma(\alpha)=\alpha
        \end{equation*}
        pues, $\alpha\in E^G$. Puesto que $E/F$ es separable, entonces $f(X)=X-\alpha\in F[X]$, luego $\alpha\in F$. Por tanto, $E^G=F$.

        $(3)\Rightarrow(1)$: Suponga que $E^G=F$. Sea $\alpha\in E$ y $f(X)=\irr(\alpha,F,X)$. Definamos:
        \begin{equation*}
            A=\left\{\sigma(\alpha)\Big|\sigma\in G \right\}
        \end{equation*}
        $A$ es un subconjunto de $E$ no vacío. Notemos que $A$ es un conjunto de raíces de $f(X)$ de $E$. Luego, $A$ es finito. así:
        \begin{equation*}
            |A|\leq\deg(f(X))
        \end{equation*}
        tomemos $m=|A|$ y
        \begin{equation*}
            A=\left\{\sigma_1(\alpha),...,\sigma_m(\alpha) \right\}
        \end{equation*}
        Sea $\theta\in G$. Tenemos que
        \begin{equation*}
            |\left\{(\theta\circ\sigma_1)(\alpha),...,(\theta\circ\sigma_m)(\alpha) \right\}|=m
        \end{equation*}
        (por ser biyección) con el conjunto de adentro tal que $\left\{(\theta\circ\sigma_1)(\alpha),...,(\theta\circ\sigma_m)(\alpha)\right\}\subseteq A$, así
        \begin{equation*}
            A=\left\{(\theta\circ\sigma_1)(\alpha),...,(\theta\circ\sigma_m)(\alpha)\right\}
        \end{equation*}
        es decir que $\theta$ permuta a los elementos de $A$ para todo $\theta\in G$. Definimos
        \begin{equation*}
            g(X)=(X-\sigma_1(\alpha))\cdot...\cdot(X-\sigma_m(\alpha))\in E[X]
        \end{equation*}
        Por lo anterior para cada $\theta\in G$,
        \begin{equation*}
            g^\theta(X)=g(X)
        \end{equation*}
        por tanto, $g(X)\in E^{G}[X]=F[X]$ donde $g(\alpha)=0$. Por tanto, $f(X)\divides g(X)\Rightarrow f(X)$ es separable sobre $F$ y todas las raíces de $f(X)$ están en $E$ (más aún, $g(X)=f(X)$). De esta forma se sigue $E/F$ es normal y separable, es decir, de Galois.
    \end{proof}

\end{document}