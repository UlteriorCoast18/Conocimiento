\documentclass[12pt]{report}
\usepackage[spanish]{babel}
\usepackage[utf8]{inputenc}
\usepackage{amsmath}
\usepackage{amssymb}
\usepackage{amsthm}
\usepackage{graphics}
\usepackage{subfigure}
\usepackage{lipsum}
\usepackage{array}
\usepackage{multicol}
\usepackage{enumerate}
\usepackage[framemethod=TikZ]{mdframed}
\usepackage[a4paper, margin = 1.5cm]{geometry}

%En esta parte se hacen redefiniciones de algunos comandos para que resulte agradable el verlos%

\def\proof{\paragraph{Demostración:\\}}
\def\endproof{\hfill$\square$\\}
\renewcommand{\theenumii}{\roman{enumii}}

%En esta parte se definen los comandos a usar dentro del documento para enlistar%

\newtheoremstyle{largebreak}
  {}% use the default space above
  {}% use the default space below
  {\normalfont}% body font
  {}% indent (0pt)
  {\bfseries}% header font
  {}% punctuation
  {\newline}% break after header
  {}% header spec

\theoremstyle{largebreak}

\newmdtheoremenv[
    leftmargin=0em,
    rightmargin=0em,
    innertopmargin=-2pt,
    innerbottommargin=8pt,
    hidealllines = true,
    roundcorner = 5pt,
    backgroundcolor = gray!60!red!30
]{exa}{Ejemplo}[section]

\newmdtheoremenv[
    leftmargin=0em,
    rightmargin=0em,
    innertopmargin=-2pt,
    innerbottommargin=8pt,
    hidealllines = true,
    roundcorner = 5pt,
    backgroundcolor = gray!50!blue!30
]{obs}{Observación}[section]

\newmdtheoremenv[
    leftmargin=0em,
    rightmargin=0em,
    innertopmargin=-2pt,
    innerbottommargin=8pt,
    rightline = false,
    leftline = false
]{theor}{Teorema}[section]

\newmdtheoremenv[
    leftmargin=0em,
    rightmargin=0em,
    innertopmargin=-2pt,
    innerbottommargin=8pt,
    rightline = false,
    leftline = false
]{propo}{Proposición}[section]

\newmdtheoremenv[
    leftmargin=0em,
    rightmargin=0em,
    innertopmargin=-2pt,
    innerbottommargin=8pt,
    rightline = false,
    leftline = false
]{cor}{Corolario}[section]

\newmdtheoremenv[
    leftmargin=0em,
    rightmargin=0em,
    innertopmargin=-2pt,
    innerbottommargin=8pt,
    rightline = false,
    leftline = false
]{lema}{Lema}[section]

\newmdtheoremenv[
    leftmargin=0em,
    rightmargin=0em,
    innertopmargin=-2pt,
    innerbottommargin=8pt,
    roundcorner=5pt,
    backgroundcolor = gray!30,
    hidealllines = true
]{mydef}{Definición}[section]

\newmdtheoremenv[
    leftmargin=0em,
    rightmargin=0em,
    innertopmargin=-2pt,
    innerbottommargin=8pt,
    roundcorner=5pt
]{excer}{Ejercicio}[section]

%En esta parte se colocan comandos que definen la forma en la que se van a escribir ciertas funciones%

\newcommand\abs[1]{\ensuremath{\lvert#1\rvert}}
\newcommand\divides{\ensuremath{\bigm|}}
\DeclareMathOperator{\car}{car}
\DeclareMathOperator{\irr}{irr}

%recuerda usar \clearpage para hacer un salto de página

\begin{document}
    \title{Notas Extensiones Separables AM III}
    \author{Cristo Daniel Alvarado}
    \date{Diciembre de 2023}
    \maketitle

    \tableofcontents %Con este comando se genera el índice general del libro%

    \setcounter{chapter}{3} %En esta parte lo que se hace es cambiar la enumeración del capítulo%
    
    \chapter{Extensiones Separables}
    
    \section{Resultados preeliminares}
    
    Para enunciar lo que es una extensión separable, se necesitarán demostrar algunos resultados preeliminares para enunciarlo de forma adecuada.

    \begin{propo}
        Sea $F$ un campo y $f(X)\in F[X]$ un polinomio no constante. Si
        \begin{enumerate}
            \item $\car(F)=0$, entonces $f'(X)=0$. \label{F_1}
            \item $\car(F)=p>0$, entonces $f'(X)=0$  si y sólo si $\exists g(X)\in F[X]$ tal que $f(X)=g(X^{p})$. \label{F_2}
        \end{enumerate}
    \end{propo}

    \begin{proof}
        En ambos casos, para la demostración se requiere de usar el polinomio $f'(X)$. Expresamos
        \begin{equation}
            f(X)=a_0+a_1x+\cdots+a_nx^n,\quad n\geq1,\textup{ }a_n\neq0
        \end{equation}
        
        De (\ref{F_1}): Se tiene que
        \begin{equation*}
            f'(X)=\cdots+na_nx^{n-1}
        \end{equation*}
        donde $na_n\neq0$ ya que $\car(F)=0$. Por tanto, $f'(X)\neq0$. 

        De (\ref{F_2}): Se probará el si, sólo si.

        $\Leftarrow)$: Supongamos que $\exists$ $g(X)\in F[X]$ tal que $f(X)=g(X^p)$. Expresamos a $g(X)=b_0+b_1x+\cdots+b_mx^m$, donde $b_m\neq0$. Entonces
        \begin{equation*}
            \begin{split}
                f(X)=&g(X^p)\\
                =&b_0+b_1X^p+\cdots+b_mX^{pm}\\
                \Rightarrow f'(X)=&pb_1X^{p-1}+\cdots+pmb_mX^{pm-1}\\
                =&0\cdot X^{p-1}+\cdots+0\cdot X^{pm-1}\\
                =&0\\
            \end{split}
        \end{equation*}  

        $\Rightarrow)$: Supongamos que $f'(X)=0$, donde $f'(X)=\sum_{i=1}^{m}ia_ix^{i-1}$, entonces $ia_i=0$, para todo $i=1,\cdots,m$. Si $a_i\neq0$ para algún $i$, entonces debe suceder que $i\cdot1=i=0$, por lo cual $\car(F)=p\divides i$. Luego si $a_i\neq0$, existe $m_i\in\mathbb{N}$ tal que $i=pm_i$. Escribiendo a $f(X)$ con todos sus términos no cero, se tiene que
        \begin{equation*}
            \begin{split}
                f(X)=&a_0+a_{pm_1}X^{pm_1}+\cdots+a_{pm_n}X^{pm_n}\\
                =&a_0+a_{pm_1}(X^p)^{m_1}+\cdots+a_{pm_n}(X^p)^{m_n}\\
                =&g(X^p)
            \end{split}
        \end{equation*}
        donde $g(X)=a_0+a_{pm_1}X+\cdots+a_{pm_n}X^{m_n}$, siendo $a_{pm_n}\neq0$, pues $f(X)\neq0$
    \end{proof}

    De este teorema anterior y de un teorema del capítlo pasado, se deduce de forma inmediata el siguiente corolario:

    \begin{cor}
        Sea $F$ un campo y $f(X)\in F[X]$ un polinomio irreducible. Si
        \begin{enumerate}
            \item $\car(F)=0$, entonces todas las raíces de $f(X)$ son simples.
            \item $\car(F)=p>0$, entonces $f(X)$ tiene una raíz simple si y sólo si, $\exists g(X)\in F[X]$ tal que $f(X)=g(X^{p})$.
        \end{enumerate}
    \end{cor}

    El siguiente teorema tiene como objetivo caracterizar las extensiones separables, enunciando un resultado importante para su definición.

    \begin{theor}
        Sea $F$ un campo con $\car(F)=p>0$. Sea $f(X)\in F[X]$ un polinomio irreducible, y $e\in\mathbb{N}^{*}$ tal que $f(X)\in F[x^{p^e}]$, pero $f(X)\notin F[x^{p^{e+1}}]$. Sea $\Psi(X)\in F[X]$ el polinomio tal que $f(X)=\Psi(X^{p^e})$. Entonces
        \begin{enumerate}
            \item $\Psi(X)$ es un polinomio irredicible en $F[X]$.\label{T_1_1_1}
            \item Todas las raíces de $\Psi(X)$ son simples.\label{T_1_1_2}
            \item Todas las raíces de $f(X)$ tienen la misma multiplicidad, a saber, $p^e$.\label{T_1_1_3}
            \item Si $m=\deg(\Psi)$, entonces $\deg(f)=p^em$.\label{T_1_1_4}
        \end{enumerate}
    \end{theor}
    \begin{proof}
        De (1): Supongamos que $\Psi(X)$ es descomponible, entonces existen $g(X),h(X)\in F[X]$ con grados $\geq1$ tales que
        \begin{equation*}
            \begin{split}
                \Psi(X)=&g(X)h(X)\\
                \Rightarrow f(X)=&g(X^p)h(X^p)\\
                =&g_1(X)h_1(X)\\
            \end{split}
        \end{equation*}
        donde $g_1(X)=g(X^p)$ y $h_1(X)=h(X^p)$ con grados $\geq1$, lo cual implicaría que $f(X)$ es reducible. Luego $\Psi(X)$ tiene que ser irredicible.

        De (2): Supongamos que $\Psi(X)$ admite una raíz multiple, entonces $\exists$ $g(X)\in F[X]$ tal que $\Psi(X)=g(X^p)$. Así
        \begin{equation*}
            \begin{split}
                f(X)=&\Psi(X^{p^{e}})\\
                =&g(X^{p^{e+1}})\\
                \in&F[x^{p^{e+1}}]
            \end{split}
        \end{equation*}
        lo cual es una contradicción. Por lo tanto $\Psi(X)$ debe tenera todas sus raíces simples.

        De (3): Sea $m=\deg(\Psi)$. Sean $\beta_1,\cdots,\beta_m\in\bar{F}$ todas las raíces de $\Psi(X)$ en alguna cerradura algebraica de $F$. Se tiene entonces que
        \begin{equation*}
            \begin{split}
                \Psi(X)=&a\left(x-\beta_1\right)\cdots\left(x-\beta_m\right)\\
                \Rightarrow f(X)=&\Psi(X^{p^e})\\
                =&a\left(x^{p^e}-\beta_1\right)\cdots\left(x^{p^e}-\beta_m\right)\\
            \end{split}
        \end{equation*}
        Donde $a\in F$ es alguna constante. Ahora, para cada $i=1,\cdots,m$ sea $\alpha_i\in\bar{F}$ una raíz del polinomio $X^{p^e}-\beta_i=0$, esto es $\beta_i=\alpha^{p^e}$. Notemos que si $i\neq j$, debe suceder que $\alpha_i\neq\alpha_j$. Por tanto
        \begin{equation*}
            asd
        \end{equation*}

        De (4): Es inmediata.
    \end{proof}

    Se deduce de forma inmediata el siguiente corolario.

    \begin{cor}
        Sea $F$ campo y $f(X)\in F[X]$ un polinomio irredicible. Entonces todas las raíces de $f(X)$ tienen la misma multiplicidad. Si $\car(F)=0$, la multiplicidad de estas raíces es $1$, y si $\car(F)=p>0$, tienen multiplicidad $p^e$, para algún $e\in\mathbb{N}^{*}$ (este $e$ se obtiene del teorema anterior). 
    \end{cor}

    \section{Extensiones separables}

    Ahora estamos en las condiciones de enunciar la definición de separabilidad.

    \begin{mydef}
        De acuerdo con las notaciones del teorema anterior y de su demostracion, tenemos que el número $\deg(\Psi)$ es llamado \textbf{el grado de separabilidad de $f$}, y al entero no negativo $e$ es llamado \textbf{el grado de inseparabilidad de $f$}.
    \end{mydef}

    En otras palabras, podemos ver que el grado de separabilidad de $f$ es el número de raíces distintas de $f$.

    \begin{mydef}
        Sea $F$ un campo y $\bar{F}$ una cerradura algebraica de $F$. Si $\alpha\in \bar{F}$ y $f(X)=\irr(\alpha, F, X)$, entonces se define \textbf{el grado de separabilidad de $\alpha$}, como el grado de separabilidad de $f$, y al exponente $e$ de inseparabilidad de $f$, será el \textbf{exponente de inseparabilidad de $\alpha$}.
    \end{mydef}

    En el caso en que $\car(F)=0$, el exponente y grado de inseparabilidad de $f$ y $\alpha$ no tienen sentido en estar definidos, pues en ambos casos su valor siempre será de $1$.

    En cualquier caso, si $\alpha\in \bar{F}$ se denota al grado de separabilidad de $\alpha$ como
    \begin{equation}
        \left[F(\alpha):F\right]_s
    \end{equation}

    En el caso de que $\car(F)=0$, se tiene que
    \begin{equation}
        \left[F(\alpha):F\right]_s=\left[F(\alpha):F\right]=\deg(\irr(\alpha, F, X))
    \end{equation}
    y, si $\car(F)=p>0$, entonces
    \begin{equation}
        \left[F(\alpha):F\right]_s=\frac{\left[F(\alpha):F\right]}{p^e}
    \end{equation}

    \begin{propo}
        Sea $F$ un campo, $\bar{F}$ una cerradora algebraica de $F$ y $\alpha\in\bar{F}$. Entonces, $\left[F(\alpha):F\right]_s=N$, donde $N\in\mathbb{N}$ es el número de $F$-homomorfismos de $F(\alpha)$ en $\bar{F}$.
    \end{propo}
    
    \begin{proof}
        
    \end{proof}

    \begin{mydef}
        Sea $E/F$ una extensión algebraica. Se define \textbf{el grado de separabilidad de $E$ sobre $F$} como la cardinalidad del conjunto de $F$-homomorfismos que van de $E$ en $\bar{F}$, donde $\bar{F}$ es una cerradura algebraica de $F$ que contiene a $E$. Tal cardinal es denotado por $\left[E:F\right]_s$.
    \end{mydef}

    De resultados de capítulo anterior, se deduce de forma inmediata el siguiente teorema.

    \begin{theor}
        Sea $E/F$ una extensión finita y $K$ un campo intermedio de la extensión $E/F$. Entonces
        \begin{equation}
            \left[E:F\right]_s=\left[E:K\right]_s \left[K:F\right]_s
        \end{equation}
    \end{theor}

    \begin{mydef}
        Sea $F$ un campo y $\alpha\in\bar{F}$. Decimos que \textbf{$\alpha$ es separable sobre $F$} si $\left[F(\alpha):F\right]_s=\left[F(\alpha):F\right]$. Si $E/F$ es una extensión algebraica, entonces se dice que \textbf{$E/F$ es separable} o \textbf{$E$ es separable sobre $F$}, si todo elemento de $E$ es separable sobre $F$.
    \end{mydef}

    Veremos ahora algunas caracterizaciones de las extensiones separables.

    \begin{obs}
        Sea $F$ camop y $\bar{F}$ cerradura algebraica de $F$.
        \begin{enumerate}
            \item Si $\alpha\in\bar{F}$, entonces $\alpha$ es separable sobre $F$ si y sólo si $f(X)=\irr(\alpha, F, X)$ es tal que todas sus raíces son simples. Cuando esto ocurra decimos que \textbf{$f(X)$ es separable sobre $F$}.
            \item Si $g(X)\in F[X]$, decimos que \textbf{$g(X)$ es separable sobre $F$} si todos sus factores irreducibles son separables sobre $F$.
        \end{enumerate}
    \end{obs}

    \begin{propo}
        Sea $E/F$ una extensión finita con $\car(F)=p>0$. Entonces existe un elemento $t\in\mathbb{N}^{*}$ tal que
        \begin{equation}
            \left[E:F\right]=p^t\left[E:F\right]_s
        \end{equation}
        En particular, si $p\nmid \left[E:F\right]$, enotnces $\left[E:F\right]=\left[E:F\right]_s$.
    \end{propo}

    \begin{obs}
        Si $E/F$ es una extensión finita y $\car(F)=0$, entonces $\left[E:F\right]=\left[E:F\right]_s$.
    \end{obs}

    \begin{propo}
        Sea $E/F$ una extension de campos con $\car(F)=p>0$ y $\alpha\in E$ algebraico sobre $F$. Sea $e$ el exponente de inseparabilidad de $\alpha$ sobre $F$. Entonces
        \begin{enumerate}
            \item $\alpha^{p^e}$ es separable sobre $F$.
            \item Las siguientes condiciones son equivalentes:
            \begin{enumerate}
                \item $\alpha$ es separable sobre $F$.
                \item $\left[F(\alpha):F\right]_s=\left[F(\alpha):F\right]$.
                \item $e=0$.
                \item $F(\alpha)=F(\alpha^p)$.
            \end{enumerate}
        \end{enumerate}
    \end{propo}

    \begin{proof}
        De 
    \end{proof}

    \begin{propo}
        Sea $E/F$ una extensión finita. Entonces $E/F$ es separable si y sólo si $\left[E:F\right]_s=\left[E:F\right]$.
    \end{propo}

    \begin{proof}
        
    \end{proof}

    \begin{obs}
        Sea $F\subseteq K \subseteq E$ una torre de campos y $\alpha\in E$ separable sobre $F$. Entonces $\alpha$ es separable sobre $K$. Más generalmente, sean $E/F$ y $K/F$ extensiones de campos y $\alpha\in E$ separable sobre $F$. Si $\alpha$ es elemento de un campo $L$ extensión de $K$, entonces $\alpha$ es separable sobre $K$.
    \end{obs}

    \begin{propo}
        Sea $E/F$ una extensión de campos y $S\subseteq E$ tal que $E=F(S)$. Sea.
        \begin{equation}
            K = \left\{\alpha\in E | \alpha \textup{ es separable sobre }F\right\}\label{equation:k_sep_cer}
        \end{equation}
        Entonces
        \begin{enumerate}
            \item $K$ es un subcampo intermedio de la extensión $E/F$.
            \item $E/F$ es separable si y sólo si $\alpha$ es separable sobre $F$, para todo $\alpha\in S$.
        \end{enumerate}
    \end{propo}

    \begin{proof}
        De (1): Probaremos que $K$ es campo y que $F\subseteq K\subseteq E$. En efecto, sea $\alpha\in F$, se tiene que $\alpha$ es algebraico sobre $F$, con polinomio irreducible $f(X)=X-\alpha$, el cual tiene todas sus raíces distintas, por lo cual $\alpha$ es separable sobre $F$. Entonces $F\subseteq K\subseteq E$.
        Sean ahora $\alpha,\beta\in K\neq\emptyset$, pues $F\subseteq K$. Consideremos el campo intermedio de la extensión $E/F$, $F(\alpha,\beta)$. Se tiene entonces la torre de campos
        \begin{equation*}
            F\subseteq F(\alpha)\subseteq F(\alpha, \beta)\subseteq E
        \end{equation*}
        Como $\beta$ es separable sobre $F$, lo es sobre $F(\alpha)$, luego como el grado de separabilidad es multiplicativo, se tiene que
        \begin{equation*}
            \begin{split}
                \left[F(\alpha,\beta):F\right]_s=&\left[F(\alpha,\beta):F(\alpha)\right]_s\left[F(\alpha):F\right]_s\\
                =&\left[F(\alpha,\beta):F(\alpha)\right]\left[F(\alpha):F\right]\\
                =&\left[F(\alpha,\beta):F\right]\\
            \end{split}
        \end{equation*}
        por lo cual, la extensión $F(\alpha,\beta)/F$ es separable, luego los elementos $\alpha-\beta, \alpha\beta, \alpha^{-1}\in F(\alpha,\beta)$ son separables sobre $F$. Por tanto, $K$ es campo y por lo anterior, es subcampo intermedio de la extensión $E/F$.
        
        De (2): Veamos que

        $\Rightarrow$): Es inmediata, pues si $E/F$ es separable todo elemento de $E$ es separable sobre $F$. En particular todo elemento de $S$ es separable sobre $F$.

        $\Leftarrow$): Supongamos que $\alpha$ es separable sobre $F$, para todo $\alpha\in S$. Por (1) se tiene que $S\subseteq K$ y $F\subseteq K$, pero como $K$ es subcampo de $E$, se tiene que $F(S)\subseteq K$, por lo cual $F(S)=E=K$. Así, todos los elementos de $E$ son separables sobre $F$, es decir $E/F$ es una extensión separable. 
    \end{proof}

    \begin{mydef}
        El campo $K$ de la definición (\ref{equation:k_sep_cer}) es llamado \textbf{la cerradura separable} o de la extensión $E/F$ o simplemente de $E/F$, o de $F$ en $E$.

        Si consideramos la extensión $\bar{F}/F$, entonces la cerradura separable de $F$ en $\bar{F}$ simplemente se dice es la \textbf{cerradura separable de F}.
    \end{mydef}
    
    \begin{obs}
        Si $E/F$ es una extensión algebraica de tal manera que $E\subseteq \bar{F}$, entonces la cerradura separable de $F$ en $E$, $K$, es la intersección de la cerradura separable de $F$ con $E$.
    \end{obs}

    \begin{obs}
        En la literatura no existe notación establecida para referirse a la cerradura normal. En este momento nosotros acordaremos la siguiente. Sobre la extensión $E/F$, se denotará a la cerradura separable de $F$ en $E$ por:
        \begin{equation*}
            F_{S,E/F}\quad\textup{o}\quad F_{S,F}^{E}
        \end{equation*} 
        Cuando la extensión es $\bar{F}/F$ será
        \begin{equation*}
            F_S
        \end{equation*}
        y a veces a la cerradura algebraica se le denota por $\bar{F}=F^{a}$.
    \end{obs}

    \begin{propo}
        Sea $E/F$ una extensión normal \& $F_S$ la cerrradura separable de $E/F$. Entonces, la extensión $F_S/F$ es normal.
    \end{propo}

    \begin{proof}
        Sea $\alpha\in F_S$ con $f(X)=\irr(\alpha, F, X)$, y $\beta\in \bar{F}$ tal que $\alpha$ y $\beta$ son $F$-conjugados, es decir que ambos son raíces del polinomio $f(X)$.
        Como la extensión $E/F$ es normal, entonces $\beta\in E$, donde $\irr(\beta, F, X)=f(X)$ es separable sobre $F$, pues $\alpha$ es separable sobre $F$, es decir, $\beta$ es separable sobre $F$.
        Luego $\beta\in F_S$.
        Por tanto, la extensión $F_S/F$ es normal.
    \end{proof}

    \begin{obs}
        Si $F$ es campo, la extensión $\bar{F}/F$ es normal, por lo cual las extensiones $\bar{F}/F_S$ y $F_S/F$ son ambas normales (siendo $F_S$ la cerradura separable de $F$). 
    \end{obs}

    \begin{propo}
        Sea $E/F$ una extensión finita. y $F_S$ la cerradura separable de $F$ en $E$. Entonces,
        \begin{equation}
            \left[F_S:F\right]=\left[E:F\right]_s
        \end{equation}
    \end{propo}

    \begin{proof}
        Tenemos dos casos:
        \begin{itemize}
            \item Si $\car(F)=0$, entonces la extensión $E/F$ es separable y por tanto $F_S=E$. Por tanto
            \begin{equation*}
                \begin{split}
                    \left[F_S:F\right]=&\left[E:F\right]\\
                    =&\left[E:F\right]_s\\
                \end{split}
            \end{equation*}
            \item Si $\car(F)=p>0$. Tenemos que
            \begin{equation*}
                \begin{split}
                    \left[E:F\right]_S=&\left[E:F_S\right]_S\left[F_S:F\right]_S\\
                    =&\left[E:F_S\right]_S\left[F_S:F\right]\\
                \end{split}
            \end{equation*}
            Para probar el resultado, basta con probar que $\left[E:F_S\right]_S=1$. Recordemos que $\left[E:F_S\right]_S$ es el cardinal de $F_S$-homomorfismos de $E$ en $\bar{F}=\bar{F_S}$. Sea entonces $f:E\rightarrow \bar{F}$ un $F_s$-homomorfismo.
            Sea $\alpha\in E$. Si $\alpha\in F_S$m entonces $f(\alpha)=\alpha$. Si $\alpha\notin F_S$, se tiene por definción de $F_S$ que $\alpha$ no es separable sobre $F$. 
            
            Sea $f(X)=\irr(\alpha, F, X)$, y tomemos $e\in\mathbb{N}^{*}$ su exponente de inseparabilidad. 
            Por un resultado anterior sucede que $\alpha^{p^e}$ es separable sobre $F$, es decir $\alpha^{p^e}\in F_S$. Luego,
            \begin{equation*}
                \begin{split}
                    f(\alpha^{p^e})=&\alpha^{p^e}\\
                    \Rightarrow \left(\alpha-f(\alpha)\right)^{p^e}=&\alpha^{p^e}-f(\alpha)^{p^e}\\
                    =&0\\
                    \Rightarrow f(\alpha)-\alpha=&0\\
                    \Rightarrow f(\alpha)=&\alpha\\
                \end{split}
            \end{equation*}
            Es decir, $f=\textup{id}_E$. Por tanto $\left[E:F_S\right]=1$. Así por la ecuación anterior
            \begin{equation*}
                \left[E:F\right]_S\left[F_S:F\right]
            \end{equation*}
        \end{itemize}
    \end{proof}

    \begin{theor}
        La clase de extensiones separables es una clase distinguida.
    \end{theor}

    \begin{proof}
        De (a): Sea $F\subseteq K\subseteq E$ una torre de campos. Probaremos que $E/F$ es separable si, y sólo si $E/K$ y $K/F$ son separables.

        $\Rightarrow$): Supongamos que $E/F$ es separable. Sabemos ya que $E/K$ es separable. Pero, por otro lado, es claro que la extensión $K/F$ es separable.

        $\Leftarrow$): Supongamos que las extensiones $E/K$ y $K/F$ son separables. Sea $\alpha\in E$ arbitrario y tomemos $f(X)=\irr(\alpha, K, X)$, digamos
        \begin{equation*}
            f(X)=a_0+a_1X+\cdots+a_{m-1}X^{m-1}+X^m\in K[X]
        \end{equation*}
        Tenemos que $f(X)$ es separable sobre $K$, es decir todas las raíces de $f(X)$ son simples. Consideremos la torre de campos:
        \begin{equation*}
            F\subseteq F(a_0,a_1,\dots,a_{m-1})\subseteq F(a_0,a_1,\dots,a_{m-1},\alpha)
        \end{equation*}
        donde $F(a_0,a_1,\dots,a_{m-1})/F$ es finita y separable, al igual que $F(a_0,a_1,\dots,a_{m-1},\alpha)/F(a_0,a_1,\dots,a_{m-1})$. Notemos que $f(X)=\irr(\alpha, F(a_0,a_1,\dots,a_{m-1}), X)$. Entonces
        \begin{equation*}
            \begin{split}
                \left[F(a_0,a_1,\dots,a_{m-1},\alpha):F\right]=&\left[F(a_0,a_1,\dots,a_{m-1},\alpha):Fa_0,a_1,\dots,a_{m-1}\right]\left[F(a_0,a_1,\dots,a_{m-1}):F\right]\\
                =&\left[F(a_0,a_1,\dots,a_{m-1},\alpha):Fa_0,a_1,\dots,a_{m-1}\right]_s\left[F(a_0,a_1,\dots,a_{m-1}):F\right]_s\\
                =&\left[F(a_0,a_1,\dots,a_{m-1},\alpha):F\right]_s\\
            \end{split}
        \end{equation*}
        es decir, $F(a_0,a_1,\dots,a_{m-1},\alpha)/F$ es una extensión separable, en particular se tiene que $\alpha$ es separable sobre $F$. Por ser el $\alpha$ arbitrario en $E$, se sigue que $E/F$ es una extensión separable.

        De (b): 
    \end{proof}

    %Falta de campos acabar esto y más cosas

    \begin{theor}
        Todo campo finito es perfecto.
    \end{theor}

    \section{Extesiones puramente inseparables}

    \begin{mydef}
        Sea $E/F$ una extensión de campos con $\car(F)=p>0$ y $\alpha\in E$. Decimos que $\alpha$ \textbf{es puramente inseparable} si existe $t\in \mathbb{Z}$, $t\geq 0$ tal que $\alpha^{p^t}\in F$.
        La extensión $E/F$ \textbf{es p.i.} si todo elemento de $E$ es p.i. sobre $F$.
    \end{mydef}

    \begin{obs}
        Si $E/F$ es una extensión de campos, entonces todos los elementos de $F$ son p.i. (separables) sobre $F$. 
    \end{obs}

    \begin{propo}
        Sea $E/F$ una extensión de campos con $\car(F)=p>0$. Sea
        \begin{equation*}
            K:=\left\{\alpha\in E|\alpha\textup{ es puramente inseparable sobre }F\right\}
        \end{equation*}
        (por la observación anterior, $K\neq \emptyset$). Entonces, $K$ es subcampo de $E$ que contiene a $F$.
    \end{propo}

    \begin{proof}
        Es claro que $K\neq \emptyset$ y $F\subseteq K \subseteq E$. Sean $\alpha, \beta\in K$, y $t_1,t_2\in \mathbb{Z}_{\geq0}$ tales que
        \begin{equation*}
            \alpha^{p^t_1},\beta^{p^t_2}\in F
        \end{equation*}
        Sea $t=\max\left\{t_1,t_2\right\}$. Por lo cual $\alpha^{p^t},\beta^{p^t}\in F$, así
        \begin{equation*}
            \begin{split}
                \left(\alpha-\beta\right)^{p^t}&=\alpha^{p^t}-\beta^{p^t}\in F\\
                \left(\alpha\beta\right)^{p^t}&=\alpha^{p^t}\beta^{p^t}\in F\\
                \left(\alpha^{-1}\right)^{p^t}&=\left(\alpha^{p^t}\right)^{-1}\in F\textup{ donde $\alpha\neq0$}
            \end{split}
        \end{equation*}
        por lo cual $K$ es campo intermedio de la extensión $E/F$.
    \end{proof}

    \begin{propo}
        Sea $E/F$ una extensión algebraica, con $\car(F)=p>0$. Sea $S\subseteq E$ tal que $E=F(S)$. Entonces, las siguientes condiciones son equivalentes:
        
        \begin{enumerate}
            \item $E/F$ es puramente inseparable.
            \item Todo elemento de $S$ es puramente inseparable sobre $F$.
            \item Los elementos de $E$ que son puramente inseparables y separables sobre $F$ son exactamente los de $F$.
            \item Si $\phi:E\rightarrow \Bar{F}$ es un $F$-homomorfismo, entonces $\phi(\alpha)=\alpha$, para todo $\alpha\in E$.
        \end{enumerate}
    \end{propo}

    \begin{proof}
        $(1)\Rightarrow(2)$: Es inmediato.

        $(2)\Rightarrow(3)$: Sea $\alpha\in E$ tal que es puramente inseparable sobre $F$ y separable sobre $F$, y $e\in\mathbb{Z}_{\geq0}$ el exponente de inseparablilidad de $\alpha$ sobre $F$.

        Tenemos que $\alpha^{p^e}$ es separable sobre $F$ (por una proposición anterior). Por otro lado, sea $t\in\mathbb{Z}_{\geq0}$ tal que $\alpha^{p^e}\in F$. Podemos suponer que $t\geq e$. Luego, $\alpha$ es raíz del polinomio $g(X)=X^{p^t}-\alpha^{p^t}=(X-\alpha)^{p^t}$, por lo cual $f(X)|g(X)$, donde $f(X)=\irr(\alpha,F,X)$.

        Así $f(x)=(X-\alpha)^{p^t}$. Como $\alpha$ es separable sobre $F$, se tiene que $e=0$, es decir que $f(X)=X-\alpha\in F[X]$, en particular, $\alpha\in F$.

        $(3)\Rightarrow(4)$: Sea $\phi:E\rightarrow\Bar{F}$ un $F$-homomorfismo arbitrario, y $\alpha\in E$, con $e\in\mathbb{Z}_{\geq0}$ su exponente de inseparabilidad. Sabemos que $\alpha^{p^e}$ es separable sobre $F$. Por hipótesis, $\alpha^{p^e}\in F$. Por lo cual
        \begin{equation*}
            \begin{split}
                \phi(\alpha^{p^e})&=\alpha^{p^e}\\
                \Rightarrow (\phi(\alpha)-\alpha)^{p^e}=\left(\phi\alpha^{p^e}\right)-\alpha^{p^e}&=0\\
                \Rightarrow \phi(\alpha)&=\alpha\\
            \end{split}
        \end{equation*} 

        $(4)\Rightarrow(1)$: sea $\alpha\in E$ arbitrario. Probaremos que existe $t\in\mathbb{Z}_{\geq0}$ tal que $\alpha^{p^t}\in F$. Sea $\beta\in\bar{F}$ un $F$-conjugado de $\alpha$. Sabemos que existe un $F$-isomorfismo $\psi:F(\alpha)\rightarrow F(\beta)$ tal que $\psi(\alpha)=\beta$. Extendemos $\psi$ a un $F$-homomorfismo $\phi:E\rightarrow \Bar{F}$. Por hipótesis, se tiene que $\phi(\gamma)=\gamma$, para todo $\gamma\in E$, en particular $\beta = \psi(\alpha)=\phi(\alpha)=\alpha$. Luego, si $e\in\mathbb{Z}_{\geq0}$ es el exponente de inseparabilidad de $\alpha$, entonces
        \begin{equation*}
            \begin{split}
                f(X)=&\irr(\alpha,F,X)\\
                =&(X-\alpha)^{p^e}\\
                =&X^{p^e}-\alpha^{p^e}\in F[X]\\
            \end{split}
        \end{equation*}
        por tanto $\alpha^{p^e}\in F$. Luego $\alpha$ es p.i. sobre $F$.
    \end{proof}

    \begin{mydef}
        Si $E/F$ es una extensión algebraica con $\car(F)=p>0$, entonces la \textbf{cerradura puramente inseparable} de la extensión $E/F$ es el campo intermedio de todos los elementos $\alpha\in R$ tal que son puramente inseparables sobre $F$.
    \end{mydef}

    \begin{obs}
        Si $E/F$ es finita, entonces $E/F$ es p.i. $\iff$ $[E:F]_S=1$.
    \end{obs}

\end{document}