\documentclass[12pt]{report}
\usepackage[spanish]{babel}
\usepackage[utf8]{inputenc}
\usepackage{amsmath}
\usepackage{amssymb}
\usepackage{amsthm}
\usepackage{graphics}
\usepackage{subfigure}
\usepackage{lipsum}
\usepackage{array}
\usepackage{multicol}
\usepackage{enumerate}
\usepackage[framemethod=TikZ]{mdframed}
\usepackage[a4paper, margin = 1.5cm]{geometry}

%En esta parte se hacen redefiniciones de algunos comandos para que resulte agradable el verlos%

\renewcommand{\theenumii}{\roman{enumii}}

\def\proof{\paragraph{Demostración:\\}}
\def\endproof{\hfill$\blacksquare$}

\def\sol{\paragraph{Solución:\\}}
\def\endsol{\hfill$\square$}

%En esta parte se definen los comandos a usar dentro del documento para enlistar%

\newtheoremstyle{largebreak}
  {}% use the default space above
  {}% use the default space below
  {\normalfont}% body font
  {}% indent (0pt)
  {\bfseries}% header font
  {}% punctuation
  {\newline}% break after header
  {}% header spec

\theoremstyle{largebreak}

\newmdtheoremenv[
    leftmargin=0em,
    rightmargin=0em,
    innertopmargin=-2pt,
    innerbottommargin=8pt,
    hidealllines = true,
    roundcorner = 5pt,
    backgroundcolor = gray!60!red!30
]{exa}{Ejemplo}[section]

\newmdtheoremenv[
    leftmargin=0em,
    rightmargin=0em,
    innertopmargin=-2pt,
    innerbottommargin=8pt,
    hidealllines = true,
    roundcorner = 5pt,
    backgroundcolor = gray!50!blue!30
]{obs}{Observación}[section]

\newmdtheoremenv[
    leftmargin=0em,
    rightmargin=0em,
    innertopmargin=-2pt,
    innerbottommargin=8pt,
    rightline = false,
    leftline = false
]{theor}{Teorema}[section]

\newmdtheoremenv[
    leftmargin=0em,
    rightmargin=0em,
    innertopmargin=-2pt,
    innerbottommargin=8pt,
    rightline = false,
    leftline = false
]{propo}{Proposición}[section]

\newmdtheoremenv[
    leftmargin=0em,
    rightmargin=0em,
    innertopmargin=-2pt,
    innerbottommargin=8pt,
    rightline = false,
    leftline = false
]{cor}{Corolario}[section]

\newmdtheoremenv[
    leftmargin=0em,
    rightmargin=0em,
    innertopmargin=-2pt,
    innerbottommargin=8pt,
    rightline = false,
    leftline = false
]{lema}{Lema}[section]

\newmdtheoremenv[
    leftmargin=0em,
    rightmargin=0em,
    innertopmargin=-2pt,
    innerbottommargin=8pt,
    roundcorner=5pt,
    backgroundcolor = gray!30,
    hidealllines = true
]{mydef}{Definición}[section]

\newmdtheoremenv[
    leftmargin=0em,
    rightmargin=0em,
    innertopmargin=-2pt,
    innerbottommargin=8pt,
    roundcorner=5pt
]{excer}{Ejercicio}[section]

%En esta parte se colocan comandos que definen la forma en la que se van a escribir ciertas funciones%

\newcommand\abs[1]{\ensuremath{\left|#1\right|}}
\newcommand\divides{\ensuremath{\bigm|}}
\newcommand\cf[3]{\ensuremath{#1:#2\rightarrow#3}}

%recuerda usar \clearpage para hacer un salto de página

\begin{document}
    \setlength{\parskip}{5pt} % Añade 5 puntos de espacio entre párrafos
    \setlength{\parindent}{12pt} % Pone la sangría como me gusta
    \title{Curso de Variable Compleja}
    \author{Cristo Daniel Alvarado}
    \maketitle

    \tableofcontents %Con este comando se genera el índice general del libro%

    \chapter{Introducción}
    
    \section{Fundamentos}

    El objetivo principal de la teoría de las funciones analíticas es el análisis de funciones que localmente pueden ser descritas en términos de una serie de potencias convergente, dispuesta como:
    \begin{equation}
        \begin{split}
            f(x)&=a_0+a_1(x-x_0)+a_2(x-x_0)^2+...+a_n(x-x_0)^n+...\\
            &=\sum_{ k=0}^\infty a_k(x-x_0)^k,\quad\forall x\in]x_0-\delta,x_0+\delta[
        \end{split}
    \end{equation}
    siendo $\cf{f}{I}{\mathbb{R}}$ con $I$ un intervalo, $x_0\in I$ y $\delta>0$ tal que $]x_0-\delta,x_0+\delta[\subseteq]a,b[$. Cuando una funcion de este tipo puede ser descrita de la forma anterior para algún par $x_0$ y $\delta$, decimos en este caso que \textbf{$f$ es analítica en $x_0$}.

    En el caso que $I$ sea un intervalo abierto y $f$ sea analítica en $x_0$ para todo $x_0\in I$, decimos que \textbf{$f$ es analítica en $I$}.

    \begin{exa}
        Las funciones $x\mapsto P(x)$, $x\mapsto e^x$, $x\mapsto \sin x$ y $x\mapsto \cos x$ son analíticas en $\mathbb{R}$
    \end{exa}

    Debido a que como resultado de efectuar operaciones algebraicas y analíticas (suma, resta, multiplicación, división, integración y derivación) sobre series de potencias resulta nuevamente en una serie de potencias convergente, es de gran interés conocer las propiedades de estas funciones (más que nada debido a las ecuaciones diferenciales). Esto motiva el estudio particular de este tipo  de funciones.

    A pesar de lo amplia que es esta clase de funciones, ésta solamente forma una parte regular de las funciones \textit{infinitamente diferenciables}.

    \begin{propo}
        Sea $\cf{f}{]x_0-r,x_0+r[}{\mathbb{R}}$ una función, siendo $r>0$ y $x_0\in\mathbb{R}$. Entonces, $f$ es analítica en $x_0$ si y sólo si se satisfacen las condiciones siguientes:
        \begin{enumerate}
            \item $f$ tiene derivadas de todos los órdenes en un entorno de $x_0$.
            \item Existen $\delta,M>0$ tales que para todo $x\in]x_0-\delta,x_0+\delta[$ y para todo $k\in\mathbb{N}$ se cumple:
            \begin{equation*}
                \abs{f^{(k)}(x)}<M\frac{k!}{\delta^k}
            \end{equation*}
        \end{enumerate}
    \end{propo}

    \begin{proof}
        $\Rightarrow):$ Suponga que $f$ es analítica en $x_0$, entonces existen $a_0,a_1,...,a_n,...\in\mathbb{R}$ y $\rho>0$ tal que
        \begin{equation*}
            f(x)=a_0+a_1(x-x_0)+a_2(x-x_0)^2+...
        \end{equation*}
        para todo $x\in]x_0-\rho,x_0+\rho[$ (note que $\rho<r$). Se sabe por resultados de análisis real que $f$ tiene derivadas de todos los órdenes en $]x_0-\rho,x_0+\rho[$ y, en particular para todo $k\in\mathbb{N}$ se tiene que
        \begin{equation*}
            f^{(k)}(x)=k!a_k+\frac{(k+1)!}{1!}a_{ k+1}(x-x_0)+\frac{(k+2)!}{2!}a_{ k+2}(x-x_0)^2+...+\frac{(k+n)!}{n!}a_{ k+n}(x-x_0)^n+...
        \end{equation*}
        para todo $x\in]x_0-\rho,x_0+\rho[$. Fijemos $k\in\mathbb{N}$ y tomemos $\delta>0$ tal que $0<2\delta<\rho$. Si $x=x_0+2\delta$, entonces la serie anterior convergerá y, por ende en el límite debe suceder que
        \begin{equation*}
            \begin{split}
                \lim_{ n\rightarrow\infty}\frac{(k+n)!}{n!}a_{ k+n}(x-x_0)^n&=0\\
                \iff \lim_{ n\rightarrow\infty}a_{ k+n}(x_0+2\delta-x_0)^n&=0\\
                \iff a^k\lim_{ n\rightarrow\infty}a_{n}(2\delta)^n&=0\\
                \iff \lim_{ n\rightarrow\infty}a_{n}(2\delta)^n&=0\\
            \end{split}
        \end{equation*}
        (pues, $\lim_{ n\rightarrow\infty}\frac{(k+n)!}{n!}=1$). En particular, de lo anterior se deduce que $\left\{a_n(2\delta)^n \right\}_{ n=1}^{\infty}$ es una sucesión acotada, luego existe $M>0$ tal que
        \begin{equation*}
            \abs{a_n(2\delta)^n}<M',\quad\forall n\in\mathbb{N}
        \end{equation*}
        Por tanto, se tiene que para todo $x\in]x_0-\delta,x_0+\delta[$, al ser la serie de potencias convergente y ser el espacio $(\mathbb{R},\abs{\cdot})$ completo, es absolutamente convergente, luego:
        \begin{equation*}
            \begin{split}
                \abs{f^{(k)}(x)}&\leq k!\abs{a_k}+\frac{(k+1)!}{1!}\abs{a_{ k+1}}\abs{x-x_0}+...+\frac{(k+n)!}{n!}\abs{a_{ k+n}}\abs{x-x_0}^n+...\\
                &\leq k!\abs{a_k}+\frac{(k+1)!}{1!}\abs{a_{ k+1}}\abs{x_0+\delta-x_0}+...+\frac{(k+n)!}{n!}\abs{a_{ k+n}}\abs{x_0+\delta-x_0}^n+...\\
                &\leq k!\abs{a_k}+\frac{(k+1)!}{1!}\abs{a_{ k+1}}\delta+...+\frac{(k+n)!}{n!}\abs{a_{ k+n}}\delta^n+...\\
                &< k!\frac{M'}{(2\delta)^k}+\frac{(k+1)!}{1!}\cdot\frac{M'}{(2\delta)^{ k+1}}\delta+...+\frac{(k+n)!}{n!}\cdot\frac{M'}{(2\delta)^{ k+n}}\delta^n+...\\
                &=\frac{k!M'}{2^k}\left[1+\frac{k+1}{1!}\cdot\frac{1}{2}+...+\frac{(k+1)(k+2)\cdot...\cdot(k+n)}{n!}\cdot\frac{1}{2^n}+... \right]\\
            \end{split}
        \end{equation*}

    \end{proof}

    

    \chapter*{Bibliografía}

    \begin{itemize}
        \item A. Markusevich, \textit{Teoría de las funciones analíticas}, Ed. Mir Moscu.
    \end{itemize}
    
\end{document}