\documentclass[12pt]{report}
\usepackage[spanish]{babel}
\usepackage[utf8]{inputenc}
\usepackage{amsmath}
\usepackage{amssymb}
\usepackage{amsthm}
\usepackage{graphics}
\usepackage{subfigure}
\usepackage{lipsum}
\usepackage{array}
\usepackage{multicol}
\usepackage{enumerate}
\usepackage[framemethod=TikZ]{mdframed}
\usepackage[a4paper, margin = 1.5cm]{geometry}
\usepackage{tikz}
\usepackage{pgffor}
\usepackage{ifthen}
\usepackage{enumitem}
\usepackage{hyperref}
\usepackage{bbm}

\usepackage{listings}

%Gestión de Hipervínculos

\hypersetup{
    colorlinks=true,
    linkcolor=black,
    filecolor=magenta,      
    urlcolor=cyan
}

%Gestión de Código de Programación

\definecolor{listing-background}{HTML}{F7F7F7}
\definecolor{listing-rule}{HTML}{B3B2B3}
\definecolor{listing-numbers}{HTML}{B3B2B3}
\definecolor{listing-text-color}{HTML}{000000}
\definecolor{listing-keyword}{HTML}{435489}
\definecolor{listing-keyword-2}{HTML}{1284CA} % additional keywords
\definecolor{listing-keyword-3}{HTML}{9137CB} % additional keywords
\definecolor{listing-identifier}{HTML}{435489}
\definecolor{listing-string}{HTML}{00999A}
\definecolor{listing-comment}{HTML}{8E8E8E}

\lstdefinestyle{myStyle}{
    language         = C++,
    alsolanguage     = scala,
    numbers          = left,
    xleftmargin      = 2.7em,
    framexleftmargin = 2.5em,
    backgroundcolor  = \color{gray!15},
    basicstyle       = \color{listing-text-color}\linespread{1.0}\ttfamily,
    breaklines       = true,
    frameshape       = {RYR}{Y}{Y}{RYR},
    rulecolor        = \color{black},
    tabsize          = 2,
    numberstyle      = \color{listing-numbers}\linespread{1.0}\small\ttfamily,
    aboveskip        = 1.0em,
    belowskip        = 0.1em,
    abovecaptionskip = 0em,
    belowcaptionskip = 1.0em,
    keywordstyle     = {\color{listing-keyword}\bfseries},
    keywordstyle     = {[2]\color{listing-keyword-2}\bfseries},
    keywordstyle     = {[3]\color{listing-keyword-3}\bfseries\itshape},
    sensitive        = true,
    identifierstyle  = \color{listing-identifier},
    commentstyle     = \color{listing-comment},
    stringstyle      = \color{listing-string},
    showstringspaces = false,
    label            = lst:bar,
    captionpos       = b,
}

\lstset{style = myStyle}

%Estilo del capítulo y sección

\makeatletter
\def\thickhrulefill{\leavevmode \leaders \hrule height 1ex \hfill \kern \z@}
\def\@makechapterhead#1{%
  {\parindent \z@ \raggedright
    \reset@font
    \hrule
    \vspace*{10\p@}%
    \par
    \center \LARGE \scshape \@chapapp{} \huge \thechapter
    \vspace*{10\p@}%
    \par\nobreak
    \vspace*{10\p@}%
    \par
    \vspace*{1\p@}%
    \hrule
    %\vskip 40\p@
    \vspace*{60\p@}
    \Huge #1\par\nobreak
    \vskip 50\p@
  }}

\def\section#1{%
  \par\bigskip\bigskip
  \hrule\par\nobreak\noindent
  \refstepcounter{section}%
  \addcontentsline{toc}{chapter}{#1}%
  \reset@font
  { \large \scshape
    \strut\S \thesection \quad
    #1}% 
    \hrule   
  \par
  \medskip
}

\def\subsection#1{%
  \par\bigskip\bigskip
  \hrule\par\nobreak\noindent
  \refstepcounter{subsection}%
  \addcontentsline{toc}{section}{#1}%
  \reset@font
  { \normalsize \scshape
    \strut\S \thesubsection \quad
    #1}% 
    \hrule   
  \par
  \medskip
}

%Gestión marca de agua

\usetikzlibrary{shapes.multipart}

\newcounter{it}
\newcommand*\watermarktext[1]{\begin{tabular}{c}
    \setcounter{it}{1}%
    \whiledo{\theit<100}{%
    \foreach \col in {0,...,15}{#1\ \ } \\ \\ \\
    \stepcounter{it}%
    }
    \end{tabular}
    }

\AddToHook{shipout/foreground}{
    \begin{tikzpicture}[remember picture,overlay, every text node part/.style={align=center}]
        \node[rectangle,black,rotate=30,scale=2,opacity=0.04] at (current page.center) {\watermarktext{Cristo Daniel Alvarado ESFM\quad}};
  \end{tikzpicture}
}

%En esta parte se hacen redefiniciones de algunos comandos para que resulte agradable el verlos%

\def\proof{\paragraph{Demostración:\\}}
\def\endproof{\hfill$\blacksquare$}

\def\sol{\paragraph{Solución:\\}}
\def\endsol{\hfill$\square$}

%En esta parte se definen los comandos a usar dentro del documento para enlistar%

\newtheoremstyle{largebreak}
  {}% use the default space above
  {}% use the default space below
  {\normalfont}% body font
  {}% indent (0pt)
  {\bfseries}% header font
  {}% punctuation
  {\newline}% break after header
  {}% header spec

\theoremstyle{largebreak}

\newmdtheoremenv[
    leftmargin=0em,
    rightmargin=0em,
    innertopmargin=0pt,
    innerbottommargin=5pt,
    hidealllines = true,
    roundcorner = 5pt,
    backgroundcolor = gray!60!red!30
]{exa}{Ejemplo}[section]

\newmdtheoremenv[
    leftmargin=0em,
    rightmargin=0em,
    innertopmargin=0pt,
    innerbottommargin=5pt,
    hidealllines = true,
    roundcorner = 5pt,
    backgroundcolor = gray!50!blue!30
]{obs}{Observación}[section]

\newmdtheoremenv[
    leftmargin=0em,
    rightmargin=0em,
    innertopmargin=0pt,
    innerbottommargin=5pt,
    rightline = false,
    leftline = false
]{theor}{Teorema}[section]

\newmdtheoremenv[
    leftmargin=0em,
    rightmargin=0em,
    innertopmargin=0pt,
    innerbottommargin=5pt,
    rightline = false,
    leftline = false
]{propo}{Proposición}[section]

\newmdtheoremenv[
    leftmargin=0em,
    rightmargin=0em,
    innertopmargin=0pt,
    innerbottommargin=5pt,
    rightline = false,
    leftline = false
]{cor}{Corolario}[section]

\newmdtheoremenv[
    leftmargin=0em,
    rightmargin=0em,
    innertopmargin=0pt,
    innerbottommargin=5pt,
    rightline = false,
    leftline = false
]{lema}{Lema}[section]

\newmdtheoremenv[
    leftmargin=0em,
    rightmargin=0em,
    innertopmargin=0pt,
    innerbottommargin=5pt,
    roundcorner=5pt,
    backgroundcolor = gray!30,
    hidealllines = true
]{mydef}{Definición}[section]

\newmdtheoremenv[
    leftmargin=0em,
    rightmargin=0em,
    innertopmargin=0pt,
    innerbottommargin=5pt,
    roundcorner=5pt
]{excer}{Ejercicio}[section]

%En esta parte se colocan comandos que definen la forma en la que se van a escribir ciertas funciones%

\newcommand\abs[1]{\ensuremath{\left|#1\right|}}
\newcommand\divides{\ensuremath{\bigm|}}
\newcommand\cf[3]{\ensuremath{#1:#2\rightarrow#3}}
\newcommand\contradiction{\ensuremath{\#_c}}
\newcommand\natint[1]{\ensuremath{\left[\big|#1\big|\right]}}
\newcommand{\bbm}[1]{\ensuremath{\mathbbm{#1}}}

\newcounter{figcount}
\setcounter{figcount}{1}

\newcommand{\tbf}[1]{\textbf{#1}}

%recuerda usar \clearpage para hacer un salto de página

\begin{document}
    \setlength{\parskip}{5pt} % Añade 5 puntos de espacio entre párrafos
    \setlength{\parindent}{12pt} % Pone la sangría como me gusta
    \title{Curso de Variable Compleja}
    \author{Cristo Daniel Alvarado}
    \maketitle

    \tableofcontents %Con este comando se genera el índice general del libro%

    \chapter{Introducción}
    
    \section{Fundamentos}

    El objetivo principal de la teoría de las funciones analíticas es el análisis de funciones que localmente pueden ser descritas en términos de una serie de potencias convergente, dispuesta como:
    \begin{equation}
        \begin{split}
            f(x)&=a_0+a_1(x-x_0)+a_2(x-x_0)^2+...+a_n(x-x_0)^n+...\\
            &=\sum_{ k=0}^\infty a_k(x-x_0)^k,\quad\forall x\in]x_0-\delta,x_0+\delta[
        \end{split}
    \end{equation}
    siendo $\cf{f}{I}{\mathbb{R}}$ con $I$ un intervalo, $x_0\in I$ y $\delta>0$ tal que $]x_0-\delta,x_0+\delta[\subseteq]a,b[$. Cuando una funcion de este tipo puede ser descrita de la forma anterior para algún par $x_0$ y $\delta$, decimos en este caso que \tbf{$f$ es analítica en $x_0$}.

    En el caso que $I$ sea un intervalo abierto y $f$ sea analítica en $x_0$ para todo $x_0\in I$, decimos que \tbf{$f$ es analítica en $I$}.

    \begin{exa}
        Las funciones $x\mapsto P(x)$, $x\mapsto e^x$, $x\mapsto \sin x$ y $x\mapsto \cos x$ son analíticas en $\mathbb{R}$
    \end{exa}

    Debido a que como resultado de efectuar operaciones algebraicas y analíticas (suma, resta, multiplicación, división, integración y derivación) sobre series de potencias resulta nuevamente en una serie de potencias convergente, es de gran interés conocer las propiedades de estas funciones (más que nada debido a las ecuaciones diferenciales). Esto motiva el estudio particular de este tipo  de funciones.

    A pesar de lo amplia que es esta clase de funciones, ésta solamente forma una parte regular de las funciones \textit{infinitamente diferenciables}.

    \begin{propo}
        Sea $\cf{f}{]x_0-r,x_0+r[}{\mathbb{R}}$ una función, siendo $r>0$ y $x_0\in\mathbb{R}$. Entonces, $f$ es analítica en $x_0$ si y sólo si se satisfacen las condiciones siguientes:
        \begin{enumerate}
            \item $f$ tiene derivadas de todos los órdenes en un entorno de $x_0$.
            \item Existen $\delta,M>0$ tales que para todo $x\in]x_0-\delta,x_0+\delta[$ y para todo $k\in\mathbb{N}$ se cumple:
            \begin{equation*}
                \abs{f^{(k)}(x)}<M\frac{k!}{\delta^k}
            \end{equation*}
        \end{enumerate}
    \end{propo}

    \begin{proof}
        $\Rightarrow):$ Suponga que $f$ es analítica en $x_0$, entonces existen $a_0,a_1,...,a_n,...\in\mathbb{R}$ y $\rho>0$ tal que
        \begin{equation*}
            f(x)=a_0+a_1(x-x_0)+a_2(x-x_0)^2+...
        \end{equation*}
        para todo $x\in]x_0-\rho,x_0+\rho[$ (note que $\rho<r$). Se sabe por resultados de análisis real que $f$ tiene derivadas de todos los órdenes en $]x_0-\rho,x_0+\rho[$ y, en particular para todo $k\in\mathbb{N}$ se tiene que
        \begin{equation*}
            f^{(k)}(x)=k!a_k+\frac{(k+1)!}{1!}a_{ k+1}(x-x_0)+\frac{(k+2)!}{2!}a_{ k+2}(x-x_0)^2+...+\frac{(k+n)!}{n!}a_{ k+n}(x-x_0)^n+...
        \end{equation*}
        para todo $x\in]x_0-\rho,x_0+\rho[$. Fijemos $k\in\mathbb{N}$ y tomemos $\delta>0$ tal que $0<2\delta<\rho$. Si $x=x_0+2\delta$, entonces la serie anterior convergerá y, por ende en el límite debe suceder que
        \begin{equation*}
            \begin{split}
                \lim_{ n\rightarrow\infty}\frac{(k+n)!}{n!}a_{ k+n}(x-x_0)^n&=0\\
                \iff \lim_{ n\rightarrow\infty}a_{ k+n}(x_0+2\delta-x_0)^n&=0\\
                \iff a^k\lim_{ n\rightarrow\infty}a_{n}(2\delta)^n&=0\\
                \iff \lim_{ n\rightarrow\infty}a_{n}(2\delta)^n&=0\\
            \end{split}
        \end{equation*}
        (pues, $\lim_{ n\rightarrow\infty}\frac{(k+n)!}{n!}=1$). En particular, de lo anterior se deduce que $\left\{a_n(2\delta)^n \right\}_{ n=1}^{\infty}$ es una sucesión acotada, luego existe $M>0$ tal que
        \begin{equation*}
            \abs{a_n(2\delta)^n}<M',\quad\forall n\in\mathbb{N}
        \end{equation*}
        Por tanto, se tiene que para todo $x\in]x_0-\delta,x_0+\delta[$, al ser la serie de potencias convergente y ser el espacio $(\mathbb{R},\abs{\cdot})$ completo, es absolutamente convergente, luego:
        \begin{equation*}
            \begin{split}
                \abs{f^{(k)}(x)}&\leq k!\abs{a_k}+\frac{(k+1)!}{1!}\abs{a_{ k+1}}\abs{x-x_0}+...+\frac{(k+n)!}{n!}\abs{a_{ k+n}}\abs{x-x_0}^n+...\\
                &\leq k!\abs{a_k}+\frac{(k+1)!}{1!}\abs{a_{ k+1}}\abs{x_0+\delta-x_0}+...+\frac{(k+n)!}{n!}\abs{a_{ k+n}}\abs{x_0+\delta-x_0}^n+...\\
                &\leq k!\abs{a_k}+\frac{(k+1)!}{1!}\abs{a_{ k+1}}\delta+...+\frac{(k+n)!}{n!}\abs{a_{ k+n}}\delta^n+...\\
                &< k!\frac{M'}{(2\delta)^k}+\frac{(k+1)!}{1!}\cdot\frac{M'}{(2\delta)^{ k+1}}\delta+...+\frac{(k+n)!}{n!}\cdot\frac{M'}{(2\delta)^{ k+n}}\delta^n+...\\
                &=\frac{k!M'}{2^k}\left[1+\frac{k+1}{1!}\cdot\frac{1}{2}+...+\frac{(k+1)(k+2)\cdot...\cdot(k+n)}{n!}\cdot\frac{1}{2^n}+... \right]\\
            \end{split}
        \end{equation*}

    \end{proof}

    \chapter{Propiedades Elementales y Ejemplos de Funciones Analíticas}
    
    \section{Series de Potencias}

    Se darán ejemplos y se hablará sobre las propiedades fundamentales de las series de potencias.

    \begin{mydef}
        Sea $\left\{ a_n \right\}_{ n=1}^\infty$ una sucesión en $\mathbb{C}$. Decimos que la serie de $\sum_{n=1}^\infty a_n$ \tbf{converge a $z\in\mathbb{C}$}, si para todo $\varepsilon>0$ existe $N\in\mathbb{N}$ tal que
        \begin{equation*}
            n\geq N\Rightarrow\abs{\sum_{ k=1}^na_n-z }<\varepsilon
        \end{equation*}
        La serie $\sum_{n=1}^\infty a_n$ \tbf{converge absolutamente}, si $\sum_{n=1}^\infty\abs{a_n}$ converge.
    \end{mydef}

    \begin{propo}
        Si $\sum_{n=1}^\infty a_n$ converge absolutamente, entonces $\sum_{n=1}^\infty a_n$ es convergente.
    \end{propo}

    \begin{proof}
        Inmediata de las propiedades del módulo.
    \end{proof}

    \begin{mydef}
        Sea $\left\{a_n \right\}_{ n=1}^\infty$ una sucesión en $\mathbb{R}$. Se definen:
        \begin{itemize}
            \item $\limsup_{ n\rightarrow\infty}a_n=\lim_{ n\rightarrow\infty}\sup\left\{a_n,a_{ n+1},... \right\}$.
            \item $\liminf_{ n\rightarrow\infty}a_n=\lim_{ n\rightarrow\infty}\inf\left\{a_n,a_{ n+1},... \right\}$.
        \end{itemize}
        como el número real en $\overline{\mathbb{R}}$.
    \end{mydef}

    \begin{obs}
        El límite superior y límite inferior de una sucesión siempre existe.
    \end{obs}

    \begin{mydef}
        Una \tbf{serie de potencias alrededor de $a\in\mathbb{C}$} es una serie de la forma $\sum_{ n=0}^\infty a_n(z-a)^n$, siendo $\left\{ a_n \right\}_{ n=1}^\infty$ una sucesión en $\mathbb{C}$ y $z\in\mathbb{C}$.
    \end{mydef}

    \begin{exa}
        La serie geométrica
        \begin{equation*}
            \sum_{ n=0}^\infty z^n
        \end{equation*}
        es convergente a $\frac{1}{1-z}$ si y sólo si $\abs{z}<1$.
    \end{exa}

    \begin{theor}[\tbf{Critero $M$ de Weierestrass}]
        Para cada $n\in\mathbb{N}$, sea $\cf{u_n}{X}{\mathbb{C}}$ una función tal que existe $M_n\in\mathbb{R}$ tal que
        \begin{equation*}
            \abs{u_n}(x)<M_n,\quad\forall x\in X
        \end{equation*}
        entonces, si $\sum_{ n=1}^\infty M_n<\infty$ se tiene que la serie
        \begin{equation*}
            \sum_{ n=1}^\infty u_n
        \end{equation*}
        converge uniformemente.
    \end{theor}

    \begin{proof}
        Se hizo en Análisis Matemático I.
    \end{proof}

    \begin{theor}
        Sea $\sum_{ n=0}^\infty a_n(z-a)^n$ una serie de potencias, defina el número $R\in[0,\infty]$ tal que
        \begin{equation*}
            \frac{1}{R}=\limsup_{ n\rightarrow\infty}\abs{a_n}^{ 1/n}
        \end{equation*}
        entonces:
        \begin{enumerate}
            \item Si $z\in\mathbb{C}$ es tal que $\abs{z-a}<R$, la serie converge absolutamente.
            \item Si $z\in\mathbb{C}$ es tal que $\abs{z-a}>R$, los términos de la serie no son acotados, por lo que la serie diverge.
            \item Si $0<r<R$, la serie converge uniformemente en $\left\{ z\in\mathbb{C}\Big|\abs{z}\leq r \right\}$.
        \end{enumerate}
        El elemento no negativo de la recta real extendida $R$ es el único con las propiedades (1) y (2), y es llamado el \tbf{radio de convergencia de la serie $\sum_{ n=0}^\infty a_n(z-a)^n$}.
    \end{theor}

    \begin{proof}
        De (1): Se tienen dos casos; si $R>0$ y $R=0$:
        \begin{itemize}
            \item \tbf{$R>0$}: Podemos suponer que $a=0$. Sea $z\in\mathbb{C}$ tal que $\abs{z}<R$. Existe $r\in\mathbb{R}$ tal que $\abs{z}<r<R$. $\frac{1}{R}=\limsup_{ n\rightarrow\infty}\abs{a_n}^{ 1/n}$, entonces existe $N\in\mathbb{N}$ tal que
            \begin{equation*}
                \abs{a_n}^{ 1/n}<\frac{1}{r},\quad\forall n\geq N
            \end{equation*}
            pues, $1/R<1/r$. Entonces:
            \begin{equation*}
                \abs{a_n}<\frac{1}{r^n},\quad\forall n\geq N
            \end{equation*}
            se sigue así que:
            \begin{equation*}
                \abs{a_nz^n}\leq\left(\frac{\abs{z}}{r}\right)^n,\quad\forall n\geq N
            \end{equation*}
            Por ende:
            \begin{equation*}
                \sum_{ n=N}^\infty\abs{a_nz^n}\leq\sum_{ n=N}^\infty\left(\frac{\abs{z}}{r}\right)^n<\infty
            \end{equation*}
            pues $\abs{z}/r<1$. Por tanto, la serie de potencias
            \begin{equation*}
                \sum_{ n=1}^\infty a_nz^n
            \end{equation*}
            es absolutamente convergente.
            \item Si $R=0$, entonces no existe $z\in\mathbb{C}$ tal que $\abs{z}<R$. Por tanto, el resultado se cumple por vacuidad.
        \end{itemize}

        De (2): El procedimiento es análogo (pero mostrando la divergencia de una serie geométrica) al de (1).

        De (3): Sea $r\in\mathbb{R}$ tal que $0<r<R$. Tomemos $\rho\in\mathbb{R}$ tal que $r<\rho<R$. Como en (1) existe $N\in\mathbb{N}$ tal que
        \begin{equation*}
            \abs{a_n}<\frac{1}{\rho},\quad\forall n\geq N
        \end{equation*}
        Entonces, si $\abs{z}\leq r$ se tiene que
        \begin{equation*}
            \abs{a_nz^n}\leq\left(\frac{r}{\rho}\right)^n,\quad\forall n\geq N
        \end{equation*}
        siendo $r/\rho<1$. Por el Criterio $M$ de Weierestrass se sigue que la serie de potencias
        \begin{equation*}
            \sum_{ n=1}^\infty \abs{a_nz^n}
        \end{equation*}
        converge uniformemente en $\left\{ z\in\mathbb{C}\Big|\abs{z}\leq r \right\}$.

        La unicidad de $R$ se sigue de (1) y (2).
    \end{proof}

    \begin{propo}
        Si $\sum_{ n=1}^\infty a_n(z-a)^n$ es una serie de potencias con radio de convergencia $R\in[0,\infty]$, entonces
        \begin{equation*}
            R=\lim_{n\rightarrow\infty}\abs{\frac{a_n}{a_{ n+1}}}
        \end{equation*}
        si el límite existe.
    \end{propo}

    \begin{proof}
        Poedmos suponer que $a=0$. Si el límite anterior existe, denotémoslo por
        \begin{equation*}
            \alpha=\lim_{n\rightarrow\infty}\abs{\frac{a_n}{a_{ n+1}}}
        \end{equation*}
        Probaremos que $\alpha=R$. Sea $z\in\mathbb{C}$:
        \begin{itemize}
            \item Suponga que $r\in\mathbb{R}$ es tal que $\abs{z}<r<\alpha$. Existe $N\in\mathbb{N}$ tal que
            \begin{equation*}
                \abs{\frac{a_{ n}}{a_{ n+1}}}>r,\quad\forall n\geq N
            \end{equation*}
            (pues, el límite converge a $\alpha$). Sea $B=\abs{a_n}r^N$. Entonces:
            \begin{equation*}
                \begin{split}
                    \abs{a_{ N+1}}r^{N+1}&=\abs{a_{ N+1}}rr^{N}\\
                    &<\abs{a_N}r^N\\
                    &=B\\
                \end{split}
            \end{equation*}
            por inducción se prueba rápidamente que:
            \begin{equation*}
                \abs{a_nr^n}\leq B,\quad\forall n\geq N
            \end{equation*}
            Entonces:
            \begin{equation*}
                \begin{split}
                    \abs{a_nz^n}&=\abs{a_nr^n}\cdot\frac{\abs{z^n}}{r^n}\\
                    &=B\cdot\frac{\abs{z}^n}{r^n},\quad\forall n\geq N \\
                \end{split}
            \end{equation*}
            Como $\abs{z}<r$, entonces $\frac{\abs{z}}{r}<1$. Por lo cual la serie
            \begin{equation*}
                \sum_{ n=1}^\infty a_nz^n
            \end{equation*}
            es absolutamente convergente para todo $z\in\mathbb{C}$ tal que $\abs{z}<\alpha$. Por (1) del Teorema anterior, debe suceder que $\alpha\leq R$.
            \item Un procedimiento análogo al anterior pero con $\abs{z}>\alpha$ prueba que la serie
            \begin{equation*}
                \sum_{ n=1}^\infty a_nz^n
            \end{equation*}
            no converge para todo $z\in\mathbb{C}$ tal que $\alpha<\abs{z}$. Por ende, $R\leq \alpha$
        \end{itemize}
        Por los dos incisos anteriores se sigue que $\alpha=R$.
    \end{proof}

    \begin{exa}
        La serie de potencias
        \begin{equation*}
            \sum_{ n=1}^\infty\frac{z^n}{n!}
        \end{equation*}
        tiene radio de convergencia $R=\infty$.
    \end{exa}

    \begin{proof}
        En efecto, veamos que:
        \begin{equation*}
            \begin{split}
                \lim_{n\rightarrow\infty}\abs{\frac{a_{ n}}{a_{n+1}}}&=\lim_{n\rightarrow\infty}\frac{(n+1)!}{n!}\\
                &=\lim_{n\rightarrow\infty}(n+1)\\
                &=\infty\\
            \end{split}
        \end{equation*}
        por lo cual, $R=\infty$.
    \end{proof}

    \begin{mydef}
        Se define la \tbf{función exponencial}, como la función $\cf{\exp}{\mathbb{C}}{\mathbb{C}}$ dada por:
        \begin{equation*}
            z\mapsto e^z=\exp(z)=\sum_{ n=0}^\infty\frac{z^n}{n!}
        \end{equation*}
        por la parte anterior, esta serie es absolutamente convergente en $\mathbb{C}$, por lo que la función $\exp$ está bien definida.
    \end{mydef}

    \begin{propo}
        \label{MultSumSeriesPot}
        Sean $\sum_{n=0}^\infty a_n$ y $\sum_{n=0}^\infty b_n$ dos series absolutamente convergentes, y sea
        \begin{equation*}
            c_n=\sum_{ k=0}^n b_ka_{ n-k},\quad\forall n\in\mathbb{N}\cup\left\{0\right\}
        \end{equation*}
        entonces, $\sum_{n=0}^\infty c_n$ es absolutamente convergente, y:
        \begin{equation*}
            \sum_{n=0}^\infty c_n=\left(\sum_{n=0}^\infty a_n\right)\cdot\left(\sum_{n=0}^\infty b_n\right)
        \end{equation*}
    \end{propo}

    \begin{proof}
        %TODO
        Ejercicio.
    \end{proof}

    \section{Funciones Analíticas}

    Se definen las funciones analíticas y se dan algunos ejemplos.

    \begin{mydef}
        Sea $G\subseteq\mathbb{C}$ abierto y $\cf{f}{G}{\mathbb{C}}$ una función. Entonces, $f$ es \tbf{diferenciable en $a\in G$}, si el límite:
        \begin{equation*}
            \lim_{ h\rightarrow0}\frac{f(a+h)-f(a)}{h}
        \end{equation*}
        existe; el valor de este límite es denotado por $f'(a)$ y es llamado la \tbf{derivada de $f$ en $a$}. Si $f$ es diferenciable en todo punto de $G$, decimos que $f$ es \tbf{diferenciable en $G$}.

        Puede entonces definirse una función $\cf{f'}{G'\subseteq G}{\mathbb{C}}$, donde $G'\subseteq G$ es el conjunto de puntos donde $f$ es diferenciable. En caso de que $f$ sea diferenciable en todo $G$, se sigue que $G'=G$.

        Si $f'$ es continua, decimos que $f$ es \tbf{diferenciable continua}. Si $f'$ es diferenciable, decimos que $f$ es \tbf{dos veces diferenciable}, continuando, una función $f$ tal que cada derivada sucesiva es diferenciable se dice \tbf{infinitamente diferenciable}.
    \end{mydef}

    \begin{propo}
        Si una función $\cf{f}{G}{\mathbb{C}}$ es diferenciable en $a\in G$, entonces $f$ es continua en $a$.
    \end{propo}

    \begin{proof}
        Veamos que:
        \begin{equation*}
            \begin{split}
                \lim_{ z\rightarrow a}\abs{f(z)-f(a)}&=\lim_{ z\rightarrow a}\left(\frac{\abs{f(z)-f(a)}}{\abs{z-a}}\cdot\abs{z-a}\right)\\
                &=\lim_{ z\rightarrow a}\left(\frac{\abs{f(z)-f(a)}}{\abs{z-a}}\right)\cdot\lim_{ z\rightarrow a}\abs{z-a}\\
                &=\abs{\lim_{ z\rightarrow a}\frac{f(z)-f(a)}{z-a}}\cdot 0\\
                &=\abs{f'(a)}\cdot 0\\
                &=0\\
            \end{split}
        \end{equation*}
        lo cual prueba el resultado.
    \end{proof}

    \begin{mydef}
        Una función $\cf{f}{G}{\mathbb{C}}$ es \tbf{analítica} si es diferenciable continua en $G$.
    \end{mydef}

    Se sigue rápidamente (como en cálculo), que las sumas y productos de funciones analíticas siguen siendo analíticas. Si $f$ y $g$ son analíticas en $G$ y $G_1$ es el conjunto de puntos donde $g$ no es cero, entonces $f/g$ es analítica en $G_1$.

    Como la función constante y $z$ son analíticas, se sigue que todas las funciones racionales son analíticas en el complemento de los ceros del denominador.

    Más aún, todas las leyes de diferenciación de sumas, productos y cocientes siguen siendo válidas.

    \begin{theor}[\tbf{Regla de la Cadena}]
        Sean $f,g$ funciones analíticas en $G$ y $\Omega$, respectivamente y suponga que $f(G)\subseteq\Omega$. Entonces, $f\circ g$ es analítica en $G$ y:
        \begin{equation*}
            (g\circ f)'(z)=g'(f(z))\cdot f'(z),\quad\forall z\in G
        \end{equation*}
    \end{theor}

    \begin{proof}
        Sea $z_0\in G$. Como $G$ es abierto, existe $r>0$ tal que
        \begin{equation*}
            \left\{\abs{z_0-z}<r\Big|z\in\mathbb{C} \right\}\subseteq G
        \end{equation*}
        para probar que el límite
        \begin{equation*}
            \lim_{ h\rightarrow 0}\frac{g\circ f(z_0+h)-g\circ f(z_0)}{h}
        \end{equation*}
        existe, basta con mostrar que para toda sucesión $\left\{h_n \right\}_{ n=1}^\infty$ en $\mathbb{C}\setminus\left\{0\right\}$ que converja a 0 se cumple que el límite
        \begin{equation*}
            \lim_{ n\rightarrow\infty}\frac{g\circ f(z_0+h_n)-g\circ f(z_0)}{h_n}
        \end{equation*}
        existe y es igual a $g'(f(z_0))\cdot f'(z_0)$. En efecto, sea $\left\{h_n \right\}_{ n=1}^\infty$ una sucesión que converge a 0, podemos asumir que:
        \begin{equation*}
            0<\abs{h_n}<r,\quad\forall n\in\mathbb{N}
        \end{equation*}
        Haremos la prueba por casos:
        \begin{enumerate}[label = \textit{\quad Caso \arabic*:}]
            \item Suponga que $f(z_0)\neq f(z_0+h_n)$ para todo $n\in\mathbb{N}$. En este caso:
            \begin{equation*}
                \begin{split}
                    \frac{g\circ f(z_0+h_n)-g\circ f(z_0)}{h_n}&=\frac{g\circ f(z_0+h_n)-g\circ f(z_0)}{f(z_0+h_n)-f(z_0)}\cdot\frac{f(z_0+h_n)-f(z_0)}{h_n}\\
                \end{split}
            \end{equation*}
            por ser $f$ continua, se sigue que:
            \begin{equation*}
                \lim_{ n\rightarrow\infty}f(z_0+h_n)-f(z_0)=0
            \end{equation*}
            por lo que, al tomar límite cuando $n\rightarrow\infty$ se obtiene que:
            \begin{equation*}
                \lim_{ n\rightarrow\infty}\frac{g\circ f(z_0+h_n)-g\circ f(z_0)}{h_n}=g'(f(z_0))\cdot f'(z_0)
            \end{equation*}
            \item $f(z_0)=f(z_0+h_n)$ para todo $n\in J\subseteq\mathbb{N}$, donde $J$ es un conjunto infinito.%TODO
        \end{enumerate}
    \end{proof}

    \begin{mydef}
        Una función compleja $f$ se dirá \tbf{analítica en $A\subseteq\mathbb{C}$}, si existe $G\subseteq\mathbb{C}$ tal que $f$ es analítica en $G$ y $A\subseteq G$.
    \end{mydef}

    En lo que sigue de este curso, se hará el mayor esfuerzo para ver porqué la teoría de las funciones analíticas es \textit{extremandamente} diferente del cálculo tradicional.

    \begin{propo}
        Sea $f(z)=\sum_{ n=0}^\infty a_n(z-a)^n$ una serie de potencias con radio de convergencia $R>0$. Entonces:
        \begin{enumerate}[label = \textit{(\arabic*)}]
            \item Para cada $k\geq 1$, la serie:
            \begin{equation}
                \label{derivadaSeriePotencias}
                \sum_{n=k}^\infty n(n-1)\cdots(n-k+1)a_n(z-a)^{ n-k}
            \end{equation}
            tiene radio de convergencia $R>0$.
            \item La función $f$ es infinitamente diferenciable en $B(a,R)$ y más aún, $f^{(k)}$ está dada por la serie en la ecuación (\ref{derivadaSeriePotencias}).
            \item Para todo $n\geq0$:
            \begin{equation*}
                a_n=\frac{1}{n!}f^{(n)}(a)
            \end{equation*}
        \end{enumerate}
    \end{propo}

    \begin{proof}
        Podemos asumir que $a=0$, ya que la función $z\mapsto z-a$ es diferenciable en $\mathbb{C}$ y composición de funciones diferenciables es diferenciable.

        De \textit{(a)}: Basta probar el caso con $k=1$, ya que por inducción se sigue rápidamente que se cumple para todo $k\geq1$. Probaremos que el radio de convergencia de la serie:
        \begin{equation*}
            \sum_{ n=1}^\infty na_nz^{ n-1}
        \end{equation*}
        es $R$. Para ello, recordemos que como $\sum_{ n=0}^\infty a_n(z-a)^n$ tiene radio de convergencia $R$, se tiene:
        \begin{equation*}
            \frac{1}{R}=\limsup_{ n\rightarrow\infty}\abs{a_n}^{ 1/n}
        \end{equation*}
        queremos probar que:
        \begin{equation*}
            \frac{1}{R}=\limsup_{ n\rightarrow\infty}\abs{na_n}^{ 1/{n-1}}
        \end{equation*}
        veamos que el límite $\lim_{n\rightarrow\infty}n^{\frac{1}{n-1}}$ existe, pues se tiene que:
        \begin{equation*}
            \begin{split}
                \lim_{ n\rightarrow\infty}\frac{\ln n}{n-1}&=\lim_{ n\rightarrow\infty}\frac{1}{n}\\
                &=0\\
                \Rightarrow\lim_{ n\rightarrow\infty}\ln\left(n^{\frac{1}{n-1}} \right)&=0\\
            \end{split}
        \end{equation*}
        al ser $x\mapsto \ln x$ una función continua, se sigue que:
        \begin{equation*}
            \lim_{ n\rightarrow\infty}n^{\frac{1}{n-1}}=1
        \end{equation*}
        Del Ejercicio (\ref{productoSucesionesLimSupLim}) se sigue que:
        \begin{equation*}
            \begin{split}
                \limsup_{ n\rightarrow\infty}\abs{na_n}^{\frac{1}{n-1}}&=1\cdot\limsup_{ n\rightarrow\infty}\abs{a_n}^{\frac{1}{n-1}}\\
                &=\limsup_{ n\rightarrow\infty}\abs{a_n}^{\frac{1}{n-1}}\\
            \end{split}
        \end{equation*}
        veamos que el límite superior anterior es $\frac{1}{R}$. Sea $R'>0$ tal que:
        \begin{equation*}
            \frac{1}{R'}=\limsup_{ n\rightarrow\infty}\abs{a_n}^{\frac{1}{n-1}}\\
        \end{equation*}
        Entonces, $R'$ es el radio de convergencia de la serie:
        \begin{equation*}
            \sum_{ n=1}^\infty a_nz^{ n-1}=\sum_{ n=0}^\infty a_{n+1}z^{n}
        \end{equation*}
        Observemos ahora que:
        \begin{equation*}
            \sum_{ n=0}^\infty a_nz^n=z\sum_{ n=0}^\infty a_{ n+1}z^n+a_0
        \end{equation*}
        Por lo que la sere de la derecha converge si y sólo si la de la izquierda lo hace, es decir:
        \begin{equation*}
            \abs{z}<R'\iff \abs{z}<R'
        \end{equation*}
        Por ende, $R=R'$.

        De \textit{(b)}: Nuevamente, solo hay que probar que la función es diferenciable. Sea:
        \begin{equation*}
            g(z)=\sum_{ n=1}^\infty na_nz^{ n-1}
        \end{equation*}
        Esta serie tiene radio de convergencia $R>0$. Tomemos:
        \begin{equation*}
            s_n(z)=\sum_{ k=0}^n a_kz^k\quad\textup{y}\quad R_n(z)=\sum_{ k=n+1}^\infty a_kz^k
        \end{equation*}
        Sea $0<r<R$, $\varepsilon>0$ y tomemos $w\in\bbm{C}$ tal que $\abs{w}<r$. Probaremos que $f'(w)$ existe y es igual a $g(w)$. En efecto, tomemos $\delta>0$ tal que:
        \begin{equation*}
            \overline{B(w,\delta)}\subseteq B(w,r)
        \end{equation*}
        Sea $z\in B(w,\delta)$, entonces:
        \begin{equation*}
            \frac{f(z)-f(w)}{z-w}-g(w)=\left[\frac{s_n(z)-s_n(w)}{z-w}-s_n'(w) \right]+[s_n'(w)-g(w)]+\left[\frac{R_n(z)-R_n(w)}{z-w}\right] 
        \end{equation*}
        Ahora:
        \begin{equation*}
            \begin{split}
                \frac{R_n(z)-R_n(w)}{z-w}&=\frac{1}{z-w}\sum_{ k=n+1}^\infty a_k(z^k-w^k)\\
                &=\sum_{ k=n+1}^\infty a_k\left(\frac{z^k-w^k}{z-w} \right)\\
            \end{split}
        \end{equation*}
        con:
        \begin{equation*}
            \abs{\frac{z^k-w^k}{z-w}}=\abs{z^{k-1}+z^{k-2}w+\cdots+zw^{ k-2}+w^{ k-1} }\leq kr^{ k-1}
        \end{equation*}
        Por tanto:
        \begin{equation*}
            \abs{\frac{R_n(z)-R_n(w)}{z-w}}\leq\sum_{ k=n+1}^\infty a_kkr^{ k-1}
        \end{equation*}
        Como $r<R$, entonces la serie anterior converge, así que existe $N_1\in\bbm{N}$ tal que:
        \begin{equation*}
            n\geq N_1\Rightarrow \abs{\frac{R_n(z)-R_n(w)}{z-w}}<\frac{\varepsilon}{3}
        \end{equation*}
        Adicionalmente, como $\lim_{ n\rightarrow\infty }s_n'(w)=g(w)$, existe $N_2\in\bbm{N}$ tal que:
        \begin{equation*}
            n\geq N_2\Rightarrow \abs{s_n'(w)-g(w)}<\frac{\varepsilon}{3}
        \end{equation*}
        Sea $N=\max\left\{N_1,N_2 \right\}$. Como $s_N$ es diferenciable con derivada $s_N'$, se tiene que existe $\delta>0$ tal que:
        \begin{equation*}
            0<\abs{z-w}<\delta\Rightarrow \abs{\frac{s_n(z)-s_n(w)}{z-w}-s_n'(w)}<\frac{\varepsilon}{3}
        \end{equation*}
        Por tanto, poniendo todas las desigualdades juntas resulta que:
        \begin{equation*}
            0<\abs{z-w}<\delta\Rightarrow\abs{\frac{f(z)-f(w)}{z-w}-g(w)}<\varepsilon
        \end{equation*}
        Por tanto, $f'(w)=g(w)$.

        De \textit{(c)}: Es inmediato de evaluar las derivadas en $z=a$.
    \end{proof}

    \begin{cor}
        Si la serie de potencias $\sum_{ n=0}^\infty a_n(z-a)^n$ tiene radio de convergencia $R>0$, entonces $f(z)=\sum_{ n=0}^\infty a_n(z-a)^n$ es analítica en $B(a,R)$.
    \end{cor}

    \begin{proof}
        Inmediata del teorema anterior.
    \end{proof}

    De forma inmediata se sigue que la función:
    \begin{equation*}
        z\mapsto e^z=\sum_{ n=0}^\infty\frac{z^n}{n!}
    \end{equation*}
    es analítica en $\bbm{C}$.

    \begin{propo}
        Si $G\subseteq\bbm{C}$ es abierto y conexo, y la función $\cf{f}{G}{\bbm{C}}$ es diferenciable con $f'(z)=0$ para todo $z\in G$, entonces $f$ es constante.
    \end{propo}

    \begin{proof}
        Sea $z_0\in G$ y tomemos $\omega=f(z_0)$. Probaremos que el conjunto:
        \begin{equation*}
            A=\left\{z\in G\Big|f(z)=\omega \right\}
        \end{equation*}
        es abierto y cerrado no vacío, por ende es todo $G$. En efecto, sea $z\in\overline{A}\cap G$, entonces existe una sucesión $\left\{z_n \right\}_{ n=1}^\infty$ en $A$ tal que:
        \begin{equation*}
            \lim_{ n\rightarrow\infty}z_n=z
        \end{equation*}
        Como $f$ es diferenciable, en particular es continua, por lo que:
        \begin{equation*}
            \omega=\lim_{ n\rightarrow}f(z_n)=f(z)
        \end{equation*}
        Así que $z\in A$. Por tanto, $A=\overline{A}^G$, es decir que $A$ es cerrado en $G$.

        Ahora, sea $z\in A$, entonces existe $\varepsilon>0$ tal que $B(z,\varepsilon)\subseteq G$ y tomemos $w\in B(z,w)$. Definimos la función $\cf{g}{[0,1]}{\bbm{C}}$ dada por:
        \begin{equation*}
            g(t)=f(tz+(1-t)w),\quad\forall t\in[0,1]
        \end{equation*}
        Entonces, se tiene que $g$ es diferenciable en $[0,1]$ y:
        \begin{equation*}
            \begin{split}
                \frac{dg}{dt}(t)&=f'(tz+(1-t)w)\cdot \frac{d}{dt}(tz+(1-t)w)\\
                &=f'(tz+(1-t)w)\cdot(z-w)\\
                &=0\\
            \end{split}
        \end{equation*}
        por el Ejercicio (\ref{normalDerivadaComplejaDerivadaCompos}) y ya que $f'=0$ en $G$. Por tanto, $g'=0$, es decir que $g$ es constante, en particular:
        \begin{equation*}
            f(w)=g(0)=g(1)=f(z)=\omega
        \end{equation*}
        Con lo que $w\in A$. Así que $B(z,\varepsilon)\subseteq A$.
    \end{proof}

    \begin{mydef}
        Una función $f$ es \textbf{periódica con período $c\in\bbm{C}$} si $f(z+c)=f(z)$ para todo $z\in\bbm{C}$.
    \end{mydef}

    En particular, se tiene de algunos ejercicios que:
    \begin{equation*}
        e^z=e^{ z+c}\iff c=i\theta
    \end{equation*}
    con $\theta\in\bbm{R}$. Más aún, de las propiedades de la función seno y coseno se sigue que:
    \begin{equation*}
        \theta=2\pi ik,\quad\textup{ para algún }k\in\bbm{Z}
    \end{equation*}

    De esta forma se sigue que la función exponencial es periódica.

    Queremos ahora definir la función logarítmo, pero no lo podemos hacer con expansión en serie de potencias (ya que los limitaríamos a un disco) ni con la integral de $t\mapsto\frac{1}{t}$, para ello procederemos de la siguiente manera:

    Queremos una función tal que:
    \begin{equation*}
        w\mapsto\log w=z\iff w=e^z
    \end{equation*}

    Como $e^z$ nunca es cero, logaritmo no puede estar definido en cero. Por tanto, supongamos que $e^z=w$ y $w\neq 0$. Si $z=x+iy$, entonces $\abs{w}=e^x$ y $y=\arg w+2\pi k$ para alguna $k\in\bbm{Z}$ (por la periodicidad de $e^z$).
    
    Así que el conjunto:
    \begin{equation*}
        \left\{\ln\abs{w}+i(\arg w+2\pi k)\Big|k\in\bbm{Z} \right\}
    \end{equation*}
    es un conjunto solución de $e^z=w$ (siendo $\ln\abs{w}$ el logaritmo usual).

    \begin{mydef}[\textbf{Rama de logarítmo}]
        Si $G$ es un conjunto abierto de $\bbm{C}$ y $\cf{f}{G}{\bbm{C}}$ es continua tal que:
        \begin{equation*}
            z=e^{ f(z)},\quad\forall z\in G
        \end{equation*}
        decimos que $f$ es una \textbf{rama de logarítmo}.
    \end{mydef}

    \begin{obs}
        Si $f$ es una rama de logarítmo en el conjunto abierto conexo $G$, entonces $g(z)=f(z)+2\pi ik$ también lo es. ¿El converso se cumple?
    \end{obs}

    \begin{propo}[\textbf{Caracterización de las ramas de logarítmo}]
        Sea $G\subseteq\bbm{C}$ abierto y conexo, $f$ una rama de logarítmo en $G$. Entonces todas las ramas de logarítmo en $G$ son de la forma:
        \begin{equation*}
            f(z)+2\pi ik
        \end{equation*}
        para algún $k\in\bbm{Z}$.
    \end{propo}

    \begin{proof}
        
    \end{proof}

    \section{Ejercicios}

    Ejercicios de cada una de las secciones.

    \subsection{Series de Potencias}

    \begin{excer}
        Pruebe que si $z_1,z_2\in\mathbb{C}$, entonces:
        \begin{equation*}
            e^{z_1+z_2}=e^{ z_1}e^{ z_1}
        \end{equation*}
    \end{excer}

    \begin{proof}
        En efecto, veamos que:
        \begin{equation*}
            \begin{split}
                e^{ x+iz_2}&=\sum_{ n=0}^\infty\frac{(z_1+z_2)^n}{n!}\\
                &=\sum_{ n=0}^\infty\frac{1}{n!}\sum_{ k=0}^n\binom{n}{k}z_1^kz_2^{ n-k}\\
                &=\sum_{ n=0}^\infty\sum_{ k=0}^n\frac{1}{n!}\cdot\frac{n!}{k!(n-k)!}z_1^kz_2^{ n-k}\\
                &=\sum_{ n=0}^\infty\sum_{ k=0}^n\frac{z_1^k}{k!}\cdot\frac{z_2^{n-k}}{(n-k)!}\\
            \end{split}
        \end{equation*}
        donde, recordemos que:
        \begin{equation*}
            e^z_1=\sum_{ n=0}^\infty\frac{z_1^n}{n!}\quad\textup{y}\quad e^ {z_2}=\sum_{ n=0}^\infty\frac{z_2^n}{n!}
        \end{equation*}
        por tanto, de la Proposición \ref{MultSumSeriesPot} se sigue que:
        \begin{equation*}
            e^{ z_1+z_2}=e^z_1e^{ z_2}
        \end{equation*}
    \end{proof}

    \begin{excer}
        Pruebe que
        \begin{equation*}
            e^z=e^x(\cos y+i\sin y)
        \end{equation*}
        donde $z=x+iy$.
    \end{excer}

    \begin{proof}
        Sea $z\in\mathbb{C}$. Se tiene que:
        \begin{equation*}
            \begin{split}
                e^z&=e^{x}e^{iy}
            \end{split}
        \end{equation*}
        Veamos que:
        \begin{equation*}
            \begin{split}
                e^{ iy}&=\sum_{n=0}^\infty\frac{(iy)^n}{n!}\\
                &=\sum_{n=0}^\infty\frac{(iy)^{2k}}{(2k)!}+\sum_{k=0}^\infty\frac{(iy)^{2k+1}}{(2k+1)!}\\
            \end{split}
        \end{equation*}
    \end{proof}

    \begin{excer}
        \label{productoSucesionesLimSupLim}
        Pruebe que si $\left\{a_n\right\}_{ n=1}^\infty$ y $\left\{b_n\right\}_{ n=1}^\infty$ son dos sucesiones de números no negativos tales que $0\leq b=\lim_{ n\rightarrow\infty}b_n$ y $a=\limsup_{ n\rightarrow\infty}a_n$, entonces:
        \begin{equation*}
            \limsup_{ n\rightarrow\infty}(a_nb_n)=ab
        \end{equation*}
    \end{excer}

    \begin{proof}
        Antes, notemos que al tenerse:
        \begin{equation*}
            \limsup_{ n\rightarrow\infty}a_n=a
        \end{equation*}
        se tiene que el siguiente límite existe:
        \begin{equation*}
            \lim_{ n\rightarrow\infty}\left(\sup_{ k\geq n}a_k \right)
        \end{equation*}
        por tanto, la sucesión $\left\{\sup_{ k\geq n}a_k\right\}_{ n=1}^\infty$ es acotada. Así que, para cada $n\in\mathbb{N}$, se tiene que el supremo:
        \begin{equation*}
            \sup_{ k\geq n}a_k
        \end{equation*}
        existe. Veamos ahora que:
        \begin{equation*}
            \begin{split}
                \limsup_{ n\rightarrow\infty}(a_nb_n)&=\lim_{ n\rightarrow\infty}\left(\sup_{ k\geq n}a_kb_k \right)\\
                &=\lim_{ n\rightarrow\infty}\left(\left(\sup_{ k\geq n}a_k\right)\left(\sup_{ k\geq n}b_k\right) \right)\\
            \end{split}
        \end{equation*}
        donde el supremo se puede separar ya que ambos supremos existen y ser las dos sucesiones acotadas y de números no negativos. Por ende:
        \begin{equation*}
            \begin{split}
                \limsup_{ n\rightarrow\infty}(a_nb_n)&=\lim_{ n\rightarrow\infty}\left(\left(\sup_{ k\geq n}a_k\right)\left(\sup_{ k\geq n}b_k\right) \right)\\
                &=\lim_{ n\rightarrow\infty}\left(\sup_{ k\geq n} a_k\right)\cdot\lim_{ n\rightarrow\infty}\left(\sup_{ k\geq n}b_k\right)\\
                &=\left(\limsup_{ n\rightarrow\infty}a_n\right)\cdot\left(\limsup_{ n\rightarrow\infty}b_n\right)\\
                &=\left(\limsup_{ n\rightarrow\infty}a_n\right)\cdot\left(\lim_{ n\rightarrow\infty}b_n\right)\\
                &=ab\\
            \end{split}
        \end{equation*}
    \end{proof}

    \begin{excer}
        \label{normalDerivadaComplejaDerivadaCompos}
        Sea $\cf{f}{G}{\bbm{C}}$ analítica en $G$ y $\cf{h}{[0,1]}{G}$ función real diferenciable. Entonces la función $\cf{f\circ h}{[0,1]}{\bbm{C}}$ cumple que:
        \begin{equation*}
            \lim_{ t\rightarrow s}\frac{f\circ h(s)-f\circ h(t)}{s-t}=f'(h(s))\cdot h'(s)
        \end{equation*}
        para todo $s\in[0,1]$.
    \end{excer}

    \subsection{Funciones Analíticas}

    \chapter*{Bibliografía}

    \begin{itemize}
        \item A. Markusevich, \textit{Teoría de las funciones analíticas}, Ed. Mir Moscu.
        \item J. Conway, \textit{Complex Analysis}, Ed. Mir Moscu.
    \end{itemize}
    
\end{document}