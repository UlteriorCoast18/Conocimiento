\documentclass[12pt]{report}
\usepackage[spanish]{babel}
\usepackage[utf8]{inputenc}
\usepackage{amsmath}
\usepackage{amssymb}
\usepackage{amsthm}
\usepackage{graphics}
\usepackage{subfigure}
\usepackage{lipsum}
\usepackage{array}
\usepackage{multicol}
\usepackage{enumerate}
\usepackage[framemethod=TikZ]{mdframed}
\usepackage[a4paper, margin = 1.5cm]{geometry}

%En esta parte se hacen redefiniciones de algunos comandos para que resulte agradable el verlos%

\renewcommand{\theenumii}{\roman{enumii}}

\def\proof{\paragraph{Demostración:\\}}
\def\endproof{\hfill$\blacksquare$}

\def\sol{\paragraph{Solución:\\}}
\def\endsol{\hfill$\square$}

%En esta parte se definen los comandos a usar dentro del documento para enlistar%

\newtheoremstyle{largebreak}
  {}% use the default space above
  {}% use the default space below
  {\normalfont}% body font
  {}% indent (0pt)
  {\bfseries}% header font
  {}% punctuation
  {\newline}% break after header
  {}% header spec

\theoremstyle{largebreak}

\newmdtheoremenv[
    leftmargin=0em,
    rightmargin=0em,
    innertopmargin=-2pt,
    innerbottommargin=8pt,
    hidealllines = true,
    roundcorner = 5pt,
    backgroundcolor = gray!60!red!30
]{exa}{Ejemplo}[section]

\newmdtheoremenv[
    leftmargin=0em,
    rightmargin=0em,
    innertopmargin=-2pt,
    innerbottommargin=8pt,
    hidealllines = true,
    roundcorner = 5pt,
    backgroundcolor = gray!50!blue!30
]{obs}{Observación}[section]

\newmdtheoremenv[
    leftmargin=0em,
    rightmargin=0em,
    innertopmargin=-2pt,
    innerbottommargin=8pt,
    rightline = false,
    leftline = false
]{theor}{Teorema}[section]

\newmdtheoremenv[
    leftmargin=0em,
    rightmargin=0em,
    innertopmargin=-2pt,
    innerbottommargin=8pt,
    rightline = false,
    leftline = false
]{propo}{Proposición}[section]

\newmdtheoremenv[
    leftmargin=0em,
    rightmargin=0em,
    innertopmargin=-2pt,
    innerbottommargin=8pt,
    rightline = false,
    leftline = false
]{cor}{Corolario}[section]

\newmdtheoremenv[
    leftmargin=0em,
    rightmargin=0em,
    innertopmargin=-2pt,
    innerbottommargin=8pt,
    rightline = false,
    leftline = false
]{lema}{Lema}[section]

\newmdtheoremenv[
    leftmargin=0em,
    rightmargin=0em,
    innertopmargin=-2pt,
    innerbottommargin=8pt,
    roundcorner=5pt,
    backgroundcolor = gray!30,
    hidealllines = true
]{mydef}{Definición}[section]

\newmdtheoremenv[
    leftmargin=0em,
    rightmargin=0em,
    innertopmargin=-2pt,
    innerbottommargin=8pt,
    roundcorner=5pt
]{excer}{Ejercicio}[section]

%En esta parte se colocan comandos que definen la forma en la que se van a escribir ciertas funciones%

\newcommand\abs[1]{\ensuremath{\left|#1\right|}}
\newcommand\divides{\ensuremath{\bigm|}}
\newcommand\cf[3]{\ensuremath{#1:#2\rightarrow#3}}
\newcommand\natint[1]{\ensuremath{\left[\!\left[ #1\right]\!\right]}}
\newcommand{\afa}{\:
    \begin{tikzpicture}
        \draw [line width = 0.17 mm, black] (0,0) -- (-0.115,0.29);
        \draw [line width = 0.17 mm, black] (0,0) -- (0.115,0.29);
        \draw [line width = 0.17 mm, black] (-0.12,0) arc (190:-10:0.12cm);
    \end{tikzpicture}
    \:
}
\newcommand{\conj}[1]{\ensuremath{\overline{#1}}}
%Este símvolo es para casi todo salvo una cantidad finita

%recuerda usar \clearpage para hacer un salto de página

\begin{document}
    \setlength{\parskip}{5pt} % Añade 5 puntos de espacio entre párrafos
    \setlength{\parindent}{12pt} % Pone la sangría como me gusta
    \title{Notas, Ejercicios y Problemas sobre funciones de variable compleja}
    \author{Cristo Daniel Alvarado}
    \maketitle

    \tableofcontents %Con este comando se genera el índice general del libro%

    \renewcommand{\theenumi}{\roman{enumi}}

    %\setcounter{chapter}{3} %En esta parte lo que se hace es cambiar la enumeración del capítulo%
    
    \chapter{Ejercicios y Problemas}
    
    \section{Parte 1}
    
    Primero se verán algunas definiciones fundamentales para poder entender los ejercicios.
    
    \subsection{Problemas y Ejercicios}

    \begin{excer}
        Encuentre los números complejos tales que sus conjugados son iguales a:
        \begin{enumerate}
            \item Sus cuadrados.
            \item Sus cubos.
        \end{enumerate}
    \end{excer}

    \begin{sol}
        De (i): Consideremos un número complejo $z=a+ib\in\mathbb{C}$ tal que
        \begin{equation*}
            \begin{split}
                &\conj{z}=z^2\\
                \iff& a-ib=(a^2-b^2)+2iab\\
                \iff& \left\{
                    \begin{array}{rl}
                        a=& a^2-b^2\\
                        -b=&2ab\\
                    \end{array}
                \right.\\
                \iff& \left\{
                    \begin{array}{rl}
                        a=& a^2-b^2\\
                        (2a-1)b=&0\\
                    \end{array}
                \right.\\
            \end{split}
        \end{equation*}
        de la segunda ecuación se deduce que $a=\frac{1}{2}$ o $b=0$. Si $a=\frac{1}{2}$, entonces
        \begin{equation*}
            \begin{split}
                a^2-b^2&=a\\
                \iff b^2&=a^2-a\\
                \iff b^2&=\frac{1}{4}-\frac{1}{2}\\
                \iff b^2&=-\frac{1}{4}\\
            \end{split}
        \end{equation*}
        siendo $b\in\mathbb{R}$, tal cosa no puede suceder, por lo que $b=0$, lo cual implica que
        \begin{equation*}
            \begin{split}
                a^2-b^2&=a\\
                \iff a^2&=a\\
                \iff a(a-1)=0\\
            \end{split}
        \end{equation*}
        es decir, si $a=0$ o si $a=1$.

        De (ii): Consideremos un número complejo $z=a+ib\in\mathbb{C}$ tal que
        \begin{equation*}
            \begin{split}
                &\conj{z}=z^3\\
                \iff& a-ib=a^3+3ia^2b-3ab^2-ib^3\\
                \iff& a-ib=a^3-3ab^2+i(3a^2b-b^3)\\
                \iff& \left\{
                    \begin{array}{rl}
                        a = & a^3-3ab^2\\
                        -b = & 3a^2b-b^3\\
                    \end{array}
                \right.\\
                \iff& \left\{
                    \begin{array}{rl}
                        a(a^2-3b^2-1) = & 0 \\
                        b(3a^2-b^2+1) = & 0 \\
                    \end{array}
                \right.\\
            \end{split}
        \end{equation*}
        Analicemos por casos cada ecuación.
        \begin{itemize}
            \item Suponga que $a=0$, entonces:
            \begin{equation*}
                \begin{split}
                    b(1-b^2)&=0\\
                \end{split}
            \end{equation*}
            luego $b=0$ o $1-b^2=0\Rightarrow b^2=1$, es decir que $z=0$ ó $z=\pm i$.
            \item Suponga que $a^2-3b^2-1=0$, entonces $a^2=3b^2+1$. Por tanto:
            \begin{equation*}
                \begin{split}
                    b(3a^2-b^2+1) = & 0 \\
                    \Rightarrow b(9b^2+3-b^2+1)=0\\
                    \Rightarrow b(9b^2+3-b^2+1)=0\\
                    \Rightarrow 4b(2b^2+1)=0\\
                    \Rightarrow 4b(2b^2+1)=0\\
                \end{split}
            \end{equation*}
            esto es, $b=0$ o $2b^2+1=0$, el segundo caso no puede ocurrir pues $b\in\mathbb{R}$, luego si $b=0$ se tiene que $a^2=1$, es decir que $z=\pm1$.
        \end{itemize}
        En resumen, $z$ toma uno y sólo uno de los valores del conjunto $\left\{0,-1,1,-i,i \right\}$.
    \end{sol}

    \begin{excer}
        Suponga que un número complejo $u\in\mathbb{C}$ es obtenido a partir de aplicar un número finito de veces operaciones racionales a los números complejos $z_1,...,z_n\in\mathbb{C}$. Pruebe que las mismas operaciones aplicadas a $\conj{z_1},...,\conj{z_n}\in\mathbb{C}$ resultan en $\conj{u}$. 
    \end{excer}

    \begin{proof}
        Notemos que cada operación (suma y multiplicación) son operaciones binarias y que, el aplicar un número finito de veces operaciones racionales a los números complejos $z_1,...,z_n$ es equivalente a componer un número finito de veces las operaciones binarias de suma y multiplicación (tomando suma de inverso aditivo en caso de la resta y multiplicación por inverso multiplicativo en caso de la división), por lo que para probar el resultado basta con probar que
        \begin{equation*}
            \conj{x+y}=\conj{x}+\conj{y}\quad\textup{y}\quad\conj{x\cdot y}=\conj{x}\cdot\conj{y},\quad\forall x,y\in\mathbb{C}
        \end{equation*}
        En efecto, sean $x=a+ib$ y $y=c+id$ números complejos. Entonces:
        \begin{equation*}
            \begin{split}
                \conj{x+y}&=\conj{a+ib+c+id}\\
                &=\conj{(a+c)+i(b+d)}\\
                &=a+c-i(b+d)\\
                &=(a-ib)+(c-id)\\
                &=\conj{x}+\conj{y}\\
            \end{split}
        \end{equation*}
        (de forma análoga se prueba la otra igualdad).
    \end{proof}

    \begin{excer}
        Por un argumento puramente geométrico, pruebe la desigualdad:
        \begin{equation*}
            \abs{z-1}\leq\abs{\abs{z}-1}+\abs{z}\abs{\arg z}
        \end{equation*}
    \end{excer}
    
    \begin{proof}
        
    \end{proof}

    \begin{excer}
        Resuelva las siguientes ecuaciones:
        \begin{itemize}
            \item $\abs{z}-z=1+2i$.
            \item $\abs{z}+z=2+i$.
        \end{itemize}
    \end{excer}

    \begin{sol}
        De (i): Suponga que existe $z=a+ib\in\mathbb{C}$ tal que
        \begin{equation*}
            \abs{z}-z=1+2i
        \end{equation*}
        entonces,
        \begin{equation*}
            \begin{split}
                &\sqrt{a^2+b^2}-a-ib=1+2i\\
                \iff& \left\{\begin{array}{rcl}
                    \sqrt{a^2+b^2}-a&=&1\\
                    -b&=&2\\
                \end{array} \right.\\
                \iff& \left\{\begin{array}{rcl}
                    \sqrt{a^2+b^2}-a&=&1\\
                    b&=&-2\\
                \end{array} \right.\\
                \iff& \left\{\begin{array}{rcl}
                    \sqrt{a^2+4}&=&1+a\\
                    b&=&-2\\
                \end{array} \right.\\
                \iff& \left\{\begin{array}{rcl}
                    a^2+4&=&1+2a+a^2\\
                    b&=&-2\\
                \end{array} \right.\\
                \iff& \left\{\begin{array}{rcl}
                    a^2+4&=&1+2a+a^2\\
                    b&=&-2\\
                \end{array} \right.\\
                \iff& \left\{\begin{array}{rcl}
                    2a&=&3\\
                    b&=&-2\\
                \end{array} \right.\\
                \iff& \left\{\begin{array}{rcl}
                    a&=&\frac{3}{2}\\
                    b&=&-2\\
                \end{array} \right.\\
            \end{split}
        \end{equation*}
        por tanto, $z=\frac{3}{2}-2i$.

        De (ii): Suponga que...
    \end{sol}

    \begin{excer}
        Pruebe que todo número excepto $-1$ de módulo unitario puede ser expresado en la forma
        \begin{equation*}
            z=\frac{1+it}{1-it}
        \end{equation*}
        donde $t\in\mathbb{R}$.
    \end{excer}

    \begin{proof}
        Sea $z\in\mathbb{C}$ y tomemos $t=\tan\frac{\arg z}{2}$. Se tiene que $t\in\mathbb{R}$ si $z\neq-1$ (ya que $\frac{\arg z}{2}=\frac{p}{2}$, el único punto donde no está definida ya que $0\leq\frac{\arg z}{2}\leq\pi$). Además, a cada punto de módulo unitario le corresponde uno y sólo un argumento $\arg z$.

        Afirmamos que se cumple la igualdad. En efecto:
        \begin{equation*}
            \begin{split}
                \frac{1+it}{1-it}&=\frac{1+it}{1-it}\cdot\frac{1+it}{1+it}\\
                &=\frac{1-t^2+2it}{1+t^2}\\
                &=\frac{1-\tan^2\frac{\arg z}{2}}{1+\tan^2\frac{\arg z}{2}}+i\frac{2\tan\frac{\arg z}{2}}{1+\tan^2\frac{\arg z}{2}}\\
                &=\cos\left(2\cdot\frac{\arg z}{2}\right)+i\sin\left(2\cdot\frac{\arg z}{2}\right)\\
                &=\cos\arg z+i\sin\arg z\\
                &=z\\
            \end{split}
        \end{equation*}
        pues, para todo $\theta\in\mathbb{R}$ (para el que todo esté bien definido), se cumplen las igualdades:
        \begin{equation*}
            \left\{
                \begin{array}{rl}
                    \sin\theta &= \frac{2\tan\frac{\theta}{2}}{1+\tan^2\frac{\theta}{2}}\\
                    \cos\theta &= \frac{1-\tan^2\frac{\theta}{2}}{1+\tan^2\frac{\theta}{2}}\\
                    \tan\theta &= \frac{2\tan^2\frac{\theta}{2}}{1-\tan^2\frac{\theta}{2}}\\
                \end{array}
            \right.
        \end{equation*}
        %TODO: Una demostración de este hecho no estaría mal.
        Lo que prueba el resultado.
    \end{proof}

    \begin{excer}
        Pruebe que si $\abs{z}<\frac{1}{2}$, entonces $\abs{(1+i)z^3+iz}<\frac{3}{4}$.
    \end{excer}

    \begin{proof}
        Sea $z\in\mathbb{C}$ tal que $\abs{z}<\frac{1}{2}$, se tiene que
        \begin{equation*}
            \begin{split}
                \abs{(1+i)z^3+iz}&\leq\abs{(1+i)z^3}+\abs{iz}\\
                &\leq\abs{1+i}\abs{z^3}+\abs{i}\abs{z}\\
                &<(\abs{1}+\abs{i})\abs{z}^3+\frac{1}{2}\\
                &<\frac{2}{2^3}+\frac{1}{2}\\
                &\leq\frac{1}{4}+\frac{1}{2}\\
                &=\frac{3}{4}\\
            \end{split}
        \end{equation*}
    \end{proof}

    \begin{excer}
        Pruebe que
        \begin{equation*}
            \arg\frac{z_3-z_1}{z_3-z_2}=\frac{1}{2}\arg\frac{z_2}{z_1}
        \end{equation*}
        siendo $z_1,z_2,z_3\in\mathbb{C}$ tales que $\abs{z_1}=\abs{z_2}=\abs{z_3}$.
    \end{excer}

    \begin{proof}
        Considere los tres números escritos en su forma trigonométrica, es decir:
        \begin{equation*}
            \left\{
                \begin{split}
                    z_1&=r_1\cos\theta_1+i\sin\theta_1\\
                    z_2&=r_1\cos\theta_2+i\sin\theta_2\\
                    z_3&=r_1\cos\theta_3+i\sin\theta_3\\
                \end{split}
            \right.
        \end{equation*}
        Entonces,
        \begin{equation*}
            \begin{split}
                \arg\frac{z_2}{z_1}&=\arg\left(\frac{\abs{z_2}}{\abs{z_1}}\left[\cos\left(\theta_2-\theta_1\right)+i\sin\left(\theta_2-\theta_1\right)\right]\right) \\
                &=\arg\left(\cos\left(\theta_2-\theta_1\right)+i\sin\left(\theta_2-\theta_1\right)\right) \\
                &=\theta_2-\theta_1\\
            \end{split}
        \end{equation*}
        pues, $\abs{z_1}\neq0$ y se tiene la igualdad de ambos módulos.

        Ahora, también se cumple:
        \begin{equation*}
            \begin{split}
                \frac{z_3-z_1}{z_3-z_2}&=
            \end{split}
        \end{equation*}
    \end{proof}
    
    

\end{document}