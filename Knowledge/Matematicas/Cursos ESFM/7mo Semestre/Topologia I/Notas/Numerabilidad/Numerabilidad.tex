\documentclass[12pt]{report}
\usepackage[spanish]{babel}
\usepackage[utf8]{inputenc}
\usepackage{amsmath}
\usepackage{amssymb}
\usepackage{amsthm}
\usepackage{graphics}
\usepackage{subfigure}
\usepackage{lipsum}
\usepackage{array}
\usepackage{multicol}
\usepackage{enumerate}
\usepackage[framemethod=TikZ]{mdframed}
\usepackage[a4paper, margin = 1.5cm]{geometry}

%En esta parte se hacen redefiniciones de algunos comandos para que resulte agradable el verlos%

\renewcommand{\theenumii}{\roman{enumii}}

\def\proof{\paragraph{Demostración:\\}}
\def\endproof{\hfill$\blacksquare$}

\def\sol{\paragraph{Solución:\\}}
\def\endsol{\hfill$\square$}

%En esta parte se definen los comandos a usar dentro del documento para enlistar%

\newtheoremstyle{largebreak}
  {}% use the default space above
  {}% use the default space below
  {\normalfont}% body font
  {}% indent (0pt)
  {\bfseries}% header font
  {}% punctuation
  {\newline}% break after header
  {}% header spec

\theoremstyle{largebreak}

\newmdtheoremenv[
    leftmargin=0em,
    rightmargin=0em,
    innertopmargin=-2pt,
    innerbottommargin=8pt,
    hidealllines = true,
    roundcorner = 5pt,
    backgroundcolor = gray!60!red!30
]{exa}{Ejemplo}[section]

\newmdtheoremenv[
    leftmargin=0em,
    rightmargin=0em,
    innertopmargin=-2pt,
    innerbottommargin=8pt,
    hidealllines = true,
    roundcorner = 5pt,
    backgroundcolor = gray!50!blue!30
]{obs}{Observación}[section]

\newmdtheoremenv[
    leftmargin=0em,
    rightmargin=0em,
    innertopmargin=-2pt,
    innerbottommargin=8pt,
    rightline = false,
    leftline = false
]{theor}{Teorema}[section]

\newmdtheoremenv[
    leftmargin=0em,
    rightmargin=0em,
    innertopmargin=-2pt,
    innerbottommargin=8pt,
    rightline = false,
    leftline = false
]{propo}{Proposición}[section]

\newmdtheoremenv[
    leftmargin=0em,
    rightmargin=0em,
    innertopmargin=-2pt,
    innerbottommargin=8pt,
    rightline = false,
    leftline = false
]{cor}{Corolario}[section]

\newmdtheoremenv[
    leftmargin=0em,
    rightmargin=0em,
    innertopmargin=-2pt,
    innerbottommargin=8pt,
    rightline = false,
    leftline = false
]{lema}{Lema}[section]

\newmdtheoremenv[
    leftmargin=0em,
    rightmargin=0em,
    innertopmargin=-2pt,
    innerbottommargin=8pt,
    roundcorner=5pt,
    backgroundcolor = gray!30,
    hidealllines = true
]{mydef}{Definición}[section]

\newmdtheoremenv[
    leftmargin=0em,
    rightmargin=0em,
    innertopmargin=-2pt,
    innerbottommargin=8pt,
    roundcorner=5pt
]{excer}{Ejercicio}[section]

%En esta parte se colocan comandos que definen la forma en la que se van a escribir ciertas funciones%

\newcommand\abs[1]{\ensuremath{\left|#1\right|}}
\newcommand\divides{\ensuremath{\bigm|}}
\newcommand\cf[3]{\ensuremath{#1:#2\rightarrow#3}}
\newcommand\contradiction{\ensuremath{\#_c}}
\newcommand{\V}[1]{\ensuremath{\mathcal{V}(#1)}}
\newcommand{\Int}[1]{\ensuremath{\mathring{#1}}}
\newcommand{\Cls}[1]{\ensuremath{\overline{#1}}}
\newcommand{\Fr}[1]{\ensuremath{\textup{Fr}(#1)}}
\newcommand{\natint}[1]{\ensuremath{\left[\!\left[#1\right]\!\right]}}
\newcommand{\floor}[1]{\ensuremath{\lfloor#1\rfloor}}
\newcommand{\Card}[1]{\ensuremath{\textup{Card}\left(#1\right)}}
\newcommand{\Pot}[1]{\ensuremath{\mathcal{P}\left(#1\right)}}
\newcommand{\id}[1]{\ensuremath{\textup{id}_{#1}}}

%recuerda usar \clearpage para hacer un salto de página

\begin{document}
    \setlength{\parskip}{5pt} % Añade 5 puntos de espacio entre párrafos
    \setlength{\parindent}{12pt} % Pone la sangría como me gusta
    \title{Notas Curso Topología I
    
    Axiomas de Numerabilidad}
    \author{Cristo Daniel Alvarado}
    \maketitle

    \tableofcontents %Con este comando se genera el índice general del libro%

    \setcounter{chapter}{4} %En esta parte lo que se hace es cambiar la enumeración del capítulo%
    
    \chapter{Axiomas de Numerabilidad}
    
    \section{Conceptos Fundamentales}

    \begin{obs}
        De ahora en adelante numerable será equivalente a lo sumo numerable.
    \end{obs}

    \begin{mydef}
        Sea $(X,\tau)$ un espacio topológico.
        \begin{enumerate}
            \item  Sean $x\in X$ y $\mathcal{U}$ una colección de vecindades de $x$. Diremos que $\mathcal{U}$ es un \textbf{sistema fundamental de vecindades de $x$} si dada $V\in\mathcal{V}(x)$ existe $U\in\mathcal{U}$ tal que $U\subseteq V$. Si $\mathcal{U}$ es numerable, $\mathcal{U}$ se dice un \textbf{sistema fundamental numerable de vecindades de $x$}.
            \item Si dado $x\in X$ existe un sistema fundamental numerable de vecindades de $x$, el espacio $(X,\tau)$ se dice \textbf{primero numerable}.
            \item El espacio $(X,\tau)$ se dice un \textbf{espacio segundo numerable} si su topología tiene una base numerable.
            \item El espacio $(X,\tau)$ se dice un \textbf{espacio separable} si existe $A\subseteq X$ tal que $A$ es numerable y además $\Cls{A}=X$ (es decir que es denso en $X$).
            \item El espacio $(X,\tau)$ se dice un \textbf{espacio de Lindelöf} si cada cubierta abierta del espacio tiene una subcubierta numerable.
        \end{enumerate}
    \end{mydef}

    \section{Espacios Primero Numerables}

    \begin{propo}
        Sea $(X,\tau)$ un espacio primero numerable. Si $Y\subseteq X$ entonces $(Y,\tau_Y)$ es primero numerable.
    \end{propo}

    \begin{proof}
        Sea $Y\subseteq X$. Sea $y\in Y$, en particular $y\in X$. Como $(X,\tau)$ es primero numeable, existe un sistema fundamental de vecindades de $x$ en $(X,\tau)$, digamos $\mathcal{U}'$, es decir que para este $\mathcal{U}'$ se cumple:
        \begin{equation*}
            \forall V\in\mathcal{V}(x)\exists U\in\mathcal{U}'\textup{ tal que }U\subseteq V
        \end{equation*}
        Sea
        \begin{equation*}
            \mathcal{U}=\left\{Y\cap U\Big|U\in\mathcal{U}' \right\}
        \end{equation*}
        Tenemos que $U\in\mathcal{U}'$, $Y\cap U$ es una vecindad de $y$ en $(Y,\tau_Y)$ y, como $\mathcal{U}'$ es numerable, también $\mathcal{U}$ lo es.

        Sea $W\subseteq Y$ una vecindad de $y$ en $(Y,\tau_Y)$, luego existe $V\in\tau$ tal que
        \begin{equation*}
            y\in Y\cap V\subseteq W
        \end{equation*}
        Como en particular $V$ es una vecindad de $y$ en $(X,\tau)$, entonces existe $U\in\mathcal{U}'$ tal que
        \begin{equation*}
            U\subseteq V
        \end{equation*}
        luego,
        \begin{equation*}
            Y\cap U\subseteq Y\cap V\subseteq W
        \end{equation*} 
        donde $Y\cap U\in\mathcal{U}$. Así, $\mathcal{U}$ es un sistema fundamental de vecindades de $y$ en $(Y,\tau_Y)$. Como $y\in Y$ fue arbitrario, se sigue que $(Y,\tau_Y)$ es primero numerable.
    \end{proof}

    \begin{propo}
        La propiedad de ser primero numerable es topológica.
    \end{propo}

    \begin{proof}
        Sean $(X_1,\tau_1)$ y $(X_2,\tau_2)$ espacios topológicos homeomorfos tales que $(X_1,\tau_1)$ es primero numerable. Sea $\cf{f}{(X_1,\tau_1)}{(X_2,\tau_2)}$ el homeomorfismo entre tales espacios. Veamos que $(X_2,\tau_2)$ es primero numeable.

        En efecto, sea $x_2\in X_2$, entonces existe un único $x_1\in X_1$ tal que $f(x_1)=x_2$. Como $(X_1,\tau_1)$ es primero numerable, entonces existe $\mathcal{U}_1$ sistema fundamental numerable de vecindades de $x_1$. Sea
        \begin{equation*}
            \mathcal{U}_2=\left\{f(U_1)\Big|U_1\in\mathcal{U}_1 \right\}
        \end{equation*}
        Como $\mathcal{U}_1$ es numerable, $\mathcal{U}_2$ también lo es. Y, como $U_1\in\mathcal{U}_1$ es una vecindad de $x_1$, entonces $f(U_1)$ es una vecindad de $x_2$ (por ser $f$ homeomorfismo). Por tanto, $\mathcal{U}_2$ es una colección de vecindades de $x_2$. Ahora, sea $V\in\mathcal{V}(x_2)$ una vecindad de $x_2$. Como $f$ es homeomorfismo entonces
        \begin{equation*}
            f^{-1}(V)\in\mathcal{V}(x_1)
        \end{equation*}
        Luego, existe $U\in\mathcal{U}_1$ tal que
        \begin{equation*}
            U\subseteq f^{-1}(V)\Rightarrow f(U)\subseteq V
        \end{equation*}
        por ser $f$ biyección, donde $f(U)\in\mathcal{U}_2$.

        Así, $\mathcal{U}_2$ es un sistema fundamental numerable de vecindades de $x_2$. Como el elemento $x_2$ fue arbitrario, se sigue que $(X_2,\tau_2)$ es primero numerable. Luego, la propiedad de ser primero numerable es topológica.
    \end{proof}

    \begin{propo}
        Sean $\left\{(X_k,\tau_k) \right\}_{ k\in\mathbb{N}}$ una familia numerable de espacios topológicos y
        \begin{equation*}
            X=\prod_{ k\in\mathbb{N}}X_k
        \end{equation*}
        Entonces, $(X,\tau_p)$ es primero numerable si y sólo si $(X_k,\tau_k)$ es primero numerable, para todo $k\in\mathbb{N}$.
    \end{propo}

    \begin{proof}
        $\Rightarrow)$: Es inmediato del hecho de que la propiedad de ser primero numerable es hereditaria y topológica.

        $\Leftarrow)$: Suponga que $(X_k,\tau_k)$ es primero numerable para todo $k\in\mathbb{N}$. Sea $x=(x_n)_{n\in\mathbb{N}}\in X$. Si $k\in\mathbb{N}$, se tiene que $(X_k,\tau_k)$ es primero numerable. Para $x_k\in X_k$ existe
        \begin{equation*}
            \mathcal{U}_k=\left\{U_m^k \right\}_{ m\in\mathbb{N}}
        \end{equation*}
        sistema fundamental numerable de vecindades de $x_k$ en $(X_k,\tau_k)$. Definimos
        \begin{equation*}
            \begin{split}
                \mathcal{U}=\left\{\prod_{ l\in\mathbb{N}}A_l\Big|\right.&\textup{ existe }I=\left\{i_1,...,i_t \right\} \subseteq\mathbb{N}\textup{ finito con }i_r<i_s\textup{ si }r<s\textup{ tal que } \\
                &\left.l\in\mathbb{N}-I\Rightarrow A_l=X_l\textup{ y }l\in I\Rightarrow A_k\in\mathcal{U}_l \right\}\\
            \end{split}
        \end{equation*}
        veamos que $\mathcal{U}\subseteq\mathcal{V}(x)$ y además $\mathcal{U}$  es un sistema fundamental de vecindades de $x$. Sea $U=\prod_{t\in\mathbb{N}}U_t$ un básico de la topología producto tal que $x\in U$. Tenemos que existe $I\subseteq\mathbb{N}$ finito tal que
        \begin{equation*}
            l\in\mathbb{N}-I\Rightarrow U_l=X_l\textup{ y }l\in I\Rightarrow x_l\in U_l\in\tau_l
        \end{equation*}
        Para $l\in I$ existe $U_{ m_l}^l\in\mathcal{U}_l$ tal que $x_l\in U_{ m_l}^l\subseteq U_l$. Sea
        \begin{equation*}
            A=\prod_{ l\in\mathbb{N}}A_l
        \end{equation*}
        donde,
        \begin{equation*}
            l\in\mathbb{N}-I\Rightarrow A_l=X_l\textup{ y }l\in I\Rightarrow A_l=U_{ m_l}^l
        \end{equation*}
        por tanto, $A\in\mathcal{U}$ y es tal que $x\in A\subseteq U$.
    \end{proof}

\end{document}