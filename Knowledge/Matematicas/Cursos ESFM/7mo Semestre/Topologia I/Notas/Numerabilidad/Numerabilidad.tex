\documentclass[12pt]{report}
\usepackage[spanish]{babel}
\usepackage[utf8]{inputenc}
\usepackage{amsmath}
\usepackage{amssymb}
\usepackage{amsthm}
\usepackage{graphics}
\usepackage{subfigure}
\usepackage{lipsum}
\usepackage{array}
\usepackage{multicol}
\usepackage{enumerate}
\usepackage[framemethod=TikZ]{mdframed}
\usepackage[a4paper, margin = 1.5cm]{geometry}

%En esta parte se hacen redefiniciones de algunos comandos para que resulte agradable el verlos%

\renewcommand{\theenumii}{\roman{enumii}}

\def\proof{\paragraph{Demostración:\\}}
\def\endproof{\hfill$\blacksquare$}

\def\sol{\paragraph{Solución:\\}}
\def\endsol{\hfill$\square$}

%En esta parte se definen los comandos a usar dentro del documento para enlistar%

\newtheoremstyle{largebreak}
  {}% use the default space above
  {}% use the default space below
  {\normalfont}% body font
  {}% indent (0pt)
  {\bfseries}% header font
  {}% punctuation
  {\newline}% break after header
  {}% header spec

\theoremstyle{largebreak}

\newmdtheoremenv[
    leftmargin=0em,
    rightmargin=0em,
    innertopmargin=-2pt,
    innerbottommargin=8pt,
    hidealllines = true,
    roundcorner = 5pt,
    backgroundcolor = gray!60!red!30
]{exa}{Ejemplo}[section]

\newmdtheoremenv[
    leftmargin=0em,
    rightmargin=0em,
    innertopmargin=-2pt,
    innerbottommargin=8pt,
    hidealllines = true,
    roundcorner = 5pt,
    backgroundcolor = gray!50!blue!30
]{obs}{Observación}[section]

\newmdtheoremenv[
    leftmargin=0em,
    rightmargin=0em,
    innertopmargin=-2pt,
    innerbottommargin=8pt,
    rightline = false,
    leftline = false
]{theor}{Teorema}[section]

\newmdtheoremenv[
    leftmargin=0em,
    rightmargin=0em,
    innertopmargin=-2pt,
    innerbottommargin=8pt,
    rightline = false,
    leftline = false
]{propo}{Proposición}[section]

\newmdtheoremenv[
    leftmargin=0em,
    rightmargin=0em,
    innertopmargin=-2pt,
    innerbottommargin=8pt,
    rightline = false,
    leftline = false
]{cor}{Corolario}[section]

\newmdtheoremenv[
    leftmargin=0em,
    rightmargin=0em,
    innertopmargin=-2pt,
    innerbottommargin=8pt,
    rightline = false,
    leftline = false
]{lema}{Lema}[section]

\newmdtheoremenv[
    leftmargin=0em,
    rightmargin=0em,
    innertopmargin=-2pt,
    innerbottommargin=8pt,
    roundcorner=5pt,
    backgroundcolor = gray!30,
    hidealllines = true
]{mydef}{Definición}[section]

\newmdtheoremenv[
    leftmargin=0em,
    rightmargin=0em,
    innertopmargin=-2pt,
    innerbottommargin=8pt,
    roundcorner=5pt
]{excer}{Ejercicio}[section]

%En esta parte se colocan comandos que definen la forma en la que se van a escribir ciertas funciones%

\newcommand\abs[1]{\ensuremath{\left|#1\right|}}
\newcommand\divides{\ensuremath{\bigm|}}
\newcommand\cf[3]{\ensuremath{#1:#2\rightarrow#3}}
\newcommand\contradiction{\ensuremath{\#_c}}
\newcommand{\V}[1]{\ensuremath{\mathcal{V}(#1)}}
\newcommand{\Int}[1]{\ensuremath{\mathring{#1}}}
\newcommand{\Cls}[1]{\ensuremath{\overline{#1}}}
\newcommand{\Fr}[1]{\ensuremath{\textup{Fr}(#1)}}
\newcommand{\natint}[1]{\ensuremath{\left[\!\left[#1\right]\!\right]}}
\newcommand{\floor}[1]{\ensuremath{\lfloor#1\rfloor}}
\newcommand{\Card}[1]{\ensuremath{\textup{Card}\left(#1\right)}}
\newcommand{\Pot}[1]{\ensuremath{\mathcal{P}\left(#1\right)}}
\newcommand{\id}[1]{\ensuremath{\textup{id}_{#1}}}

%recuerda usar \clearpage para hacer un salto de página

\begin{document}
    \setlength{\parskip}{5pt} % Añade 5 puntos de espacio entre párrafos
    \setlength{\parindent}{12pt} % Pone la sangría como me gusta
    \title{Notas Curso Topología I
    
    Axiomas de Numerabilidad}
    \author{Cristo Daniel Alvarado}
    \maketitle

    \tableofcontents %Con este comando se genera el índice general del libro%

    \setcounter{chapter}{4} %En esta parte lo que se hace es cambiar la enumeración del capítulo%
    
    \chapter{Axiomas de Numerabilidad}
    
    \section{Conceptos Fundamentales}

    \begin{obs}
        De ahora en adelante numerable será equivalente a lo sumo numerable.
    \end{obs}

    \begin{mydef}
        Sea $(X,\tau)$ un espacio topológico.
        \begin{enumerate}
            \item  Sean $x\in X$ y $\mathcal{U}$ una colección de vecindades de $x$. Diremos que $\mathcal{U}$ es un \textbf{sistema fundamental de vecindades de $x$} si dada $V\in\mathcal{V}(x)$ existe $U\in\mathcal{U}$ tal que $U\subseteq V$. Si $\mathcal{U}$ es numerable, $\mathcal{U}$ se dice un \textbf{sistema fundamental numerable de vecindades de $x$}.
            \item Si dado $x\in X$ existe un sistema fundamental numerable de vecindades de $x$, el espacio $(X,\tau)$ se dice \textbf{primero numerable}.
            \item El espacio $(X,\tau)$ se dice un \textbf{espacio segundo numerable} si su topología tiene una base numerable.
            \item El espacio $(X,\tau)$ se dice un \textbf{espacio separable} si existe $A\subseteq X$ tal que $A$ es numerable y además $\Cls{A}=X$ (es decir que es denso en $X$).
            \item El espacio $(X,\tau)$ se dice un \textbf{espacio de Lindelöf} si cada cubierta abierta del espacio tiene una subcubierta numerable.
        \end{enumerate}
    \end{mydef}

    \section{Espacios Primero Numerables}

    \begin{propo}
        Sea $(X,\tau)$ un espacio primero numerable. Si $Y\subseteq X$ entonces $(Y,\tau_Y)$ es primero numerable.
    \end{propo}

    \begin{proof}
        Sea $Y\subseteq X$. Sea $y\in Y$, en particular $y\in X$. Como $(X,\tau)$ es primero numeable, existe un sistema fundamental de vecindades de $x$ en $(X,\tau)$, digamos $\mathcal{U}'$, es decir que para este $\mathcal{U}'$ se cumple:
        \begin{equation*}
            \forall V\in\mathcal{V}(x)\exists U\in\mathcal{U}'\textup{ tal que }U\subseteq V
        \end{equation*}
        Sea
        \begin{equation*}
            \mathcal{U}=\left\{Y\cap U\Big|U\in\mathcal{U}' \right\}
        \end{equation*}
        Tenemos que $U\in\mathcal{U}'$, $Y\cap U$ es una vecindad de $y$ en $(Y,\tau_Y)$ y, como $\mathcal{U}'$ es numerable, también $\mathcal{U}$ lo es.

        Sea $W\subseteq Y$ una vecindad de $y$ en $(Y,\tau_Y)$, luego existe $V\in\tau$ tal que
        \begin{equation*}
            y\in Y\cap V\subseteq W
        \end{equation*}
        Como en particular $V$ es una vecindad de $y$ en $(X,\tau)$, entonces existe $U\in\mathcal{U}'$ tal que
        \begin{equation*}
            U\subseteq V
        \end{equation*}
        luego,
        \begin{equation*}
            Y\cap U\subseteq Y\cap V\subseteq W
        \end{equation*} 
        donde $Y\cap U\in\mathcal{U}$. Así, $\mathcal{U}$ es un sistema fundamental de vecindades de $y$ en $(Y,\tau_Y)$. Como $y\in Y$ fue arbitrario, se sigue que $(Y,\tau_Y)$ es primero numerable.
    \end{proof}

    \begin{propo}
        La propiedad de ser primero numerable es topológica.
    \end{propo}

    \begin{proof}
        Sean $(X_1,\tau_1)$ y $(X_2,\tau_2)$ espacios topológicos homeomorfos tales que $(X_1,\tau_1)$ es primero numerable. Sea $\cf{f}{(X_1,\tau_1)}{(X_2,\tau_2)}$ el homeomorfismo entre tales espacios. Veamos que $(X_2,\tau_2)$ es primero numeable.

        En efecto, sea $x_2\in X_2$, entonces existe un único $x_1\in X_1$ tal que $f(x_1)=x_2$. Como $(X_1,\tau_1)$ es primero numerable, entonces existe $\mathcal{U}_1$ sistema fundamental numerable de vecindades de $x_1$. Sea
        \begin{equation*}
            \mathcal{U}_2=\left\{f(U_1)\Big|U_1\in\mathcal{U}_1 \right\}
        \end{equation*}
        Como $\mathcal{U}_1$ es numerable, $\mathcal{U}_2$ también lo es. Y, como $U_1\in\mathcal{U}_1$ es una vecindad de $x_1$, entonces $f(U_1)$ es una vecindad de $x_2$ (por ser $f$ homeomorfismo). Por tanto, $\mathcal{U}_2$ es una colección de vecindades de $x_2$. Ahora, sea $V\in\mathcal{V}(x_2)$ una vecindad de $x_2$. Como $f$ es homeomorfismo entonces
        \begin{equation*}
            f^{-1}(V)\in\mathcal{V}(x_1)
        \end{equation*}
        Luego, existe $U\in\mathcal{U}_1$ tal que
        \begin{equation*}
            U\subseteq f^{-1}(V)\Rightarrow f(U)\subseteq V
        \end{equation*}
        por ser $f$ biyección, donde $f(U)\in\mathcal{U}_2$.

        Así, $\mathcal{U}_2$ es un sistema fundamental numerable de vecindades de $x_2$. Como el elemento $x_2$ fue arbitrario, se sigue que $(X_2,\tau_2)$ es primero numerable. Luego, la propiedad de ser primero numerable es topológica.
    \end{proof}

    \begin{propo}
        Sean $\left\{(X_k,\tau_k) \right\}_{ k\in\mathbb{N}}$ una familia numerable de espacios topológicos y
        \begin{equation*}
            X=\prod_{ k\in\mathbb{N}}X_k
        \end{equation*}
        Entonces, $(X,\tau_p)$ es primero numerable si y sólo si $(X_k,\tau_k)$ es primero numerable, para todo $k\in\mathbb{N}$.
    \end{propo}

    \begin{proof}
        $\Rightarrow)$: Es inmediato del hecho de que la propiedad de ser primero numerable es hereditaria y topológica.

        $\Leftarrow)$: Suponga que $(X_k,\tau_k)$ es primero numerable para todo $k\in\mathbb{N}$. Sea $x=(x_n)_{n\in\mathbb{N}}\in X$. Si $k\in\mathbb{N}$, se tiene que $(X_k,\tau_k)$ es primero numerable. Para $x_k\in X_k$ existe
        \begin{equation*}
            \mathcal{U}_k=\left\{U_m^k \right\}_{ m\in\mathbb{N}}
        \end{equation*}
        sistema fundamental numerable de vecindades de $x_k$ en $(X_k,\tau_k)$. Definimos
        \begin{equation*}
            \begin{split}
                \mathcal{U}=\left\{\prod_{ l\in\mathbb{N}}A_l\Big|\right.&\textup{ existe }I=\left\{i_1,...,i_t \right\} \subseteq\mathbb{N}\textup{ finito con }i_r<i_s\textup{ si }r<s\textup{ tal que } \\
                &\left.l\in\mathbb{N}-I\Rightarrow A_l=X_l\textup{ y }l\in I\Rightarrow A_k\in\mathcal{U}_l \right\}\\
            \end{split}
        \end{equation*}
        veamos que $\mathcal{U}\subseteq\mathcal{V}(x)$ y además $\mathcal{U}$  es un sistema fundamental de vecindades de $x$. Sea $U=\prod_{t\in\mathbb{N}}U_t$ un básico de la topología producto tal que $x\in U$. Tenemos que existe $I\subseteq\mathbb{N}$ finito tal que
        \begin{equation*}
            l\in\mathbb{N}-I\Rightarrow U_l=X_l\textup{ y }l\in I\Rightarrow x_l\in U_l\in\tau_l
        \end{equation*}
        Para $l\in I$ existe $U_{ m_l}^l\in\mathcal{U}_l$ tal que $x_l\in U_{ m_l}^l\subseteq U_l$. Sea
        \begin{equation*}
            A=\prod_{ l\in\mathbb{N}}A_l
        \end{equation*}
        donde,
        \begin{equation*}
            l\in\mathbb{N}-I\Rightarrow A_l=X_l\textup{ y }l\in I\Rightarrow A_l=U_{ m_l}^l
        \end{equation*}
        por tanto, $A\in\mathcal{U}$ y es tal que $x\in A\subseteq U$. 
        
        Veamos ahora que $\mathcal{U}$ es numerable. Sea $A=\prod_{ l\in\mathbb{N}}A_l\in\mathcal{U}$, entonces existe $I\subseteq\mathbb{N}$ finito, digamos $I=\left\{i_1,...,i_t \right\}$ (ordenados de forma estrictamente creciente y siendo todos distintos) tales que $l\in\mathbb{N}-I$ entonces $A_l=X_l$.Y, si $l\in I$ entonces $A_l=U_{ m_l}^l\in\mathcal{U}_l$. Sea $(i_1,...,i_t,m_{i_1},...,m_{i_t})\in\mathbb{N}^{ 2t}$. 
        
        Definimos la función
        \begin{equation*}
            \cf{f}{\mathcal{U}}{\bigcup_{t\in\mathbb{N}}\mathbb{N}^{2t}}
        \end{equation*}
        (donde $\mathbb{N}^{ 2t}$ expresa el producto cartesiano de $\mathbb{N}$ consigo mismo $2t$-veces) tal que $A\mapsto (i_1,...,i_t,m_{i_1},...,m_{i_t})$ (siendo el $A$ de la forma en que se expresó anteriormente). Se tiene por la elección de los elementos de $\mathcal{U}$, que la función $f$ está bien definida y es inyectiva. Por tanto, $\mathcal{U}$ es numerable.

        Luego, $(X,\tau_p)$ es primero numerable.
    \end{proof}

    \begin{propo}
        Sea $(X,\tau)$ un espacio primero numerable.
        \begin{enumerate}
            \item Sea $A\subseteq X$ y $x\in X$. Entonces $x\in\Cls{A}$ si y sólo si existe una sucesión de puntos $\left\{x_n \right\}_{ n=1}^\infty$ de $A$ que converge a $x$.
            \item Sean $(X',\tau')$ espacio topológico y $\cf{f}{(X,\tau)}{(X',\tau')}$ una función. Entonces, para $x\in X$, $f$ es continua en $X$ si y sólo si para cada sucesión $\left\{ x_n\right\}_{ n=1}^\infty$ de puntos en $X$ que converge a $x$, se tiene que la sucesión $\left\{f(x_n) \right\}_{ n=1}^\infty$ converge a $f(x)$.
        \end{enumerate}
    \end{propo}

    \begin{proof}
        De (1): Se probará la doble implicación.

        $\Rightarrow)$: Sea $x\in\Cls{A}$ y $\left\{ B_n\right\}_{n\in\mathbb{N}}$ un sistema fundamental numerable de vecindades de $x$. Entonces
        \begin{equation*}
            B_1\cap A\neq\emptyset
        \end{equation*}
        pues $x\in\Cls{A}$ y $B_1$ es vecindad de $x$. Tomemos $x_1\in B_1\cap A$. Para cada $n\in\mathbb{N}$, como
        \begin{equation*}
            B_1\cap\cdots\cap B_n
        \end{equation*} 
        es vecindad de $x$, entonces su intersección con $A$ es no vacía. Tome así $x_n\in B_1\cap\cdots\cap B_n\cap A$ y constrúyase así la sucesión $\left\{x_n \right\}_{ n\in\mathbb{N}}$. Veamos que esta sucesión converge a $x$. En efecto, sea $U\in\tau$ tal que $x\in\tau$. Como este es un sistema fundamental de vecindades, existe $l\in\mathbb{N}$ tal que $B_l\subseteq U$, luego
        \begin{equation*}
            x_{l+m}\in B_l\subseteq U
        \end{equation*}
        para todo $m\geq 0$. Por tanto, la sucesión converge a $x$.

        $\Leftarrow)$: Sea $\left\{ x_n\right\}_{ n=1}^\infty$ una sucesión de puntos de $A$ tal que $x_n\rightarrow\infty$. Tomemos $M\in\tau$ tal que $x\in M$, luego existe $k\in\mathbb{N}$ tal que $x_{ k+m}\in M$, para todo $m\geq 0$, así $M\cap A\neq\emptyset$. Por tanto, $x\in\Cls{A}$.

        De (2): Se probará la doble implicación.

        $\Rightarrow)$: Suponga que $f$ es continua en $x$. Sea $\left\{x_n \right\}$ una sucesión de puntos que converge a $x$. Sea $V\in\tau'$ tal que $f(x)\in V$, entonces $x\in f^{-1}(V)$, donde $f^{-1}(V)\in\tau$ por ser $f$ continua en $x$. Luego, existe $k\in\mathbb{N}$ tal que
        \begin{equation*}
            x_{ k+m}\in f^{-1}(V),\quad\forall m\geq0
        \end{equation*}
        es decir que
        \begin{equation*}
            f(x_{ k+m})\in f(f^{-1}(V))\subseteq V,\quad\forall m\geq0
        \end{equation*}
        Por tanto, $\left\{f(x_n) \right\}_{ n=1}^\infty$ converge a $f(x)$.

        $\Leftarrow)$: Veamos que dado $A\subseteq X$ se cumple que $f(\Cls{A})\subseteq\Cls{f(A)}$. En efecto, sea $x\in\Cls{A}$. Por 1) al ser $(X,\tau)$ primero numerable existe una sucesión $\left\{x_n \right\}_{ n=1}^\infty$ de puntos de $A$ que converge a $x$. Entonces $\left\{f(x_n) \right\}_{ n=1}^\infty$ es una sucesión de puntos de $f(A)$ que converge a $f(x)$. Por tanto, $f(x)\in\Cls{f(A)}$ (en la prueba de la suficiencia no es necesario que $(X,\tau)$ sea primero numerable, así que en este caso no se ocupa que $(X',\tau')$ sea primero numerable). Por tanto, $f(\Cls{A})\subseteq\Cls{f(A)}$
    \end{proof}

    \section{Espacios Segundo Numerables}

    \begin{propo}
        La propiedad de ser segundo numerable es hereditaria.
    \end{propo}

    \begin{proof}
        Sea $(X,\tau)$ un espacio topológico segundo numerable y $Y\subseteq X$ subconjunto. Veamos que $(Y,\tau_Y)$ es segundo numerable. En efecto, como $(X,\tau)$ es primero numerable, existe $\mathcal{B}=\left\{B_n\right\}_{ n\in\mathbb{N}}$ una base para la topología $\tau$ que es a lo sumo numerable. Se tiene que
        \begin{equation*}
            \mathcal{B}'=\left\{Y\cap B\Big|B\in\mathcal{B} \right\}
        \end{equation*}
        es una base para $\tau_Y$ (por una proposición anterior). Como $\mathcal{B}$ es numerable, se sigue que $\mathcal{B}'$ es numerable. Por tanto, $(Y,\tau_Y)$ es segundo numerable.
    \end{proof}

    \begin{propo}
        La propiedad de ser segundo numerable es topológica.
    \end{propo}

    \begin{proof}
        Sean $(X,\tau)$ y $(Y,\sigma)$ espacios topológicos homeomorfos con $\cf{f}{(X,\tau)}{(Y,\sigma)}$ el homeomorfismo y, suponga que $(X,\tau)$ es segundo numerable y sea $\mathcal{B}=\left\{B_n \right\}_{ n\in\mathbb{N}}$ una base de $\tau$. Entonces, la por una proposición, la colección:
        \begin{equation*}
            \mathcal{B}'=\left\{f(B)\Big|B\in\mathcal{B} \right\}
        \end{equation*}
        es una base para la topología $\sigma$ (por ser $f$ homeomorfismo) la cual es a lo sumo numerable. Por tanto, $(Y,\sigma)$ es segundo numerable.

        Así, la propiedad de ser segundo numerable es topológica.
    \end{proof}

    \begin{excer}
        Sea $\left\{(X_n,\tau_n) \right\}_{ n=1}^\infty$ una familia de espacios topológicos segundo numerables y, tomemos
        \begin{equation*}
            X=\prod_{ n=1}^\infty X_n
        \end{equation*}
        Entonces, $(X,\tau_p)$ es segundo numerable.
    \end{excer}

    \begin{proof}
        
    \end{proof}

    \begin{theor}
        Sea $(X,\tau)$ un espacio topológico.
        \begin{enumerate}
            \item Si $(X,\tau)$ es segundo numerable, entonces es primero numerable.
            \item Si $(X,\tau)$ se segundo numerable, entonces el espacio es de Lindelöf.
            \item Si $(X,\tau)$ es segundo numerable, entonces es separable.
        \end{enumerate}
    \end{theor}
     
    \begin{proof}
        De (1): Sea $\left\{ B_n\right\}_{ n\in\mathbb{N}}$ una base para la topología $\tau$. Tomemos $x\in X$ y defina
        \begin{equation*}
            \mathcal{B}_x=\left\{B\in\mathcal{B} \Big|x\in B \right\}
        \end{equation*}
        Se tiene que $\mathcal{B}_x$ es a lo sumo numerable. Sea $U\in\tau$ tal que $x\in U$, luego como $\mathcal{B}$ es base existe $B\in\mathcal{B}$ tal que $x\in B\subseteq U$, luego $B\in\mathcal{B}_x$. Por tanto, $\mathcal{B}_x$ es un sistema fundamental de vecindades de $x$ el cual es a lo sumo numerable. Al ser $x\in X$ arbitrario, se sigue que $(X,\tau)$ es primero numerable.

        De (2): Sea $\left\{ B_n\right\}_{ n\in\mathbb{N}}$ una base para la topología $\tau$ y sea $\mathcal{A}$ una cubierta abierta de $X$. Dado $x\in X$, como $A$ es una cubierta existe $A_x\in\mathcal{A}$ tal que
        \begin{equation*}
            x\in A_x\in\tau
        \end{equation*}
        luego, existe $B_x\in\mathcal{B}$ tal que $x\in\ B_x\subseteq A_x$. Sea
        \begin{equation*}
            \mathcal{K}=\left\{m\in\mathbb{N}\Big|\exists A\in\mathcal{A}\textup{ tal que }B_m\subseteq A \right\}
        \end{equation*}
        por la observación anterior, $\mathcal{K}\neq\emptyset$. Dado $k\in\mathcal{K}$ escogemos un único $A_k\in\mathcal{A}$ tal que $B_k\subseteq A_k$. Sea
        \begin{equation*}
            \mathcal{A}'=\left\{ A_n\right\}_{ n\in\mathbb{N}}
        \end{equation*}
        $\mathcal{A}'\subseteq\mathcal{A}$ es una subcolección a lo sumo numerable.

        Sea $x\in X$, tomemos $A\in\mathcal{A}$ tal que $x\in A$. Por ser $\mathcal{B}$ base existe $B_i\in\mathcal{B}$ tal que
        \begin{equation*}
            x\in B_i\subseteq A
        \end{equation*}
        Luego, $i\in\mathcal{K}$ por ende $x\in A_i$ siendo $A_i\in\mathcal{A}'$. Por tanto:
        \begin{equation*}
            X=\bigcup_{ i=1}^\infty A_i
        \end{equation*}
        luego, $(X,\tau)$ es Lindelöf.

        De (3): Sea $\mathcal{B}=\left\{B_n\right\}_{ n\in\mathbb{N}}$ base para $\tau$. Dado $n\in\mathbb{N}$ si $B_n\neq\emptyset$, escogemos $x_n\in B_n$ y con estos puntos formamos al conjunto numerable $A=\left\{x_n\Big|n\in\mathbb{N} \right\}$.

        Veamos que $\Cls{A}=X$. En efecto, sea $U\in\tau$ tal que $U\neq\emptyset$, veamos que $U\cap A\neq\emptyset$. En efecto, sea $x\in U$, luego existe $m\in\mathbb{N}$ tal que $x\in B_m\subseteq U$. Como $B_m\cap A\neq\emptyset$ entonces $U\cap A\neq\emptyset$. Se sigue que $\Cls{A}=X$.
    \end{proof}

    \begin{propo}
        Sean $(X,\tau)$ un espacio segundo numerable y $\mathcal{B}$ una base para su topología $\tau$. Entonces, $\mathcal{B}$ contiene una base numerable para $\tau$.
    \end{propo}

    \begin{proof}
        Sea $\mathcal{B}=\left\{B_\alpha\right\}_{\alpha\in I}$ una base para $\tau$ y, como $(X,\tau)$ es segundo numerable, existe $\mathcal{A}=\left\{A_n\right\}_{ n\in\mathbb{N}}$ base a lo sumo numerable de $\tau$.
        \renewcommand{\theenumi}{\alph{enumi}}
        \begin{enumerate}
            \item Sea $\mathcal{U}\in\tau$. Definimos:
            \begin{equation*}
                \mathcal{U}^*=\left\{A\in\mathcal{A}\Big|\exists U\in\mathcal{U}\textup{ tal que }A\subseteq U \right\}
            \end{equation*}
            dado $A\in\mathcal{U}^*$ escogemos un único $U_A\in\mathcal{U}$ tal que $A\subseteq U_A$. Defina
            \begin{equation*}
                \mathcal{U}'=\left\{U_A\in\mathcal{U} \Big|A\in\mathcal{U}^* \right\}
            \end{equation*}
            se tiene que $\mathcal{U}'$ es numerable por ser $\mathcal{A}$ numerable. Como $\mathcal{U}'\subseteq\mathcal{U}$, entonces
            \begin{equation*}
                \bigcup\mathcal{U}'\subseteq\bigcup\mathcal{U}
            \end{equation*}
            Veamos que se cumple la otra contención. Sea $x\in\bigcup\mathcal{U}$, luego existe $U\in\mathcal{U}$ tal que $x\in U$. Como $\mathcal{A}$ es una base y $U\in\tau$, existe $A\in \mathcal{A}$ tal que
            \begin{equation*}
                x\in A\subseteq U
            \end{equation*}
            así, $A\in\mathcal{U}^*$, luego $x\in A\subseteq U_A$ por lo cual $x\in\bigcup\mathcal{U}'$. Así,
            \begin{equation*}
                \bigcup\mathcal{U}'=\bigcup\mathcal{U}
            \end{equation*}
            \item Sea $A\in\mathcal{A}$, $A\in\tau$ luego existe $\mathcal{B}_A\subseteq\mathcal{B}$ tal que
            \begin{equation*}
                A=\bigcup\mathcal{B}_A
            \end{equation*}
            Por (a) existe $\mathcal{B}_A'\subseteq\mathcal{B}_A$ tal que $\mathcal{B}_A'$ es numerable y
            \begin{equation*}
                A=\bigcup\mathcal{B}_A'
            \end{equation*}
            Luego, $\bigcup\left\{\mathcal{B}_A'\Big|A\in\mathcal{A} \right\}$ es un conjunto a lo sumo numerable contenida en $\mathcal{B}$ tal que es una base para $\tau$.
        \end{enumerate}
        Por los dos incisos anteriores, se tiene el resultado.
    \end{proof}

    \begin{exa}
        Sea $X=\left\{0,1\right\}$ y tomemos $\tau_D=\left\{X,\emptyset,\left\{0\right\},\left\{1\right\}\right\}$. El espacio $(X,\tau_D)$ es segundo numerable, en particular primero numerable, Lindelöf y separable (además, metrizable pues $\tau_D$ es la topología discreta).
    \end{exa}

    \begin{exa}
        Considere $X=\left\{0,1\right\}$ y tomemos $\tau=\tau_D$. Para $r\in\mathbb{R}$ definimos $X_r=X$ y $\tau_r=\tau$. Veamos que $\left(X=\prod_{ r\in\mathbb{R}}X_r,\tau_p \right)$ no es primero numerable.
    \end{exa}

    \begin{proof}
        En efecto, sea $x=\left(x_r\right)_{ r\in\mathbb{R}}\in X$ tal que
        \begin{equation*}
            x_r=0,\quad\forall r\in\mathbb{R}
        \end{equation*}
        Sea $\mathcal{V}=\left\{V_n \right\}_{ n\in\mathbb{N}}$ una familia numerable de vecindades de $x$. Dado $m\in\mathbb{N}$ existe un básico $B_m\in\tau_p$ tal que
        \begin{equation*}
            x_m\in B_m\subseteq V_m
        \end{equation*}
        como $B_m$ es un básico de $\tau_p$, luego existe $J_m\subseteq\mathbb{R}$ finito tal que
        \begin{equation*}
            B_m=\prod_{ r\in\mathbb{R}}W_r
        \end{equation*}
        con $W_r\in\tau_r$, para cada $r\in J_m$ y $W_r=X_r$ para todo $r\in\mathbb{R}-J_m$. Por lo tanto, si
        \begin{equation*}
            V_m=\prod_{ r\in\mathbb{R}}K_r
        \end{equation*}
        entonces para todo $r\in\mathbb{R}-J_m$ se tiene que $K_r=X_r$. Tomemos
        \begin{equation*}
            J=\bigcup_{ m\in\mathbb{N}}J_m
        \end{equation*}
        este conjunto es a lo sumo numerable, siendo tal que $J\subseteq\mathbb{R}$, luego $\mathbb{R}-J$ es no vacío. Sea $t\in\mathbb{R}-J$, se tiene que para todo $m\in\mathbb{N}$, $t\notin J_m$. Sea
        \begin{equation*}
            U=\prod_{ r\in\mathbb{R}}U_r
        \end{equation*}
        donde
        \begin{equation*}
            U_r=\left\{ 
                \begin{array}{lcr}
                    \left\{0\right\} & \textup{ si } & r=t\\
                    X_r & \textup{ si } & r\neq t\\ 
                \end{array}
            \right.
        \end{equation*}
        $U\in\tau_p$ además, $x\in U$. Se cumple además que $V_m\nsubseteq U$ para todo $m\in\mathbb{N}$. Suponga que $\exists m_0\in\mathbb{N}$ tal que
        \begin{equation*}
            V_{ m_0}\subseteq U
        \end{equation*}
        Se tiene que
        \begin{equation*}
            \left\{0,1\right\}=X_t=K_t=p_t(V_{ m_0})\subseteq p_t(U)=\left\{0\right\}
        \end{equation*}
        lo cual es una contradicción. Por tanto, $\mathcal{V}$ no puede ser un sistema fundamnetal de vecindades para $x$, así que no es primero numerable.
    \end{proof}

    \begin{obs}
        En el ejemplo anterior, $\left(\prod_{ r\in\mathbb{R}}U_r,\tau_p \right)$ no es segundo numerable, pues no es primero numerable. Pero, $(X_r,\tau_r)$ es segundo numerable, para todo $r\in\mathbb{R}$.

        Tampoco es metrizable, siendo $(X_r,\tau_r)$ para todo $r\in\mathbb{R}$, pues metrizable implica primero numerable.
    \end{obs}

    \begin{propo}
        Sea $(X,\tau)$ un espacio metrizable. Entonces, $(X,\tau)$ es primero numerable.
    \end{propo}

    \begin{proof}
        Sea $\cf{d}{X\times X}{\mathbb{R}}$ una métrica tal que $X=X_d$. Sea $x\in X$. Para $m\in\mathbb{N}$ definimos
        \begin{equation*}
            B_n=B_d\left(x,\frac{1}{n}\right)
        \end{equation*}
        Entonces, $\left\{B_n\right\}_{ n\in\mathbb{N}}$ es un sistema fundamental de vecindades para $x$ el cual es a lo sumo numerable.
    \end{proof}

    \begin{exa}
        
    \end{exa}



\end{document}