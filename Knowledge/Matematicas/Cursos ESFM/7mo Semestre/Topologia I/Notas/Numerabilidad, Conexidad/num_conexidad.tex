\documentclass[12pt]{report}
\usepackage[spanish]{babel}
\usepackage[utf8]{inputenc}
\usepackage{amsmath}
\usepackage{amssymb}
\usepackage{amsthm}
\usepackage{graphics}
\usepackage{subfigure}
\usepackage{lipsum}
\usepackage{array}
\usepackage{multicol}
\usepackage{enumerate}
\usepackage[framemethod=TikZ]{mdframed}
\usepackage[a4paper, margin = 1.5cm]{geometry}

%En esta parte se hacen redefiniciones de algunos comandos para que resulte agradable el verlos%

\renewcommand{\theenumii}{\roman{enumii}}

\def\proof{\paragraph{Demostración:\\}}
\def\endproof{\hfill$\blacksquare$}

\def\sol{\paragraph{Solución:\\}}
\def\endsol{\hfill$\square$}

%En esta parte se definen los comandos a usar dentro del documento para enlistar%

\newtheoremstyle{largebreak}
  {}% use the default space above
  {}% use the default space below
  {\normalfont}% body font
  {}% indent (0pt)
  {\bfseries}% header font
  {}% punctuation
  {\newline}% break after header
  {}% header spec

\theoremstyle{largebreak}

\newmdtheoremenv[
    leftmargin=0em,
    rightmargin=0em,
    innertopmargin=-2pt,
    innerbottommargin=8pt,
    hidealllines = true,
    roundcorner = 5pt,
    backgroundcolor = gray!60!red!30
]{exa}{Ejemplo}[section]

\newmdtheoremenv[
    leftmargin=0em,
    rightmargin=0em,
    innertopmargin=-2pt,
    innerbottommargin=8pt,
    hidealllines = true,
    roundcorner = 5pt,
    backgroundcolor = gray!50!blue!30
]{obs}{Observación}[section]

\newmdtheoremenv[
    leftmargin=0em,
    rightmargin=0em,
    innertopmargin=-2pt,
    innerbottommargin=8pt,
    rightline = false,
    leftline = false
]{theor}{Teorema}[section]

\newmdtheoremenv[
    leftmargin=0em,
    rightmargin=0em,
    innertopmargin=-2pt,
    innerbottommargin=8pt,
    rightline = false,
    leftline = false
]{propo}{Proposición}[section]

\newmdtheoremenv[
    leftmargin=0em,
    rightmargin=0em,
    innertopmargin=-2pt,
    innerbottommargin=8pt,
    rightline = false,
    leftline = false
]{cor}{Corolario}[section]

\newmdtheoremenv[
    leftmargin=0em,
    rightmargin=0em,
    innertopmargin=-2pt,
    innerbottommargin=8pt,
    rightline = false,
    leftline = false
]{lema}{Lema}[section]

\newmdtheoremenv[
    leftmargin=0em,
    rightmargin=0em,
    innertopmargin=-2pt,
    innerbottommargin=8pt,
    roundcorner=5pt,
    backgroundcolor = gray!30,
    hidealllines = true
]{mydef}{Definición}[section]

\newmdtheoremenv[
    leftmargin=0em,
    rightmargin=0em,
    innertopmargin=-2pt,
    innerbottommargin=8pt,
    roundcorner=5pt
]{excer}{Ejercicio}[section]

%En esta parte se colocan comandos que definen la forma en la que se van a escribir ciertas funciones%

\newcommand\abs[1]{\ensuremath{\left|#1\right|}}
\newcommand\divides{\ensuremath{\bigm|}}
\newcommand\cf[3]{\ensuremath{#1:#2\rightarrow#3}}
\newcommand\contradiction{\ensuremath{\#_c}}
\newcommand{\V}[1]{\ensuremath{\mathcal{V}(#1)}}
\newcommand{\Int}[1]{\ensuremath{\mathring{#1}}}
\newcommand{\Cls}[1]{\ensuremath{\overline{#1}}}
\newcommand{\Fr}[1]{\ensuremath{\textup{Fr}(#1)}}
\newcommand{\natint}[1]{\ensuremath{\left[\!\left[#1\right]\!\right]}}
\newcommand{\floor}[1]{\ensuremath{\lfloor#1\rfloor}}
\newcommand{\Card}[1]{\ensuremath{\textup{Card}\left(#1\right)}}
\newcommand{\Pot}[1]{\ensuremath{\mathcal{P}\left(#1\right)}}
\newcommand{\id}[1]{\ensuremath{\textup{id}_{#1}}}

%recuerda usar \clearpage para hacer un salto de página

\begin{document}
    \setlength{\parskip}{5pt} % Añade 5 puntos de espacio entre párrafos
    \setlength{\parindent}{12pt} % Pone la sangría como me gusta
    \title{Notas Curso Topología I
    
    Axiomas de Numerabilidad}
    \author{Cristo Daniel Alvarado}
    \maketitle

    \tableofcontents %Con este comando se genera el índice general del libro%

    \setcounter{chapter}{4} %En esta parte lo que se hace es cambiar la enumeración del capítulo%
    
    \chapter{Axiomas de Numerabilidad}
    
    \section{Conceptos Fundamentales}

    \begin{obs}
        De ahora en adelante numerable será equivalente a lo sumo numerable.
    \end{obs}

    \begin{mydef}
        Sea $(X,\tau)$ un espacio topológico.
        \begin{enumerate}
            \item  Sean $x\in X$ y $\mathcal{U}$ una colección de vecindades de $x$. Diremos que $\mathcal{U}$ es un \textbf{sistema fundamental de vecindades de $x$} si dada $V\in\mathcal{V}(x)$ existe $U\in\mathcal{U}$ tal que $U\subseteq V$. Si $\mathcal{U}$ es numerable, $\mathcal{U}$ se dice un \textbf{sistema fundamental numerable de vecindades de $x$}.
            \item Si dado $x\in X$ existe un sistema fundamental numerable de vecindades de $x$, el espacio $(X,\tau)$ se dice \textbf{primero numerable}.
            \item El espacio $(X,\tau)$ se dice un \textbf{espacio segundo numerable} si su topología tiene una base numerable.
            \item El espacio $(X,\tau)$ se dice un \textbf{espacio separable} si existe $A\subseteq X$ tal que $A$ es numerable y además $\Cls{A}=X$ (es decir que es denso en $X$).
            \item El espacio $(X,\tau)$ se dice un \textbf{espacio de Lindelöf} si cada cubierta abierta del espacio tiene una subcubierta numerable.
        \end{enumerate}
    \end{mydef}

    \section{Espacios Primero Numerables}

    \begin{propo}
        Sea $(X,\tau)$ un espacio primero numerable. Si $Y\subseteq X$ entonces $(Y,\tau_Y)$ es primero numerable.
    \end{propo}

    \begin{proof}
        Sea $Y\subseteq X$. Sea $y\in Y$, en particular $y\in X$. Como $(X,\tau)$ es primero numeable, existe un sistema fundamental de vecindades de $y$ en $(X,\tau)$, digamos $\mathcal{U}'$, es decir que para este $\mathcal{U}'$ se cumple:
        \begin{equation*}
            \forall V\in\mathcal{V}(y)\exists U\in\mathcal{U}'\textup{ tal que }U\subseteq V
        \end{equation*}
        Sea
        \begin{equation*}
            \mathcal{U}=\left\{Y\cap U\Big|U\in\mathcal{U}' \right\}
        \end{equation*}
        Tenemos que $U\in\mathcal{U}'$, $Y\cap U$ es una vecindad de $y$ en $(Y,\tau_Y)$ y, como $\mathcal{U}'$ es numerable, también $\mathcal{U}$ lo es.

        Sea $W\subseteq Y$ una vecindad de $y$ en $(Y,\tau_Y)$, luego existe $V\in\tau$ tal que
        \begin{equation*}
            y\in Y\cap V\subseteq W
        \end{equation*}
        Como en particular $V$ es una vecindad de $y$ en $(X,\tau)$, entonces existe $U\in\mathcal{U}'$ tal que
        \begin{equation*}
            U\subseteq V
        \end{equation*}
        luego,
        \begin{equation*}
            Y\cap U\subseteq Y\cap V\subseteq W
        \end{equation*} 
        donde $Y\cap U\in\mathcal{U}$. Así, $\mathcal{U}$ es un sistema fundamental de vecindades de $y$ en $(Y,\tau_Y)$. Como $y\in Y$ fue arbitrario, se sigue que $(Y,\tau_Y)$ es primero numerable.
    \end{proof}

    \begin{propo}
        La propiedad de ser primero numerable es topológica.
    \end{propo}

    \begin{proof}
        Sean $(X_1,\tau_1)$ y $(X_2,\tau_2)$ espacios topológicos homeomorfos tales que $(X_1,\tau_1)$ es primero numerable. Sea $\cf{f}{(X_1,\tau_1)}{(X_2,\tau_2)}$ el homeomorfismo entre tales espacios. Veamos que $(X_2,\tau_2)$ es primero numeable.

        En efecto, sea $x_2\in X_2$, entonces existe un único $x_1\in X_1$ tal que $f(x_1)=x_2$. Como $(X_1,\tau_1)$ es primero numerable, entonces existe $\mathcal{U}_1$ sistema fundamental numerable de vecindades de $x_1$. Sea
        \begin{equation*}
            \mathcal{U}_2=\left\{f(U_1)\Big|U_1\in\mathcal{U}_1 \right\}
        \end{equation*}
        Como $\mathcal{U}_1$ es numerable, $\mathcal{U}_2$ también lo es. Y, como $U_1\in\mathcal{U}_1$ es una vecindad de $x_1$, entonces $f(U_1)$ es una vecindad de $x_2$ (por ser $f$ homeomorfismo). Por tanto, $\mathcal{U}_2$ es una colección de vecindades de $x_2$. Ahora, sea $V\in\mathcal{V}(x_2)$ una vecindad de $x_2$. Como $f$ es homeomorfismo entonces
        \begin{equation*}
            f^{-1}(V)\in\mathcal{V}(x_1)
        \end{equation*}
        Luego, existe $U\in\mathcal{U}_1$ tal que
        \begin{equation*}
            U\subseteq f^{-1}(V)\Rightarrow f(U)\subseteq V
        \end{equation*}
        por ser $f$ biyección, donde $f(U)\in\mathcal{U}_2$.

        Así, $\mathcal{U}_2$ es un sistema fundamental numerable de vecindades de $x_2$. Como el elemento $x_2$ fue arbitrario, se sigue que $(X_2,\tau_2)$ es primero numerable. Luego, la propiedad de ser primero numerable es topológica.
    \end{proof}

    \begin{propo}
        Sean $\left\{(X_k,\tau_k) \right\}_{ k\in\mathbb{N}}$ una familia numerable de espacios topológicos y
        \begin{equation*}
            X=\prod_{ k\in\mathbb{N}}X_k
        \end{equation*}
        Entonces, $(X,\tau_p)$ es primero numerable si y sólo si $(X_k,\tau_k)$ es primero numerable, para todo $k\in\mathbb{N}$.
    \end{propo}

    \begin{proof}
        $\Rightarrow)$: Es inmediato del hecho de que la propiedad de ser primero numerable es hereditaria y topológica.

        $\Leftarrow)$: Suponga que $(X_k,\tau_k)$ es primero numerable para todo $k\in\mathbb{N}$. Sea $x=(x_n)_{n\in\mathbb{N}}\in X$. Para $x_k\in X_k$ existe
        \begin{equation*}
            \mathcal{U}_k=\left\{U_m^k \right\}_{ m\in\mathbb{N}}
        \end{equation*}
        sistema fundamental numerable de vecindades de $x_k$ en $(X_k,\tau_k)$. Definimos
        \begin{equation*}
            \begin{split}
                \mathcal{U}=\left\{\prod_{ l\in\mathbb{N}}A_l\Big|\right.&\textup{ existe }I=\left\{i_1,...,i_t \right\} \subseteq\mathbb{N}\textup{ finito con }i_r<i_s\textup{ si }r<s\textup{ tal que } \\
                &\left.l\in\mathbb{N}-I\Rightarrow A_l=X_l\textup{ y }l\in I\Rightarrow A_k\in\mathcal{U}_l \right\}\\
            \end{split}
        \end{equation*}
        veamos que $\mathcal{U}\subseteq\mathcal{V}(x)$ y además $\mathcal{U}$  es un sistema fundamental de vecindades de $x$. Sea $U=\prod_{t\in\mathbb{N}}U_t$ un básico de la topología producto tal que $x\in U$. Tenemos que existe $I\subseteq\mathbb{N}$ finito tal que
        \begin{equation*}
            l\in\mathbb{N}-I\Rightarrow U_l=X_l\textup{ y }l\in I\Rightarrow x_l\in U_l\in\tau_l
        \end{equation*}
        Para $l\in I$ existe $U_{ m_l}^l\in\mathcal{U}_l$ tal que $x_l\in U_{ m_l}^l\subseteq U_l$. Sea
        \begin{equation*}
            A=\prod_{ l\in\mathbb{N}}A_l
        \end{equation*}
        donde,
        \begin{equation*}
            l\in\mathbb{N}-I\Rightarrow A_l=X_l\textup{ y }l\in I\Rightarrow A_l=U_{ m_l}^l
        \end{equation*}
        por tanto, $A\in\mathcal{U}$ y es tal que $x\in A\subseteq U$. 
        
        Veamos ahora que $\mathcal{U}$ es numerable. Sea $A=\prod_{ l\in\mathbb{N}}A_l\in\mathcal{U}$, entonces existe $I\subseteq\mathbb{N}$ finito, digamos $I=\left\{i_1,...,i_t \right\}$ (ordenados de forma estrictamente creciente y siendo todos distintos) tales que $l\in\mathbb{N}-I$ entonces $A_l=X_l$.Y, si $l\in I$ entonces $A_l=U_{ m_l}^l\in\mathcal{U}_l$. Sea $(i_1,...,i_t,m_{i_1},...,m_{i_t})\in\mathbb{N}^{ 2t}$. 
        
        Definimos la función
        \begin{equation*}
            \cf{f}{\mathcal{U}}{\bigcup_{t\in\mathbb{N}}\mathbb{N}^{2t}}
        \end{equation*}
        (donde $\mathbb{N}^{ 2t}$ expresa el producto cartesiano de $\mathbb{N}$ consigo mismo $2t$-veces) tal que $A\mapsto (i_1,...,i_t,m_{i_1},...,m_{i_t})$ (siendo el $A$ de la forma en que se expresó anteriormente). Se tiene por la elección de los elementos de $\mathcal{U}$, que la función $f$ está bien definida y es inyectiva. Por tanto, $\mathcal{U}$ es numerable.

        Luego, $(X,\tau_p)$ es primero numerable.
    \end{proof}

    \begin{propo}
        Sea $(X,\tau)$ un espacio primero numerable.
        \begin{enumerate}
            \item Sea $A\subseteq X$ y $x\in X$. Entonces $x\in\Cls{A}$ si y sólo si existe una sucesión de puntos $\left\{x_n \right\}_{ n=1}^\infty$ de $A$ que converge a $x$.
            \item Sean $(X',\tau')$ espacio topológico y $\cf{f}{(X,\tau)}{(X',\tau')}$ una función. Entonces, para $x\in X$, $f$ es continua en $X$ si y sólo si para cada sucesión $\left\{ x_n\right\}_{ n=1}^\infty$ de puntos en $X$ que converge a $x$, se tiene que la sucesión $\left\{f(x_n) \right\}_{ n=1}^\infty$ converge a $f(x)$.
        \end{enumerate}
    \end{propo}

    \begin{proof}
        De (1): Se probará la doble implicación.

        $\Rightarrow)$: Sea $x\in\Cls{A}$ y $\left\{ B_n\right\}_{n\in\mathbb{N}}$ un sistema fundamental numerable de vecindades de $x$. Entonces
        \begin{equation*}
            B_1\cap A\neq\emptyset
        \end{equation*}
        pues $x\in\Cls{A}$ y $B_1$ es vecindad de $x$. Tomemos $x_1\in B_1\cap A$. Para cada $n\in\mathbb{N}$, como
        \begin{equation*}
            B_1\cap\cdots\cap B_n
        \end{equation*} 
        es vecindad de $x$, entonces su intersección con $A$ es no vacía. Tome así $x_n\in B_1\cap\cdots\cap B_n\cap A$ y constrúyase así la sucesión $\left\{x_n \right\}_{ n\in\mathbb{N}}$. Veamos que esta sucesión converge a $x$. En efecto, sea $U\in\tau$ tal que $x\in\tau$. Como este es un sistema fundamental de vecindades, existe $l\in\mathbb{N}$ tal que $B_l\subseteq U$, luego
        \begin{equation*}
            x_{l+m}\in B_l\subseteq U
        \end{equation*}
        para todo $m\geq 0$. Por tanto, la sucesión converge a $x$.

        $\Leftarrow)$: Sea $\left\{ x_n\right\}_{ n=1}^\infty$ una sucesión de puntos de $A$ tal que $x_n\rightarrow\infty$. Tomemos $M\in\tau$ tal que $x\in M$, luego existe $k\in\mathbb{N}$ tal que $x_{ k+m}\in M$, para todo $m\geq 0$, así $M\cap A\neq\emptyset$. Por tanto, $x\in\Cls{A}$.

        De (2): Se probará la doble implicación.

        $\Rightarrow)$: Suponga que $f$ es continua en $x$. Sea $\left\{x_n \right\}$ una sucesión de puntos que converge a $x$. Sea $V\in\tau'$ tal que $f(x)\in V$, entonces $x\in f^{-1}(V)$, donde $f^{-1}(V)\in\tau$ por ser $f$ continua en $x$. Luego, existe $k\in\mathbb{N}$ tal que
        \begin{equation*}
            x_{ k+m}\in f^{-1}(V),\quad\forall m\geq0
        \end{equation*}
        es decir que
        \begin{equation*}
            f(x_{ k+m})\in f(f^{-1}(V))\subseteq V,\quad\forall m\geq0
        \end{equation*}
        Por tanto, $\left\{f(x_n) \right\}_{ n=1}^\infty$ converge a $f(x)$.

        $\Leftarrow)$: Veamos que dado $A\subseteq X$ se cumple que $f(\Cls{A})\subseteq\Cls{f(A)}$. En efecto, sea $x\in\Cls{A}$. Por 1) al ser $(X,\tau)$ primero numerable existe una sucesión $\left\{x_n \right\}_{ n=1}^\infty$ de puntos de $A$ que converge a $x$. Entonces $\left\{f(x_n) \right\}_{ n=1}^\infty$ es una sucesión de puntos de $f(A)$ que converge a $f(x)$. Por tanto, $f(x)\in\Cls{f(A)}$ (en la prueba de la suficiencia no es necesario que $(X,\tau)$ sea primero numerable, así que en este caso no se ocupa que $(X',\tau')$ sea primero numerable). Por tanto, $f(\Cls{A})\subseteq\Cls{f(A)}$
    \end{proof}

    \section{Espacios Segundo Numerables}

    \begin{propo}
        La propiedad de ser segundo numerable es hereditaria.
    \end{propo}

    \begin{proof}
        Sea $(X,\tau)$ un espacio topológico segundo numerable y $Y\subseteq X$ subconjunto. Veamos que $(Y,\tau_Y)$ es segundo numerable. En efecto, como $(X,\tau)$ es primero numerable, existe $\mathcal{B}=\left\{B_n\right\}_{ n\in\mathbb{N}}$ una base para la topología $\tau$ que es a lo sumo numerable. Se tiene que
        \begin{equation*}
            \mathcal{B}'=\left\{Y\cap B\Big|B\in\mathcal{B} \right\}
        \end{equation*}
        es una base para $\tau_Y$ (por una proposición anterior). Como $\mathcal{B}$ es numerable, se sigue que $\mathcal{B}'$ es numerable. Por tanto, $(Y,\tau_Y)$ es segundo numerable.
    \end{proof}

    \begin{propo}
        La propiedad de ser segundo numerable es topológica.
    \end{propo}

    \begin{proof}
        Sean $(X,\tau)$ y $(Y,\sigma)$ espacios topológicos homeomorfos con $\cf{f}{(X,\tau)}{(Y,\sigma)}$ el homeomorfismo y, suponga que $(X,\tau)$ es segundo numerable y sea $\mathcal{B}=\left\{B_n \right\}_{ n\in\mathbb{N}}$ una base de $\tau$. Entonces, la por una proposición, la colección:
        \begin{equation*}
            \mathcal{B}'=\left\{f(B)\Big|B\in\mathcal{B} \right\}
        \end{equation*}
        es una base para la topología $\sigma$ (por ser $f$ homeomorfismo) la cual es a lo sumo numerable. Por tanto, $(Y,\sigma)$ es segundo numerable.

        Así, la propiedad de ser segundo numerable es topológica.
    \end{proof}

    \begin{excer}
        Sea $\left\{(X_n,\tau_n) \right\}_{ n=1}^\infty$ una familia de espacios topológicos segundo numerables y, tomemos
        \begin{equation*}
            X=\prod_{ n=1}^\infty X_n
        \end{equation*}
        Entonces, $(X,\tau_p)$ es segundo numerable.
    \end{excer}

    \begin{proof}
        
    \end{proof}

    \begin{theor}
        Sea $(X,\tau)$ un espacio topológico.
        \begin{enumerate}
            \item Si $(X,\tau)$ es segundo numerable, entonces es primero numerable.
            \item Si $(X,\tau)$ se segundo numerable, entonces el espacio es de Lindelöf.
            \item Si $(X,\tau)$ es segundo numerable, entonces es separable.
        \end{enumerate}
    \end{theor}
     
    \begin{proof}
        De (1): Sea $\left\{ B_n\right\}_{ n\in\mathbb{N}}$ una base para la topología $\tau$. Tomemos $x\in X$ y defina
        \begin{equation*}
            \mathcal{B}_x=\left\{B\in\mathcal{B} \Big|x\in B \right\}
        \end{equation*}
        Se tiene que $\mathcal{B}_x$ es a lo sumo numerable. Sea $U\in\tau$ tal que $x\in U$, luego como $\mathcal{B}$ es base existe $B\in\mathcal{B}$ tal que $x\in B\subseteq U$, luego $B\in\mathcal{B}_x$. Por tanto, $\mathcal{B}_x$ es un sistema fundamental de vecindades de $x$ el cual es a lo sumo numerable. Al ser $x\in X$ arbitrario, se sigue que $(X,\tau)$ es primero numerable.

        De (2): Sea $\left\{ B_n\right\}_{ n\in\mathbb{N}}$ una base para la topología $\tau$ y sea $\mathcal{A}$ una cubierta abierta de $X$. Dado $x\in X$, como $A$ es una cubierta existe $A_x\in\mathcal{A}$ tal que
        \begin{equation*}
            x\in A_x\in\tau
        \end{equation*}
        luego, existe $B_x\in\mathcal{B}$ tal que $x\in\ B_x\subseteq A_x$. Sea
        \begin{equation*}
            \mathcal{K}=\left\{m\in\mathbb{N}\Big|\exists A\in\mathcal{A}\textup{ tal que }B_m\subseteq A \right\}
        \end{equation*}
        por la observación anterior, $\mathcal{K}\neq\emptyset$. Dado $k\in\mathcal{K}$ escogemos un único $A_k\in\mathcal{A}$ tal que $B_k\subseteq A_k$. Sea
        \begin{equation*}
            \mathcal{A}'=\left\{ A_n\right\}_{ n\in\mathbb{N}}
        \end{equation*}
        $\mathcal{A}'\subseteq\mathcal{A}$ es una subcolección a lo sumo numerable.

        Sea $x\in X$, tomemos $A\in\mathcal{A}$ tal que $x\in A$. Por ser $\mathcal{B}$ base existe $B_i\in\mathcal{B}$ tal que
        \begin{equation*}
            x\in B_i\subseteq A
        \end{equation*}
        Luego, $i\in\mathcal{K}$ por ende $x\in A_i$ siendo $A_i\in\mathcal{A}'$. Por tanto:
        \begin{equation*}
            X=\bigcup_{ i=1}^\infty A_i
        \end{equation*}
        luego, $(X,\tau)$ es Lindelöf.

        De (3): Sea $\mathcal{B}=\left\{B_n\right\}_{ n\in\mathbb{N}}$ base para $\tau$. Dado $n\in\mathbb{N}$ si $B_n\neq\emptyset$, escogemos $x_n\in B_n$ y con estos puntos formamos al conjunto numerable $A=\left\{x_n\Big|n\in\mathbb{N} \right\}$.

        Veamos que $\Cls{A}=X$. En efecto, sea $U\in\tau$ tal que $U\neq\emptyset$, veamos que $U\cap A\neq\emptyset$. En efecto, sea $x\in U$, luego existe $m\in\mathbb{N}$ tal que $x\in B_m\subseteq U$. Como $B_m\cap A\neq\emptyset$ entonces $U\cap A\neq\emptyset$. Se sigue que $\Cls{A}=X$.
    \end{proof}

    \begin{propo}
        Sean $(X,\tau)$ un espacio segundo numerable y $\mathcal{B}$ una base para su topología $\tau$. Entonces, $\mathcal{B}$ contiene una base numerable para $\tau$.
    \end{propo}

    \begin{proof}
        Sea $\mathcal{B}=\left\{B_\alpha\right\}_{\alpha\in I}$ una base para $\tau$ y, como $(X,\tau)$ es segundo numerable, existe $\mathcal{A}=\left\{A_n\right\}_{ n\in\mathbb{N}}$ base a lo sumo numerable de $\tau$.
        \renewcommand{\theenumi}{\alph{enumi}}
        \begin{enumerate}
            \item Sea $\mathcal{U}\in\tau$. Definimos:
            \begin{equation*}
                \mathcal{U}^*=\left\{A\in\mathcal{A}\Big|\exists U\in\mathcal{U}\textup{ tal que }A\subseteq U \right\}
            \end{equation*}
            dado $A\in\mathcal{U}^*$ escogemos un único $U_A\in\mathcal{U}$ tal que $A\subseteq U_A$. Defina
            \begin{equation*}
                \mathcal{U}'=\left\{U_A\in\mathcal{U} \Big|A\in\mathcal{U}^* \right\}
            \end{equation*}
            se tiene que $\mathcal{U}'$ es numerable por ser $\mathcal{A}$ numerable. Como $\mathcal{U}'\subseteq\mathcal{U}$, entonces
            \begin{equation*}
                \bigcup\mathcal{U}'\subseteq\bigcup\mathcal{U}
            \end{equation*}
            Veamos que se cumple la otra contención. Sea $x\in\bigcup\mathcal{U}$, luego existe $U\in\mathcal{U}$ tal que $x\in U$. Como $\mathcal{A}$ es una base y $U\in\tau$, existe $A\in \mathcal{A}$ tal que
            \begin{equation*}
                x\in A\subseteq U
            \end{equation*}
            así, $A\in\mathcal{U}^*$, luego $x\in A\subseteq U_A$ por lo cual $x\in\bigcup\mathcal{U}'$. Así,
            \begin{equation*}
                \bigcup\mathcal{U}'=\bigcup\mathcal{U}
            \end{equation*}
            \item Sea $A\in\mathcal{A}$, $A\in\tau$ luego existe $\mathcal{B}_A\subseteq\mathcal{B}$ tal que
            \begin{equation*}
                A=\bigcup\mathcal{B}_A
            \end{equation*}
            Por (a) existe $\mathcal{B}_A'\subseteq\mathcal{B}_A$ tal que $\mathcal{B}_A'$ es numerable y
            \begin{equation*}
                A=\bigcup\mathcal{B}_A'
            \end{equation*}
            Luego, $\bigcup\left\{\mathcal{B}_A'\Big|A\in\mathcal{A} \right\}$ es un conjunto a lo sumo numerable contenida en $\mathcal{B}$ tal que es una base para $\tau$.
        \end{enumerate}
        Por los dos incisos anteriores, se tiene el resultado.
    \end{proof}

    \begin{exa}
        Sea $X=\left\{0,1\right\}$ y tomemos $\tau_D=\left\{X,\emptyset,\left\{0\right\},\left\{1\right\}\right\}$. El espacio $(X,\tau_D)$ es segundo numerable, en particular primero numerable, Lindelöf y separable (además, metrizable pues $\tau_D$ es la topología discreta).
    \end{exa}

    \begin{exa}
        Considere $X=\left\{0,1\right\}$ y tomemos $\tau=\tau_D$. Para $r\in\mathbb{R}$ definimos $X_r=X$ y $\tau_r=\tau$. Veamos que $\left(X=\prod_{ r\in\mathbb{R}}X_r,\tau_p \right)$ no es primero numerable.
    \end{exa}

    \begin{proof}
        En efecto, sea $x=\left(x_r\right)_{ r\in\mathbb{R}}\in X$ tal que
        \begin{equation*}
            x_r=0,\quad\forall r\in\mathbb{R}
        \end{equation*}
        Sea $\mathcal{V}=\left\{V_n \right\}_{ n\in\mathbb{N}}$ una familia numerable de vecindades de $x$. Dado $m\in\mathbb{N}$ existe un básico $B_m\in\tau_p$ tal que
        \begin{equation*}
            x_m\in B_m\subseteq V_m
        \end{equation*}
        como $B_m$ es un básico de $\tau_p$, luego existe $J_m\subseteq\mathbb{R}$ finito tal que
        \begin{equation*}
            B_m=\prod_{ r\in\mathbb{R}}W_r
        \end{equation*}
        con $W_r\in\tau_r$, para cada $r\in J_m$ y $W_r=X_r$ para todo $r\in\mathbb{R}-J_m$. Por lo tanto, si
        \begin{equation*}
            V_m=\prod_{ r\in\mathbb{R}}K_r
        \end{equation*}
        entonces para todo $r\in\mathbb{R}-J_m$ se tiene que $K_r=X_r$. Tomemos
        \begin{equation*}
            J=\bigcup_{ m\in\mathbb{N}}J_m
        \end{equation*}
        este conjunto es a lo sumo numerable, siendo tal que $J\subseteq\mathbb{R}$, luego $\mathbb{R}-J$ es no vacío. Sea $t\in\mathbb{R}-J$, se tiene que para todo $m\in\mathbb{N}$, $t\notin J_m$. Sea
        \begin{equation*}
            U=\prod_{ r\in\mathbb{R}}U_r
        \end{equation*}
        donde
        \begin{equation*}
            U_r=\left\{ 
                \begin{array}{lcr}
                    \left\{0\right\} & \textup{ si } & r=t\\
                    X_r & \textup{ si } & r\neq t\\ 
                \end{array}
            \right.
        \end{equation*}
        $U\in\tau_p$ además, $x\in U$. Se cumple además que $V_m\nsubseteq U$ para todo $m\in\mathbb{N}$. Suponga que $\exists m_0\in\mathbb{N}$ tal que
        \begin{equation*}
            V_{ m_0}\subseteq U
        \end{equation*}
        Se tiene que
        \begin{equation*}
            \left\{0,1\right\}=X_t=K_t=p_t(V_{ m_0})\subseteq p_t(U)=\left\{0\right\}
        \end{equation*}
        lo cual es una contradicción. Por tanto, $\mathcal{V}$ no puede ser un sistema fundamnetal de vecindades para $x$, así que no es primero numerable.
    \end{proof}

    \begin{obs}
        En el ejemplo anterior, $\left(\prod_{ r\in\mathbb{R}}U_r,\tau_p \right)$ no es segundo numerable, pues no es primero numerable. Pero, $(X_r,\tau_r)$ es segundo numerable, para todo $r\in\mathbb{R}$.

        Tampoco es metrizable, siendo $(X_r,\tau_r)$ para todo $r\in\mathbb{R}$, pues metrizable implica primero numerable.
    \end{obs}

    \begin{propo}
        Sea $(X,\tau)$ un espacio metrizable. Entonces, $(X,\tau)$ es primero numerable.
    \end{propo}

    \begin{proof}
        Sea $\cf{d}{X\times X}{\mathbb{R}}$ una métrica tal que $X=X_d$. Sea $x\in X$. Para $m\in\mathbb{N}$ definimos
        \begin{equation*}
            B_n=B_d\left(x,\frac{1}{m}\right)
        \end{equation*}
        Entonces, $\left\{B_n\right\}_{ n\in\mathbb{N}}$ es un sistema fundamental de vecindades para $x$ el cual es a lo sumo numerable. En efecto, sea $U\in\tau$ tal que $x\in U$. Entonces, como el sistema de bolas abiertas forma una base para la topología $\tau$ se tiene que existe $r>0$ tal que
        \begin{equation*}
            B(x,r)\subseteq U
        \end{equation*}
        Por la propiedad arquimediana existe $n\in\mathbb{N}$ tal que $\frac{1}{n}<r$. Así,
        \begin{equation*}
            B\left(x,\frac{1}{n}\right)\subseteq U
        \end{equation*}
        siendo $B(x,\frac{1}{n})$ un elemento del sistema fundamental de vecindades.
    \end{proof}

    \begin{exa}
        Sea $\mathcal{B}_l=\left\{[a,b)\Big|a,b\in\mathbb{R} \right\}$. Ya se sabe que $\mathcal{B}_l$ es una base para una topología sobre $\mathbb{R}$, la cual se denota por $\tau_l$. A la pareja $(\mathbb{R},\tau_l)$ se suele escribir simplemente como $\mathbb{R}_l$.
        \begin{enumerate}
            \item $\mathbb{R}_l$ es de Hausdorff (esto se deduce de forma casi inmediata).
            \item $\mathbb{R}_l$ es primero numerable. En efecto, sea $r\in\mathbb{R}$, entonces el conjunto:
            \begin{equation*}
                \mathcal{V}=\left\{[r,r+\frac{1}{n})\Big|n\in\mathbb{N} \right\}
            \end{equation*}
            es un sistema fundamental de vecindades de $r$. En efecto, sea $U\in\tau_l$ tal que $r\in U$. Considere $[l,k)\in\mathcal{B}_l$ que cumpla
            \begin{equation*}
                r\in [l,k)\subseteq U
            \end{equation*}
            Entonces, $l\leq r<k$. Por la propiedad arquimediana existe $n\in\mathbb{N}$ tal que
            \begin{equation*}
                r+\frac{1}{n}<k
            \end{equation*}
            luego,
            \begin{equation*}
                [r,r+\frac{1}{n})\subseteq [l,k)\subseteq U
            \end{equation*}
            Así, $\mathcal{V}$ es un sistema fundamental de vecindades de $r$.
            \item $\mathbb{R}_l$ no es segundo numerable. Sea $\mathcal{B}$ una base para $\tau_l$. Dado $x\in\mathbb{R}$, escogemos $B_x\in\mathcal{B}$ tal que
            \begin{equation*}
                x\in B_x\subseteq [x,x+1)
            \end{equation*}
            Tenemos $x=\inf B_x$ luego dados $x,y\in\mathbb{R}$ con $x<y$ existen $B_x,B_y\in\mathcal{B}$ tales que $B_x\neq B_y$. Por tanto, $\mathcal{B}$ no puede ser numerable, así que $\mathbb{R}_l$ no puede ser segundo numerable.
            \item $\mathbb{R}_l$ es separable. Considere $\mathbb{Q}\subseteq\mathbb{R}$. Este conjunto es numerable y denso en $\mathbb{R}_l$.
            \item $\mathbb{R}_l$ es de Lindelöf. Sea $\mathcal{A}$ una cubierta de $\mathbb{R}_l$ formada por básicos. Suponga que
            \begin{equation*}
                \mathcal{A}=\left\{[a_\alpha,b_\alpha)\Big|\alpha\in I \right\}
            \end{equation*}
            Sea
            \begin{equation*}
                C=\bigcup_{\alpha\in I}(a_\alpha,b_\alpha)
            \end{equation*}
            Considere $C$ como subespacio de $(\mathbb{R},\tau_u)$. El espacio $(\mathbb{R},\tau_u)$ es segundo numerable, luego $(X,{\tau_u}_C)$ es segundo numerable. Por lo tanto, $(X,{\tau_u}_C)$ es de Lindelöf. Tenemos que existe $J=\left\{\alpha_1,...,\alpha_m,... \right\} \subseteq I$ numerable tal que
            \begin{equation*}
                C=\bigcup_{ i=1}^\infty (a_{\alpha_i},b_{\alpha_i})
            \end{equation*}
            Sea
            \begin{equation*}
                \mathcal{A}'=\left\{[a_\alpha,b_\alpha)\Big|\alpha\in J \right\}
            \end{equation*}
            Se tiene que
            \begin{equation*}
                C\subseteq\bigcup_{\alpha\in J}[a_\alpha,b_\alpha)
            \end{equation*}
            Tomemos
            \begin{equation*}
                D=\mathbb{R}-C
            \end{equation*}
            Veamos que $D$ es numerable. En efecto, sea $x\in D$, luego $x\in\mathbb{R}-C$. Así, para todo $\alpha\in I$,
            \begin{equation*}
                x\notin (a_\alpha,b_\alpha)
            \end{equation*}
            Luego, existe $\alpha_0\in I$ tal que $x= a_{\alpha_0}$. Sea $q_x\in(a_{\alpha_0},b_{\alpha_0})\cap\mathbb{Q}$, entonces
            \begin{equation*}
                (x,q_x)\subseteq C
            \end{equation*}
            Sea $\cf{f}{D}{\mathbb{Q}}$ la función definida por: dado $x\in D$, $x\mapsto q_x$. Veamos que $f$ es inyectiva. Sean $x,y\in D$ con $x<y$.
            \begin{enumerate}
                \item Suponga que $q_y\leq q_x$. Se tiene que $x<y<q_x\leq q_y$ (por la elección de los $q$). Por tanto, $y\in (x,q_x)\subseteq C\Rightarrow y\notin D$\contradiction.
                \item Por (i), $q_x<q_y$. Así, $f$ es inyectiva. Luego, $D$ es a lo sumo numerable.
            \end{enumerate}
            Dado $d\in D$ escogemos un único elemento $A_d\in\mathcal{A}$ tal que $d\in A$. Sea
            \begin{equation*}
                \mathcal{A}''=\left\{A_d\Big|d\in D \right\}
            \end{equation*}
            Se tiene que $\mathcal{A}'$ y $\mathcal{A}''$ son a lo sumo numerables, luego su unión también lo es y es tal que
            \begin{equation*}
                \mathbb{R}\subseteq \bigcup\mathcal{A}'\cup\mathcal{A}''
            \end{equation*}
            por tanto, $\mathbb{R}_l$ es Lindelöf.
            \item $\mathbb{R}_l^2=\mathbb{R}_l\times\mathbb{R}_l$ no es de Lindelöf. Sea
            \begin{equation*}
                \mathcal{L}=\left\{(x,-x)\Big|x\in\mathbb{R} \right\}
            \end{equation*}
            Afirmamos que $\mathcal{L}\subseteq \mathbb{R}_l^2$ es cerrado. Sea
            \begin{equation*}
                \mathcal{A}=\left\{\mathbb{R}-\mathcal{L} \right\}\bigcup\left\{[a,b)\times[-a,d)\Big|a,b,d\in\mathbb{R} \right\}
            \end{equation*}
            Se tiene que $\mathcal{A}$ es una cubierta abierta de $\mathbb{R}_l^2$.
            Además, para $U=[a,b)\times[-a,d)$ tenemos que $U\cap\mathcal{L}=\left\{(a,-a) \right\}$. Luego, para todo $a\in\mathbb{R}$
            \begin{equation*}
                \left\{(a,-a) \right\}\in{\tau_l^2}_{\mathcal{L}}
            \end{equation*}
            pero entonces $\mathcal{A}$ no puede tener una subcolección numerable que cubra a $\mathbb{R}_l^2$.
            \item $\mathbb{R}_l^2$ es separable pues $\mathbb{Q}^2\subseteq\mathbb{R}_l^2$ es numerbale y denso.
            \item $\mathcal{L}$ como subespacio de $\mathbb{R}_l^2$ no es separable. Sea $A\subseteq\mathcal{L}$ numerable. Se tiene que ${\tau_l^2}_{\mathcal{L}}$ coincide con la topología discreta. Luego $\mathcal{L}-A$ es abierto, así $\mathcal{A}$ es cerrado (todo esto en la topología del subespacio), así $A$ no es denso en $(L,)$. Por tanto, el espacio no puede ser separable.
        \end{enumerate}
    \end{exa}

    \chapter{Espacios Conexos}

    \section{Conceptos Fundamentales}

    \renewcommand{\theenumi}{\alph{enumi}}

    \begin{mydef}
        Sea $(X,\tau)$ un espacio topológico.
        \begin{enumerate}
            \item Una \textbf{partición de $X$} es una pareja formada por dos conjuntos abiertos $U,V$ no vacíos tales que $U\cap V=\emptyset$ y $X=U\cap V$.
            \item Dos subconjuntos $A,B$ de $X$ se dicen \textbf{mutuamente separados} en $(X,\tau)$ si $A\cap\Cls{B}=\Cls{A}\cap B=\emptyset$.
            \item $(X,\tau)$ se llama un \textbf{espacio conexo} si no existe una partición de $X$ y, en caso contrario lo llamaremos \textbf{espacio disconexo}. Si $A\subseteq X$ se dice que $A$ es un \textbf{conjunto conexo} si $(A,\tau_A)$ es conexo.
        \end{enumerate}
    \end{mydef}

    \begin{exa}
        Dado $(X,\tau_I=\left\{X,\emptyset \right\})$ es un conjunto conexo.
    \end{exa}

    \begin{exa}
        Sea $X$ un conjunto con al menos dos puntos distintos. Entonces, $(X,\tau_D)$ no es conexo.
    \end{exa}

    \begin{exa}
        Sean $\tau_1$ y $\tau_2$ dos topologías definidas sobre el conjunto $X$ tales que $\tau_2\subseteq\tau_1$. Si $(X,\tau_1)$ es conexo, entonces $(X,\tau_2)$ también lo es.
    \end{exa}

    \begin{proof}
        
    \end{proof}

    \begin{exa}
        Sea $X=\left\{a,b,c\right\}$ y considere la topología $\tau=\left\{X,\emptyset,\left\{a,b\right\},\left\{b,c \right\},\left\{b\right\} \right\}$. Entonces, $(X,\tau)$ es conexo.

        Pero, $A=\left\{a,c\right\}$ es un conjunto tal que $(A,\tau_A=\left\{A,\emptyset,\left\{a\right\},\left\{c\right\} \right\})$ no es conexo.

        Por tanto, la propiedad de ser conexo no se hereda.
    \end{exa}

    \renewcommand{\theenumi}{\arabic{enumi}}
    
    \begin{propo}
        Sea $(\mathcal{L},\prec)$ un conjunto ordenado tal que:
        \begin{enumerate}
            \item Dados $x,y\in\mathcal{L}$ tales que $x\prec y$, existe $z\in\mathcal{L}$ que cumple $x\prec z\prec y$.
            \item Todo subconjunto no vacío de $\mathcal{L}$ acotado superiormente tiene mínima cota superior.
        \end{enumerate}
        Entonces, al considerar $(\mathcal{L},\tau_{\prec})$ tenemos que en $\mathcal{L}$, el mismo $\mathcal{L}$, cada intervalo abierto, cerrado, semi-abierto y cualquier rayo, son conjuntos conexos
    \end{propo}

    \begin{proof}
        Sea $Y\subseteq\mathcal{L}$ tal que $Y=\mathcal{L}$ o $Y$ es un intervalo o $Y$ es un rayo.

        Tenemos que dados $p,q\in Y$ con $p\prec q$ se cumple que $[p,q]\subseteq Y$, es decir que $Y$ es un conjunto convexo. Mostremos que $Y$ es conexo.

        Sean $A,B\subseteq Y$ tales que $A,B\in{\tau_{\prec}}_Y$ son ambos no vacíos y $A\cap B\neq\emptyset$. Mostraremos que $A\cup B$ es un subconjunto propio de $Y$, es decir que
        \begin{equation*}
            A\cup B\subsetneqq Y
        \end{equation*}
        Sean $a\in A$ y $b\in B$. Podemos suponer que $a\prec b$. Como $Y$ es convexo, entonces $[a,b]\subseteq Y$. Sea
        \begin{equation*}
            A_0=A\cap [a,b]\neq\emptyset
        \end{equation*}
        y
        \begin{equation*}
            B_0=B\cap [a,b]\neq\emptyset
        \end{equation*}
        Entonces $A_0,B_0$ son dos conjuntos abiertos no vacíos en $([a,b],{\tau_\prec}_{[a,b]})$. Para todo $x\in A_0$ se tiene que $x\prec b$. Existe pues $c\in\mathcal{L}$ tal que $c$ es la mínima cota superior de $A_0$. Probemos que $c\in[a,b]$ y que $c\notin A\cup B$.
        \begin{enumerate}
            \item $c\notin A_0$. Suponga que $c\in A_0$, entonces $a\preceq c\prec b$. Como $A_0$ es abierto en $[a,b]$ existe $d\in\mathcal{L}$ tal que $[c,d)\subseteq A_0$ y $c\prec d$. Como $c\prec d$ entonces existe $y\in\mathcal{L}$ tal que $c\prec y\prec d$. Luego $y\in A_0$\contradiction. Por tanto, $c\notin A_0$.
            \item $c\in[a,b]$. Sea $y_0\in A_0=A\cap [a,b]$. Por la parte anterior se tiene que $y\prec c\preceq b$, luego $c\in [y_0,b]\subseteq [a,b]$.
            \item $c\notin A$.
            \item $c\notin B_0$. Suponga que $c\in B_0=B\cap [a,b]$, entonces $a\prec c$. $B_0$ es abierto en $[a,b]$, luego existe $d\in[a,b]$ tal que $d\prec c$ y $(d,c]\subseteq B_0$. Existe entonces $x\in A_0$ tal que $d\prec x\prec c$, luego $x\in A$ y $x\in B$ pues $(d,c]\subseteq B_0$\contradiction. Luego, $c\notin B_0$.
            \item $c\notin B$.
        \end{enumerate}
        Entonces, $c\notin A\cup B$. Así, $A\cup B$ no puede formar una partición de $Y$, es decir que $Y$ es conexo.
    \end{proof}

    \begin{cor}
        Consideremos $(\mathbb{R},\tau_u)$, entonces cada intervalo, cada rayo y el mismo conjunto $\mathbb{R}$ son subconjuntos conexos de $(\mathbb{R},\tau_u)$.
    \end{cor}

    \begin{proof}
        Es inmediato del teorema anterior.
    \end{proof}

    \begin{propo}
        Sea $C$ un subconjunto de $(\mathbb{R},\tau_u)$. Entonces, $C$ es conexo si y sólo si $C$ es un intervalo o $C$ es un rayo o $C=\mathbb{R}$ o $C=\emptyset$ o $C=\left\{r\right\}$ con $r\in\mathbb{R}$.
    \end{propo}

    \begin{proof}
        $\Rightarrow):$ Sea $C\subseteq\mathbb{R}$ tal que $C\neq\mathbb{R}$, $C\neq\emptyset$, $C$ no es un intervalo ni un rayo ni un conjunto unipuntual. Entonces, existen $a,b\in C$ y un punto $x\in\mathbb{R}-C$ tal que
        \begin{equation*}
            a<x<b
        \end{equation*}
        Sea
        \begin{equation*}
            A=\left\{c\in C\Big|c<x \right\}\quad\textup{y}\quad B=\left\{c\in C\Big|x<c \right\}
        \end{equation*}
        tanto $A$ como $B$ son conjuntos no vacíos. Otra forma de expresarlos es como:
        \begin{equation*}
            A=(-\infty,x)\cap C\quad\textup{y}\quad B=(x,\infty)\cap C
        \end{equation*}
        $A$ y $B$ son dos conjuntos no vacíos abiertos en $(C,{\tau_u}_C)$ tales que $A\cap B=\emptyset$. Además, $A\cup B=C$. Luego $C$ no es conexo. 

        $\Leftarrow):$ Es inmediata del teorema anterior.
    \end{proof}

    \begin{obs}
        Sea $(X,\tau)$ un espacio toplógico no conexo. Entonces, existen $U,V\in\tau-\left\{\emptyset \right\}$ tales que
        \begin{equation*}
            U\cap V=X\quad X=U\dot{\cup}V
        \end{equation*}
        por ende, $U=X-V$ y $V=X-U$ son cerrados disjuntos tales que
        \begin{equation*}
            \Int{U}=U=\Cls{U}
        \end{equation*}
        Análogamente
        \begin{equation*}
            \Int{V}=V=\Cls{V}
        \end{equation*}
        Además, $U\cap\Cls{V}=\Cls{U}\cap V=\emptyset$. También, $\Fr{U}=\Fr{V}=\emptyset$.
    \end{obs}

    \begin{propo}
        Sea $(X,\tau)$ un espacio topológico, entonces los siguientes enunciados son equivalentes:
        \begin{enumerate}
            \item $(X,\tau)$ es conexo.
            \item Los únicos subconjuntos de $X$ que son a la vez abiertos y cerrados son $X$ y $\emptyset$.
            \item Los únicos subconjuntos de $X$ con frontera vacía son $X$ y $\emptyset$.
        \end{enumerate}
    \end{propo}

    \begin{proof}
        $(1)\Rightarrow(2)$: Sea $A\subseteq X$ tal que $A$ es abierto y cerrado a la vez, es decir que $A,X-A\in\tau$. Suponga que $A\neq X,\emptyset$, se tiene pues que
        \begin{equation*}
            X=A\cup (X-A)\quad\textup{y}\quad A\cap (X-A)=\emptyset
        \end{equation*}
        siendo $A,X-A\neq\emptyset$. Luego esto implicaría que $(X,\tau)$ no es conexo\contradiction. Por tanto, $A=\emptyset$ o $A=\mathbb{R}$.

        $(2)\Rightarrow(3):$ Sea $A\subseteq X$ tal que $\Fr{A}=\emptyset$. Entonces,
        \begin{equation*}
            \emptyset=\Fr{A}=\Cls{A}-\Int{A}\Rightarrow \Int{A}=\Cls{A}
        \end{equation*}
        luego $A$ es cerrado y abierto en $(X,\tau)$. Por tanto, $A=X$ o $A=\emptyset$.

        $(3)\Rightarrow (1):$ Suponga que $U,V\in\tau$ son tales que
        \begin{equation*}
            U\cap V=\emptyset\quad\textup{y}\quad U\cup V=X
        \end{equation*}
        Se tiene que $U=X-V$ y $V=X-U$ donde se sigue que $U,V$ son cerrados en $(X,\tau)$. Así,
        \begin{equation*}
            \Cls{U}=U=\Int{U}\quad\textup{y}\quad\Cls{V}=V=\Int{V}
        \end{equation*}
        por tanto, $\Fr{U}=\emptyset$, es decir que $U=\emptyset$ y $V=X$, o $U=X$ y $V=\emptyset$. Luego, $(X,\tau)$ es conexo.
    \end{proof}

    \begin{mydef}
        Sea $(X,\tau)$ un espacio topológico. Dos conjuntos $U,V\in\tau$ se dicen \textbf{mutuamente separados} si $U\cap\Cls{V}=\Cls{U}\cap V=\emptyset$.
    \end{mydef}

    \begin{mydef}
        Sea $(X,\tau)$ un espacio topológico y sea $Y\subseteq X$. Una pareja $A,B$ de subconjuntos de $X$ mutuamente separados en $(X,\tau)$ es una \textbf{separación de $Y$ en $(X,\tau)$} si
        \begin{equation*}
            Y=A\cup B,\quad Y\cap A\textup{ y }Y\cap B \neq\emptyset
        \end{equation*}
    \end{mydef}

    \begin{propo}
        Sean $(X,\tau)$ un espacio topológico y $Y\subseteq X$. Entonces $(Y,\tau_Y)$ es conexo si y sólo si no existe una separación de $Y$ en $X$.
    \end{propo}

    \begin{proof}
        $\Rightarrow):$ Suponga que $A,B\subseteq X$ son una separación de $Y$ en $(X,\tau)$. Tenemos que
        \begin{equation*}
            \Cls{A}\cap B=A\cap\Cls{B}=\emptyset
        \end{equation*}
        también,
        \begin{equation*}
            Y=A\cup B\quad Y\cap A\neq\emptyset\textup{ y }Y\cap B\neq\emptyset
        \end{equation*}
        Se tiene pues que
        \begin{equation*}
            \begin{split}
                \Cls{A}\cap Y &=\Cls{A}\cap (A\cup B)\\
                &=(\Cls{A}\cap A)\cup (\Cls{A}\cap B)\\
                &=A\\
            \end{split}
        \end{equation*}
        análogamente se prueba que $\Cls{B}\cap Y=B$. Por tanto, $A,B$ forman una partición de $(Y,\tau_Y)$\contradiction. Por tanto, $(Y,\tau_Y)$ es conexo.

        $\Leftarrow):$ Suponga que $(Y,\tau_Y)$ no es conexo. Entonces existen $A,B\in\tau_Y$ con $A,B\neq\emptyset$ tales que
        \begin{equation*}
            A\cap B=\emptyset\quad\textup{y}\quad A\cup B=Y
        \end{equation*}
        Luego $A$ y $B$ son conjuntos abiertos y cerrados en $(Y,\tau_Y)$.
        \begin{equation*}
            A=\Cls{A}\cap Y\quad\textup{y}\quad B=\Cls{B}\cap Y
        \end{equation*}
        Siendo tales que
        \begin{equation*}
            \emptyset=A\cap B=(\Cls{A}\cap Y)\cap B=\Cls{A}\cap(Y\cap B)=\Cls{A}\cap B
        \end{equation*}
        de forma análoga $A\cap\Cls{B}=\emptyset$. Así, $A$ y $B$ forman una separación de $Y$ en $(X,\tau)$.
    \end{proof}

    \begin{cor}
        Sea $(X,\tau)$ un espacio topológico. Entonces, $(X,\tau)$ es conexo si y sólo si no existen $A,B\subseteq X$ no vacíos tales que
        \begin{equation*}
            X=A\cup B\quad A\cap \Cls{B}=\emptyset=\Cls{A}\cap B
        \end{equation*}
    \end{cor}

    \begin{proof}
        Inmediata de la proposición anterior.
    \end{proof}

    \begin{propo}
        Sea $(X,\tau)$ un espacio topológico y sean $Y,Z\subseteq X$ tales que $Y\subseteq Z$. Si $U,V$ es una separación de $Z$ en $(X,\tau)$ y $Y$ es conexo, entonces $Y\subseteq U$ ó $Y\subseteq V$.
    \end{propo}

    \begin{proof}
        Se tiene que $Y\subseteq U\cup V$. Sea
        \begin{equation*}
            U_1=Y\cup U\quad\textup{y}\quad V_1=Y\cap V
        \end{equation*}
        entonces,
        \begin{equation*}
            Y=U_1\cup V_1
        \end{equation*}
        Como $U\cap\Cls{V}=\emptyset=\Cls{U}\cap V$, entonces
        \begin{equation*}
            \begin{split}
                \Cls{U_1}\cap V_1&=\Cls{Y\cap U}\cap (Y\cap V)\\
                &\subseteq\Cls{U}\cap (Y\cap V)\\
                &=(\Cls{U}\cap V)\cap Y\\
                &=\emptyset\\
                \Rightarrow \Cls{U_1}\cap V_1&=\emptyset\\
            \end{split}
        \end{equation*}
        de forma análoga se obtiene que $U_1\cap\Cls{V_1}=\emptyset$. Como $Y$ es conexo entonces $U_1=\emptyset$ o $V_1=\emptyset$, es decir que $Y\subseteq V$ o $Y\subseteq U$.
    \end{proof}

    \begin{propo}
        Sea $(X,\tau)$ un espacio topológico y $\left\{A_\alpha\right\}_{\alpha\in I}$ una familia de subconjuntos conexos de $X$ tales que
        \begin{equation*}
            \bigcap_{\alpha\in I}A_\alpha\neq\emptyset
        \end{equation*}
        Entonces $\bigcup_{\alpha\in I}A_\alpha$ es conexo.
    \end{propo}

    \begin{proof}
        Sea $A=\bigcup_{\alpha\in I}A_\alpha$. Supongamos que $A$ no es conexo, entonces existe una separación $U,V\in\tau$ de $A$ en $X$. Tomemos $\beta\in I$. Como $A_\beta\subseteq A$ y $A_\beta$ es conexo, entonces por la proposición anterior se tiene que:
        \begin{equation*}
            A_\beta\subseteq U\quad\textup{ó}\quad A_\beta\subseteq V
        \end{equation*}
        Podemos suponer sin pérdida de generalidad que $A_\beta\subseteq U$. Como $\bigcap_{\alpha\in I}A_\alpha\subseteq A_\beta$, entonces para todo $\gamma\in I$ se tiene que $A_\gamma\cap U\neq\emptyset$, luego por ser cada $A_\gamma$ conexo debe suceder que:
        \begin{equation*}
            A_\gamma\subseteq U
        \end{equation*}
        para todo $\gamma\in I$. Por tanto:
        \begin{equation*}
            A=\bigcup_{\alpha\in I}A_\alpha\subseteq U
        \end{equation*}
        así, $A\cap V=\emptyset$\contradiction pues $U$ y $V$ forman una separación de $A$. Por tanto $A$ debe ser conexo.
    \end{proof}

    \begin{propo}
        Sean $(X_1,\tau_1)$ y $(X_2,\tau_2)$ espacios topológicos tales que existe una función continua y suprayectiva $\cf{f}{(X_1,\tau_1)}{(X_2,\tau_2)}$. Si $(X_1,\tau_1)$ es conexo, entonces $(X_2,\tau_2)$ también lo es.
    \end{propo}

    \begin{proof}
        Sea $A\subseteq X_2$ tal que $A,X_2-A\in\tau_2$. Suponga que $A\neq\emptyset$, para probar que $(X_2,\tau_2)$ es conexo basta con ver que $A=X_2$. En efecto, veamos que como $f$ es suprayectiva entonces $f^{-1}(A)\neq\emptyset$ y, al ser $f$ continua se tiene que
        \begin{equation*}
            f^{-1}(A)\in\tau_1
        \end{equation*}
        Pero,
        \begin{equation*}
            f^{-1}(X_2-A)=X_1-f^{-1}(A)
        \end{equation*}
        donde $X_2-A\in\tau_2$, luego $X_1-f^{-1}(A)\in\tau_1$. Por ser $(X_1,\tau_1)$ conexo, al ser $f^{-1}(A)\neq\emptyset$ debe tenerse pues que:
        \begin{equation*}
            f^{-1}(A)=X_1
        \end{equation*}
        (pues $f^{-1}(A)$ y $X_1-f^{-1}(A)$ están en $\tau_1$). Por tanto
        \begin{equation*}
            A=f(f^{-1}(A))=f(X_1)=X_2
        \end{equation*}
        lo que prueba el resultado.
    \end{proof}

    \begin{cor}
        La propiedad de ser conexo es topológica.
    \end{cor}

    \begin{proof}
        Es inmediata del teorema anterior.
    \end{proof}

    \begin{propo}
        Sea $(X,\tau)$ un espacio topológico, y sea $Y=\left\{a,b\right\}$ dotado de la topología discreta $\tau_D=\left\{\emptyset,T,\left\{a\right\},\left\{b\right\} \right\}$. Entonces $(X,\tau)$ conexo si y sólo si no es posible definir una función $\cf{f}{(X,\tau)}{(Y,\tau_D)}$ que sea suprayectiva y continua.
    \end{propo}

    \begin{proof}
        $\Rightarrow):$ Suponga que se puede definir tal función, entonces por la proposición anterior se seguiría que $(Y,\tau_D)$ es conexo\contradiction, pues $Y=\left\{a\right\}\cup\left\{b\right\}$ siendo $\left\{a\right\},\left\{b\right\}\in\tau_D$ tales que $\left\{a\right\}\cap\left\{b\right\}=\emptyset$. Por tanto, no es posible definir una función con tales propiedades.

        $\Leftarrow):$ Suponga que $(X,\tau)$ no es conexo, entonces existen $U,V\in\tau-\left\{\emptyset\right\}$ tales que
        \begin{equation*}
            X=U\cup V\quad\textup{y}\quad U\cap V=\emptyset
        \end{equation*}
        defina $\cf{f}{(X,\tau)}{(Y,\tau_D)}$ como sigue:
        \begin{equation*}
            f(x)=\left\{
                \begin{array}{lcr}
                    a & \textup{ si } & x\in U\\
                    b & \textup{ si } & x\in V\\
                \end{array}
            \right.,\quad\forall x\in X.
        \end{equation*}
        se tiene que $f^{-1}(\left\{a\right\})=U$, $f^{-1}(\left\{b\right\})=V$, luego $f$ es continua. Además por definición $f$ es suprayectiva. Lo anterior prueba la contrapositiva.
    \end{proof}

    \begin{propo}
        Sean $(X,\tau)$ un espacio topológico, $A,B\subseteq X$ tales que $A\subseteq B\subseteq \Cls{A}$. Si $A$ es conexo, entonces $B$ es conexo.
    \end{propo}

    \begin{proof}
        Suponga que $B$ no es conexo. Podemos definir una función $\cf{f}{(B,\tau_B)}{(Y,\tau_D)}$ continua y suprayectiva, donde $Y=\left\{a,b\right\}$. Como $B\subseteq \Cls{A}$ se tiene que:
        \begin{equation*}
            \Cls{A}^B=\Cls{A}\cap B=B
        \end{equation*}
        Por lo cual $f\left(\Cls{A}^B\right)=f(B)=Y$, por ser $f$ continua,
        \begin{equation*}
            Y=f\left(\Cls{A}^B\right)\subseteq\Cls{f(A)}=f(A)\Rightarrow f(A)=Y
        \end{equation*}
        %TODO ¿\Cls{f(A)}=f(A)?
        Tenemos pues que $\cf{f\big|_A}{(A,\tau_A)}{(Y,\tau_D)}$ es una función continua (por ser reestricción) y suprayectiva. Por ende, $A$ no es conexo\contradiction. Por tanto, $B$ es conexo. 
    \end{proof}

    \begin{cor}
        Sea $(X,\tau)$ es un espacio topológico. Si $A\subseteq X$ es conexo, entonces $\Cls{A}$ es conexo.  
    \end{cor}

    \begin{proof}
        Es inmediato del teorema anterior.
    \end{proof}

    \begin{theor}[\textbf{Teorema del valor medio}]
        Sea $(X,\tau)$ un espacio conexo, $(Y,\prec)$ un conjunto ordenado y $\cf{f}{(X,\tau)}{(Y,\tau_\prec)}$ una función continua. Si $a,b\in X$ y $\gamma\in Y$ es tal que:
        \begin{equation*}
            f(a)\prec \gamma\prec f(b)
        \end{equation*}
        entonces existe $c\in X$ tal que $f(x)=\gamma$.
    \end{theor}

    \begin{proof}
        Suponga que no existe $c\in X$ tal que $f(c)=\gamma$
    \end{proof}

    \begin{propo}
        Sean $(X_1,\tau_1)$ y $(X_2,\tau_2)$ dos espacios conexos. Entonces $(X_1\times X_2,\tau_p )$ es un espacio conexo.
    \end{propo}

    \begin{proof}
        

        Entonces, para todo $x\in X_1$, tenemos que $T_x=(X_1\times\left\{b\right\})\cup(\left\{x\right\}\times X_2)$ es un conexo.

        Además, para todo $x\in X_1$, $(a,b)\in T_x$ (recordando que $a\in X_1$ es arbitrario fijo), luego $\bigcup_{ x\in X_1}T_x$ es conexo. Veamos que
        \begin{equation*}
            \bigcup_{ x\in X_1}T_x=X_1\times X_2
        \end{equation*}
        En efecto, sea $(p,q)\in X_1\times X_2$, entonces $(p,q)\in T_p\subseteq\bigcup_{x\in X_1}T_x$.
    
        Se sigue entonces que $(X_1\times X_2,\tau_p)$ es conexo.
    \end{proof}

    \begin{excer}
        Si $\left\{(X_1,\tau_1),...,(X_n,\tau_n)\right\}$ son espacios topológicos conexos, entonces
        \begin{equation*}
            X=\prod_{ i=1}^n X_i
        \end{equation*}
        dotado de la topología producto es un espacio conexo.
        
        \textit{Sugerencia}. Se puede demostrar que $(X_1\times...\times X_{ n-1})\times X_n$ es homeomorfo a $X_1\times...\times X_n$.
    \end{excer}

    \begin{proof}
        %TODO
    \end{proof}

    \begin{propo}
        Sea $\left\{(X_\alpha,\tau_\alpha)\right\}_{\alpha\in I}$ una familia arbitraria de espacios topológicos y sea
        \begin{equation*}
            X=\prod_{\alpha\in I}X_\alpha
        \end{equation*}
        Entonces $(X,\tau_p)$ es conexo si y sólo si para todo $\alpha\in I$, $(X_\alpha,\tau_\alpha)$ es un espacio conexo.
    \end{propo}

    \begin{proof}
        $\Rightarrow):$ Sea $\alpha\in I$ y considere la función $\cf{p_\alpha}{(X,\tau_p)}{(X_\alpha,\tau_\alpha)}$. Esta función es continua y suprayectiva, se sigue entonces que $(X_\alpha,\tau_\alpha)$ es conexo.

        $\Leftarrow):$ Sea $b=(b_\alpha)_{\alpha\in I}\in X$ elemento arbitrario fijo de  $X$ y, sea $J=\left\{\alpha_1,...,\alpha_n \right\}\subseteq I$. Definimos
        \begin{equation*}
            X_J=\left\{(x_\alpha)_{\alpha\in I}\in X\Big|x_\alpha=b_\alpha \textup{ para }\alpha\notin J \right\}
        \end{equation*}
        Se tiene que $X_J\neq\emptyset$ pues $b\in X_J$. Podemos escribir $X_J$ como
        \begin{equation*}
            X_J=\prod_{\alpha\in I}Y_\alpha
        \end{equation*}
        donde
        \begin{equation*}
            Y_\alpha=\left\{
                \begin{array}{lcr}
                    \left\{b_\alpha\right\} & \textup{ si } & \alpha\notin J\\
                    X_\alpha & \textup{ si } & \alpha\in J\\
                \end{array}
            \right.
        \end{equation*}
        Sea $X'=\prod_{i=1}^\infty X_{\alpha_i}$. Definamos $\cf{\varphi}{(X',\tau_p)}{(X_J,{\tau_p}_{X_J})}$ tal que
        \begin{equation*}
            (x_{\alpha_1},...,x_{\alpha_n})\mapsto (y_\alpha)_{\alpha\in I}
        \end{equation*}
        donde
        \begin{equation*}
            y_\alpha=\left\{
                \begin{array}{lcr}
                    b_\alpha & \textup{ si } & \alpha\notin J\\
                    x_\alpha & \textup{ si } & \alpha\in J\\
                \end{array}
            \right.
        \end{equation*}
        \begin{enumerate}
            \item \textbf{$\varphi$ es suprayectiva}. Veamos que $\varphi(X')=X_J$. En efecto, sea $\zeta=(\zeta_{\alpha})_{\alpha\in I}\in X_J$, es decir que si $\alpha\notin J$ se tiene que $\zeta_\alpha=b_\alpha$, luego:
            \begin{equation*}
                \begin{split}
                    \varphi((\zeta_{\alpha_1},...,\zeta_{\alpha_n}))=\zeta
                \end{split}
            \end{equation*}
            se concluye que $\varphi(X')=X_J$.
            \item \textbf{$\varphi$ es continua}. Sea $U=\prod_{\alpha\in I}U_\alpha$ un básico de $(X,\tau_p)$, es decir que $U_\alpha\in\tau_\alpha$ para todo $\alpha\in I$ (y coincide con $X_\alpha$ para casi todo $\alpha\in I$ salvo una cantidad finita). Tomemos
            \begin{equation*}
                U'=U\cap X_J\in {\tau_p}_{ X_J}-\left\{\emptyset \right\}
            \end{equation*}
            Se tiene que $U'\in{\tau_p}_{X_J}$, más aún:
            \begin{equation*}
                \begin{split}
                    U'&=\left(\prod_{\alpha\in I}U_\alpha \right)\cap\left(\prod_{\alpha\in I}Y_\alpha \right)\\
                    &=\prod_{\alpha\in I}(U_\alpha\cap Y_\alpha)\\
                \end{split}
            \end{equation*}
            donde
            \begin{equation*}
                U_\alpha\cap Y_\alpha=\left\{
                    \begin{array}{lcr}
                        \left\{b_\alpha\right\} & \textup{ si } & \alpha\notin J\\
                        U_\alpha & \textup{ si } & \alpha\in J\\
                    \end{array}
                \right.,\quad\forall \alpha\in I
            \end{equation*}
            Por tanto
            \begin{equation*}
                \varphi^{-1}(U')=\prod_{\alpha\in J}U_\alpha\in{\tau_p}_{X'}
            \end{equation*}
            luego $\varphi$ es una función continua.
        \end{enumerate}
        Por el ejercicio anterior se tiene que $(X',{\tau_p}_{X'})$ es conexo, entonces $(X_J,\tau_p)$ es conexo (por ser $\varphi$ continua y suprayectiva).

        Sea
        \begin{equation*}
            \mathcal{F}=\left\{J\subseteq I\Big|J\textup{ es un conjunto finito} \right\}
        \end{equation*}
        Para todo $J\in\mathcal{F}$, $X_J$ es conexo por lo probado anteriormente para el cual $b\in X_J$. Por ende, el conjunto
        \begin{equation*}
            \bigcup_{ J\in\mathcal{F}}X_J=Y
        \end{equation*}
        es conexo en $(X,\tau_p)$. Veamos que
        \begin{equation*}
            \Cls{Y}=X
        \end{equation*}
        En efecto, sea $W=\prod_{\alpha\in I}W_\alpha$ un básico de $\tau_p$ con $W\neq\emptyset$. Se tiene que para todo $\alpha\in I$, $W_\alpha\in\tau_\alpha$ y además existe $K\in\mathcal{F}$ tal que si $\alpha\notin K$, $W_\alpha=X_\alpha$.

        Para $\alpha\in K$, $x_\alpha\in X_\alpha$ y definimos
        \begin{equation*}
            y_\alpha=\left\{
                \begin{array}{lcr}
                    x_\alpha & \textup{ si } & \alpha\in K\\
                    b_\alpha & \textup{ si } & \alpha\notin K\\
                \end{array}
            \right.,\quad\forall\alpha\in I
        \end{equation*}
        Entonces $y=(y_\alpha)_{\alpha\in I}\in X_K\cap W$ lo que implica que $Y\cap W\neq\emptyset$. Luego $\Cls{Y}=X$ y así, $(X,\tau_p)$ es conexo.
    \end{proof}

    \begin{mydef}
        Sea $(X,\tau)$ un espacio topológico y sea $p\in X$. Tomemos
        \begin{equation*}
            \mathcal{C}=\left\{C\subseteq X\Big|C\textup{ es conexo y }p\in C \right\}
        \end{equation*}
        tenemos que $\left\{p\right\}\in\mathcal{C}$ y además para todo $C\in\mathcal{C}$, $p\in C$. Por tanto $C_p=\bigcup_{ C\in\mathcal{C}}C$ es un conjunto conexo, el cual llamaremos \textbf{la componente conexa de $p$}.
    \end{mydef}

    \begin{obs}
        Se tiene lo siguiente:
        \begin{enumerate}
            \item $C_p$ es el máximo conexo de $X$ que contiene a $p\in X$.
            \item $C_p$ es un conjunto cerrado.
            \item Sean $p,q\in X$, entonces $C_p=C_q$ ó $C_p\cap C_q=\emptyset$.
        \end{enumerate}
    \end{obs}

    \begin{proof}
        De 1): Es inmediata de la definición.

        De 2): Como $C_p$ es conexo, entonces $\Cls{C_p}$ es conexo, luego por maximalidad $\Cls{C_p}\subseteq C_p$ lo cual implica que $C_p$ es cerrado.

        De 3): Si $C_p\cap C_q\neq\emptyset$ entonces $C_p\cup C_q$ es conexo, pero es tal que contiene a $p$ y $q$, luego
        \begin{equation*}
            C_p\subseteq C_p\cup C_q\subseteq C_p\quad\textup{y}\quad C_q\subseteq C_p\cup C_q\subseteq C_q
        \end{equation*}
        por tanto, $C_p\cup C_q=C_p=C_q$.
    \end{proof}

    \begin{mydef}
        Sea $(X,\tau)$ un espacio topológico, definimos sobre $X$ la relación $\sim$ siguiente:
        \begin{equation*}
            x\sim y\iff\textup{no existen }A,B\in\tau\textup{ tales que }A\cap B=\emptyset, A\cup B=X, x\in A\textup{ y }y\in B
        \end{equation*}
        Esta es una realción de equivalencia sobre $X$. Esta relación de equivalencia dice básicamente que dos elementos están relacionados si y sólo si están en la misma componente conexa.
    \end{mydef}

    \begin{proof}
        
    \end{proof}

\end{document}