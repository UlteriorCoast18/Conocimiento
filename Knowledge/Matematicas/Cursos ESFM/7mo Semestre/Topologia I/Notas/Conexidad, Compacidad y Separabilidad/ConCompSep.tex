\documentclass[12pt]{report}
\usepackage[spanish]{babel}
\usepackage[utf8]{inputenc}
\usepackage{amsmath}
\usepackage{amssymb}
\usepackage{amsthm}
\usepackage{graphics}
\usepackage{subfigure}
\usepackage{lipsum}
\usepackage{array}
\usepackage{multicol}
\usepackage{enumerate}
\usepackage[framemethod=TikZ]{mdframed}
\usepackage[a4paper, margin = 1.5cm]{geometry}

%En esta parte se hacen redefiniciones de algunos comandos para que resulte agradable el verlos%

\renewcommand{\theenumii}{\roman{enumii}}

\def\proof{\paragraph{Demostración:\\}}
\def\endproof{\hfill$\blacksquare$}

\def\sol{\paragraph{Solución:\\}}
\def\endsol{\hfill$\square$}

%En esta parte se definen los comandos a usar dentro del documento para enlistar%

\newtheoremstyle{largebreak}
  {}% use the default space above
  {}% use the default space below
  {\normalfont}% body font
  {}% indent (0pt)
  {\bfseries}% header font
  {}% punctuation
  {\newline}% break after header
  {}% header spec

\theoremstyle{largebreak}

\newmdtheoremenv[
    leftmargin=0em,
    rightmargin=0em,
    innertopmargin=-2pt,
    innerbottommargin=8pt,
    hidealllines = true,
    roundcorner = 5pt,
    backgroundcolor = gray!60!red!30
]{exa}{Ejemplo}[section]

\newmdtheoremenv[
    leftmargin=0em,
    rightmargin=0em,
    innertopmargin=-2pt,
    innerbottommargin=8pt,
    hidealllines = true,
    roundcorner = 5pt,
    backgroundcolor = gray!50!blue!30
]{obs}{Observación}[section]

\newmdtheoremenv[
    leftmargin=0em,
    rightmargin=0em,
    innertopmargin=-2pt,
    innerbottommargin=8pt,
    rightline = false,
    leftline = false
]{theor}{Teorema}[section]

\newmdtheoremenv[
    leftmargin=0em,
    rightmargin=0em,
    innertopmargin=-2pt,
    innerbottommargin=8pt,
    rightline = false,
    leftline = false
]{propo}{Proposición}[section]

\newmdtheoremenv[
    leftmargin=0em,
    rightmargin=0em,
    innertopmargin=-2pt,
    innerbottommargin=8pt,
    rightline = false,
    leftline = false
]{cor}{Corolario}[section]

\newmdtheoremenv[
    leftmargin=0em,
    rightmargin=0em,
    innertopmargin=-2pt,
    innerbottommargin=8pt,
    rightline = false,
    leftline = false
]{lema}{Lema}[section]

\newmdtheoremenv[
    leftmargin=0em,
    rightmargin=0em,
    innertopmargin=-2pt,
    innerbottommargin=8pt,
    roundcorner=5pt,
    backgroundcolor = gray!30,
    hidealllines = true
]{mydef}{Definición}[section]

\newmdtheoremenv[
    leftmargin=0em,
    rightmargin=0em,
    innertopmargin=-2pt,
    innerbottommargin=8pt,
    roundcorner=5pt
]{excer}{Ejercicio}[section]

%En esta parte se colocan comandos que definen la forma en la que se van a escribir ciertas funciones%

\newcommand\abs[1]{\ensuremath{\biglvert#1\bigrvert}}
\newcommand\divides{\ensuremath{\bigm|}}
\newcommand\cf[3]{\ensuremath{#1:#2\rightarrow#3}}
\newcommand\contradiction{\ensuremath{\#_c}}
\newcommand{\V}[1]{\ensuremath{\mathcal{V}(#1)}}
\newcommand{\Int}[1]{\ensuremath{\mathring{#1}}}
\newcommand{\Cls}[1]{\ensuremath{\overline{#1}}}
\newcommand{\Fr}[1]{\ensuremath{\textup{Fr}(#1)}}
\newcommand{\natint}[1]{\ensuremath{\left[|#1|\right]}}
\newcommand{\floor}[1]{\ensuremath{\lfloor#1\rfloor}}
\newcommand{\Card}[1]{\ensuremath{\textup{Card}\left(#1\right)}}
\newcommand{\Pot}[1]{\ensuremath{\mathcal{P}\left(#1\right)}}
\newcommand{\id}[1]{\ensuremath{\textup{id}_{#1}}}

%recuerda usar \clearpage para hacer un salto de página

\begin{document}
    \setlength{\parskip}{5pt} % Añade 5 puntos de espacio entre párrafos
    \setlength{\parindent}{12pt} % Pone la sangría como me gusta
    \title{Notas curso Topología I.
    
    Separabilidad, Numerabilidad y Conexidad}
    \author{Cristo Daniel Alvarado}
    \maketitle

    \tableofcontents %Con este comando se genera el índice general del libro%

    \setcounter{chapter}{1} %En esta parte lo que se hace es cambiar la enumeración del capítulo%
    
    \chapter{Separabilidad}
    
    \section{Axiomas de separación}

    \begin{mydef}
        Sea $(X,\tau)$ un espacio topológico.
        \begin{enumerate}
            \item $(X,\tau)$ se dice un \textbf{espacio $T_0$} si dados $a,b\in X$ con $a\neq b$ existe un abierto que contiene a alguno de los dos puntos, pero no contiene al otro.
            \item $(X,\tau)$ se dice un \textbf{espacio $T_1$} si dados $a,b\in X$ con $a\neq b$ existen $U,V\subseteq X$ abiertos tales que $a\in U$, $b\in V$ y, $a\notin V$, $b\notin U$.
            \item $(X,\tau)$ se dice un \textbf{espacio $T_2$} si dados $a,b\in X$ con $a\neq b$ existen $U,V\subseteq X$ abiertos tales que $a\in U$, $b\in V$ y, $U\cap V=\emptyset$. Esto es equivalente a que el espacio sea de Hausdorff.
            \item $(X,\tau)$ se dice un \textbf{espacio $T_3$} si dados $p\in X$ y $A\subseteq X$ cerrado tal que $p\notin A$, existen $U,V\in\tau$ tales que $p\in U$, $A\subseteq V$ y $U\cap V=\emptyset$.
            \item $(X,\tau)$ se dice un \textbf{espacio $T_4$} si dados $A,B\subseteq X$ cerrados y disjuntos, existen $U,V\in\tau$ tales que $A\subseteq U$, $B\subseteq V$ y, $U\cap V=\emptyset$.
            \item $(X,\tau)$ se dice un \textbf{espacio regular} si es un espacio $T_3$ y $T_1$.
            \item $(X,\tau)$ se dice un \textbf{espacio normal} si es un espacio $T_4$ y $T_1$.
        \end{enumerate}
    \end{mydef}

    \begin{obs}
        Notemos que:
        \begin{equation*}
            T_2\Rightarrow T_1\Rightarrow T_0
        \end{equation*}
    \end{obs}

    \begin{exa}
        Considere al conjunto $X=\left\{1,2 \right\}$ y $\tau=\left\{X,\emptyset,\left\{1 \right\} \right\}$. Afirmamos que $(X,\tau)$ es $T_0$, pero no es $T_1$ y, por ende tampoco puede ser $T_2$.
    \end{exa}

    
    \begin{exa}
        Sea $(\mathbb{R},\tau_{cf})$. Afirmamos que $(\mathbb{R},\tau_u)$ es $T_1$. En efecto, sean $r,s\in\mathbb{R}$ tales que $r\neq s$. Los conjuntos $U=\mathbb{R}-\left\{s \right\},V=\mathbb{R}-\left\{r \right\}\in\tau_{cf}$ pues sus complementos son finitos, además:
        \begin{equation*}
            r\in U\quad\textup{y}\quad s\in V
        \end{equation*}
        además, $r\notin V$ y $s\notin U$. Por tanto, el espacio de $T_1$. Pero no es $T_2$.

        En efecto, suponga que existen $U,V\in\tau_{cf}$ abiertos tales que $\varphi=\frac{1+\sqrt{5}}{2}\in U$, $\frac{1}{\pi}\in V$ y $U\cap V=\emptyset$. En particular, se tiene que $\mathbb{R}-U$ y $\mathbb{R}-V$ son finitos. Por tanto:
        \begin{equation*}
            \begin{split}
                (\mathbb{R}-U)\cup(\mathbb{R}-V)&=\mathbb{R}-(U\cap V)\\
                &=\mathbb{R}\\
            \end{split}
        \end{equation*}
        es finito, por tanto, $\mathbb{R}$ es finito\contradiction.
    \end{exa}

    \begin{exa}
        Considere al espacio $(\mathbb{R},\tau_I=\left\{X,\emptyset \right\})$. Afirmamos que $(\mathbb{R},\tau_I)$ es $T_4$ y $T_3$, pero NO es $T_0$, pues si $\varphi=\frac{1+\sqrt{5}}{2},\frac{1}{\pi}\in\mathbb{R}$, solo hay un abierto que contiene a alguno de los dos puntos, el cual es $\mathbb{R}$, que siempre tiene a los dos puntos. Por ende, el espacio no es $T_0$.        
    \end{exa}

    \begin{propo}
        $T_4$ y $T_1\Rightarrow T_3$ y $T_1\Rightarrow T_2\Rightarrow T_1\Rightarrow T_0$.
    \end{propo}

    \begin{proof}
        La prueba se hará más adelante.
    \end{proof}

    \section{Espacios $T_1$}
    
    \begin{propo}
        Sea $(X,\tau)$ un espacio topológico. Entonces $(X,\tau)$ es un espacio $T_1$ si y sólo si todo subconjunto unitario de $X$ es cerrado.
    \end{propo}

    \begin{proof}
        Se probará la doble implicación.

        $\Rightarrow):$ Suponga que $(X,\tau)$ es $T_1$. Sea $x\in X$. Hay que probar que $X-\left\{x \right\}\in\tau$. En efecto, sea $y\in X-\left\{x \right\}$, entonces $x\neq y$. Como el espacio es $T_1$ existen un par de abiertos $U,V\in\tau$ tales que $x\in U$, $y\in V$ y $x\notin V$ y $y\notin U$.

        Como $y\in V$ y $x\notin V$, entonces $y\in V\subseteq X-\left\{x\right\}$. Luego $X-\left\{x\right\}$ es unión arbitraria de abiertos, luego es abierto. Por ende, $\left\{x\right\}$ es cerrado.

        $\Leftarrow):$ Suponga que todo subconjunto unitario de $X$ es cerrado. Sean $x,y\in X$ tales que $x\neq y$. Como $\left\{x\right\},\left\{y\right\}$ son cerrados, entonces $U=X=\left\{y\right\}$ y $V=X-\left\{x\right\}$ son abiertos y cumplen que:
        \begin{equation*}
            x\in U,y\in V\quad x\notin V,y\notin U
        \end{equation*}
        por tanto, como fueron arbitrarios los dos elementos $x,y\in X$ distintos, se sigue que $(X,\tau)$ es $T_1$.
    \end{proof}

    \begin{cor}
        Sea $(X,\tau)$ un espacio topológico. $(X,\tau)$ es $T_1$ si y sólo si todo subconjunto finito de $X$ es cerrado.
    \end{cor}

    \begin{proof}
        Es inmediata de la proposición anterior.
    \end{proof}

    \begin{cor}
        Sea $X$ un conjunto finito y $\tau$ una topología definida sobre $X$. $(X,\tau)$ es $T_1$ si y sólo $\tau=\tau_D$.
    \end{cor}

    \begin{proof}
        Es inmediata de la proposición anterior.
    \end{proof}

    \begin{propo}
        Sea $(X,\tau)$ un espacio topológico. Entonces, $(X,\tau)$ es $T_1$ si y sólo si $\tau_{cf}\subseteq\tau$.
    \end{propo}

    \begin{proof}
        Se probarán las dos implicaciones.

        $\Rightarrow):$ Sea $A\in\tau_{cf}$ con $A\neq\emptyset$, luego $X-A$ es un conjunto finito. Como $(X,\tau)$ es $T_1$, entonces $X-A$ es cerrado (debe serlo por ser finito), luego $A$ es abierto, es decir $A\in\tau$.
        
        $\Leftarrow):$ Supongamos que $\tau_{cf}\subseteq\tau$. Sean $x\in X$. El conjunto $X-\left\{x\right\}$ es finito, luego $X-\left\{x\right\}\in\tau$, por ende el conjunto $\left\{x\right\}$ es cerrado. Como el $x$ fue arbitrario, se sigue que todo conjunto unipuntual es cerrado luego, por una proposición anterior, se sigue que $(X,\tau)$ es $T_1$.
    \end{proof}

    \begin{cor}
        La topología $\tau_{cf}$ es la topología más gruesa (o menos fina) que podemos definir sobre un conjunto para que el espacio topológico $(X,\tau_{cf})$ sea $T_1$.
    \end{cor}

    \begin{proof}
        Es inmediata de la proposición anterior.
    \end{proof}

    \begin{propo}
        La propiedad de ser un espacio topológico $T_1$ es hereditaria. 
    \end{propo}

    \begin{proof}
        Sea $(X,\tau)$ un espacio topológico $T_1$ y, tomemos $Y\subseteq X$. Formemos así al espacio $(Y,\tau_Y)$, queremos ver que este espacio es $T_1$. En efecto, sea $y\in Y$, entonces:
        \begin{equation*}
            \left\{y\right\}=\left\{y \right\}\cap Y
        \end{equation*}
        luego, $\left\{y\right\}\subseteq Y$ es un conjunto cerrado en $(Y,\tau_Y)$, ya que $\left\{y\right\}\subseteq X$ es un conjunto cerrado en $(X,\tau)$. Por ende, todo conjunto unipuntual es cerrado en $(Y,\tau_Y)$, luego este subespacio es $T_1$.
    \end{proof}

    \begin{propo}
        La propiedad de ser un espacio topológico $T_1$ es topológica.
    \end{propo}

    \begin{proof}
        Sean $(X_1,\tau_1)$ y $(X_2,\tau_2)$ espacios topológicos homeomorfos y, suponga que $(X_1,\tau_1)$ es un espacio $T_1$. Sea $\cf{h}{(X_1,\tau_1)}{(X_2,\tau_2)}$ el homeomorfismo entre estos dos espacios. Como esta función es homeomorfismo, es una biyección cerrada y continua. Sea $x_2\in X_2$. Entonces, existe $x_1\in X_1$ tal que:
        \begin{equation*}
            h(x_1)=x_2
        \end{equation*}
        luego, por ser biyección:
        \begin{equation*}
            h(\left\{x_1\right\})=\left\{x_2\right\}
        \end{equation*}
        donde $\left\{x_1\right\}$ es cerado en $(X_1,\tau_1)$. Como $h$ es cerrada entonces, $\left\{x_2\right\}$ es cerrado en $(X_2,\tau_2)$. Por tanto, todo conjunto unipuntual es cerrado en $(X_2,\tau_2)$, así $(X_2,\tau_2)$ es $T_1$. 
    \end{proof}

    \begin{propo}
        Sea $\left\{(X_\alpha,\tau_\alpha)\right\}_{\alpha\in I}$ una familia de espacios topológicos. Sea
        \begin{equation*}
            X=\prod_{\alpha\in I}X_\alpha
        \end{equation*}
        entonces, $(X,\tau_p)$ es $T_1$ si y sólo si $(X_\alpha,\tau_\alpha)$ es $T_1$, para todo $\alpha\in I$.
    \end{propo}

    \begin{proof}
        Se probarán las dos implicaciones.

        $\Rightarrow):$ Suponga que $(X,\tau_p)$ es $T_1$. Como la propiedad de ser un espacio $T_1$ es hereditaria y topológica, entonces al tenerse que $(X_\alpha,\tau_\alpha)$ es homeomorfo a un subespacio de $(X,\tau_p)$, tal subespacio es $T_1$ y la propiedad se conserva bajo homeomorfismos luego, se tiene que $(X_\alpha,\tau_\alpha)$ es $T_1$, para todo $\alpha\in I$.

        $\Leftarrow):$ Suponga que $(X_\alpha,\tau_\alpha)$ es $T_1$, para todo $\alpha\in I$. Sean $x=\left(x_\alpha\right)_{\alpha\in I},y=\left(y_\alpha\right)_{\alpha\in I}\in X$ con $x\neq y$. Por ser diferentes, existe $\alpha_0\in I$ tal que
        \begin{equation*}
            x_{\alpha_0}\neq y_{\alpha_0}
        \end{equation*}
        Como $(X_{\alpha_0},\tau_{\alpha_0})$ es $T_1$, existen $U,V\in\tau_{\alpha_0}$ tales que:
        \begin{equation*}
            x_{\alpha_0}\in U,y_{\alpha_0}\in V\quad x_{\alpha_0}\notin V,y_{\alpha_0}\notin U
        \end{equation*}
        tomemos $M=\prod_{\alpha\in I}M_\alpha$ y $N=\prod_{\alpha\in I}N_\alpha$, donde:
        \begin{equation*}
            M_\alpha=\left\{
                \begin{array}{lcr}
                    X_\alpha & \textup{ si } & \alpha\neq\alpha_0\\
                    U & \textup{ si } & \alpha=\alpha_0\\
                \end{array}
            \right.
        \end{equation*}
        y
        \begin{equation*}
            N_\alpha=\left\{
                \begin{array}{lcr}
                    X_\alpha & \textup{ si } & \alpha\neq\alpha_0\\
                    V & \textup{ si } & \alpha=\alpha_0\\
                \end{array}
            \right.
        \end{equation*}
        para todo $\alpha\in I$. Entonces, $x\in M$, $y\in N$ con $N,M\in\tau_p$, pero $x\notin N$, $y\notin M$.

        Por tanto, $(X,\tau_p)$ es $T_1$.
    \end{proof}

    \begin{propo}
        Sea $(X,\tau)$ un espacio topológico, y sea
        \begin{equation*}
            \Delta=\left\{(x,x)\Big|x\in X \right\}
        \end{equation*}
        entonces, $(X,\tau)$ es $T_2$ si y sólo si $\Delta$ es un subconjunto cerrado de $(X\times X,\tau_p)$ (da igual si es la topología producto o de caja ya que ambas coinciden).
    \end{propo}

    \begin{proof}
        Se probarán las dos implicaciones.

        $\Rightarrow):$ Suponga que $(X,\tau)$ es $T_2$. Veamos que $\Delta$ es cerrado en $(X\times X,\tau_p)$. Tomemos $(a,b)\in X\times X$ tal que $(a,b)\notin\Delta$, luego $a\neq b$. Como $(X,\tau)$ es $T_2$, existen dos abiertos $U,V\in\tau$ tales que:
        \begin{equation*}
            a\in U,b\in V\quad U\cap V=\emptyset
        \end{equation*}
        Sea $L=U\times V$. Se tiene que $(a,b)\in L$ y $L\in\tau_p$. Además, $\Delta\cap L=\emptyset$. En efecto, suponga que existe un elemento $(x,x)\in L$, entonces $x\in U$ y $x\in V$, luego $U\cap V\neq\emptyset$\contradiction. Por tanto, $\Delta\cap L=\emptyset$. Así, el conjunto $X\times X-\Delta$ es abierto por ser unión arbitraria de abiertos, luego $\Delta$ es cerrado en $(X\times X,\tau_p)$.

        $\Leftarrow):$ Suponga que $\Delta$ es cerrado en $(X\times X,\tau_p)$. Sean $x,y\in X$ con $x\neq y$. Entonces, $(x,y)\notin\Delta$, luego $(x,y)\in X\times X-\Delta$ el cual es abierto, luego existe un básico $B=U\times V$ tal que $(x,y)\in U\times V\subseteq X\times X-\Delta$, siendo $U,V\in\tau$.

        Por la parte anterior, se tiene que $U\cap V=\emptyset$. Por tanto:
        \begin{equation*}
            x\in U,y\in V\quad U\cap V=\emptyset
        \end{equation*}
        por ende, al ser los elementos diferentes $x,y\in X$ arbitrarios, se sigue que $(X,\tau)$ es $T_2$.
    \end{proof}

    \section{Espacios $T_3$}



\end{document}