\documentclass[12pt]{report}
\usepackage[spanish]{babel}
\usepackage[utf8]{inputenc}
\usepackage{amsmath}
\usepackage{amssymb}
\usepackage{amsthm}
\usepackage{graphics}
\usepackage{subfigure}
\usepackage{lipsum}
\usepackage{array}
\usepackage{multicol}
\usepackage{enumerate}
\usepackage[framemethod=TikZ]{mdframed}
\usepackage[a4paper, margin = 1.5cm]{geometry}

%En esta parte se hacen redefiniciones de algunos comandos para que resulte agradable el verlos%

\renewcommand{\theenumii}{\roman{enumii}}

\def\proof{\paragraph{Demostración:\\}}
\def\endproof{\hfill$\blacksquare$}

\def\sol{\paragraph{Solución:\\}}
\def\endsol{\hfill$\square$}

%En esta parte se definen los comandos a usar dentro del documento para enlistar%

\newtheoremstyle{largebreak}
  {}% use the default space above
  {}% use the default space below
  {\normalfont}% body font
  {}% indent (0pt)
  {\bfseries}% header font
  {}% punctuation
  {\newline}% break after header
  {}% header spec

\theoremstyle{largebreak}

\newmdtheoremenv[
    leftmargin=0em,
    rightmargin=0em,
    innertopmargin=-2pt,
    innerbottommargin=8pt,
    hidealllines = true,
    roundcorner = 5pt,
    backgroundcolor = gray!60!red!30
]{exa}{Ejemplo}[section]

\newmdtheoremenv[
    leftmargin=0em,
    rightmargin=0em,
    innertopmargin=-2pt,
    innerbottommargin=8pt,
    hidealllines = true,
    roundcorner = 5pt,
    backgroundcolor = gray!50!blue!30
]{obs}{Observación}[section]

\newmdtheoremenv[
    leftmargin=0em,
    rightmargin=0em,
    innertopmargin=-2pt,
    innerbottommargin=8pt,
    rightline = false,
    leftline = false
]{theor}{Teorema}[section]

\newmdtheoremenv[
    leftmargin=0em,
    rightmargin=0em,
    innertopmargin=-2pt,
    innerbottommargin=8pt,
    rightline = false,
    leftline = false
]{propo}{Proposición}[section]

\newmdtheoremenv[
    leftmargin=0em,
    rightmargin=0em,
    innertopmargin=-2pt,
    innerbottommargin=8pt,
    rightline = false,
    leftline = false
]{cor}{Corolario}[section]

\newmdtheoremenv[
    leftmargin=0em,
    rightmargin=0em,
    innertopmargin=-2pt,
    innerbottommargin=8pt,
    rightline = false,
    leftline = false
]{lema}{Lema}[section]

\newmdtheoremenv[
    leftmargin=0em,
    rightmargin=0em,
    innertopmargin=-2pt,
    innerbottommargin=8pt,
    roundcorner=5pt,
    backgroundcolor = gray!30,
    hidealllines = true
]{mydef}{Definición}[section]

\newmdtheoremenv[
    leftmargin=0em,
    rightmargin=0em,
    innertopmargin=-2pt,
    innerbottommargin=8pt,
    roundcorner=5pt
]{excer}{Ejercicio}[section]

%En esta parte se colocan comandos que definen la forma en la que se van a escribir ciertas funciones%

\newcommand\abs[1]{\ensuremath{\biglvert#1\bigrvert}}
\newcommand\divides{\ensuremath{\bigm|}}
\newcommand\cf[3]{\ensuremath{#1:#2\rightarrow#3}}
\newcommand\contradiction{\ensuremath{\#_c}}
\newcommand{\V}[1]{\ensuremath{\mathcal{V}(#1)}}
\newcommand{\Int}[1]{\ensuremath{\mathring{#1}}}
\newcommand{\Cls}[1]{\ensuremath{\overline{#1}}}
\newcommand{\Fr}[1]{\ensuremath{\textup{Fr}(#1)}}
\newcommand{\natint}[1]{\ensuremath{\left[\!\left[#1\right]\!\right]}}
\newcommand{\floor}[1]{\ensuremath{\lfloor#1\rfloor}}
\newcommand{\Card}[1]{\ensuremath{\textup{Card}\left(#1\right)}}
\newcommand{\Pot}[1]{\ensuremath{\mathcal{P}\left(#1\right)}}
\newcommand{\id}[1]{\ensuremath{\textup{id}_{#1}}}

%recuerda usar \clearpage para hacer un salto de página

\begin{document}
    \setlength{\parskip}{5pt} % Añade 5 puntos de espacio entre párrafos
    \setlength{\parindent}{12pt} % Pone la sangría como me gusta
    \title{Notas curso Topología I.
    
    Separabilidad, Filtros}
    \author{Cristo Daniel Alvarado}
    \maketitle

    \tableofcontents %Con este comando se genera el índice general del libro%

    \setcounter{chapter}{1} %En esta parte lo que se hace es cambiar la enumeración del capítulo%
    
    \chapter{Separabilidad}
    
    \section{Axiomas de separación}

    \begin{mydef}
        Sea $(X,\tau)$ un espacio topológico.
        \begin{enumerate}
            \item $(X,\tau)$ se dice un \textbf{espacio $T_0$} si dados $a,b\in X$ con $a\neq b$ existe un abierto que contiene a alguno de los dos puntos, pero no contiene al otro.
            \item $(X,\tau)$ se dice un \textbf{espacio $T_1$} si dados $a,b\in X$ con $a\neq b$ existen $U,V\subseteq X$ abiertos tales que $a\in U$, $b\in V$ y, $a\notin V$, $b\notin U$.
            \item $(X,\tau)$ se dice un \textbf{espacio $T_2$} si dados $a,b\in X$ con $a\neq b$ existen $U,V\subseteq X$ abiertos tales que $a\in U$, $b\in V$ y, $U\cap V=\emptyset$. Esto es equivalente a que el espacio sea de Hausdorff.
            \item $(X,\tau)$ se dice un \textbf{espacio $T_3$} si dados $p\in X$ y $A\subseteq X$ cerrado tal que $p\notin A$, existen $U,V\in\tau$ tales que $p\in U$, $A\subseteq V$ y $U\cap V=\emptyset$.
            \item $(X,\tau)$ se dice un \textbf{espacio $T_4$} si dados $A,B\subseteq X$ cerrados y disjuntos, existen $U,V\in\tau$ tales que $A\subseteq U$, $B\subseteq V$ y, $U\cap V=\emptyset$.
            \item $(X,\tau)$ se dice un \textbf{espacio regular} si es un espacio $T_3$ y $T_1$.
            \item $(X,\tau)$ se dice un \textbf{espacio normal} si es un espacio $T_4$ y $T_1$.
        \end{enumerate}
    \end{mydef}

    \begin{obs}
        Notemos que:
        \begin{equation*}
            T_2\Rightarrow T_1\Rightarrow T_0
        \end{equation*}
    \end{obs}

    \begin{exa}
        Considere al conjunto $X=\left\{1,2 \right\}$ y $\tau=\left\{X,\emptyset,\left\{1 \right\} \right\}$. Afirmamos que $(X,\tau)$ es $T_0$, pero no es $T_1$ y, por ende tampoco puede ser $T_2$.
    \end{exa}

    
    \begin{exa}
        Sea $(\mathbb{R},\tau_{cf})$. Afirmamos que $(\mathbb{R},\tau_u)$ es $T_1$. En efecto, sean $r,s\in\mathbb{R}$ tales que $r\neq s$. Los conjuntos $U=\mathbb{R}-\left\{s \right\},V=\mathbb{R}-\left\{r \right\}\in\tau_{cf}$ pues sus complementos son finitos, además:
        \begin{equation*}
            r\in U\quad\textup{y}\quad s\in V
        \end{equation*}
        además, $r\notin V$ y $s\notin U$. Por tanto, el espacio de $T_1$. Pero no es $T_2$.

        En efecto, suponga que existen $U,V\in\tau_{cf}$ abiertos tales que $\varphi=\frac{1+\sqrt{5}}{2}\in U$, $\frac{1}{\pi}\in V$ y $U\cap V=\emptyset$. En particular, se tiene que $\mathbb{R}-U$ y $\mathbb{R}-V$ son finitos. Por tanto:
        \begin{equation*}
            \begin{split}
                (\mathbb{R}-U)\cup(\mathbb{R}-V)&=\mathbb{R}-(U\cap V)\\
                &=\mathbb{R}\\
            \end{split}
        \end{equation*}
        es finito, por tanto, $\mathbb{R}$ es finito\contradiction.
    \end{exa}

    \begin{exa}
        Considere al espacio $(\mathbb{R},\tau_I=\left\{X,\emptyset \right\})$. Afirmamos que $(\mathbb{R},\tau_I)$ es $T_4$ y $T_3$, pero NO es $T_0$, pues si $\varphi=\frac{1+\sqrt{5}}{2},\frac{1}{\pi}\in\mathbb{R}$, solo hay un abierto que contiene a alguno de los dos puntos, el cual es $\mathbb{R}$, que siempre tiene a los dos puntos. Por ende, el espacio no es $T_0$.        
    \end{exa}

    \begin{propo}
        $T_4$ y $T_1\Rightarrow T_3$ y $T_1\Rightarrow T_2\Rightarrow T_1\Rightarrow T_0$.
    \end{propo}

    \begin{proof}
        La prueba se hará más adelante.
    \end{proof}

    \section{Espacios $T_1$}
    
    \begin{propo}
        Sea $(X,\tau)$ un espacio topológico. Entonces $(X,\tau)$ es un espacio $T_1$ si y sólo si todo subconjunto unitario de $X$ es cerrado.
    \end{propo}

    \begin{proof}
        Se probará la doble implicación.

        $\Rightarrow):$ Suponga que $(X,\tau)$ es $T_1$. Sea $x\in X$. Hay que probar que $X-\left\{x \right\}\in\tau$. En efecto, sea $y\in X-\left\{x \right\}$, entonces $x\neq y$. Como el espacio es $T_1$ existen un par de abiertos $U,V\in\tau$ tales que $x\in U$, $y\in V$ y $x\notin V$ y $y\notin U$.

        Como $y\in V$ y $x\notin V$, entonces $y\in V\subseteq X-\left\{x\right\}$. Luego $X-\left\{x\right\}$ es unión arbitraria de abiertos, luego es abierto. Por ende, $\left\{x\right\}$ es cerrado.

        $\Leftarrow):$ Suponga que todo subconjunto unitario de $X$ es cerrado. Sean $x,y\in X$ tales que $x\neq y$. Como $\left\{x\right\},\left\{y\right\}$ son cerrados, entonces $U=X=\left\{y\right\}$ y $V=X-\left\{x\right\}$ son abiertos y cumplen que:
        \begin{equation*}
            x\in U,y\in V\quad x\notin V,y\notin U
        \end{equation*}
        por tanto, como fueron arbitrarios los dos elementos $x,y\in X$ distintos, se sigue que $(X,\tau)$ es $T_1$.
    \end{proof}

    \begin{cor}
        Sea $(X,\tau)$ un espacio topológico. $(X,\tau)$ es $T_1$ si y sólo si todo subconjunto finito de $X$ es cerrado.
    \end{cor}

    \begin{proof}
        Es inmediata de la proposición anterior.
    \end{proof}

    \begin{cor}
        Sea $X$ un conjunto finito y $\tau$ una topología definida sobre $X$. $(X,\tau)$ es $T_1$ si y sólo $\tau=\tau_D$.
    \end{cor}

    \begin{proof}
        Es inmediata de la proposición anterior.
    \end{proof}

    \begin{propo}
        Sea $(X,\tau)$ un espacio topológico. Entonces, $(X,\tau)$ es $T_1$ si y sólo si $\tau_{cf}\subseteq\tau$.
    \end{propo}

    \begin{proof}
        Se probarán las dos implicaciones.

        $\Rightarrow):$ Sea $A\in\tau_{cf}$ con $A\neq\emptyset$, luego $X-A$ es un conjunto finito. Como $(X,\tau)$ es $T_1$, entonces $X-A$ es cerrado (debe serlo por ser finito), luego $A$ es abierto, es decir $A\in\tau$.
        
        $\Leftarrow):$ Supongamos que $\tau_{cf}\subseteq\tau$. Sean $x\in X$. El conjunto $X-\left\{x\right\}$ es finito, luego $X-\left\{x\right\}\in\tau$, por ende el conjunto $\left\{x\right\}$ es cerrado. Como el $x$ fue arbitrario, se sigue que todo conjunto unipuntual es cerrado luego, por una proposición anterior, se sigue que $(X,\tau)$ es $T_1$.
    \end{proof}

    \begin{cor}
        La topología $\tau_{cf}$ es la topología más gruesa (o menos fina) que podemos definir sobre un conjunto para que el espacio topológico $(X,\tau_{cf})$ sea $T_1$.
    \end{cor}

    \begin{proof}
        Es inmediata de la proposición anterior.
    \end{proof}

    \begin{propo}
        La propiedad de ser un espacio topológico $T_1$ es hereditaria. 
    \end{propo}

    \begin{proof}
        Sea $(X,\tau)$ un espacio topológico $T_1$ y, tomemos $Y\subseteq X$. Formemos así al espacio $(Y,\tau_Y)$, queremos ver que este espacio es $T_1$. En efecto, sea $y\in Y$, entonces:
        \begin{equation*}
            \left\{y\right\}=\left\{y \right\}\cap Y
        \end{equation*}
        luego, $\left\{y\right\}\subseteq Y$ es un conjunto cerrado en $(Y,\tau_Y)$, ya que $\left\{y\right\}\subseteq X$ es un conjunto cerrado en $(X,\tau)$. Por ende, todo conjunto unipuntual es cerrado en $(Y,\tau_Y)$, luego este subespacio es $T_1$.
    \end{proof}

    \begin{propo}
        La propiedad de ser un espacio topológico $T_1$ es topológica.
    \end{propo}

    \begin{proof}
        Sean $(X_1,\tau_1)$ y $(X_2,\tau_2)$ espacios topológicos homeomorfos y, suponga que $(X_1,\tau_1)$ es un espacio $T_1$. Sea $\cf{h}{(X_1,\tau_1)}{(X_2,\tau_2)}$ el homeomorfismo entre estos dos espacios. Como esta función es homeomorfismo, es una biyección cerrada y continua. Sea $x_2\in X_2$. Entonces, existe $x_1\in X_1$ tal que:
        \begin{equation*}
            h(x_1)=x_2
        \end{equation*}
        luego, por ser biyección:
        \begin{equation*}
            h(\left\{x_1\right\})=\left\{x_2\right\}
        \end{equation*}
        donde $\left\{x_1\right\}$ es cerado en $(X_1,\tau_1)$. Como $h$ es cerrada entonces, $\left\{x_2\right\}$ es cerrado en $(X_2,\tau_2)$. Por tanto, todo conjunto unipuntual es cerrado en $(X_2,\tau_2)$, así $(X_2,\tau_2)$ es $T_1$. 
    \end{proof}

    \begin{propo}
        Sea $\left\{(X_\alpha,\tau_\alpha)\right\}_{\alpha\in I}$ una familia de espacios topológicos. Sea
        \begin{equation*}
            X=\prod_{\alpha\in I}X_\alpha
        \end{equation*}
        entonces, $(X,\tau_p)$ es $T_1$ si y sólo si $(X_\alpha,\tau_\alpha)$ es $T_1$, para todo $\alpha\in I$.
    \end{propo}

    \begin{proof}
        Se probarán las dos implicaciones.

        $\Rightarrow):$ Suponga que $(X,\tau_p)$ es $T_1$. Como la propiedad de ser un espacio $T_1$ es hereditaria y topológica, entonces al tenerse que $(X_\alpha,\tau_\alpha)$ es homeomorfo a un subespacio de $(X,\tau_p)$, tal subespacio es $T_1$ y la propiedad se conserva bajo homeomorfismos luego, se tiene que $(X_\alpha,\tau_\alpha)$ es $T_1$, para todo $\alpha\in I$.

        $\Leftarrow):$ Suponga que $(X_\alpha,\tau_\alpha)$ es $T_1$, para todo $\alpha\in I$. Sean $x=\left(x_\alpha\right)_{\alpha\in I},y=\left(y_\alpha\right)_{\alpha\in I}\in X$ con $x\neq y$. Por ser diferentes, existe $\alpha_0\in I$ tal que
        \begin{equation*}
            x_{\alpha_0}\neq y_{\alpha_0}
        \end{equation*}
        Como $(X_{\alpha_0},\tau_{\alpha_0})$ es $T_1$, existen $U,V\in\tau_{\alpha_0}$ tales que:
        \begin{equation*}
            x_{\alpha_0}\in U,y_{\alpha_0}\in V\quad x_{\alpha_0}\notin V,y_{\alpha_0}\notin U
        \end{equation*}
        tomemos $M=\prod_{\alpha\in I}M_\alpha$ y $N=\prod_{\alpha\in I}N_\alpha$, donde:
        \begin{equation*}
            M_\alpha=\left\{
                \begin{array}{lcr}
                    X_\alpha & \textup{ si } & \alpha\neq\alpha_0\\
                    U & \textup{ si } & \alpha=\alpha_0\\
                \end{array}
            \right.
        \end{equation*}
        y
        \begin{equation*}
            N_\alpha=\left\{
                \begin{array}{lcr}
                    X_\alpha & \textup{ si } & \alpha\neq\alpha_0\\
                    V & \textup{ si } & \alpha=\alpha_0\\
                \end{array}
            \right.
        \end{equation*}
        para todo $\alpha\in I$. Entonces, $x\in M$, $y\in N$ con $N,M\in\tau_p$, pero $x\notin N$, $y\notin M$.

        Por tanto, $(X,\tau_p)$ es $T_1$.
    \end{proof}

    \begin{propo}
        Sea $(X,\tau)$ un espacio topológico, y sea
        \begin{equation*}
            \Delta=\left\{(x,x)\Big|x\in X \right\}
        \end{equation*}
        entonces, $(X,\tau)$ es $T_2$ si y sólo si $\Delta$ es un subconjunto cerrado de $(X\times X,\tau_p)$ (da igual si es la topología producto o de caja ya que ambas coinciden).
    \end{propo}

    \begin{proof}
        Se probarán las dos implicaciones.

        $\Rightarrow):$ Suponga que $(X,\tau)$ es $T_2$. Veamos que $\Delta$ es cerrado en $(X\times X,\tau_p)$. Tomemos $(a,b)\in X\times X$ tal que $(a,b)\notin\Delta$, luego $a\neq b$. Como $(X,\tau)$ es $T_2$, existen dos abiertos $U,V\in\tau$ tales que:
        \begin{equation*}
            a\in U,b\in V\quad U\cap V=\emptyset
        \end{equation*}
        Sea $L=U\times V$. Se tiene que $(a,b)\in L$ y $L\in\tau_p$. Además, $\Delta\cap L=\emptyset$. En efecto, suponga que existe un elemento $(x,x)\in L$, entonces $x\in U$ y $x\in V$, luego $U\cap V\neq\emptyset$\contradiction. Por tanto, $\Delta\cap L=\emptyset$. Así, el conjunto $X\times X-\Delta$ es abierto por ser unión arbitraria de abiertos, luego $\Delta$ es cerrado en $(X\times X,\tau_p)$.

        $\Leftarrow):$ Suponga que $\Delta$ es cerrado en $(X\times X,\tau_p)$. Sean $x,y\in X$ con $x\neq y$. Entonces, $(x,y)\notin\Delta$, luego $(x,y)\in X\times X-\Delta$ el cual es abierto, luego existe un básico $B=U\times V$ tal que $(x,y)\in U\times V\subseteq X\times X-\Delta$, siendo $U,V\in\tau$.

        Por la parte anterior, se tiene que $U\cap V=\emptyset$. Por tanto:
        \begin{equation*}
            x\in U,y\in V\quad U\cap V=\emptyset
        \end{equation*}
        por ende, al ser los elementos diferentes $x,y\in X$ arbitrarios, se sigue que $(X,\tau)$ es $T_2$.
    \end{proof}

    \section{Espacios $T_3$}

    \begin{propo}
        Sea $(X,\tau)$ un espacio topológico. Entonces, el espacio es $T_3$ si y sólo si dado $x\in X$ y $U\in\tau$ tal que $x\in U$ existe $V\in\tau$ tal que $x\in V$ y $\Cls{V}\subseteq U$.
    \end{propo}

    \begin{proof}
        $\Rightarrow):$ Suponga que $(X,\tau)$ es $T_3$. Sea $x\in X$ y $U\in\tau$ tal que $x\in U$, luego $x\notin X-U$, el cual es cerrado, luego por ser el espacio $T_3$ existen $W,V\in\tau$ abiertos disjuntos tales que:
        \begin{equation*}
            x\in V\quad\textup{ y }\quad X-U\subseteq W
        \end{equation*}
        es claro que $V\subseteq U$ (pues, $W\subseteq X-U$ y $W\cap V=\emptyset$). Veamos que $\Cls{V}\subseteq U$. En efecto, supongamos que $y\in\Cls{V}$ y $y\notin U$, entonces $y\in W$, luego el conjunto $W\cap V\neq\emptyset$\contradiction. Por ende, $\Cls{V}\subseteq U$.

        $\Leftarrow):$ Sea $x\in X$ y $F\subseteq X$ cerrado tal que $x\notin F$, entonces $x\in X-F$ el cual es abierto. Luego por hipótesis existe un cerrado $\Cls{V}$ tal que $x\in V\subseteq\Cls{V}\subseteq X-F$.

        Luego, $F\subseteq X-\Cls{V}$. Tomemos $W=X-\Cls{V}$. Entonces, $V$ y $W$ son abiertos tales que $x\in V$, $F\subseteq W$ y, $W\cap V=\emptyset$. Por tanto, $(X,\tau)$ es $T_3$.
    \end{proof}

    \begin{exa}
        Considere el espacio topológico $(X=\left\{1,2 \right\},\tau_I)$. Este espacio es $T_3$ pero no es $T_0$.
    \end{exa}

    \begin{exa}
        Sea $K=\left\{\frac{1}{n}\Big|n\in\mathbb{N}\right\}$, tomemos $\mathcal{B}$ la colección de subconjuntos de $\mathbb{R}$ formada por los siguientes conjuntos:
        \begin{enumerate}
            \item Todos los intervalos abiertos $(a,b)$.
            \item Todos los conjuntos de la forma $(a,b)-K$.
        \end{enumerate}
        Tenemos que $\mathcal{B}$ es una base para una topología sobre $\mathbb{R}$.

        Sea $\tau_K$ la topología generada por la colección $\mathcal{B}$. Tenemos que $\tau_u\subseteq \tau_K$. Por ende, como $(\mathbb{R},\tau_u)$ es $T_2$, se sigue que $(\mathbb{R},\tau_K)$ también lo es.

        Sean $l\notin\mathbb{R}-K$ y $L=(l-1,l+1)-K$. Tenemos que $l\in L$. El conjunto $L$ es un básico y, además, $L\subseteq\mathbb{R}-\mathbb{K}$. Por tanto, $\mathbb{R}-K\in\tau_K$, luego $K$ es un conjunto cerrado en $(\mathbb{R},\tau_K)$.

        Tenemos que $0\notin K$. Suponga que $U,V\in\tau$ son abiertos tales que $0\in U$, $K\subseteq V$ y $U\cap V=\emptyset$. Como $0\in U$. Sea $B\in\mathcal{B}$ un básico tal que $x\in B\subseteq V$. Tenemos que, dado un intervalo abierto que contenga al $0$, este siempre contiene puntos de $K$, luego $B$ debe ser de la forma $B=(a,b)-K$.

        Sea $m\in\mathbb{N}$ tal que $\frac{1}{m}\in(a,b)$. Se tiene que $\frac{1}{m}\in K\subseteq V$, luego existe un básico $(c,d)$ (debe ser de esta forma) tal que $\frac{1}{m}\in(c,d)\subseteq V$. Ahora, podemos suponer que $a<0<c<d<b$.
        Sea $\zeta\mathbb{R}$ tal que $\zeta<\frac{1}{m}$ y $\max\left\{c,\frac{1}{m+1} \right\}<\zeta$, luego:
        \begin{equation*}
            c<\zeta<\frac{1}{m}
        \end{equation*}
        entonces, en particular, $\zeta\in(c,d)$, $\zeta\notin K$ ya que $\frac{1}{m+1}<\zeta<\frac{1}{m}$ y $\zeta\in(a,b)$. Por tanto, $\zeta\in U\cap V$\contradiction. Así, $(\mathbb{R},\tau_K)$ no es $T_3$.
    \end{exa}

    \begin{propo}
        La propiedad de ser $T_3$ cumple:
        \begin{enumerate}
            \item Se hereda.
            \item Es topológica.
        \end{enumerate}
    \end{propo}

    \begin{proof}
        De (1): Sea $(X,\tau)$ un espacio topológico $T_3$ y sea $Y\subseteq X$. Probaremos que $(Y,\tau_Y)$ es $T_3$. Tomemos $A\subseteq Y$ cerrado con la topología $\tau_Y$ y $p\in Y-A$.

        Como $A$ es cerrado en el subespacio, existe $C\subseteq X$ cerrado en $(X,\tau)$ tal que:
        \begin{equation*}
            A=Y\cap C
        \end{equation*}
        En particular, $A\subseteq C$, es decir que $Y-C\subseteq Y-A$, luego $p\notin C$. Como $(X,\tau)$ es $T_3$, existen $U,V\in\tau$ disjuntos tales que:
        \begin{equation*}
            p\in V\quad\textup{y}\quad C\subseteq U
        \end{equation*}
        luego, los conjuntos $Y\cap U,Y\cap V\in \tau_Y$ son tales que:
        \begin{equation*}
            p\in Y\cap V\quad\textup{y}\quad A=Y\cap C\subseteq Y\cap U
        \end{equation*}
        siendo estos disjuntos (pues $U$ y $V$ lo son). Por tanto, $(Y,\tau_Y)$ es $T_3$.

        De (2): Sean $(X,\tau)$ y $(Y,\sigma)$ espacios topológicos homeomorfos, y $\cf{f}{(X,\tau)}{(Y,\sigma)}$ el homeomorfismo entre ambos.
    
        Suponga que $(X,\tau)$ es $T_3$. Probaremos que $(Y,\sigma)$ también es $T_3$. En efecto, sean $p\in Y$ y $F\subseteq Y$ cerrado tales que $p\notin F$, es decir que $p\in Y-F$.
        Sea
        \begin{equation*}
            F'=f^{-1}(F)
        \end{equation*}
        y $p'=f^{-1}(p)$. Por ser homeomorfismo, se tiene que $F'$ es cerrado en $(X,\tau)$ y, por ser inyectiva se tiene que $p'\notin F'$. Luego, como $(X,\tau)$ es $T_3$ existen $U',V'\in\tau$ disjuntos tales que:
        \begin{equation*}
            p'\in V'\quad\textup{y}\quad F'\subseteq U'
        \end{equation*}
        Sean $U=f(U')$ y $V=f(V')$, los cuales son abiertos en $(Y,\sigma)$ tales que:
        \begin{equation*}
            p=f(p')\in V\quad\textup{y}\quad F=f(F')\subseteq U
        \end{equation*}
        siendo $U,V$ disjuntos por serlo $U',V'$. Luego, $(Y,\sigma)$ es $T_3$.
    \end{proof}

    \begin{propo}
        Sea $\left\{(X_\alpha,\tau_\alpha) \right\}_{\alpha\in I}$ una familia de espacios topológicos, sea
        \begin{equation*}
            X=\prod_{\alpha\in I}X_\alpha
        \end{equation*}
        entonces, $(X,\tau_p)$ es $T_3$ si y sólo si $(X_\alpha,\tau_\alpha)$ es $T_3$, para todo $\alpha\in I$.
    \end{propo}

    \begin{proof}
        $\Rightarrow):$ Es inmediata del hecho de que la propiedad de que un espacio sea $T_3$ es hereditaria y topológica.

        $\Leftarrow):$ Suponga que para todo $\alpha\in I$, $(X_\alpha,\tau_\alpha)$ es $T_3$. Veamos que $(X,\tau_p)$ es $T_3$. Sea $x\in X$ y $U\in\tau_p$ un abierto tal que $x\in U$.

        Como $U\in\tau_p$, podemos encontrar un básico $B$, que podemos expresar como $B=\prod_{\alpha\in I}B_\alpha$, donde $B_\alpha=X_\alpha$ para casi todo salvo una cantidad finita de $\alpha\in I$, y $B_\alpha$ es abierto en $(X_\alpha,\tau_\alpha)$ para todo $\alpha\in I$.

        Como cada $(X_\alpha,\tau_\alpha)$ es $T_3$, entonces para cada $B_\alpha$ existe $V_\alpha\in\tau_\alpha$ tal que $x_\alpha\in V_\alpha$ y $\Cls{V_\alpha}\subseteq B_\alpha$, para todo $\alpha\in I$.

        Si $B_\alpha=X_\alpha$, tomemos $V_\alpha=X_\alpha$, en caso contrario lo dejamos igual. Entonces, el conjunto $V=\prod_{\alpha\in I}V_\alpha$ es un básico, en particular, abierto, tal que $x\in V$, y
        \begin{equation*}
            \Cls{V}=\Cls{\prod_{\alpha\in I}V_\alpha}=\prod_{\alpha\in I}\Cls{V_\alpha}\subseteq\prod_{\alpha\in I}B_\alpha=B\subseteq U
        \end{equation*}
        por tanto, $(X,\tau_p)$ es $T_3$.
    \end{proof}

    \begin{cor}
        Sea $(X,\tau)$ un espacio topológico.
        \begin{enumerate}
            \item Si $(X,\tau)$ es regular, entonces y $Y\subseteq X$, entonces $(Y,\tau_Y)$ es regular.
            \item Si $(X,\tau)$ y $(X',\tau')$ son espacios homeomorfos y, $(X,\tau)$ es regular, entonces $(X',\tau')$ es regular.
            \item Si $\left\{(X_\alpha,\tau_\alpha) \right\}_{\alpha\in I}$ es una familia de espacios topológicos. Si $X=\prod_{\alpha\in I}$, entonces $(X,\tau_p)$ es regular si y sólo si $(X_\alpha,\tau_\alpha)$ es regular, para todo $\alpha\in I$.
        \end{enumerate}
    \end{cor}

    \begin{proof}
        Son inmediatas del hecho que la propiedad de ser $T_1$ y $T_3$ se hereda y es topológicsa y, de que esta propiedad se preserva bajo productos y elementos del producto.
    \end{proof}

    \section{Espacios $T_4$}

    \begin{propo}
        Sea $(X.\tau)$ un espacio topológico. Entonces, $(X,\tau)$ es $T_4$ si y sólo si dados $A\subseteq X$ cerrado y $U\in\tau$ tales que $A\subseteq U$, existe un abierto $V$ tal que $A\subseteq V$ y $\Cls{V}\subseteq U$. 
    \end{propo}

    \begin{proof}
        $\Rightarrow):$ Supongamos que $(X,\tau)$ es $T_4$. Sean $A\subseteq X$ cerrado y $U\in\tau$ tal que $A\subseteq U$. El conjunto $B=X-U$ es un cerrado tal que $A\cap B=\emptyset$. Como el espacio $(X,\tau)$ es $T_4$, existen dos abiertos $V,W\in\tau$ tales que:
        \begin{equation*}
            A\subseteq V\quad\textup{y}\quad B\subseteq W
        \end{equation*}
        y, $V\cap W=\emptyset$. Como $V\cap W=\emptyset$, entonces $V\subseteq X-W\subseteq X-B= U$. Afirmamos que $\Cls{V}\subseteq U$. En efecto, notemos que $X-W$ es un cerrado que contiene a $V$, por ende $\Cls{V}\subseteq X-W\subseteq U$, luego $\Cls{V}\subseteq U$. Con lo cual se sigue el resultado.
        
        $\Leftarrow):$ Sean $A,B\subseteq X$ cerrados tales que $A\cap B=\emptyset$. Se tiene entonces que:
        \begin{equation*}
            A\subseteq X-B
        \end{equation*}
        donde $X-B\in\tau$, luego por hipótesis existe $U\in\tau$ abierto tal que:
        \begin{equation*}
            \begin{split}
                A\subseteq U\subseteq\Cls{U}\subseteq X-B
            \end{split}
        \end{equation*}
        el conjunto $V=X-\Cls{U}$ es un abierto para el cual, se tiene que $B\subseteq V$. Luego, $U,V\in\tau$ son tales que $A\subseteq U$, $B\subseteq V$ y $U\cap V=\emptyset$. Luego el espacio es $T_4$.
    \end{proof}

    \begin{propo}
        Sea $(X,\tau)$ un espacio $T_4$ y sea $A\subseteq X$ un conjunto cerrado. Entonces, $(A,\tau_A)$ es $T_4$.
    \end{propo}

    \begin{proof}
        Sean $M,N\subseteq(A,\tau_A)$ cerrados tales que $M\cap N=\emptyset$. Como $A$ es cerrado en $(X,\tau)$, entonces $M,N$ son cerrados en $(X,\tau)$. Luego, como $(X,\tau)$ es $T_4$, existen dos abiertos $U',V'\in\tau$ tales que
        \begin{equation*}
            M\subseteq U',\quad N\subseteq V',\quad U'\cap V'=\emptyset
        \end{equation*}
        Luego, los conjuntos $U=A\cap U',V=A\cap V'\in\tau_A$ son disjuntos tales que $M\subseteq U$ y $N\subseteq V$, ya que $M,N\subseteq A$. Así, $(A,\tau_A)$ es $T_4$.
    \end{proof}

    \begin{lema}
        Sean $(X_1,\tau_1)$ y $(X_2,\tau_2)$ dos espacios topológicos homeomorfos. Entonces, si $\cf{f}{(X_1,\tau_1)}{(X_2,\tau_2)}$ es el homeomorfismo entre ambos espacios, se tiene que $f(\Cls{A})=f(\Cls{A})$, para todo $A\subseteq X_1$.
    \end{lema}

    \begin{proof}
        Como $f$ es homeomorfismo, en particular es continua
    \end{proof}

    \begin{propo}
        La propiedad de ser $T_4$ es topológica.
    \end{propo}

    \begin{proof}
        Sean $(X_1,\tau_1)$ y $(X_2,\tau_2)$ espacios topológicos homeomorfos tales que $(X_1,\tau_1)$ es $T_4$. Sea $\cf{f}{(X_1,\tau_1)}{(X_2,\tau_2)}$ el homeomorfismo entre ellos.

        Veamos que $(X_2,\tau_2)$ es $T_4$. En efecto, sea $A\subseteq X_2$ cerrado y $U\in\tau_2$ abierto tal que $A\subseteq U$. Como $f$ es homeomorfismo, entonces
        $f^{-1}(A)\subseteq X_1$ es cerrado y, $f^{-1}(U)\in\tau_1$ son tales que:
        \begin{equation*}
            f^{-1}(A)\subseteq f^{-1}(U)
        \end{equation*}
        Luego, como $(X_1,\tau_1)$ es $T_4$, existe $W\in\tau_1$ tal que:
        \begin{equation*}
            f^{-1}(A)\subseteq W\subseteq\Cls{W}\subseteq f^{-1}(U)
        \end{equation*}
        Sea $V=f(W)$. Como $f$ es homeomorfismo, es una función abierta, luego $V\in\tau_2$, para la cual se cumple que:
        \begin{equation*}
            A\subseteq V\subseteq U
        \end{equation*}
        pero, $f(\Cls{V})=\Cls{f(V)}$ (por ser $f$ homeomorfismo), se tiene que:
        \begin{equation*}
            A\subseteq V\subseteq \Cls{V}\subseteq U
        \end{equation*}
        por tanto, $(X_2,\tau_2)$ es $T_4$.
    \end{proof}

    \begin{lema}[\textbf{Lema de Urysohn}]
        Sea $(X,\tau)$ un espacio topológico. Entonces, $(X,\tau)$ es $T_4$ si y sólo si para todos $A,B\subseteq X$ cerrados disjuntos, existe una función continua $\cf{f}{(X,\tau)}{([0,1],\tau_u)}$ tal que $f(A)=\left\{1\right\}$ y $f(B)=\left\{0\right\}$.
    \end{lema}

    \begin{proof}
        $\Rightarrow):$ Para probar el resultado, debemos hacer varias cosas antes:
        \begin{enumerate}
            \item Sea
            \begin{equation*}
                P=\mathbb{Q}\cap[0,1]
            \end{equation*}
            Nuestro objetivo es que para cada $p\in P$ le asignemos un conjunto abierto $U_p\subseteq X$ tal que si $p,q\in P$ son tales que
            \begin{equation*}
                p<q\Rightarrow \Cls{U}_p \subseteq U_q
            \end{equation*}
            de esta forma, la familia $\left\{U_p\Big|p\in P \right\}$ estará simplemente ordenada de la misma forma en la que sus subíndices lo están en $P$. Como el conjunto $P$ es numerable, podemos usar inducción para definir cada uno de los $U_p$. Ordenemos los elementos de $P$ en una sucesión de tal forma que los números $0$ y $1$ son los primeros de la sucesión (denotada de ahora en adelante por $\left\{p_n \right\}_{n=1}^\infty$).

            Definiremos ahora los conjuntos $U_p$ como sigue: defina
            \begin{equation*}
                U_1=X-B
            \end{equation*}
            Como $A$ es un cerrado contenido en $U_1$, por ser $(X,\tau)$ $T_4$, se tiene que existe un conjunto abierto $U_0\subseteq X$ tal que
            \begin{equation*}
                A\subseteq U_0\quad\textup{y}\quad\Cls{U}_0\subseteq U_1
            \end{equation*}
            En general, sea $P_n$ el conjunto de los primeros $n$ números racionales en la sucesión de los elementos de $P$. Suponga que $U_p$ está definido para cada $p\in P_n$ y, satisface la condición:
            \begin{equation*}
                p,q\in P_n\textup{ tal que }p<q\Rightarrow \Cls{U}_p\subseteq U_q
            \end{equation*}
            Sea $r$ el siguiente número racional en la sucesión $\left\{p_n \right\}_{n=1}^\infty$, esto es $r=p_{ n+1}$. Definiremos $U_r$.

            Considere el conjunto
            \begin{equation*}
                P_{ n+1}=P_n\cup\left\{r \right\}
            \end{equation*}
            Este es un subconjunto finito del intervalo $[0,1]$ y, tiene un orden simple derivado del orden simple $<$ de $[0,1]$.

            En un conjunto finito simplemente ordenado, todo elemento tiene un predecesor inmediato y un sucesor inmediato. El número $0$ es el elemento más pequeño y, $1$ es el elemento más grande de $P_{n+1}$ y, $r$ no es $0$ o $1$. Por tanto, $r$ tiene un sucesor y un predecesor inmediato, denotados respectivamente por $q$ y $p$. Los conjuntos $U_p$ y $U_q$ están definidos y son tlaes que
            \begin{equation*}
                \Cls{U}_p\subseteq U_q
            \end{equation*}
            por hipótesis de inducción. Como $(X,\tau)$ es $T_4$, entonces existe un conjunto abierto $U_r\subseteq X$ tal que
            \begin{equation*}
                \Cls{U}_p\subseteq U_r\quad\textup{y}\quad\Cls{U}_r\subseteq U_q
            \end{equation*}
            Es claro (pues los conjuntos $U_p$ con $p\in P_n$ están ordenados por la contención), que
            \begin{equation*}
                p,q\in P_{ n+1}\textup{ tal que }p<q\Rightarrow \Cls{U}_p\subseteq U_q
            \end{equation*}
            Usando inducción, tenemos definidos los conjuntos $U_p$, para todo $p\in P$.

            \item Ahora que se tiene definido $U_p$ para todo número en $\mathbb{Q}\cap[0,1]$, extenderemos esta definición a todo $\mathbb{Q}$, haciendo
            \begin{equation*}
                \begin{split}
                    U_p&=\emptyset,\quad p<0\\
                    U_p&=X,\quad 1<p\\
                \end{split}
            \end{equation*}
            para todo $p\in\mathbb{Q}$. Se sigue cumpliendo que para todo $p,q\in\mathbb{Q}$
            \begin{equation*}
                p<q\Rightarrow \Cls{U}_p\subseteq U_q
            \end{equation*}
            \item Dado un punto $p\in X$, definamos el conjunto $\mathbb{Q}(x)$ como el conjunto de todos los números racionales $p\in\mathbb{Q}$ tales que los correspondientes $U_p$ contengan a $x$, es decir:
            \begin{equation*}
                \mathbb{Q}(x)=\left\{p\in\mathbb{Q}\Big|x\in U_p \right\}
            \end{equation*}
            Este conjunto no contiene a ningún número menor que $0$ ya que $x\notin U_p$ para todo $p\in\mathbb{Q}^-$, además, contiene a todo número mayor que $1$, pues $x\in U_p$ para todo $p\in\mathbb{Q}$, $p>1$. Por tanto, $\mathbb{Q}(x)$ es acotado inferiormente y no vacío, luego tiene ínfimo en el intervalo $[0,1]$. Defina
            \begin{equation*}
                f(x)=\inf\mathbb{Q}(x)=\inf\left\{p\in\mathbb{Q} \Big| x\in U_p \right\}
            \end{equation*}
            \item Afirmamos que $f$ es la función deseada. Si $x\in A$, entonces $x\in U_p$ para todo $p\in\mathbb{Q}_{\geq0}$, luego
            \begin{equation*}
                f(x)=\inf\mathbb{Q}(x)=0
            \end{equation*}
            Similarmente, si $x\in B$, entonces $x\notin U_p$ para todo $p\in\mathbb{Q}$ con $p\leq 1$. Luego, $\mathbb{Q}(x)$ consiste de todos los números racionales mayores a $1$ y, por ende, $f(x)=1$.

            Probaremos que $f$ es continua. Para ello, probaremos que se cumplen dos cosas:
            \begin{enumerate}
                \item $x\in\Cls{U}_r$ implica que $f(x)\leq r$.
                \item $x\notin U_r$ implica que $f(x)\geq r$.
            \end{enumerate}
            Para probar (1), notemos que si $x\in\Cls{U}_r$, entonces $x\in U_s$, para todo $s>r$. Entonces, $\mathbb{Q}(x)$ contiene a todos los números racionales mayores que $r$, así que, por definición tenemos que
            \begin{equation*}
                f(x)=\inf\mathbb{Q}(x)\leq r
            \end{equation*}

            Para probar (2), notemos que si $x\notin U_r$, entonces $x$ no está en $U_s$ para todo $s<r$. Por tanto, $\mathbb{Q}(x)$ no contiene números racionales menores que $r$, por lo cual
            \begin{equation*}
                f(x)=\inf\mathbb{Q}(x)\geq r
            \end{equation*}
            Ahora probaremos la continuidad de $f$. Sea $x_0\in X$ y un intervalo abierto $(c,d)$ en $\mathbb{R}$ tal que
            \begin{equation*}
                c<f(x_0)<d
            \end{equation*}
            podemos encontrar números racionales $p,q\in\mathbb{Q}$ tales que
            \begin{equation*}
                c<p<f(x_0)<q<d
            \end{equation*}
            Afirmamos que el conjunto
            \begin{equation*}
                U=U_q-\Cls{U}_p
            \end{equation*}
            es un abierto que cumple que $f(U)\subseteq(c,d)$ y es tal que $x_0\in U$. En efecto, notemos que $x_0\in U_q$ pues $f(x_0)<q$ implica por (2) que $f(x_0)\in U_q$ y, como $p<f(x_0)$, implica por (1) que $f(x_0)\notin \Cls{U}_p$. Por tanto, $f(x_0)\in U$.

            Sea $x\in U$, entonces $x\in U_q\subseteq \Cls{U}_q$, por lo cual de (1), $f(x)\leq q$ y, $x\notin \Cls{U}_p$ implica que $x\notin\Cls{U}_p$ por lo cual de (2) se sigue que $p\leq f(x)$. Por tanto, $f(x)\in[p,q]\subseteq (c,d)$.

            Luego, $f(U)\subseteq(c,d)$. Así, $f$ es continua en $x_0\in X$. Como el punto fue arbitrario, se sigue que $f$ es continua en $X$.
        \end{enumerate}
        Por los 4 incisos anteriores, se sigue el resultado.

        $\Leftarrow):$ Sean $A,B\subseteq X$ cerrados disjuntos. Por hipótesis existe una función continua $\cf{f}{(X,\tau)}{([0,1],\tau_u)}$ tal que $f(A)=1$ y $f(B)=0$. Los conjuntos $U=f^{-1}((r,1])$ $V=f^{-1}([0,r))$, donde $r\in(0,1)$, son dos abiertos (ya que $f$ es continua y $[0,r),(r,1],\in\tau_u$) tales que:
        \begin{equation*}
            A\subseteq U\quad B\subseteq V
        \end{equation*}
        y, $U\cap V=\emptyset$.
    \end{proof}

    \chapter{Filtros}

    \section{Conceptos Fundamentales}

    \begin{mydef}
        Sean $X$ un conjunto no vacío y $\mathcal{F}$ una familia no vacía de subconjuntos no vacíos de $X$. $\mathcal{F}$ se dice que es un \textbf{filtro} si cumple lo siguiente:
        \begin{enumerate}
            \item $\emptyset\notin\mathcal{F}$.
            \item Si $F_1,F_2\in\mathcal{F}$, entonces $F_1\cap F_2\in\mathcal{F}$.
            \item Si $K\subseteq X$ y $F\subseteq K$ para algún $F\in\mathcal{F}$, entonces $K\subseteq F$. (\textit{Propiedad de absorción}).
        \end{enumerate}
    \end{mydef}

    \begin{exa}
        Sea $X$ un conjunto no vacío. Entonces, $\left\{X\right\}$ es un filtro sobre $X$.
    \end{exa}

    \begin{obs}
        Si $\mathcal{F}$ es un filtro sobre un conjunto no vacío $X$ entonces, se cumple lo siguiente:
        \begin{enumerate}
            \item $\forall A,B\in\mathcal{F}$, $A\cap B\neq\emptyset$.
            \item Si $A_1,...,A_n\in\mathcal{F}$, entonces $\bigcap_{i=1}^nA_i\in\mathcal{F}$ y es no vacía.
        \end{enumerate}
    \end{obs}

    \begin{exa}
        Sea $X$ un conjunto no vacío y $A\subseteq X$ no vacío. Entonces,
        \begin{equation*}
            \mathcal{F}_A=\left\{M\subseteq X\Big|A\subseteq M \right\}
        \end{equation*}
        es un filtro sobre $X$.
    \end{exa}

    \begin{obs}
        Si $A=\left\{ x\right\}$, escribiremos $\mathcal{F}_{x}$ en vez de $\mathcal{F}_{\left\{x\right\}}$.
    \end{obs}

    \begin{exa}
        Sea $(X,\tau)$ un espacio topológico con $X$. Sea
        \begin{equation*}
            \xi_x=\left\{V\subseteq X\Big|V\in\V{x}\right\}
        \end{equation*}
        con $x\in X$ (recordando que $\V{x}$ es el conjunto de todas las vecindades de $x$). Entonces, $\xi_x$ es un filtro sobre $X$. Este filtro es llamado el \textbf{filtro de vecindades sobre el punto $x$}.
    \end{exa}

    \begin{proof}
        Tenemos que verificar 4 condiciones:
        \begin{enumerate}
            \item $X\in\xi_x$.
            \item $\emptyset\notin\xi_x$.
            \item $M,N\in\V{x}$ implica que $M\cap N\in\V{x}$.
            \item Seea $L\subseteq X$ tal que $V\in\V{x}$ cumple que $V\subseteq L$, entonces $L\in\V{x}$.
        \end{enumerate}
        Luego, $\xi_x$ es un filtro sobre $X$.
    \end{proof}

    \begin{obs}
        Si $\mathcal{F}$ es un filtro sobre $X$, entonces $X\in\mathcal{F}$.
    \end{obs}

    \begin{propo}
        Sean $X$ un conjunto no vacío y $\left\{\mathcal{F}_\alpha\right\}_{\alpha\in I}$ una familia de filtros sobre $X$. Entonces, $\bigcap_{\alpha\in I}\mathcal{F}_\alpha$ es un filtro en $X$.
    \end{propo}

    \begin{proof}
        Sea
        \begin{equation*}
            \mathcal{K}=\bigcap_{\alpha\in I}\mathcal{F}_\alpha
        \end{equation*}
        \begin{enumerate}
            \item $\mathcal{K}\neq\emptyset$, pues $X\in\mathcal{F}_\alpha$, para todo $\alpha\in I$.
            \item $\emptyset\notin\mathcal{K}$, pues $\emptyset\notin\mathcal{F}_\alpha$ para todo $\alpha\in I$.
            \item Sean $A,B\in\mathcal{K}$, entonces $A,B\in\mathcal{F}_\alpha$ para todo $\alpha\in I$. Por ser filtros se sigue que $A\cap B\in\mathcal{F}_\alpha$ para todo $\alpha\in I$, luego $A\cap B\in\mathcal{K}$.
            \item Sea $M\subseteq X$ y sea $L\in\mathcal{K}$ tal que $L\subseteq M$, entonces $L\in\mathcal{F}_\alpha$ para todo $\alpha\in I$. Como cada $\mathcal{F}_\alpha$ cumple la propiedad de absorción, se tiene que $M\in\mathcal{F}_\alpha$ para todo $\alpha\in I$, luego $M\in\mathcal{K}$.
        \end{enumerate}
        Por los 4 incisos anteriores, se sigue que $\mathcal{K}$ es un filtro sobre $X$.
    \end{proof}

    \begin{exa}
        Sea $X=\left\{a,b \right\}$ con $a\neq b$. Tomemos $\mathcal{F}_1=\left\{X,\left\{ a\right\} \right\}$ y $\mathcal{F}_2=\left\{X,\left\{ b\right\} \right\}$. Entonces $\mathcal{F}_1\cup\mathcal{F}_2$ no es un filtro, ya que en caso contario se tendría que $\left\{a \right\}\cap\left\{b \right\}=\emptyset\in\mathcal{F}_1\cap\mathcal{F}_2$, lo cual no puede ser.

        Así, la unión de filros no necesariamente es un filtro.
    \end{exa}

    \begin{propo}
        Si $\left\{\mathcal{F}_\alpha \right\}_{\alpha\in I}$ es una familia de filtros sobre $X$ tal que dados $\alpha,\beta\in I$ se tiene que
        \begin{equation*}
            \mathcal{F}_\alpha\subseteq\mathcal{F}_\beta\textup{ o }\mathcal{F}_\beta\subseteq\mathcal{F}_\alpha
        \end{equation*}
        entonces $\mathcal{F}=\bigcup_{\alpha\in I}\mathcal{F}_\alpha$ es un filtro.
    \end{propo}

    \begin{proof}
        En efecto, veamos que $\mathcal{F}$ es un filtro.
        \begin{enumerate}
            \item $\mathcal{F}\neq\emptyset$ ya que $X\in\mathcal{F}_\alpha$ para algún $\alpha\in I$.
            \item $\emptyset\notin\mathcal{F}$, pues $\emptyset\notin\mathcal{F}_\alpha$ para todo $\alpha\in I$.
            \item Sean $A,B\in\mathcal{F}$. Entonces, existen $\alpha,\beta\in I$ tales que $A\in\mathcal{F}_\alpha$ y $B\in\mathcal{F}_\beta$, entonces se tiene una de las dos contenciones:
            \begin{equation*}
                \mathcal{F}_\alpha\subseteq\mathcal{F}_\beta\textup{ o }\mathcal{F}_\beta\subseteq\mathcal{F}_\alpha
            \end{equation*}
            supongamos que $\mathcal{F}_\alpha\subseteq\mathcal{F}_\beta$, entonces $A,B\in\mathcal{F}_\beta$. Por tanto, $A\cap B\in\mathcal{F}_\beta$. Así, $A\cap B\in\mathcal{F}$.
            \item Sea $M\subseteq X$ y $L\in\mathcal{F}$ tal que $L\subseteq M$. Como $L\in\mathcal{F}$ existe $\alpha\in I$ tal que $L\in\mathcal{F}_\alpha$, luego por la propiedad de absorción $M\in\mathcal{F}_\alpha$. Por tanto, $M\in\mathcal{F}$.
        \end{enumerate}
        Por los cuatro incisos anteriores, se sigue que $\mathcal{F}$ es un filtro sobre $X$.
    \end{proof}

    \begin{mydef}
        Sea $\mathcal{F}$ un filtro sobre $X$. Una familia no vacía $\mathcal{B}$ de subconjuntos de $X$ de $X$ es \textbf{una base para el filtro $\mathcal{F}$} si se cumple lo siguiente:
        \begin{enumerate}
            \item $\mathcal{B}\subseteq\mathcal{F}$.
            \item $\forall F\in\mathcal{F}$ existe $B\in\mathcal{B}$ tal que $B\subseteq F$.
        \end{enumerate}
    \end{mydef}

    \begin{obs}
        Observamos que
        \begin{enumerate}
            \item Si $\mathcal{F}$ es un filtro sobre un conjunto $X$, entonces $\mathcal{F}$ es una base para sí mismo.
            \item Si $\mathcal{B}$ es una base para el filtro $\mathcal{F}$ sobre $X$ y, $B_1,B_2\in\mathcal{B}$, entonces existe $B_3\in\mathcal{B}$ tal que $B_3\subseteq B_1\cap B_2$.
        \end{enumerate}
    \end{obs}

    \begin{mydef}
        Sea $X$ un conjunto no vacío y $\mathcal{B}$ una familia no vacía de subconjuntos no vacíos de $X$. Se dice que $\mathcal{B}$ es \textbf{una base de filtro en $X$}, si se cumple lo siguiente: Dados $B_1,B_2\in\mathcal{B}$ existe $B_3\in\mathcal{B}$ tal que $B_3\subseteq B_1\cap B_2$. 
    \end{mydef}

    \begin{propo}
        Sea $X$ un conjunto no vacío y $\mathcal{B}$ una base de filtro en $X$. Entonces:
        \begin{equation*}
            \mathcal{B}^+=\left\{A\subseteq X\Big|\exists B\in\mathcal{B}\textup{ tal que }B\subseteq A \right\}
        \end{equation*}
        es un filtro en $X$ y este se dice \textbf{el filtro generado por la base $\mathcal{B}$}. Además, $\mathcal{B}$ es una base para $\mathcal{B}^+$.
    \end{propo}

    \begin{proof}
        Se tienen que probar dos cosas:
        \begin{enumerate}
            \item Es claro que $\mathcal{B}\subseteq\mathcal{B}^+$. Por tanto, $\mathcal{B}^+\neq\emptyset$.
            \item $\emptyset\notin\mathcal{B}^+$ es cierto pues $\emptyset\notin\mathcal{B}$, ya que $\mathcal{B}$ es una subcolección no vacía de conjuntos no vacíos.
            \item Tomemos $K,M\in\mathcal{B}^+$ luego, existen $B_1,B_2\in\mathcal{B}$ tal que $B_1\subseteq K$ y $B_2\subseteq M$. Por tanto, $B_1\cap B_2\subseteq K\cap M$. Por ser $\mathcal{B}$ base para un filtro sobre $X$, existe $B_3\in\mathcal{B}$ tal que $B_3\subseteq B_1\cap B_2\subseteq K\cap M$. Luego, $K\cap M\in\mathcal{B}^+$.
            \item Sea $W\subseteq X$ y $L\in\mathcal{B}^+$ tal que $L\subseteq W$. Existe $B\in\mathcal{B}$ tal que $B\subseteq L\subseteq W$, luego $B\subseteq W$. Por tanto, $W\in\mathcal{B}^+$.
        \end{enumerate}
        Por los cuatro incisos anteriores, se sigue que $\mathcal{B}^+$ es un filtro sobre $X$.
    \end{proof}

    \begin{propo}
        Sea $\mathcal{F}$ un filtro sobre $X$ y $A\subseteq X$ tal que $\forall F\in\mathcal{F}$, $A\cap F\neq\emptyset$. Entonces
        \begin{equation*}
            \mathcal{B}=\left\{A\cap F\Big|F\in\mathcal{F} \right\}
        \end{equation*}
        es una base de filtro y, el filtro generado por ella $\mathcal{B}^+$ cumple lo siguiente:
        \begin{enumerate}
            \item $\mathcal{F}\subseteq\mathcal{B}^+$.
            \item $A\in\mathcal{B}^+$.
        \end{enumerate}
    \end{propo}

    \begin{proof}
        Se deben cumplir varios incisos:
        \begin{enumerate}
            \item $\mathcal{B}\neq\emptyset$, pues el conjunto $A\cap X=A\in\mathcal{B}$ ya que $X\in\mathcal{F}$.
            \item $\emptyset\notin\mathcal{B}$ ya que se contradeciría la hipótesis de que $A\cap F=\emptyset$ para todo $F\in\mathcal{F}$.
            \item $B_1,B_2\in\mathcal{B}$ implica que existen $F_1,F_2\in\mathcal{F}$ tales que $B_1=A\cap F_1$ y $B_2= A\cap F_2$. Por tanto
            \begin{equation*}
                B_1\cap B_2=A\cap (F_1\cap F_2)
            \end{equation*}
            donde, $A\cap(F_1\cap F_2)\in\mathcal{B}$ pues, $\mathcal{F}$ es un filtro sobre $X$. Luego, tomando $B_3=B_1\cap B_2\in\mathcal{B}$, se sigue que $B_3\subseteq B_1\cap B_2$.
        \end{enumerate}
        por los tres incisos anteriores, se sigue que $\mathcal{B}$ es base para un filtro sobre $X$. Ya se tiene que $A\in\mathcal{B}^+$, pues $\mathcal{B}\subseteq\mathcal{B}^+$.

        Sea ahora $F\in\mathcal{F}$. Entonces, $F\cap A\in\mathcal{B}^+$. Por propiedad de absorción se debe tener que como $F\cap A\subseteq F$, entonces $F\in\mathcal{B}^+$.
    \end{proof}

    \begin{propo}
        Sean $X$ y $Y$ dos conjuntos no vacíos. Sea $\mathcal{F}$ un filtro en $X$ y $\cf{f}{X}{Y}$ una función. Entonces,
        \begin{equation*}
            \mathcal{B}=\left\{f(A)\Big|A\in\mathcal{F} \right\}
        \end{equation*}
        es una base de filtro en $Y$. En este caso, se denotará por $f(\mathcal{F})$ a $\mathcal{B}^+$, esto es $f(\mathcal{F})=\mathcal{B}^+$.
    \end{propo}

    \begin{proof}
        Se deben verificar tres condiciones
        \begin{enumerate}
            \item $\mathcal{B}\neq\emptyset$, pues $f(X)\in\mathcal{B}$.
            \item Todos los elementos de $\mathcal{B}$ son no vacíos, pues como $\mathcal{F}$ es un filtro sobre $X$, todos sus elementos son no vacíos, así $f(F)\neq\emptyset$ para todo $F\in\mathcal{F}$.
            \item Si $B_1,B_2\in\mathcal{B}$, entonces existen $F_1,F_2\in\mathcal{F}$ tales que $B_1=f(F_1)$ y $B_2=f(F_2)$. Por tanto, el conjunto
            \begin{equation*}
                B_3=f(F_1\cap F_2)\subseteq f(F_1)\cap f(F_2)=B_1\cap B_2
            \end{equation*}
            es tal que $B_3\in\mathcal{B}$, ya que $F_1\cap F_2\in\mathcal{F}$.
        \end{enumerate}
        por los incisos anteriores, se sigue que $\mathcal{B}$ es base de un filtro en $Y$.
    \end{proof}

    \begin{exa}
        Considere $X=\left\{a,b \right\}$, $a\neq b$. Sea $\cf{f}{X}{X}$ dada como sigue:
        \begin{equation*}
            f(a)=a=f(b)
        \end{equation*}
        el conjunto $\mathcal{F}=\left\{X,\left\{a\right\} \right\}$ es un filtro sobre $X$. la colección
        \begin{equation*}
            f(\mathcal{F})=\left\{\left\{a\right\} \right\}
        \end{equation*}
        no es un filtro en $X$ ya que $X\notin f(\mathcal{F})$.
    \end{exa}

    \begin{propo}
        Sean $X$ y $Y$ dos conjuntos no vacíos, $\mathcal{F}$ un filtro en $X$ y $\cf{f}{X}{Y}$ una función. Entonces, $f$ es una función suprayectiva si y sólo si $\left\{f(F)\Big|F\in\mathcal{F} \right\}$ es un filtro en $Y$.
    \end{propo}

    \begin{proof}
        Necesidad: Suponga que $f$ es suprayectiva. Ya se sabe que
        \begin{equation*}
            \mathcal{K}=\left\{f(F)\Big|F\in\mathcal{F} \right\}
        \end{equation*}
        es una base de filtro. Se tiene que por ser $f$ suprayectiva que
        \begin{equation*}
            f(f^{-1}(A))=A,\quad\forall A\subseteq Y
        \end{equation*}
        Se cumplen tres condiciones:
        \begin{enumerate}
            \item $\mathcal{K}\neq\emptyset$ pues, $f(X)\in\mathcal{K}$.
            \item Como $\mathcal{F}$ no contiene al vacío, entonces $\mathcal{K}$ tampoco lo contiene.
            \item Sea $L\subseteq Y$ tal que existe $F\in\mathcal{F}$ con $f(F)\subseteq L$. Entonces:
            \begin{equation*}
                F\subseteq f^{-1}(f(F))\subseteq f^{-1}(L)
            \end{equation*}
            por ser $\mathcal{F}$ un filtro, luego: $f^{-1}(L)\in\mathcal{F}$. Así, $L=f(f^{-1}(L))\in\mathcal{K}$.
            \item Sean $F_1,F_2\in\mathcal{F}$. Se tiene que $f(F_1),f(F_2)\in\mathcal{K}$. Luego:
            \begin{equation*}
                \begin{split}
                    F_1\cap F_2\in\mathcal{F}
                \end{split}
            \end{equation*}
            además, $f(F_1\cap F_2)\subseteq f(F_1)\cap f(F_2)$. Luego, por la propiedad anterior se sigue que $f(F_1)\cap f(F_2)\in\mathcal{K}$ ya que $f(F_1\cap F_2)\in\mathcal{K}$.
        \end{enumerate}
        Por los 4 incisos anteriores se sigue que $\mathcal{K}$ es un filtro.

        Suficiencia: Es inmediata del hecho de que $Y$ está en la familia $f(\mathcal{F})$, luego $f(X)=Y$.

    \end{proof}

    \begin{mydef}
        Sea $X$ un conjunto no vacío. Un \textbf{ultrafiltro} $\mathcal{F}$ en $X$ es un filtro maximal respecto a la inclusión.
    \end{mydef}

    \begin{propo}
        Sea $X$ un conjunto no vacío y sea $\xi$ un filtro en $X$. Entonces existe un ultrafiltro $\mathcal{U}$ en $X$ tal que
        $\xi\subseteq\mathcal{U}$. 
    \end{propo}

    \begin{proof}
        Considere la familia
        \begin{equation*}
            \mathcal{L}=\left\{\xi_\alpha\Big|\alpha\in I \right\}
        \end{equation*}
        de todos los filtros $\xi_\alpha$ en $X$ que contienen a $\xi$. Esta familia es no vacía ya que $\xi\in\mathcal{L}$. Además, esta familia está parcialmente ordenada bajo la relación $\subseteq$. Sea $\mathcal{C}$ una cadena de $(\mathcal{L},\subseteq )$. Tomemos
        \begin{equation*}
            \rho=\bigcup_{\xi\in\mathcal{L}}\xi
        \end{equation*}
        por tanto, $\rho$ es un filtro en $X$ (ver observación anterior para garantizar que la unión de filtros sea un filtro); además, $\xi\subseteq\rho$. Tenemos que $\rho\in\mathcal{L}$ y, por construcción para todo $\mathcal{F}\in\mathcal{C}$, $\mathcal{F}\subseteq\rho$.

        Por el lema de Zorn existe un elemento maximal $\mathcal{U}$ de $(\mathcal{L},\subseteq )$ el cual es el ultrafiltro buscado que contiene a $\xi$.
    \end{proof}

    \begin{exa}
        Sea $X=\left\{a,b \right\}$ con $a\neq b$. Tomemos al filtro
        \begin{equation*}
            \mathcal{F}=\left\{X \right\},\quad\mathcal{U}_1=\left\{X,\left\{a\right\} \right\}\quad\mathcal{U}_2=\left\{X,\left\{b\right\} \right\}
        \end{equation*}
        se tiene que $\mathcal{F}$ es un filtro y, $\mathcal{U}_1,\mathcal{U}_2$ son ultrafiltros en $X$. Además $\mathcal{F}\subseteq\mathcal{U}_1$ y $\mathcal{F}\subseteq\mathcal{U}_2$.

        Es decir, la existencia del ultrafiltro no es única.
    \end{exa}

    \begin{propo}
        Sea $\xi$ un ultrafiltro en el conjunto no vacío $X$. Entonces se cumple lo siguiente:
        \begin{enumerate}
            \item Si $A,B\subseteq X$ y $A\cup B\in\xi$, entonces alguno de los dos $A,B$ es elemento del filtro.
            \item Si $A_1,...,A_k\subseteq X$ tales que $\bigcup_{ i=1}^k A_i\in\xi$, entoncex existe $l\in\natint{1,k}$ tal que $A_l\in\xi$.
        \end{enumerate}
    \end{propo}

    \begin{proof}
        De (1): Sean $A,B\subseteq X$ tales que $A\cup B\in\xi$. Se tienen varios casos:
        \begin{enumerate}
            \item Suoponga que para todo $C\in\xi$ se tiene que $C\cap A\neq\emptyset$. Entonces, el conjunto
            \begin{equation*}
                \mathcal{B}=\left\{C\cap A\Big|A\in\xi \right\}
            \end{equation*}
            es una base de filtro y, $\mathcal{B}$ cumple que $\xi\subseteq\mathcal{B}$. Por tanto, $\xi=\mathcal{B}$. Pero, $A\in\mathcal{B}$, luego $A\in\mathcal{B}$.
            \item Suponga que existe $C_0\in\xi$ tal que $C_0\cap A=\emptyset$. Entonces:
            \begin{equation*}
                \begin{split}
                    C\cap B&=\left(C\cap A \right)\cup\left(C\cap B \right)\\
                    &=C\cap\left(A\cup B \right)\\
                \end{split}
            \end{equation*}
            donde $C\in\xi$ y $A\cup B\in\xi$. Por tanto, $C\cap B\in\xi$. Pero, $C\cap B\subseteq B$, luego por absorción se sigue que $B\in\xi$.
        \end{enumerate}

        De (2): Se hace usando inducción sobre $k$.
    \end{proof}

    \begin{propo}
        Sea $\xi$ un filtro en $X$. Entonces, $\xi$ es un ultrafiltro si y sólo si para todo subconjunto $A\subseteq X$, $A\in\xi$ ó $X-A\in\xi$.
    \end{propo}

    \begin{proof}
        Necesidad: Sea $A\subseteq X$. Escribimos $X=A\cup(X-A)$. Como $\xi$ es un filtro, entonces $X\in\xi$. Por la proposición anterior se tiene que $A\in\xi$ ó $X-A\in\xi$.

        Suficiencia: Sea $\eta$ un filtro en $X$ tal que $\xi\subseteq \eta$. Tomemos $A\subseteq X$ tal que $A\in\eta$. Entonces, $X-A\notin\eta$, luego $X-A\notin\xi$. Por hipótesis, debe suceder que $A\in\xi$. De esta forma, $\xi=\eta$.

        Luego, $\xi$ es un ultrafiltro.
    \end{proof}

    \begin{excer}
        Sean $X$ y $Y$ conjuntos, $\xi$ un ultrafiltro de $X$ y sea $\cf{f}{X}{Y}$ una función suprayectiva. Entonces, $f(\xi)$ es un ultrafiltro en $Y$.
    \end{excer}

    \begin{proof}
        Ya se tiene que $f(\xi)$ es un filtro en $Y$. 
    \end{proof}

    \begin{exa}
        Sea $X$ un conjunto no vacío y sea $\left\{x_n \right\}_{n\in\mathbb{N}}$ una sucesión de puntos en $X$, entonces
        \begin{equation*}
            \rho=\left\{A\subseteq X\Big|\exists N\in\mathbb{N}\textup{ tal que }\forall k\geq N, x_{k}\in A \right\}
        \end{equation*}
        es un filtro en $X$ y, se dice el \textbf{filtro asociado a la sucesión}.
    \end{exa}

    \begin{proof}
        Hay que verificar 4 condiciones:
        \begin{enumerate}
            \item $\emptyset\notin\rho$.
            \item $X\in\rho$ ya que la sucesión está en $X$.
            \item Sean $A,B\in\rho$. Entonces, existen $N,M\in\mathbb{N}$ tales que $k\geq N\Rightarrow x_k\in A$ y, $k\geq M\Rightarrow x_k\in B$. Sea
            \begin{equation*}
                N_0=\max\left\{N,M \right\}\in\mathbb{N}
            \end{equation*}
            entonces, si $k\geq N_0$ se tiene que $x_k\in A$ y $x_k\in B$, luego $x_k\in A\cap B$. Por ende, $A\cap B\in \rho$.
            \item Sea $L\subseteqq X$ y $A\in\rho$ tal que $A\subseteq L$. Como $A\in\rho$ entonces existe $N\in\mathbb{N}$ tal que si $k\geq N\Rightarrow x_k\in A\subseteq L$. Por ende, $L\in\rho$.
        \end{enumerate}
        por los incisos anteriores, se sigue que $\rho$ es un filtro en $X$.
    \end{proof}

    \begin{mydef}
        Sea $(X,\tau)$ un espacio topológico. Una sucesión de puntos de $X$, $\left\{x_n \right\}_{ n\in\mathbb{N}}$ se dice que \textbf{converge} a un punto $l\in X$ si para todo $U\in\tau$ tal que $l\in U$ existe $N\in\mathbb{N}$ tal que para todo $k\geq N$, $x_k\in U$.
    \end{mydef}

    \begin{propo}
        Sea $\left\{x_n \right\}_{ n\in\mathbb{N}}$ una sucesión de puntos de $X$ y sea $\rho$ el filtro asociado a la sucesión. Sea $l\in X$, entonces $\left\{x_n \right\}_{ n\in\mathbb{N}}$ converge al punto $l$ si y sólo si $\xi_l\subseteq \rho$.        
    \end{propo}

    \begin{proof}
        Necesidad: Suponga que la sucesión converge a $l\in X$. Sea $V\in\xi_l$, entonces existe un abierto $U\subseteq X$ tal que $l\in U\subseteq V$. Como la sucesión converge a $l$ y $U$ es abierto, existe $N\in\mathbb{N}$ tal que para todo $k\geq N$ se tiene que $x_k\in U\subseteq V$. Por tanto, $V\in\rho$.

        Luego, $\xi_l\subseteq\rho$.

        Suficiencia: Suponga que $\xi_l\subseteq\rho$. Si $U\subseteq X$ es abierto tal que $l\in U$, entonces $U\in\xi_l\subseteq \rho$, luego existe $N\in\mathbb{N}$ tal que $k\geq N$ implica que $x_k\in U$.

        Así, la sucesión $\left\{x_n \right\}_{ n\in\mathbb{N}}$ converge a $l$.
    \end{proof}

    \begin{exa}
        Sea $X=\left\{1,2,3 \right\}$ y considere la topología $\tau=\left\{X,\emptyset,\left\{1,2 \right\},\left\{3\right\} \right\}$. Para cada $n\in\mathbb{N}$ se define $X_n=1$.

        Se tiene que la sucesión $\left\{x_n\right\}_{ n\in\mathbb{N}}$ converge a $1$ y $2$.
    \end{exa}

\end{document}