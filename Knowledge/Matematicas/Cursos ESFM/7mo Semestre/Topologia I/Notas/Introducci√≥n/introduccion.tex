\documentclass[12pt]{report}
\usepackage[spanish]{babel}
\usepackage[utf8]{inputenc}
\usepackage{amsmath}
\usepackage{amssymb}
\usepackage{amsthm}
\usepackage{graphics}
\usepackage{subfigure}
\usepackage{lipsum}
\usepackage{array}
\usepackage{multicol}
\usepackage{enumerate}
\usepackage[framemethod=TikZ]{mdframed}
\usepackage[a4paper, margin = 1.5cm]{geometry}
\usepackage{cancel}

%En esta parte se hacen redefiniciones de algunos comandos para que resulte agradable el verlos%

\renewcommand{\theenumii}{\roman{enumii}}

\def\proof{\paragraph{Demostración:\\}}
\def\endproof{\hfill$\blacksquare$}

\def\sol{\paragraph{Solución:\\}}
\def\endsol{\hfill$\square$}

%En esta parte se definen los comandos a usar dentro del documento para enlistar%

\newtheoremstyle{largebreak}
  {}% use the default space above
  {}% use the default space below
  {\normalfont}% body font
  {}% indent (0pt)
  {\bfseries}% header font
  {}% punctuation
  {\newline}% break after header
  {}% header spec

\theoremstyle{largebreak}

\newmdtheoremenv[
    leftmargin=0em,
    rightmargin=0em,
    innertopmargin=0pt,
    innerbottommargin=5pt,
    hidealllines = true,
    roundcorner = 5pt,
    backgroundcolor = gray!60!red!30
]{exa}{Ejemplo}[section]

\newmdtheoremenv[
    leftmargin=0em,
    rightmargin=0em,
    innertopmargin=0pt,
    innerbottommargin=5pt,
    hidealllines = true,
    roundcorner = 5pt,
    backgroundcolor = gray!50!blue!30
]{obs}{Observación}[section]

\newmdtheoremenv[
    leftmargin=0em,
    rightmargin=0em,
    innertopmargin=0pt,
    innerbottommargin=5pt,
    rightline = false,
    leftline = false
]{theor}{Teorema}[section]

\newmdtheoremenv[
    leftmargin=0em,
    rightmargin=0em,
    innertopmargin=0pt,
    innerbottommargin=5pt,
    rightline = false,
    leftline = false
]{propo}{Proposición}[section]

\newmdtheoremenv[
    leftmargin=0em,
    rightmargin=0em,
    innertopmargin=0pt,
    innerbottommargin=5pt,
    rightline = false,
    leftline = false
]{cor}{Corolario}[section]

\newmdtheoremenv[
    leftmargin=0em,
    rightmargin=0em,
    innertopmargin=0pt,
    innerbottommargin=5pt,
    rightline = false,
    leftline = false
]{lema}{Lema}[section]

\newmdtheoremenv[
    leftmargin=0em,
    rightmargin=0em,
    innertopmargin=0pt,
    innerbottommargin=5pt,
    roundcorner=5pt,
    backgroundcolor = gray!30,
    hidealllines = true
]{mydef}{Definición}[section]

\newmdtheoremenv[
    leftmargin=0em,
    rightmargin=0em,
    innertopmargin=0pt,
    innerbottommargin=5pt,
    roundcorner=5pt
]{excer}{Ejercicio}[section]

%En esta parte se colocan comandos que definen la forma en la que se van a escribir ciertas funciones%

\newcommand\abs[1]{\ensuremath{\biglvert#1\bigrvert}}
\newcommand\divides{\ensuremath{\bigm|}}
\newcommand\cf[3]{\ensuremath{#1:#2\rightarrow#3}}
\newcommand\contradiction{\ensuremath{\#_c}}
\newcommand{\V}[1]{\ensuremath{\mathcal{V}(#1)}}
\newcommand{\Int}[1]{\ensuremath{\mathring{#1}}}
\newcommand{\Cls}[1]{\ensuremath{\overline{#1}}}
\newcommand{\Fr}[1]{\ensuremath{\textup{Fr}(#1)}}
\newcommand{\natint}[1]{\ensuremath{\left[|#1|\right]}}
\newcommand{\floor}[1]{\ensuremath{\lfloor#1\rfloor}}
\newcommand{\Card}[1]{\ensuremath{\textup{Card}\left(#1\right)}}
\newcommand{\Pot}[1]{\ensuremath{\mathcal{P}\left(#1\right)}}
\newcommand{\id}[1]{\ensuremath{\textup{id}_{#1}}}

%recuerda usar \clearpage para hacer un salto de página

\begin{document}
    \setlength{\parskip}{5pt} % Añade 5 puntos de espacio entre párrafos
    \setlength{\parindent}{12pt} % Pone la sangría como me gusta
    \title{Notas del curso Topología I}
    \author{Cristo Daniel Alvarado}
    \maketitle

    \tableofcontents %Con este comando se genera el índice general del libro%

    \setcounter{chapter}{-1} %En esta parte lo que se hace es cambiar la enumeración del capítulo%
    
    \chapter{Introduccion}
    
    \section{Temario}
    
    Checar el Munkres

    \section{Bibliografía}    

    \begin{enumerate}
        \item J. R. Munkres 'Topología' - Prentices Hall.
        \item M. Gemignsni 'Elementary Topology' -  Dover.
        \item J. Dugundji 'Topology' -  Allyn Bacon.
    \end{enumerate}

    \chapter{Conceptos Fundamentales}

    \section{Fundamentos}

    \begin{mydef}
        Sea $X$ un conjunto y $\mathcal{A}$ una familia no vacía de subconjuntos de $X$. Definamos los \textbf{complementos de $\mathcal{A}$}
        \begin{equation*}
            \mathcal{A}':=\left\{X-A\big| A\in\mathcal{A} \right\}
        \end{equation*}
        (básicamente es el conjunto de todos los complementos de los conjuntos en $\mathcal{A}$). Para no perder ambiguedad, no denotaremos al complemento de un conjunto por $B^c$, sino por $X-B$ (para denotar quien es el conjunto sobre el que se toma el complemento del conjunto).

        La \textbf{unión de los elementos} de $\mathcal{A}$ se define como el conjunto:
        \begin{equation*}
            \bigcup\mathcal{A}=\bigcup_{A\in\mathcal{A}}A=\left\{x\in X\big| x\in A\textup{ para algún elemento }A\in\mathcal{A} \right\}
        \end{equation*}
        denotada por el símbolo de la izquierda.

        La \textbf{intersección de los elementos} de $\mathcal{A}$ se define como el conjunto:
        \begin{equation*}
            \bigcap\mathcal{A}=\bigcap_{A\in\mathcal{A}}A=\left\{x\in X\big| x\in A\textup{ para todo elemento }A\in\mathcal{A} \right\}
        \end{equation*}
    \end{mydef}

    \begin{obs}
        En caso de que la colección $\mathcal{A}$ sea vacía, no se puede hacer lo que marca la definición anterior. Como $\mathcal{A}$ es vacía, entonces $\mathcal{A}'$ también es vacía.
        \begin{enumerate}
            \item Suponga que $\cup\mathcal{A}\neq\emptyset$, entonces existe $x\in X$ tal que $x\in \cup\mathcal{A}$, luego existe algún elemento $A\in\mathcal{A}$ tal que $x\in A$, pero esto no puede suceder, pues la familia $\mathcal{A}$ es vacía. \contradiction. Por tanto, $\cup\mathcal{A}=\emptyset$.
            \item Ahora, si aplicamos las leyes de Morgan, y tomamos
            \begin{equation*}
                X-\cap\mathcal{A}=X-\cap\emptyset=\cup\emptyset'=\cup\emptyset=\emptyset
            \end{equation*}
            luego, $\cap\mathcal{A}=X$.
        \end{enumerate}
        En definitiva, si $\mathcal{A}$ es una colección vacía, entonces definimos $\cup\mathcal{A}=\emptyset$ y $\cap\mathcal{A}=X$.
    \end{obs}

    La observación junto con la definición anterior se usarán a lo largo de todo el curos y serán de utilidad.

    \begin{mydef} 
        Sea $X$ un conjunto y sea $\tau$ una familia de subconjuntos de $X$. Se dice que $\tau$ es una \textbf{una topología definida sobre $X$} si se cumple lo siguiente:
        \begin{enumerate}
            \item $\emptyset,X\in\tau$.
            \item Si $\mathcal{A}$ es una subcolección de $\tau$, entonces $\bigcup\mathcal{A}\in\tau$.
            \item Si $A,B\in\tau$, entonces $A\cap B\in\tau$.
        \end{enumerate}
    \end{mydef}

    \begin{obs}
        En algunos libros viejos viene la siguiente condición adicional a la definición:
        \begin{enumerate}
            \setcounter{enumi}{3}
            \item Si $p,q\in X$ con $p\neq q$, entonces existen $U, V\in\tau$ tales que $p\in U$, $q\in V$ y $U\cap V=\emptyset$.
        \end{enumerate}
        en este caso se dirá que el espacio es \textbf{Hausdorff}.
    \end{obs}

    \begin{obs}
        Se tienen las siguientes observaciones:
        \begin{enumerate}
            \item Sea $X$ un conjunto y $\mathcal{A}$ una familia de subconjuntos de $X$. Si
            \begin{equation*}
                \mathcal{A}=\left\{A_\alpha\big|\alpha\in I \right\}
            \end{equation*}
            entonces podemos escribir
            \begin{equation*}
                \bigcup\mathcal{A}=\bigcup_{A\in\mathcal{A}}A=\bigcup_{\alpha\in I}A_\alpha
            \end{equation*}
            e igual con la intersección:
            \begin{equation*}
                \bigcap\mathcal{A}=\bigcap_{A\in\mathcal{A}}A=\bigcap_{\alpha\in I}A_\alpha
            \end{equation*}
            Si $\mathcal{A}$ es una familia vacía, y se toma como definición lo dicho en la observación 1.0.1, entonces podemos omitir el primer inciso de la definición anterior.
            \item Si $\tau$ es una topología sobre $X$ y para $n\in\mathbb{N}$, $A_1,...,A_n\in\tau$, entonces $A_1\cap...\cap A_n\in\tau$.
        \end{enumerate}
    \end{obs}

    \begin{exa}
        Sea $X$ un conjunto no vacío.
        \begin{enumerate}
            \item El conjunto potencia (denotado por $\mathcal{P}$) de $X$ es una topología sobre $X$, la cual se llama la \textbf{topología discreta}, y se denota por $\tau_D$.
            \item La colección formada únicamente por $X$ y $\emptyset$ es una topolgía sobre $X$, es decir $\tau=\left\{\emptyset,X \right\}$ es llamada la \textbf{topología indiscreta}, y se escribe como $\tau_I$.
            \item En el caso de que $X=\left\{1\right\}$, se tendría que $\tau_D=\left\{\emptyset,\left\{1\right\} \right\}$ y $\tau_I=\left\{\emptyset,\left\{1\right\} \right\}$.
            
            Si $X=\left\{1,\zeta\right\}$, entonces $\tau_D=\left\{\emptyset,\left\{1\right\},\left\{\zeta\right\},\left\{1,\zeta\right\} \right\}$ y $\tau_I=\left\{\emptyset,\left\{1,\zeta\right\} \right\}$.

            \item Si $\tau$ es una topología sobre $X$, entonces
            \begin{equation*}
                \tau_I\subseteq\tau\subseteq\tau_D
            \end{equation*}
            \item Sea $a\in X$. Entonces $\tau=\left\{\emptyset,X,\left\{a\right\},\right\}$ es una topología sobre $X$.
            \item Sea $A\subseteq X$ y sea $\tau\left(A\right)=\left\{B\subseteq X\big| A\subseteq B \right\}\bigcup\left\{\emptyset\right\}$. Esta familia $\tau\left(A\right)$ es una topología sobre $X$.
        \end{enumerate} 
    \end{exa}

    \begin{sol}
        Para el inciso 6., veamos que $\tau(A)$ es una topología sobre $X$. En efecto, verificaremos que se cumplen las 3 condiciones:
        \begin{enumerate}
            \item Claro que $\emptyset\in\tau(A)$ por definición de $\tau(A)$. Además $X\in\tau(A)$ ya que $X\subseteq X$ y $A\subseteq X$.
            \item Sea $\mathcal{B}$ una familia no vacía de subconjuntos de $\tau(A)$, entonces existe $B_0\in\mathcal{B}$ tal que $A\subseteq B_0$, por lo cual
            \begin{equation*}
                A\subseteq B_0\subseteq\bigcup_{B\in\mathcal{B}}B\subseteq X
            \end{equation*}
            por tanto $\bigcup_{B\in\mathcal{B}}B\in\tau(A)$.
            \item Sean $C,D\in\tau(A)$, entonces $A\subseteq C$ y $A\subseteq B$, por ende $A\subseteq B\cap C\subseteq X$. Así, $B\cap C\in\tau(A)$.
        \end{enumerate}
        Por los incisos anteriores, la familia descrita en el inciso 6. es una topología sobre $X$.
    \end{sol}

    \begin{obs}
        Sea $X$ un conjunto no vacío. Si $A=\left\{a\right\}\subseteq X$, entonces escribimos $\tau_a$ en vez de $\tau\left(A\right)$.
    \end{obs}

    \setcounter{exa}{0}
    \begin{exa}
        Se continuan con los ejemplos anteriores:
        \begin{enumerate}
            \setcounter{enumi}{6}
            \item Sea $\tau_{cf}=\left\{A\subseteq X\big| X-A\textup{ es un conjunto finito} \right\}\bigcup\left\{\emptyset\right\}$. Esta es una topología sobre $X$ y se llama la \textbf{topología de los complementos finitos}.
            \item Si $X$ es un conjunto finito, entonces $\tau_{cf}=\tau_D=\mathcal{P}$.
            \item Considere (en un conjunto finito $X$) a $\tau_{cf}$ y sean $a,b\in X$ con $a\neq b$. Si $U_a=X-\left\{b\right\}$, $U_b=X-\left\{a\right\}$, entonces $U_a,U_b\in\tau_{cf}$ y además, $a\in U_a$ pero $b\notin U_a$ y $a\notin U_b$ pero $b\in U_b$. Esta propiedad es muy importante tenerla en mente pues más adelante se usará.
        \end{enumerate}
    \end{exa}

    \begin{sol}
        Veamos que la famila del ejemplo 7. es una topología sobre $X$. En efecto, veamos que se cumplen las 3 condiciones:
        \begin{enumerate}
            \item Claro que $\emptyset\in\tau_{cf}$ (por definición de $\tau_{cf}$). Y además $X\in\tau_{cf}$ ya que $\emptyset=X-X$ es un conjunto finito y $X\subseteq X$.
            \item Sea $\mathcal{A}$ una familia no vacía de subconjuntos de $\tau_{cf}$. Se cumple entonces que existe $A_0\in\mathcal{A}$ tal que $X-A_0$ es finito. Por lo cual como
            \begin{equation*}
                X-\bigcup\mathcal{A}\subseteq X-A
            \end{equation*}
            ya que $A\subseteq\bigcup\mathcal{A}$, se tiene que $X-\bigcup\mathcal{A}$ es finito y $\bigcup\mathcal{A}\subseteq X$. Por tanto, $\bigcup\mathcal{A}\in\mathcal{A}$.
            \item Sean $A,B\in\tau_{cf}$. Probaremos que $A\cap B\in\tau_{cf}$. Afirmamos que $X-A\cap B$ es finito, en efecto, por leyes de Morgan se tiene que
            \begin{equation*}
                X-(A\cap B)=(X-A)\cup(X-B)\subseteq X
            \end{equation*}
            donde $X-A$ y $X-B$ son finitos, por lo cual su unión también lo es. Por tanto $A\cap B\in\tau_{cf}$.
        \end{enumerate}
        Por los tres incisos anteriores, se sigue que $\tau_{cf}$ es una topología sobre $X$.
    \end{sol}

    A continuación se verá una proposición la cual tiene como objetivo el inducir una topología sobre un espacio métrico $(X,d)$ arbitrario.

    \begin{propo}
        Sea $(X,d)$ un espacio métrico. Dados $a\in X$ y $\varepsilon\in\mathbb{R}^+$, al conjunto $B_d(x,\varepsilon)=\left\{y\in X\big|d(x,y)<\varepsilon \right\}$ se llama \textbf{$\varepsilon$-bola con centro en $x$ y radio $\varepsilon$}. 
        
        Sea
        \begin{equation*}
            \tau_d=\left\{A\subseteq X\big| \forall a\in A \exists r>0\textup{ tal que }B_d(a,r)\subseteq A \right\}
        \end{equation*}
        Esta colección es una topología sobre $X$.
    \end{propo}

    \begin{proof}
        Se verificará que se cumplen las tres condiciones.
        \begin{enumerate}
            \item Por vacuidad, $\emptyset\in\tau_d$. Además, $X\in\tau_d$, pues para todo $x\in X$, $B_d(x,1)\subseteq X$.
            \item Sean $\mathcal{A}$ una familia no vacía de subconjuntos de $\tau_d$. Sea $p\in\cup\mathcal{A}$, es decir que existe $A_\beta\in\mathcal{A}$ tal que $p\in A_\beta$, así existe $r>0$ tal que $B_d(a,r)\subseteq A_\beta\subseteq\cup\mathcal{A}$, luego $\cup\mathcal{A}\in\tau_d$.
            \item Sean $M,N\in\tau_d$, y sea $p\in M\cap N$, es decir que $p\in M$ y $p\in N$, por lo cual existen $r_1,r_2>0$ tales que $B_d(p,r_1)\subseteq M$ y $B_d(p,r_2)\subseteq N$. Sea $r=\min\left\{r_1,r_2\right\}$, es inmediato que $B_d(p,r)\subseteq B_d(p,r_i)$, para $i=1,2$. Por tanto, $B_d(p,r)\subseteq M\cap N$. Luego, como el $p$ fue arbitrario,se sigue que $M\cap N\in \tau_d$.
        \end{enumerate}
    \end{proof}

    \begin{mydef}
        La topología de la proposición anterior es llamada la \textbf{topología generada por la métrica $d$}.
    \end{mydef}

    \begin{excer}
        Sea $(X,d)$ espacio métrico. Veamos que, dados $x\in X$ y $r>0$, se cumple que $B_d(x,r)\in\tau_d$.
    \end{excer}

    \begin{sol}
        Sea $y\in B_d(x,r)$, entonces $d(x,y)<r$. Sea $\varepsilon=d(x,y)$ y, supongamos que $x\neq y$ (pues en caso contrario, el caso es inmediato ya que $B_d(x,r)\subseteq B_d(x,r)$) luego $\varepsilon>0$ y además $\varepsilon<r$. Sea $s=r-\varepsilon\in\mathbb{R}^+$.

        Afirmamos que $B_d(y,s)\subseteq B_d(x,r)$. En efecto, sea $z\in B_d(y,s)$, entonces
        \begin{equation*}
            \begin{split}
                d(z,y)&<s\\
                \Rightarrow d(z,y)&<r-\varepsilon\\
                \Rightarrow d(z,y)+\varepsilon&<r\\
                \Rightarrow d(z,y)+d(y,x)&<r\\
                \Rightarrow d(z,x)&<r\\
            \end{split}
        \end{equation*}
        por tanto, $x\in B_d(x,r)$. Luego, $B_d(x,r)\in\tau_d$.
    \end{sol}

    \begin{lema}
        Todo espacio métrico $(X,d)$ es Hausdorff. 
    \end{lema}

    \begin{proof}
        Veamos que dados $x,y\in X$, $x\neq y$ existen $r,s\in\mathbb{R}^+$ tales que $B_d(x,r)\cap B_d(y,s)=\emptyset$. Como $x\neq y$ entonces $d(x,y)=m\in\mathbb{R}^+$. Tomemos $r=\frac{m}{\pi}$ y $s=\frac{\pi-1}{\pi}m$ y veamos que la intersección es vacía.
    
        En efecto, en caso de que no lo fuese se tendría que si existiera $p\in B_d(x,r)\cap B_d(y,s)$, entonces $d(p,x)<\frac{m}{\pi}$ y $d(p,y)<\frac{\pi-1}{\pi}m$, por lo cual de la desigualdad triangular se sigue que:
        \begin{equation*}
            d(x,y)\leq d(p,x)+d(p,y)<\frac{1+\pi-1}{\pi}m=m=d(x,y)
        \end{equation*}
        lo cual es una contradicción\contradiction. Por tanto, la intersección es vacía.
    \end{proof}

    Retomando al espacio métrico $(X,d)$, tenemos que para $A\subseteq X$, $A\in\tau_d$ si y sólo si existen $\left\{a_\alpha\right\}_{\alpha\in I}\subseteq A$ y $\left\{\varepsilon_\alpha\right\}_{\alpha\in I}\subseteq\mathbb{R}^+$ tales que
    \begin{equation*}
        \bigcup_{\alpha\in I }B_d(a_\alpha,\varepsilon_\alpha)=A
    \end{equation*}
    donde $\forall\alpha\in I$ se tiene que $A_\alpha\in \mathcal{A}$.

    \begin{cor}
        Sea $(X,d)$ un espacio métrico y
        \begin{equation*}
            \mathcal{B}_d=\left\{B_d(x,\varepsilon)|x\in X,\varepsilon\in\mathbb{R}^+ \right\}
        \end{equation*}
        entonces, para $A\subseteq X$ se tiene que $A\in\tau_d$ si y sólo si existe una colección $\left\{B_\alpha \right\}_{\alpha\in I}\subseteq \mathcal{B}_d$ tal que $A=\bigcup_{\alpha\in I}B_\alpha$. La colección $\mathcal{B}_d\subseteq\tau_d$.
    \end{cor}

    \begin{exa}
        Sea $m\in\mathbb{N}$ y considere el espacio métrico $\mathbb{R}^m$ con la métrica $d_u$, siendo:
        \begin{equation*}
            d_u(x,y)=[(x_1-y_1)^2+...+(x_m-y_m)^2]^{\frac{1}{2}}
        \end{equation*}
        para $x=(x_1,...,x_m),y=(y_1,...,y_m)\in\mathbb{R}^m$. Esta métrica será denominada \textbf{métrica usual}. Vamos a escribir a la topología generada por esta métrica como $\tau_u$, y se dice la \textbf{topología usual definida sobre $\mathbb{R}^m$}. En particular, cuando $m=1$ tenemos que $\tau_u$ la topología usual definida sobre $\mathbb{R}$. En este caso, se tiene que $A\in\tau_u$ si y sólo si existen $\left\{a_\alpha\right\}_{\alpha\in I}$ y $\left\{B_\alpha\right\}_{\alpha\in I}$ subfamilias de $\mathbb{R}$ tal que $A=\bigcup_{\alpha\in I}\left(a_\alpha,b_\alpha\right)$.
    \end{exa}

    \begin{obs}
        Tenemos que para todo $n\in\mathbb{N}$, los conjuntos $\left(-\frac{1}{n},\frac{1}{n}\right)\in\tau_u$, y $\bigcap_{n\in\mathbb{N}}\left(-\frac{1}{n},\frac{1}{n}\right)=\left\{0\right\}\notin\tau_u$. Es decir, que la topología solo es cerrada (en general) bajo intersecciones finitas.
    \end{obs}

    \begin{mydef}
        Sea $X$ un conjunto, y sean $\tau_1$ y $\tau_2$ topologías sobre $X$. Decimos que $\tau_2$ es \textbf{más fina} que la topología $\tau_1$ si se tiene que $\tau_1\subseteq\tau_2$ (a veces también se dice que $\tau_1$ es \textbf{menos fina} que $\tau_2$).
    \end{mydef}

    \begin{exa}
        Sea $X=\left\{1,2,3\right\}$, $\tau_1=\left\{X,\emptyset,\left\{1\right\} \right\}$, $\tau_2=\left\{X,\emptyset,\left\{2\right\} \right\}$. Tomemos
        \begin{equation*}
            \tau_1\cup\tau_2=\left\{X,\emptyset,\left\{1\right\},\left\{2\right\}\right\}
        \end{equation*}
        la familia $\tau_1\cup\tau_2$ no es una topología sobre $X$, pues no es cerrada bajo uniones arbitrarias. Con esto se tiene que la unión de dos topologías no necesariamente es una topología.
    \end{exa}

    \begin{theor}
        Sea $X$ un conjunto, y sea $\left\{\tau_\alpha\right\}_{\alpha\in I}$ una familia de topologías sobre $X$, entonces $\tau=\bigcap_{\alpha\in I}\tau_\alpha$ es una topología sobre $X$.
    \end{theor}

    \begin{proof}
        Veamos que se cumplen las tres condiciones.
        \begin{enumerate}
            \item Claro que $X,\emptyset\in\tau$, pues $X,\emptyset\in\tau_\alpha$, para todo $\alpha\in I$.
            \item Sea $\mathcal{A}=\left\{A_\beta\right\}_{\beta\in J}\subseteq\tau=\bigcap_{\alpha\in I}\tau_\alpha$ una subcolección arbitraria de elementos de $\tau$. Por ser $\tau_\alpha$ una topología, se sigue que $\bigcup\mathcal{A}\in\tau_\alpha$, para todo $\alpha\in I$. Por tanto, $\bigcup\mathcal{A}\in\tau$.
            \item Sean $K,L\in\tau$, entonces $K,L\in\tau_\alpha$, para todo $\alpha\in I$, luego como $\tau_\alpha$ es una topología sobre $X$, se tiene que $L\cap K\in \tau_\alpha$, para todo $\alpha\in I$, por tanto $L\cap K\in\tau$.
        \end{enumerate}
        Por los tres incisos anteriores, se sigue que $\tau$ es una topología sobre $X$.
    \end{proof}

    \begin{cor}
        Sea $X$ un conjunto y sean $\mathcal{A}$ una familia de subconjuntos de $X$. Definimos
        \begin{equation*}
            \mathcal{K}=\left\{\tau|\tau\textup{ es una topología sobre }X\textup{ y }\mathcal{A}\subseteq\tau \right\}
        \end{equation*}
        Entonces:
        \begin{enumerate}
            \item $\tau_D\in\mathcal{K}$.
            \item Definiendo $\tau(\mathcal{A})=\bigcap_{\tau\in\mathcal{K}}\tau$, se tiene que $\tau(\mathcal{A})$ es una topología sobre $X$.
            \item Para toda topología $\tau\in\mathcal{K}$, $\tau(\mathcal{A})\subseteq\tau$.
            \item $\tau(\mathcal{A})\in\mathcal{K}$.
        \end{enumerate}
    \end{cor}

    \begin{proof}
        De 1. Es inmediato, pues como $\mathcal{A}\subseteq\mathcal{P}=\tau_D$ y $\tau_D$ es una topología sobre $X$, se sigue que $\tau_D\in\mathcal{K}$.

        De 2. Es inmediato del teorema anterior.

        De 3. Como $\tau(\mathcal{A})=\bigcap_{\tau\in\mathcal{K}}\tau$, entonces $\tau(\mathcal{A})\subseteq\tau$, para toda $\tau\in\mathcal{K}$.

        De 4. Por 2. $\tau(\mathcal{A})$ es una topología sobre $X$, y además $\mathcal{A}\subseteq\tau(\mathcal{A})$, pues $\mathcal{A}\subseteq\tau$, para todo $\tau\in\mathcal{K}$, luego $\mathcal{A}\subseteq\bigcap_{\tau\in\mathcal{K}}\tau=\tau(\mathcal{A})$. Por ende, $\tau(\mathcal{A})\in\mathcal{K}$.
    \end{proof}

    \begin{mydef}
        Un \textbf{espacio topológico} es una pareja $(X,\tau)$ en donde $X$ es un conjunto y $\tau$ es una topología sobre $X$. A los elementos de $\tau$ los llamaremos los \textbf{conjuntos abiertos} del espacio $(X,\tau)$ a veces también se les nombra como los \textbf{$\tau$-abiertos de $X$}.
    \end{mydef}

    \begin{exa}
        Ejemplos de espacios topológicos son $(\mathbb{R},\tau_D)$, $(\mathbb{R},\tau_I)$, $(\mathbb{R},\tau_{cf})$, $(\mathbb{R},\tau_u)$, etc... Las diferencias notables son que $\left\{1,\sqrt{2} \right\}$ es abierto en $(\mathbb{R},\tau_D)$, pero no en $(\mathbb{R},\tau_u)$.
    \end{exa}

    Sea $X$ un conjunto y $\mathcal{A}\subseteq\mathcal{P}$. Por el corolario anterior, podemos trabajar con la topología $\tau(\mathcal{A})$, y tenemos así al espacio topológico $(X,\tau(\mathcal{A}))$, el cual en particular tiene como abiertos a los elementos de la familia $\mathcal{A}$.
    
    \begin{mydef}
        Sea $(X,\tau)$ un espacio topológico.
        \begin{enumerate}
            \item Un subconjunto $C\subseteq X$ es un \textbf{conjunto cerrado} del espacio topológico $(X,\tau)$ si $X-C\in\tau$.
        \end{enumerate}
    \end{mydef}

    \begin{exa}
        En $(\mathbb{R},\tau_u)$ se tiene que $\mathbb{R}$ y $\emptyset$ son abiertos y cerrados a la vez, pero el conjunto $[1,2[$ no es abierto ni cerrado, $]1,2[$ es abierto pero no cerrado y $[1,2]$ no es abierto pero sí es cerrado.
    \end{exa}

    \begin{propo}
        Sea $(X,\tau)$ un espacio topológico.
        \begin{enumerate}
            \item Si $A_1,...,A_n$ son subconjuntos cerrados de $(X,\tau)$, entonces su unión $A_1\cup...\cup A_n$ es un cerrado de $(X,\tau)$.
            \item Si $\mathcal{A}$ es una familia arbitraria de conjuntos cerrados en $(X,\tau)$, entonces $\bigcap\mathcal{A}$ es un conjunto cerrado.
        \end{enumerate}
    \end{propo}

    \begin{proof}
        De (1): Consideremos el complemento de la unión. Se tiene que:
        \begin{equation*}
            \begin{split}
                X-\bigcup_{i=1}^nA_i=&\bigcap_{i=1}^n(X-A_i)\\
            \end{split}
        \end{equation*}
        el cuál es abierto por ser intersección finita de abiertos. Luego $\bigcap_{i=1}^nA_i$ es cerrado.

        De (2): Basta con aplicar leyes de Morgan.
    \end{proof}

    \begin{exa}
        Considere $(\mathbb{R},\tau_u)$ y, para $n\in\mathbb{N}$ definimos $A_n=(-\frac{1}{n},\frac{1}{n})$, es claro que cada uno de estos conjuntos es abierto. Sea $B_n=\mathbb{R}-A_n=(-\infty,-\frac{1}{n}]\cup[\frac{1}{n},\infty)$.

        Se tiene que:
        \begin{equation*}
            \bigcup_{n\in\mathbb{N}}B_n=\bigcup_{n\in\mathbb{N}}\mathbb{R}-A_n=\mathbb{R}-\bigcap_{n\in\mathbb{N}}A_n=\mathbb{R}-\left\{0\right\}
        \end{equation*}
        el cual es abierto. Por tanto, la unión arbitraria de cerrados no es cerrada (en general).
    \end{exa}

    \begin{mydef}
        Sea $(X,\tau)$ un espacio topológico y, sean $x\in X$ y $V\subseteq X$ tal que $x\in V$. Se dice que $V$ es una \textbf{vecindad de $x$} si existe $U\in\tau$ abierto tal que $x\in U$ y $U\subseteq V$.
        \begin{enumerate}
            \item Si $V$ es una vecindad de $x$ y $V\in\tau$, decimos que $V$ es una \textbf{vecindad abierta de $x$}.
            \item Si $V$ es una vecindad de $x$ y $X-V\in\tau$, decimos que $V$ es una \textbf{vecindad cerrada de $x$}.
        \end{enumerate}
        Al conjunto de todas las vecindades del punto $x$ lo denotamos por $\mathcal{V}(x)$. Tenemos que $X\in\mathcal{V}(x)$ para todo $x\in X$.
    \end{mydef}

    \begin{mydef}
        Se define el conjunto $\natint{1,n}$ llamado \textbf{intervalo natural de $1$ a $n$} como el conjunto $\left\{1,2,...,n\right\}$.
    \end{mydef}
        
    \begin{excer}
        Sea $(X,\tau)$ un espacio topológico.
        \begin{enumerate}
            \item Si $V_1,...,V_n\in\mathcal{V}(x)$ para $x\in X$, entonces $V_1\cap...\cap V_n\in\mathcal{V}(x)$.
            \item Si $\left\{V_\alpha\right\}_{\alpha\in I}\subseteq\mathcal{V}(x)$ para $x\in X$, entonces $\bigcup_{\alpha\in I}V_\alpha\in\mathcal{V}(x)$.
        \end{enumerate}
    \end{excer}

    \begin{sol}
        Probaremos ambos incisos:
        
        De (1): Como $x\in V_i$ para $i\in\natint{1,n}$, entonces existen $U_1,...,U_n$ abiertos en $X$ tales que $x\in U_i\subseteq V_i$ para todo $i\in\natint{1,n}$, luego $x\in\bigcap_{i=1}^n U_i\subseteq V_1\cap...\cap V_n$ donde el primer conjunto es abierto, luego $V_1\cap...\cap V_n\in\V{x}$.

        De (2): Es inmediato.
    \end{sol}

    \begin{mydef}
        Sea $(X,\tau)$ un espacio topológico y $A\subseteq X$.
        \begin{enumerate}
            \item Sea $x\in X$. $x$ es un \textbf{punto de acumulación de $A$} si para todo $U$ abierto que contiene a $x$ se tiene que $(U-\left\{x\right\})\cap A\neq\emptyset$ ($U$ contiene un punto de $A$ diferente de $x$). Al conjunto de todos los puntos de acumulación lo llamaremos el \textbf{conjunto derivado de $A$}, y se denota por $A'$.
            \item Un elemento $a\in A$ es un \textbf{punto interior} de $A$, si $A$ es una vecindad de $x$ (es decir, $A\in\V{x}$). 
            \textbf{El interior de $A$} es el conjunto de todos los puntos interiores de $A$ y se escribe $\Int{A}$. Es claro que $\Int{A}\subseteq A$.
            \item Sea
            \begin{equation*}
                \mathcal{C}=\left\{C\subseteq X\big|X-C\in\tau,A\subseteq C \right\}
            \end{equation*}
            es claro que $\mathcal{C}$ es no vacía, pues $X\in\mathcal{C}$. La \textbf{cerradura de $A$} es el conjunto $\bigcap_{C\in\mathcal{C}}C$ y se denota por $\overline{A}$. Si $x\in\overline{A}$, diremos que \textbf{$x$ es un punto adherente de $A$}. Es claro que $A\subseteq\overline{A}$.
            \item La \textbf{frontera de $A$} es el conjunto $\Cls{A}\cap\Cls{X-A}$ y se denota por $\Fr{A}$.
        \end{enumerate}
    \end{mydef}

    \begin{propo}
        Sea $(X,\tau)$ un espacio topológico, $x\in X$ y sean $A,B\subseteq X$. Entonces:
        \begin{enumerate}
            \item $\Int{A}\subseteq A\subseteq\Cls{A}$.
            \item $\Int{A}=\bigcup\left\{U\in\tau\big|U\subseteq A \right\}=\bigcup\mathcal{A}$.
            \item $\Int{A}\in\tau$.
            \item Si $V\in\tau$ tal que $V\subseteq A$, entonces $V\subseteq\Int{A}$.
            \item $A$ es abierto si y sólo si $\Int{A}=A$.
            \item $\Int{\Int{A}}=\Int{A}$.
            \item $\Int{\widehat{A\cap B}}=\Int{A}\cap\Int{B}$.
            \item $\Int{A}\cup\Int{B}\subseteq \Int{\widehat{A\cup B}}$.
            \item $\Cls{A}$ es un conjunto cerrado.
            \item Si $K\subseteq X$ es cerrado de $(X,\tau)$ y $A\subseteq K$, entonces $\Cls{A}\subseteq K$.
            \item $A$ es cerrado si y sólo si $\Cls{A}=A$.
            \item $\Cls{\Cls{A}}=\Cls{A}$.
            \item $\Cls{A\cup B}=\Cls{A}\cup\Cls{B}$.
            \item $\Cls{A\cap B}\subseteq\Cls{A}\cap\Cls{B}$.
            \item $\emptyset=\Int{\emptyset}=\Cls{\emptyset}$ y $X=\Int{X}=\Cls{X}$.
            \item Si $A\subseteq B$, entonces $\Int{A}\subseteq \Int{B}$ y $\Cls{A}\subseteq\Cls{B}$.
            \item $x\in\Cls{A}$ si y sólo si para todo abierto $U\subseteq X$ tal que $x\in U$ se tiene que $U\cap A\neq\emptyset$.
            \item $x\in\Fr{A}$ si y sólo si para todo abierto $U$ tal que $x\in U$ se cumple que $U\cap A\neq\emptyset$ y $U\cap (A-X)\neq\emptyset$.
            \item $\Cls{A}=A\cup A'$.
            \item $A$ es un conjunto cerrado si y sólo si $A'\subseteq A$.
            \item $\Cls{A}=\Int{A}\cup\Fr{A}$.
            \item $\Fr{A}=\Fr{X-A}$.
            \item $\Cls{A}-\Fr{A}=\Int{A}$.
        \end{enumerate}
    \end{propo}

    \begin{proof}
        Se probarán todos los incisos.

        De (1): Si $x\in\Int{A}$, entonces $A\in\V{x}$, luego $x\in A$. Por tanto, $\Int{A}\subseteq A$. Ahora, es claro que $A\subseteq \Cls{A}$, pues de la definción de cerradura de $A$, todos los elementos de la intersección en esta definición contienen a $A$, luego $A$ está contenida en la intersección.

        De (2): Veamos que se tienen las dos contenciones:
        \begin{itemize}
            \item $\Int{A}\subseteq\bigcup\mathcal{A}$. Sea $x\in \Int{A}$, entonces $A\in\V{x}$, por lo cual existe un abierto $U\in\tau$ tal que $x\in U\subseteq A$, luego $U\in\mathcal{A}$, es decir que $x\in \bigcup\mathcal{A}$.
            \item $\bigcup\mathcal{A}\subseteq\Int{A}$. Sea $x\in \bigcup\mathcal{A}$, entonces existe $U\in\tau$ con $U\subseteq A$ tal que $x\in U$, por lo cual $A\in\V{x}$, luego $x\in\Int{A}$.
        \end{itemize}
        por los dos incisos anteriores, se sigue que $\Int{A}=\bigcup\mathcal{A}$, es decir que el interior de $A$ es el conjunto abierto más grande contenido en $A$.

        De (3): Es inmediato de (2).

        De (4): Es inmediato de (2).

        De (5): Supongamos que $A$ es abierto, entonces $A\in\tau$. Además, $A\subseteq A$, por lo cual de (4) se sigue que $A\subseteq\Int{A}$. Ya se tiene que $\Int{A}\subseteq A$, por tanto $A=\Int{A}$.

        La recíproca es inmediata.

        De (6): Por (3), se tiene que $\Int{A}$ es abierto, luego por (5) se sigue que $\Int{\Int{A}}=\Int{A}$.

        De (7): Probaremos las dos contenciones:
        \begin{itemize}
            \item $\Int{\widehat{A\cap B}}\subseteq\Int{A}\cap\Int{B}$. Si $x\in\Int{\widehat{A\cap B}}\subseteq A\cap B$, entonces existe $U\in\tau$ tal que $x\in U\subseteq A\cap B$, en particular $x\in U\subseteq A$ y $x\in U\subseteq B$, luego $x\in \Int{A}$ y $x\in \Int{B}\Rightarrow x\in \Int{A}\cap\Int{B}$. Por tanto, $\Int{\widehat{A\cap B}}\subseteq\Int{A}\cap\Int{B}$.
            \item $\Int{A}\cap\Int{B}\subseteq\Int{\widehat{A\cap B}}$. El conjunto $\Int{A}\cap\Int{B}\in\tau$ y $\Int{A}\cap\Int{B}\subseteq A\cap B$. Por (4), se sigue que $\Int{A}\cap\Int{B}\subseteq\Int{\widehat{A\cap B}}$.
        \end{itemize}
        de los dos incisos anteriores, se sigue que $\Int{\widehat{A\cap B}}=\Int{A}\cap\Int{B}$.

        De (8): Se tiene que $\Int{A}\cup\Int{B}\in\tau$ es tal que $\Int{A}\cup\Int{B}\subseteq A\cup B$, luego por (4) se sigue que $\Int{A}\cup\Int{B}\subseteq\Int{\widehat{A\cup B}}$.

        De (9): Es inmediato de la definición de $\Cls{A}$, pues este conjunto es intersección arbitraria de cerrados.

        De (10): Es inmediato de la definición de $\Cls{A}$. Esto significa que la cerradura de un conjunto es el cerrado más pequeño que contiene a $A$.

        De (11): Suponga que $A$ es cerrado, entonces como $A\subseteq A$, se tiene por (10) que $\Cls{A}\subseteq A$. Luego, como $A\subseteq \Cls{A}$ por (1), se sigue que $A=\Cls{A}$.
        
        La recíproca es inmediata de (9).

        De (12): Por (9), $\Cls{A}$ es cerrado, luego por (11) se tiene que $\Cls{A}=\Cls{\Cls{A}}$.

        De (13): Proaremos las dos contenciones:
        \begin{itemize}
            \item $\Cls{A\cup B}\subseteq\Cls{A}\cup\Cls{B}$. El conjunto $\Cls{A}\cup\Cls{B}$ es un cerrado que contiene a $A\cup B$, por tanto del inciso (10) se tiene que $\Cls{A\cup B}\subseteq\Cls{A}\cup\Cls{B}$.
            \item Como $A, B\subseteq A\cup B$, entonces $A,B\subseteq \Cls{A\cup B}$, luego por (10) se tiene que $\Cls{A},\Cls{B}\subseteq \Cls{A\cup B}$. Por tanto, $\Cls{A}\cup\Cls{B}\subseteq\Cls{A\cup B}$.
        \end{itemize}
        de los dos incisos anteriores, se sigue que $\Cls{A\cup B}=\Cls{A}\cup\Cls{B}$.

        De (14): Como $A\subseteq\Cls{A}$ y $B\subseteq\Cls{B}$, entonces $A\cap B\subseteq\Cls{A}\cap\Cls{B}$. Por (10), se sigue que $\Cls{A\cap B}\subseteq\Cls{A}\cap\Cls{B}$.

        De (15): Se dividirá en dos partes:
        \begin{itemize}
            \item $\emptyset=\Int{\emptyset}=\Cls{\emptyset}$. Como $\emptyset\subseteq\Int{\emptyset}\subseteq\emptyset$, entonces $\emptyset=\Int{\emptyset}$. Ahora, como $\emptyset$ es un cerrado que contiene a $\emptyset$, se sigue por (10) que $\Cls{\emptyset}\subseteq\emptyset$. Por ende, $\Cls{\emptyset}=\emptyset$.
            \item Para $X$ el caso es casi análogo a $\emptyset$ (al final todo esto resulta más en un juego de palabras que en otra cosa).
        \end{itemize}

        De (16). Como $A\subseteq B$, entonces $\Int{A}\subseteq B$, y $A\subseteq\Cls{B}$, por (4) y (10), se debe tener que $\Int{A}\subseteq\Int{B}$ y $\Cls{A}\subseteq\Cls{B}$.

        De (17): Sea $x\in X$:
        
        $\Rightarrow$): Suponga que $x\in\Cls{A}$, entonces para todo $C\subseteq X$ cerrado tal que $A\subseteq C$, $x\in C$. Suponga que existe $U_0\in\tau$ abierto tal que $x\in U_0$ y $U_0\cap A=\emptyset$. Entonces $A\subseteq X-U_0$ es un cerrado que contiene a $A$, luego $x\in X-U_0$, es decir $x\notin U_0$\contradiction. Por tanto, $U\cap A\neq\emptyset$ para todo $U\in\tau$ tal que $x\in U$.

        $\Leftarrow$): Sea $L\subseteq X$ un cerrado tal que $A\subseteq L$. Probaremos que $x\in L$, suponiendo la tesis para este $x\in X$. Suponga que $x\notin L$, entonces $x\in X-L$ el cual es abierto, por tanto $(X-L)\cap A\neq\emptyset$, es decir $A\nsubseteq L$\contradiction. Por tanto, $x\in L$.
        
        De (18): Es inmediato de la definición de $\Fr{A}=\Cls{A}\cap\Cls{X-A}$ y del inciso (17).

        De (19): Se probarán las dos contenciones:
        \begin{itemize}
            \item $\Cls{A}\subseteq A\cup A'$. Sea $x\in\Cls{A}$. Si $x\in A$, se tiene el resultado. Suponga que $x\notin A$. Como $x\in\Cls{A}$, por (17) para todo abierto $U\subseteq X$ se tiene que $U\cap A\neq\emptyset$, pero $x\notin A$, por lo cual $(U-\left\{x\right\})\cap A\neq\emptyset$. Por tanto, $x\in A'$.
            \item $A\cup A'\subseteq\Cls{A}$. Es inmediata de la definición de $\Cls{A}$ y $A'$.
        \end{itemize}
        por los dos incisos anteriores se sigue que $\Cls{A}=A\cup A'$.

        De (20): Suponga que $A$ es cerrado, entonces por (11), $\Cls{A}=A$, luego por (19) se tiene que $A\cup A'=\Cls{A}=A$, es decir que $A'\subseteq A$.
        
        Si $A'\subseteq A$, entonces $A=A\cup A'=\Cls{A}$ por (11), luego $A=\Cls{A}$, es decir que $A$ es cerrado.
        
        De (21): Es claro que $\Int{A}\cap\Fr{A}\subseteq\Cls{A}$, pues $\Int{A},\Fr{A}\subseteq\Cls{A}$. Ahora, si $x\in\Cls{A}$ sea $U\subseteq X$ tal que $x\in U$. Se tienen dos casos:
        \begin{itemize}
            \item $U\subseteq A$, en este caso se sigue de la definición que $x\in\Int{A}$.
            \item $U\nsubseteq A$, entonces existe $y\in U$ tal que $y\notin A$, es decir que $U\cap(X-A)\neq\emptyset$. Como $x\in \Cls{A}$, entonces $U\cap A\neq\emptyset$. Por ser el $U$ arbitrario, se sigue por (18) que $x\in\Fr{A}$.
        \end{itemize}
        es decir que $x\in\Int{A}\cup\Fr{A}$. Por tanto, $\Cls{A}\subseteq\Int{A}\cup\Fr{A}$. Así, $\Cls{A}=\Int{A}\cup\Fr{A}$.

        De (22): Veamos que $A=X-(X-A)$, por lo cual:
        \begin{equation*}
            \Fr{A}=\Cls{A}\cap\Cls{X-A}=\Cls{X-A}\cap\Cls{A}=\Cls{X-A}\cap\Cls{X-(X-A)}=\Fr{X-A}
        \end{equation*}

        De (23): Observemos que: $x\in\Cls{A}-\Fr{A}$ si y sólo si se cumple que
        \begin{itemize}
            \item Para todo $U$ abierto tal que $x\in U$ se cumple que $U\cap A\neq\emptyset$.
            \item Existe $U_0$ abierto tal que $x\in U_0$ y, $U_0\cap A=\emptyset$ o $U_0\cap (X-A)=\emptyset$.
        \end{itemize}
        Por ambas condiciones, debe suceder que $U_0\cap A\neq\emptyset$ y $U_0\cap(X-A)=\emptyset$, es decir que $U_0 \subseteq A$, esto es que $x\in\Int{A}$. Por tanto, $x\in \Cls{A}-\Fr{A}$ si y sólo si $x\in\Int{A}$. Luego se tiene la igualdad.
    \end{proof}

    \renewcommand{\theenumi}{\arabic{enumi}}

    \begin{propo}
        Sea $(X,\tau)$ un espacio topológico y $\left\{A_\alpha \right\}_{\alpha\in I}\subseteq\mathcal{P}(X)$.
        \begin{enumerate}
            \item $\bigcup_{\alpha\in I}\Cls{A_\alpha}\subseteq\Cls{\bigcup_{\alpha\in I}A_\alpha}$.
            \item $\Cls{\bigcap_{\alpha\in I}A_\alpha}\subseteq\bigcap_{\alpha\in I}\Cls{A_\alpha}$. 
        \end{enumerate}
    \end{propo}
    
    \begin{proof}
        Probemos ambos incisos:
        
        De (1): Si $x\in \bigcup_{\alpha\in I}\Cls{A_\alpha}$, sea $U\in\tau$ tal que $x\in U$, luego $\exists \alpha\in I$ tal que $x\in \Cls{A_\alpha}$, por ende $U\cap A_\alpha\neq\emptyset$, por tanto $U\cap\left(\bigcup_{\alpha\in I}A_\alpha\right)\cap \neq\emptyset$. Como el $U\in\tau$ fue arbitrario, se sigue que $x\in\Cls{\bigcup_{\alpha\in I}A_\alpha}$.

        De (2): Si $x\in \Cls{\bigcap_{\alpha\in I}A_\alpha}$, entonces para $U\in\tau$ tal que $x\in U$ se cumple que $\left(\bigcap_{\alpha\in I}A_\alpha\right)\cap U\neq\emptyset\Rightarrow\bigcap_{\alpha\in I}\left(A_\alpha\cap U \right)\neq\emptyset$, es decir que $U\cap A_\alpha\neq\emptyset$, para todo $\alpha\in I$, luego como $U\in\tau$ fue arbitrario, se sigue que $x\in \Cls{A_\alpha}$ para todo $\alpha\in I$. Así $x\in\bigcap_{\alpha\in I}\Cls{A_\alpha}$. 
    \end{proof}

    \begin{exa}
        Considere al espacio topológico $(\mathbb{R},\tau_u)$. Tomemos
        \begin{enumerate}
            \item $A=]0,1]\cup\left\{9\right\}$. Tenemos que $\Cls{A}=[0,1]\cup\left\{9\right\}$, $A'=[0,1]$, por lo cual no podemos relacionar (al menos de forma directa) a $A$ junto con su $A'$ (esto es, uno no está contenido dentro del otro).
            \item Sea $n\in\mathbb{N}$, defina $A_n=[\frac{1}{n},1]$. Se tiene que
            \begin{equation*}
                \bigcup_{n\in\mathbb{N}}\Cls{A_n}=]0,1]
            \end{equation*}
            y,
            \begin{equation*}
                \bigcup_{n\in\mathbb{N}}A_n=]0,1]\Rightarrow \Cls{\bigcup_{n\in\mathbb{N}}A_n}=[0,1]
            \end{equation*}
            es decir que $\Cls{\bigcup_{n\in\mathbb{N}}A_n}\nsubseteq\bigcup_{n\in\mathbb{N}}\Cls{A_n}$.
            \item Considere $X=\left\{a,b\right\}$, tomemos al espacio topológico $(X,\tau=\left\{X,\emptyset,\left\{a\right\} \right\})$. Si $A=\left\{a\right\}$ y $b=\left\{b\right\}$, entonces $\Int{A}=\left\{a\right\}$, $\Int{B}=\emptyset$, $\Cls{A}=X$, $\Cls{B}=B$. Luego $X=\Int{\widehat{A\cup B}}\nsubseteq \Int{A}\cup\Int{B}=A$.
            
            Además $A\cap B=\emptyset\Rightarrow\Cls{A\cap B}=\emptyset$. Por ende, $B=\Cls{A}\cap \Cls{B}\nsubseteq\Cls{A\cap B}$.
        \end{enumerate}
    \end{exa}

    \begin{mydef}
        Para $x\in\mathbb{R}$, se define el \textbf{suelo de $x$} (denotado por $\floor{x}$) como el máximo entero tal que $\floor{x}\leq x$.
    \end{mydef}

    \begin{excer}
        Considere $(\mathbb{R},\tau_u)$. Encuentre $\Int{\mathbb{N}}$, $\Cls{\mathbb{N}}$, $\mathbb{N}'$, $\Fr{\mathbb{N}}$.
    \end{excer}

    \begin{sol}
        Hagamos cada uno de los incisos:
        \begin{enumerate}
            \item $\Int{\mathbb{N}}$) Afirmamos que $\Int{\mathbb{N}}=\emptyset$. En efecto, si fuese el caso contrario, existiría $U$ abierto no vacío en $(\mathbb{R},\tau_u)$ tal que $U\subseteq\mathbb{N}$, luego si $x\in U\cap\mathbb{N}\neq\emptyset$ (por ser $U$ no vacío), entonces existe $r>0$ tal que $]x-r,x+r[\subseteq U$.
            
            Sea $\delta=\min\left\{1,r\right\}>0$, entonces $]x-\delta,x+\delta[\subseteq U$, pero como $x\in\mathbb{N}$, no puede ser que $x+\frac{1}{2}\in\mathbb{N}$, lo cual contradice el hecho de que $U\subseteq\mathbb{N}$. Por tanto, $\Int{\mathbb{N}}=\emptyset$.
            \item $\Cls{\mathbb{N}}$) Probaremos que $\mathbb{N}$ es cerrado. Si $x\in\mathbb{R}-\mathbb{N}$, entonces existe $r=\min\left\{x-\floor{x},1-x+\floor{x}\right\}>0$ (pues $x\notin\mathbb{N}$, luego $\floor{x}<x$) tal que $]x-r,x+r[\subseteq\mathbb{R}-\mathbb{N}$. 
            
            En efecto, supongamos que $x-\floor{x}\leq1-x+\floor{x}$, entonces
            \begin{equation*}
                \begin{split}
                    ]x-r,x+r[\subseteq &]\floor{x},x+1-x+\floor{x}[ \\
                    \subseteq &]\floor{x},\floor{x}+1[ \\
                \end{split}
            \end{equation*}
            es decir, $]x-r,x+r[\subseteq\mathbb{R}-\mathbb{N}$. Si $x-\floor{x}\leq1-x+\floor{x}$ el caso es análogo. Por tanto, $\mathbb{R}-\mathbb{N}$ es abierto, luego $\mathbb{N}$ es cerrado y, por ende $\Cls{\mathbb{N}}=\mathbb{N}$.

            \item $\mathbb{N}'$) Afirmamos que el conjunto es vacío. Sea $x\in\mathbb{R}$. Se tienen dos casos:
            \begin{itemize}
                \item $x\in\mathbb{N}$) En este caso, existe $r=\frac{1}{2}>0$ tal que $(]x-r,x+r[-\left\{x\right\})\cap\mathbb{N}=\emptyset$.
                \item $x\notin\mathbb{N}$) En este caso, existe $r=\min\left\{x-\floor{x},1-x+\floor{x}\right\}>0$ tal que (como se vió en (2)) $]x-r,x+r[\cap\mathbb{N}=\emptyset$, en particular $(]x-r,x+r[-\left\{x\right\})\cap\mathbb{N}=\emptyset$.
            \end{itemize}
            Por ambos incisos, se sigue que $\mathbb{N}'=\emptyset$.

            \item $\Fr{\mathbb{N}}$) Afirmamos que $\Fr{\mathbb{N}}=\mathbb{N}$. En efecto, ya se sabe que $\mathbb{N}=\Cls{\mathbb{N}}$. Probaremos que $\mathbb{N}\subseteq\Cls{\mathbb{R}-\mathbb{N}}$ y con ello se tendría el resultado. 
            
            Sea $x\in\mathbb{N}$, entonces si $U$ es un abierto tal que $x\in U$, entonces existe $r>0$ tal que $]x-r,x+r[\subseteq U$, luego si $\delta=\min\left\{1,r\right\}>0$, se tiene que el elemento $x+\frac{\delta}{2}\in\mathbb{R}-\mathbb{N}$, es decir que $U\cap(\mathbb{R}-\mathbb{N})\neq\emptyset$. Por tanto, $x\in\Cls{\mathbb{R}-\mathbb{N}}$, así $\mathbb{N}\subseteq\Cls{\mathbb{R}-\mathbb{N}}$.
        \end{enumerate}
    \end{sol}

    \begin{excer}
        Considere $(\mathbb{R},\tau_I)$, $(\mathbb{R},\tau_D)$ y $(\mathbb{R},\tau_{cf})$. Encuentre en cada uno de los espacios anteriores $\Int{\mathbb{Z}}$, $\Cls{\mathbb{Z}}$, $\mathbb{Z}'$, $\Fr{\mathbb{Z}}$.
    \end{excer}

    \begin{sol}
        Consideremos cada una de las topologías por separado.
        \begin{enumerate}
            \item En $(\mathbb{R},\tau_I)$:
            \begin{itemize}
                \item $\Int{\mathbb{Z}}$) Afirmamos que $\Int{\mathbb{Z}}=\emptyset$. En efecto, como $\tau_I=\left\{\emptyset,\mathbb{R} \right\}$, el único abierto contenido en $\mathbb{Z}$ es $\emptyset$, luego $\Int{\mathbb{Z}}=\emptyset$.
                \item $\Cls{\mathbb{Z}}$) Como el único cerrado que contiene a $\mathbb{Z}$ es $\mathbb{R}$, se sigue que $\Cls{\mathbb{Z}}=\mathbb{R}$.
                \item $\mathbb{Z}'$) Sea $x\in\mathbb{R}$, si $U$ es un abierto tal que $x\in U$, entonces debe suceder que $U=\mathbb{R}$, luego $(U-\left\{x\right\})\cap\mathbb{Z}\neq\emptyset$. Por tanto $x\in\mathbb{Z}'$. Así, $\mathbb{R}=\mathbb{Z}'$.
                \item $\Fr{\mathbb{Z}}$) Como $\Cls{\mathbb{R}-\mathbb{Z}}=\mathbb{R}$, entonces se tiene que $\Fr{\mathbb{Z}}=\mathbb{R}$.
            \end{itemize}
            \item En $(\mathbb{R},\tau_D)$:
            \begin{itemize}
                \item $\Int{\mathbb{Z}}$) Es claro que $\Int{\mathbb{Z}}=\mathbb{Z}$, pues en la topología discreta todo subconjunto de $\mathbb{R}$ es abierto.
                \item $\Cls{\mathbb{Z}}$) Es claro que $\Cls{\mathbb{Z}}=\mathbb{Z}$, pues en la topología discreta todo subconjunto de $\mathbb{R}$ es cerrado.
                \item $\mathbb{Z}'$) Afirmamos que $\mathbb{Z}'=\emptyset$. En efecto, si $x\in\mathbb{Z}$, entonces $\left\{x\right\}$ es un abierto en $\mathbb{R}$ tal que $x\in\left\{x\right\}$, y se cumple que $(\left\{x\right\}-\left\{x\right\})\cap \mathbb{Z}=\emptyset$. Luego $\mathbb{Z}'=\emptyset$.
                \item $\Fr{\mathbb{Z}}$) Como $\mathbb{R}-\mathbb{Z}$ es cerrado, por un inciso anterior se sigue que $\Fr{\mathbb{Z}}=\emptyset$.
            \end{itemize}
            \item En $(\mathbb{R},\tau_{cf})$:
            \begin{itemize}
                \item $\Int{\mathbb{Z}}$) Sea $U\subseteq \mathbb{Z}$ abierto, es decir que $\mathbb{R}-U$ es finito o $U=\emptyset$. Se tiene entonces que:
                \begin{equation*}
                    \mathbb{R}-\mathbb{Z}\subseteq\mathbb{R}-U
                \end{equation*}
                como $\mathbb{R}-\mathbb{Z}$ es infinito, entonces por Cantor-Bernstein debe suceder que $\mathbb{R}-U$ también sea infinito. Por tanto, $U=\emptyset$. Luego entonces $\Int{\mathbb{Z}}=\emptyset$.
                \item $\Cls{\mathbb{Z}}$) Sea $C\subseteq\mathbb{R}$ un cerrado tal que $\mathbb{Z}\subseteq C$. Como en $\tau_{cf}$ los cerrados son todos los subconjuntos finitos o $\mathbb{R}$, entonces al ser $\mathbb{Z}$ infinito no puede ser que $C$ sea finito, luego $C=\mathbb{R}$. Por ende, $\Cls{\mathbb{Z}}=\mathbb{R}$.
                \item $\mathbb{Z}'$) Sea $x\in\mathbb{R}$, afirmamos que $x\in\mathbb{Z}'$. En efecto, si $U\subseteq\mathbb{R}$ es abierto tal que $x\in U$, entonces $\mathbb{R}-U$ es finito, luego como $\mathbb{Z}$ es infinito, existe $z\in\mathbb{Z}$ tal que $z< y$, para todo $y\in \mathbb{R}-U$ y $x<y$. Es decir que $y\in(U-\left\{x\right\})\cap\mathbb{Z}$. Por tanto, $(U-\left\{x\right\})\cap\mathbb{Z}\neq\emptyset$, es decir que $x\in\mathbb{Z}'$.
                \item $\Fr{\mathbb{Z}}$) Computemos $\Cls{\mathbb{R}-\mathbb{Z}}$. Sea $C\subseteq\mathbb{R}$ cerrado tal que $\mathbb{R}-\mathbb{Z}\subseteq C$, entonces como $C$ es cerrado, $C$ es finito o $C=\mathbb{R}$, pero $C$ no puede ser finito ya que $\mathbb{R}-\mathbb{Z}$ es infinito, luego $C=\mathbb{R}$. Así, $\Cls{\mathbb{R}-\mathbb{Z}}=\mathbb{R}$. Por tanto, $\Fr{\mathbb{Z}}=\mathbb{Z}$.
            \end{itemize}
        \end{enumerate}
    \end{sol}

    \begin{mydef}
        Sea $(X,\tau)$ un espacio topológico. Se dice que el espacio $(X,\tau)$ es de \textbf{Hausdorff} si para todo $x_1,x_2\in X$ distintos existen $U_1,U_2\in\tau$ tales que $x_1\in U_1$, $x_2\in U_2$ y $U_1\cap U_2=\emptyset$.
    \end{mydef}

    \begin{exa}
        Considere $(X,\tau)$ donde $X=\left\{1,2\right\}$ y $\tau=\left\{X,\emptyset, \left\{1\right\} \right\}$, entonces $(X,\tau)$ no es de Hausdorff.
    \end{exa}

    \begin{exa}
        $(\mathbb{R},\tau_I)$ no es de Hausdorff (cuando el espacio tiene más de un elemento esto se sigue cumpliendo).
    \end{exa}

    \begin{exa}
        Sea $(X,d)$ un espacio métrico y consideremos al espacio topológico $(X,\tau_d)$. Este espacio es de Hausdorff.
    \end{exa}

    \begin{exa}
        Sea $(X,d)$ es espacio métrico tal que la métrica de él está definida como:
        \begin{equation*}
            d(x,y)=\left\{
            \begin{array}{lr}
                0 & \textup{ si }x\neq y\\
                1 & \textup{ si }x = y\\
            \end{array}
            \right.
        \end{equation*}
        dado $p\in X$ considere $B_d(p,1)=\left\{p\right\}$. Entonces para todo $p\in X$, $\left\{p\right\}\in\tau_D\Rightarrow \forall A\subseteq X$, $A\in\tau_d$, es decir que $\tau_d=\tau_D$. 
    \end{exa}

    \begin{mydef}
        Un espacio topológico $(X,\tau)$ se dice \textbf{metrizable} si existe una métrica $\cf{d}{X\times X}{\mathbb{R}}$ tal que $\tau_d=\tau$. 
    \end{mydef}

    \begin{propo}
        Si $(X,\tau)$ es un espacio metrizable, entonces $(X,\tau)$ es un espacio de Hausdorff.
    \end{propo}

    \begin{proof}
        Es inmediata de la definición de espacio metrizable y del ejemplo 1.1.10.
    \end{proof}

    \begin{exa}
        Considere $X=\left\{1,2\right\}$, si tomamos al espacio topológico $(X,\tau=\left\{X,\emptyset,\left\{2\right\} \right\})$ obtenemos que este espacio no es metrizable por no ser de Hausdorff.
    \end{exa}

    \begin{exa}
        Considere $(X,\tau_D)$. Este espacio es metrizable tomando la métrica discreta.
    \end{exa}

    \section{Bases de una topología}

    \begin{mydef}
        Sea $(X,\tau)$ un espacio topológico. Una subcolección $\mathcal{B}$ de $\tau$ es una \textbf{base para la toplogía $\tau$} si todo $U\in \tau$ puede escribirse como unión arbitraria de elementos de $\mathcal{B}$.

        Si $\mathcal{B}$ es una base para $\tau$, a sus elementos los llamaremos \textbf{básicos}.
    \end{mydef}

    \begin{obs}
        Cualquier topología es una base para sí misma.
    \end{obs}

    Considere al espacio topolóigco $(X,\tau)$. Una base $\mathcal{B}$ de $\tau$ cumple que:
    \begin{enumerate}
        \item $\mathcal{B}\subseteq\tau$.
        \item Si $U\in\tau$ entonces existe $\left\{U_\alpha\right\}_{\alpha\in I}\subseteq\mathcal{B}$ tal que $U=\bigcup_{\alpha\in A}U_\alpha$.
    \end{enumerate}

    ¿Qué pretendemos con esta definición?

    Básicamente lo que se pretende es descibir a todos los elementos de la topología mediante un conjunto más pequeño de elementos (esto permite que sea más fácil de manejar y que las propiedades deseadas para los elementos de la topología se sigan cumpliendo).

    \begin{exa}
        Sea $X$ un conjunto y sea $\mathcal{M}=\left\{\left\{p\right\}\big| p\in X \right\}$. Esta es una base para $\tau_D$ definida sobre $X$.
    \end{exa}

    \begin{exa}
        Sea $(X,d)$ un espacio métrico y sea $\tau_d$ la topología generada por la métrica $d$. Entonces, las colecciones:
        \begin{equation*}
            \mathcal{B}_1 =\left\{B_d(x,r)\big| x\in X, r\in\mathbb{R}^+ \right\}
        \end{equation*}
        es una base para la topología generada por $\tau_d$. Además,
        \begin{equation*}
            \mathcal{B}_2=\left\{B_d(x,q)\big| x\in X, q\in\mathbb{Q}^+ \right\}
        \end{equation*}
        es otra base. Más aún:
        \begin{equation*}
            \mathcal{B}_3=\left\{B_d\left(x,\frac{1}{n}\right) \big| x\in X, n\in\mathbb{N} \right\}
        \end{equation*}
        es otra base.
    \end{exa}

    \begin{exa}
        Sea $X=\left\{a,b,c,d\right\}$ y $\kappa=\left\{\left\{a,b,c\right\}, \left\{c,d\right\} \right\}$. Afirmamos que no existe una topología definida sobre $X$ tal que $\kappa$ sea base de ella.

        En efecto, suponga que $\tau$ es una topología sobre $X$ y $\kappa$ es una base para $\tau$, entonces $\left\{a,b,c\right\}$ y $\left\{c,d\right\}$ están en $\tau$, luego su intersección $\left\{c\right\}\in\tau$. Pero, $\left\{c\right\}$ no puede ser escrito como unión de elementos de $\kappa$.
    \end{exa}

    \begin{propo}
        Sea $(X,\tau)$ un espacio topológico y sea $\mathcal{B}\subseteq\tau$. Entonces, $\mathcal{B}$ es una base para la topología $\tau$ si y sólo si dados $U\in \tau$ y $u\in U$ existe $B\in\mathcal{B}$ tal que $u\in B\subseteq U$.
    \end{propo}

    \begin{proof}
        Probaremos la doble implicación.

        $\Rightarrow$): Suponga que $\mathcal{B}$ es una base para la topología $\tau$. Sea $U\in \tau$ y $u\in U$. Como $\mathcal{B}$ es una base entonces exise una subcolección $\mathcal{C}\subseteq\mathcal{B}$ tal que $U=\bigcup\mathcal{C}$, luego existe $C_\alpha\in\mathcal{C}$ tal que $u\in C_\alpha$. Por ende $u\in C_\alpha\subseteq U$. Tomando $B=C_\alpha\in\mathcal{B}$ se tiene el resultado.

        $\Leftarrow)$: Suponga que se cumple la tesis. Ya se tiene que $\mathcal{B}\subseteq\tau$. Sea entonces $U\in\tau$. Para cada $x\in U$ existe $B_x\in\mathcal{B}$ tal que $x\in B_x\subseteq U$, luego la colección:
        \begin{equation*}
            \left\{U_x\in\mathcal{B}\big| x\in U \right\}
        \end{equation*}
        es una subcolección de $\mathcal{B}$ tal que $\bigcup_{x\in U}U_x=U$. Por tanto, $\mathcal{B}$ es una base de $\tau$.
    \end{proof}

    \begin{cor}
        Sea $(X,\tau)$ un espacio topológico y sea $\mathcal{B}$ una base de la topología $\tau$. Sea $U\subseteq X$, entonces $U$ es abierto en $\tau$ si y sólo si dados $x\in U$ existe $B\in\mathcal{B}$ tal que $x\in B\subseteq U$.
    \end{cor}

    \begin{proof}
        Es inmediato de la proposición anterior.
    \end{proof}

    \begin{cor}
        Sea $X$ un conjunto y sean $\tau_1,\tau_2$ dos topologías definidas sobre $X$. Tomemos $\mathcal{B}_1,\mathcal{B}_2$ bases para $\tau_1,\tau_2$, respectivamente, entonces los siguientes resultados son equivalentes:
        \begin{enumerate}
            \item $\tau_1\subseteq\tau_2$.
            \item Dados $x\in X$ y $B_1\in\mathcal{B}_1$ tal que $x\in B_1$ existe $B_2\in\mathcal{B}_2$ tal que $x\in B_2$ y $B_2\subseteq B_1$.
        \end{enumerate}
    \end{cor}

    \begin{proof}
        Probaremos la doble implicación.
        
        $1) \Rightarrow 2)$: Sean $x\in X$ y $B_1\in\mathcal{B_2}$ tal que $x\in B_1$. Como $\tau_1\subseteq\tau_2$, entonces en particular $B_1$ es abierto de $\tau_2$, luego existe $B_2\in\mathcal{B}_2$ tal que $x\in B_2\subseteq B_1$.

        $2) \Rightarrow 1)$: Sea $U\in\tau_1$, como $\mathcal{B}_1$ es base de $\tau_1$ entonces existe $\left\{B_\alpha \right\}_{\alpha\in I}$ subcolección de $\mathcal{B}_1$ tal que:
        \begin{equation*}
            \bigcup_{\alpha\in I}B_\alpha=U
        \end{equation*}
        Sea $u\in U$, entonces existe $\beta\in I$ tal que $u\in B_\beta$. Por (2) existe $C_\beta$ tal que $u\in C_\beta\subseteq B_\beta$. Formamos la subcolección $\left\{C_\alpha \right\}_{\alpha\in I}$, por lo anterior se sigue que:
        \begin{equation*}
            \begin{split}
                U=&\bigcup_{\alpha\in I}C_\alpha\\
                \subseteq&\bigcup_{\alpha\in I}B_\alpha\\
                =&U\\
                \Rightarrow U =&\bigcup_{\alpha\in I}C_\alpha\\
            \end{split}
        \end{equation*}
        donde $\bigcup_{\alpha\in I}C_\alpha\in\tau_2$ por ser $\mathcal{B}_2$ una base de $\tau_2$. Por tanto, $U\in\tau_2$. Finalmente, se sigue que $\tau_1\subseteq\tau_2$.

    \end{proof}

    \begin{cor}
        Sean $d_1$ y $d_2$ métricas definidas sobre el conjunto $X$. Consideremos $\tau_{d_1}$ y $\tau_{d_2}$. Entonces $\tau_{d_1}\subseteq \tau_{d_2}$ si y sólo si dado $x\in X$ y $\varepsilon\in\mathbb{R}^{+}$ existe $\delta\in\mathbb{R}^{+}$ tal que $B_{d_2}(x,\varepsilon)\subseteq B_{d_1}(x,\delta)$
    \end{cor}

    \begin{proof}
        Es inmediato del corolario anterior.
    \end{proof}

    \begin{cor}
        Sea $X$ un conjunto y sea $\mathcal{B}$ una colección de subconjuntos de $X$ tal que $\mathcal{B}$ es base para dos topologías $\tau_1$ y $\tau_2$ definidas sobre $X$. Entonces, $\tau_1=\tau_2$.
    \end{cor}

    \begin{proof}
        Es inmediato del corolario 1.2.2.
    \end{proof}

    \begin{cor}
        Sea $(X,\tau)$ un espacio topológico y sea $\mathcal{B}$ una base para $\tau$. Entonces, se cumple lo siguiente:
        \begin{enumerate}
            \item La intersección de dos elementos de $\mathcal{B}$ se puede escribir como una unión de elementos de $\mathcal{B}$.
            \item Existe $\left\{B_\alpha \right\}_{\alpha\in I}\subseteq\mathcal{B}$ tal que
            \begin{equation*}
                \bigcup_{\alpha\in I}B_\alpha=X
            \end{equation*}
        \end{enumerate} 
    \end{cor}

    \begin{proof}
        Es inmediato de la definición de base.
    \end{proof}

    ¿Es posible prescindir de un espacio topológico para definir lo que es una base?

    \begin{mydef}
        Sea $X$ un conjunto arbitrario y sea $\mathcal{A}$ una familia de subconjuntos de $X$, $\mathcal{A}$ se dice que es \textbf{una base para una topología sobre $X$} si cumple lo siguiente:
        \begin{enumerate}
            \item La intersección de dos elementos de $\mathcal{A}$ se puede escribir como una unión de elementos de $\mathcal{A}$.
            \item $X$ se puede escribir como una unión de elementos de $\mathcal{A}$.
        \end{enumerate}
    \end{mydef}

    \begin{propo}
        Sea $\mathcal{A}$ una base para una topología sobre el conjunto $X$. Entonces, la colección $\tau_{\mathcal{A}}$ dada por:
        \begin{equation*}
            \tau_{\mathcal{A}}=\left\{U\subseteq X\big| U\textup{ se puede escribir como una unión de elementos de }\mathcal{A} \right\}
        \end{equation*}
        es una topología sobre $X$ y $\mathcal{A}$ es una base para ella. La topología $\tau_{\mathcal{A}}$ es llamada \textbf{topología generada por $\mathcal{A}$}.
    \end{propo}

    \begin{proof}
        Veamos que $\tau_{\mathcal{A}}$ es una topología sobre $X$.
        \begin{enumerate}
            \item Es claro que $X\in\tau_{\mathcal{A}}$ y además $\emptyset\in\tau_{\mathcal{A}}$ ya que se puede ver como la unión de los elementos de la familia vacía.
            \item Sea $\mathcal{U}=\left\{U_\alpha\right\}_{\alpha\in I} \subseteq\mathcal{\tau_{\mathcal{A}}}$, entonces dado $\alpha\in I$ existe $\left\{A^\alpha_\beta\right\}_{\beta\in J_\alpha}\subseteq\mathcal{A}$ tal que
            \begin{equation*}
                U_\alpha=\bigcup_{\beta\in J_\alpha}A^\alpha_\beta
            \end{equation*}
            luego
            \begin{equation*}
                \begin{split}
                    \bigcup_{\alpha\in I}U_\alpha=&\bigcup_{\alpha\in I}\left(\bigcup_{\beta\in J_\alpha}A^\alpha_\beta \right)\in\tau_{\mathcal{A}} \\
                \end{split}
            \end{equation*}
            donde la unión está en $\tau_{\mathcal{A}}$ por definición de la misma.
            \item Sean $U,V\in\tau_{\mathcal{A}}$, entonces existen $\left\{A_\alpha \right\}_{\alpha\in I}$ y $\left\{B_\beta\right\}_{\beta\in J}$ subcolecciones de $\mathcal{A}$ tales que:
            \begin{equation*}
                U=\bigcup_{\alpha\in I}A_\alpha\quad \textup{y}\quad V=\bigcup_{\beta\in J}B_\beta
            \end{equation*}
            se tiene que:
            \begin{equation*}
                \begin{split}
                    U\cap V =& \left(\bigcup_{\alpha\in I}A_\alpha\right) \cap \left(\bigcup_{\beta\in J}B_\beta\right)\\
                    =& \left(\bigcup_{\alpha\in I\textup{ y }\beta\in J}A_\alpha\cap B_\beta\right)\\
                    =& \left(\bigcup_{(\alpha,\beta)\in I\times J}A_\alpha\cap B_\beta\right)\\
                \end{split}
            \end{equation*}
            sabemos que para $\alpha\in I$ y $\beta\in J$, el conjunto $A_\alpha\cap B_\beta$ se puede escribir como una unión de elementos de $\mathcal{A}$, por tanto $U\cap V$ es una unión de elementos de $\mathcal{A}$.
        \end{enumerate}
        por los tres incisos anteriores, se sigue que $\tau_{\mathcal{A}}$ es una topología sobre $X$. El hecho de que $\mathcal{A}$ sea una base para esta topología es inmediato de la definición de $\tau_{\mathcal{A}}$.
    \end{proof}

    \section{Sub-bases}

    \begin{mydef}
        Sea $X$ un conjunto y $\mathcal{S}$ una colección no vacía de subconjuntos de $X$. Entonces, se dice que $\mathcal{S}$ es una \textbf{sub-base para $\tau(\mathcal{S})$}.
    \end{mydef}

    \begin{exa}
        Sea $X=\left\{a,e,i,o,u\right\}$, $\mathcal{S}=\left\{\left\{a,e\right\},\left\{e,i\right\} \right\}$. $\mathcal{S}$ es una sub-base para $\tau(\mathcal{S})$ pero no es una base para $\tau(\mathcal{S})$.

        Sea $\mathcal{S}'=\left\{\left\{a,e\right\},\left\{e,i\right\},\left\{e\right\} \right\}$. Entonces $\tau(\mathcal{S})=\tau(\mathcal{S}')$, pues 
        \begin{enumerate}
            \item $\mathcal{S}\subseteq\mathcal{S}'$ implica que $\tau(\mathcal{S})\subseteq\tau(\mathcal{S}')$.
            \item Como $\left\{a,e\right\},\left\{e,i\right\}\in\tau(\mathcal{S})$ entonces $\mathcal{S}'\subseteq\tau(\mathcal{S})$ lo cual implica que $\tau(\mathcal{S}')\subseteq\tau(\mathcal{S})$.
        \end{enumerate}
    \end{exa}

    \begin{propo}
        Sea $X$ un conjunto arbitrario y sea $\mathcal{S}$ una colección no vacía de subconjuntos de $X$. Sea 
        \begin{equation*}
            \mathcal{B}=\left\{B\subseteq X\big| B\textup{ se puede escribir como una intersección finita de elementos de }\mathcal{S} \right\}\cup\left\{X\right\}
        \end{equation*}
        Entonces, 
        \begin{enumerate}
            \item $\mathcal{B}$ es una base para una topología sobre $X$.
            \item $\tau_{\mathcal{B}}$ es la topología más gruesa definida sobre $X$ para la cual $\mathcal{S}$ es una colección de conjuntos abiertos, es decir que $\tau_{\mathcal{B}}=\tau(\mathcal{S})$.
        \end{enumerate}
    \end{propo}

    \begin{proof}
        Notemos antes que $\mathcal{S}\subseteq\mathcal{B}$.

        De (1): Sean $M,N\in\mathcal{B}$, entonces existen $S_{M,1},...,S_{M,k},S_{N,1},...,S_{N,l}\in\mathcal{S}$ tales que:
        \begin{equation*}
            M=\bigcap_{i=1 }^kS_{M,i }\quad \textup{y}\quad N=\bigcap_{j=1 }^nN_{N,j }
        \end{equation*}
        por tanto:
        \begin{equation*}
            \begin{split}
                M\cap N=&\left(\bigcap_{i=1 }^kS_{M,i }\right)\cap\left(\bigcap_{j=1 }^nN_{N,j }\right) \\
                =&\left(\bigcap_{i=1,j=1 }^{k,n } S_{M,i }\cap S_{N,j }  \right)
            \end{split}
        \end{equation*}
        luego, por definición de $\mathcal{B}$ se sigue que $M\cap N\in\mathcal{B}$.

        Además, de la definición es claro que $X\in\mathcal{B}$. Por tanto, $\mathcal{B}$ es una base de una topología sobre $X$.

        De (2): De la observacíon que se hizo al inicio, se tiene que $\mathcal{S}\subseteq\tau_{\mathcal{B}}$, es decir que $\tau_{\mathcal{B}}$ es una topología que contiene a $\mathcal{S}$, luego $\tau(\mathcal{S})\subseteq \tau_{\mathcal{B}}$.

        Suponga que $\tau$ es una topología sobre $X$ tal que $\mathcal{S}\subseteq\tau$. Es claro que $\mathcal{B}\subseteq\tau$ ya que $\tau$ es cerrado bajo intersecciones finitas. Por tanto, $\tau_{\mathcal{B}}\subseteq\tau$. Luego:
        \begin{equation*}
            \tau_{\mathcal{B}}\subseteq\tau(\mathcal{S})
        \end{equation*}
        Por ambas contenciones se sigue la igualdad.
    \end{proof}

    \begin{obs}
        Sea $X$ un conjunto y $\mathcal{S}$ una colección no vacía de subconjuntos de $X$. Sea $M\in\tau(\mathcal{S})$, es decir que existe $\left\{A_\alpha \right\}_{\alpha\in I } \subseteq\mathcal{P}(X)$ tal que:
        \begin{equation*}
            M=\bigcup_{\alpha\in I}A_\alpha
        \end{equation*}
        y además, dado $\beta\in I$ existen $S_{\beta,1},...,S_{\beta,n_\beta}\in\mathcal{S}$ tales que:
        \begin{equation*}
            A_\beta=\bigcap_{i=1 }^{n_\beta }S_{\beta,i}
        \end{equation*}
        por tanto,
        \begin{equation*}
            M=\bigcup_{\alpha\in I}\left(\bigcap_{i=1 }^{n_\alpha }S_{\alpha,i} \right)
        \end{equation*}
        lo cual caracteriza a los elementos de $\tau(\mathcal{S})$.
    \end{obs}

    \begin{excer}
        Demuestre que las siguientes colecciones de subconjuntos de $\mathbb{R}$ son base para una topología sobre $\mathbb{R}$:
        \begin{enumerate}
            \item $\mathcal{B}_1=\left\{]a,b[\Big|a,b\in\mathbb{R},a<b \right\}$.
            \item $\mathcal{B}_2=\left\{[a,b[\Big|a,b\in\mathbb{R},a<b \right\}$.
            \item $\mathcal{B}_3=\left\{]a,b]\Big|a,b\in\mathbb{R},a<b \right\}$.
            \item  $\mathcal{B}_4=\left\{B-K \Big|B\in\mathcal{B}_1 \right\}\cup\mathcal{B}_1$, con $K=\left\{\frac{1}{n}\Big|n\in\mathbb{N} \right\}$.
            \item $\mathcal{B}_5=\left\{]a,\infty[\Big|a\in\mathbb{R}\right\}$.
            \item $\mathcal{B}_6=\left\{]-\infty,b[\Big|b\in\mathbb{R} \right\}$.
            \item $\mathcal{B}_7=\left\{A\subseteq\mathbb{R}\Big| \mathbb{R}-A\textup{ es finito} \right\}$.
        \end{enumerate}
    \end{excer}

    \begin{sol}
        %TODO (4) y (7)%

        La demostrción de (1)-(3) es muy similar, por lo que solo se probará para (3).
        
        De (3): Tenemos que verificar que la intersección de dos elementos de $\mathcal{B}_3$ se puede escribir como unión de elementos de $\mathcal{B}_3$ y que $\mathbb{R}$ puede ser escrito como unión de elementos de esta colección. En efecto:
        \begin{enumerate}
            \item Sean $]a_1,b_1],]a_2,b_2]\in\mathcal{B}_3$. Se tienen dos casos:
            \begin{itemize}
                \item $]a_1,b_1]\cap]a_2,b_2]=\emptyset$. En este caso la intersección se escribe como la unión de los elementos de la familia vacía.
                \item $]a_1,b_1]\cap]a_2,b_2]\neq \emptyset$. Analicemos este caso.
            \end{itemize}
            \item Notemos que:
            \begin{equation*}
                \mathbb{R}=\bigcup_{m\in\mathbb{Z} }]m,m+1]
            \end{equation*}
            donde $]m,m+1]\in\mathcal{B}_3$ para todo $m\in\mathbb{Z}$.
        \end{enumerate}
        por 1) y 2) se sigue que $\mathcal{B}_3$ es una base para una topología sobre $\mathbb{R}$.

        De (4): 

        La prueba de (5) y (6) es muy similar, por lo cual solo se hará la de (5).
        
        De (5): Se tienen que verificar dos condiciones:
        \begin{enumerate}
            \item Sean $a,b\in\mathbb{R}$. Sea $c=\max\left\{a,b\right\}\in\mathbb{R}$, tenemos que:
            \begin{equation*}
                ]a,\infty[\cap]b,\infty[=]c,\infty[
            \end{equation*}
            en efecto, si $x\in ]a,\infty[\cap]b,\infty[$, entonces $x>a$ y $x>b$, luego $x>\max\left\{a,b\right\}=c$, así pues $x\in ]c,\infty[$. Si $x\in]c,\infty[$ es claro que $x\in]a,\infty[\cap]b,\infty[$. Luego la intersección de estos dos elementos de $\mathcal{B}_5$ se escribe como unión de elementos de $\mathcal{B}_5$, pues $]c,\infty[\in\mathcal{B}_5$.

            \item Notemos que:
            \begin{equation}
                \mathbb{R}=\bigcup_{m\in\mathbb{N}}]-m,\infty[
            \end{equation}
            donde $]-m,\infty[\in\mathcal{B}_5$ para todo $m\in\mathbb{N}$.
        \end{enumerate}

        Por los dos incisos anteriores, se sigue que $\mathcal{B}_5$ es una base para una topología sobre $\mathbb{R}$.

        De (7):
    \end{sol}

    \begin{obs}
        Usamos la notación:
        \begin{equation*}
            \mathcal{B}_l=\left\{[a,b[\Big|a,b\in\mathbb{R},a<b \right\}
        \end{equation*}
        a la topología $\tau_{\mathcal{B}_l}$ la llamaremos la \textbf{topología del límite inferior}, y se denota por $\tau_l$.
    \end{obs}

    \section{Subespacios topológicos}

    \begin{excer}
        Sea $(X,\tau)$ un espacio topológico y $Y\subseteq X$. Demostrar que
        \begin{equation*}
            \tau_Y=\left\{Y\cap U\Big|U\in\tau \right\}
        \end{equation*}
        es una topología sobre $Y$.
    \end{excer}

    \begin{proof}
        Verifiquemos que se cumplen las tres condiciones:
        \begin{enumerate}
            \item Es claro que $\emptyset\in\tau_Y$, pues $\emptyset=Y\cap\emptyset$ donde $\emptyset\in\tau$. Además, $Y\in\tau_Y$ pues $Y=Y\cap X$ con $X\in\tau$.
            \item Sea $\left\{A_\alpha \right\}_{\alpha\in I}\subseteq\tau_Y$ una subcoleccíón no vacía de elementos de $\tau_Y$. Entonces, para cada $\alpha\in I$ existe $U_\alpha\in\tau$ tal que
            \begin{equation*}
                A_\alpha=Y\cap U_\alpha
            \end{equation*}
            por lo cual:
            \begin{equation*}
                \begin{split}
                    \bigcup_{\alpha\in I }A_\alpha=&\bigcup_{\alpha\in I }\left(Y\cap U_\alpha \right) \\
                    =&Y\cap \bigcup_{\alpha\in I } U_\alpha\\
                \end{split}
            \end{equation*}
            donde $\bigcup_{\alpha\in I } U_\alpha\in\tau$. Por tanto, $\bigcup_{\alpha\in I }A_\alpha\in\tau_Y$.
            \item Sean $A,B\in\tau_Y$ entonces, existen $U,V\in\tau$ tales que:
            \begin{equation*}
                A=Y\cap U\quad\textup{y}\quad B=Y\cap V
            \end{equation*}
            por tanto:
            \begin{equation*}
                \begin{split}
                    A\cap B=&(Y\cap U)\cap (Y\cap V)\\
                    =&Y\cap (U\cap (Y\cap V))\\
                    =&Y\cap (Y\cap (U\cap V))\\
                    =&Y\cap (U\cap V)\\
                \end{split}
            \end{equation*}
            donde $U\cap V\in\tau$. Así $A\cap B\in\tau_Y$.
        \end{enumerate}
        por los tres incisos anteriores se sigue que $\tau_Y$ es una topología sobre $Y$.
    \end{proof}
    
    \begin{mydef}
        Sea $(X,\tau)$ un espacio topológico y sea $Y\subseteq X$. A la topología sobre $Y$,
        \begin{equation*}
            \tau_Y=\left\{Y\cap U\Big|U\in\tau \right\}
        \end{equation*}
        la llamaremos \textbf{la topología inducida por $\tau$ en $Y$}. A la pareja $(Y,\tau_Y)$ la llamaremos \textbf{un subespacio topológico de $(X,\tau)$}.

        Si $A\in\tau_Y$, se dice que $A$ es un \textbf{abierto en $Y$}. Si $A\subseteq Y$ y cumple que $Y-A\in\tau_Y$, se dice que $A$ es un \textbf{cerrado en $Y$}.
    \end{mydef}

    \begin{exa}
        Considere $(\mathbb{R},\tau_u)$, $Y=[0,1[$, $A=[0,\frac{1}{2}[$. Podemos escribir:
        \begin{equation*}
            A=]-1,\frac{1}{2}[\cap Y
        \end{equation*}
        donde $]-1,\frac{1}{2}[\in\tau_u$, por ende $A\in\tau_Y$, pero $A$ no es abierto en $(\mathbb{R},\tau_u)$.

        Se tiene además que $\Int{A}=]0,\frac{1}{2}[$ y, como $A\in\tau_Y$, entonces $\Int{A}^Y=A=[0,\frac{1}{2}[$.
    \end{exa}

    \begin{exa}
        Considere $(\mathbb{R},\tau_u)$. Tomemos al subconjunto $\mathbb{N}$. Como:
        \begin{equation*}
            \left\{m\right\}=\mathbb{N}\cap]m-\frac{1}{2},m+\frac{1}{2}[,\quad\forall m\in\mathbb{N}
        \end{equation*}
        así, $\left\{m \right\}\in\tau_{u_\mathbb{N}}$, es decir que coincide con la topología discreta de $\mathbb{N}$, pero $\left\{m\right\}\notin\tau_u$.
    \end{exa}

    \begin{exa}
        Considere $(\mathbb{R},\tau_u)$, $Y=[0,1[$, $A=[0,\frac{1}{2}[\in\tau_{u_Y }$. Sea $B=[\frac{1}{2},1[\subseteq Y$. Se tiene que $B=Y-A$, es decir que $B$ es un cerrado de $(Y,\tau_{u_Y })$, pero no es cerrado en $(\mathbb{R},\tau_u)$.

        Además, $\Cls{B}=[\frac{1}{2},1]$, y $\Cls{B}^Y=B=[\frac{1}{2},1[$ (por ser cerrado en la topología del subespacio).

        Sea $M\subseteq Y$, denotamos por $\Fr{M}_Y$ a la frontera de $M$ en $(Y,\tau_{u_Y })$.
    \end{exa}

    \begin{propo}
        Sean $(X,\tau)$ un espacio topológico y $Y\subseteq X$.
        \begin{enumerate}
            \item Si $Y$ es abierto (respectivamente, cerrado) en $(X,\tau)$ y $U\subseteq Y$ es un conjunto abierto (respectivamente, cerrado) en $(Y,\tau_Y)$, entonces $U$ es abierto (respectivamente, cerrado) en $(X,\tau)$.
            \item Si $\mathcal{B}$ es una base para $\tau$, entonces la colección:
            \begin{equation*}
                \mathcal{B}_Y=\left\{Y\cap B\Big| B\in\mathcal{B} \right\}
            \end{equation*}
            es una base para la topología $\tau_Y$.
            \item Sea $A\subseteq Y$. Entonces, $A$ es cerrado en $Y$ si y sólo si existe $C\subseteq X$ cerrado en $X$ tal que $A=Y\cap C$.
            \item Sea $B\subseteq Y$. Si $\Cls{B}$ es la cerradura de $B$ en $X$, entonces la cerradura de $B$ en $Y$, denotada por $\Cls{B}^Y$, es $Y\cap \Cls{B}$.
            \item Sea $A\subseteq Y$, si $\Int{\widehat{A}}^Y$ denota al interior de $A$ en $Y$, entonces $Y\cap\Int{A}\subseteq \Int{\widehat{A}}^Y$.
            \item Sea $A\subseteq Y$, si $\Fr{A}_Y$ es la frontera de $A$ en $Y$, entonces $\Fr{A}_Y\subseteq Y\cap \Fr{A}$.
        \end{enumerate}
    \end{propo}

    \begin{proof}
        De (1): En ambos casos la prueba es inmediata de la definción de subespacio de un espacio topológico.

        De (2): Se deben verificar dos condiciones:
        \begin{enumerate}
            \item $\mathcal{B}_Y\subseteq\tau_Y$, esto es inmediato pues si $Y\cap B\in\mathcal{B}_Y$, entonces $B\in\mathcal{B}$, luego $B\in\tau$ y, por ende $Y\cap B\in\tau_Y$.
            \item Sea $U\in\tau_Y$ abierto no vacío. Entonces existe $V\in\tau$ tal que $U=Y\cap V$. Sea $\left\{B_\alpha \right\}_{\alpha\in I}\subseteq\mathcal{B}$ tal que $V=\bigcup_{\alpha\in I}B_\alpha$, entonces:
            \begin{equation*}
                \begin{split}
                    U=&Y\cap V\\
                =&Y\cap\left(\bigcup_{\alpha\in I}B_\alpha\right) \\
                =&\left(\bigcup_{\alpha\in I}Y\cap B_\alpha\right) \\
                \end{split}
            \end{equation*}
            por tanto, $U$ es unión de elementos de $\mathcal{B}_Y$.
        \end{enumerate}
        
        por los dos incisos anteriores se sigue que $\mathcal{B}_Y$ es base de $\tau_Y$.

        De (3): Probaremos la doble implicación:
        
        $\Rightarrow)$: Suponga que $A$ es cerrado en $Y$, entonces $Y-A\in\tau_Y$, luego existe $U\in\tau$ tal que $Y-A=Y\cap U$. Tomemos $C=X-U$, se tiene que:
        \begin{equation*}
            \begin{split}
                Y\cap C=&(X-U)\cap Y\\
                =&(X\cap Y)-(U\cap Y)\\
                =&Y-(U\cap Y)\\
                =&Y-(Y\cap U)\\
                =&Y-(Y-A)\\
                =&A\\
                \Rightarrow A=& Y\cap C\\
            \end{split}
        \end{equation*}

        \textit{Prueba alternativa de la igualdad de conjuntos}. Sea $a\in A$, en particular $a\in Y$. Si $a\in U$, entonces $a\notin A$, luego esto contradice el hecho de que $Y-A=Y\cap U$, por tanto $a\in C$. Así, $a\in C\cap Y$.

        Si $p\in Y\cap C$, entonces $p\in Y$ y $p\notin U$, luego $p\notin Y-A$, es decir $p\in A$.

        Por lo anterior se sigue que $A=Y\cap C$.

        $\Leftarrow)$: Suponga que existe $C\subseteq X$ cerrado en $X$ tal que $A=Y\cap C$. Hay que ver que $A$ es cerrado en $Y$, para ello, notemos que:
        \begin{equation*}
            \begin{split}
                Y-A=&Y-(Y\cap C) \\
                =&Y-(Y-Y\cap U) \\
                =&Y\cap U \\
            \end{split}
        \end{equation*}
        donde $U=X-C$ es abierto en $X$ y, por ende $Y\cap U$ es abierto en $Y$, luego $Y-A$ es abierto en $Y$ lo que implica que $A$ es cerrado.

        De (4): Se probarán las dos contenciones:
        \begin{itemize}
            \item $Y\cap\Cls{B}\subseteq\Cls{B}^Y$) Por el inciso anterior, $Y\cap\Cls{B}$ es un cerrado en $Y$ el cual contiene a $B$ (pues $B\subseteq\Cls{B},Y$), luego $\Cls{B}^Y\subseteq Y\cap \Cls{B}$.
            \item $\Cls{B}^Y\subseteq Y\cap\Cls{B}$) Sea $M\subseteq Y$ cerrado en $Y$ tal que $B\subseteq M$, luego por (3) existe $K\subseteq X$ cerrado tal que $M=Y\cap K$, siendo $K$ un cerrado que contiene a $B$, luego $\Cls{B}\subseteq K$. Por tanto, $Y\cap \Cls{B}\subseteq M$. Por ende, al ser $M$ un cerrado en $Y$ arbitrario que contiene a $B$ se sigue que $Y\cap\Cls{M}\subseteq \Cls{B}^Y$.
        \end{itemize}

        Por las dos contenciones se sigue que $\Cls{B}^Y=Y\cap\Cls{B}$.

        De (5): Es inmediato.

        De (6): Observemos que:
        \begin{equation*}
            \begin{split}
                \Fr{A}_Y=&\Cls{A}^Y\cap\Cls{Y-A}^Y\\
                =&(Y\cap\Cls{A})\cap(Y\cap\Cls{Y-A})\\
                =&Y\cap(\Cls{A}\cap\Cls{Y-A}) \\
                \subseteq&Y\cap(\Cls{A}\cap\Cls{X-A}) \\
                =&Y\cap\Fr{A}\\
                \Rightarrow \Fr{A}_Y\subseteq& Y\cap\Fr{A}\\
            \end{split}
        \end{equation*}
        pues, $Y-A\subseteq X-A$.

    \end{proof}

    \begin{obs}
        En el inciso (3) de la demostración anterior, notemos que:
        \begin{equation*}
            (Y\cap U)\cup (Y\cap C)=Y
        \end{equation*}
        pues, $U\cup C = X$ donde $U$ y $C$ son disjuntos, luego $Y\cap U$ y $Y\cap C$ lo son , así $Y-Y\cap U=Y\cap C$. Esto justifica un paso en la demostración de la vuelta de (3).
    \end{obs}

    \begin{exa}
        Considere $(\mathbb{R},\tau_u)$ y, considere el subespacio $(\mathbb{Z},\tau_{u\mathbb{Z}})$. Entonces, $\Int{\mathbb{N}}=\emptyset$ y, $\Int{\widehat{\mathbb{N}}}^Y=\mathbb{N}$. Por ende, $\Int{\widehat{\mathbb{N}}}^Y\nsubseteq\Int{\mathbb{N}}$.
    \end{exa}

    \begin{exa}
        Considere $(\mathbb{R}^2,\tau_u)$ y el subespacio $(Y,\tau_{uY})$ con:
        \begin{equation*}
            Y=\left\{(x,y)\in\mathbb{R}^2\Big|y=0 \right\}
        \end{equation*}
        entonces, $\Fr{Y}=Y$ y $\Fr{Y}_Y=\emptyset$.
    \end{exa}

    \begin{obs}
        Sea $(X,\tau)$ un espacio topológico y sean $Y,Z\subseteq X$ tales que $Z\subseteq Y$. Tenemos que podemos considerar a $(Z,\tau_Z)$ como subespacio de $X$.

        También, podemos considerar a $(Z,\tau_{Y_Z})$ como subespacio de $(Y,\tau_Y)$.

        ¿Es cierto que $\tau_{Y_Z}=\tau_{Y}$? La respuesta es que sí:
        \begin{itemize}
            \item Sea $M\in\tau_Z$, entonces, $M=Z\cap U$ donde $U\in\tau$, luego como $M\subseteq Y$ se sigue que: $M=Z\cap(Y\cap U)$ siendo $Y\cap U\in\tau_Y$, así $M\in \tau_{Y_Z}$.
            \item Sea $K\in \tau_{Y_Z}$, entonces existe $L\in\tau_Y$ tal que $K=Z\cap L$, pero como $L\in\tau_Y$ entonces existe $U\in\tau$ tal que $L=Y\cap U$, por tanto: $K=Z\cap(Y\cap U)=Z\cap U$ pues $Z\subseteq Y$, luego $K\in\tau_Z$.
        \end{itemize}
        por ambos incisos, se sigue la igualdad.

        El objetivo de esta aclaración es que podamos considerar de forma más sencillas subespacios dentro de subespacios.
    \end{obs}

    \begin{mydef}
        Sea $(X,\tau)$ un espacio topológico. Una propiedad $P$ que se cumple para $(X,\tau)$ se dice que es una \textbf{propiedad que se hereda}, si se verifica en cualquier subespacio topológico de $(X,\tau)$. A veces simplemente se dice que $P$ es una \textbf{propiedad hereditaria}.
    \end{mydef}

    \begin{exa}
        La propiedad de ser un espacio de Hausdorff es hereditaria.
    \end{exa}

    \begin{proof}
        Sea $(X,\tau)$ un espacio de Hausdorff y sea $Y\subseteq X$ arbitrario. Sean $p,q\in Y$ con $p\neq q$, en particular como $X$ es de Hausdorff, existen $M,N\in\tau$ tales que $p\in M$, $q\in N$ y $M\cap N=\emptyset$.

        En particular, $p\in Y\cap M$ y $q\in Y\cap N$, donde ambos conjuntos son abiertos en $Y$ y, además $(Y\cap M)\cap (Y\cap N)=\emptyset$. Por tanto, $(Y,\tau_Y)$ es de Hausdorff.
    \end{proof}

    \begin{exa}
        Sea $(X,\tau)$ un espacio topológico tal que $\tau$ tiene una base numerable, sea $\mathcal{B}$ tal base. Si $Y\subseteq X$ es arbitrario, sabemos que
        \begin{equation*}
            \mathcal{B}_Y=\left\{Y\cap B\Big|B\in\mathcal{B} \right\}
        \end{equation*}
        es una base para $\tau_Y$, la cual es numerable por ser $\mathcal{B}$ numerable. Luego esta propiedad es hereditaria.
    \end{exa}

    \begin{excer}
        La propiedad de ser metrizable se hereda.
    \end{excer}

    \begin{proof}
        Sea $(X,\tau)$ un espacio topológico metrizable, entonces existe una métrica $\cf{d}{X\times X}{\mathbb{R}}$ tal que $\tau_d=\tau$.

        Sea ahora $(Y,\tau_Y)$ un subespacio de $(X,\tau)$. Considere la restricción de $d$ a $Y\times Y$, es decir:
        \begin{equation*}
            \rho=d\Big|_{Y\times Y}
        \end{equation*}
        es claro que $\rho$ es una métrica sobre $Y$. Para ver que $(Y,\tau_Y)$ es metrizable, hay que ver que:
        \begin{equation*}
            \tau_\rho=\tau_Y
        \end{equation*}
        donde
        \begin{equation*}
            \tau_\rho=\left\{A\subseteq Y\Big|\forall x\in A\exists r_x\in\mathbb{R}^+\textup{ tal que }B_\rho(x,r_x)\subseteq A \right\}
        \end{equation*}
        Sea $A\in\tau_\rho$, entonces para cada $x\in A$ existe $r_x\in\mathbb{R}^+$ tal que $B_\rho(x,r_x)\subseteq A$, esto es:
        \begin{equation*}
            A=\bigcup_{x\in A}B_\rho(x,r_x)
        \end{equation*}
        pero, notemos que:
        \begin{equation*}
            \begin{split}
                B_\rho(x,r_x)&=\left\{y\in Y\Big|\rho(x,y)<r_x \right\}\\
                &=\left\{y\in Y\Big|d(x,y)<r_x \right\}\\
                &=\left\{u\in X\Big|d(x,u)<r_x \right\}\cap Y \\
                &= B_d(x,r_x)\cap Y
            \end{split}
        \end{equation*}
        %TODO
    \end{proof}

    \section{Relaciones de orden y la topología del orden}

    \begin{mydef}
        Una relación $\mathcal{R}$ definida sobre un conjunto $A$ es una \textbf{relación de orden lineal} si se cumple lo siguiente:
        \begin{enumerate}
            \item Dados $a,b\in A$ distintos se tiene que $a\mathcal{R}b$ ó $b\mathcal{R}a$.
            \item Para todo elemento de $a\in A$, $a\cancel{\mathcal{R}}a$.
            \item Si $a,b,c\in A$ son tales que $a\mathcal{R}b$ y $b\mathcal{R}c$, entonces $a\mathcal{R}c$.
        \end{enumerate}
    \end{mydef}

    \begin{mydef}
        Si $\mathcal{R}$ es una relación de orden lineal definida sobre el conjunto $A$, diremos que $(A,\mathcal{R})$ es un \textbf{conjunto ordenado}.
    \end{mydef}

    \begin{propo}
        Sea $(X,\mathcal{R})$ un conjunto ordenado y sea $B\subseteq X$.
        \begin{enumerate}
            \item Si existe $b\in B$ tal que $\forall x\in B-\left\{b\right\}$, $b\mathcal{R} x$, entonces $b$ es único y se dice el \textbf{elemento mínimo} (a veces también llamado \textbf{primer elemento}) de $B$.
            \item Si existe $b\in B$ tal que $\forall x\in B-\left\{b\right\}$, $x\mathcal{R} b$, entonces $b$ es único y se dice el \textbf{elemento máximo} (a veces también llamado \textbf{último elemento}) de $B$.
        \end{enumerate}
    \end{propo}

    \begin{proof}
        De 1): Suponga que existe $b\in B$ tal que:
        \begin{equation*}
            b\mathcal{R}x,\quad\forall x\in B-\left\{b\right\}
        \end{equation*}
        si $b'\in B$ es diferente de $b$ y tal que
        \begin{equation*}
            b'\mathcal{R}x,\quad\forall x\in B-\left\{b'\right\}
        \end{equation*}
        entonces se tendría que $b\mathcal{R}b'$ y $b'\mathcal{R}b$, lo cual es una contradicción ya que $\mathcal{R}$ es un orden lineal.

        Por tanto, tal $b$ es único.

        De 2): Es análoga a (1)

    \end{proof}

    \begin{obs}
        Si $(X,\prec)$ es un conjunto ordenado y $a,b\in X$, escribimos $a\preceq b$ si $a\prec b$ o $a=b$.
    \end{obs}

    \begin{mydef}
        Sea $(A,\prec)$ un conjunto ordenado y $B\subseteq A$. 
        \begin{enumerate}
            \item Si existe $a\in A$ tal que para todo $x\in B$ se cumple que $a\preceq x$, diremos que $B$ está \textbf{acotado inferiormente por $A$}. En este caso, $a$ se dice una \textbf{cota inferior de $B$}.
            \item Si existe $a'\in A$ tal que para todo $x\in B$ se cumple que $x\preceq a'$, diremos que $B$ está \textbf{acotado superiormente por $A$}. En este caso, $a'$ se dice una \textbf{cota superior de $B$}.
            \item Si $B$ está acotado inferiormente y el conjunto de cotas inferiores de $B$ tiene elemento máximo, diremos que tal elemento es \textbf{la máxima cota inferior de $B$} (abreviado \textbf{máx. c.i.}).
            \item Si $B$ está acotado superiormente y el conjunto de cotas superiores de $B$ tiene elemento mínimo, diremos que tal elemento es \textbf{la mínima cota superior de $B$} (abreviado \textbf{mín. c.s.}).
            \item Si cada subconjunto no vacío acotado superiormente (resp. inferiormente) del conjunto ordenado $(A,\prec)$ tiene mínima cota superior (resp. máxima cota inferior), se dice que $(A,\prec)$ tiene la propiedad de la \textbf{mínima cota superior} (resp. \textbf{máxima cota inferior}).
        \end{enumerate}
    \end{mydef}

    \begin{mydef}
        Un conjunto ordenado $(A,\prec)$ es un \textbf{continuum lineal} si cumple:
        \begin{enumerate}
            \item $(A,\prec)$ tiene la propiedad de la mínima cota superior.
            \item Si $a,b\in A$ tales que $a\prec b$, entonces existe $c\in A$ tal que $a\prec c$ y $c\prec b$ (a veces escrito como $a\prec c\prec b$).
        \end{enumerate}
    \end{mydef}

    \begin{mydef}
        Sea $(A,\prec)$ un conjunto ordenado y sean $a,b\in A$ tales que $a\prec b$. Definimos los siguientes conjuntos:
        \begin{enumerate}
            \item $(a,b)=\left\{x\in A\Big|a\prec x\prec b \right\}$, llamado el \textbf{intervalo abierto con extremos $a$ y $b$}.
            \item Si $(a,b)=\emptyset$, $a$ se dice el \textbf{predecesor inmediato de $b$}, y $b$ el \textbf{sucedor inmediato de $a$}.
            \item $[a,b]=\left\{x\in A\Big|a\preceq x\preceq b \right\}$, llamado el \textbf{intervalo cerrado con extremos $a$ y $b$}.
            \item $(a,b]=\left\{x\in A\Big|a\prec x\preceq b \right\}$, llamado el \textbf{intervalo abierto por la izquierda y cerrado por la derecha con extremos $a$ y $b$}.
            \item $[a,b)=\left\{x\in A\Big|a\preceq x\prec b \right\}$, llamado el \textbf{intervalo abierto por la derecha y cerrado por la izquierda con extremos $a$ y $b$}.
            \item Los siguientes cuatro conjuntos se llaman \textbf{rayos determinados por el elemento $a$}:
            \begin{enumerate}
                \item $(a,+\infty)=\left\{x\in A\Big|a\prec x \right\}$.
                \item $[a,+\infty)=\left\{x\in A\Big|a\preceq x \right\}$.
                \item $(-\infty,a)=\left\{x\in A\Big|x\prec a \right\}$.
                \item $(-\infty, a]=\left\{x\in A\Big|x\preceq a \right\}$.
            \end{enumerate}
            \item Al rayo $(-\infty, a)$ también se le conoce como la \textbf{sección definida por el elemento $a$}, y se escribe $\mathcal{S}_a$, es decir:
            \begin{equation*}
                \mathcal{S}_a=\left\{x\in A\Big|x\prec a \right\}
            \end{equation*}
        \end{enumerate}
    \end{mydef}

    \begin{propo}
        Sea $(X,\prec)$ un conjunto ordenado y sea $\mathcal{B}$ la colección de todos los subconjuntos de $X$ de los siguientes tipos:
        \begin{enumerate}
            \item Todos los intervalos abiertos de $X$.
            \item Todos los invervalos de la forma $[a_0,b)$ donde $a_0$ es el elemento mínimo de $(X,\prec)$ (si es que tal $a_0$ existe).
            \item Todos los invervalos de la forma $(a,b_0]$ donde $b_0$ es el elemento máximo de $(X,\prec)$ (si es que tal $b_0$ existe).
        \end{enumerate}
        Entonces, $\mathcal{B}$ es una base para una topología sobre $X$, la cual llamaremos la \textbf{topología del orden} y se denota por $\tau_\prec$.

        Tenemos además, que la colección $\mathcal{S}_a$ de todos los rayos de la forma $(a,+\infty)$ con $a\in X$ es una sub-base para $\tau_\prec$.
    \end{propo}
    
    \begin{proof}
        Tenemos cuatro casos:
        \begin{enumerate}
            \item \textbf{$X$ no tiene elemento máximo y mínimo}. Se tienen que verificar dos cosas:
            \begin{enumerate}
                \item Si $(a_1,b_1),(a_2,b_2)$ es un intervalo abierto en $X$, entonces la intersección de ambos es unión de intervalos abiertos. En efecto, si la intersección es no vacía, entonces:
                \begin{equation*}
                    (a_1,b_1)\cap(a_2,b_2)=(\max\left\{a_1,a_2\right\},\min\left\{b_1,b_2\right\})
                \end{equation*}
                el cual es un intervalo abierto. 
                \item $X$ es unión de intevalos abiertos. En efecto, sea
                \begin{equation*}
                    \mathcal{A}=\left\{(a,b)\Big|a,b\in X, a\prec b \right\}
                \end{equation*}
                entonces:
                \begin{equation*}
                    X=\bigcup\mathcal{A}
                \end{equation*}
            \end{enumerate}
            \item \textbf{$X$ tiene elemento máximo pero no mínimo}.
        \end{enumerate}
    \end{proof}

    \begin{mydef}
        Sean $(A,\prec_A)$ y $(B,\prec_B)$ dos conjuntos ordenados. definimos la relación $\prec$ sobre $A\times B$ de la siguiente manera
        \begin{equation*}
            (a,b)\prec(c,d)\iff a\prec_A c\textup{ o, }a = c \textup{ y }b\prec_B d
        \end{equation*}
        esta relación es un orden definido sobre $A\times B$ y se dice el \textbf{orden del diccionario}.
    \end{mydef}

    \begin{proof}
        Se deben cumplir tres cosas:
        \begin{enumerate}
            \item Sean $(a,b),(c,d)\in A\times B$ y suponga que $(a,b)\neq(c,d)$, se tienen dos casos:
            \begin{enumerate}
                \item $a\neq c$, entonces $a\prec_A c$ ó $a\prec c$. En el primer caso se tiene que $(a,b)\prec(c,d)$. En caso contrario se tendría que $(c,d)\prec(a,b)$.
                \item $a=c$, entonces, puede suceder $b\prec_B d$ ó $d\prec_B b$ ó $b=d$, por tanto $(a,b)\prec(c,d)$ ó $(c,d)\prec(a,b)$ ó $(a,b)=(c,d)$, en cuyo caso los elementos son iguales, cosa que no puede suceder. Por tanto solo puede suceder que $b\prec_B d$ ó $d\prec_B b$.
            \end{enumerate}
            por tanto, $(a,b)\prec(c,d)$ o $(c,d)\prec(a,b)$.
            \item Como $\prec_A$ y $\prec_B$ son antireflexivos, entonces siempre se tiene que $(a,b)\nprec (a,b)$.
            \item Sean $(x_1,y_1),(x_2,y_2),(x_3,y_3)\in A\times B$ tales que $(x_1,y_1)\prec(x_2,y_2)$ y $(x_2,y_2)\prec(x_3,y_3)$. Entonces
            \begin{equation*}
                \begin{split}
                    &(x_1\prec_A x_2 \textup{ ó } x_1=x_2 \textup{ y } y_1\prec_B y_2)\textup{ y }(x_2\prec_A x_3 \textup{ ó } x_2=x_3 \textup{ y } y_2\prec_B y_3)\\
                    \Rightarrow &[(x_1\prec_A x_2 \textup{ ó } x_1=x_2 \textup{ y } y_1\prec_B y_2)\textup{ y }x_2\prec_A x_3]\textup{ ó }\\
                    &[(x_1\prec_A x_2 \textup{ ó } x_1=x_2 \textup{ y } y_1\prec_B y_2)\textup{ y }(x_2=x_3 \textup{ y } y_2\prec_B y_3)]\\
                    \Rightarrow &[(x_1\prec_A x_2\textup{ y }x_2\prec_A x_3)\textup{ ó }(x_1=x_2 \textup{ y } y_1\prec_B y_2\textup{ y }x_2\prec_A x_3)]\textup{ ó }\\
                    &[(x_1\prec_A x_2\textup{ y }x_2=x_3 \textup{ y } y_2\prec_B y_3)\textup{ ó }(x_1=x_2 \textup{ y } y_1\prec_B y_2\textup{ y }x_2=x_3 \textup{ y } y_2\prec_B y_3)]\\
                    \Rightarrow &[(x_1\prec_A x_3)\textup{ ó }(x_1\prec_A x_3)]\textup{ ó }\\
                    &[(x_1\prec_A x_3)\textup{ ó }(x_1=x_3 \textup{ y } y_1\prec_B y_3)]\\
                    \Rightarrow &x_1\prec_A x_3\textup{ ó }x_1=x_3 \textup{ y } y_1\prec_B y_3\\
                    \Rightarrow &(x_1,y_2)\prec (x_3,y_3)\\
                \end{split}
            \end{equation*}
            por los tres incisos anteriores, se sigue que $\prec$ es un orden en $A\times B$.
        \end{enumerate}
    \end{proof}

    \begin{mydef}
        Un conjunto ordenado $(X,\mathcal{R})$ se dice que está \textbf{bien ordenado} si todo subconjunto no vacío de $X$ tiene primer elemento o elemento mínimo.
    \end{mydef}

    \section{Estudio del espacio topológico $(\Cls{\mathcal{S}_\omega},\tau_\prec)$}

    \begin{propo}
        Existe un conjunto bien ordenado no numerable, en el cual toda sección de él es numerable.
    \end{propo}

    \begin{proof}
        Sean $X=\left\{1,2\right\}$ y $\alpha=(1,2)$. Tenemos que $(X,\alpha)$ es un conjunto bien ordenado. Tomemos ahora sea $Y$ un conjunto no numerable y sea $\beta$ un buen orden definido sobre $Y$, luego la pareja $(Y,\beta)$ es un conjunto bien ordenado.

        Sea $Z=X\times Y$ y consideremos la relación $\prec$ definida sobre $Z$ de la siguiente manera:
        \begin{equation*}
            (a,b)\prec(c,d)\iff a\alpha c\textup{ ó }a = c\textup{ y } b\beta d
        \end{equation*}
        Ya tenemos que $(Z,\prec)$ es un conjunto ordenado (por la proposición anterior), el cual es no numerable.

        Veamos que $(Z,\prec)$ está bien ordenado. Sea $A\subseteq Z$ no vacío. Se tienen dos casos:
        \begin{enumerate}
            \item Suponga que existe $y\in Y$ tal que $(1,y)\in A$. Entonces, el conjunto:
            \begin{equation*}
                \mathcal{B}=\left\{l\in Y\Big|(1,l)\in A \right\}
            \end{equation*}
            este conjunto es no vacío pues $(1,y)\in\mathcal{B}$. Sea $m$ el primer elemento de $\mathcal{B}$, el cual existe por ser $Y$ bien ordenado. Veamos que $(1,m)$ es el primer elemento de $A$. Como $m\in B$, tenemos que $(1,m)\in A$. Sea $(x,y)\in A$, se tienen dos casos:
            \begin{enumerate}
                \item $x=1$, en cuyo caso $y\in \mathcal{B}$ y, por ende $m\beta y$ o $m=\beta$, lo cual implica que $(1,m)\preceq(x,y)$.
                \item $x=2$, en cuyo caso se tiene que $1\alpha x$ y, por ende, $(1,m)\prec(x,y)$.
            \end{enumerate}
            en cualquier caso, $(1,m)\preceq(x,y)$. Luego este elemento es el primer elemento de $A$.
            \item Suponga que para todo $(x,y)\in A$, $x=2$. Sea
            \begin{equation*}
                \mathcal{C}=\left\{l\in Y\Big|(2,l)\in A \right\}
            \end{equation*}
            el cual es no vacío pues $A\neq\emptyset$. $\mathcal{C}\subseteq Y$ no vacío el cual es bien ordenado, luego tiene primer elemento, digamos $m\in\mathcal{C}$. Por tanto, $(2,m)\in A$ y afirmamos que es el primer elemento de $A$, pues si $(x,y)\in A$ se tiene que $x=2$ y, por definición de $\mathcal{C}$ se sigue que $m\beta y$ o $m=y$, en cuyo caso se tiene que $(2,m)\preceq (x,y)$, lo cual prueba la afirmación.
        \end{enumerate}
        por ambos incisos, se sigue que $A$ tiene primer elemento. Por ser $A$ no vacío arbitrario, se tiene que $(Z,\prec)$ es un conjunto no numerable bien ordenado.

        Además, tenemos que para todo $y\in Y$, $\mathcal{S}_{(2,y)}$ es una sección de $(Z,\prec)$ no numerable, pues si $l\in Y$, entonces $(1,l)\prec (2,y)$, es decir que para todo $l\in Y$, $(1,l)\in\mathcal{S}_{(2,y)}$, con lo cual esta sección es no numerable. Sea
        \begin{equation*}
            \mathcal{W}=\left\{z\in Z\Big|S_z\textup{ es una sección no numerable de }(Z,\prec) \right\}
        \end{equation*}
        por lo anterior, $\mathcal{W}\neq\emptyset$. Sea $\omega$ el primer elemento de $\mathcal{W}$, es decir que la sección $S_\omega$ es no numerable y, para todo $z\in Z$, $w\preceq z$.
        
        Tenemos que la pareja $(S_\omega, \prec)$ es un conjunto bien ordenado no numerable en el que toda sección de él es numerable. Recordemos que:
        \begin{equation*}
            \mathcal{S}_\omega=\left\{z\in Z\Big|z\prec\omega \right\}
        \end{equation*}
        y este conjunto es bien ordenado por $\prec$, pues es subconjunto de $Z$ y es no numerable por como se eligió. Vamos a ver que toda sección de él es numerable.

        Sea $r\in\mathcal{S}_\omega$, entonces:
        \begin{equation*}
            \mathcal{S}_r=\left\{z\in\mathcal{S}_\omega\Big|z\prec r \right\}
        \end{equation*}
        como $r\prec\omega$ se tiene que por elección de $\omega$ debe suceder que $\mathcal{S}_r$ no puede ser no numerable, es decir que es a lo sumo numerable.

        Veamos que es numerable. En efecto, suponga que existe $p\in\mathcal{S}_\omega$ tal que $\mathcal{S}_p$ es una sección finita. Se tiene entonces que
    \end{proof}

    \begin{obs}
        Denotaremos por $\Cls{\mathcal{S}_\omega}=\mathcal{S}_\omega\cup\left\{\omega \right\}$, es decir:
        \begin{equation*}
            \Cls{\mathcal{S}_\omega}=\left\{z\in Z\Big|z\preceq\omega \right\}
        \end{equation*}
        Sea $\tau_\prec$ la topología generada por el buen orden $\prec$ en $Z$, y considere a $\Cls{\mathcal{S}_\omega}$ con la topología del subespacio $\tau_{\prec_{\Cls{\mathcal{S}_\omega}}}$, la cual denotaremos simplemente por $\tau_\prec$.
    \end{obs}

    \begin{propo}
        Si $A\subseteq \mathcal{S}_\omega$ numerable, entonces existe $s\in\mathcal{S}_\omega$ tal que para todo $a\in A$, $a\prec s$.
    \end{propo}

    \begin{proof}
        Tenemos que para todo $a\in A$, el conjunto $\mathcal{S}_a$ es numerable, luego
        \begin{equation*}
            B=\bigcup_{ a\in A}\mathcal{S}_a
        \end{equation*}
        es numerable. Por lo tanto, existe $s\in\mathcal{S}_\omega-(A\cup B)$. Veamos que para todo $a\in A$, $a\prec s$. Suponga que $s\prec k$ para algún $k\in A$, es decir que $s\in \mathcal{S}_k\subseteq B$, luego $s\in A\cup B$\contradiction. Por tanto, tal $s$ cumple lo deseado.
    \end{proof}

    \begin{propo}
        $(\Cls{\mathcal{S}_\omega},\tau_\prec)$ es un espacio de Hausdorff.
    \end{propo}

    \begin{proof}
        Sea $p$ el primer elemento de $\Cls{\mathcal{S}_\omega}$. Además, sean $a,b\in\Cls{\mathcal{S}_\omega}$ tales que $a\prec b$. Se tienen dos casos:
        \begin{enumerate}
            \item Suponga que $b=\omega$, entonces existe $c\in S_\omega$ tal que $a\prec c\prec b$ (en caso contrario se tendría que $\mathcal{S}_\omega=\mathcal{S}_a\cup\left\{a\right\}$, donde un lado es no numerable y el otro sí, lo cual no puede suceder). Entonces:
            \begin{equation*}
                a\in [p,c)\quad\textup{y}\quad b\in(c,\omega]
            \end{equation*}
            donde $[p,c),(c,\omega]\in\tau_\prec$ y su intersección es vacía.
            \item Suponga que $b\prec\omega$:
            \begin{enumerate}
                \item Si no existe $c\in\mathcal{S}_\omega$ tal que $a\prec c$ y $c\prec b$, entonces $a\in [p,b)$ y $b\in (a,\omega]$ y, $[p,b),(a,\omega]\in\tau_\prec$ son disjuntos.
                \item Si existe $c\in\mathcal{S}_\omega$ tal que $a\prec c\prec b$, entonces $a\in [p,c)$ y $b\in (c,\omega]$, donde $[p,c),(c,\omega]\in\tau_\prec$ son disjuntos.
            \end{enumerate}
        \end{enumerate}
        por los dos incisos anteriores, se sigue que el espacio es de Hausdorff.
    \end{proof}

    \begin{propo}
        $\omega$ es un punto de acumulación de $\mathcal{S}_\omega$.
    \end{propo}

    \begin{proof}
        Sea $B$ un básico de $\tau_\prec$ tal que $\omega\in B$, entonces existe $a\in\mathcal{S}_\omega$ tal que $(a,\omega]\subseteq B$. Suponga que $B\cap \mathcal{S}_\omega=\emptyset$, en particular $(a,\omega]\cap\mathcal{S}_\omega=\emptyset$, es decir que:
        \begin{equation*}
            \mathcal{S}_\omega=\mathcal{S}_a\cup\left\{a\right\}
        \end{equation*}
        lo cual no puede suceder ya que entonces se tendría que $\mathcal{S}_\omega$ es numerable\contradiction. Por tanto, la intersección es no vacía, es decir que existe $x\in \mathcal{S}_\omega$ tal que $x\in B$.
    \end{proof}

    \section{Funciones Continuas}

    \begin{mydef}
        Sean $(X,d)$ y $(Y,\rho)$ dos espacios métricos, y $\cf{f}{(X,d)}{(Y,\rho)}$ una función. La función $f$ se dice una \textbf{función continua} si dado $x_0\in X$ y $\varepsilon\in\mathbb{R}^+$, existe $\delta\in\mathbb{R}^+$ tales que si
        \begin{equation*}
            x\in B_d(x_0,\delta)\Rightarrow f(x)\in B_\rho(f(x_0),\varepsilon)
        \end{equation*}
        que es equivalente a decir que $f(B_d(x_0,\delta))\subseteq B_\rho(f(x_0),\varepsilon)$.
    \end{mydef}

    \begin{propo}
        Sean $(X,d)$ y $(Y,\rho)$ dos espacios métricos, y $\cf{f}{(X,d)}{(Y,\rho)}$ una función. Entonces, $f$ es una función continua si y sólo si dado $U\subseteq Y$ abierto, $f^{-1}(U)\subseteq X$ es abierto.
    \end{propo}

    \begin{proof}
        $\Rightarrow)$: Suponga que $f$ es continua y sea $U\subseteq Y$ abierto. Si $x\in f^{-1}(U)$, entonces $f(x)\in U$. Como $U$ es abierto, existe $\varepsilon\in\mathbb{R}^+$ tal que $B_\rho(f(x),\varepsilon)\subseteq U$. Pero, como $f$ es continua entonces existe $\delta\in\mathbb{R}^+$ tal que 
        \begin{equation*}
            f(B_d(x,\delta))\subseteq B_\rho(f(x),\varepsilon)\subseteq U
        \end{equation*}
        es decir que $B_d(x,\delta)\subseteq f^{-1}(U)$. Por tanto, al ser $x\in f^{-1}(U)$ arbitrario, se sigue que $f^{-1}(U)$ es abierto.

        $\Leftarrow)$: Suponga que para todo $U\subseteq Y$ abierto, $f^{-1}(U)$ es abierto en $X$. Sean ahora $x_0\in X$ y $\varepsilon\in\mathbb{R}^+$. Como el conjunto $B_\rho(f(x_0),\varepsilon)$ es abierto, entonces $f^{-1}(B_\rho(f(x_0),\varepsilon))$ donde $x_0\in f^{-1}(B_\rho(f(x_0),\varepsilon))$, por ende existe $\delta\in\mathbb{R}^+$ tal que
        \begin{equation*}
            \begin{split}
                B_d(x_0,\delta)&\subseteq f^{-1}(B_\rho(f(x_0),\varepsilon))\\
                \Rightarrow f(B_d(x_0,\delta))&\subseteq B_\rho(f(x_0),\varepsilon)\\
            \end{split}
        \end{equation*}
        por tanto, como el $x_0\in X$ fue arbitrario se sigue que $f$ es continua en $X$.
    \end{proof}

    \begin{mydef}
        Sean $(X_1,\tau_1)$ y $(X_2,\tau_2)$ dos espacios topológicos y $\cf{f}{(X_1,\tau_1)}{(X_2,\tau_2)}$ una función. Decimos que $f$ es una \textbf{función continua} si para todo $U\in\tau_2$ se tiene que $f^{-1}(U)\in\tau_1$ (imágenes inversas de abiertos son abiertas).
    \end{mydef}

    \begin{exa}
        Sea $(X_1,\tau_1)$ un espacio topológico tal que $\tau_1$ es la topología discreta, es decir que $\tau_1=\mathcal{P}(X)$. Sea $(X_2,\tau_2)$ un espacio topológico arbitrario. Entonces, toda función $\cf{f}{(X_1,\tau_1)}{(X_2,\tau_2)}$ es continua.
    \end{exa}

    \begin{exa}
        Sea $(X_1,\tau_1)$ un espacio toplógico arbitrario y, sea $(X_2,\tau_2)$ un espacio topológico tal que $\tau_2=\tau_I$. Entonces, toda función $\cf{f}{(X_1,\tau_1)}{(X_2,\tau_2)}$ es continua.
    \end{exa}

    \begin{propo}
        Sean $(X_1,\tau_1)$ y $(X_2,\tau_2)$ dos espacios topológicos y $\cf{f}{(X_1,\tau_1)}{(X_2,\tau_2)}$ una función. Entonces, $f$ es una función continua si y sólo si dados $x\in X_1$ y $V\in\V{f(x)}$ existe $U\in\V{x}$ tal que $f(U)\subseteq V$.
    \end{propo}

    \begin{proof}
        $\Rightarrow)$: Suponga que $f$ es continua. Sea $x\in X_1$ y $V\in\V{f(x)}$, entonces existe $W\in\tau_2$ tal que $f(x)\in W\subseteq V$, es decir que $x\in f^{-1}(W)$ donde al ser $f$ continua se tiene que $f^{-1}(W)\in\tau_1$, esto es que $U=f^{-1}(W)\in\V{x}$. Además, $U=f^{-1}(W)\subseteq f(V)$.

        $\Leftarrow)$: Sea $V\in\tau_2$ y sea $x\in f^{-1}(V)$, entonces $f(x)\in V$ donde $V\in\V{f(x)}$. Así, por la tesis existe $U\in \V{x}$ tal que $f(U)\subseteq V$, lo cual implica que:
        \begin{equation*}
            x\in U\subseteq f^{-1}(f(U))\subseteq f^{-1}(V)
        \end{equation*}
        por tanto, $f^{-1}(V)\in\tau_1$.
    \end{proof}

    \begin{cor}
        Sean $(X_1,\tau_1)$ y $(X_2,\tau_2)$ dos espacios topológicos y $\cf{f}{(X_1,\tau_1)}{(X_2,\tau_2)}$ una función. Entonces, $f$ es continua si y sólo si dado $x\in X_1$ y dado $V\in\tau_2$ tales que $f(x)\in V$ existe $U\in\tau_1$ tal que $x\in U$ y $f(U)\subseteq V$.
    \end{cor}

    \begin{proof}
        Es inmediato de la proposición anterior.
    \end{proof}

    \begin{propo}
        Sean $(X_1,\tau_1)$ y $(X_2,\tau_2)$ dos espacios topológicos, $\mathcal{B}$ una base para $\tau_2$ y $\cf{f}{(X_1,\tau_1)}{(X_2,\tau_2)}$ una función. Entonces, $f$ es continua si y sólo si para todo $B\in\mathcal{B}$ se tiene que $f^{-1}(B)\in\tau_1$.
    \end{propo}

    \begin{proof}
        $\Rightarrow)$: Es inmediata.

        $\Leftarrow)$: Sea $U\in\tau_2$, como $\mathcal{B}$ es base, entonces existe $\left\{B_\alpha \right\}_{\alpha\in I}\subseteq\mathcal{B}$ tal que
        \begin{equation*}
            U=\bigcup_{\alpha\in I}B_\alpha
        \end{equation*}
        por lo cual:
        \begin{equation*}
            f^{-1}(U)=f^{-1}\left(\bigcup_{\alpha\in I}B_\alpha\right)=\bigcup_{\alpha\in I}f^{-1}(B_\alpha)
        \end{equation*}
        donde $f^{-1}(B_\alpha)\in\tau_1$, para todo $\alpha\in I$. Luego, $f^{-1}(U)\in\tau_1$.
    \end{proof}

    \begin{propo}
        Sean $(X_1,\tau_1)$ y $(X_2,\tau_2)$ dos espacios topológicos y, $\cf{f}{(X_1,\tau_1)}{(X_2,\tau_2)}$ una función. Entonces, lo siguientes enunciados son equivalentes:
        \begin{enumerate}
            \item $f$ es continua.
            \item $\forall A\subseteq X_1$, $f(\Cls{A})\subseteq\Cls{f(A)}$.
            \item Si $A\subseteq X_2$ cerrado, entonces $f^{-1}(A)$ es cerrado de $X_1$. 
        \end{enumerate}
        
        \begin{proof}
            $1)\Rightarrow 2)$: Sea $A\subseteq X_1$ y $x\in \Cls{A}$. Queremos ver que dado $V\subseteq X_2$ abierto tal que $f(x)\in X_2$ contiene puntos de $f(A)$, i.e. $V\cap f(A)\neq\emptyset$.
            
            Como $f$ es continua, entonces $f^{-1}(V)$ es abierto en $X_1$, luego como $x\in\Cls{A}$ entonces $f^{-1}(V)\cap A\neq\emptyset$, así existe $a\in f^{-1}(V)\cap A$ por lo cual $f(A)\in V$ y $f(a)\in f(A)$, así $V\cap f(A)\neq\emptyset$. Finalmente, se sigue que $f(\Cls{A})\subseteq\Cls{f(A)}$.

            $2)\Rightarrow 3)$: Sea $A\subseteq X_2$ cerrado. Por (2) se tiene que
            \begin{equation*}
                \Cls{f^{-1}(A)}\subseteq f^{-1}(f(\Cls{f^{-1}(A)}))\subseteq f^{-1}(\Cls{f(f^{-1}(A))})\subseteq f^{-1}(\Cls{A})=f^{-1}(A)
            \end{equation*}
            (la segunda contención se da por (2)) y pues $f(f^{-1}(A))\subseteq A$.
            
            $3)\Rightarrow 1)$: Sea $U\in\tau_2$, entonces $X_2-U$ es cerrado en $X_2$, luego $f^{-1}(X_2-U)=X_1-f^{-1}(U)$ siendo $f^{-1}(U)$ cerrado, luego $f^{-1}(U)\in\tau_1$.
        \end{proof}
    \end{propo}

    \begin{exa}
        Sean $(X_1,\tau_1)$ y $(X_2,\tau_2)$ dos espacios topológicos y $y_0\in X_2$. Definimos la función $\cf{\underline{y_0}}{(X_1,\tau_1)}{(X_2,\tau_2)}$ tal que $\forall x\in X_1$ se tiene que $\underline{y_0}(x)=y_0$. Esta función es una función continua y se llama una \textbf{función constante}.

        Sea $\cf{f}{(\mathbb{R},\tau_u)}{(\mathbb{R},\tau_I)}$ tal que para todo $x\in\mathbb{R}$, $f(x)=4$ (esto es que $f=\underline{4}$). Por lo anterior esta es una función continua. Se tiene por ende que:
        \begin{equation*}
            \Cls{f(\mathbb{N})}=\mathbb{R}\quad\textup{y}\quad f(\Cls{\mathbb{N}})=4
        \end{equation*}
        es decir que $\Cls{f(\mathbb{N})}\nsubseteq f(\Cls{\mathbb{N}})$.
    \end{exa}

    \begin{exa}
        Sean $(X_1,\tau_1)$, $(X_2,\tau_2)$ y $(X_3,\tau_3)$ tres espacios topológicos.
        \begin{enumerate}
            \item Tomemos $A\subseteq X_1$, y sea $\cf{i_A}{(A,\tau_{1_A})}{(X_1,\tau_1)}$ la función definida de por:
            \begin{equation*}
                \forall a\in A, i_A(a)=a
            \end{equation*}
            tenemos que esta función es una función continua y se llama \textbf{la función inclusión de $A$ en $X_1$}. Además, tenemos que para $U\in \tau_1$, $i^{-1}(U)=U\cap A\in\tau_{1_A}$.
            \item Sea $\cf{f}{(X_1,\tau_1)}{(X_2,\tau_2)}$ una función continua.
            \begin{enumerate}
                \item Si $B\subseteq X_2$ cumple que $f(X_1)=B$, entonces, la función $\cf{F}{(X_1,\tau_1)}{(B,\tau_{2_B})}$ definida para todo $x\in X_1$, $F(x)=f(x)$ es continua. $F$ se dice que es \textbf{una reestricción del rango de $f$}.
                
                La continuidad se sigue del hecho de que si $U\in\tau_{2_B}$, entonces existe $V\in\tau_2$ tal que $U=V\cap B$, luego
                \begin{equation*}
                    F^{-1}(U)=f^{-1}(U)=f^{-1}(V\cap B)=f^{-1}(V)\cap f^{-1}(B)=f^{-1}(V)\cap X_1=f^{-1}(V)
                \end{equation*}
                donde el miembro de la derecha está en $\tau_1$. Por tanto, $F$ es continua.
                \item Si $(X_3,\tau_3)$ es un espacio topológico que tiene a $(X_2,\tau_2)$ como subespacio, entonces la función $\cf{F}{(X_1,\tau_1)}{(X_3,\tau_3)}$ tal que $\forall x\in X_1$, $F(x)=f(x)$ se llama \textbf{una expansión del rango de $f$}.
                
                La continuidad se sigue del hecho de que si $U\in\tau_3$, entonces:
                \begin{equation*}
                    F^{-1}(U)=F^{-1}(U\cap X_2)\cup F^{-1}(U\cap (X_3-X_2))=F^{-1}(U\cap X_2)\cup\emptyset=f^{-1}(U\cap X_2)
                \end{equation*}
                donde el mimebro de la derecha está en $\tau_1$ ya que $U\cap X_2$ está en $\tau_2$ por ser $f$ continua.

                \item Si $A\subseteq X_1$, entonces la función $\cf{f_A}{(A,\tau_{1_A})}{(X_2,\tau_2)}$ tal que para todo $x\in A$, $f_A(x)=f(x)$ es una función continua y se dice \textbf{la función reestringida (del dominio) de $f$ al conjunto $A$}. Esta función también es continua.
            \end{enumerate}
            \item Si $\cf{f}{(X_1,\tau_1)}{(X_2,\tau_2)}$ es una función y $\left\{A_\alpha\right\}_{ \alpha\in I}\subseteq \tau_1$ tal que:
            \begin{equation*}
                \bigcup_{\alpha\in I}A_\alpha=X_1
            \end{equation*}
            entonces, $f$ es continua si y sólo si $\forall \alpha\in I$, $\cf{f_{A_\alpha}}{(A_\alpha,\tau_{1_{A_\alpha}})}{(X_2,\tau_2)}$ es una función continua.
            \item Si $\cf{f}{(X_1,\tau_1)}{(X_2,\tau_2)}$ es una función y $\left\{ C_\alpha\right\}_{\alpha\in I}$ es una familia \textit{finita} de conjuntos cerrado sen $(X_1,\tau_1)$, tal que
            \begin{equation*}
                \bigcup_{\alpha\in I}C_\alpha=X_1
            \end{equation*}
            entonces, $f$ es continua si y sólo si $\forall \alpha\in I$, $\cf{f_{C_\alpha}}{(C_\alpha,\tau_{1_{C_\alpha}})}{(X_2,\tau_2)}$ es una función continua.

            \item Sean $A,B$ subconjuntos abiertos (respectivamente, cerrados) de $(X_1,\tau_1)$ tales que $X_1=A\cup B$. Si $\cf{f_1}{(A,\tau_{1_A})}{(X_2,\tau_2)}$ y $\cf{f_2}{(B,\tau_{1_B})}{(X_2,\tau_2)}$ son funciones continuas tales que para todo $x\in A\cap B$ se tiene que $f_1(x)=f_2(x)$, entonces, la función $\cf{f}{(X_1,\tau_1)}{(X_2,\tau_2)}$ definida como:
            \begin{equation*}
                \forall x\in X_1,\quad f(x)=\left\{ \begin{array}{lcr}
                    f_1(x) &\textup{ si }& x\in A\\
                    f_2(x) &\textup{ si }& x\in B\\
                \end{array}\right.
            \end{equation*}
            es una función continua.

            \item Si $\cf{f}{(X_1,\tau_1)}{(X_2,\tau_2)}$ y $\cf{g}{(X_2,\tau_2)}{(X_3,\tau_3)}$ son funciones continuas, entonces $\cf{g\circ f}{(X_1,\tau_1)}{(X_3,\tau_3)}$ es continua.
        \end{enumerate}
    \end{exa}

    \begin{proof}
        De (3): La necesidad es inmediata del inciso anterior.

        Para la suficiencia, suponga que para todo $\alpha\in I$, la función $f_{A_\alpha}$ es continua. Sea $U\in\tau_2$, entonces por hipótesis se tiene que $f_{A_\alpha}^{-1}(U)\in\tau_{1_{A_\alpha}}$, por tanto:
        \begin{equation*}
            \forall \alpha\in I, f^{-1}(U)\cap A_\alpha\in \tau_{1_{A_\alpha}}
        \end{equation*}
        por otro lado,
        \begin{equation*}
            f^{-1}(U)=f^{-1}(U)\cap X_1=f^{-1}(U)\cap\left(\bigcup_{\alpha\in I}A_\alpha \right)=\left(\bigcup_{\alpha\in I}f^{-1}(U)\cap A_\alpha \right)
        \end{equation*}
        donde todos los miembros de unión en la derecha están en $\tau_1$ ya que cada $A_\alpha$ es abierto, y, al ser $f^{-1}(U)$ abierto en $\tau_{A_\alpha}$, se sigue que $f^{-1}(U)$ debe ser elemento de $\tau_1$. Por tanto, $f^{-1}(U)$ es abierto.

        De (4): La necesidad es inmediata de un inciso anterior.

        Para la suficiencia... %TODO

        De (5): Supongamos que $A,B$ son abiertos. Sea $U\in\tau_2$. Como $f_1$ y $f_2$ son continuas, entonces $f_1^{-1}(U)\in\tau_{1_A}$ y $f_2^{-1}(U)\in\tau_{1_B}$, como $A$ y $B$ son abiertos en $(X_1,\tau_1)$, entonces $f_1^{-1}(U)$ y $f_2^{-1}(U)$ son abiertos en $(X_1,\tau_1)$. Por tanto,
        \begin{equation*}
            f^{-1}(U)=f_1^{-1}(U)\cup f_2^{-1}(U)\in\tau_1
        \end{equation*}
        se sigue entonces que $f$ es continua.

        De (6): Sea $U\in\tau_3$, entonces $g^{-1}(U)\in \tau_2$, por lo cual $f^{-1}(g^{-1}(U))\in\tau_1$. Como:
        \begin{equation*}
            (g\circ f)^{-1}(U)=f^{-1}(g^{-1}(U))\in\tau_1
        \end{equation*}
        entonces, se sigue que $g\circ f$ es continua.
    \end{proof}

    \section{Funciones abiertas, cerradas y homemorfismos}

    \begin{mydef}
        Sean $(X_1,\tau_1)$ y $(X_2,\tau_2)$ espacios topológicos y $\cf{f}{(X_1,\tau_1)}{(X_2,\tau_2)}$ una función entre ellas.
        \begin{enumerate}
            \item Decimos que $f$ es una \textbf{función abierta}, si para todo $U\in\tau_1$, $f(U)\in\tau_2$.
            \item Decimos que $f$ es una \textbf{función cerrada}, si para todo $X_1-U\in\tau_1$, $X_2-f(U)\in\tau_2$ (en otras palabras, si $U\subseteq X_1$ es cerrado, entonces $f(U)\subseteq X_2$ es cerrado).
        \end{enumerate}
    \end{mydef}

    Veamos ejemplos de funciones que sean abiertas, cerradas y continuas.

    Considere $X=\left\{a,b \right\}$ con $a\neq b$, $\tau_1=\left\{ X,\emptyset,\left\{a\right\} \right\}$ y $\tau_2=\left\{ X,\emptyset,\left\{b\right\} \right\}$

    \begin{enumerate}
        \item Tomemos $\cf{id}{(X,\tau_1)}{(X,\tau_2)}$. Esta función no es continua, tampoco es abierta ni cerrada.
        \item Tomemos $\cf{id}{(X,\tau_I=\left\{X,\emptyset \right\})}{(X,\tau_D)}$. Esta función no es continua, pero si es abierta y cerrada.
        \item Tomemos $\cf{id=id^{-1}}{(X,\tau_D)}{(X,\tau_I=\left\{X,\emptyset \right\})}$. Esta función es continua, pero no es abierta ni cerrada.
        \item Tomemos $\cf{id=id^{-1}}{(X,\tau_1)}{(X,\tau_1)}$. Esta función es continua, abierta y cerrada.
        \item Tomemos $\cf{\underline{a}}{(X,\tau_D)}{(X,\tau_1)}$. Esta función es continua, abierta pero no es cerrada.
        \item Tomemos $\cf{\underline{b}}{(X,\tau_D)}{(X,\tau_1)}$. Esta función es continua, pero no es abierta y sí es cerrada.
        \item Tomemos $\cf{f}{(X,\tau_D)}{(\mathbb{R},\tau_u)}$ tal que $f(a)=0$ y $f(b)=1$. Esta función es continua, pero no abierta ni cerrada.
    \end{enumerate}

    Los ejemplos anteriores se resumen en la siguinte tabla.

    \begin{center}
        \begin{tabular}{c|c|c|c}
            Ejemplo & Continua & Abierta & Cerrada \\
            \hline
            1 & $\times$ & $\times$ & $\times$ \\
            2 & $\times$ & $\surd$ & $\surd$ \\
            3 & $\surd$ & $\times$ & $\times$ \\
            4 & $\surd$ & $\surd$ & $\surd$ \\
            5 & $\surd$ & $\surd$ & $\times$ \\
            6 & $\surd$ & $\times$ & $\surd$ \\
            7 & $\surd$ & $\times$ & $\times$ \\
        \end{tabular}
    \end{center}

    \begin{mydef}
        Sean $(X_1,\tau_1)$ y $(X_2,\tau_2)$ espacios topológicos. Se dice que los espacios topológicos son \textbf{homeomorfos}, si existe una función $\cf{h}{(X_1,\tau_1)}{(X_2,\tau_2)}$ biyectiva tal que $h$ y $h^{-1}$ son continuas y, en tal caso, se dice que $h$ es un \textbf{homeomorfismo entre los espacios $(X_1,\tau_1)$ y $(X_2,\tau_2)$}, o simplemente un \textbf{homeomorfismo}, y se escribe
        \begin{equation*}
            (X_1,\tau_1) \simeq (X_2,\tau_2)
        \end{equation*}
    \end{mydef}
    
    \begin{propo}
        Sean $(X_1,\tau_1)$ y $(X_2,\tau_2)$ espacios topológicos, y $\cf{f}{(X_1,\tau_1)}{(X_2,\tau_2)}$ una función biyectiva. Entonces, los siguientes enunciados son equivalentes:
        \begin{enumerate}
            \item $f$ es un homeomorfismo.
            \item $f$ es continua y abierta.
            \item $f$ es continua y cerrada.
            \item $A\subseteq X_2$ es cerrado si y sólo si $f^{-1}(A)\subseteq X_1$ es cerrado.
            \item $A\in\tau_2$ si y sólo si $f^{-1}(A)\in\tau_1$.
            \item Si $\mathcal{B}$ es una base para $\tau_1$, entonces la colección
            \begin{equation*}
                f(\mathcal{B}):=\left\{f(B)\Big|B\in\mathcal{B} \right\}
            \end{equation*}
            es una base para $\tau_2$.
        \end{enumerate}
    \end{propo}

    \begin{proof}
        $1)\Rightarrow 2)$: Es claro que $f$ es continua. Por ser homeomorfismo, se sigue que $f^{-1}$ también es abierta. Para ver que $f$ es abierta, sea $U\in\tau_1$, entonces
        \begin{equation*}
            f(U)=(f^{-1})^{-1}(U)
        \end{equation*}
        luego, como $f^{-1}$ es continua, se sigue que $f(U)\in\tau_2$.

        $2)\Rightarrow 3)$: Ya se tiene que $f$ es continua. Sea $C\subseteq X_1$ cerrado. Como $f$ es abierta, entonces el conjunto $X_2-f(U)\in\tau_2$, luego $f(U)$ es cerado.

        $3)\Rightarrow 4)$: Haremos la doble implicación.

        $\Rightarrow)$: Sea $A\subseteq X_2$ cerrado, como $f$ es continua, entonces $f^{-1}(A)\subseteq X_1$ es cerrado.

        $\Leftarrow)$: Sea $A\subseteq X_2$ tal que $f^{-1}(A)\subseteq X_1$ es cerrado. Como $f$ es cerrada, entonces $f(f^{-1}(A))=A$ (pues es biyección) es cerrado.

        $4)\Rightarrow 5)$: $A\in\tau_2$ sii $X_2-A$ es un cerrado sii $f^{-1}(X_2-A)$ es un cerrado sii $X_1-f^{-1}(A)$ es cerrado sii $f^{-1}(A)\in\tau_1$.

        $5)\Rightarrow 6)$: Sea $\mathcal{B}$ una base de $\tau_1$. Tenemos que dado $B\in\mathcal{B}$, $B\in\tau_1$. Como $f$ es biyectiva, entonces $B=f^{-1}(f(B))$, luego $f(B)\in\tau_2$.

        Por tanto, $f(\mathcal{B})\subseteq \tau_2$. Ahora, si $A\in\tau_2$, por lo anterior se tiene que $f^{-1}(A)\in\tau_1$, luego existe una subcolección $\left\{B_\alpha \right\}_{\alpha\in I}$ tal que:
        \begin{equation*}
            f^{-1}(A)=\bigcup_{\alpha\in I}B_\alpha
        \end{equation*}
        luego:
        \begin{equation*}
            A=f(f^{-1}(A))=f\left(\bigcup_{\alpha\in I}B_\alpha\right)=\bigcup_{\alpha\in I}f(B_\alpha)
        \end{equation*}
        donde $f(B_\alpha)\in f(\mathcal{B})$ para todo $\alpha\in I$. Por tanto, $f(\mathcal{B})$ es una base para $\tau_2$.

        $6)\Rightarrow 1)$: Sea $\mathcal{B}$ una base para $\tau_1$. Por hipótesis se tiene que $f(\mathcal{B})$ es una base de $\tau_2$.

        Sea ahora $U\in\tau_2$, entonces existe $\left\{B_\alpha \right\}_{\alpha\in I}\subseteq\mathcal{B}$ tal que 
        \begin{equation*}
            U=\bigcup_{\alpha\in I}f(B_\alpha)
        \end{equation*}
        luego,
        \begin{equation*}
            f^{-1}(U)=f^{-1}\left(\bigcup_{\alpha\in I}f(B_\alpha)\right)=\bigcup_{\alpha\in I}f^{-1}(f(B_\alpha))=\bigcup_{\alpha\in I}B_\alpha
        \end{equation*}
        donde el lado de la derecha es abierto por ser $\mathcal{B}$ base de $\tau_1$. Así, $f$ es continua.

        Sea ahora $V\in\tau_1$, luego existe $\left\{B_\beta \right\}_{\beta\in J}$ tal que:
        \begin{equation*}
            V=\bigcup_{\beta\in J}B_\beta
        \end{equation*}
        luego,
        \begin{equation*}
            f(V)=f\left(\bigcup_{\beta\in J}B_\beta\right)=\bigcup_{\beta\in J}f(B_\beta)
        \end{equation*}
        donde los elementos de la izquierda están en $f(\mathcal{B})$, así $f(V)=(f^{-1})^{-1}(V)$ es abierto en $X_2$. Por ende, $f^{-1}$ es continua.

        Como ambas funciones $f$ y $f^{-1}$ son continuas, se sigue que $f$ es homeomorfismo.
    \end{proof}

    \begin{cor}
        Sean $(X_1,\tau_1)$ y $(X_2,\tau_2)$ espacios topológicos, y $\cf{f}{(X_1,\tau_1)}{(X_2,\tau_2)}$ una función biyectiva. Entonces, $f$ es un homeomorfismo $\iff$ $f$ es continua y abierta $\iff$ $f$ es continua y cerrada.
    \end{cor}

    \begin{proof}
        Es inmediato del teorema anterior.
    \end{proof}

    \begin{excer}
        Sea $X$ un conjunto y $\tau_1,\tau_2$ dos topologías sobre $X$. Sea $\cf{\id{X}}{(X,\tau_1)}{(X,\tau_2)}$. Entonces, $\id{X}$ es homeomorfismo si y sólo si $\tau_1=\tau_2$.
    \end{excer}

    \begin{proof}
        Se probarán las dos implcaciones.

        $\Rightarrow)$: 

        $\Leftarrow)$:

    \end{proof}

    \begin{mydef}
        Las propiedades de los espacios topológicos que se pueden definir por medio de sus subconjuntos abiertos (de forma equivalente, cerrados), son llamados \textbf{invariantes con respecto a homeomorfismos}.

        Así, diremos que una propiedad $P$ que cumple un espacio topológico es una \textbf{propiedad topológica} si todo espacio topológico que es homeomorfo a $(X,\tau)$ posee esa propiedad, es decir $P$ puede considerarse como una propiedad de la clase de los espacios topológicamente equivalentes a $(X,\tau)$.
    \end{mydef}

    \begin{propo}
        La propiedad de ser un espacio de Hausdorff es topológica.
    \end{propo}

    \begin{proof}
        Sean $(X_1,\tau_1)$ y $(X_2,\tau_2)$ espacios topológicos, $\cf{f}{(X_1,\tau_1)}{(X_2,\tau_2)}$ un homeomorfismo entre ellos. Supongamos que $(X_1,\tau_1)$ es Hausdorff, probaremos que $(X_2,\tau_2)$ también lo es.

        En efecto, sean $p,q\in X_2$, $p\neq q$. Como $f^{-1}(p),f^{-1}(q)\in X_1$ son distintos por ser $f$ biyectiva y al ser $(X_1,\tau_1)$ Hausdorff, entonces existen abiertos $V_1,V_2\in\tau_1$ tales que
        \begin{equation*}
            f^{-1}(p)\in V_1,\quad f^{-1}(q)\in V_2,\quad V_1\cap V_2=\emptyset
        \end{equation*}
        Tomemos $U_1=f(V_1)$ y $U_2=f(V_2)$ abiertos en $(X_1,\tau_1)$ ya que $f$ es una función abierta. Se tiene que $p\in U_1$, $q\in U_2$ y:
        \begin{equation*}
            U_1\cap U_2=f(V_1)\cap f(V_2)=f(V_1\cap V_2)=f(\emptyset)=\emptyset
        \end{equation*}
        por tanto, el espacio $(X_2,\tau_2)$ es Hausdorff.
    \end{proof}

    \begin{excer}
        La propiedad de ser un espacio metrizable es topológica.
    \end{excer}

    \begin{proof}
        
    \end{proof}

    \begin{excer}
        La propiedad de tener una base numerable es topológica.
    \end{excer}

    \begin{proof}
        
    \end{proof}

    \section{Topología Producto}

    \begin{propo}
        Sean $Y$ un conjunto y $(X,\tau)$ un espacio topológico. Si $\cf{f}{Y}{X}$ es una función, entonces
        \begin{equation*}
            \tau'=\left\{f^{-1}(U)\Big|U\in\tau \right\}
        \end{equation*}
        es una topología sobre $Y$ y es la topología más gruesa definida sobre $Y$ tal que $\cf{f}{(Y,\tau')}{(X,\tau)}$ es una función continua.
    \end{propo}

    \begin{proof}
        Primero veremos que $\tau'$ es una topología sobre $Y$.
        \begin{enumerate}
            \item Es claro que $Y=f^{-1}(X),\emptyset=f^{-1}(\emptyset)\in\tau'$ ya que $X,\emptyset\in\tau$.
            \item Sean $M_1,M_2\in\tau'$, entonces existen $U_1,U_2\in\tau$ tales que $M_1=f^{-1}(U_1)$ y $M_2=f^{-1}(U_2)$, luego:
            \begin{equation*}
                M_1\cap M_2=f^{-1}(U_1)\cap f^{-1}(U_2)=f^{-1}(U_1\cap U_2)\in\tau'
            \end{equation*}
            pues, $U_1\cap U_2\in\tau$.
            \item  Sea $\left\{M_\alpha \right\}_{\alpha\in I}=\mathcal{A}\subseteq\tau'$ una colección de elementos de $\tau'$, entonces existe $\left\{U_\alpha \right\}_{\alpha\in I}\subseteq \tau$ tal que
            \begin{equation*}
                M_\alpha=f^{-1}(U_\alpha),\quad\forall\alpha\in I
            \end{equation*}
            luego,
            \begin{equation*}
                \begin{split}
                    \bigcup_{\alpha\in I}M_\alpha&=\bigcup_{\alpha\in I}M_\alpha\\
                    &=\bigcup_{\alpha\in I}f^{-1}(U_\alpha)\\
                    &=f^{-1}\left(\bigcup_{\alpha\in I}U_\alpha \right)\\
                \end{split}
            \end{equation*}
            donde $\bigcup_{\alpha\in I}U_\alpha\in\tau$, por ende $\bigcup_{\alpha\in I}M_\alpha\in\tau'$.
        \end{enumerate}
        por los tres incisos anteriores, se sigue que $\tau'$ es una topología sobre $Y$.

        Es claro que la función $f$ es continua (pues imagen inversa de abiertos simpre es abierta). Sea ahora $\sigma$ una topología sobre $Y$ tal que $\cf{f}{(Y,\tau)}{(X,\tau)}$ es una función continua. Hay que probar que $\tau'\subseteq\sigma$. En efecto, ...
    \end{proof}

    \begin{mydef}
        Sean $\left\{X_\alpha \right\}_{\alpha\in I}$ una familia no vacía de subconjuntos de $X$ y $\cf{x}{I}{\bigcup_{\alpha\in I}X_\alpha}$ tal que para todo $\alpha\in I$, $x(\alpha)\in X_\alpha$. Denotamos a
        \begin{equation*}
            x(\alpha):=x_\alpha,\quad\forall \alpha\in I
        \end{equation*}
        y, a la función $x$ la denotaremos simplemente por $\left(x_\alpha \right)_{\alpha\in I}$. Formamos así el conjunto:
        \begin{equation*}
            \prod_{\alpha\in I}X_\alpha=\left\{\cf{x}{I}{\bigcup_{\alpha\in I}X_\alpha}\Big|x\textup{ es una función tal que para todo }\alpha\in I,x_\alpha\in X_\alpha \right\}
        \end{equation*}
        le llamaremos el \textbf{producto cartesiano de la famila $\left\{X_\alpha \right\}_{\alpha\in I}$}.
    \end{mydef}

\end{document}