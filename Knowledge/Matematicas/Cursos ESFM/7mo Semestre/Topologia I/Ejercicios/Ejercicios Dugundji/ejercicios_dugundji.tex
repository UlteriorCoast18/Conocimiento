\documentclass[12pt]{report}
\usepackage[spanish]{babel}
\usepackage[utf8]{inputenc}
\usepackage{amsmath}
\usepackage{amssymb}
\usepackage{amsthm}
\usepackage{graphics}
\usepackage{subfigure}
\usepackage{lipsum}
\usepackage{array}
\usepackage{multicol}
\usepackage{enumerate}
\usepackage[framemethod=TikZ]{mdframed}
\usepackage[a4paper, margin = 1.5cm]{geometry}

%En esta parte se hacen redefiniciones de algunos comandos para que resulte agradable el verlos%

\renewcommand{\theenumii}{\roman{enumii}}

\def\proof{\paragraph{Demostración:\\}}
\def\endproof{\hfill$\blacksquare$}

\def\sol{\paragraph{Solución:\\}}
\def\endsol{\hfill$\square$}

%En esta parte se definen los comandos a usar dentro del documento para enlistar%

\newtheoremstyle{largebreak}
  {}% use the default space above
  {}% use the default space below
  {\normalfont}% body font
  {}% indent (0pt)
  {\bfseries}% header font
  {}% punctuation
  {\newline}% break after header
  {}% header spec

\theoremstyle{largebreak}

\newmdtheoremenv[
    leftmargin=0em,
    rightmargin=0em,
    innertopmargin=0pt,
    innerbottommargin=5pt,
    hidealllines = true,
    roundcorner = 5pt,
    backgroundcolor = gray!60!red!30
]{exa}{Ejemplo}[section]

\newmdtheoremenv[
    leftmargin=0em,
    rightmargin=0em,
    innertopmargin=0pt,
    innerbottommargin=5pt,
    hidealllines = true,
    roundcorner = 5pt,
    backgroundcolor = gray!50!blue!30
]{obs}{Observación}[section]

\newmdtheoremenv[
    leftmargin=0em,
    rightmargin=0em,
    innertopmargin=0pt,
    innerbottommargin=5pt,
    rightline = false,
    leftline = false
]{theor}{Teorema}[section]

\newmdtheoremenv[
    leftmargin=0em,
    rightmargin=0em,
    innertopmargin=0pt,
    innerbottommargin=5pt,
    rightline = false,
    leftline = false
]{propo}{Proposición}[section]

\newmdtheoremenv[
    leftmargin=0em,
    rightmargin=0em,
    innertopmargin=0pt,
    innerbottommargin=5pt,
    rightline = false,
    leftline = false
]{cor}{Corolario}[section]

\newmdtheoremenv[
    leftmargin=0em,
    rightmargin=0em,
    innertopmargin=0pt,
    innerbottommargin=5pt,
    rightline = false,
    leftline = false
]{lema}{Lema}[section]

\newmdtheoremenv[
    leftmargin=0em,
    rightmargin=0em,
    innertopmargin=0pt,
    innerbottommargin=5pt,
    roundcorner=5pt,
    backgroundcolor = gray!30,
    hidealllines = true
]{mydef}{Definición}[section]

\newmdtheoremenv[
    leftmargin=0em,
    rightmargin=0em,
    innertopmargin=0pt,
    innerbottommargin=5pt,
    roundcorner=5pt
]{excer}{Ejercicio}[section]

%En esta parte se colocan comandos que definen la forma en la que se van a escribir ciertas funciones%

\newcommand\abs[1]{\ensuremath{\left|#1\right|}}
\newcommand\divides{\ensuremath{\bigm|}}
\newcommand\cf[3]{\ensuremath{#1:#2\rightarrow#3}}
\newcommand\contradiction{\ensuremath{\#_c}}
\newcommand\natint[1]{\ensuremath{\left[\!\left[#1\right]\!\right] }}

\newcommand{\Int}[1]{\ensuremath{\mathring{#1}}}
\newcommand{\Cls}[1]{\ensuremath{\overline{#1}}}
\newcommand{\Fr}[1]{\ensuremath{\textup{Fr}\left(#1\right)}}
\newcommand{\Ext}[1]{\ensuremath{\textup{Ext}\left(#1\right)}}

\begin{document}
    \setlength{\parskip}{5pt} % Añade 5 puntos de espacio entre párrafos
    \setlength{\parindent}{12pt} % Pone la sangría como me gusta
    \title{Ejercicios Dugundji Topology y Problemas Varios}
    \author{Cristo Daniel Alvarado}
    \maketitle

    \tableofcontents %Con este comando se genera el índice general del libro%

    %\setcounter{chapter}{3} %En esta parte lo que se hace es cambiar la enumeración del capítulo%

    \chapter{Espacios Topológicos}

    \section{Conceptos Fundamentales}

    \begin{obs}
        El símbolo $\aleph(X)$, donde $X$ es un conjunto, denota al cardinal del conjunto (realmente denota a otra cosa que viene a ser lo mismo, pero para usos prácticos tomaremos lo anterior como cierto).
    \end{obs}

    \begin{excer}
        Pruebe lo siguiente:
        \begin{enumerate}
            \item Sea $X$ un conjunto infinto. Pruebe que $\mathcal{A}_0=\left\{A\subseteq X \big|X-A\textup{ es finito} \right\}\cup\left\{\emptyset \right\}$ es una topología sobre $X$.
            \item Sea $\aleph(X)\geq\aleph_0$. Pruebe que $\mathcal{A}_1=\left\{A\subseteq X \big|\aleph(X-A)<\aleph(X) \right\}\cup\left\{\emptyset \right\}$ es una topología sobre $X$.
        \end{enumerate}
    \end{excer}

    \begin{proof}
        De (1): Es la topología de los complementos finitos (la prueba de esto se hizo en las notas).

        De (2): Veamos que se verifican las tres condiciones:
        \begin{enumerate}
            \item Por definición de $\mathcal{A}_1$ se tiene que $\emptyset\in\mathcal{A}_1$ y, como $\aleph(\emptyset)<\aleph_0$, entonces $\aleph(X-X)<\aleph(X)$, por ende $X\in\mathcal{A}_1$.
            \item Sea $\mathcal{E}$ una subfamilia no vacía arbitraria de $\mathcal{A}_1$. Considere a $\bigcup\mathcal{E}$. Como la familia es no vacía, existe $E_0\in\mathcal{E}$, se tiene así que:
            \begin{equation*}
                \begin{split}
                    E_0\subseteq \bigcup\mathcal{E}\Rightarrow& X- \bigcup\mathcal{E}\subseteq X-E_0\\
                    \Rightarrow& \aleph\left(X- \bigcup\mathcal{E}\right) \subseteq \aleph(X-E_0) \\
                \end{split}
            \end{equation*}
            por Cantor-Bernstein. Por lo cual al tenerse que $\bigcup\mathcal{E}\subseteq X$, se sigue que $\bigcup\mathcal{E}\in\mathcal{A}_1$.
            \item Sean $A,B\in\mathcal{A}_1$, entonces $\aleph(X-A)<\aleph(X)$ y $\aleph(X-B)<\aleph(X)$. Notemos que
            \begin{equation*}
                X-(A\cap B)=(X-A)\cup (X-B)
            \end{equation*}
            Entonces $\aleph(X-(A\cap B))=\aleph((X-A)\cup (X-B))\leq\aleph(X-A)+\aleph(X-B)< \aleph(X)+\aleph(X)=2\aleph(X)=\aleph(X)$, pues $\aleph(X)\geq\aleph_0$. Por tanto, al ser $A\cap B\subseteq X$, se sigue que $A\cap B\in \mathcal{A}_1$.
        \end{enumerate}
        Por las tres condiciones anteriores, se sigue que $\mathcal{A}_1$ es una topología sobre $X$.
    \end{proof}

    \begin{excer}
        ¿Cuántas topologías distintas puede tener un conjunto de tres elemento? ¿Cuál es su orden parcial?
    \end{excer}

    \begin{sol}
        Considere $X=\left\{a,b,c\right\}$. De todas las topologías que puede tener, deben de estar al menos la topología discreta y la indiscreta, formada por los conjuntos:
        \begin{equation*}
            \begin{split}
                \tau_D&=\left\{\emptyset, \left\{a\right\},\left\{b\right\}, \left\{c\right\}, \left\{a,b\right\},\left\{b,c\right\},\left\{c,a\right\},\left\{a,b,c\right\} \right\}=\mathcal{P}(\left\{a,b,c\right\})\\
                \tau_I&=\left\{\emptyset,\left\{a,b,c\right\} \right\}
            \end{split}
        \end{equation*}
        Ahora, las otras que se pueden tener son aquellas que solo contienen a uno de los elementos, es decir las siguientes:
        \begin{equation*}
            \begin{split}
                \tau_a&=\left\{\emptyset,\left\{a\right\},\left\{a,b,c\right\}\right\}\\
                \tau_b&=\left\{\emptyset,\left\{b\right\},\left\{a,b,c\right\}\right\}\\
                \tau_c&=\left\{\emptyset,\left\{c\right\},\left\{a,b,c\right\}\right\}\\
            \end{split}
        \end{equation*}
        y, también aquellas que contengan a un par de elementos, pero de esta forma: $\left\{a,b\right\}$, que serían las siguientes:
        \begin{equation*}
            \begin{split}
                \tau_{a,b}&=\left\{\emptyset,\left\{a,b\right\},\left\{a,b,c\right\}\right\}\\
                \tau_{b,c}&=\left\{\emptyset,\left\{b,c\right\},\left\{a,b,c\right\}\right\}\\
                \tau_{c,a}&=\left\{\emptyset,\left\{c,a\right\},\left\{a,b,c\right\}\right\}\\
            \end{split}
        \end{equation*}
        (en esta se verifica casi de forma inmediata que es una topología sobre $X$). Ahora, se deben considerar aquellas en las que se tiene más de un elemento no trivial (cuando menciono la palabra trivial, me refiero a que no sea alguno de $\emptyset$ o $X=\left\{a,b,c\right\}$). Por ejemplo, consideremos a $\left\{a,b\right\}$ un elemento no trivial, y sea $\tau$ una topología sobre $X$ que contiene a este elemento. Se tienen seis casos:
        \begin{enumerate}
            \item ${a}\in\tau$, entonces al ser cerrado bajo uniones e intersecciones se tiene que (al menos) $\tau$ debe ser de la forma:
            \begin{equation*}
                \tau=\left\{\emptyset,\left\{a\right\},\left\{a,b\right\},\left\{a,b,c\right\}\right\}
            \end{equation*}
            \item $\left\{b\right\}\in\tau$, como con el caso anterior, se tendría que (al menos) $\tau$ debe ser de la forma:
            \begin{equation*}
                \tau=\left\{\emptyset,\left\{b\right\},\left\{a,b\right\},\left\{a,b,c\right\}\right\}
            \end{equation*}
            Ahora, si $\left\{a\right\}\in\tau$, entonces (al menos) $\tau$ debe ser de la forma:
            \begin{equation*}
                \tau=\left\{\emptyset,\left\{a\right\},\left\{b\right\},\left\{a,b\right\},\left\{a,b,c\right\}\right\}
            \end{equation*}
            \item $\left\{c\right\}\in\tau$, se tiene entonces que una topología sobre $X$ (al menos), debe ser:
            \begin{equation*}
                \tau=\left\{\emptyset,\left\{c\right\},\left\{a,b\right\},\left\{a,b,c\right\} \right\}
            \end{equation*}
            \item $\left\{b,c\right\}\in\tau$, se tiene entonces que $\tau$ debe ser de la forma (al menos):
            \begin{equation*}
                \tau=\left\{\emptyset,\left\{b\right\},\left\{b,c\right\},\left\{a,b\right\},\left\{a,b,c\right\} \right\}
            \end{equation*}
        \end{enumerate}
        Son un vergo, nmms.
    \end{sol}

    \begin{excer}
        Sean $\tau_X$ y $\tau_Y$ dos topologías en $X$ y $Y$, respectivamente. ¿Es
        \begin{equation*}
            \tau=\left\{A\times B\big| A\in\tau_X, B\in\tau_Y \right\}
        \end{equation*}
        una topología en $X\times Y$?
    \end{excer}

    \begin{sol}
        Veamos si se cumplen las tres condiciones para que $\tau$ sea una topología sobre $X$.
        \begin{enumerate}
            \item Es claro que $\emptyset,X\times Y \in\tau$, pues $\emptyset\in\tau_X,\tau_Y$ y, $X\in\tau_X$ y $Y\in\tau_Y$.
            \item Sea $\mathcal{C}$ una subfamilia no vacía de $\tau$. Entonces, cada elemento de $\mathcal{C}=\left\{C_\alpha\big|\alpha\in I \right\}$ es de la forma:
            \begin{equation*}
                C_\alpha=A_\alpha\times B_\alpha
            \end{equation*}
            donde $A_\alpha\in \tau_X$ y $B_\alpha\in\tau_Y$, para todo $\alpha\in I$. Luego:
            \begin{equation*}
                \begin{split}
                    \bigcup_{\alpha\in I}C_\alpha&=\bigcup_{\alpha\in I}A_\alpha\times B_\alpha \\
                \end{split}
            \end{equation*}
            Veamos que en general no es cierto que $\bigcup_{\alpha\in I}C_\alpha\in\tau$. En efecto, tomemos $X=Y=\mathbb{R}$ (con la topología usual) y como conjuntos de la familia a: $C_1=(0,1)\times(0,1)$, y $C_2=(1,2)\times(1,2)$. Se tiene que:
            \begin{equation*}
                C_1\cup C_2\notin\tau
            \end{equation*}
            ya que, en caso contrario se tendría que $C_1\cup C_2=A\times B$, con $A,B\subseteq\mathbb{R}$ abiertos con la topología usual. 
            
            Entonces, en particular los elementos $(\frac{1}{2},\frac{1}{2}),(\frac{3}{2},\frac{3}{2})\in C_1\cup C_2$, por lo cual los elementos $(\frac{1}{2},\frac{3}{2}),(\frac{3}{2},\frac{1}{2})\in C_1\cup C_2$\contradiction, por la forma en que se tomaron $C_1$ y $C_2$. Por lo cual, $C_1\cup C_2$ no puede expresarse como el producto cartesiano de dos abiertos.
            \item Sean $C,D\in\tau$, es decir que $C=A_1\times B_1$ y $D=A_2\times B_2$, donde $A_i\in\tau_X$ y $B_i\in\tau_Y$ para $i\in\left\{1,2\right\}$. Entonces:
            \begin{equation*}
                \begin{split}
                    C\cap B&= (A_1\times B_1)\cap(A_2\times B_2)\\
                    &= (A_1\cap A_2)\times (B_1\cap B_2) \\
                \end{split}
            \end{equation*}
            donde $A_1\cap A_2\in\tau_X$ y $B_1\cap B_2\in\tau_Y$, por ende $C\cap B\in\tau$.
        \end{enumerate}
        Por el inciso (2), se tiene que $\tau$ (al menos en un caso particular) no es una topología sobre $X\times Y$.
    \end{sol}

    Recordemos la definición de un preorden y orden parcial:

    \begin{mydef}
        Una relación binaria $R$ en un conjunto $A$ es llamada un \textbf{preorden} si es reflexiva y transitiva, esto es:
        \begin{enumerate}
            \item $\forall a\in A, aRa$.
            \item $(aRb)\lor(bRc)\Rightarrow aRc$.
        \end{enumerate}
        denotamos (en general) al preorden por $\prec$.
    \end{mydef}

    \begin{mydef}
        Sea $(A,\prec)$ un conjunto preordenado.
        \begin{enumerate}
            \item $m\in A$ es llamado \textbf{elemento maximal} en $A$ si para todo $a\in A$ tal que $m\prec a\Rightarrow a\prec m$.
            \item Un elemento $a_0\in A$ es llamado \textbf{cota superior de un subconjunto $B\subseteq A$} si para todo $b\in B$, $b\prec a_0$.
            \item Un subconjunto $B\subseteq A$ es llamado una \textbf{cadena} si cualesquiera dos elementos de $B$ están relacionados, es decir que $a,b\in B$ implica que $a\prec b$ o $b\prec a$.
        \end{enumerate}
    \end{mydef}
    
    \begin{mydef}
        Sea $A$ un conjunto preordenado. Un \textbf{orden parcial} es un preorden en $A$ junto con la propiedad adicional:
        \begin{equation*}
            (a\prec b)\land (b\prec a)\Rightarrow (a=b)
        \end{equation*}
        esta propiedad es llamada antisimetría. Un conjunto $A$ adjutandole además un orden parcial es llamado un \textbf{conjunto parcialmente ordenado}. Un conjunto parcialente ordenado que es también una cadena es llamado un \textbf{conjunto totalmente ordenado}.
    \end{mydef}

    \begin{excer}
        Sea $X$ un conjunto parcialmente ordenado. Defina $U\subseteq X$ abierto si y sólo si satisface la condición: $(x\in U)\land (y\prec x)\Rightarrow y\in U$. Pruebe que la familia
        \begin{equation*}
            \mathcal{A}=\left\{U\subseteq X\big| U\textup{ es abierto} \right\}
        \end{equation*}
        es una topología sobre $X$.
    \end{excer}

    \begin{proof}
        Se deben verificar que se cumplen las tres condiciones.
        \begin{enumerate}
            \item $\emptyset\in\mathcal{A}$, pues por vacuidad se cumple que $\emptyset$ satisface la condición. Ahora, sea $x\in X$ y $y\prec x$, entonces $y\in X$ (pues es dónde se define el preorden). Por tanto, $X\in\mathcal{A}$.
            \item Sea $\mathcal{B}$ una familia no vacía de subconjuntos de $\mathcal{A}$. Si $x\in\bigcup\mathcal{B}$, entonces existe $B_0\in\mathcal{B}$ tal que $x\in B_0$.
            
            Ahora, si $y\in X$ es tal que $y\prec x$, como $x\in B_0$, por ser $B_0$ abierto se tiene que $y\in B_0\subseteq\bigcup\mathcal{B}$. Por lo cual $\bigcup\mathcal{B}$ es abierto.

            \item Sean $U,V\in\mathcal{A}$, si $U\cap V=\emptyset$ es claro que $U\cap V\in\mathcal{A}$. Suponga que la intersección es no vacía y sean $x\in U\cap V$ y $y\in X$ tal que $y\prec x$. En particular $(x\in U)\land(y\prec x)$ y $(x\in V)\land(y\prec x)$, por ende $y\in U\cap V$, es decir que $U\cap V\in\mathcal{A}$.
        \end{enumerate}
        Por los incisos anteriores, se tiene que $\mathcal{A}$ es una topología sobre $X$.
    \end{proof}

    \begin{excer}
        En $\mathbb{Z}^+$ defina $U\subseteq\mathbb{Z}^+$ que sea abierto si satisface la condición $n\in U\Rightarrow$ cada divisor de $n$ pertenece a $U$. Pruebe que esta es una topología en $\mathbb{Z}^+$ y que no es la topología discreta.
    \end{excer}

    \begin{proof}
        Llamemos $\tau$ a la familia de todos los conjuntos abiertos en $\mathbb{Z}^+$. Veamos que para $\tau$ se cumplen las tres condiciones:
        \begin{enumerate}
            \item $\emptyset\in\tau$, esto es cierto por vacuidad. Ahora si $n\in\mathbb{Z}^+$, entonces todos sus divisores están en $\mathbb{Z}^+$ (divisores positivos), por lo cual $\mathbb{Z}^+\in\tau$.
            \item Sea $\mathcal{A}$ una familia no vacía de elementos de $\tau$, y sea $n\in\bigcap\mathcal{A}$, entonces existe $A_0$ tal que $n\in A_0$, pero $A_0$ es abierto, por lo cual contiene a todos los divisores de $n$. Como $A\subseteq\bigcup\mathcal{A}$ entonces $\bigcup\mathcal{A}$ contiene a todos los divisores de $n$, luego $\bigcup\mathcal{A}\in\tau$.
            \item Sean $A,B\in\tau$ tales que $A\cap B\neq\emptyset$. Si $n\in A\cap B$ entonces $n\in A$ y $n\in B$, como $A$ y $B$ son abiertos, entonces estos dos conjuntos cumplen que cada divisor de $n$ pertenece a $A$ y $B$, en particular cada divisor de $n$ pertenece a $A\cap B$. Por tanto, $A\cap B\in\tau$. 
        \end{enumerate}
        Por los tres incisos anteriores, se sigue que $\tau$ es una topología sobre $\mathbb{Z}^+$.
    \end{proof}

    \begin{excer}
        Pruebe lo siguiente: $\tau$ es la topología discreta en $X$ si y sólo si todo punto de $X$ es un conjunto abierto (hablando de los conjuntos unipuntuales). 
    \end{excer}

    \begin{proof}
        Se probará la doble implicación:
        $\Rightarrow)$: Suponga que $\tau$ es la topología discreta, entonces $\tau=\mathcal{P}(X)$, en particular $\left\{x\right\}\in \mathcal{P}(X)$, para cada $x\in X$, esto es $\left\{x\right\}\in\tau$.

        $\Leftarrow)$: Suponga que todo conjunto unipuntual de $X$ está en $\tau$, y sea $A\in\mathcal{P}(X)$, entonces:
        \begin{equation*}
            A=\bigcup_{a\in A}\left\{a\right\}
        \end{equation*}
        donde $\left\{a\right\}$ es abierto y, por ende $A$ es abierto al ser una unión arbitraria de abiertos. Por tanto, $A\in\tau$, Por ende $\mathcal{P}(X)\subseteq\tau$, pero siempre se tiene que $\tau\subseteq\mathcal{P}(X)$, luego $\tau=\mathcal{P}(X)=\tau_D$.
    \end{proof}

    \setcounter{section}{2}

    \newpage

    \section{Creación de topologías dados conjuntos}

    Mis ejercicios de la sección: 5 y 9.
    
    \begin{excer}
        
    \end{excer}

    \newpage

     \section{Conceptos Elementales}

     Mis ejercicios de la sección: 8, 10, 14, 18, 22.

    \setcounter{excer}{3}

     \begin{excer}
        Sea $(X,\tau)$ un espacio topológico. Pruebe que $G$ es abierto en $X$, si y sólo si $\Cls{G\cap\Cls{A}}=\Cls{G\cap A}$ para todo $A\subseteq X$.
     \end{excer}

    \begin{proof}
        Se probará la doble implicación.

        $\Rightarrow)$: Suponga que $G$ es abierto, Como $A\subseteq \Cls{A}$ para todo $A\in X$, se tiene entonces que:
        \begin{equation*}
            \begin{split}
                G\cap A&\subseteq G\cap \Cls{A}\\
                \Rightarrow \Cls{G\cap A}&\subseteq \Cls{G\cap \Cls{A}} \\
            \end{split}
        \end{equation*}
        por lo cual basta probar la otra contención. Si $x\in\Cls{G\cap \Cls{A}}$, entonces para toda vecindad $U$ de $x$ se cumple que $U\cap (G\cap \Cls{A})\neq\emptyset$, sea $y\in U\cap (G\cap \Cls{A})$ entonces, como el conjunto $U\cap G$ es una vecindad de $y$, se tiene que $U\cap (G\cap A)\neq\emptyset$, es decir que existe un elemento $z\in U$ tal que $z\in G\cap A$, pero $U$ originalmente era una vecindad de $x$, luego $x\in\Cls{G\cap A }$, lo cual prueba la otra contención.
    \end{proof}

    \begin{excer}
        Pruebe que si $(X,\tau)$ es un espacio topológico, entonces si $A\subseteq X$ tal que $A'=\emptyset$ implica que $A$ es cerrado.
    \end{excer}

    \begin{proof}
        Sea $A\subseteq X$ tal que $A'=0$. Como
        \begin{equation*}
            \Cls{A}=A\cup A'=A
        \end{equation*}
        se tiene entonces que $A$ coincide con su cerradura, la cual es cerrada. Por tanto, $A$ es cerrado.
    \end{proof}

    \begin{excer}
        Sea $A=\left\{\frac{1}{n}+\frac{1}{m}\Big|m,n\in\mathbb{N} \right\}\subseteq\mathbb{E}^1$. Pruebe que $A'=\left\{\frac{1}{n}\Big|n\in\mathbb{N} \right\}\cup\left\{0\right\}$ y que $A''=\left\{0\right\}$.
    \end{excer}

    \begin{proof}
        
    \end{proof}

    \begin{excer}
        Sea $\left\{A_\alpha \right\}_{\alpha\in I}$ una familia de subconjuntos de $X$. Suponga que $\bigcup_{\alpha\in I}\Cls{A_\alpha}$ es cerrado. Pruebe que $\bigcup_{\alpha\in I}\Cls{A_\alpha}=\Cls{\bigcup_{\alpha\in I}A_\alpha}$.
    \end{excer}

    \begin{proof}
        Ya se sabe que
        \begin{equation*}
            \bigcup_{\alpha\in I}\Cls{A_\alpha}\subseteq\Cls{\bigcup_{\alpha\in I}A_\alpha}
        \end{equation*}
        Hay que ver la otra contención. Observemos que $\bigcup_{\alpha\in I}\Cls{A_\alpha}$ es un cerrado que contiene a $\bigcup_{\alpha\in I}A_\alpha$, luego, por minimalidad de la cerradura, debe suceder que:
        \begin{equation*}
            \Cls{\bigcup_{\alpha\in I}A_\alpha}\subseteq\bigcup_{\alpha\in I}\Cls{A_\alpha}
        \end{equation*}
        pues, la cerradura de un conjunto debe estar contenida en cualquier cerrado que contenga al conjnunto. Luego, por las dos contenciones, se sigue que:
        \begin{equation*}
            \bigcup_{\alpha\in I}\Cls{A_\alpha}=\Cls{\bigcup_{\alpha\in I}A_\alpha}
        \end{equation*}
    \end{proof}

    \begin{excer}
        Pruebe que $\Fr{A}=\emptyset$ si y sólo si $A$ es abierto y cerrado.
    \end{excer}

    \begin{proof}
        $\Rightarrow):$ Suponga que $\Fr{A}=\emptyset$. Se tiene entonces que:
        \begin{equation*}
            \emptyset = \Fr{A}=\Cls{A}\cap\Cls{X-A}
        \end{equation*}
        Afirmamos que $A=\Cls{A}$ y que $X-A=\Cls{X-A}$. En efecto, ya se sabe que $A\subseteq \Cls{A}$.

        Suponga que existe $x\in\Cls{A}$ tal que $x\notin A$, como $X=A\cup (X-A)$, se sigue que $x\in X-A\subseteq \Cls{X-A}$, luego $x\in \Cls{A}\cap\Cls{X-A}$\contradiction lo cual contradice la igualdad anterior. Por tanto, $\Cls{A}\subseteq A$, se decir que $A=\Cls{A}$.

        De forma análoga se prueba que $X-A=\Cls{X-A}$. Entonces, $A$ es un conjunto cerrado, ya que coincide con su cerradura, y abierto ya que su complemento es cerrado. Por ende, $A$ es abierto y cerrado.

        $\Leftarrow):$ Suponga que $A$ es abierto y cerrado, entonces se tiene que $A=\Cls{A}$ y $X-A\Cls{X-A}$. Por ende:
        \begin{equation*}
            \Fr{A}=\Cls{A}\cap\Cls{X-A}=A\cap(X-A)=\emptyset
        \end{equation*}
        como se quería demostrar.

    \end{proof}

    \begin{excer}
        Pruebe las siguientes fórmulas:
        \begin{enumerate}
            \item $\Fr{\Fr{\Fr{A}}}=\Fr{\Fr{A}}$.
            \item $\Fr{\Int{A}}\subseteq\Fr{A}$.
            \item $\Int{\overbrace{A-B}}\subseteq\Int{A}-\Int{B}$.
        \end{enumerate}
    \end{excer}

    \begin{proof}
        De (1): Sea $B\subseteq X$. Entonces,
        \begin{equation*}
            \Fr{B}=\Cls{B}\cap\Cls{X-B}
        \end{equation*}
        por tanto,
        \begin{equation*}
            \begin{split}
                \Fr{\Fr{B}}&=\Cls{\Cls{B}\cap\Cls{X-B}}\cap\Cls{X-\Cls{B}\cap\Cls{X-B}}\\
                &=(\Cls{B}\cap\Cls{X-B})\cap\Cls{X-\Cls{B}\cap\Cls{X-B}}\\
                &=\Fr{B}\cap\Cls{X-\Fr{B}}\\
            \end{split}
        \end{equation*}
        por tanto, $\Fr{\Fr{B}}\subseteq \Fr{B}$. En particular, para $A\subseteq X$ se sigue que $\Fr{\Fr{\Fr{A}}}\subseteq\Fr{\Fr{A}}$ (tomando $B=\Fr{A}$).

        Ahora, tenemos que:
        \begin{equation*}
            \Fr{\Fr{\Fr{A}}}=\Fr{\Fr{A}}\cap\Cls{X-\Fr{\Fr{A}}}
        \end{equation*}

        Probaremos la otra contención. Afirmamos que $\Fr{\Fr{A}}\subseteq\Cls{X-\Fr{\Fr{A}}}$. En efecto, si $x\in $
        
        Suponga que $x\in\Fr{\Fr{A}}$ es tal que $x\notin\Cls{X-\Fr{\Fr{A}}}$. Entonces, existe $U\subseteq X$ abierto que contiene a $x$ tal que:
        \begin{equation*}
            U\cap(X-\Fr{\Fr{A}})=\emptyset
        \end{equation*}
        por tanto, $U\subseteq\Fr{\Fr{A}}$. De esta forma, al ser $x$ arbitrario, se sigue que el conjunto $\Fr{\Fr{A}}-\Cls{X-\Fr{\Fr{A}}}$ es abierto, en particular, $\Fr{\Fr{A}}-\Cls{X-\Fr{\Fr{A}}}\subseteq\Int{\overbrace{\Fr{\Fr{A}}}}$.

        De (2): Se tiene que
        \begin{equation*}
            \Int{A}\subseteq A\subseteq \Cls{A}
        \end{equation*}
        Por tanto,
        \begin{equation*}
            \Cls{\Int{A}}\subseteq\Cls{A}
        \end{equation*}
        Además,
        \begin{equation*}
            \begin{split}
                X-A\subseteq& X-\Int{A}\\
                \Rightarrow\Cls{X-A}\subseteq\Cls{X-\Int{A}}\\
            \end{split}
        \end{equation*}
        pero, $X-\Int{A}$ es cerrado, luego $X-\Int{A}=\Cls{X-\Int{A}}$. Por ende:
        \begin{equation*}
            \Cls{X-A}\subseteq X-\Int{A}
        \end{equation*}
        Para probar el resultado, basta con probar que $\Cls{X-A}=X-\Int{A}$. Si $x\in X-\Int{A}$ entonces, $x\notin\Int{A}$ por tanto, para todo abierto $U\subseteq X$ tal que $x\in U$, se tiene que:
        \begin{equation*}
            U\nsubseteq A
        \end{equation*}
        por tanto, $U\cap (X-A)\neq\emptyset$. Se sigue entonces que $x\in \Cls{X-A}$, de donde se sigue que $X-\Int{A}\subseteq\Cls{X-A}$. Por la contención anterior, se tiene que $\Cls{X-A}=X-\Int{A}$. Así:
        \begin{equation*}
            \begin{split}
                \Fr{\Int{A}}&=\Cls{\Int{A}}\cap\Cls{X-\Int{A}}\\
                &=\Cls{\Int{A}}\cap(X-\Int{A})\\
                &=\Cls{\Int{A}}\cap\Cls{X-A} \\
                &\subseteq\Cls{A}\cap\Cls{X-A} \\
                &=\Fr{A}\\
                \Rightarrow \Fr{\Int{A}}&\subseteq\Fr{A}\\
            \end{split}
        \end{equation*}

        De (3): 

    \end{proof}

    \begin{excer}
        Suponga que $\Fr{A}\cap\Fr{B}=\emptyset$. Pruebe que $\Int{\overbrace{A\cup B}}=\Int{A}\cup\Int{B}$ y que $\Fr{A\cap B}=\left[\Cls{A}\cap\Fr{B} \right]\cup\left[\Fr{A}\cap\Cls{B} \right]$.
    \end{excer}

    \begin{proof}
        Ya se sabe que $\Int{A}\cup\Int{B}\subseteq\Int{\overbrace{A\cup B}}$. Probemos la otra contención.

        Suponga que existe $x\in\Int{\overbrace{A\cup B}}$ tal que $x\notin\Int{A}\cup\Int{B}$, es decir que $x\in X-(\Int{A}\cup\Int{B})=(X-\Int{A})\cap(X-\Int{B})$.

        Por tanto, para todo abierto $U\subseteq X$ tal que $x\in U$ se tiene que $U\nsubseteq A$ y $U\nsubseteq B$. Se tienen tres casos:
        \begin{enumerate}
            \item $x\in A\cap B$: En tal caso, se sigue que $x\in\Fr{A}\cap\Fr{B}$, ya que los conjuntos:
            \begin{equation*}
                U\cap A,U\cap (X-A),U\cap B,U\cap(X-B)\neq\emptyset
            \end{equation*}
            son no vacíos, para todo $U$ abierto que contiene a $x$, pero esto es una contradicción, ya que $\Fr{A}\cap\Fr{B}=\emptyset$\contradiction.
            \item $x\in A-B$. Como $x\in\Int{\overbrace{A\cup B}}$, existe un abierto $V\subseteq X$ que contiene a $x$ tal que $V\subseteq A\cup B$. Sea $U\subseteq X$ abierto que contiene a $x$. Se tiene que:
            \begin{equation*}
                U\cap A, U\cap (X-B)\neq\emptyset
            \end{equation*}
            pues, $x$ está en ambos conjuntos. Ahora, como $U\nsubseteq A$, entonces la intersección $U\cap (X-A)\neq\emptyset$, luego $x\in \Fr{A}$.

            Considere al abierto $U_0=U\cap V$. Este es un abierto que contiene a $x$ tal que $U_0\subseteq A\cup B$ (pues, $V\subseteq A\cup B$). Pero, como $U_0\nsubseteq A$, debe tenerse que existe $y\in U_0$ tal que $y\in B$. Luego, la intersección:
            \begin{equation*}
                U_0\cap B\neq\emptyset\Rightarrow U\cap B\neq\emptyset
            \end{equation*}
            por ende, $x\in\Fr{B}$, de donde se sigue que $x\in\Fr{A}\cap\Fr{B}$, pero esto es una contradicción, ya que $\Fr{A}\cap\Fr{B}=\emptyset$\contradiction.
            \item $x\in B-A$. De forma similar al caso anterior, se llega a que $x\in \Fr{A}\cap\Fr{B}$\contradiction.
        \end{enumerate}
        los tres incisos llevan a que $x\in\Fr{A}\cap\Fr{B}$\contradiction. Por tanto, $x\in\Int{A}\cup\Int{B}$.

        Para la segunda parte, observemos que:
        \begin{equation*}
            \begin{split}
                \Fr{A\cap B}&=\Cls{A\cap B}\cap\Cls{X-A\cap B}\\
                &=\Cls{A\cap B}\cap\Cls{(X-A)\cup(X-B)}\\
                &=\Cls{A\cap B}\cap(\Cls{(X-A)}\cup\Cls{X-B})\\
                &=(\Cls{A\cap B}\cap\Cls{(X-A)})\cup(\Cls{A\cap B}\cap\Cls{X-B})\\
            \end{split}
        \end{equation*}
        para probar el resultado, basta con probar que $\Cls{A\cap B}=\Cls{A}\cap\Cls{B}$. Para ello, probaremos que si $C\subseteq X$, entonces:
        \begin{equation*}
            X-\Cls{C}=\Int{\overbrace{X-C}}
        \end{equation*}
        En efecto, como $X-\Cls{C}\subseteq X-C$ siendo el primer conjunto abierto, se sigue que $X-\Cls{C}\subseteq\Int{\overbrace{X-C}}$. Ahora, el conjunto $X-\Int{\overbrace{X-C}}$ es un cerrado que contiene a $C$. En efecto, es cerrado por ser el complemento de un abierto.

        Ahora, si $x\in C$, entonces $x\in X-(X-C)$. Como $\Int{\overbrace{X-C}}\subseteq X-C$, se sigue que $x\in X-\Int{\overbrace{X-C}}$. Por tanto, $C\subseteq X-\Int{\overbrace{X-C}}$. Luego, por minimalidad de la cerradura, se sigue que $\Cls{C}\subseteq X-\Int{\overbrace{X-C}}$, es decir que $\Int{\overbrace{X-C}}=X-(X-\Int{\overbrace{X-C}})\subseteq X-\Cls{C}$.

        Se tienen las contenciones $X-\Cls{C}\subseteq\Int{\overbrace{X-C}}$ y $\Int{\overbrace{X-C}}\subseteq X-\Cls{C}$, por tanto, se sigue que $X-\Cls{C}=\Int{\overbrace{X-C}}$.

        Con esto probado, tomemos $C=A\cap B$, entonces:
        \begin{equation*}
            \begin{split}
                X-\Cls{A\cap B}&=\Int{\overbrace{X-A\cap B}}\\
                &=\Int{\overbrace{(X-A)\cup (X-B)}}\\
                &=\Int{\overbrace{X-A}}\cup\Int{\overbrace{X-B}}\\
                &=(X-\Cls{A})\cup(X-\Cls{B})\\
                &=X-\Cls{A}\cap\Cls{B}\\
                \Rightarrow \Cls{A\cap B}&=\Cls{A}\cap\Cls{B}\\
            \end{split}
        \end{equation*}
        donde, el paso de la segunda a la tercera igualdad se da ya que $\Fr{X-A}\cap\Fr{X-B}=\Fr{A}\cap\Fr{B}=\emptyset$. Luego, se tiene que:
        \begin{equation*}
            \begin{split}
                \Fr{A\cap B}&=(\Cls{A\cap B}\cap\Cls{(X-A)})\cup(\Cls{A\cap B}\cap\Cls{X-B})\\
                &=(\Cls{A}\cap\Cls{B} \cap\Cls{(X-A)})\cup(\Cls{A}\cap\Cls{B}\cap\Cls{X-B})\\
                &=([\Cls{A}\cap\Cls{X-A}]\cap\Cls{B})\cup(\Cls{A}\cap[\Cls{B}\cap\Cls{X-B}])\\
                &=(\Fr{A}\cap\Cls{B})\cup(\Cls{A}\cap\Fr{B})\\
                &=\left[\Cls{A}\cap\Fr{B}\right]\cup\left[\Fr{A}\cap\Cls{B}\right]\\
            \end{split}
        \end{equation*}
        lo cual prueba el resultado.
    \end{proof}

    \begin{excer}
        ¿Para qué espacios topológicos $(X,\tau)$ el único conjunto denso es $X$?
    \end{excer}

    \begin{proof}
        Sea $(X,\tau)$ un espacio topológico tal que $X$ es es el único conjunto denso en sí mismo. Si $x\in X$, entonces el conjunto $X-\left\{x \right\}$ no es denso en $X$, por lo cual:
        \begin{equation*}
            \Cls{X-\left\{x\right\}}=X-\left\{x\right\}
        \end{equation*}
        luego, $X-\left\{x \right\}$ es cerrado en $X$, es decir que $\left\{x\right\}$ es abierto. Como $x\in X$ fue arbitrario, se sigue que $\left\{ x\right\}$ es abierto, para todo $x\in X$. Por ende, $\tau=\tau_D$ (en caso que de $X$ no sea vacío).

        Por tanto, los únicos espacios en los que ocurre esto, son aquellos en los que la topología es la discreta.
    \end{proof}

    \begin{excer}
        Sea $(X,\tau)$ espacio topológico y $E,G\subseteq X$ abiertos densos en $X$. Pruebe que $E\cap G$ es denso en $X$.
    \end{excer}

    \begin{proof}
        Sea $U\subseteq X$ abierto. Para probar el resultado, debemos probar que $U\cap (E\cap G)\neq\emptyset$. Como $U\cap E$ es abierto, entonces $(U\cap E)\cap G=U\cap (E\cap G)\neq\emptyset$.

        En este caso, no es necesario que los dos sean abiertos a la vez, basta con que uno de ellos lo sea.
    \end{proof}

    \begin{excer}
        Sean $(X,\tau)$ un espacio topológico y $D\subseteq X$ un conjunto denso en $X$. Pruebe que $\Cls{D\cap G}=\Cls{G}$, para todo $G\subseteq X$ abierto.
    \end{excer}

    \begin{proof}
        Sea $G\subseteq X$ abierto. Ya se tiene que:
        \begin{equation*}
            \Cls{D\cap G}\subseteq\Cls{G}
        \end{equation*}
        pues, $D\cap G\subseteq G$. Se ahora $x\in\Cls{G}$, entonces si $U\subseteq X$ es abierto, se tiene que $U\cap G\neq\emptyset$. Como $D$ es denso en $X$, entonces $U\cap(D\cap G)=U\cap(G\cap D)=(U\cap G)\cap D\neq\emptyset$, es decir que $x\in\Cls{G\cap D}$.

        De aquí se sigue la otra contención. Por las dos, se tiene que $\Cls{D\cap G}=\Cls{G}$.
    \end{proof}

    \begin{excer}
        Sean $(X,\tau)$ un espacio topológico y $\mathcal{S}$ una sub-base para $\tau$, y $D\subseteq X$ tal que $D\cap S\neq\emptyset$ para todo $S\in\mathcal{S}$ ¿Esto implica que $D$ es denso en $X$?
    \end{excer}

    \begin{proof}
        Como $\mathcal{S}$ es una sub-base de $\tau$, entonces la colección formada por todas las intersecciones finitas de elementos de $\mathbb{S}$, forman una base de la topología $\tau$.
        No necesariamente se tiene que $D$ es denso en $X$, pues si $B\in\mathcal{B}$ es un básico, entonces existen $S_1,...,S_n\in\mathcal{S}$ tales que:
        \begin{equation*}
            B=\bigcap_{i=1}^nS_i
        \end{equation*}
        Luego, $D\cap S_i$ es no vacío para todo $i\in\natint{1,n}$, pero no necesariamente $D\cap\bigcap_{ i=1}^nS_i\neq\emptyset$.

        En efecto, considere el espacio $X=\left\{a,e,i,o,u \right\}$ y $\mathcal{S}=\left\{\left\{ a,e\right\},\left\{ e,i\right\} \right\}$. Se tiene que $\mathcal{S}$ es subbase de de la topología $\tau=\tau(\mathcal{S})=\left\{X,\left\{ a,e,i\right\},\left\{ a,e\right\},\left\{ e,i\right\},\left\{e\right\},\emptyset \right\}$. En el espacio topológico $(X,\tau)$, el conjunto $D=\left\{a,i\right\}$ cumple que $D\cap S\neq\emptyset$ para todo $S\in\mathcal{S}$, pero $D$ no es denso en $X$ ya que el abierto $\left\{e\right\}$ no contiene puntos de $D$.
    \end{proof}

    \begin{excer}
        
    \end{excer}

    \begin{excer}
        
    \end{excer}
    
    \begin{excer}
        
    \end{excer}

    \begin{excer}
        Se define el \textbf{exterior de un conjunto $A\subseteq X$},denotado por $\Ext{A}$, como el conjunto $\Ext{A}=\Int{\overbrace{X-A}}$. Pruebe lo siguiente:
        \begin{enumerate}
            \item $\Ext{A\cup B}=\Ext{A}\cap\Ext{B}$.
            \item $A\cap\Ext{A}=\emptyset$.
            \item $X=\Ext{\emptyset}$.
            \item $\Ext{X-\Ext{A}}=\Ext{A}$.
        \end{enumerate}
    \end{excer}

    \begin{proof}
        De (1): Notemos que:
        \begin{equation*}
            \begin{split}
                \Ext{A\cup B}&=\Int{\overbrace{X-A\cup B}}\\
                &=\Int{\overbrace{(X-A)\cap(X-B)}}\\
                &=\Int{\overbrace{X-A}}\cap\Int{\overbrace{X-B}}\\
                &=\Ext{A}\cap\Ext{B}\\
            \end{split}
        \end{equation*}
        
        De (2): Sea $A\subseteq X$, se tiene que $\Int{\overbrace{X-A}}\subseteq X-A$, por tanto, $A\cap\Ext{A}\subseteq A\cap (X-A)=\emptyset$. Luego, $A\cap\Ext{A}=\emptyset$.

        De (3): Notemos que:
        \begin{equation*}
            \begin{split}
                \Ext{\emptyset}&=\Int{\overbrace{X-\emptyset}}\\
                &=\Int{X}\\
                &=X\\
            \end{split}
        \end{equation*}
        pues, el conjunto $X$ es abierto.

        De (4): Sea $A\subseteq X$. Entonces:
        \begin{equation*}
            \begin{split}
                \Ext{X-\Ext{A}}&=\Int{\overbrace{X-\Ext{A}}}\\
                &=\Int{\overbrace{X-(X-\Ext{A})}}\\
                &=\Int{\Ext{A}}\\
                &=\Ext{A}\\
            \end{split}
        \end{equation*}
        pues, $\Ext{A}$ es un conjunto abierto.
    \end{proof}

    \begin{excer}
        
    \end{excer}

    \begin{excer}
        
    \end{excer}

    \begin{excer}
        
    \end{excer}

    \begin{excer}
        Un conjunto abierto $U\subseteq X$ de un espacio topológico $(X,\tau)$ es llamado \textbf{abierto regular} si $U=\Int{\Cls{U}}$; un conjunto cerrado $C\subseteq X$ es llamado \textbf{cerrado regular}, si $C=\Cls{\Int{C}}$. Pruebe lo siguiente:
        \begin{enumerate}
            \item Si $A$ es cerrado, entonces $\Int{A}$ es un conjunto abierto regular.
            \item Si $U$ es abierto, entonces $\Cls{U}$ es un conjunto cerrado regular.
            \item El complemento de un conjunto abierto regular (resp. cerrado) es un conjunto cerrado regular (resp. abierto).
            \item Si $U,V\subseteq X$ son conjuntos abiertos regulares, entonces $U\subseteq V$ si y sólo si $\Cls{U}\subseteq \Cls{V}$.
            \item Si $A,B\subseteq X$ son conjuntos cerrados regulares, entonces $A\subseteq B$ si y sólo si $\Int{A}\subseteq \Int{B}$.
            \item Si $U,V\subseteq X$ son abiertos regulares, entonces $U\cap V$ también es abierto regular.
            \item Si $A,B\subseteq X$ son cerrados regulares, entonces $A\cup B$ también es cerrado regular.
        \end{enumerate}
    \end{excer}

    \begin{proof}
        De (1): Sea $A\subseteq X$ un conjunto cerrado. Hay que probar que $\Int{A}$ es abierto regular, es decir, que:
        \begin{equation*}
            \Int{A}=\Int{\Cls{\Int{A}}}
        \end{equation*}
        Notemos que
        \begin{equation*}
            \begin{split}
                \Int{A}&\subseteq A\\
                \Rightarrow \Cls{\Int{A}}&\subseteq \Cls{A}=A\\
                \Rightarrow \Int{\Cls{\Int{A}}}&\subseteq \Int{A}\\
            \end{split}
        \end{equation*}
        para la otra contención analicemos. $\Cls{\Int{A}}$ es un cerrado para el que se cumple que $\Int{A}\subseteq\Cls{\Int{A}}$, luego sacando interior de ambos lados, se sigue que:
        \begin{equation*}
            \Int{A}=\Int{\Int{A}}\subseteq\Int{\Cls{\Int{A}}}
        \end{equation*}
        por tanto, de las dos contenciones se sigue que:
        \begin{equation*}
            \Int{A}=\Int{\Cls{\Int{A}}}
        \end{equation*}
        luego, $\Int{A}$ es un abierto regular.

        De (2): Sea $U\subseteq X$ abierto. Hay que probar que:
        \begin{equation*}
            \Cls{U}=\Cls{\Int{\Cls{U}}}
        \end{equation*}
        En efecto, veamos que:
        \begin{equation*}
            \begin{split}
                \Int{\Cls{U}}&\subseteq\Cls{U}\\
                \Rightarrow \Cls{\Int{\Cls{U}}}&\subseteq\Cls{\Cls{U}}=\Cls{U}\\
            \end{split}
        \end{equation*}
        Ahora, 
        \begin{equation*}
            \begin{split}
                \begin{split}
                    U&\subseteq\Cls{U}\\\
                    \Rightarrow U=\Int{U}&\subseteq\Int{\Cls{U}}\\
                    \Rightarrow \Cls{U}&\subseteq \Cls{\Int{\Cls{U}}}\\
                \end{split}
            \end{split}
        \end{equation*}
        lo cual prueba la otra contención, así $\Cls{U}=\Cls{\Int{\Cls{U}}}$. Luego, $\Cls{U}=\Cls{\Int{\Cls{U}}}$ por lo cual, $\Cls{U}$ es cerrado regular.

        De (3): Basta con probar que el complemento de un conjunto abierto regular es un conjunto cerrado regular. Sea $U\subseteq X$ abierto regular, es decir que:
        \begin{equation*}
            U=\Int{\Cls{U}}
        \end{equation*}
        Entonces, su complemento $C=X-U$ cumple que:
        \begin{equation*}
            \Int{C}\subseteq C
            \Rightarrow \Cls{\Int{C}}\subseteq C
        \end{equation*}
        Si $x\in C$, entonces $x\in X-U=X-\Int{\Cls{U}}$, luego $x\notin\Int{\Cls{U}}$, por tanto, para todo abierto $V\subseteq X$ que contiene a $x$ se tiene que $V\nsubseteq\Cls{U}$, es decir, que existe un $y\in V$ tal que $y\notin \Cls{U}$ esto es $y\in X-\Cls{U}$.

        Pero, $X-\Cls{U}=\Int{\overbrace{X-U}}$ (esto se probó en un ejercicio anterior), es decir que $y\in\Int{C}$. Por tanto, $V\cap\Int{C}\neq\emptyset$. Luego, $x\in \Cls{\Int{C}}$.

        Así, se tiene la contención $C\subseteq\Cls{\Int{C}}$. Por esta y otra contención, se sigue que $C=\Cls{\Int{C}}$, es decir que $X-U$ es cerrado regular.

        De (4): La ida es inmediata. Suponga que $\Cls{U}\subseteq\Cls{V}$, tomando interiores se sigue que $U=\Int{\Cls{U}}\subseteq\Int{\Cls{V}}=V$, lo cual pruebra el resultado.

        De (5): Es análogo a (4).

        De (6): Sean $U,V\subseteq X$ abiertos regulares, es decir que: $U=\Int{\Cls{U}}$ y $V=\Int{\Cls{V}}$. Se tiene que:
        \begin{equation*}
            \begin{split}
                \Int{\overbrace{\Cls{U\cap V}}}&\subseteq\Int{\overbrace{\Cls{U}\cap\Cls{V}}}\\
                &=\Int{\Cls{U}}\cap\Int{\Cls{V}}\\
                &=U\cap V
            \end{split}
        \end{equation*}
        para ver la otra contención, notemos que
        \begin{equation*}
            \begin{split}
                U\cap V&\subseteq U\cap V\\
                \Rightarrow U\cap V&\subseteq \Cls{U\cap V}\\
                \Rightarrow U\cap V=\Int{\overbrace{U\cap V}} &\subseteq \Int{\overbrace{\Cls{U\cap V}}}\\
            \end{split}
        \end{equation*}
        de las dos contenciones se sigue que $U\cap V = \Int{\overbrace{\Cls{U\cap V}}}$.

        De (7): Es análogo a (6).
    \end{proof}

    \newpage

    \section{Creando topologías a partir de operaciones elementales}

    \begin{excer}
        Sean $X$ un conjunto, y $A\mapsto u(A)$, $A\mapsto v(A)$ dos operaciones de cerradura, es decir que cumplen que:
        \begin{enumerate}
            \item $u(\emptyset)=\emptyset$.
            \item $A\subseteq u(A)$, para todo $A\subseteq X$.
            \item $u\circ u(A)=u(A)$, para todo $A\subseteq X$.
            \item $u(A\cup B)=u(A)\cup u(B)$, para todos $A,B\subseteq X$.
        \end{enumerate}
        (por un resultado anterior, la familia $\tau_u=\left\{X-u(A)\Big|A\subseteq X \right\}$ es una topología sobre $X$. Lo análogo se cumple para $v$).

        Suponga que se cumple que $v\circ u(A)$ es $u$-cerrado para todo $A\subseteq X$. Pruebe que $A\mapsto v\circ u(A)$ es una operación de cerradura y que $v\circ u(A)$ es de hecho la intersección de todos los conjuntos que contienen a $A$ que son cerrados tanto en $v$ como en $u$.

        Finalmente, muestre que $u\circ v(A)\subseteq v\circ u(A)$.
    \end{excer}

    \begin{proof}
        Probaremos varias cosas:
        \begin{enumerate}
            \item $A\mapsto v\circ u(A)$ es una operación de cerradura. En efecto, hay que verificar que se cumplen varias condiciones:
            \begin{enumerate}
                \item Se tiene que:
                \begin{equation*}
                    \begin{split}
                        v\circ u(\emptyset)&=v(u(\emptyset))\\
                        &=v(\emptyset)\\
                        &=\emptyset\\
                    \end{split}
                \end{equation*}
                \item Sea $A\subseteq X$. Se tiene que $A\subseteq u(A)$, luego $v(A)\subseteq v\circ u(A)$, como $A\subseteq v(A)$, entonces se sigue que $A\subseteq v\circ u(A)$.
                \item Sea $A\subseteq X$. Como $v\circ u(A)$ es $u$-cerrado, entonces $u((v\circ u)(A))=v\circ u(A)$, aplicando $v$ se sigue que $(v\circ u)\circ(v\circ u)(A)=v\circ (v\circ u)(A)=(v\circ v)\circ u(A)=v\circ u(A)$.
                \item Sean $A,B\subseteq X$, entonce:
                \begin{equation*}
                    \begin{split}
                        v\circ u(A\cup B)&=v(u(A\cup B))\\
                        &=v(u(A)\cup u(B))\\
                        &=v(u(A))\cup v(u(B))\\
                        &=v\circ u(A)\cup v\circ u(B)\\
                    \end{split}
                \end{equation*}
            \end{enumerate}
            por los incisos i)-iv) se sigue que $A\mapsto v\circ u(A)$ es una operación de cerradura.

            \item Sea $A\subseteq X$. El conjunto $v\circ u(A)$ es $v$-cerrado y, por hipótesis es $u$-cerrado.
        
            Sea
            \begin{equation*}
                \mathcal{C}=\left\{C\subseteq X\Big|C\textup{ es $u$-cerrado y $v$-cerrado y }A\subseteq C \right\}
            \end{equation*}
            Tomemos $\widehat{C}=\bigcap\mathcal{C}$. Por la observación anterior, como $A\subseteq v\circ u(A)\in\mathcal{C}$ (por ser operación de cerradura), se tiene que $\widehat{C}\subseteq v\circ u(A)$ ya que $v\circ u(A)\in\mathcal{C}$.

            Sea ahora $C\in\mathcal{C}$. Para probar el resultado, hay que ver que $v\circ u(A)\subseteq C$. Como $C$ es $u$-cerrado, y $A\subseteq C$, entocnes $u(A)\subseteq u(C)=C$. Pero, además $C$ es $v$-cerrado, por lo cual $v\circ u(A)\subseteq v(C)=C$.

            Por tanto, $v\circ u(A)=\widehat{C}$.

            \item Sea $A\subseteq X$, entonces $A\subseteq u(A)$ y, por ende $v(A)\subseteq v\circ u(A)$. Como $v\circ u(A)$ es $u$-cerrado, entonces:
            \begin{equation*}
                u\circ v(A)\subseteq u(v\circ u(A))=v\circ u(A)
            \end{equation*}
            como se quería demostrar.
        \end{enumerate}
    \end{proof}

    \begin{excer}
        Sean $X,Y$ conjuntos y $\cf{\varphi}{X}{\mathcal{P}(Y)}$ una función. Para $A\subseteq X$, defina:
        \begin{equation*}
            \varphi(A)=\bigcup\left\{\varphi(x)\Big|x\in A \right\}
        \end{equation*}
        y, para $B\subseteq Y$, sea $\varphi^{-1}(B)=\left\{\varphi(x)\subseteq B\textup{ y }\varphi(x)\neq\emptyset \right\}$. Pruebe que $u(A)=\varphi\circ\varphi^{-1}(A)$ satisface lo siguiente:
        \begin{enumerate}
            \item $u(\emptyset)=\emptyset$.
            \item $A\subseteq u(A)$, para todo $A\subseteq X$.
            \item $u\circ u(A)=u(A)$, para todo $A\subseteq X$.
            \item $(A\subseteq B)\Rightarrow(u(A)\subseteq u(B))$, para todo $A,B\subseteq X$.
        \end{enumerate}
    \end{excer}

    \begin{proof}
        
    \end{proof}

    \begin{excer}
        Sea $(X,\tau)$ un espacio topológico, y sea $\cf{\tau}{\mathcal{P}(X)\times\mathcal{P}(X)}{\mathcal{P}(X)}$ una función que cumple lo siguiente:
        \begin{enumerate}
            \item $\tau(A,B\cup C)\cup\tau(B,C\cup A)=\tau(A\cup B, C)\cup\tau(A,B)$
            \item $\tau(\emptyset, X)=\emptyset$.
            \item $\tau(\Cls{A},\Cls{X-A})\subseteq\Cls{A}$.
            \item $\tau(A,B)\subseteq A\cup B$.
        \end{enumerate}
        Pruebe que $\tau(A,B)=(A\cap\Cls{B})\cup(\Cls{A}\cap B)$.
    \end{excer}

    \begin{proof}
        %TODO este ejercicio es para reflexionar.
        Veamos que propiedades cumple esta operación. Sean $A,B,C\subseteq X$. Se cumple que:
        \begin{equation*}
            \begin{split}
                \tau(A,A)&\subseteq A\cup A\\
                &=A\\
                \Rightarrow \tau(A,A)&\subseteq A\\
            \end{split}
        \end{equation*}
        Además,
        \begin{equation*}
            \begin{split}
                \tau(\emptyset,B\cup C)\cup\tau(B, C)
                &=\tau(\emptyset,B\cup C)\cup\tau(B, C\cup\emptyset)\\
                &=\tau(\emptyset\cup B,C)\cup\tau(\emptyset, B) \\
            \end{split}
        \end{equation*}
        Tomando $B=X$ se tiene que:
        \begin{equation*}
            \begin{split}
                \tau(\emptyset,X\cup C)\cup\tau(X, C)
            \end{split}
        \end{equation*}
    \end{proof}

    \newpage

    \section{$G_\delta$, $F_\sigma$ y conjuntos de Borel}

    Mis ejercicios de la sección: 4.

    \newpage

    \section{Relativización}

    Mis ejercicios de la sección: 2, 7 y 12.

    \newpage

    \section{Funciones continuas}

    Mis ejercicios de la sección: 6 y 10.

    \newpage

    \section{Definición por partes de funciones}

    Mis ejercicios de la sección: 2.

    \newpage

    \section{Funciones continuas en $\mathbb{E}^1$}

    \newpage

    \section{Funciones abiertos y cerradas}

    \newpage

    \section{Homeomorfismos}
    
    \chapter{Segundo Parical}

    \section{Axiomas de Separación}

    \begin{excer}
        Sean $(X,\tau)$ y $(Y,\sigma)$ espacios topológicos siendo $(Y,\sigma)$ un espacio Hausdorff. Si $\cf{f,g}{(X,\tau)}{(Y,\sigma)}$ son funciones continuas, entonces
        \begin{enumerate}
            \item El conjunto $\left\{x\in X\Big|f(x)=g(x) \right\}$ es cerrado en $(X,\tau)$.
            \item Si $D\subseteq X$ es denso y $f\big|_D=g\big|_D$, entonces $f=g$ en $X$.
            \item La gráfica de la función continua $\cf{f}{(X,\tau)}{(Y,\tau)}$, esto es, el conjunto
            \begin{equation*}
                \Gamma(f)=\left\{(x,f(x))\in X\times Y \Big|x\in X \right\}
            \end{equation*}
            es cerrado en $X\times Y$ con la topología producto.
            \item Si $f$ es inyectiva y continua, entonces $(X,\tau)$ es Hausdorff.
        \end{enumerate}
    \end{excer}

    \begin{proof}
        De (1): Sea
        \begin{equation*}
            C=\left\{x\in X\Big|f(x)=g(x) \right\}
        \end{equation*}
        para probar que este conjunto es cerrado, se probará que $U=X-C$ es abierto en $(X,\tau)$. En efecto, si $x\in X-C$ se tiene que
        \begin{equation*}
            f(x)\neq g(x)
        \end{equation*}
        como el espacio $(Y,\sigma)$ es $T_2$, existen dos abiertos $U,V\subseteq Y$ tales que
        \begin{equation*}
            f(x)\in U,\quad g(x)\in V\quad U\cap V=\emptyset
        \end{equation*}
        se tiene entonces que $x\in W=f^{-1}(U)\cap g^{-1}(V)\neq\emptyset$, donde el conjunto $W\subseteq X$ es abierto por ser intersección de dos abiertos y ser las funciones $f,g$ continuas. Afirmamos que
        \begin{equation*}
            W\subseteq X-C
        \end{equation*}
        Procederemos por contradicción. Suponga que existe $y\in W$ tal que $y\notin X-C$, es decir $y\in C$. Como $y\in W$ se tiene que
        \begin{equation*}
            f(y)\in U,\quad g(y)\in V
        \end{equation*}
        además, al tenerse que $y\in C$ se sigue que $f(y)=g(y)$. Por tanto, $U\cap V\neq\emptyset$\contradiction. Luego debe suceder que $W\subseteq X-C$. Así, para cada $x\in X-C$ se tiene que existe un abierto tal que $x\in W\subseteq X-C$. Se sigue entonces que el conjunto $X-C$ es abierto, es decir que $C$ es cerrado en $(X,\tau)$.

        De (2): Hay que probar que
        \begin{equation*}
            f(x)=g(x),\quad\forall x\in X
        \end{equation*}
        se tienen dos casos (en caso de que $D\subsetneqq X$, si $D=X$ el resultado es inmediato):
        \begin{enumerate}
            \item $x\in D$, como $f|_D=g|_D$ se sigue que $f(x)=f|_D(x)=g|_D(x)=g(x)$.
            \item $x\in X-D$. Procederemos por contradicción. Suponga que $f(x)\neq g(x)$. Como $(Y,\sigma)$ es $T_2$ existen dos abiertos $V_1,V_2\subseteq Y$ tales que
            \begin{equation*}
                f(x)\in V_1,\quad g(x)\in V_2,\quad V_1\cap V_2=\emptyset
            \end{equation*}
            se tiene que $x\in U=f^{-1}(V_1)\cap g^{-1}(V_2)\in\tau$, pues las funciones son continuas. Como $D$ es denso en $X$ y $U\subseteq X$ es un abierto no vacío, existe un elemento $y\in D$ tal que $y\in U$, esto es que
            \begin{equation*}
                f(y)\in V_1\quad g(y)\in V_2
            \end{equation*}
            donde, al tenerse que $f(y)=g(y)$ se sigue que $V_1\cap V_2\neq\emptyset$\contradiction. Por tanto, debe suceder que $f(x)=g(x)$.
        \end{enumerate}
        por los dos incisos anteriores se sigue que $f=g$ en $X$.

        De (3): Sea $A=X\times Y-\Gamma(f)$. Probaremos que $A$ es abierto. En efecto, si $(x,y)\in C$ se tiene que $y\neq f(x)$. Como $(Y,\sigma)$ es $T_2$ existen dos abiertos $V_1,V_2\subseteq Y$ tales que 
        \begin{equation*}
            y\in V_1,\quad f(x)\in V_2\quad V_1\cap V_2=\emptyset
        \end{equation*}
        Como $f$ es continua, el conjunto $U=f^{-1}(V_2)\subseteq X$ es abierto. Ahora, el conjunto
        \begin{equation*}
            W=U\times V
        \end{equation*}
        donde $V=V_1$ es un básico (en particular un abierto) para el cual se tiene que $(x,y)\in W$ y $W\subseteq A$. En efecto, lo primero se tiene de forma inmediata. Suponga que existe $(z,w)\in W$ tal que $(z,w)\notin A$, entonces
        \begin{equation*}
            z\in U,\quad w\in V,\quad\textup{y}\quad w=f(z)
        \end{equation*}
        es decir,
        \begin{equation*}
            z\in f^{-1}(V_2),\quad f(z)\in V_1
        \end{equation*}
        por lo cual
        \begin{equation*}
            f(z)\in V_2, f(z)\in V_1\Rightarrow V_1\cap V_2\neq\emptyset\contradiction
        \end{equation*}
        por ende, $W\subseteq A$. Luego como en (1) debe tenerse que $A$ es abierto en $(X\times Y,\tau_p)$, es decir que $\Gamma(f)$ es cerrado en $(X\times Y,\tau_p)$.

        De (4): Sean $x,y\in X$ tales que $x\neq y$. Como $f$ es inyectiva se sigue que $f(x)\neq f(y)$, luego por ser $(Y,\sigma)$ $T_2$ existen dos abiertos $V_1,V_2\subseteq Y$ tales
        que
        \begin{equation*}
            f(x)\in V_1,\quad f(y)\in V_2,\quad V_1\cap V_2=\emptyset
        \end{equation*}
        sean $U_i=f^{-1}(V_i)$ para $i=1,2$. Estos conjuntos son abiertos en $(X,\tau)$ ya que $f$ es continua. Además
        \begin{equation*}
            U_1\cap U_2=\emptyset
        \end{equation*}
        ya que en caso contrario se tendría que si $z\in U_1\cap U_2$ entonces $f(z)\in V_1\cap V_2=\emptyset$\contradiction. Por ende, $U_1\cap U_2=\emptyset$, siendo tales que $x\in U_1$ y $y\in U_2$. Por ser los $x,y$ arbitrarois en $X$ se tiene entonces que $(X,\tau)$ es $T_2$.
    \end{proof}

    \begin{excer}
        Sea $(Y,\tau)$ un espacio Hausdorff $T_3$ y $A\subseteq Y$ un conjunto infinito. Entonces, existe una familia
        \begin{equation*}
            \left\{U_n\subseteq Y\Big|U_n\textup{ es abierto para todo }n\in\mathbb{N}^* \right\}
        \end{equation*}
        de conjuntos cuyas cerraduras son disjuntas a pares y tales que
        \begin{equation*}
            A\cap U_n\neq\emptyset,\quad\forall n\in\mathbb{N}
        \end{equation*}
    \end{excer}

    \begin{proof}
        Tomemos $U_0=\emptyset$. Suponga elegidos $U_1,...,U_n\subset X$ abiertos con cerraduras disjuntas a pares tales que
        \begin{equation*}
            A\cap U_k=\emptyset,\quad\forall k\in\natint{1,n}
        \end{equation*}
        siendo el conjunto
        \begin{equation*}
            A_n=A-\bigcup_{ k=1}^n\Cls{U_k}
        \end{equation*}
        infinito. Tomemos $a,b\in A_n$ con $a\neq b$. Como el espacio es Hausdorff se tiene que $\left\{b \right\}\subseteq X$ es un conjunto cerrado y es tal que $a\notin\left\{b \right\}$. Ahora, $A-\left(\bigcup_{ k=1}^n\Cls{U_k}\cup\left\{b\right\} \right)$ es un abierto que contiene a $a$. Como el espacio es $T_3$ existe un abierto $V\subseteq X$ tal que
        \begin{equation*}
            a\in V\subseteq\Cls{V}\subseteq A-\left(\bigcup_{ k=1}^n\Cls{U_k}\cup\left\{b\right\} \right) 
        \end{equation*}
        y con ello un abierto $W\subseteq X$ tal que
        \begin{equation*}
            b\in W\subseteq\Cls{W}\subseteq A-\left(\bigcup_{ k=1}^n\Cls{U_k}\cup\Cls{V}\right)
        \end{equation*}
        definamos
        \begin{equation*}
            U_{ n+1}=\left\{
                \begin{array}{lcr}
                    V & \textup{ si } & A\cap\Cls{V}\textup{ es finito} \\
                    W & \textup{ e.o.c} & \\
                \end{array}
            \right.
        \end{equation*}
        es claro que $U_{ n+1}$ es abierto. Se tienen dos casos:
        \begin{enumerate}
            \item $U_{ n+1}=V$:
            \item $U_{ n+1}=W$: 
        \end{enumerate}
    \end{proof}

    \begin{excer}
        Sea $X$ un conjunto infinito. Demuestre que $(X,\tau_{cf})$ no es un espacio $T_2$.
    \end{excer}

    \begin{proof}
        
    \end{proof}

    \begin{excer}
        Sea $\left\{(X_\alpha,\tau_\alpha) \right\}_{\alpha\in I}$ una familia de espacios topológicos. Tomando
        \begin{equation*}
            X=\prod_{\alpha\in I}X_\alpha
        \end{equation*}
        se tiene que si el espacio $(X,\tau_p)$ es normal, entonces $(X_\alpha,\tau_\alpha)$ es normal para todo $\alpha\in I$. 
    \end{excer}

    \begin{excer}
        Sea $X$ un conjunto, $p\in X$ y $K\subseteq X$ tal que $\abs{K}\geq2$
    \end{excer}

    \begin{proof}
        
    \end{proof}

    \begin{excer}
        Sea $(X,\tau)$ un espacio topológico. Demuestre que $(X,\tau)$ es $T_2$ si y sólo si el conjunto
        \begin{equation*}
            \Delta=\left\{(x,y)\in X\times X\Big|x=y \right\}
        \end{equation*}
        es cerrado en $(X\times X,\tau_p)$.
    \end{excer}

    \begin{proof}
        
    \end{proof}

    \begin{excer}
        Sea $(X,\tau)$ un espacio topológico $T_1$. Demuestre que
        \begin{enumerate}
            \item Si $x\in X$ y $A=\left\{x\right\}$, entonces $A'=\emptyset$.
            \item Si $x_1,...,x_n\in X$ y $A=\left\{x_1,...,x_n\right\}$, entonces $A'=\emptyset$.
        \end{enumerate}
    \end{excer}

    \begin{proof}
        
    \end{proof}

    \begin{excer}
        Sea $(X,\tau)$ un espacio topológico $T_2$ y, $A,B\subseteq X$ compactos disjuntos. Pruebe que existen $U,V\in\tau$ tales que
        \begin{equation*}
            A\subseteq U,\quad B\subseteq V\quad\textup{y}\quad U\cap V=\emptyset
        \end{equation*}
    \end{excer}

    \begin{proof}
        
    \end{proof}

    \begin{excer}
        Sea $(X,\tau)$ un espacio topológico regular. Demuestre que, dados $x,y\in X$ distintos existen $U,V\in\tau$ tales que
        \begin{equation*}
            x\in U,\quad y\in V,\quad\textup{y}\quad\Cls{U}\cap\Cls{V}=\emptyset
        \end{equation*}
    \end{excer}

    \begin{proof}
        
    \end{proof}

    \renewcommand{\theenumi}{\roman{enumi}}

    \begin{excer}
        Sea $(X,\tau)$ un espacio $T_2$. Pruebe que si $x\in X$
        \begin{enumerate}
            \item $\bigcap\left\{F\subseteq X\Big|x\in F\textup{ y }F\textup{ es cerrado} \right\}=\left\{x\right\}$.
            \item $\bigcap\left\{U\subseteq X\Big|x\in U\textup{ y }U\textup{ es abierto} \right\}=\left\{x\right\}$.
        \end{enumerate}
        pruebe que ninguna de las dos propiedades anteriores es equivalente a que el espacio sea $T_2$.
    \end{excer}

    \begin{proof}
        De (i): Como $(X,\tau)$ es $T_2$, en particular es $T_1$, luego $\tau_{cf}\subseteq\tau$, así que el conjunto
        \begin{equation*}
            \left\{x\right\}
        \end{equation*}
        es cerrado (por ser finito) y tal que $x\in\left\{x\right\}$. Por tanto,
        \begin{equation*}
            \bigcap\left\{F\subseteq X\Big|x\in F\textup{ y }F\textup{ es cerrado} \right\}=\left\{x\right\}
        \end{equation*}
        (pues la otra contención se tiene de forma inmediata). Observe que no es necesario que el espacio sea $T_2$ para que la condición anterior se cumpla.

        De (ii): Suponga que
        \begin{equation*}
            \bigcap\left\{U\subseteq X\Big|x\in U\textup{ y }U\textup{ es abierto} \right\}\neq\left\{x\right\}
        \end{equation*}
        como $x$ se encunetra en todo abierto que lo contiene, forzosamente debe existir $y\in X$ tal que
        \begin{equation*}
            y\in\bigcap\left\{U\subseteq X\Big|x\in U\textup{ y }U\textup{ es abierto} \right\}
        \end{equation*}
        Entonces, para todo $U\in\tau$ tal que $x\in\tau$ implica que $y\in\tau$. Como $(X,\tau)$ es $T_2$ existen $V,W\in\tau$ tales que
        \begin{equation*}
            x\in W,\quad y\in V,\quad\textup{ y }\quad W\cap V=\emptyset
        \end{equation*}
        pero, la segunda condición implica que $y\notin W$\contradiction. Por tanto,
        \begin{equation*}
            \bigcap\left\{U\subseteq X\Big|x\in U\textup{ y }U\textup{ es abierto} \right\}=\left\{x\right\}
        \end{equation*}
        Para el ejemplo, tome $(\mathbb{N},\tau_{cf})$. Este espacio claramente no es $T_2$, pero si cumple la condición requerida.
    \end{proof}

    \begin{excer}
        Sea $X$ un conjunto finito. Pruebe que la única topología $\tau$ que hace de $(X,\tau)$ un espacio $T_2$ es la discreta. 
    \end{excer}

    \begin{proof}
        Supongamos que $X=\left\{x_1,...,x_n\right\}$. Hay que probar que
        \begin{equation*}
            \tau_D\subseteq\tau
        \end{equation*}
        (suponiendo que $(X,\tau)$ es $T_2$). En efecto, como $(X,\tau)$ es $T_2$, entonces los conjuntos $\left\{x_i\right\}$ son cerrados, para todo $i\in\natint{1.n}$. En particular, para todo $i\in\natint{1,n}$:
        \begin{equation*}
            \bigcup_{ j\in\natint{1,n},j\neq i}\left\{x_j\right\}=X-\left\{x_i\right\}
        \end{equation*}
        es cerrado (por ser unión finita de cerrados), luego su complemento $\left\{x_i\right\}$ es abierto en $(X,\tau)$. Por tanto, $\tau_D\subseteq\tau$.
    \end{proof}

    \begin{excer}
        Sea $(X,\tau)$ un espacio $T_2$. Pruebe que si $A\subseteq X$, entonces
        \begin{enumerate}
            \item $A'$ es cerrado.
            \item $\left(A'\right)'\subseteq A'$.
            \item $\left(\Cls{A}\right)'= A'$.
        \end{enumerate}
    \end{excer}

    \begin{proof}
        Sea $A\subseteq X$.
        
        De (i): Veamos que $X-A'$ es abierto. En efecto, primero recordemos que
        \begin{equation*}
            x\in A'\iff \forall V\in\tau\textup{ tal que }x\in V\Rightarrow (V-\left\{x\right\})\cap A\neq\emptyset
        \end{equation*}
        Por tanto,
        \begin{equation*}
            x\in X-A'\iff \exists U_0\in\tau \textup{ tal que }x\in U_0\textup{ y }(U_0-\left\{x\right\})\cap A=\emptyset
        \end{equation*}
        Afirmamos que para $x\in X-A'$, $U_0\subseteq X-A'$. En efecto, en caso contrario si existiera $y\in U_0$ tal que $y\in A'$ (en particular, $y\neq x$ pues $x\in X-A'$), por la primera condición:
        \begin{equation*}
            (U_0-\left\{y\right\})\cap A\neq\emptyset
        \end{equation*}
        Si $x\notin A$, entonces
        \begin{equation*}
            (U_0-\left\{x\right\})\cap A\supseteq(U_0-\left\{y\right\})\cap A\neq\emptyset 
        \end{equation*}
        lo cual es una contradición. Si $x\in A$, como el espacio es $T_2$ existen dos abiertos $W,V\in\tau$ tales que
        \begin{equation*}
            y\in V,\quad x\in W,\quad V\cap W=\emptyset
        \end{equation*}
        Tomemos $V_0=U_0\cap V$. Se tiene que
        \begin{equation*}
            (V_0-\left\{y\right\})\cap A\neq\emptyset
        \end{equation*}
        pues $y\in A'$. Luego, como $x\notin V_0$ (pues $x\notin V$), se sigue que existe $z\in (V_0-\left\{y\right\})\cap A$, en particular $z\neq x$ y $z\in U_0$, luego
        \begin{equation*}
            (U_0-\left\{x\right\})\cap A=\emptyset
        \end{equation*}
        lo cual es una contradicción. Por ende, $U_0\subseteq X-A'$. Así, el conjunto $X-A'$ es abierto, luego $A'$ es cerrado.

        De (ii): Si $x\in\left(A'\right)'$, entonces
        \begin{equation*}
            \forall U\in\tau\textup{ tal que }x\in U\Rightarrow \left(U-\left\{x\right\} \right)\cap A'\neq\emptyset
        \end{equation*}
        sea $V=U-\left\{x\right\}=U\cap(X-\left\{x\right\})\in\tau$ (pues $\left\{x\right\}$ es abierto). Entonces, al tenerse que $V\cap A'\neq\emptyset$, existe $y\in X$ tal que $y\in V\cap A'$, en particular $y\in A'$, luego como $V\in\tau$ es tal que $y\in V$, se sigue que
        \begin{equation*}
            (V-\left\{y\right\})\cap A\neq\emptyset
        \end{equation*} 
        así, existe $z\in V-\left\{y\right\}$ tal que $z\in A$, en particular,
        \begin{equation*}
            z\in V\cap A\Rightarrow (U-\left\{x\right\})\cap A\neq\emptyset
        \end{equation*}
        es decir, que $x\in A'$. Por tanto,
        \begin{equation*}
            \left(A'\right)'\subseteq A'
        \end{equation*}

        De (iii): Una contención es inmediata del hecho de que $A\subseteq\Cls{A}$, pues
        \begin{equation*}
            \begin{split}
                x\in A'\iff& \forall U\in\tau\textup{ tal que }x\in U\Rightarrow (U-\left\{x\right\})\cap A\neq\emptyset\\
                \Rightarrow\:&\forall U\in\tau\textup{ tal que }x\in U\Rightarrow (U-\left\{x\right\})\cap\Cls{A}\neq\emptyset\\
                \iff& x\in\left(\Cls{A}\right)'\\
            \end{split}
        \end{equation*}
        Así, $A'\subseteq\left(\Cls{A}\right)'$. Si $x\in\left(\Cls{A}\right)'$, entonces
        \begin{equation*}
            \forall U\in\tau\textup{ tal que }x\in U\Rightarrow (U-\left\{x\right\})\cap\Cls{A}\neq\emptyset
        \end{equation*}
        entonces, existe $y\in U-\left\{x\right\}$ tal que $y\in\Cls{A}$. Si $y\in A$ hemos terminado. Suponga que $y\notin A$. Como $V=U-\left\{x\right\}$ es un abierto tal que $y\in V$, entonces
        \begin{equation*}
            V\cap A\neq\emptyset
        \end{equation*}
        así, existe $z\in V$ tal que $z\in A$, en particular
        \begin{equation*}
            (U-\left\{x\right\})\cap A\neq\emptyset
        \end{equation*}
        por tanto, $x\in A'$.
    \end{proof}

    \begin{excer}
        Sea $\cf{f}{(X,\tau)}{(Y,\sigma)}$ y $\cf{g}{(Y,\sigma)}{(X,\tau)}$ funciones continuas tales que $g\circ f=1_X$. Pruebe que si $(Y,\tau)$ es $T_2$ implica que $(X,\tau)$ es $T_2$, y que $f(X)$ es cerrado en $(Y,\sigma)$.
    \end{excer}

    \begin{proof}
        Notemos que como
        \begin{equation*}
            g\circ f=1_X
        \end{equation*}
        entonces $g$ es suprayectiva y $f$ es inyectiva. En efecto, veamos que $g$ es suprayectiva, sea $x\in X$, entonces existe $f(x)\in Y$ tal que
        \begin{equation*}
            g(f(x))=x
        \end{equation*}
        Ahora, sean $x_1,x_2\in X$ tales que $f(x_1)=f(x_2)$, entonces
        \begin{equation*}
            g\circ f(x_1)=g\circ f(x_2)\Rightarrow x_1=x_2
        \end{equation*}
        Luego, se sigue lo deseado.

        Suponga que $(Y,\sigma)$ es $T_2$. Sean $x,y\in X$ tales que $x\neq y$. Como $(Y,\sigma)$ es $T_2$ y $f(x)\neq f(y)$ entonces existen dos abiertos $V_1,V_2\in\sigma$ tales que
        \begin{equation*}
            f(x)\in V_1,\quad f(y)\in V_2,\quad\textup{y}\quad V_1\cap V_2=\emptyset
        \end{equation*}
        tomemos $U_1=f^{-1}(V_1)$ y $U_2=f^{-1}(V_2)$. Se tiene que
        \begin{equation*}
            x_1\in U_1\quad x_2\in U_2
        \end{equation*}
        y, $U_1\cap U_2=\emptyset$. En efecto, si $x\in U_1\cap U_2$ entonces $x\in f^{-1}(V_1)\cap f^{-1}(V_2)$, esto es que $f(x)\in V_1$ y $f(x)\in V_2$\contradiction. Por ende, $U_1\cap U_2=\emptyset$.

        Así, $(X,\tau)$ es $T_2$.

        Ahora, veamos que $f(X)$ es cerrado en $(Y,\sigma)$. En efecto, veamos que su complemento es abierto. Si $y\in Y-f(X)$, entonces
        \begin{equation*}
            \begin{split}
                y\in Y-f(X)&\iff y\notin f(X)\\
                &\iff y\neq f(x),\quad\forall 
                x\in X\\
            \end{split}
        \end{equation*}
        %TODO
    \end{proof}

    \section{Filtros}

    \begin{excer}
        Sea $(Y,d)$ un espacio métrico, y $\left\{ y_n\right\}_{ n=1}^\infty$ una sucesión en $Y$. Pruebe que $y_n\rightarrow y_0$ si y sólo si $d(y_n,y_0)\rightarrow0$.
    \end{excer}

    \begin{proof}
        $\Rightarrow)$: Suponga que $y_n\rightarrow y_0$ con $y_0\in Y$. Entonces
        \begin{equation*}
            U\in\tau\textup{ tal que }y_0\in U\Rightarrow\exists N\in\mathbb{N}\textup{ para el cual }n\geq N\textup{ implica que }y_n\in U
        \end{equation*}
        en particular:
        \begin{equation*}
            \forall \varepsilon>0\exists N\in\mathbb{N}\textup{ tal que }n\geq N\Rightarrow y_n\in B(y_0,\varepsilon)
        \end{equation*}
        es decir que
        \begin{equation*}
            \forall \varepsilon>0\exists N\in\mathbb{N}\textup{ tal que }n\geq N\Rightarrow \abs{d(y_n,y_0)-0}<\varepsilon
        \end{equation*}
        lo cual prueba el resultado.
    \end{proof}

    \begin{excer}
        Sea $(X,\tau)$ un espacio topológico.
        \begin{enumerate}
            \item Encuentre un ejemplo de un filtro definido sobre $X$ que converja a dos puntos distintos.
            \item Sea $\Delta=\left\{(x,y)\in X\times X\Big|x=y \right\}$, demuestre que $\Delta$ es un conjunto cerrado en $(X\times X,\tau_p)$ si y sólo si dado un filtro $\mathcal{F}$ sobre $X$ convergente, este converge a un único punto.
        \end{enumerate}
    \end{excer}

    \begin{excer}
        Sean $X,Y$ dos conjuntos y $\cf{f}{X}{Y}$ una función. Si $\xi$ es un ultrafiltro sobre $X$, entonces el filtro $f(\xi)^+$, generado por la base de filtro $f(\xi)$ es un ultrafiltro en $Y$.
    \end{excer}

    \begin{proof}
        Recordemos que
        \begin{equation*}
            f(\xi)^+=\left\{A\subseteq Y\Big|\textup{existe }E\in\xi\textup{ tal que }f(E)\subseteq A \right\}
        \end{equation*}
        Por una proposición anterior ya se sabe que $f(\xi)^+$ es filtro sobre $Y$. Veamos que es ultrafiltro. Primero, probaremos que dado un conjunto $A\subseteq Y$, uno de los dos conjuntos $A,Y-A$ está en $f(\xi)^+$. En efecto, sea
        \begin{equation*}
            B=f^{-1}(A)
        \end{equation*}
        Como $B\subseteq X$, entonces $B\in\xi$ ó $X-B\in \xi$.
        \begin{itemize}
            \item Suponga que $B\in\xi$, entonces se sigue por la definición de $B$ que
            \begin{equation*}
                f(B)\in f(\xi)
            \end{equation*}
            y, como $f(f^{-1}(A))\subseteq A$, se sigue que $f(B)\subseteq A$, luego $A\in\xi$.
            \item Suponga que $X-B\in\xi$. Afirmamos que
            \begin{equation*}
                X-B=X-f^{-1}(A)=f^{-1}(Y-A)
            \end{equation*}
            en efecto, veamos que
            \begin{equation*}
                \begin{split}
                    x\in X-f^{-1}(A)&\iff f(x)\notin A\\
                    &\iff f(x)\in Y-A\\
                    &\iff x\in f^{-1}(Y-A)\\
                \end{split}
            \end{equation*}
            lo cual prueba el resultado. Por tanto, como $X-B=X-f^{-1}(A)\in \xi$, se sigue que $f^{-1}(Y-A)\in f(\xi)$. Luego, como $f(f^{-1}(Y-A))\subseteq Y-A$, entonces se tiene que $Y-A\in f(\xi)^+$.
        \end{itemize}
        Por tanto, $A\in f(\xi)^+$ ó $Y-A\in f(\xi)^+$ (no pueden estar los dos a la vez por ser $f(\xi)^+$ filtro). Luego, por una proposición $f(\xi)^+$ es ultrafiltro.
    \end{proof}

    \begin{excer}
        Sean $\mathcal{F}_1,...,\mathcal{F}_n$ filtros definidos sobre $X$ y $\mathcal{U}$ un ultrafiltro sobre $X$ tal que $\mathcal{F}_1\cap \cdots\cap \mathcal{F}_n\subseteq\mathcal{U}$. Demuestre que existe $i\in\natint{1,n}$ tal que $\mathcal{F}_i\subseteq\mathcal{U}$.
    \end{excer}

    \begin{proof}
        Procederemos por inducción sobre $n$.
        \begin{itemize}
            \item Considere el caso $n=2$. Suponga que $\mathcal{F}_1,\mathcal{F}_2\nsubseteq\mathcal{U}$, entonces existen $A_1\in\mathcal{F}_1$ y $A_2\in\mathcal{F}_2$ tales que $A_1,A_2\notin\mathcal{U}$, en particular por ser $\mathcal{U}$ ultrafiltro se tiene que $A_1\cup A_2\notin\mathcal{U}$. Tomemos
            \begin{equation*}
                A=A_1\cup A_2
            \end{equation*}
            se tiene por absorción que $A_1\cup A_2 \in\mathcal{F}_1,\mathcal{F}_2$. Por tanto,
            \begin{equation*}
                A_1\cup A_2\in\mathcal{F}_1\cap\mathcal{F}_2\subseteq\mathcal{U}
            \end{equation*}
            luego, $A_1\cup A_2\in\mathcal{U}$\contradiction. Por tanto, alguno de los dos $\mathcal{F}_1,\mathcal{F}_2$ tiene que estar contenido en $\mathcal{U}$.
            \item Suponga que existe $k\in\mathbb{N}$, $k\geq 2$ tal que si $\mathcal{F}_1,...,\mathcal{F}_k$ son filtros sobre $X$ y $\mathcal{U}$ un ultrafiltro tal que $\mathcal{F}_1\cap...\cap\mathcal{F}_k\subseteq\mathcal{U}$, entonces existe $i\in\natint{1,k}$ tal que $\mathcal{F}_i\subseteq\mathcal{U}$.
            \item Veamos que se cumple para $k+1$. En efecto, sean $\mathcal{F}_1,...,\mathcal{F}_{ k+1}$ filtros sobre $X$ y $\mathcal{U}$ un ultrafiltro sobre $X$ tal que
            \begin{equation*}
                \mathcal{F}_1\cap...\cap\mathcal{F}_k\cap\mathcal{F}_{ k+1}\subseteq\mathcal{U}
            \end{equation*}
            Sea $\xi=\mathcal{F}_1\cap...\cap\mathcal{F}_k$. Por una proposición $\xi$ es un filtro, luego por el caso $n=2$ se debe tener que $\xi\subseteq\mathcal{U}$ o $\mathcal{F}_{ k+1}\subseteq\mathcal{U}$. Si se tiene el segundo caso, se sigue el resultado tomando $i=k+1$. En el primer caso, usando la hipótesis de inducción se sigue que existe $i\in\natint{1,k}$ tal que $\mathcal{F}_i\subseteq\mathcal{U}$.

            En ambos casos, se tiene que existe $i\in\natint{1,k+1}$ tal que $\mathcal{F}_i\subseteq\mathcal{U}$.
        \end{itemize}
        Aplicando inducción, el resultado se tiene para todo $n\in\mathbb{N}$.
    \end{proof}

    \begin{excer}
        Sea $\mathcal{U}=\left\{A_\alpha\right\}_{\alpha\in I}$ una base de filtro en $X$ y $\mathcal{B}=\left\{B_\beta\right\}_{\beta\in J}$ una base de filtro en $Y$ (siendo $X,Y$ conjuntos no vacíos). Pruebe que
        \begin{equation*}
            \mathcal{U}\times\mathcal{B}=\left\{A_\alpha\times B_\beta\Big|(\alpha,\beta)\in I\times J \right\}
        \end{equation*}
        es una base de filtro en $X\times Y$.
    \end{excer}

    \begin{proof}
        En efecto, hay que verificar que se cumplen dos condiciones:
        \begin{enumerate}
            \item Claramente la familia $\mathcal{U}\times\mathcal{B}$ es de conjuntos no vacíos, pues cada una de $\mathcal{U}$ y $\mathcal{B}$ es no vacía de conjuntos no vacíos, luego todo elemento del producto de ambos es no vacío.
            \item Sean $A_{\alpha_1}\times B_{\beta_1},A_{\alpha_2}\times B_{\beta_2}\in\mathcal{U}\times\mathcal{B}$. Como $A_{\alpha_1},A_{\alpha_2}\in\mathcal{U}$ y $B_{\beta_1},B_{\beta_2}\in\mathcal{B}$, entonces al ser bases de filtro se tiene que existen $A_{\alpha_3}\in\mathcal{U}$ y $B_{\beta_3}\in\mathcal{B}$ tales que
            \begin{equation*}
                A_{\alpha_3}\subseteq A_{\alpha_1}\cap A_{\alpha_2}\quad\textup{y}\quad B_{\beta_3}\subseteq B_{\beta_1}\cap B_{\beta_2}
            \end{equation*}
            Se tiene luego que $A_{\alpha_3}\times B_{\beta_3}\in\mathcal{U}\times\mathcal{B}$. Además,
            \begin{equation*}
                A_{\alpha_3}\times B_{\beta_3}\subseteq A_{\alpha_1}\times B_{\beta_1}\cap A_{\alpha_2}\times B_{\beta_2}
            \end{equation*}
            por la forma en que se tomaron estos elementos.
        \end{enumerate}
        por los dos incisos anteriores, se sigue que $\mathcal{U}\times\mathcal{B}$ es base de filtro sobre $X\times Y$.
    \end{proof}

    \begin{excer}
        Sea $\cf{f}{X}{Y}$ una función (siendo $X,Y$ conjuntos no vacíos) y $\mathcal{B}$ una base de filtro en $Y$. Pruebe que
        \begin{equation*}
            f^{-1}(\mathcal{B})=\left\{f^{-1}(B)\Big|B\in\mathcal{B} \right\}
        \end{equation*}
        es una base de filtro en $X$ si y sólo si $f^{-1}(B)\neq\emptyset$ para todo $B\in\mathcal{B}$.
    \end{excer}

    \begin{proof}
        $\Rightarrow):$ Suponga que $f^{-1}(\mathcal{B})$ es una base de filtro en $X$, en particular se tiene que es una familia no vacía de conjuntos no vacíos, es decir que
        \begin{equation*}
            f^{-1}(B)\neq\emptyset,\quad\forall B\in\mathcal{B}
        \end{equation*}
        (pues todo elemento de la base de filtro es de esa forma) lo que prueba el resultado.

        $\Leftarrow):$ Suponga que $f^{-1}(B)\neq\emptyset$ para todo $B\in\mathcal{B}$. Veamos que $f^{-1}(B)$ es base de filtro. En efecto, se deben verificar dos condiciones:
        \begin{enumerate}
            \item $f^{-1}(\mathcal{B})$ es una familia no vacía, pues $\mathcal{B}$ es una familia no vacía, y es de conjuntos no vacíos, pues por hipótesis todo elemento es de la forma
            \begin{equation*}
                f^{-1}(B)\neq\emptyset,\quad\forall B\in\mathcal{B}
            \end{equation*}
            el cual es no vacío.
            \item Sean $B_1,B_2\in\mathcal{B}$. Entonces, existe $B_3\in\mathcal{B}$ tal que
            \begin{equation*}
                B_3\subseteq B_1\cap B_2
            \end{equation*}
            tomando imágenes inversas se tiene que
            \begin{equation*}
                f^{-1}(B_3)\subseteq f^{-1}(B_1\cap B_2)
            \end{equation*}
            Ahora, veamos que
            \begin{equation*}
                \begin{split}
                    x\in f^{-1}(B_1\cap B_2)&\iff f(x)\in B_1\cap B_2\\
                    &\iff f(x)\in B_1\textup{ y }f(x)\in B_2\\
                    &\iff x\in f^{-1}(B_1)\textup{ y }x\in f^{-1}(B_2)\\
                    &\iff x\in f^{-1}(B_1)\cap f^{-1}(B_2)\\
                \end{split}
            \end{equation*}
            por tanto,
            \begin{equation*}
                f^{-1}(B_3)\subseteq f^{-1}(B_1)\cap f^{-1}(B_2)
            \end{equation*}
            donde los dos elementos de la derecha están en $f^{-1}(\mathcal{B})$ y el de la izquierda también lo está.
        \end{enumerate}
        Por los dos incisos anteriores se sigue que $f^{-1}(\mathcal{B})$ es base de filtro.
    \end{proof}

    \begin{excer}
        Pruebe que el conjunto de puntos de acumulación de una base de filtro es cerrado (posiblemente vacío).
    \end{excer}

    \begin{proof}
        Sea $(X,\tau)$ un espacio topológico y $\mathcal{B}$ una base de filtro. Recordemos que
        \begin{equation*}
            x\in X\textup{ es punto de acumulación de }\mathcal{B}\iff x\in \Cls{B},\quad\forall B\in\mathcal{B}
        \end{equation*}
        es decir que $x$ es punto de acumulación de $\mathcal{B}$ si y sólo si
        \begin{equation*}
            x\in\bigcap_{ i\in I}\Cls{B_i}
        \end{equation*}
        donde $\mathcal{B}=\left\{B_i\right\}_{ i\in I}$. Es decir que el conjunto de puntos de acumulación de una base de filtro es
        \begin{equation*}
            \mathcal{B}'=\bigcap_{ i\in I}\Cls{B_i}
        \end{equation*}
        el cual es cerrado por ser intersección arbitraria de cerrados.
    \end{proof}

    \section{Compacidad}

    \begin{excer}
        Sea $(X,\tau)$ un espacio topológico que no es compacto. Pruebe que
        \begin{equation*}
            \mathcal{B}=\left\{U\subseteq X\Big|U=X-C\textup{ donde }C\subseteq X\textup{ es compacto} \right\}
        \end{equation*}
        es base de un filtro sobre $X$. Pruebe además que si $(X,\tau)$ es compacto, entonces $\mathcal{B}$ no es un filtro.
    \end{excer}

    \begin{proof}
        Primero veamos que si el espacio $(X,\tau)$ es compacto, $\mathcal{B}$ no es filtro. En efecto, en particular se tendría que $X\subseteq X$ es compacto, luego
        \begin{equation*}
            \emptyset = X-X\in\mathcal{F}
        \end{equation*}
        así, $\mathcal{B}$ no puede ser filtro.

        Suponga que $(X,\tau)$ no es compacto. Veamos que $\mathcal{F}$ es un fitro. En efecto, se deben cumplir cuatro condiciones:
        \begin{enumerate}
            \item $\emptyset\notin\mathcal{B}$, como $(X,\tau)$ no es compacto, entonces $X$ no es un subconjunto compacto de $(X,\tau)$, luego $\emptyset=X-X\notin\mathcal{B}$.
            \item $\mathcal{B}\neq\emptyset$, como $\emptyset$ es compacto en $(X,\tau)$, entonces $X=X-\emptyset\in\mathcal{B}$, luego $\mathcal{B}\neq\emptyset$.
            \item Sean $A,B\in\mathcal{B}$, entonces existen $C_1,C_2\subseteq X$ compactos tales que
            \begin{equation*}
                A=X-C_1,\quad\textup{y}\quad B=X-C_2
            \end{equation*}
            así,
            \begin{equation*}
                \begin{split}
                    A\cap B&=(X-C_1)\cap (X-C_2)\\
                    &=X-(C_1\cup C_2)\\
                \end{split}
            \end{equation*}
            donde $C_1\cup C_2$ es compacto en $(X,\tau)$ (por ser unión finita de compactos). Luego, $A\cap B\in\mathcal{B}$.
        \end{enumerate}
        Por los tres incisos anteriores, se sigue que $\mathcal{B}$ es base de un filtro sobre $X$.
    \end{proof}

    \begin{excer}
        Sea $\cf{f}{X}{Y}$ una función. Pruebe que $f$ es inyectiva si y sólo si para todo filtro $\mathcal{F}$ sobre $Y$, $f^{-1}(\mathcal{F})$ es filtro sobre $X$.
    \end{excer}

    \begin{proof}
        
    \end{proof}
        
    \begin{excer}
        Sean $(X_1,\tau_1)$ y $(X_2,\tau_2)$ espacios topológicos $T_2$ localmente compactos que no son compactos. Si $(X_1,\tau_1)$ y $(X_2,\tau_2)$ son homeomorfos, demuestre que $(\hat{X_1},\hat{\tau_1})$ y $(\hat{X_2},\hat{\tau_2})$ también lo son.
    \end{excer}
    
    \begin{proof}
        Supongamos que la compactificación de Alexandroff es tal que
        \begin{equation*}
            \hat{X_1}=X_1\cup\left\{\infty_1 \right\}\quad\textup{y}\quad\hat{X_2}=X_2\cup\left\{\infty_2 \right\}
        \end{equation*}
        donde $\infty_1\notin X_1$ y $\infty_2\notin X_2$.

        Sea $\cf{f}{(X_1,\tau_1)}{(X_2,\tau_2)}$ el homeomorfismo entre ambos espacios. Defina $\cf{\hat{f}}{(\hat{X_1},\hat{\tau_1})}{(\hat{X_2},\hat{\tau_2})}$ como sigue
        \begin{equation*}
            \hat{f}(x)=\left\{ 
                \begin{array}{lcr}
                    f(x) & \textup{ si } & x\in X_1\\
                    \infty_2 & \textup{ si } & x=\infty_1\\
                \end{array}
            \right.
        \end{equation*}
        Veamos que esta función es homeomorfismo. En efecto, primero veamos que es biyección.
        \begin{itemize}
            \item \textbf{$\hat{f}$ es inyectiva}. Sean $x,y\in \hat{X_1}$ tales que $x\neq y$. Si $x,y\in X_1$ entonces como $f$ es inyectiva, se tiene que
            \begin{equation*}
                \hat{f}(x)=f(x)\neq f(y)=\hat{f}(y)
            \end{equation*}
            Si $x=\infty_1$, entonces $\hat{f}(x)=\infty_2\notin X_2$, luego $\hat{f}(x)\neq \hat{f}(y)$ (independientemente del valor de $y$). De forma análoga se tiene el resultado si $y=\infty_1$.
            
            Por tanto, $\hat{f}$ es inyectiva.
            \item \textbf{$\hat{f}$ es suprayectiva}. Sea $u\in \hat{X_2}$, si $u\in X_2$ como $f$ es suprayectiva, existe $x\in X_1$ tal que $f(x)=\hat{f}(x)=u$. Si $u=\infty_2$, existe $x=\infty_1$ tal que $\hat{f}(x)=u$.
        \end{itemize}
        Por tanto, de los dos incisos anteriores se sigue que $\hat{f}$ es biyectiva. Para ver que es homeomorfismo, basta con verificar que es continua y abierta.
        \begin{itemize}
            \item \textbf{$\hat{f}$ es continua}. Sea $V\in\hat{\tau_2}$. Se tienen dos casos:
            \begin{enumerate}
                \item $V\in\tau_2$, en cuyo caso se tiene que
                \begin{equation*}
                    \hat{f}^{-1}(V)=f^{-1}(V)\in\tau_1\subseteq\hat{\tau_1}
                \end{equation*}
                pues, $V\subseteq X_2$.
                \item $V=\hat{X_2}-C$, donde $C\subseteq X_2$ es compacto en $(X_2,\tau_2)$. Veamos que
                \begin{equation*}
                    \begin{split}
                        \hat{f}^{-1}(V)&=\hat{f}^{-1}(X_2-C)\\
                        &=\hat{f}^{-1}(X_2)-\hat{f}^{-1}(C)\\
                        &=X_1-f^{-1}(C)\\
                    \end{split}
                \end{equation*}
                pues, $C\subseteq X_2$. Afirmamos que $f^{-1}(C)$ es compacto. En efecto, como $\cf{f^{-1}}{(X_2,\tau_2)}{(X_1,\tau_1)}$ es continua, la imagen de compactos es compacta, luego $f^{-1}(C)$ es un compacto en $(X_1,\tau_1)$, luego $\hat{f}^{-1}(V)\in\hat{\tau_1}$.
            \end{enumerate}
            por los dos incisos anteriores se sigue que $\hat{f}$ es continua.
            \item \textbf{$\hat{f}$ es abierta}. Sea $U\in\hat{\tau_1}$, se tienen dos casos:
            \begin{enumerate}
                \item $U\in\tau_2$, entonce
                \begin{equation*}
                    \begin{split}
                        \hat{f}(U)&=f(U)\in\tau_2\subseteq\hat{\tau_2}
                    \end{split}
                \end{equation*}
                \item $U=\hat{X_1}-C$ donde $C\subseteq (X_1,\tau_1)$ es compacto. Veamos que
                \begin{equation*}
                    \begin{split}
                        \hat{f}(U)&=\hat{f}(\hat{X_1}-C)\\
                        &=\hat{f}(\hat{X_1})-\hat{f}(C)\\
                        &=\hat{X_2}-f(C)\\
                    \end{split}
                \end{equation*}
                donde $f(C)$ es compacto en $(X_2,\tau_2)$ pues $C$ es compacto en $(X_1,\tau_1)$, luego $\hat{f}(U)\in\hat{\tau_2}$.
            \end{enumerate}
            por los dos incisos anteriores se sigue que $\hat{f}$ es abierta.
        \end{itemize}
        Como $f$ es una función biyectiva, continua y abierta, se sigue que $\hat{f}$ es homeomorfismo. Así, $(\hat{X_1},\hat{\tau_1})$ y $(\hat{X_2},\hat{\tau_2})$ son homeomorfos.
    \end{proof}

    \begin{excer}
        Sea $\left\{(X_\alpha,\tau_\alpha) \right\}_{\alpha\in I}$ una familia de espacios topológicos localmente compactos y suponga que existe $J\subseteq I$ finito tal que
        \begin{equation*}
            \forall \beta\in I-J, (X_\beta,\tau_\beta)\textup{ es compacto}
        \end{equation*}
        Demuestre que $\left(X=\prod_{\alpha\in I}X_\alpha,\tau_p \right)$ es localmente compacto.
    \end{excer}

    \begin{proof}
        Sea $x=\left( x_i\right)_{ i\in I} \in X$ arbitrario. Debemos encontrar una vecindad $C=\prod_{i\in I}C_i \subseteq X$ compacta de $x$.
        
        Sea $i\in I$, se tienen dos casos:
        \begin{itemize}
            \item $i\in I-J$: Tomemos $C_i=X_i$, el cual es compacto en $(X_i,\tau_i)$.
            \item $i\in J$: Como $(X_i,\tau_i)$ es localmente compacto y $x_i\in X_i$, existe $C_i\subseteq C_i$ vecindad compacta de $X$.
        \end{itemize}
        Tomemos $C=\prod_{ i\in I}C_i$. Esta es una vecindad (\textit{¿Por qué?}) de $x$. Además es compacta pues cada $C_i$ es compacto, luego por Tikhonov el producto cartesiano dotado de la topología producto es compacto.
    \end{proof}

    \chapter{Tercer Parcial}

    \section{Axiomas de Numerabilidad}

    \section{Separabilidad}

    \begin{excer}
        Sea $(X,\tau)$ un espacio topológico separable. Sea $\mathcal{U}=\left\{U_\alpha \right\}_{\alpha\in I}\subseteq\tau$ tal que
        \begin{equation*}
            U_\alpha\cap U_\beta=\emptyset,\quad\forall \alpha,\beta\in I,\alpha\neq\beta
        \end{equation*}
        Demuestre que $\mathcal{U}$ es a lo sumo numerable.
    \end{excer}

    \begin{proof}
        Como el espacio $(X,\tau)$ es separble, existe $A\subseteq X$ a lo sumo numerable tal que
        \begin{equation*}
            \Cls{A}=X
        \end{equation*}
        (es decir que $A$ es denso en $(X,\tau)$). Veamos que $\mathcal{U}$ es a lo sumo numerable. En efecto, como $A$ es denso en $(X,\tau)$, entonces para cada $\alpha\in I$ se tiene que
        \begin{equation*}
            U_\alpha\cap A\neq\emptyset, \textup{ pues }U_\alpha\in\tau
        \end{equation*}
        Para cada $\alpha\in I$ escogemos un único $x_\alpha\in X$ tal que $x_\alpha\in U_\alpha\cap A$. Construyamos así al función $\cf{f}{I}{X}$ tal que $\alpha\mapsto x_\alpha$. Por construcción esta función está bien definida.

        Afirmamos que $f$ es inyectiva. En efecto, sean $\alpha,\beta\in I$ tales que $\alpha\neq\beta$. Como $\alpha\neq\beta$ entonces
        \begin{equation*}
            U_\alpha\cap U_\beta=\emptyset
        \end{equation*}
        Dado a que $x_\alpha\in U_\alpha\cap A$ y $x_\beta\in U_\beta\cap A$, no puede suceder que $x_\alpha=x_\beta$, pues en tal caso se tendría que $U_\alpha\cap U_\beta\neq\emptyset$\contradiction. Luego, $f$ es inyectiva. Por Cantor-Bernstein se sigue que
        \begin{equation*}
            \abs{I}\leq\abs{f(I)}\leq\abs{A}\leq\aleph_0
        \end{equation*}
        pues, $f(I)\subseteq A$. Por tanto, $I$ es a lo sumo numerable, es decir que $\mathcal{U}$ es a lo sumo numerable.
    \end{proof}

    \begin{mydef}
        Sea $(X,\tau)$ un espacio topológico y $A\subseteq X$. Dado $x\in X$ se dice que \textbf{$x$ es punto de condensación de $A$} si para todo $U\in\mathcal{V}_x$ se tiene que $U\cap A$ es un conjunto no numerable.
    \end{mydef}

    \begin{propo}
        Sean $(X,\tau)$ un espacio Lindelöf y $A\subseteq X$ cerrado. Entonces, $(A,\tau_A)$ es Lindelöf.
    \end{propo}

    \begin{proof}
        Sea $\mathcal{V}=\left\{V_\alpha\right\}_{\alpha\in I}\subseteq\tau_A$ una cubierta abierta de $A$, es decir
        \begin{equation*}
            \bigcup_{\alpha\in I}V_\alpha=A
        \end{equation*}
    \end{proof}

    \begin{excer}
        Si $(X,\tau)$ es un espacio Lindelöf y $A\subseteq X$ es no numerable, demuestre que existe $x\in X$ tal que $x$ es punto de condensación de $A$.
    \end{excer}

    \begin{proof}
        Procederemos por contradicción. Suponga que para todo $x\in X$, $x$ no es punto de condensación de $A$, es decir que para todo $x\in X$ existe $U_x\in\mathcal{V}_x$ (una vecindad que en particular podemos tomar abierta de $x$) tal que
        \begin{equation*}
            U_x\cap A\textup{ es a lo sumo numerable}
        \end{equation*}
        Constrúyase así la cubierta abierta $\left\{U_x\right\}_{x\in X}$ de $X$. Como $(X,\tau)$ es Lindelöf, entonces existe $\left\{U_{x_i} \right\}_{ i=1}^\infty$ subcubierta a lo sumo numerable de $X$, esto es
        \begin{equation*}
            X=\bigcup_{ i=1}^\infty U_{ x_i}
        \end{equation*}
        luego,
        \begin{equation*}
            \begin{split}
                A=\bigcup_{i=1}^\infty U_{x_i}\cap A
            \end{split}
        \end{equation*}
        donde $U_{ x_i}\cap A$ es a lo sumo numerable, luego $A$ es a lo sumo numerable\contradiction. Por tanto, $A$ tiene al menos un punto de condensación.
    \end{proof}

    \begin{mydef}
        Sea $(X,\tau)$ un espacio topológico y $Y$ un conjunto. Si función $\cf{f}{X}{Y}$, entonces
        \begin{equation*}
            \tau_f=\left\{A\subseteq Y\Big|f^{-1}(A)\in\tau \right\}
        \end{equation*}  
        es una topología sobre $Y$ y es la topología más gruesa que se puede definir sobre $Y$ tal que la función $\cf{f}{(X,\tau)}{(Y,\tau_f)}$ es continua.
    \end{mydef}
    
    \begin{excer}
        Sean $(X_1,\tau_1)$ y $(X_1,\tau_2)$ espacios topológicos y $\cf{f}{(X_1,\tau_1)}{(X_2,\tau_2)}$ una función continua, suprayectiva y abierta. Entonces, $\tau_2=\tau_f$.
    \end{excer}

    \begin{proof}
        Veamos que se cumple la doble contención.
        \begin{itemize}
            \item Sea $U\in\tau_2$, entonces se tiene que como la función $f$ es abierta, $f(U)\in\tau_2$. En particular como $f$ es suprayectiva, 
        \end{itemize}
    \end{proof}

    \begin{excer}
        Sean $(X_1,\tau_1)$ y $(X_2,\tau_2)$ espacios topológicos y $Y$ un conjunto. Sea $\cf{f}{X_1}{Y}$ una función. Demuestre que una función $\cf{g}{(Y,\tau_f)}{(X_2,\tau_2)}$ es continua si y sólo si $\cf{g\circ f}{(X_1,\tau_1)}{(X_2,\tau_2)}$ es continua.
    \end{excer}

    \begin{proof}
        
    \end{proof}

    \begin{excer}
        Sean $(X,\tau)$ un espacio con una base a lo sumo numerable y $A\subseteq X$ subconjunto no numerable de $X$. Pruebe que una cantidad no numerable de puntos de $A$ son puntos límites de $A$.
    \end{excer}

    \begin{proof}
        
    \end{proof}

    \section{Conexidad}

    \begin{excer}
        Sean $\tau$ y $\tau'$ dos topologías definidas sobre $X$. Si $\tau\subseteq \tau'$, ¿qué se puede decir de la conexión de $X$ respecto de una topología y respecto de la otra?
    \end{excer}

    \begin{proof}
        Si $(X,\tau)$ es disconexo, entonces también lo será $(X,\tau')$, ya que como $(X,\tau)$ es disconexo existen dos conjuntos abiertos $A,B\in\tau$ tales que
        \begin{equation*}
            A\cap B=\emptyset\quad\textup{y}\quad A\cup B=X
        \end{equation*}
        En particular, $A,B\in\tau'$, luego $(X,\tau')$ es disconexo.

        Si $(X,\tau')$ es conexo, entonces $(X,\tau)$ es conexo también, ya que no existen conjuntos abiertos y cerrados en $\tau$ que sean abiertos y cerrados, pues éstos no existen en $\tau'$.
    \end{proof}

    \begin{excer}
        Sea $(X,\tau)$ un espacio topológico y $\left\{A_n\right\}_{ n=1}^\infty$ una sucesión de conjuntos conexos de $(X,\tau)$ tal que para todo $n\in\mathbb{N}$, $A_n\cap A_{ n+1}\neq\emptyset$. Demuestre que $\bigcup_{ n=1}^\infty A_n$ es conexo.
    \end{excer}

    \begin{proof}
        Para todo $m\in\mathbb{N}$ defina
        \begin{equation*}
            B_m=\bigcup_{ k=1}^m A_k
        \end{equation*}
        Afirmamos que $B_m$ es conexo, para todo $m\in\mathbb{N}$. En efecto, procederemos por inducción sobre $m$.
        \begin{itemize}
            \item Si $m=2$, veamos que
            \begin{equation*}
                B_2=A_1\cup A_2
            \end{equation*}
            donde $A_1,A_2$ son conexos y son tales que $A_1\cap A_2=A_1\cap A_{ 1+1}\neq\emptyset$. Por tanto, de un teorema se sigue que $B_2$ es conexo.
            \item Suponga que existe $m\in\mathbb{N}$ tal que $B_m$ es conexo. Veamos que $B_{ m+1}$ es conexo. En efecto, se tiene que:
            \begin{equation*}
                B_{ m+1}=\bigcup_{ k=1}^{ m+1}A_k=\left(\bigcup_{ k=1}^{ m}A_k\right)\cup A_{ m+1}=B_m\cup A_m
            \end{equation*}
            donde $B_m$ y $A_{ m+1}$ son conexos en $(X,\tau)$ tales que
            \begin{equation*}
                \emptyset\neq A_m\cap A_{ m+1}\subseteq B_m\cap A_{ m+1}
            \end{equation*}
            luego $B_m\cup A_{ m+1}$ es conexo, es decir que $B_{ m+1}$ es conexo.
        \end{itemize}
        Por inducción se sigue que $B_m$ es conexo para todo $m\in\mathbb{N}$. Ahora, si $m,n\in\mathbb{N}$ son tales que $n<m$, se tiene que
        \begin{equation*}
            B_m\cap B_n\neq\emptyset
        \end{equation*}
        ya que $m\leq n$ o $n\leq m$ (de donde se sigue que $B_n\subseteq B_m$ o $B_m\subseteq B_n$) y, $A_1\neq\emptyset$ (pues $A_1\cap A_2\neq\emptyset$). Por tanto, al ser cada uno conexo y tener intersección no vacía a pares, se sigue que
        \begin{equation*}
            \bigcup_{ m=1}^{\infty}B_m=\bigcup_{ m=1}^{\infty}\bigcup_{ n=1}^{m}A_n=\bigcup_{ n=1}^{\infty}A_n
        \end{equation*}
        es conexo en $(X,\tau)$.
    \end{proof}

    \begin{excer}
        Sean $\left\{A_\alpha \right\}_{\alpha\in I}$ una sucesión de espacios conexos en un espacio topológico $(X,\tau)$ y $A$ un subespacio conexo de $(X,\tau)$. Demuestre que si $A\cap A_\alpha\neq\emptyset$ para todo $\alpha\in I$, entonces $A\cup\left(\bigcup_{\alpha\in I}A_\alpha \right)$ es conexo en $(X,\tau)$. 
    \end{excer}

    \begin{proof}
        
    \end{proof}

    \begin{excer}
        Demuestre que si $X$ es un conjunto infinito, entonces $X$ es conexo con la topología de los complementos finitos (o topología cofinita).
    \end{excer}

    \begin{proof}
        Sea $X$ un conjunto infinito y considere el espacio topológico $(X,\tau_{cf})$. Recordemos que
        \begin{equation*}
            \tau_{cf}=\left\{A\subseteq X\Big|X-A\textup{ es finito} \right\}\cup\left\{\emptyset \right\}
        \end{equation*}
        Probaremos que si $A,B\subseteq X$ son abiertos no vacíos, entonces $A\cap B\neq\emptyset$. En efecto, si $A,B\in\tau_{cf}-\left\{\emptyset \right\}$ son tales que $A\cap B=\emptyset$ se tiene que
        \begin{equation*}
            X-A\cap B=X\Rightarrow X=(X-A)\cup(X-B)
        \end{equation*}
        donde $X-A$ y $X-B$ son finitos, luego $X$ es finito\contradiction. Por ende $A\cap B\neq\emptyset$. Pero lo anterior implica que no existe una partición de $X$ en dos conjuntos abiertos no vacíos, pues siempre sucede que $A\cap B\neq\emptyset$. Por tanto, $(X,\tau_{cf})$ debe ser conexo.
    \end{proof}

    \begin{mydef}
        Un espacio topológico $(X,\tau)$ es \textbf{totalmente disconexo} si sus únicos subespacios conexos son los conjuntos unipuntuales.
    \end{mydef}

    \begin{excer}
        Sea $(X,\tau_D)$ un espacio topológico dotado de la topología discreta. Pruebe que $(X,\tau_D)$ es totalmente disconexo. ¿Es cierto el recíproco?
    \end{excer}

    \begin{proof}
        
    \end{proof}

    \begin{excer}
        Sea $A\subseteq X$. Demuestre que si $C$ es un subespacio conexo de $X$ que interseca tanto a $A$ como a $X-A$, entonces $C$ interseca a $\Fr{A}$.
    \end{excer}

    \begin{proof}
        Se tiene que
        \begin{equation*}
            C\cap A\neq\emptyset\quad\textup{y}\quad C\cap(X-A)=\emptyset
        \end{equation*}
        Suponga que $C\cap\Fr{A}=\emptyset$. Como
        \begin{equation*}
            X=\Int{\overbrace{X-A}}\cup\Fr{A}\cup\Int{A}
        \end{equation*}
        donde los tres conjuntos de la derecha son disjuntos a pares. Debe suceder entonces que
        \begin{equation*}
            C\subseteq \Int{\overbrace{X-A}}\cup\Int{A}
        \end{equation*}
        Afirmamos que $C\cap\Int{A}\neq\emptyset$ y $C\cap\Int{\overbrace{X-A}}\neq\emptyset$. En efecto, si $C\cap\Int{A}=\emptyset$ se tendría que
        \begin{equation*}
            C\cap A\subseteq \Int{\overbrace{X-A}}\subseteq X-A
        \end{equation*}
        lo cual no puede suceder. De forma análoga, $C\cap\Int{\overbrace{X-A}}\neq\emptyset$. Tomemos $U=C\cap\Int{A}$ y $V=C\cap\Int{\overbrace{X-A}}$, se tiene que $U,V\in\tau_C$ son no vacíos para los cuales
        \begin{equation*}
            C=U\cup V=(C\cap\Int{\overbrace{X-A}})\cup(C\cap\Int{A})=C\cap\left(\Int{\overbrace{X-A}}\cup\Int{A}\right)
        \end{equation*}
        se tendría entonces que $C$ no es conexo\contradiction. Por tanto, $C\cap\Fr{A}\neq\emptyset$.
    \end{proof}

    \begin{excer}
        Sean $A$ un subconjunto propio de $X$ y $B$ un subconjunto propio de $Y$. Si $X$ e $Y$ son conexos, demuestre que
        \begin{equation*}
            (X\times Y)-(A\times B)
        \end{equation*}
        es conexo.
    \end{excer}

    \begin{proof}
        
    \end{proof}

    \begin{excer}
        Sea $Y\subseteq X$ y supongamos que $X$ e $Y$ son conexos. Demuestre que si $A$ y $B$ forman una separación de $X-Y$, entonces $Y\cup A$ e $Y\cup B$ son conexos.
    \end{excer}

    \begin{proof}
        
    \end{proof}

    \begin{excer}
        En $\mathbb{R}^n$ la conexidad se preserva bajo traslaciones.
    \end{excer}

    \begin{proof}
        Sea $C\subseteq\mathbb{R}^n$ conexo y $x\in\mathbb{R}^n$. Probaremos que el conjunto
        \begin{equation*}
            C+x=\left\{y+x\in\mathbb{R}^n \Big|y\in C \right\}
        \end{equation*}
        es conexo en $\mathbb{R}^n$. En efecto, considere la aplicación $\cf{f}{\mathbb{R}^n}{\mathbb{R}^n}$ tal que $y\mapsto y+x$. Es claro que esta función es continua y, se tiene que
        \begin{equation*}
            f\big|_{C}:C\rightarrow C+x
        \end{equation*}
        es una función continua suprayectiva, luego por un teorema se sigue que $C+x$ es conexo en $\mathbb{R}^n$.
    \end{proof}

    \begin{excer}
        Sea $n>1$, y $B\subseteq\mathbb{R}^n$ numerable. Entonces, $\mathbb{R}^n-B$ es conexo.
    \end{excer}

    \begin{proof}
        Antes, para cada par $x,y\in\mathbb{R}^n$ se define el segmento que une a $x$ y $y$ como el conjunto:
        \begin{equation*}
            L_{ xy}=\left\{ty+(1-t)x\Big|t\in[0,1] \right\}
        \end{equation*}
        Este conjunto es conexo pues es homeomorfo al subespacio conexo $([0,1],{\tau_u}_{[0,1]})$ y tiene más de dos puntos siempre que $x\neq y$.

        Ahora, suponga que $B\neq\emptyset$ (pues si $B=\emptyset$ el resultado es inmediato ya que $\mathbb{R}$ es conexo). Por el ejercicio anterior podemos suponer que $0\notin B$. Para cada $x\in\mathbb{R}^n-(B\cup\left\{0\right\})$ considere
        \begin{equation*}
            L_{0x}=\left\{tx\in\mathbb{R}^n \Big|t\in[0,1]\right\}
        \end{equation*} 
        
        y, dado $x\in\mathbb{R}^n-(B\cup\left\{0\right\})$ sea $l(x,a)$ una linea de longitud $a>0$ en $\mathbb{R}^n$ tal que $l(x,a)$ interseca a $L_{0x}$ es un punto distinto de $0$ y $x$ (lo cual es posible pues $L_x$ tiene más de dos puntos pues $x\neq0$). Afirmamos que existe $z\in l(x,a)$ tal que el conjunto
        \begin{equation*}
            L_{0z}\cup L_{zx}
        \end{equation*}
        está totalmente contenido en $\mathbb{R}^n-B$. En efecto, si para todo $z\in l(x,a)$ se tiene que $L_{0z}\cup L_{zx}$ no está totalmente contenido en $\mathbb{R}^n-B$, entonces existiría $b_z\in l_x$ tal que $b_z\in B$. No puede suceder que $b_z=0$ o $b_z=x$ pues $0,x\notin B$. Además, si $z,w\in l(x,a)$ son tales que $z\neq w$, entonces $b_z\neq b_w$, ya que los únicos puntos que tienen en común los conjuntos $L_{0z}\cup L_{zx}$ y $L_{0w}\cup L_{wx}$ son $0$ y $x$ (esto se deduce de forma inmediata mediante un sistema de ecuaciones lineales). Por tanto, la aplicación
        \begin{equation*}
            z\mapsto b_z
        \end{equation*}
        de $l(x,a)$ en $B$ es inyectiva. Como $a>0$ entonces se tendría que $B$ es no numerable (pues $l(x,a)$ es no numerable)\contradiction. Por tanto, existe $z\in l(x,a)$ tal que
        \begin{equation*}
            L_{ 0z}\cup L_{zx}
        \end{equation*}
        está totalmente contenido en $\mathbb{R}^n-B$. Este conjunto es conexo, pues $L_{ 0z}\cap L_{zx}=\left\{z\right\}$ y ambos segmentos son conexos. Defina así para cada $x\in\mathbb{R}^n-B$
        \begin{equation*}
            l_x=L_{ 0z}\cup L_{zx}
        \end{equation*}
        Entonces, se tiene que
        \begin{equation*}
            \bigcup_{x\in\mathbb{R}^n-B}l_x
        \end{equation*}
        es un conjunto conexo, ya que cada $l_x$ es conexo y, para todo par $x,y\in\mathbb{R}^n-B$, $l_x\cap l_y\neq\emptyset$, pues $0\in l_x\cap l_y$. Pero
        \begin{equation*}
            \bigcup_{x\in\mathbb{R}^n-B}l_x=\mathbb{R}^n-B
        \end{equation*}
        pues $l_x\subseteq\mathbb{R}^n-B$, para todo $x\in\mathbb{R}^n-B$. Así, $\mathbb{R}^n-B$ es conexo.
    \end{proof}

    \begin{excer}
        Sean $A,B$ subconjuntos de un espacio topológico $(X,\tau)$.
        \begin{enumerate}
            \item Demuestre que si $A$ y $B$ son cerrados y $A\cap B$ y $A\cup B$ son conexos, entonces $A$ y $B$ son conexos en. Proporcione un ejemplo donde la proposición no se cumple si uno de los dos conjuntos $A$ o $B$ no es cerrado.
            \item Suponga que $A$ y $B$ son conexos y que $\Cls{A}\cap B\neq\emptyset$, demuestre que $A\cup B$ es conexo.
        \end{enumerate}
    \end{excer}

    \begin{proof}
        
    \end{proof}

    \begin{excer}
        Sea $(X,\tau)$ un espacio topológico tal que dados $x,y\in X$ existe $C\subseteq X$ conexo tal que $x,y\in X$. Demuestre que $(X,\tau)$ es un espacio conexo.
    \end{excer}

    \begin{proof}
        
    \end{proof}

    \begin{excer}
        Sea $(X,\tau)$ un espacio localmente conexo.
        \begin{enumerate}
            \item Dado $x\in X$ sea $C_x$ la componente conexa de $x$. Demuestre que $C_x\in\tau$.
            \item Definimos una relación de equivalencia sobre $X$ por: $x\mathcal{R}y$ si y sólo si $C_x=C_y$. Demuestre que $\tau/\mathcal{R}$ es la topología discreta.
        \end{enumerate}
    \end{excer}

    \begin{proof}
        
    \end{proof}

    \begin{excer}
        Pruebe que $S^n$ es conexo para todo $n\geq1$.
    \end{excer}

    \begin{proof}
        Procederemos por inducción sobre $n$.
        \begin{itemize}
            \item $n=1$. Considere
            \begin{equation*}
                S^1=\left\{(x,y)\in\mathbb{R}^2\Big|x^2+y^2=1 \right\}
            \end{equation*}
            Sea $p=(1,0)$ elemento de $S^1$. Se sabe que para cada $(x,y)\in S^1-\left\{p\right\}$ existe un único $\theta_x\in]0,2\pi[$ tal que
            \begin{equation*}
                (x,y)=(\cos\theta_x,\sin\theta_x)
            \end{equation*}
            Defina
            \begin{equation*}
                S_x^1=\left\{(\cos t\theta_x,\sin t\theta_x)\in S^1\Big|t\in[0,1]\right\}
            \end{equation*}
            Es claro que este conjunto une a $p$ con $(x,y)$ y, está totalmente contenido en $S^1$ (por identidades trigonométricas). Además es conexo, pues existe una función continua y suprayectiva de $([0,1],{\tau_u}_{[0,1]})$ en $(S_x^1,{\tau_u}_{S^1})$ ($t\mapsto(\cos t\theta_x,\sin t\theta_x)$, sólo habría que probar que es continua, procediendo por sucesiones se tiene el resultado) donde el primer espacio es conexo. Luego, el conjunto
            \begin{equation*}
                \bigcup_{x\in S^1-\left\{p\right\}}S_x^1=S_1
            \end{equation*}
            es conexo en $(\mathbb{R}^2,\tau_u)$ pues $p\in S_x^1$, para todo $x\in S^1-\left\{p\right\}$. Por tanto, $S_1$ es conexo en $(\mathbb{R}^2,\tau_u)$.

            \item Suponga que existe $n\in\mathbb{N}$, $n\geq2$ tal que $S^{n-1}$ es conexo en $\mathbb{R}^n$. Probaremos que $S^n$ es conexo en $\mathbb{R}^{ n+1}$. En efecto, observemos que
            \begin{equation*}
                S^n=\left\{(x_1,...,x_{ n+1})\in\mathbb{R}^{n+1}\Big|\sum_{ k=1}^{n+1}x_k^2=1 \right\}
            \end{equation*}
        \end{itemize}
    \end{proof}

    \section{Espacio Cociente}

    \begin{excer}
        Sea $(X,\tau)$ un espacio topológico y sea $\mathcal{R}$ una relación ed equivalencia sobre $X$. Demuestre lo siguiente:
        \begin{enumerate}
            \item $(X/\mathcal{R},\tau/\mathcal{R})$ es $T_1$ si y sólo si $\forall x\in X$, $[x]$ es un subcojnunto cerrado de $(X,\tau)$.
            \item Si $(X/\mathcal{R},\tau/\mathcal{R})$ es un espacio $T_2$, entonces $\mathcal{R}$ es un subconjunto cerrado de $(X\times X,\tau_p)$.
            \item Sea $\cf{\phi}{X}{X/\mathcal{R}}$ la función definida por $\phi(x)=[x]$. Si $\mathcal{R}$ es un subconjunto cerrado de $(X\times X,\tau_p )$ y $\phi$ es una función abierta, entonces $(X/\mathcal{R},\tau/\mathcal{R})$ es $T_2$.
            \item Si $(X,\tau)$ es un espacio compacto, entonces $(X/\mathcal{R},\tau/\mathcal{R})$ es un espacio compacto.
        \end{enumerate}
    \end{excer}

    \begin{proof}
        
    \end{proof}

\end{document}