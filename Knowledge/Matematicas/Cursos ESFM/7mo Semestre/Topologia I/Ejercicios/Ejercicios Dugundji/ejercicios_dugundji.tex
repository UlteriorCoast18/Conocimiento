\documentclass[12pt]{report}
\usepackage[spanish]{babel}
\usepackage[utf8]{inputenc}
\usepackage{amsmath}
\usepackage{amssymb}
\usepackage{amsthm}
\usepackage{graphics}
\usepackage{subfigure}
\usepackage{lipsum}
\usepackage{array}
\usepackage{multicol}
\usepackage{enumerate}
\usepackage[framemethod=TikZ]{mdframed}
\usepackage[a4paper, margin = 1.5cm]{geometry}

%En esta parte se hacen redefiniciones de algunos comandos para que resulte agradable el verlos%

\renewcommand{\theenumii}{\roman{enumii}}

\def\proof{\paragraph{Demostración:\\}}
\def\endproof{\hfill$\blacksquare$}

\def\sol{\paragraph{Solución:\\}}
\def\endsol{\hfill$\square$}

%En esta parte se definen los comandos a usar dentro del documento para enlistar%

\newtheoremstyle{largebreak}
  {}% use the default space above
  {}% use the default space below
  {\normalfont}% body font
  {}% indent (0pt)
  {\bfseries}% header font
  {}% punctuation
  {\newline}% break after header
  {}% header spec

\theoremstyle{largebreak}

\newmdtheoremenv[
    leftmargin=0em,
    rightmargin=0em,
    innertopmargin=-2pt,
    innerbottommargin=8pt,
    hidealllines = true,
    roundcorner = 5pt,
    backgroundcolor = gray!60!red!30
]{exa}{Ejemplo}[section]

\newmdtheoremenv[
    leftmargin=0em,
    rightmargin=0em,
    innertopmargin=-2pt,
    innerbottommargin=8pt,
    hidealllines = true,
    roundcorner = 5pt,
    backgroundcolor = gray!50!blue!30
]{obs}{Observación}[section]

\newmdtheoremenv[
    leftmargin=0em,
    rightmargin=0em,
    innertopmargin=-2pt,
    innerbottommargin=8pt,
    rightline = false,
    leftline = false
]{theor}{Teorema}[section]

\newmdtheoremenv[
    leftmargin=0em,
    rightmargin=0em,
    innertopmargin=-2pt,
    innerbottommargin=8pt,
    rightline = false,
    leftline = false
]{propo}{Proposición}[section]

\newmdtheoremenv[
    leftmargin=0em,
    rightmargin=0em,
    innertopmargin=-2pt,
    innerbottommargin=8pt,
    rightline = false,
    leftline = false
]{cor}{Corolario}[section]

\newmdtheoremenv[
    leftmargin=0em,
    rightmargin=0em,
    innertopmargin=-2pt,
    innerbottommargin=8pt,
    rightline = false,
    leftline = false
]{lema}{Lema}[section]

\newmdtheoremenv[
    leftmargin=0em,
    rightmargin=0em,
    innertopmargin=-2pt,
    innerbottommargin=8pt,
    roundcorner=5pt,
    backgroundcolor = gray!30,
    hidealllines = true
]{mydef}{Definición}[section]

\newmdtheoremenv[
    leftmargin=0em,
    rightmargin=0em,
    innertopmargin=-2pt,
    innerbottommargin=8pt,
    roundcorner=5pt
]{excer}{Ejercicio}[section]

%En esta parte se colocan comandos que definen la forma en la que se van a escribir ciertas funciones%

\newcommand\abs[1]{\ensuremath{\biglvert#1\bigrvert}}
\newcommand\divides{\ensuremath{\bigm|}}
\newcommand\cf[3]{\ensuremath{#1:#2\rightarrow#3}}
\newcommand{\contradiction}{\ensuremath{\#_c}}

\newcommand{\Int}[1]{\ensuremath{\mathring{#1}}}
\newcommand{\Cls}[1]{\ensuremath{\overline{#1}}}
\newcommand{\Fr}[1]{\ensuremath{\textup{Fr}\left(#1\right)}}
\newcommand{\Ext}[1]{\ensuremath{\textup{Ext}\left(#1\right)}}

%recuerda usar \clearpage para hacer un salto de página

\begin{document}
    \title{Ejercicios Dugundji Topology y Problemas Varios}
    \author{Cristo Daniel Alvarado}
    \maketitle

    \tableofcontents %Con este comando se genera el índice general del libro%

    %\setcounter{chapter}{3} %En esta parte lo que se hace es cambiar la enumeración del capítulo%

    \chapter{Espacios Topológicos}

    \section{Conceptos Fundamentales}

    \begin{obs}
        El símbolo $\aleph(X)$, donde $X$ es un conjunto, denota al cardinal del conjunto (realmente denota a otra cosa que viene a ser lo mismo, pero para usos prácticos tomaremos lo anterior como cierto).
    \end{obs}

    \begin{excer}
        Pruebe lo siguiente:
        \begin{enumerate}
            \item Sea $X$ un conjunto infinto. Pruebe que $\mathcal{A}_0=\left\{A\subseteq X \big|X-A\textup{ es finito} \right\}\cup\left\{\emptyset \right\}$ es una topología sobre $X$.
            \item Sea $\aleph(X)\geq\aleph_0$. Pruebe que $\mathcal{A}_1=\left\{A\subseteq X \big|\aleph(X-A)<\aleph(X) \right\}\cup\left\{\emptyset \right\}$ es una topología sobre $X$.
        \end{enumerate}
    \end{excer}

    \begin{proof}
        De (1): Es la topología de los complementos finitos (la prueba de esto se hizo en las notas).

        De (2): Veamos que se verifican las tres condiciones:
        \begin{enumerate}
            \item Por definición de $\mathcal{A}_1$ se tiene que $\emptyset\in\mathcal{A}_1$ y, como $\aleph(\emptyset)<\aleph_0$, entonces $\aleph(X-X)<\aleph(X)$, por ende $X\in\mathcal{A}_1$.
            \item Sea $\mathcal{E}$ una subfamilia no vacía arbitraria de $\mathcal{A}_1$. Considere a $\bigcup\mathcal{E}$. Como la familia es no vacía, existe $E_0\in\mathcal{E}$, se tiene así que:
            \begin{equation*}
                \begin{split}
                    E_0\subseteq \bigcup\mathcal{E}\Rightarrow& X- \bigcup\mathcal{E}\subseteq X-E_0\\
                    \Rightarrow& \aleph\left(X- \bigcup\mathcal{E}\right) \subseteq \aleph(X-E_0) \\
                \end{split}
            \end{equation*}
            por Cantor-Bernstein. Por lo cual al tenerse que $\bigcup\mathcal{E}\subseteq X$, se sigue que $\bigcup\mathcal{E}\in\mathcal{A}_1$.
            \item Sean $A,B\in\mathcal{A}_1$, entonces $\aleph(X-A)<\aleph(X)$ y $\aleph(X-B)<\aleph(X)$. Notemos que
            \begin{equation*}
                X-(A\cap B)=(X-A)\cup (X-B)
            \end{equation*}
            Entonces $\aleph(X-(A\cap B))=\aleph((X-A)\cup (X-B))\leq\aleph(X-A)+\aleph(X-B)< \aleph(X)+\aleph(X)=2\aleph(X)=\aleph(X)$, pues $\aleph(X)\geq\aleph_0$. Por tanto, al ser $A\cap B\subseteq X$, se sigue que $A\cap B\in \mathcal{A}_1$.
        \end{enumerate}
        Por las tres condiciones anteriores, se sigue que $\mathcal{A}_1$ es una topología sobre $X$.
    \end{proof}

    \begin{excer}
        ¿Cuántas topologías distintas puede tener un conjunto de tres elemento? ¿Cuál es su orden parcial?
    \end{excer}

    \begin{sol}
        Considere $X=\left\{a,b,c\right\}$. De todas las topologías que puede tener, deben de estar al menos la topología discreta y la indiscreta, formada por los conjuntos:
        \begin{equation*}
            \begin{split}
                \tau_D&=\left\{\emptyset, \left\{a\right\},\left\{b\right\}, \left\{c\right\}, \left\{a,b\right\},\left\{b,c\right\},\left\{c,a\right\},\left\{a,b,c\right\} \right\}=\mathcal{P}(\left\{a,b,c\right\})\\
                \tau_I&=\left\{\emptyset,\left\{a,b,c\right\} \right\}
            \end{split}
        \end{equation*}
        Ahora, las otras que se pueden tener son aquellas que solo contienen a uno de los elementos, es decir las siguientes:
        \begin{equation*}
            \begin{split}
                \tau_a&=\left\{\emptyset,\left\{a\right\},\left\{a,b,c\right\}\right\}\\
                \tau_b&=\left\{\emptyset,\left\{b\right\},\left\{a,b,c\right\}\right\}\\
                \tau_c&=\left\{\emptyset,\left\{c\right\},\left\{a,b,c\right\}\right\}\\
            \end{split}
        \end{equation*}
        y, también aquellas que contengan a un par de elementos, pero de esta forma: $\left\{a,b\right\}$, que serían las siguientes:
        \begin{equation*}
            \begin{split}
                \tau_{a,b}&=\left\{\emptyset,\left\{a,b\right\},\left\{a,b,c\right\}\right\}\\
                \tau_{b,c}&=\left\{\emptyset,\left\{b,c\right\},\left\{a,b,c\right\}\right\}\\
                \tau_{c,a}&=\left\{\emptyset,\left\{c,a\right\},\left\{a,b,c\right\}\right\}\\
            \end{split}
        \end{equation*}
        (en esta se verifica casi de forma inmediata que es una topología sobre $X$). Ahora, se deben considerar aquellas en las que se tiene más de un elemento no trivial (cuando menciono la palabra trivial, me refiero a que no sea alguno de $\emptyset$ o $X=\left\{a,b,c\right\}$). Por ejemplo, consideremos a $\left\{a,b\right\}$ un elemento no trivial, y sea $\tau$ una topología sobre $X$ que contiene a este elemento. Se tienen seis casos:
        \begin{enumerate}
            \item ${a}\in\tau$, entonces al ser cerrado bajo uniones e intersecciones se tiene que (al menos) $\tau$ debe ser de la forma:
            \begin{equation*}
                \tau=\left\{\emptyset,\left\{a\right\},\left\{a,b\right\},\left\{a,b,c\right\}\right\}
            \end{equation*}
            \item $\left\{b\right\}\in\tau$, como con el caso anterior, se tendría que (al menos) $\tau$ debe ser de la forma:
            \begin{equation*}
                \tau=\left\{\emptyset,\left\{b\right\},\left\{a,b\right\},\left\{a,b,c\right\}\right\}
            \end{equation*}
            Ahora, si $\left\{a\right\}\in\tau$, entonces (al menos) $\tau$ debe ser de la forma:
            \begin{equation*}
                \tau=\left\{\emptyset,\left\{a\right\},\left\{b\right\},\left\{a,b\right\},\left\{a,b,c\right\}\right\}
            \end{equation*}
            \item $\left\{c\right\}\in\tau$, se tiene entonces que una topología sobre $X$ (al menos), debe ser:
            \begin{equation*}
                \tau=\left\{\emptyset,\left\{c\right\},\left\{a,b\right\},\left\{a,b,c\right\} \right\}
            \end{equation*}
            \item $\left\{b,c\right\}\in\tau$, se tiene entonces que $\tau$ debe ser de la forma (al menos):
            \begin{equation*}
                \tau=\left\{\emptyset,\left\{b\right\},\left\{b,c\right\},\left\{a,b\right\},\left\{a,b,c\right\} \right\}
            \end{equation*}
        \end{enumerate}
        Son un vergo, nmms.
    \end{sol}

    \begin{excer}
        Sean $\tau_X$ y $\tau_Y$ dos topologías en $X$ y $Y$, respectivamente. ¿Es
        \begin{equation*}
            \tau=\left\{A\times B\big| A\in\tau_X, B\in\tau_Y \right\}
        \end{equation*}
        una topología en $X\times Y$?
    \end{excer}

    \begin{sol}
        Veamos si se cumplen las tres condiciones para que $\tau$ sea una topología sobre $X$.
        \begin{enumerate}
            \item Es claro que $\emptyset,X\times Y \in\tau$, pues $\emptyset\in\tau_X,\tau_Y$ y, $X\in\tau_X$ y $Y\in\tau_Y$.
            \item Sea $\mathcal{C}$ una subfamilia no vacía de $\tau$. Entonces, cada elemento de $\mathcal{C}=\left\{C_\alpha\big|\alpha\in I \right\}$ es de la forma:
            \begin{equation*}
                C_\alpha=A_\alpha\times B_\alpha
            \end{equation*}
            donde $A_\alpha\in \tau_X$ y $B_\alpha\in\tau_Y$, para todo $\alpha\in I$. Luego:
            \begin{equation*}
                \begin{split}
                    \bigcup_{\alpha\in I}C_\alpha&=\bigcup_{\alpha\in I}A_\alpha\times B_\alpha \\
                \end{split}
            \end{equation*}
            Veamos que en general no es cierto que $\bigcup_{\alpha\in I}C_\alpha\in\tau$. En efecto, tomemos $X=Y=\mathbb{R}$ (con la topología usual) y como conjuntos de la familia a: $C_1=(0,1)\times(0,1)$, y $C_2=(1,2)\times(1,2)$. Se tiene que:
            \begin{equation*}
                C_1\cup C_2\notin\tau
            \end{equation*}
            ya que, en caso contrario se tendría que $C_1\cup C_2=A\times B$, con $A,B\subseteq\mathbb{R}$ abiertos con la topología usual. 
            
            Entonces, en particular los elementos $(\frac{1}{2},\frac{1}{2}),(\frac{3}{2},\frac{3}{2})\in C_1\cup C_2$, por lo cual los elementos $(\frac{1}{2},\frac{3}{2}),(\frac{3}{2},\frac{1}{2})\in C_1\cup C_2$\contradiction, por la forma en que se tomaron $C_1$ y $C_2$. Por lo cual, $C_1\cup C_2$ no puede expresarse como el producto cartesiano de dos abiertos.
            \item Sean $C,D\in\tau$, es decir que $C=A_1\times B_1$ y $D=A_2\times B_2$, donde $A_i\in\tau_X$ y $B_i\in\tau_Y$ para $i\in\left\{1,2\right\}$. Entonces:
            \begin{equation*}
                \begin{split}
                    C\cap B&= (A_1\times B_1)\cap(A_2\times B_2)\\
                    &= (A_1\cap A_2)\times (B_1\cap B_2) \\
                \end{split}
            \end{equation*}
            donde $A_1\cap A_2\in\tau_X$ y $B_1\cap B_2\in\tau_Y$, por ende $C\cap B\in\tau$.
        \end{enumerate}
        Por el inciso (2), se tiene que $\tau$ (al menos en un caso particular) no es una topología sobre $X\times Y$.
    \end{sol}

    Recordemos la definición de un preorden y orden parcial:

    \begin{mydef}
        Una relación binaria $R$ en un conjunto $A$ es llamada un \textbf{preorden} si es reflexiva y transitiva, esto es:
        \begin{enumerate}
            \item $\forall a\in A, aRa$.
            \item $(aRb)\lor(bRc)\Rightarrow aRc$.
        \end{enumerate}
        denotamos (en general) al preorden por $\prec$.
    \end{mydef}

    \begin{mydef}
        Sea $(A,\prec)$ un conjunto preordenado.
        \begin{enumerate}
            \item $m\in A$ es llamado \textbf{elemento maximal} en $A$ si para todo $a\in A$ tal que $m\prec a\Rightarrow a\prec m$.
            \item Un elemento $a_0\in A$ es llamado \textbf{cota superior de un subconjunto $B\subseteq A$} si para todo $b\in B$, $b\prec a_0$.
            \item Un subconjunto $B\subseteq A$ es llamado una \textbf{cadena} si cualesquiera dos elementos de $B$ están relacionados, es decir que $a,b\in B$ implica que $a\prec b$ o $b\prec a$.
        \end{enumerate}
    \end{mydef}
    
    \begin{mydef}
        Sea $A$ un conjunto preordenado. Un \textbf{orden parcial} es un preorden en $A$ junto con la propiedad adicional:
        \begin{equation*}
            (a\prec b)\land (b\prec a)\Rightarrow (a=b)
        \end{equation*}
        esta propiedad es llamada antisimetría. Un conjunto $A$ adjutandole además un orden parcial es llamado un \textbf{conjunto parcialmente ordenado}. Un conjunto parcialente ordenado que es también una cadena es llamado un \textbf{conjunto totalmente ordenado}.
    \end{mydef}

    \begin{excer}
        Sea $X$ un conjunto parcialmente ordenado. Defina $U\subseteq X$ abierto si y sólo si satisface la condición: $(x\in U)\land (y\prec x)\Rightarrow y\in U$. Pruebe que la familia
        \begin{equation*}
            \mathcal{A}=\left\{U\subseteq X\big| U\textup{ es abierto} \right\}
        \end{equation*}
        es una topología sobre $X$.
    \end{excer}

    \begin{proof}
        Se deben verificar que se cumplen las tres condiciones.
        \begin{enumerate}
            \item $\emptyset\in\mathcal{A}$, pues por vacuidad se cumple que $\emptyset$ satisface la condición. Ahora, sea $x\in X$ y $y\prec x$, entonces $y\in X$ (pues es dónde se define el preorden). Por tanto, $X\in\mathcal{A}$.
            \item Sea $\mathcal{B}$ una familia no vacía de subconjuntos de $\mathcal{A}$. Si $x\in\bigcup\mathcal{B}$, entonces existe $B_0\in\mathcal{B}$ tal que $x\in B_0$.
            
            Ahora, si $y\in X$ es tal que $y\prec x$, como $x\in B_0$, por ser $B_0$ abierto se tiene que $y\in B_0\subseteq\bigcup\mathcal{B}$. Por lo cual $\bigcup\mathcal{B}$ es abierto.

            \item Sean $U,V\in\mathcal{A}$, si $U\cap V=\emptyset$ es claro que $U\cap V\in\mathcal{A}$. Suponga que la intersección es no vacía y sean $x\in U\cap V$ y $y\in X$ tal que $y\prec x$. En particular $(x\in U)\land(y\prec x)$ y $(x\in V)\land(y\prec x)$, por ende $y\in U\cap V$, es decir que $U\cap V\in\mathcal{A}$.
        \end{enumerate}
        Por los incisos anteriores, se tiene que $\mathcal{A}$ es una topología sobre $X$.
    \end{proof}

    \begin{excer}
        En $\mathbb{Z}^+$ defina $U\subseteq\mathbb{Z}^+$ que sea abierto si satisface la condición $n\in U\Rightarrow$ cada divisor de $n$ pertenece a $U$. Pruebe que esta es una topología en $\mathbb{Z}^+$ y que no es la topología discreta.
    \end{excer}

    \begin{proof}
        Llamemos $\tau$ a la familia de todos los conjuntos abiertos en $\mathbb{Z}^+$. Veamos que para $\tau$ se cumplen las tres condiciones:
        \begin{enumerate}
            \item $\emptyset\in\tau$, esto es cierto por vacuidad. Ahora si $n\in\mathbb{Z}^+$, entonces todos sus divisores están en $\mathbb{Z}^+$ (divisores positivos), por lo cual $\mathbb{Z}^+\in\tau$.
            \item Sea $\mathcal{A}$ una familia no vacía de elementos de $\tau$, y sea $n\in\bigcap\mathcal{A}$, entonces existe $A_0$ tal que $n\in A_0$, pero $A_0$ es abierto, por lo cual contiene a todos los divisores de $n$. Como $A\subseteq\bigcup\mathcal{A}$ entonces $\bigcup\mathcal{A}$ contiene a todos los divisores de $n$, luego $\bigcup\mathcal{A}\in\tau$.
            \item Sean $A,B\in\tau$ tales que $A\cap B\neq\emptyset$. Si $n\in A\cap B$ entonces $n\in A$ y $n\in B$, como $A$ y $B$ son abiertos, entonces estos dos conjuntos cumplen que cada divisor de $n$ pertenece a $A$ y $B$, en particular cada divisor de $n$ pertenece a $A\cap B$. Por tanto, $A\cap B\in\tau$. 
        \end{enumerate}
        Por los tres incisos anteriores, se sigue que $\tau$ es una topología sobre $\mathbb{Z}^+$.
    \end{proof}

    \begin{excer}
        Pruebe lo siguiente: $\tau$ es la topología discreta en $X$ si y sólo si todo punto de $X$ es un conjunto abierto (hablando de los conjuntos unipuntuales). 
    \end{excer}

    \begin{proof}
        Se probará la doble implicación:
        $\Rightarrow)$: Suponga que $\tau$ es la topología discreta, entonces $\tau=\mathcal{P}(X)$, en particular $\left\{x\right\}\in \mathcal{P}(X)$, para cada $x\in X$, esto es $\left\{x\right\}\in\tau$.

        $\Leftarrow)$: Suponga que todo conjunto unipuntual de $X$ está en $\tau$, y sea $A\in\mathcal{P}(X)$, entonces:
        \begin{equation*}
            A=\bigcup_{a\in A}\left\{a\right\}
        \end{equation*}
        donde $\left\{a\right\}$ es abierto y, por ende $A$ es abierto al ser una unión arbitraria de abiertos. Por tanto, $A\in\tau$, Por ende $\mathcal{P}(X)\subseteq\tau$, pero siempre se tiene que $\tau\subseteq\mathcal{P}(X)$, luego $\tau=\mathcal{P}(X)=\tau_D$.
    \end{proof}

    \setcounter{section}{2}

    \section{Creación de topologías dados conjuntos}

    \begin{excer}
        
    \end{excer}


    \appendix

    \chapter{Exámenes Topología I Salvador Quintín Flores García}

    \section{Primer Examen}

    \begin{excer}
        Sean $X=\mathbb{R}$ y $\tau=\left\{X,\emptyset \right\}\cup\left\{B_q\right\}_{q\in\mathbb{Q}}$, donde $B_q=(q,\infty)\cap\mathbb{Q}$. ¿Es $(\mathbb{R},\tau)$ un espacio topológico? Demuestre su respuesta.
    \end{excer}

    \begin{sol}
        
    \end{sol}

    \begin{excer}
        ¿La familia $\left\{[a,b[\big| a,b\in\mathbb{Q}, a<b \right\}$ es base en $(X,\tau_S)$? Justifique su respuesta.
    \end{excer}

    \begin{sol}
        
    \end{sol}

    \begin{excer}
        Sea $(X,\tau)$ un espacio topológico y $R$ una relación de equivalencia sobre $X$, y $\cf{p}{X}{X/R}$ la función que a cada elemento $x\mapsto [x]$ lo asigna a su clase de equivalencia. Haga lo siguiente:
        \begin{enumerate}
            \item Demuestre que la colección de todos los conjuntos cerrados en $(X/R,\tau/R)$ es:
            \begin{equation*}
                \left\{F\subseteq X/R\big| p^{-1}(F) \textup{ es cerrado en }X \right\}
            \end{equation*}
            \item Demuestre que la colección de todos los conjuntos cerrados en $(X/R, \tau/R)$ es igual a la familia:
            \begin{equation*}
                \left\{p(F)\subseteq X/R\big| F\textup{ es cerrado en }(X,\tau)\textup{ y }p^{-1}(p(F))=F \right\}
            \end{equation*}
        \end{enumerate}
    \end{excer}

    \begin{sol}
        
    \end{sol}

    \begin{excer}
        En el espacio $(X,\tau_{cf})$ y tomando $A=(0,1)$, obtener:
        \begin{enumerate}
            \item $\Int{A}$.
            \item $\Cls{A}$.
            \item $\Fr{A}$.
            \item $\Ext{A}=\Int{\widehat{X-A}}$.
        \end{enumerate}
    \end{excer}

    \begin{sol}
        
    \end{sol}

    \begin{excer}
        Sea $(X,\tau)$ un espacio topológico, para cada $A\subseteq X$ definimos $\alpha(A)=\Int{\Cls{A}}$, y $\beta(A)=\Cls{\Int{A}}$. Demuestre o refute:
        \begin{enumerate}
            \item $\alpha(\alpha(A))=\alpha(A)$, para cada $A\subseteq X$.
            \item $\beta(\beta(A))=\Cls{A}$, para cada $A\subseteq X$.
        \end{enumerate}
    \end{excer}

\end{document}