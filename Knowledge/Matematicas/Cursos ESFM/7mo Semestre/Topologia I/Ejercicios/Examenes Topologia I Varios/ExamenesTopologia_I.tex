\documentclass[12pt]{report}
\usepackage[spanish]{babel}
\usepackage[utf8]{inputenc}
\usepackage{amsmath}
\usepackage{amssymb}
\usepackage{amsthm}
\usepackage{graphics}
\usepackage{subfigure}
\usepackage{lipsum}
\usepackage{array}
\usepackage{multicol}
\usepackage{enumerate}
\usepackage[framemethod=TikZ]{mdframed}
\usepackage[a4paper, margin = 1.5cm]{geometry}

%En esta parte se hacen redefiniciones de algunos comandos para que resulte agradable el verlos%

\renewcommand{\theenumii}{\roman{enumii}}

\def\proof{\paragraph{Demostración:\\}}
\def\endproof{\hfill$\blacksquare$}

\def\sol{\paragraph{Solución:\\}}
\def\endsol{\hfill$\square$}

%En esta parte se definen los comandos a usar dentro del documento para enlistar%

\newtheoremstyle{largebreak}
  {}% use the default space above
  {}% use the default space below
  {\normalfont}% body font
  {}% indent (0pt)
  {\bfseries}% header font
  {}% punctuation
  {\newline}% break after header
  {}% header spec

\theoremstyle{largebreak}

\newmdtheoremenv[
    leftmargin=0em,
    rightmargin=0em,
    innertopmargin=0pt,
    innerbottommargin=5pt,
    hidealllines = true,
    roundcorner = 5pt,
    backgroundcolor = gray!60!red!30
]{exa}{Ejemplo}[section]

\newmdtheoremenv[
    leftmargin=0em,
    rightmargin=0em,
    innertopmargin=0pt,
    innerbottommargin=5pt,
    hidealllines = true,
    roundcorner = 5pt,
    backgroundcolor = gray!50!blue!30
]{obs}{Observación}[section]

\newmdtheoremenv[
    leftmargin=0em,
    rightmargin=0em,
    innertopmargin=0pt,
    innerbottommargin=5pt,
    rightline = false,
    leftline = false
]{theor}{Teorema}[section]

\newmdtheoremenv[
    leftmargin=0em,
    rightmargin=0em,
    innertopmargin=0pt,
    innerbottommargin=5pt,
    rightline = false,
    leftline = false
]{propo}{Proposición}[section]

\newmdtheoremenv[
    leftmargin=0em,
    rightmargin=0em,
    innertopmargin=0pt,
    innerbottommargin=5pt,
    rightline = false,
    leftline = false
]{cor}{Corolario}[section]

\newmdtheoremenv[
    leftmargin=0em,
    rightmargin=0em,
    innertopmargin=0pt,
    innerbottommargin=5pt,
    rightline = false,
    leftline = false
]{lema}{Lema}[section]

\newmdtheoremenv[
    leftmargin=0em,
    rightmargin=0em,
    innertopmargin=0pt,
    innerbottommargin=5pt,
    roundcorner=5pt,
    backgroundcolor = gray!30,
    hidealllines = true
]{mydef}{Definición}[section]

\newmdtheoremenv[
    leftmargin=0em,
    rightmargin=0em,
    innertopmargin=0pt,
    innerbottommargin=5pt,
    roundcorner=5pt
]{excer}{Ejercicio}[section]

%En esta parte se colocan comandos que definen la forma en la que se van a escribir ciertas funciones%

\newcommand\abs[1]{\ensuremath{\big|#1\big|}}
\newcommand\divides{\ensuremath{\bigm|}}
\newcommand\cf[3]{\ensuremath{#1:#2\rightarrow#3}}
\newcommand\contradiction{\ensuremath{\#_c}}
\newcommand\natint[1]{\ensuremath{\left[\big|#1\big|\right]}}

\newcommand\gen[1]{\ensuremath{\langle#1\rangle}}

\newcommand{\Int}[1]{\ensuremath{\mathring{#1}}}
\newcommand{\Cls}[1]{\ensuremath{\overline{#1}}}
\newcommand{\Fr}[1]{\ensuremath{\textup{Fr}\left(#1\right)}}
\newcommand{\Ext}[1]{\ensuremath{\textup{Ext}\left(#1\right)}}

\begin{document}
    \setlength{\parskip}{5pt} % Añade 5 puntos de espacio entre párrafos
    \setlength{\parindent}{12pt} % Pone la sangría como me gusta
    \title{Exámenes Parciales y ETS Topología I}
    \author{Cristo Daniel Alvarado}
    \maketitle

    \tableofcontents %Con este comando se genera el índice general del libro%

    %\setcounter{chapter}{3} %En esta parte lo que se hace es cambiar la enumeración del capítulo%

    \chapter{Primer Examamen Parcial}

    \section{Ejercicios I}

    \begin{excer}
        Sean $X=\mathbb{R}$ y $\tau=\left\{X,\emptyset \right\}\cup\left\{B_q\right\}_{q\in\mathbb{Q}}$, donde $B_q=(q,\infty)\cap\mathbb{Q}$. ¿Es $(\mathbb{R},\tau)$ un espacio topológico? Demuestre su respuesta.
    \end{excer}

    \begin{sol}
        
    \end{sol}

    \begin{excer}
        ¿La familia $\left\{[a,b[\big| a,b\in\mathbb{Q}, a<b \right\}$ es base en $(X,\tau_S)$? Justifique su respuesta.
    \end{excer}

    \begin{sol}
        
    \end{sol}

    \begin{excer}
        Sea $(X,\tau)$ un espacio topológico y $R$ una relación de equivalencia sobre $X$, y $\cf{p}{X}{X/R}$ la función que a cada elemento $x\mapsto [x]$ lo asigna a su clase de equivalencia. Haga lo siguiente:
        \begin{enumerate}
            \item Demuestre que la colección de todos los conjuntos cerrados en $(X/R,\tau/R)$ es:
            \begin{equation*}
                \left\{F\subseteq X/R\big| p^{-1}(F) \textup{ es cerrado en }X \right\}
            \end{equation*}
            \item Demuestre que la colección de todos los conjuntos cerrados en $(X/R, \tau/R)$ es igual a la familia:
            \begin{equation*}
                \left\{p(F)\subseteq X/R\big| F\textup{ es cerrado en }(X,\tau)\textup{ y }p^{-1}(p(F))=F \right\}
            \end{equation*}
        \end{enumerate}
    \end{excer}

    \begin{sol}
        
    \end{sol}

    \begin{excer}
        En el espacio $(X,\tau_{cf})$ y tomando $A=(0,1)$, obtener:
        \begin{enumerate}
            \item $\Int{A}$.
            \item $\Cls{A}$.
            \item $\Fr{A}$.
            \item $\Ext{A}=\Int{\widehat{X-A}}$.
        \end{enumerate}
    \end{excer}

    \begin{sol}
        
    \end{sol}

    \begin{excer}
        Sea $(X,\tau)$ un espacio topológico, para cada $A\subseteq X$ definimos $\alpha(A)=\Int{\Cls{A}}$, y $\beta(A)=\Cls{\Int{A}}$. Demuestre o refute:
        \begin{enumerate}
            \item $\alpha(\alpha(A))=\alpha(A)$, para cada $A\subseteq X$.
            \item $\beta(\beta(A))=\beta(A)$, para cada $A\subseteq X$.
        \end{enumerate}
    \end{excer}

    \begin{proof}
        
    \end{proof}

    \section{Ejercicios II}

    \begin{excer}
        Sea $\mathcal{S}=\left\{\mathbb{N},\left\{1,2 \right\},\left\{3,4,5 \right\},\left\{1,4,7 \right\} \right\}$, Encuentre explícitamente los elementos de la topología $\tau$ menos fina definida sobre $\mathbb{N}$ tal que $\mathcal{S}\subseteq\tau$ y, además, encuentre una base para $\tau$ tal que $\mathcal{B}\neq\tau$.
    \end{excer}

    \begin{sol}
        
    \end{sol}

    \begin{excer}
        Sean $(X,\tau)$ un espacio topológico, $Y$ un conjunto y $\cf{f}{X}{Y}$ una función. Demuestre que el conjunto:
        \begin{equation*}
            \tau_f:=\left\{U\subseteq Y\Big|f^{-1}(U)\in\tau \right\}
        \end{equation*}
        es una topología sobre $Y$ y es la topología más fina definida sobre $Y$ tal que la función $\cf{f}{(X,\tau)}{(Y,\tau_f)}$ es una función continua.
    \end{excer}

    \begin{proof}
        
    \end{proof}

    \begin{excer}
        Sea $(\mathbb{N},\tau_{cf})$, donde $\tau_{cf}$ es la topología de los complementos finitos. Considere el conjunto
        \begin{equation*}
            P=\left\{n\in\mathbb{N}\Big|\exists k\in\mathbb{N} \textup{ tal que }n=2k \right\}
        \end{equation*}
        encuentre $\Cls{P}, \Int{P},P'$.
    \end{excer}

    \begin{sol}
        
    \end{sol}

    \begin{excer}
        Sean $(X,\tau)$ y $(Y,\tau')$ espacios topológicos tales que $(Y,\tau')$ es un espacio topológico Hausdorff. Sean $A\subseteq X$ y $\cf{f}{(A,\tau_A)}{(Y,\tau')}$ una función continua tal que existen $\cf{g_i}{(\Cls{A},\tau_{\Cls{A}})}{(Y,\tau')}$, $i\in\left\{1,2 \right\}$ funciones continuas que cumplen:
        \begin{equation*}
            g_1(a)=g_2(a),\quad\forall a\in A
        \end{equation*}
        demuestre que $g_1=g_2$.
    \end{excer}

    \begin{proof}
        Sea $a\in\Cls{A}$. Debemos probar que $g_1(a)=g_2(a)$. Se tienen dos casos:
        \begin{enumerate}
            \item $a\in A$, en cuyo caso se sigue de hipótesis que $g_1(a)=g_2(a)$.
            \item Suponga que $a\notin A$. Si $g_1(a)\neq g_2(a)$, como el espacio $(Y,\tau')$ es Hausdorff, existen dos abiertos $M,N\in\tau'$ tales que
            \begin{equation*}
                g_1(a)\in M\quad g_2(a)\in N\quad M\cap N=\emptyset
            \end{equation*}
            por ende, los conjuntos $V=g_1^{-1}(M)$ y $U=g_2^{-1}(N)$ son dos abiertos en $(\Cls{A},\tau_{\Cls{A}})$ tales que
            \begin{equation*}
                a\in V\cap U
            \end{equation*}
            como $a\in\Cls{A}$, existe pues $a'\in A$ tal que $a'\in U\cap V$, es decir:
            \begin{equation*}
                a'\in g_1^{-1}(M)\cap g_2^{-1}(N)\Rightarrow g_1(a')\in M\quad\textup{y}\quad g_2(a')\in N\Rightarrow g_1(a')=g_2(a')\in M\cap N
            \end{equation*}
            pues, $a'\in A$. Por tanto, $M\cap N\neq\emptyset$\contradiction. Así, $g_1(a)=g_2(a)$.
        \end{enumerate}
        Por ambos incisos, se sigue que $g_1=g_2$.
    \end{proof}

    \chapter{Segundo Examen Parcial}

    \section{Ejercicios I}

    \begin{excer}
        Sean $S=\left\{(x,y)\in\mathbb{R}^2\Big|1\leq x^2+y^2\leq 4 \right\}$ y $P=\left\{(x,y,z)\in\mathbb{R}^3\Big|x^2+y^2=1\textup{ y }0\leq z\leq 1 \right\}$. Los espacios $(S,\tau_{u_S})$ y $(P,\tau_{u_P})$ son homeomorfos? Demuestre que su respuesta es correcta.
    \end{excer}

    \begin{excer}
        Sea $\beta$ un número irracional arbitrario pero fijo. Demuestre que el conjunto 
        \begin{equation*}
            G=\left\{a\beta+b\Big|a\in\mathbb{N},b\in\mathbb{Z} \right\}
        \end{equation*}
        es denso en $(\mathbb{R},\tau_u)$.
    \end{excer}

    \begin{excer}
        Considere los espacios topológicos $(\mathbb{R},\tau_{cf})$ y $(\mathbb{R},\tau_{cf}')$ donde $\tau_{cf}$ y $\tau_{cf}'$ son la topología de los complementos finitos en $\mathbb{R}$ y $\mathbb{R}^2$, respectivamente. ¿Es cierto que $\tau_{cf}\times\tau_{cf}=\tau_{cf}'$? Demuestre su respuesta.
    \end{excer}

    \begin{excer}
        Sea $n\mapsto r_n$ una biyección de $\mathbb{N}$ sobre $\mathbb{Q}\cap [0,1]$. Definimos la función $\cf{f}{([0,1],\tau_{u_{[0,1]}})}{(\mathbb{R},\tau_u)}$ como sigue; para cada $x\in [0,1]$ tomamos 
        \begin{equation*}
            f(x)=\sum_{n\in\mathcal{N}(x) }\frac{1}{2^n}
        \end{equation*}
        donde $\mathcal{N}(x)=\left\{n\in\mathbb{R}^n\Big|x<r_n \right\}$. Demuestre que la reestricción de $f$ al conjunto $B$ de todos los números irracionales $x$ en $[0,1]$ es continua.
    \end{excer}

    \section{Ejercicios II}

    \chapter{Tercer Examen Parcial}

    \section{Ejercicios}
    
    \begin{excer}
        
    \end{excer}

    \chapter{ETS Ordinario}

    \section{Ejercicios}

    \begin{excer}
        Sea $A=\left\{\sin n\Big|n\in\mathbb{N} \right\}$. Pruebe que $\Cls{A}=[-1,1]$ en $(\mathbb{R},\tau_u)$.
    \end{excer}

    \begin{proof}
        Notemos que $A=\sin(\mathbb{N})$. Sea $C=\sin(\mathbb{Z})$.

        Es claro que $C\subseteq[-1,1]$ donde $[-1,1]$ es un cerrado en $(\mathbb{R},\tau_u)$, por tanto $\Cls{C}\subseteq[-1,1]$, veremos que se cumple la otra contención. Sea $x\in[-1,1]$,
        \begin{itemize}
            \item Si $x\in C$,  es claro que $x\in\Cls{C}$ ya que $C\subseteq\Cls{C}$.
            \item Si $x\notin C$, como la función $t\mapsto\sen t$ de $\mathbb{R}$ a $[-1,1]$ es suprayectiva, entonces existe $\theta\in\mathbb{R}$ tal que $\sin\theta=x$.
            
            Ahora, por la proposición 4.2.2, el conjunto
            \begin{equation*}
                B=\left\{a+2\pi b\Big|a,b\in\mathbb{Z} \right\}
            \end{equation*}
            es denso en $\mathbb{R}$ por ser $2\pi$ irracional. Entonces, para cada $n\in\mathbb{N}$ existe $\theta_n=a_n+2\pi b_n\in B$ tal que $\abs{\theta-\theta_n}<\frac{1}{n}$, es decir que la sucesión $\left\{\theta_n \right\}_{n=1}^\infty$ converge a $\theta$. Como $t\mapsto\sin t$ es continua, entonces:
            \begin{equation*}
                \begin{split}
                    \lim_{n\rightarrow\infty }\abs{\sin\theta-\sin\theta_n}=&0\\
                    \Rightarrow \lim_{n\rightarrow\infty }\abs{x-\sin\left(a_n+2\pi b_n \right)}=&0\\
                \end{split}
            \end{equation*}
            pero,
            \begin{equation*}
                \begin{split}
                    \sin\left(a_n+2\pi b_n \right)=&\sin\left(a_n\right)\cos\left(2\pi b_n\right)+\cos\left(a_n\right)\sin\left(2\pi b_n \right)\\
                    =&\sin\left(a_n\right)\\
                \end{split}
            \end{equation*}
            pues $\cos(2\pi k)=1$ y $\sin(2\pi k)=0$, para todo $k\in\mathbb{Z}$. Luego,
            \begin{equation*}
                \begin{split}
                    \lim_{n\rightarrow\infty }\abs{x-\sin a_n}=&0\\
                \end{split}
            \end{equation*}
            es decir que para $\varepsilon>0$ existe $n\in\mathbb{N}$ tal que $\abs{x-\sin a_n}<\varepsilon$, donde $a_n\in\mathbb{Z}$. 
        \end{itemize}
        Por los dos incisos anterioes, se sigue que lo que $\Cls{C}\subseteq[-1,1]\Rightarrow \Cls{C}=[-1,1]$, es decir que $\sin(\mathbb{Z})$ es denso en $[-1,1]$, pero $t\mapsto\sin t$ es continua y periódica entre $[-1,1]$, por tanto de la proposición 4.2.3 se sigue que $A=\sin(\mathbb{N})$ es denso en $[-1,1]$.
    \end{proof}

    \begin{excer}
        Para cada par de enteros positivos primos relativos $a,b\in\mathbb{N}$ definimos:
        \begin{equation*}
            N_{a.b}=\left\{a+kb\Big|k\in\mathbb{N}^* \right\}
        \end{equation*}
        \begin{enumerate}
            \item Demuestre que la familia $B=\left\{N_{ a,b}\Big|a,b\in\mathbb{N}\textup{ tales que }(a,b)=1 \right\}$ es base de una topología sobre $\mathbb{N}$.
            \item $(\mathbb{N},\tau(B))$ es Hausdorff (donde $\tau(B)$ representa a la topología generada por la base $B$).
            \item Cualquier múltiplo de $b$ pertenece a $\Cls{N_{ a,b}}$.
        \end{enumerate}
    \end{excer}

    \begin{excer}
        Demuestre que una bola en $(\mathbb{R}^2,\tau_u|^{\mathbb{R}^2})$ no es igual al producto cartesiano de dos subconjuntos de $(\mathbb{R},\tau_u|^{\mathbb{R}})$.
    \end{excer}

    \begin{excer}
        Sea $(X,\tau)$ un espacio Hausdorff compacto. Si $(X,\tau)$ no tiene puntos aislados, demuestre que $X$ no es a lo sumo numerable.
    \end{excer}

    \begin{excer}
        Sean
        \begin{equation*}
            \begin{split}
                X&=\left\{(x,z)\in\mathbb{R}^2\Big|x=0,0\leq z\leq 1 \right\}\\
                Y&=\left\{(x,z)\in\mathbb{R}^2\Big|x^2+\left(z-\frac{1}{4} \right)^2=\frac{1}{16},x\geq 0 \right\}\\
                Z&=\left\{(x,z)\in\mathbb{R}^2\Big| x^2+\left(z-\frac{3}{4} \right)^2=\frac{1}{16},x\geq0 \right\}\\
                B&=X\cup Y\cup Z\\
                O&=\left\{(x,z)\in\mathbb{R}^2\Big|\left(x-\frac{1}{2} \right)^2+\left(z-\frac{1}{2} \right)^2=\frac{1}{4} \right\}\\
            \end{split}
        \end{equation*}
        ¿Son los espacios $(B,\tau_{ u_{B}}|^{\mathbb{R}^2})$ y $(O,\tau_{ u_{O}}|^{\mathbb{R}^2})$ homeomorfos? En caso de que su repuesta sea sí, construya explícitamente una función biyectiva y continua $\cf{f}{B}{O}$ y también exhiba su inversa mostrando también que es continua.

        En caso de que su respuesta sea negativa, justifique detalladamente porque no son homeomorfos.
    \end{excer}
    
    \section{Resultados Preeliminares}
    
    \begin{propo}
        Considere al grupo aditivo $(\mathbb{R},+)$. Entonces todo subgrupo $H$ de éste es denso en la topología $(\mathbb{R},\tau_u)$ ó es cíclico.
    \end{propo}

    \begin{proof}
        Se tienen que probar dos cosas:
        \begin{enumerate}
            \item Suponga que $G$ es denso. Se probará que $G$ no puede ser cíclico. En efecto, si $G$ fuera cíclico, existiría $g\in G$ tal que
            \begin{equation*}
                G=\gen{g}
            \end{equation*}
            es claro que $g\neq 0$, pues en caso contrario se tendría que $G=\left\{0\right\}$, que no puede suceder ya que $G$ es denso en $\mathbb{R}$, así $g>0$; además, existe $h\in G$ tal que $0<h<g$ ya que el conjunto $]0,g[$ es abierto en $\mathbb{R}$.

            Como $G=\gen{g}$ existe entonces $n\in\mathbb{N}$ tal que $g=hn$ (por ser $h,g>0$), es decir que $g\leq h$\contradiction, pues $h<g$. Por tanto, $G$ no es cíclico.

            \item Suponga que $G$ no es denso. Probaremos que $G$ es cíclico, sea
            \begin{equation*}
                g=\inf\left\{x\in G\Big|x>0 \right\}
            \end{equation*}
            Se tienen dos casos. Afirmamos que $g>0$. En efecto, suponga que $g=0$, sea $U\subseteq\mathbb{R}$ abierto no vacío y, $x\in\mathbb{R}$ y $\varepsilon>0$ tales que $]x-\varepsilon,x+\varepsilon[\subseteq U$. Como $g=0$, existe $g_\varepsilon\in G$ tal que $0<g_\varepsilon<\varepsilon$, sea ahora $k\in\mathbb{Z}$ tal que:
            \begin{equation*}
                kg_\varepsilon\leq x<(k+1)g_\varepsilon
            \end{equation*}
            es claro que $kg_\varepsilon\in G$, y además:
            \begin{equation*}
                \begin{split}
                    0\leq&x-kg_\varepsilon\\
                    <&(k+1)g_\varepsilon-kg_\varepsilon\\
                    =&g_\varepsilon\\
                    <&\varepsilon\\
                \end{split}
            \end{equation*}
            es decir, $\abs{x-kg_\varepsilon}<\varepsilon$ y por ende $kg_\varepsilon\in U$. Por tanto, $G$ es denso en $\mathbb{R}$\contradiction. Por tanto, $g>0$. Veamos ahora que $g\in G$.
            
            Suponga que $g\notin G$, entonces existen $h_1,h_2\in G$ positivos tales que:
            \begin{equation*}
                g<h_1<h_2<2g
            \end{equation*}
            (por propiedades del ínfimo), luego $h_2-h_1\in G$ y son tales que $0<h_2-h_1<g$\contradiction, pues $g$ es el ínfimo. Luego, $g\in G$.

            Sea $x\in G$, entonces existe $k\in\mathbb{Z}$ tal que
            \begin{equation*}
                kg\leq x< (k+1)g
            \end{equation*}
            Así, $kg\in G$ lo cual implica que $x-kg\in H$, por ende:
            \begin{equation*}
                \begin{split}
                    0\leq& x-kg\\
                    <&(k+1)g-kg\\
                    =g&
                \end{split}
            \end{equation*}
            al ser $g$ el ínfimo, debe suceder que $x-kg=0$, es decir que $x=kg$. Por tanto, $G=\gen{g}$.
        \end{enumerate}
        por los dos incisos anteriores, se sigue que $G$ es denso ó es cíclico.
    \end{proof}

    \begin{propo}
        Sea $\alpha\in\mathbb{R}$. Entonces el conjunto:
        \begin{equation*}
            A=\left\{a+b\alpha\Big|a,b\in\mathbb{Z} \right\}
        \end{equation*}
        es denso en $\mathbb{R}$ con la topología usual.
    \end{propo}

    \begin{proof}
        Afirmamos que $A$ es un subgrupo de $\mathbb{R}$ el cual no es cíclico, por tanto, de la proposición anterior, se sigue que $A$ es denso en $\mathbb{R}$ con la topología usual.

        Es claro que $A$ es subgrupo de $(\mathbb{R},+)$, pues si $a_1+b_1\alpha,a_2+b_2\alpha\in A$, se tiene que el elemento $(a_1-a_2)+(b_1-b_2)\alpha\in A$ ya que $a_1-a_2,b_1-b_2\in\mathbb{Z}$.

        Ahora, supongamos que $A$ es cíclico, entonces existiría $a+b\alpha\in A$ positivo (lo podemos elegir positivo y no puede ser cero ya que $\alpha\in A$) tal que $A=\gen{a+b\alpha}$. En particular, $\alpha\in A$, por tanto, existe $m\in\mathbb{Z}$ tal que
        \begin{equation*}
            \begin{split}
                \alpha=&m(a+b\alpha)\\
                \Rightarrow (1-mb)\alpha=&ma\\
            \end{split}
        \end{equation*}
        entonces, $mb=1$, lo cual implica que $m=b=\pm 1$ (en caso contrario, un lado de la ecuación sería irracional y el otro entero), y que $a=0$. Por tanto, $A=\gen{\alpha}=\gen{-\alpha}$, pero esto no puede suceder pues el elemento $1+2\alpha\notin\gen{\alpha}$, pero $1+2\alpha\in A$\contradiction.

        Por tanto, $A$ no es cíclico. Luego, de la proposición anterior, se sigue que $A$ es denso en $\mathbb{R}$ con la topología usual.
    \end{proof}
    
    \begin{propo}
        Sea $\cf{f}{\mathbb{R}}{[-1,1]}$ función continua y periódica de período $T>0$. Entonces, si $f(\mathbb{Z})$ es denso en $(\mathbb{R},\tau_u)$, entonces $f(\mathbb{N})$ también lo es.
    \end{propo}

    \begin{proof}
        Si $T$ es racional, entonces $f(\mathbb{Z})={T}$ el cual no es denso en $[-1,1]$, por tanto, $T$ debe ser irracional. Como $f$ es continua y acotada, entonces es uniformemente continua en $\mathbb{R}$.

        Sea $x\in[-1,1]$ y $\varepsilon>0$, entonces existe $m\in\mathbb{Z}$ tal que $\abs{f(m)-x}<\frac{\varepsilon}{2}$. Como $f$ es uniformemente continua, existe $\delta>0$ tal que si $\abs{u-v}<\delta$ entones $\abs{f(u)-f(v)}<\frac{\varepsilon}{2}$.

        Si $m\in\mathbb{N}$, se tiene el resultado. Suponga que $m\leq 0$. Existen $p,q\in\mathbb{N}$ tales que
        \begin{equation*}
            \abs{p-Tq}<\delta
        \end{equation*}
        donde $p>-m$ y $q>1/\delta$, esto pues el conjunto $]T,\infty[\cap\mathbb{Q}$ es denso en $[T,\infty[$. Entonces:
        \begin{equation*}
            \begin{split}
                \abs{f(m+p)-\alpha}&\leq\abs{f(m+p)-f(m)}+\abs{f(m)-\alpha} \\
                &\leq\abs{f(m+(p-Tq))-f(m)}+\abs{f(m)-\alpha} \\
                &<\frac{\varepsilon}{2}+\frac{\varepsilon}{2}\\
                &=\varepsilon\\
            \end{split}
        \end{equation*}
        con $p+m\in\mathbb{N}$. Luego $f(\mathbb{N})$ es denso en $[-1,1]$.

    \end{proof}

\end{document}