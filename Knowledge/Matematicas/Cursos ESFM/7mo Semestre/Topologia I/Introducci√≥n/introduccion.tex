\documentclass[12pt]{report}
\usepackage[spanish]{babel}
\usepackage[utf8]{inputenc}
\usepackage{amsmath}
\usepackage{amssymb}
\usepackage{amsthm}
\usepackage{graphics}
\usepackage{subfigure}
\usepackage{lipsum}
\usepackage{array}
\usepackage{multicol}
\usepackage{enumerate}
\usepackage[framemethod=TikZ]{mdframed}
\usepackage[a4paper, margin = 1.5cm]{geometry}

%En esta parte se hacen redefiniciones de algunos comandos para que resulte agradable el verlos%

\renewcommand{\theenumii}{\roman{enumii}}

\def\proof{\paragraph{Demostración:\\}}
\def\endproof{\hfill$\blacksquare$}

\def\sol{\paragraph{Solución:\\}}
\def\endsol{\hfill$\square$}

%En esta parte se definen los comandos a usar dentro del documento para enlistar%

\newtheoremstyle{largebreak}
  {}% use the default space above
  {}% use the default space below
  {\normalfont}% body font
  {}% indent (0pt)
  {\bfseries}% header font
  {}% punctuation
  {\newline}% break after header
  {}% header spec

\theoremstyle{largebreak}

\newmdtheoremenv[
    leftmargin=0em,
    rightmargin=0em,
    innertopmargin=-2pt,
    innerbottommargin=8pt,
    hidealllines = true,
    roundcorner = 5pt,
    backgroundcolor = gray!60!red!30
]{exa}{Ejemplo}[section]

\newmdtheoremenv[
    leftmargin=0em,
    rightmargin=0em,
    innertopmargin=-2pt,
    innerbottommargin=8pt,
    hidealllines = true,
    roundcorner = 5pt,
    backgroundcolor = gray!50!blue!30
]{obs}{Observación}[section]

\newmdtheoremenv[
    leftmargin=0em,
    rightmargin=0em,
    innertopmargin=-2pt,
    innerbottommargin=8pt,
    rightline = false,
    leftline = false
]{theor}{Teorema}[section]

\newmdtheoremenv[
    leftmargin=0em,
    rightmargin=0em,
    innertopmargin=-2pt,
    innerbottommargin=8pt,
    rightline = false,
    leftline = false
]{propo}{Proposición}[section]

\newmdtheoremenv[
    leftmargin=0em,
    rightmargin=0em,
    innertopmargin=-2pt,
    innerbottommargin=8pt,
    rightline = false,
    leftline = false
]{cor}{Corolario}[section]

\newmdtheoremenv[
    leftmargin=0em,
    rightmargin=0em,
    innertopmargin=-2pt,
    innerbottommargin=8pt,
    rightline = false,
    leftline = false
]{lema}{Lema}[section]

\newmdtheoremenv[
    leftmargin=0em,
    rightmargin=0em,
    innertopmargin=-2pt,
    innerbottommargin=8pt,
    roundcorner=5pt,
    backgroundcolor = gray!30,
    hidealllines = true
]{mydef}{Definición}[section]

\newmdtheoremenv[
    leftmargin=0em,
    rightmargin=0em,
    innertopmargin=-2pt,
    innerbottommargin=8pt,
    roundcorner=5pt
]{excer}{Ejercicio}[section]

%En esta parte se colocan comandos que definen la forma en la que se van a escribir ciertas funciones%

\newcommand\abs[1]{\ensuremath{\biglvert#1\bigrvert}}
\newcommand\divides{\ensuremath{\bigm|}}
\newcommand\cf[3]{\ensuremath{#1:#2\rightarrow#3}}
\newcommand\contradiction{\ensuremath{\#_c}}

%recuerda usar \clearpage para hacer un salto de página

\begin{document}
    \title{Notas del curso Topología I}
    \author{Cristo Daniel Alvarado}
    \maketitle

    \tableofcontents %Con este comando se genera el índice general del libro%

    \setcounter{chapter}{-1} %En esta parte lo que se hace es cambiar la enumeración del capítulo%
    
    \chapter{Introduccion}
    
    \section{Temario}
    
    \section{Bibliografía}    

    \begin{enumerate}
        \item J. R. Munkres 'Topología' - Prentices Hall.
        \item M. Gemignsni 'Elementary Topology' -  Dover.
        \item J. Dugundji 'Topology' -  Allyn Bacon.
    \end{enumerate}

    \chapter{Conceptos Fundamentales}

    \section{Fundamentos}

    \begin{mydef}
        Sea $X$ un conjunto y $\mathcal{A}$ una familia no vacía de subconjuntos de $X$. Definamos los \textbf{complementos de $\mathcal{A}$}
        \begin{equation*}
            \mathcal{A}':=\left\{X-A\big| A\in\mathcal{A} \right\}
        \end{equation*}
        (básicamente es el conjunto de todos los complementos de los conjuntos en $\mathcal{A}$). Para no perder ambiguedad, no denotaremos al complemento de un conjunto por $B^c$, sino por $X-B$ (para denotar quien es el conjunto sobre el que se toma el complemento del conjunto).

        La \textbf{unión de los elementos} de $\mathcal{A}$ se define com oel conjunto:
        \begin{equation*}
            \bigcup\mathcal{A}=\bigcup_{A\in\mathcal{A}}A=\left\{x\in X\big| x\in A\textup{ para algún elemento }A\in\mathcal{A} \right\}
        \end{equation*}
        denotada por el símbolo de la izquierda.

        La \textbf{intersección de los elementos} de $\mathcal{A}$ se define como el conjunto:
        \begin{equation*}
            \bigcap\mathcal{A}=\bigcap_{A\in\mathcal{A}}A=\left\{x\in X\big| x\in A\textup{ para todo elemento }A\in\mathcal{A} \right\}
        \end{equation*}
    \end{mydef}

    \begin{obs}
        En caso de que la colección $\mathcal{A}$ sea vacía, no se puede hacer lo que marca la definición anterior. Como $\mathcal{A}$ es vacía, entonces $\mathcal{A}'$ también es vacía.
        \begin{enumerate}
            \item Suponga que $\cup\mathcal{A}\neq\emptyset$, entonces existe $x\in X$ tal que $x\in \cup\mathcal{A}$, luego existe algún elemento $A\in\mathcal{A}$ tal que $x\in A$, pero esto no puede suceder, pues la familia $\mathcal{A}$ es vacía. \contradiction. Por tanto, $\cup\mathcal{A}=\emptyset$.
            \item Ahora, si aplicamos las leyes de Morgan, y tomamos
            \begin{equation*}
                X-\cap\mathcal{A}=X-\cap\emptyset=\cup\emptyset'=\cup\emptyset=\emptyset
            \end{equation*}
            luego, $\cap\mathcal{A}=X$.
        \end{enumerate}
        En definitiva, si $\mathcal{A}$ es una colección vacía, entonces definimos $\cup\mathcal{A}=\emptyset$ y $\cap\mathcal{A}=X$.
    \end{obs}

    La observación junto con la definición anterior se usarán a lo largo de todo el curos y serán de utilidad.

    \begin{mydef} 
        Sea $X$ un conjunto y sea $\tau$ una familia de subconjuntos de $X$. Se dice que $\tau$ es una \textbf{una topología definida sobre $X$} si se cumple lo siguiente:
        \begin{enumerate}
            \item $\emptyset,X\in\tau$.
            \item Si $\mathcal{A}$ es una subcolección de $\tau$, entonces $\bigcup\mathcal{A}\in\tau$.
            \item Si $A,B\in\tau$, entonces $A\cap B\in\tau$.
        \end{enumerate}
    \end{mydef}

    \begin{obs}
        En algunos libros viejos viene la siguiente condición adicional a la definición:
        \begin{enumerate}
            \setcounter{enumi}{3}
            \item Si $p,q\in X$ con $p\neq q$, entonces existen $U, V\in\tau$ tales que $p\in U$, $q\in V$ y $U\cap V=\emptyset$.
        \end{enumerate}
        en este caso se dirá que el espacio es \textbf{Hausdorff}.
    \end{obs}

    \begin{obs}
        Se tienen las siguientes observaciones:
        \begin{enumerate}
            \item Sea $X$ un conjunto y $\mathcal{A}$ una familia de subconjuntos de $X$. Si
            \begin{equation*}
                \mathcal{A}=\left\{A_\alpha\big|\alpha\in I \right\}
            \end{equation*}
            entonces podemos escribir
            \begin{equation*}
                \bigcup\mathcal{A}=\bigcup_{A_\alpha\in\mathcal{A}}A=\bigcup_{\alpha\in I}A_\alpha
            \end{equation*}
            e igual con la intersección:
            \begin{equation*}
                \bigcap\mathcal{A}=\bigcap_{A_\alpha\in\mathcal{A}}A=\bigcap_{\alpha\in I}A_\alpha
            \end{equation*}
            Si $\mathcal{A}$ es una familia vacía, y se toma como definición lo dicho en la observación 1.0.1, entonces podemos omitir el primer inciso de la definición anterior.
            \item Si $\tau$ es una topología sobre $X$ y para $n\in\mathbb{N}$, $A_1,...,A_n\in\tau$, entonces $A_1\cap...\cap A_n\in\tau$.
        \end{enumerate}
    \end{obs}

    \begin{exa}
        Sea $X$ un conjunto no vacío.
        \begin{enumerate}
            \item El conjunto potencia (denotado por $\mathcal{P}$) de $X$ es una topología sobre $X$, la cual se llama la \textbf{topología discreta}, y se denota por $\tau_D$.
            \item La colección formada únicamente por $X$ y $\emptyset$ es una topolgía sobre $X$, es decir $\tau=\left\{\emptyset,X \right\}$ es llamada la \textbf{topología indiscreta}, y se escribe como $\tau_I$.
            \item En el caso de que $X=\left\{1\right\}$, se tendría que $\tau_D=\left\{\emptyset,\left\{1\right\} \right\}$ y $\tau_I=\left\{\emptyset,\left\{1\right\} \right\}$.
            
            Si $X=\left\{1,\zeta\right\}$, entonces $\tau_D=\left\{\emptyset,\left\{1\right\},\left\{\zeta\right\},\left\{1,\zeta\right\} \right\}$ y $\tau_I=\left\{\emptyset,\left\{1,\zeta\right\} \right\}$.

            \item Si $\tau$ es una topología sobre $X$, entonces
            \begin{equation*}
                \tau_I\subseteq\tau\subseteq\tau_D
            \end{equation*}
            \item Sea $a\in X$. Entonces $\tau=\left\{\emptyset,X,\left\{a\right\},\right\}$ es una topología sobre $X$.
            \item Sea $A\subseteq X$ y sea $\tau\left(A\right)=\left\{B\subseteq X\big| A\subseteq B \right\}\bigcup\left\{\emptyset\right\}$. Esta familia $\tau\left(A\right)$ es una topología sobre $X$.
        \end{enumerate} 
    \end{exa}

    \begin{obs}
        Sea $X$ un conjunto no vacío. Si $A=\left\{a\right\}\subseteq X$, entonces escribimos $\tau_a$ en vez de $\tau\left(A\right)$.
    \end{obs}

    \setcounter{exa}{0}
    \begin{exa}
        Se continuan con los ejemplos anteriores:
        \begin{enumerate}
            \setcounter{enumi}{6}
            \item Sea $\tau_{cf}=\left\{A\subseteq X\big| X-A\textup{ es un conjunto finito} \right\}\bigcup\left\{\emptyset\right\}$. Esta es una topología sobre $X$ y se llama la \textbf{topología de los complementos finitos}.
            \item Si $X$ es un conjunto finito, entonces $\tau_{cf}=\tau_D=\mathcal{P}$.
            \item Considere $\tau_{cf}$ y sean $a,b\in X$ con $a\neq b$. Si $U_a=X-\left\{b\right\}$, $U_b=X-\left\{a\right\}$, entonces $U_a,U_b\in\tau_{cf}$ y además, $a\in U_a$ pero $b\notin U_a$ y $a\notin U_b$ pero $b\in U_b$. Esta propiedad es muy importante tenerla en mente pues más adelante se usará.
        \end{enumerate}
    \end{exa}

    \begin{propo}
        Sea $(X,d)$ un espacio métrico. Dados $a\in X$ y $\varepsilon\in\mathbb{R}^+$, al conjunto $B_d(x,\varepsilon)=\left\{y\in X\big|d(x,y)<\varepsilon \right\}$ se llama \textbf{$\varepsilon$-bola con centro en $x$ y radio $\varepsilon$}. 
        
        Sea
        \begin{equation*}
            \tau_d=\left\{A\subseteq X\big| \forall a\in A \exists r>0\textup{ tal que }B_d(a,r)\subseteq A \right\}
        \end{equation*}
        Esta colección es una topología sobre $X$.
    \end{propo}

    \newpage

    \begin{proof}
        Entorno de Prueba
    \end{proof}

    \begin{sol}
        Entorno de Solución
    \end{sol}

    \begin{theor}[Nombre]
        Teorema
    \end{theor}

    \begin{propo}[Nombre]
        Proposición
    \end{propo}

    \begin{cor}[Nombre]
        Corolario
    \end{cor}

    \begin{lema}[Nombre]
        Lema
    \end{lema}

    \begin{mydef}[Nombre]
        Definición
    \end{mydef}

    \begin{obs}[Nombre]
        Observación
    \end{obs}

    \begin{exa}[Nombre]
        Ejemplo
    \end{exa}

    \begin{excer}[Nombre]
        Ejercicio
    \end{excer}

\end{document}