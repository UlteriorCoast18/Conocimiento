\documentclass[12pt]{report}
\usepackage[spanish]{babel}
\usepackage[utf8]{inputenc}
\usepackage{amsmath}
\usepackage{amssymb}
\usepackage{amsthm}
\usepackage{graphics}
\usepackage{subfigure}
\usepackage{lipsum}
\usepackage{array}
\usepackage{multicol}
\usepackage{enumerate}
\usepackage[framemethod=TikZ]{mdframed}
\usepackage[a4paper, margin = 1.5cm]{geometry}

%En esta parte se hacen redefiniciones de algunos comandos para que resulte agradable el verlos%

\renewcommand{\theenumii}{\roman{enumii}}

\def\proof{\paragraph{Demostración:\\}}
\def\endproof{\hfill$\blacksquare$}

\def\sol{\paragraph{Solución:\\}}
\def\endsol{\hfill$\square$}

%En esta parte se definen los comandos a usar dentro del documento para enlistar%

\newtheoremstyle{largebreak}
  {}% use the default space above
  {}% use the default space below
  {\normalfont}% body font
  {}% indent (0pt)
  {\bfseries}% header font
  {}% punctuation
  {\newline}% break after header
  {}% header spec

\theoremstyle{largebreak}

\newmdtheoremenv[
    leftmargin=0em,
    rightmargin=0em,
    innertopmargin=-2pt,
    innerbottommargin=8pt,
    hidealllines = true,
    roundcorner = 5pt,
    backgroundcolor = gray!60!red!30
]{exa}{Ejemplo}[section]

\newmdtheoremenv[
    leftmargin=0em,
    rightmargin=0em,
    innertopmargin=-2pt,
    innerbottommargin=8pt,
    hidealllines = true,
    roundcorner = 5pt,
    backgroundcolor = gray!50!blue!30
]{obs}{Observación}[section]

\newmdtheoremenv[
    leftmargin=0em,
    rightmargin=0em,
    innertopmargin=-2pt,
    innerbottommargin=8pt,
    rightline = false,
    leftline = false
]{theor}{Teorema}[section]

\newmdtheoremenv[
    leftmargin=0em,
    rightmargin=0em,
    innertopmargin=-2pt,
    innerbottommargin=8pt,
    rightline = false,
    leftline = false
]{propo}{Proposición}[section]

\newmdtheoremenv[
    leftmargin=0em,
    rightmargin=0em,
    innertopmargin=-2pt,
    innerbottommargin=8pt,
    rightline = false,
    leftline = false
]{cor}{Corolario}[section]

\newmdtheoremenv[
    leftmargin=0em,
    rightmargin=0em,
    innertopmargin=-2pt,
    innerbottommargin=8pt,
    rightline = false,
    leftline = false
]{lema}{Lema}[section]

\newmdtheoremenv[
    leftmargin=0em,
    rightmargin=0em,
    innertopmargin=-2pt,
    innerbottommargin=8pt,
    roundcorner=5pt,
    backgroundcolor = gray!30,
    hidealllines = true
]{mydef}{Definición}[section]

\newmdtheoremenv[
    leftmargin=0em,
    rightmargin=0em,
    innertopmargin=-2pt,
    innerbottommargin=8pt,
    roundcorner=5pt
]{excer}{Ejercicio}[section]

%En esta parte se colocan comandos que definen la forma en la que se van a escribir ciertas funciones%

\newcommand\abs[1]{\ensuremath{\biglvert#1\bigrvert}}
\newcommand\divides{\ensuremath{\bigm|}}
\newcommand\cf[3]{\ensuremath{#1:#2\rightarrow#3}}
\newcommand\contradiction{\ensuremath{\#_c}}
\newcommand{\V}[1]{\ensuremath{\mathcal{V}(#1)}}
\newcommand{\Int}[1]{\ensuremath{\mathring{#1}}}
\newcommand{\Cls}[1]{\ensuremath{\overline{#1}}}
\newcommand{\Fr}[1]{\ensuremath{\textup{Fr}(#1)}}

%recuerda usar \clearpage para hacer un salto de página

\begin{document}
    \title{Notas del curso Topología I}
    \author{Cristo Daniel Alvarado}
    \maketitle

    \tableofcontents %Con este comando se genera el índice general del libro%

    \setcounter{chapter}{-1} %En esta parte lo que se hace es cambiar la enumeración del capítulo%
    
    \chapter{Introduccion}
    
    \section{Temario}
    
    Checar el Munkres

    \section{Bibliografía}    

    \begin{enumerate}
        \item J. R. Munkres 'Topología' - Prentices Hall.
        \item M. Gemignsni 'Elementary Topology' -  Dover.
        \item J. Dugundji 'Topology' -  Allyn Bacon.
    \end{enumerate}

    \chapter{Conceptos Fundamentales}

    \section{Fundamentos}

    \begin{mydef}
        Sea $X$ un conjunto y $\mathcal{A}$ una familia no vacía de subconjuntos de $X$. Definamos los \textbf{complementos de $\mathcal{A}$}
        \begin{equation*}
            \mathcal{A}':=\left\{X-A\big| A\in\mathcal{A} \right\}
        \end{equation*}
        (básicamente es el conjunto de todos los complementos de los conjuntos en $\mathcal{A}$). Para no perder ambiguedad, no denotaremos al complemento de un conjunto por $B^c$, sino por $X-B$ (para denotar quien es el conjunto sobre el que se toma el complemento del conjunto).

        La \textbf{unión de los elementos} de $\mathcal{A}$ se define como el conjunto:
        \begin{equation*}
            \bigcup\mathcal{A}=\bigcup_{A\in\mathcal{A}}A=\left\{x\in X\big| x\in A\textup{ para algún elemento }A\in\mathcal{A} \right\}
        \end{equation*}
        denotada por el símbolo de la izquierda.

        La \textbf{intersección de los elementos} de $\mathcal{A}$ se define como el conjunto:
        \begin{equation*}
            \bigcap\mathcal{A}=\bigcap_{A\in\mathcal{A}}A=\left\{x\in X\big| x\in A\textup{ para todo elemento }A\in\mathcal{A} \right\}
        \end{equation*}
    \end{mydef}

    \begin{obs}
        En caso de que la colección $\mathcal{A}$ sea vacía, no se puede hacer lo que marca la definición anterior. Como $\mathcal{A}$ es vacía, entonces $\mathcal{A}'$ también es vacía.
        \begin{enumerate}
            \item Suponga que $\cup\mathcal{A}\neq\emptyset$, entonces existe $x\in X$ tal que $x\in \cup\mathcal{A}$, luego existe algún elemento $A\in\mathcal{A}$ tal que $x\in A$, pero esto no puede suceder, pues la familia $\mathcal{A}$ es vacía. \contradiction. Por tanto, $\cup\mathcal{A}=\emptyset$.
            \item Ahora, si aplicamos las leyes de Morgan, y tomamos
            \begin{equation*}
                X-\cap\mathcal{A}=X-\cap\emptyset=\cup\emptyset'=\cup\emptyset=\emptyset
            \end{equation*}
            luego, $\cap\mathcal{A}=X$.
        \end{enumerate}
        En definitiva, si $\mathcal{A}$ es una colección vacía, entonces definimos $\cup\mathcal{A}=\emptyset$ y $\cap\mathcal{A}=X$.
    \end{obs}

    La observación junto con la definición anterior se usarán a lo largo de todo el curos y serán de utilidad.

    \begin{mydef} 
        Sea $X$ un conjunto y sea $\tau$ una familia de subconjuntos de $X$. Se dice que $\tau$ es una \textbf{una topología definida sobre $X$} si se cumple lo siguiente:
        \begin{enumerate}
            \item $\emptyset,X\in\tau$.
            \item Si $\mathcal{A}$ es una subcolección de $\tau$, entonces $\bigcup\mathcal{A}\in\tau$.
            \item Si $A,B\in\tau$, entonces $A\cap B\in\tau$.
        \end{enumerate}
    \end{mydef}

    \begin{obs}
        En algunos libros viejos viene la siguiente condición adicional a la definición:
        \begin{enumerate}
            \setcounter{enumi}{3}
            \item Si $p,q\in X$ con $p\neq q$, entonces existen $U, V\in\tau$ tales que $p\in U$, $q\in V$ y $U\cap V=\emptyset$.
        \end{enumerate}
        en este caso se dirá que el espacio es \textbf{Hausdorff}.
    \end{obs}

    \begin{obs}
        Se tienen las siguientes observaciones:
        \begin{enumerate}
            \item Sea $X$ un conjunto y $\mathcal{A}$ una familia de subconjuntos de $X$. Si
            \begin{equation*}
                \mathcal{A}=\left\{A_\alpha\big|\alpha\in I \right\}
            \end{equation*}
            entonces podemos escribir
            \begin{equation*}
                \bigcup\mathcal{A}=\bigcup_{A\in\mathcal{A}}A=\bigcup_{\alpha\in I}A_\alpha
            \end{equation*}
            e igual con la intersección:
            \begin{equation*}
                \bigcap\mathcal{A}=\bigcap_{A\in\mathcal{A}}A=\bigcap_{\alpha\in I}A_\alpha
            \end{equation*}
            Si $\mathcal{A}$ es una familia vacía, y se toma como definición lo dicho en la observación 1.0.1, entonces podemos omitir el primer inciso de la definición anterior.
            \item Si $\tau$ es una topología sobre $X$ y para $n\in\mathbb{N}$, $A_1,...,A_n\in\tau$, entonces $A_1\cap...\cap A_n\in\tau$.
        \end{enumerate}
    \end{obs}

    \begin{exa}
        Sea $X$ un conjunto no vacío.
        \begin{enumerate}
            \item El conjunto potencia (denotado por $\mathcal{P}$) de $X$ es una topología sobre $X$, la cual se llama la \textbf{topología discreta}, y se denota por $\tau_D$.
            \item La colección formada únicamente por $X$ y $\emptyset$ es una topolgía sobre $X$, es decir $\tau=\left\{\emptyset,X \right\}$ es llamada la \textbf{topología indiscreta}, y se escribe como $\tau_I$.
            \item En el caso de que $X=\left\{1\right\}$, se tendría que $\tau_D=\left\{\emptyset,\left\{1\right\} \right\}$ y $\tau_I=\left\{\emptyset,\left\{1\right\} \right\}$.
            
            Si $X=\left\{1,\zeta\right\}$, entonces $\tau_D=\left\{\emptyset,\left\{1\right\},\left\{\zeta\right\},\left\{1,\zeta\right\} \right\}$ y $\tau_I=\left\{\emptyset,\left\{1,\zeta\right\} \right\}$.

            \item Si $\tau$ es una topología sobre $X$, entonces
            \begin{equation*}
                \tau_I\subseteq\tau\subseteq\tau_D
            \end{equation*}
            \item Sea $a\in X$. Entonces $\tau=\left\{\emptyset,X,\left\{a\right\},\right\}$ es una topología sobre $X$.
            \item Sea $A\subseteq X$ y sea $\tau\left(A\right)=\left\{B\subseteq X\big| A\subseteq B \right\}\bigcup\left\{\emptyset\right\}$. Esta familia $\tau\left(A\right)$ es una topología sobre $X$.
        \end{enumerate} 
    \end{exa}

    \begin{sol}
        Para el inciso 6., veamos que $\tau(A)$ es una topología sobre $X$. En efecto, verificaremos que se cumplen las 3 condiciones:
        \begin{enumerate}
            \item Claro que $\emptyset\in\tau(A)$ por definición de $\tau(A)$. Además $X\in\tau(A)$ ya que $X\subseteq X$ y $A\subseteq X$.
            \item Sea $\mathcal{B}$ una familia no vacía de subconjuntos de $\tau(A)$, entonces existe $B_0\in\mathcal{B}$ tal que $A\subseteq B_0$, por lo cual
            \begin{equation*}
                A\subseteq B_0\subseteq\bigcup_{B\in\mathcal{B}}B\subseteq X
            \end{equation*}
            por tanto $\bigcup_{B\in\mathcal{B}}B\in\tau(A)$.
            \item Sean $C,D\in\tau(A)$, entonces $A\subseteq C$ y $A\subseteq B$, por ende $A\subseteq B\cap C\subseteq X$. Así, $B\cap C\in\tau(A)$.
        \end{enumerate}
        Por los incisos anteriores, la familia descrita en el inciso 6. es una topología sobre $X$.
    \end{sol}

    \begin{obs}
        Sea $X$ un conjunto no vacío. Si $A=\left\{a\right\}\subseteq X$, entonces escribimos $\tau_a$ en vez de $\tau\left(A\right)$.
    \end{obs}

    \setcounter{exa}{0}
    \begin{exa}
        Se continuan con los ejemplos anteriores:
        \begin{enumerate}
            \setcounter{enumi}{6}
            \item Sea $\tau_{cf}=\left\{A\subseteq X\big| X-A\textup{ es un conjunto finito} \right\}\bigcup\left\{\emptyset\right\}$. Esta es una topología sobre $X$ y se llama la \textbf{topología de los complementos finitos}.
            \item Si $X$ es un conjunto finito, entonces $\tau_{cf}=\tau_D=\mathcal{P}$.
            \item Considere (en un conjunto finito $X$) a $\tau_{cf}$ y sean $a,b\in X$ con $a\neq b$. Si $U_a=X-\left\{b\right\}$, $U_b=X-\left\{a\right\}$, entonces $U_a,U_b\in\tau_{cf}$ y además, $a\in U_a$ pero $b\notin U_a$ y $a\notin U_b$ pero $b\in U_b$. Esta propiedad es muy importante tenerla en mente pues más adelante se usará.
        \end{enumerate}
    \end{exa}

    \begin{sol}
        Veamos que la famila del ejemplo 7. es una topología sobre $X$. En efecto, veamos que se cumplen las 3 condiciones:
        \begin{enumerate}
            \item Claro que $\emptyset\in\tau_{cf}$ (por definición de $\tau_{cf}$). Y además $X\in\tau_{cf}$ ya que $\emptyset=X-X$ es un conjunto finito y $X\subseteq X$.
            \item Sea $\mathcal{A}$ una familia no vacía de subconjuntos de $\tau_{cf}$. Se cumple entonces que existe $A_0\in\mathcal{A}$ tal que $X-A_0$ es finito. Por lo cual como
            \begin{equation*}
                X-\bigcup\mathcal{A}\subseteq X-A
            \end{equation*}
            ya que $A\subseteq\bigcup\mathcal{A}$, se tiene que $X-\bigcup\mathcal{A}$ es finito y $\bigcup\mathcal{A}\subseteq X$. Por tanto, $\bigcup\mathcal{A}\in\mathcal{A}$.
            \item Sean $A,B\in\tau_{cf}$. Probaremos que $A\cap B\in\tau_{cf}$. Afirmamos que $X-A\cap B$ es finito, en efecto, por leyes de Morgan se tiene que
            \begin{equation*}
                X-(A\cap B)=(X-A)\cup(X-B)\subseteq X
            \end{equation*}
            donde $X-A$ y $X-B$ son finitos, por lo cual su unión también lo es. Por tanto $A\cap B\in\tau_{cf}$.
        \end{enumerate}
        Por los tres incisos anteriores, se sigue que $\tau_{cf}$ es una topología sobre $X$.
    \end{sol}

    A continuación se verá una proposición la cual tiene como objetivo el inducir una topología sobre un espacio métrico $(X,d)$ arbitrario.

    \begin{propo}
        Sea $(X,d)$ un espacio métrico. Dados $a\in X$ y $\varepsilon\in\mathbb{R}^+$, al conjunto $B_d(x,\varepsilon)=\left\{y\in X\big|d(x,y)<\varepsilon \right\}$ se llama \textbf{$\varepsilon$-bola con centro en $x$ y radio $\varepsilon$}. 
        
        Sea
        \begin{equation*}
            \tau_d=\left\{A\subseteq X\big| \forall a\in A \exists r>0\textup{ tal que }B_d(a,r)\subseteq A \right\}
        \end{equation*}
        Esta colección es una topología sobre $X$.
    \end{propo}

    \begin{proof}
        Se verificará que se cumplen las tres condiciones.
        \begin{enumerate}
            \item Por vacuidad, $\emptyset\in\tau_d$. Además, $X\in\tau_d$, pues para todo $x\in X$, $B_d(x,1)\subseteq X$.
            \item Sean $\mathcal{A}$ una familia no vacía de subconjuntos de $\tau_d$. Sea $p\in\cup\mathcal{A}$, es decir que existe $A_\beta\in\mathcal{A}$ tal que $p\in A_\beta$, así existe $r>0$ tal que $B_d(a,r)\subseteq A_\beta\subseteq\cup\mathcal{A}$, luego $\cup\mathcal{A}\in\tau_d$.
            \item Sean $M,N\in\tau_d$, y sea $p\in M\cap N$, es decir que $p\in M$ y $p\in N$, por lo cual existen $r_1,r_2>0$ tales que $B_d(p,r_1)\subseteq M$ y $B_d(p,r_2)\subseteq N$. Sea $r=\min\left\{r_1,r_2\right\}$, es inmediato que $B_d(p,r)\subseteq B_d(p,r_i)$, para $i=1,2$. Por tanto, $B_d(p,r)\subseteq M\cap N$. Luego, como el $p$ fue arbitrario,se sigue que $M\cap N\in \tau_d$.
        \end{enumerate}
    \end{proof}

    \begin{mydef}
        La topología de la proposición anterior es llamada la \textbf{topología generada por la métrica $d$}.
    \end{mydef}

    \begin{excer}
        Sea $(X,d)$ espacio métrico. Veamos que, dados $x\in X$ y $r>0$, se cumple que $B_d(x,r)\in\tau_d$.
    \end{excer}

    \begin{sol}
        Sea $y\in B_d(x,r)$, entonces $d(x,y)<r$. Sea $\varepsilon=d(x,y)$ y, supongamos que $x\neq y$ (pues en caso contrario, el caso es inmediato ya que $B_d(x,r)\subseteq B_d(x,r)$) luego $\varepsilon>0$ y además $\varepsilon<r$. Sea $s=r-\varepsilon\in\mathbb{R}^+$.

        Afirmamos que $B_d(y,s)\subseteq B_d(x,r)$. En efecto, sea $z\in B_d(y,s)$, entonces
        \begin{equation*}
            \begin{split}
                d(z,y)&<s\\
                \Rightarrow d(z,y)&<r-\varepsilon\\
                \Rightarrow d(z,y)+\varepsilon&<r\\
                \Rightarrow d(z,y)+d(y,x)&<r\\
                \Rightarrow d(z,x)&<r\\
            \end{split}
        \end{equation*}
        por tanto, $x\in B_d(x,r)$. Luego, $B_d(x,r)\in\tau_d$.
    \end{sol}

    \begin{lema}
        Todo espacio métrico $(X,d)$ es Hausdorff. 
    \end{lema}

    \begin{proof}
        Veamos que dados $x,y\in X$, $x\neq y$ existen $r,s\in\mathbb{R}^+$ tales que $B_d(x,r)\cap B_d(y,s)=\emptyset$. Como $x\neq y$ entonces $d(x,y)=m\in\mathbb{R}^+$. Tomemos $r=\frac{m}{\pi}$ y $s=\frac{\pi-1}{\pi}m$ y veamos que la intersección es vacía.
    
        En efecto, en caso de que no lo fuese se tendría que si existiera $p\in B_d(x,r)\cap B_d(y,s)$, entonces $d(p,x)<\frac{m}{\pi}$ y $d(p,y)<\frac{\pi-1}{\pi}m$, por lo cual de la desigualdad triangular se sigue que:
        \begin{equation*}
            d(x,y)\leq d(p,x)+d(p,y)<\frac{1+\pi-1}{\pi}m=m=d(x,y)
        \end{equation*}
        lo cual es una contradicción\contradiction. Por tanto, la intersección es vacía.
    \end{proof}

    Retomando al espacio métrico $(X,d)$, tenemos que para $A\subseteq X$, $A\in\tau_d$ si y sólo si existen $\left\{a_\alpha\right\}_{\alpha\in I}\subseteq A$ y $\left\{\varepsilon_\alpha\right\}_{\alpha\in I}\subseteq\mathbb{R}^+$ tales que
    \begin{equation*}
        \bigcup_{\alpha\in I }B_d(a_\alpha,\varepsilon_\alpha)=A
    \end{equation*}
    donde $\forall\alpha\in I$ se tiene que $A_\alpha\in \mathcal{A}$.

    \begin{cor}
        Sea $(X,d)$ un espacio métrico y
        \begin{equation*}
            \mathcal{B}_d=\left\{B_d(x,\varepsilon)|x\in X,\varepsilon\in\mathbb{R}^+ \right\}
        \end{equation*}
        entonces, para $A\subseteq X$ se tiene que $A\in\tau_d$ si y sólo si existe una colección $\left\{B_\alpha \right\}_{\alpha\in I}\subseteq \mathcal{B}_d$ tal que $A=\bigcup_{\alpha\in I}B_\alpha$. La colección $\mathcal{B}_d\subseteq\tau_d$.
    \end{cor}

    \begin{exa}
        Sea $m\in\mathbb{N}$ y considere el espacio métrico $\mathbb{R}^m$ con la métrica $d_u$, siendo:
        \begin{equation*}
            d_u(x,y)=[(x_1-y_1)^2+...+(x_m-y_m)^2]^{\frac{1}{2}}
        \end{equation*}
        para $x=(x_1,...,x_m),y=(y_1,...,y_m)\in\mathbb{R}^m$. Esta métrica será denominada \textbf{métrica usual}. Vamos a escribir a la topología generada por esta métrica como $\tau_u$, y se dice la \textbf{topología usual definida sobre $\mathbb{R}^m$}. En particular, cuando $m=1$ tenemos que $\tau_u$ la topología usual definida sobre $\mathbb{R}$. En este caso, se tiene que $A\in\tau_u$ si y sólo si existen $\left\{a_\alpha\right\}_{\alpha\in I}$ y $\left\{B_\alpha\right\}_{\alpha\in I}$ subfamilias de $\mathbb{R}$ tal que $A=\bigcup_{\alpha\in I}\left(a_\alpha,b_\alpha\right)$.
    \end{exa}

    \begin{obs}
        Tenemos que para todo $n\in\mathbb{N}$, los conjuntos $\left(-\frac{1}{n},\frac{1}{n}\right)\in\tau_u$, y $\bigcap_{n\in\mathbb{N}}\left(-\frac{1}{n},\frac{1}{n}\right)=\left\{0\right\}\notin\tau_u$. Es decir, que la topología solo es cerrada (en general) bajo intersecciones finitas.
    \end{obs}

    \begin{mydef}
        Sea $X$ un conjunto, y sean $\tau_1$ y $\tau_2$ topologías sobre $X$. Decimos que $\tau_2$ es \textbf{más fina} que la topología $\tau_1$ si se tiene que $\tau_1\subseteq\tau_2$ (a veces también se dice que $\tau_1$ es \textbf{menos fina} que $\tau_2$).
    \end{mydef}

    \begin{exa}
        Sea $X=\left\{1,2,3\right\}$, $\tau_1=\left\{X,\emptyset,\left\{1\right\} \right\}$, $\tau_2=\left\{X,\emptyset,\left\{2\right\} \right\}$. Tomemos
        \begin{equation*}
            \tau_1\cup\tau_2=\left\{X,\emptyset,\left\{1\right\},\left\{2\right\}\right\}
        \end{equation*}
        la familia $\tau_1\cup\tau_2$ no es una topología sobre $X$, pues no es cerrada bajo uniones arbitrarias. Con esto se tiene que la unión de dos topologías no necesariamente es una topología.
    \end{exa}

    \begin{theor}
        Sea $X$ un conjunto, y sea $\left\{\tau_\alpha\right\}_{\alpha\in I}$ una familia de topologías sobre $X$, entonces $\tau=\bigcap_{\alpha\in I}\tau_\alpha$ es una topología sobre $X$.
    \end{theor}

    \begin{proof}
        Veamos que se cumplen las tres condiciones.
        \begin{enumerate}
            \item Claro que $X,\emptyset\in\tau$, pues $X,\emptyset\in\tau_\alpha$, para todo $\alpha\in I$.
            \item Sea $\mathcal{A}=\left\{A_\beta\right\}_{\beta\in J}\subseteq\tau=\bigcap_{\alpha\in I}\tau_\alpha$ una subcolección arbitraria de elementos de $\tau$. Por ser $\tau_\alpha$ una topología, se sigue que $\bigcup\mathcal{A}\in\tau_\alpha$, para todo $\alpha\in I$. Por tanto, $\bigcup\mathcal{A}\in\tau$.
            \item Sean $K,L\in\tau$, entonces $K,L\in\tau_\alpha$, para todo $\alpha\in I$, luego como $\tau_\alpha$ es una topología sobre $X$, se tiene que $L\cap K\in \tau_\alpha$, para todo $\alpha\in I$, por tanto $L\cap K\in\tau$.
        \end{enumerate}
        Por los tres incisos anteriores, se sigue que $\tau$ es una topología sobre $X$.
    \end{proof}

    \begin{cor}
        Sea $X$ un conjunto y sean $\mathcal{A}$ una familia de subconjuntos de $X$. Definimos
        \begin{equation*}
            \mathcal{K}=\left\{\tau|\tau\textup{ es una topología sobre }X\textup{ y }\mathcal{A}\subseteq\tau \right\}
        \end{equation*}
        Entonces:
        \begin{enumerate}
            \item $\tau_D\in\mathcal{K}$.
            \item Definiendo $\tau(\mathcal{A})=\bigcap_{\tau\in\mathcal{K}}\tau$, se tiene que $\tau(\mathcal{A})$ es una topología sobre $X$.
            \item Para toda topología $\tau\in\mathcal{K}$, $\tau(\mathcal{A})\subseteq\tau$.
            \item $\tau(\mathcal{A})\in\mathcal{K}$.
        \end{enumerate}
    \end{cor}

    \begin{proof}
        De 1. Es inmediato, pues como $\mathcal{A}\subseteq\mathcal{P}=\tau_D$ y $\tau_D$ es una topología sobre $X$, se sigue que $\tau_D\in\mathcal{K}$.

        De 2. Es inmediato del teorema anterior.

        De 3. Como $\tau(\mathcal{A})=\bigcap_{\tau\in\mathcal{K}}\tau$, entonces $\tau(\mathcal{A})\subseteq\tau$, para toda $\tau\in\mathcal{K}$.

        De 4. Por 2. $\tau(\mathcal{A})$ es una topología sobre $X$, y además $\mathcal{A}\subseteq\tau(\mathcal{A})$, pues $\mathcal{A}\subseteq\tau$, para todo $\tau\in\mathcal{K}$, luego $\mathcal{A}\subseteq\bigcap_{\tau\in\mathcal{K}}\tau=\tau(\mathcal{A})$. Por ende, $\tau(\mathcal{A})\in\mathcal{K}$.
    \end{proof}

    \begin{mydef}
        Un \textbf{espacio topológico} es una pareja $(X,\tau)$ en donde $X$ es un conjunto y $\tau$ es una topología sobre $X$. A los elementos de $\tau$ los llamaremos los \textbf{conjuntos abiertos} del espacio $(X,\tau)$ a veces también se les nombra como los \textbf{$\tau$-abiertos de $X$}.
    \end{mydef}

    \begin{exa}
        Ejemplos de espacios topológicos son $(\mathbb{R},\tau_D)$, $(\mathbb{R},\tau_I)$, $(\mathbb{R},\tau_{cf})$, $(\mathbb{R},\tau_u)$, etc... Las diferencias notables son que $\left\{1,\sqrt{2} \right\}$ es abierto en $(\mathbb{R},\tau_D)$, pero no en $(\mathbb{R},\tau_u)$.
    \end{exa}

    Sea $X$ un conjunto y $\mathcal{A}\subseteq\mathcal{P}$. Por el corolario anterior, podemos trabajar con la topología $\tau(\mathcal{A})$, y tenemos así al espacio topológico $(X,\tau(\mathcal{A}))$, el cual en particular tiene como abiertos a los elementos de la familia $\mathcal{A}$.
    
    \begin{mydef}
        Sea $(X,\tau)$ un espacio topológico.
        \begin{enumerate}
            \item Un subconjunto $C\subseteq X$ es un \textbf{conjunto cerrado} del espacio topológico $(X,\tau)$ si $X-C\in\tau$.
        \end{enumerate}
    \end{mydef}

    \begin{exa}
        En $(\mathbb{R},\tau_u)$ se tiene que $\mathbb{R}$ y $\emptyset$ son abiertos y cerrados a la vez, pero el conjunto $[1,2[$ no es abierto ni cerrado, $]1,2[$ es abierto pero no cerrado y $[1,2]$ no es abierto pero sí es cerrado.
    \end{exa}

    \begin{propo}
        Sea $(X,\tau)$ un espacio topológico.
        \begin{enumerate}
            \item Si $A_1,...,A_n$ son subconjuntos cerrados de $(X,\tau)$, entonces su unión $A_1\cup...\cup A_n$ es un cerrado de $(X,\tau)$.
            \item Si $\mathcal{A}$ es una familia arbitraria de conjuntos cerrados en $(X,\tau)$, entonces $\bigcap\mathcal{A}$ es un conjunto cerrado.
        \end{enumerate}
    \end{propo}

    \begin{proof}
        De (1): Consideremos el complemento de la unión. Se tiene que:
        \begin{equation*}
            \begin{split}
                X-\bigcup_{i=1}^nA_i=&\bigcap_{i=1}^n(X-A_i)\\
            \end{split}
        \end{equation*}
        el cuál es abierto por ser intersección finita de abiertos. Luego $\bigcap_{i=1}^nA_i$ es cerrado.

        De (2): Basta con aplicar leyes de Morgan.
    \end{proof}

    \begin{exa}
        Considere $(\mathbb{R},\tau_u)$ y, para $n\in\mathbb{N}$ definimos $A_n=(-\frac{1}{n},\frac{1}{n})$, es claro que cada uno de estos conjuntos es abierto. Sea $B_n=\mathbb{R}-A_n=(-\infty,-\frac{1}{n}]\cup[\frac{1}{n},\infty)$.

        Se tiene que:
        \begin{equation*}
            \bigcup_{n\in\mathbb{N}}B_n=\bigcup_{n\in\mathbb{N}}\mathbb{R}-A_n=\mathbb{R}-\bigcap_{n\in\mathbb{N}}A_n=\mathbb{R}-\left\{0\right\}
        \end{equation*}
        el cual es abierto. Por tanto, la unión arbitraria de cerrados no es cerrada (en general).
    \end{exa}

    \begin{mydef}
        Sea $(X,\tau)$ un espacio topológico y, sean $x\in X$ y $V\subseteq X$ tal que $x\in V$. Se dice que $V$ es una \textbf{vecindad de $x$} si existe $U\in\tau$ abierto tal que $x\in U$ y $U\subseteq V$.
        \begin{enumerate}
            \item Si $V$ es una vecindad de $x$ y $V\in\tau$, decimos que $V$ es una \textbf{vecindad abierta de $x$}.
            \item Si $V$ es una vecindad de $x$ y $X-V\in\tau$, decimos que $V$ es una \textbf{vecindad cerrada de $x$}.
        \end{enumerate}
        Al conjunto de todas las vecindades del punto $x$ lo denotamos por $\mathcal{V}(x)$. Tenemos que $X\in\mathcal{V}(x)$ para todo $x\in X$.
    \end{mydef}

    \begin{excer}
        Sea $(X,\tau)$ un espacio topológico.
        \begin{enumerate}
            \item Si $V_1,...,V_n\in\mathcal{V}(x)$ para $x\in X$, entonces $V_1\cap...\cap V_n\in\mathcal{V}(x)$.
            \item Si $\left\{V_\alpha\right\}_{\alpha\in I}\subseteq\mathcal{V}(x)$ para $x\in X$, entonces $\bigcap_{\alpha\in I}V_\alpha\in\mathcal{V}(x)$.
        \end{enumerate}
    \end{excer}

    \begin{sol}
        
    \end{sol}

    \begin{mydef}
        Sea $(X,\tau)$ un espacio topológico y $A\subseteq X$.
        \begin{enumerate}
            \item Sea $x\in X$. $x$ es un \textbf{punto de acumulación de $A$} si para todo $U$ abierto que contiene a $x$ se tiene que $(U-\left\{x\right\})\cap A\neq\emptyset$ ($U$ contiene un punto de $A$ diferente de $x$). Al conjunto de todos los puntos de acumulación lo llamaremos el \textbf{conjunto derivado de $A$}, y se denota por $A'$.
            \item Un elemento $a\in A$ es un \textbf{punto interior} de $A$, si $A$ es una vecindad de $x$ (es decir, $A\in\V{x}$). 
            \textbf{El interior de $A$} es el conjunto de todos los puntos interiores de $A$ y se escribe $\Int{A}$. Es claro que $\Int{A}\subseteq A$.
            \item Sea
            \begin{equation*}
                \mathcal{C}=\left\{C\subseteq X\big|X-C\in\tau,A\subseteq C \right\}
            \end{equation*}
            es claro que $\mathcal{C}$ es no vacía, pues $X\in\mathcal{C}$. La \textbf{cerradura de $A$} es el conjunto $\bigcap_{C\in\mathcal{C}}C$ y se denota por $\overline{A}$. Si $x\in\overline{A}$, diremos que \textbf{$x$ es un punto adherente de $A$}. Es claro que $A\subseteq\overline{A}$.
            \item La \textbf{frontera de $A$} es el conjunto $\Cls{A}\cap\Cls{X-A}$ y se denota por $\Fr{A}$.
        \end{enumerate}
    \end{mydef}

    \begin{propo}
        Sea $(X,\tau)$ un espacio topológico, $x\in X$ y sean $A,B\subseteq X$. Entonces:
        \begin{enumerate}
            \item $\Int{A}\subseteq A\subseteq\Cls{A}$.
            \item $\Int{A}=\bigcup\left\{U\in\tau\big|U\subseteq A \right\}$.
            \item $\Int{A}\in\tau$.
            \item Si $V\in\tau$ tal que $V\subseteq A$, entonces $V\subseteq\Int{A}$.
            \item $A$ es abierto si y sólo si $\Int{A}=A$.
            \item $\Int{\Int{A}}=\Int{A}$.
            \item $\Int{A\cap B}=\Int{A}\cap\Int{B}$.
            \item $\Int{A}\cup\Int{B}\subseteq \Int{A\cup B}$.
            \item $\Cls{A}$ es un conjunto cerrado.
            \item Si $K\subseteq X$ es cerrado de $(X,\tau)$ y $A\subseteq K$, entonces $\Cls{A}\subseteq K$.
            \item $A$ es cerrado si y sólo si $\Cls{A}=A$.
            \item $\Cls{\Cls{A}}=\Cls{A}$.
            \item $\Cls{A\cup B}=\Cls{A}\cup\Cls{B}$.
            \item $\Cls{A\cap B}\subseteq\Cls{A}\cap\Cls{B}$.
            \item $\emptyset=\Int{\emptyset}=\Cls{\emptyset}$ y $X=\Int{X}=\Cls{X}$.
            \item Si $A\subseteq B$, entonces $\Int{A}\subseteq \Int{B}$ y $\Cls{A}\subseteq\Cls{B}$.
            \item $x\in\Cls{A}$ si y sólo si para todo abierto $U\subseteq X$ tal que $x\in U$ se tiene que $U\cap A\neq\emptyset$.
            \item $x\in\Fr{A}$ si y sólo si para todo abierto $U$ tal que $x\in U$ se cumple que $U\cap A\neq\emptyset$ y $U\cap (A-X)\neq\emptyset$.
            \item $\Cls{A}=A\cup A'$.
            \item $A$ es un conjunto cerrado si y sólo si $A'\subseteq A$.
            \item $\Cls{A}=\Int{A}\cup\Fr{A}$.
            \item $\Fr{A}=\Fr{X-A}$.
            \item $\Cls{A}-\Fr{A}=\Int{A}$.
        \end{enumerate}
    \end{propo}

    \begin{proof}
        De (1):

        De (17): Sea $x\in X$.
        
        $\Rightarrow$): Suponga que $x\in\Cls{A}$, entonces para todo $C\subseteq X$ cerrado tal que $A\subseteq C$. Suponga que existe $U_0\in\tau$ abierto tal que $x\in U_0$ y $U_0\cap A=\emptyset$. Entonces $A\subseteq X-M$ es un cerrado que contiene a $A$, luego $x\in X-M$, es decir $x\notin M$\contradiction. Por tanto, $U\cap A\neq\emptyset$.

        $\Leftarrow$): Sea $L\subseteq X$ un cerrado tal que $A\subseteq L$. Probaremos que $x\in L$, suponiendo la tesis para este $x\in X$. Suponga que $x\notin L$, entonces $x\in X-L$ el cual es abierto, por tanto $(X-L)\cap A\neq\emptyset$, es decir $A\nsubseteq L$\contradiction. Por tanto, $x\in L$.
        
        De (19): Se probarán las dos contenciones:
        \renewcommand{\theenumi}{\alph{enumi})}
        \begin{enumerate}
            \item $\Cls{A}\subseteq A\cup A'$. Sea $x\in\Cls{A}$. Si $x\in A$, se tiene el resultado. Suponga que $x\notin A$. Como $x\in\Cls{A}$, por (17) para todo abierto $U\subseteq X$ se tiene que $U\cap A\neq\emptyset$, pero $x\notin A$, por lo cual $(U-\left\{x\right\})\cap A-\neq\emptyset$. Por tanto, $x\in A'$.
            \item $A\cup A'\subseteq\Cls{A}$. Es inmediata de la definición de $\Cls{A}$ y $A'$.
        \end{enumerate}
        Por a) y b) se sigue el resultado.
    \end{proof}

    \renewcommand{\theenumi}{\arabic{enumi}}

    \begin{propo}
        Sea $(X,\tau)$ un espacio topológico y $\left\{A_\alpha \right\}_{\alpha\in I}\subseteq\mathcal{P}(X)$.
        \begin{enumerate}
            \item $\bigcup_{\alpha\in I}\Cls{A_\alpha}\subseteq\Cls{\bigcup_{\alpha\in I}A_\alpha}$.
            \item $\Cls{\bigcap_{\alpha\in I}A_\alpha}\subseteq\bigcap_{\alpha\in I}\Cls{A_\alpha}$. 
        \end{enumerate}
    \end{propo}

\end{document}