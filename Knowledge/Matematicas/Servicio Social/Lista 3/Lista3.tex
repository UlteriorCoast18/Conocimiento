\documentclass[12pt]{article}
\usepackage{lingmacros}
\usepackage{tree-dvips}
\usepackage{ragged2e}
\usepackage[spanish,es-noshorthands]{babel}
\usepackage[utf8]{inputenc}
\usepackage{amssymb}
\usepackage{tikz}
\usepackage{enumerate}
\usepackage[a4paper, margin = 1.5cm]{geometry}
\usepackage{multicol}
\usetikzlibrary{scopes}
\usepackage{amsmath,amsthm}
\usepackage{amsfonts}
\usepackage{graphicx}

\begin{document}
\title{Resolución C I. Lista 3}
\author{Alvarado Cristo Daniel}
\date{Abril de 2023}
\maketitle

Los presentes ejercicios fueron diseñados para ser resueltos conforme el lector vaya comprendiendo los conceptos y resultados dados en la teoría, si se tiene alguna duda sobre alguno(s) de ellos se recomienda sea disipada de inmediato. Se sugiere al lector redactar, según su criterio, una guía que contenga aquellos conceptos y resultados del capítulo que considere más importantes y/o útiles como referencia
rápida de consulta para la solución de los problemas. UwU

%\renewcommand\qedsymbol{$\square$}%
\renewcommand{\labelenumi}{\textbf{3.\theenumi.}}
\renewcommand{\labelenumii}{\textbf{\Roman{enumii}.}}
\providecommand{\abs}[1]{\left| #1 \right|}
\def\proof{\textit{Solución:}\\}
\newcommand{\cf}[3]{\ensuremath{#1:#2\rightarrow#3}}

\begin{enumerate}
    \item Sea
    \begin{equation*}
        f(x)=\left\{
            \begin{array}{lcr}
                1 & \textup{ si } & -3\leq x<-1\\
                \abs{x} & \textup{ si } & -1\leq x<0\\
                1 & \textup{ si } & x=1/2\\
                x^2 & \textup{ si } & 1\leq x<3 
            \end{array}
        \right.
    \end{equation*}
    \begin{enumerate}
        \item ¿\textbf{Cuál} es el dominio de $f$? \textbf{Calcule}: $f(2),f(3/2),f(\sqrt{2}),f(-1/2),f(-\sqrt{2}/2),f(-2)$. \textbf{Bosqueje} la gráfica de $f$.
        \item Defina $h(x)=f(x+1)$. \textbf{Determine} el dominio de $h$. \textbf{Calcule}: $h(1),h(1/2),h(\sqrt{2}-1),h(-3/2),h(-1-\sqrt{2}/2)$ y $h(-3)$. \textbf{Bosqueje} la gráfica de $h$. ¿Existe alguna relación entre la gráfica de $f$ y la gráfica de $h$? \textbf{Explique}.
        \item Defina $k(x)=f(x)+1$. \textbf{Determine} el dominio de $k$. \textbf{Calcule} $k(2),k(3/2),k(\sqrt{2}),k(-1/2),k(-\sqrt{2}/2)$ y $k(-2)$. \textbf{Bosqueje} la gráfica de $j$. ¿Existe alguna relación entre la gráfica de $f$ y la gráfica de $k$? \textbf{Explique}.
    \end{enumerate}

    \item \textbf{Analice} la variación de las siguientes funciones (dominio natural, raíces, intervalos de monotonía, comportamiento en los extremos de dichos intervalos, cuadro de variación y gráfica):
    \begin{enumerate}
        \item $f(x)=x^2+3x$.
        \item $g(x)=\frac{x-1}{2x+2}$.
        \item $h(x)=\abs{x}$.
    \end{enumerate}
    \item \textbf{Determine} el dominio natrual de las siguientes funciones:
    \begin{enumerate}
        \item $x\mapsto \sqrt{3-x^2}$.
        \item $y\mapsto\sqrt{1-\sqrt{1-y^2}}$.
        \item $\omega\mapsto\frac{1}{\omega-1}+\frac{1}{\omega-2}$.
        \item $u\mapsto\sqrt{1-u^2}+\sqrt{u^2-1}$.
        \item $t\mapsto\sqrt{1-t}+\sqrt{t-2}$.
    \end{enumerate}
    \item \begin{enumerate}
        \item \textbf{Muestre} que si $\abs{x-1}\leq 1$, entonces $\abs{x^2+3x-4}\leq 6\abs{x-1}$.
        \item Sea $\varepsilon>0$. Use el inciso anterior para \textbf{probar} que si $\abs{x-1}<\min\left\{1,\frac{\varepsilon}{6} \right\}$, entonces $\abs{x^2+3x-4}\leq\varepsilon$. Aplique la definición de límite para \textbf{concluir} que
        \begin{equation*}
            \lim_{x\rightarrow1}x^2+3x=4
        \end{equation*}
    \end{enumerate} 
    \item Usando la definición de límite, \textbf{demuestre} las afirmaciones siguientes.
    \begin{enumerate}
        \item $\lim_{x\rightarrow-1}\abs{x^3}=1$.
        \item $\lim_{x\rightarrow1}f(x)=1$, donde $f(x)=\left\{\begin{array}{lcr}
            \sqrt{x} & \textup{ si } & 0\leq x<1\\
            x^2 & \textup{ si } & 1 <x\leq 2 \\
        \end{array} \right.$
        \item ¿Existe $\lim_{x\rightarrow0}f(x)$, donde $f(x)=\left\{\begin{array}{lcr}
            x^3-2x^2+x & \textup{ si } & x\neq0 \\
            1 & \textup{ si } & x=0 \\
        \end{array} \right.$? \textbf{Justifique}.
    \end{enumerate}
    \item Usando primero la definición de límite, luego algunos teoremas sobre límites y finalmente la caracterización de límites con sucesiones, \textbf{determine} los límites siguientes.
    \begin{enumerate}
        \item $\lim_{t\rightarrow-4}\frac{t^2-t-20}{t+4}$.
        \item $\lim_{y\rightarrow3}\frac{y^2-9}{y^2-2y-3}$.
        \item $\lim_{x\rightarrow1}\left(\frac{1}{1-x}-\frac{x^3-x^2-2}{x^2-1}\right)$.
        \item $\lim_{x\rightarrow0}\frac{1}{3x}\left(\frac{1}{8+x}-\frac{1}{8}\right)$.
        \item $\lim_{x\rightarrow0}\frac{1}{x^2}\left(\frac{2}{x-5}-\frac{2}{x^2+x-5} \right)$.
    \end{enumerate}

    \item Suponga que no existen los límites $\lim{x\rightarrow a}f(x)$ y $\lim{x\rightarrow a}g(x)$. ¿Pueden existir $\lim{x\rightarrow a}(f(x)+g(x))$ ó $\lim{x\rightarrow a}f(x)g(x)$? \textbf{Justifique} formalmente sus respuestas o dando contraejemplos.
    \item \begin{enumerate}
        \item \textbf{Demuestre} que si exiten los límites $\lim{x\rightarrow a}(f(x)+g(x))$ y $\lim{x\rightarrow a}f(x)$, entonces también existe $\lim{x\rightarrow a}g(x)$.
        \item \textbf{Pruebe} que si existen los límites $\lim{x\rightarrow a}f(x)g(x)$ y, además, $\lim{x\rightarrow a}f(x)\neq 0$, entonces también existe $\lim{x\rightarrow a}g(x)$.
    \end{enumerate}
    \item Suponga que exista el límite $\lim{x\rightarrow a}f(x)g(x)$ y, además, $\lim{x\rightarrow a}f(x)=0$. ¿Puede existir $\lim{x\rightarrow a}g(x)$? \textbf{Justifique} formalmente sus respuestas o dando contraejemplos.
    \item \begin{enumerate}
        \item \textbf{Pruebe} que si $\abs{x-2}\leq1$, entonces $\abs{x^2+3x-1}\geq1$.
        \item Sea $\delta>0$. Use el inciso anterior para \textbf{probar} que si $x=\min\left\{5/2,2+\delta/2 \right\}$, entonces $\abs{x^2+3x-1}\geq1$. Aplique la definición de límite para \textbf{concluir} que $\lim_{x\rightarrow2}x^2+3x\neq1$.
    \end{enumerate}
    \item Considere las funciones $j(x)=x$, $s(x)=x^2$ y $h(x)=\sqrt{\abs{x}}$, para todo $x\in\mathbb{R}$.
    \begin{enumerate}
        \item \textbf{Determine} el conjunto de todos los $x\in\mathbb{R}$ para los que se cumpla que $s(x)\leq j(x)$ y \textbf{haga} un dubujo en donde aparezcan simultáneamente las gráficas de $s$ y $j$.
        \item \textbf{Determine} el conjunto de todos los $x\in\mathbb{R}$ para los que se cumpla que $h(x)\leq s(x)$ y \textbf{haga} un dubujo en donde aparezcan simultáneamente las gráficas de $h$ y $s$.
        \item \textbf{Determine} el conjunto de todos los $x\in\mathbb{R}$ para los que se cumpla que $j(x)\leq h(x)$ y \textbf{haga} un dubujo en donde aparezcan simultáneamente las gráficas de $j$ y $h$.
    \end{enumerate}
    \item Sea $S\subseteq\mathbb{R}$. Dadas dos funciones $\cf{f,g}{S}\mathbb{R}$ se define la \textbf{envoltura superior} de $f$ y $g$, como la función $\cf{\max\left(f,g\right)}{S}{\mathbb{R}}$ dada por
    \begin{equation*}
        \max\left(f,g\right)(x)=\max\left(f(x),g(x)\right),\quad\forall x\in S
    \end{equation*}
    y, la \textbf{envoltura inferior} de $f$y $g$ como la función $\cf{\min\left(f,g\right)}{S}{\mathbb{R}}$ dada por
    \begin{equation*}
        \min\left(f,g\right)(x)=\min\left(f(x),g(x)\right),\quad\forall x\in S
    \end{equation*}
    \begin{enumerate}
        \item Reconsidere las funciones $\cf{j,s,h}{\mathbb{R}}{\mathbb{R}}$ del problema anterior. \textbf{Bosqueje} la gráfica de las funciones $\max(j,s),\min(j,s),\max(s,h),\min(s,h),\max(j,h),\min(j,h)$.
        \item \textbf{Escriba} las funciones $\max(f,g)$ y $\min(f,g)$ en términos de $f$ y $g$ y del valor absoluto.
    \end{enumerate}
    \item Aplique el teorema de comparación y/o el tereoma de álgebra de límites para \textbf{calcular} los límites siguientes.
    \begin{enumerate}
        \item $\lim_{x\rightarrow0}\frac{\tan^2x}{x}$.
        \item $\lim_{x\rightarrow a}\left[\frac{\sen x-\sen a}{x-a} \right]$, donde $a\in\mathbb{R}$.
        \item $\lim_{x\rightarrow0}\sqrt{\abs{x}}\sen\left(\frac{1}{x}\right)$.
        \item Sean $\cf{f,g}{S}{\mathbb{R}}$ dos funciones y $a\in\mathbb{R}$. Suponga que $g$ es acotada en $S$ y que $\lim_{x\rightarrow a}f(x)=0$. \textbf{Demuestre} que
        \begin{equation*}
            \lim_{x\rightarrow a}f(x)g(x)=0
        \end{equation*}
    \end{enumerate}
    \item Use el teorema sobre la caracterización de límites de funciones por medio de sucesiones en los problemas siguientes.
    \begin{enumerate}
        \item \textbf{Calcule} $\lim_{x\rightarrow-2}\sqrt[3]{\frac{x-1}{2x+2}}$.
        \item \textbf{Calcule} $\lim_{x\rightarrow-3}\sqrt[3]{\abs{x}^3}$.
        \item \textbf{Muestre} que no existe $\lim_{x\rightarrow0}\sen\left(\frac{1}{x}\right)$.
        \item \textbf{Muestre} que no existe $\lim_{x\rightarrow2}E(x)$, donde $E$ es la función parte entera.
        \item \textbf{Muestre} que no existe $\lim_{x\rightarrow2}f(x)$, donde $f(x)=\left\{\begin{array}{lcr}
            \sqrt{x} & \textup{ si } & 0\leq x<2\\
            x^3 & \textup{ si } & 2<x\leq3 \\
        \end{array} \right.$.
        \item \textbf{Muestre} que no existe $\lim_{x\rightarrow a}f(x)$, donde $f$ es la función de Dirichlet y $a\in\mathbb{R}$ es arbitrario.
    \end{enumerate}
    \item Usando primero la definición de límite y después el teorema sobre caracterizació nde límites de funciones por medio de sucesiones, \textbf{pruebe} que:
    \begin{enumerate}
        \item $\lim_{x\rightarrow1}\frac{x-1}{2x+2}\neq 3$.
        \item $\lim_{x\rightarrow 2}\abs{x}\neq -1$.
    \end{enumerate}
    \item Considere la función
    \begin{equation*}
        f(x)=\left\{
            \begin{array}{lcr}
                -x & \textup{ si } & -3\leq x\leq-1\\
                x^3 & \textup{ si } & -1< x<1\\
                \sqrt{x} & \textup{ si } & 1<x\leq3 \\
                \frac{1}{3-x}+\sqrt{3} & \textup{ si } & 3<x\\ 
            \end{array}
        \right.
    \end{equation*}
    \begin{enumerate}
        \item \textbf{Bosqueje} la gráfica de $f$.
        \item \textbf{Calcule} $\lim_{x \rightarrow-1^-}f(x)$ y $\lim_{x \rightarrow-1^+}f(x)$ ¿Existe $\lim_{x \rightarrow-1}f(x)$?
        \item \item \textbf{Calcule} $\lim_{x \rightarrow1^-}f(x)$ y $\lim_{x \rightarrow1^+}f(x)$ ¿Existe $\lim_{x \rightarrow1}f(x)$?
        \item ¿Existen $\lim_{x \rightarrow3^-}f(x)$, $\lim_{x \rightarrow3^+}f(x)$ y $\lim_{x \rightarrow3}f(x)$?
        \item \textbf{Calcule} $\lim_{x \rightarrow-3}f(x)$ y $\lim_{x \rightarrow\infty}f(x)$.
        \item Si $a\in[-3,\infty[\backslash\left\{-3,-1,1,3 \right\}$. \textbf{Calcule} $\lim_{ x\rightarrow a}f(x)$, $\lim_{ x\rightarrow a^+}f(x)$ y $\lim_{ x\rightarrow a^-}f(x)$.
    \end{enumerate}
    \textbf{Justifique} usando la definición del límite correspondiente y también usando la respectiva caracterización de sucesiones.

    \item \textbf{Calcule}, justificando por medio de la definición del límite correspondiente y de la respectiva caracterización de sucesiones, los siguientes límites.
    \begin{enumerate}
        \item $\lim_{x\rightarrow\infty}x^2+3$, $\lim_{x\rightarrow\infty}-x^2-3$ y $\lim_{x\rightarrow\infty}[(x^2+3)-(-x^2-5)]$.
        \item \item $\lim_{x\rightarrow\infty}3x^2-x+5$, $\lim_{x\rightarrow\infty}x-3$ y $\lim_{x\rightarrow\infty}[(3x^2-x+5)+(x-3)]$.
        \item $\lim_{x\rightarrow3^-}\frac{5x-2}{x-3}$, $\lim_{x\rightarrow3^+}\frac{5x-2}{x-3}$ y $\lim_{x\rightarrow3}\abs{\frac{5x-2}{x-3}}$.
    \end{enumerate}
    \item \textbf{Calcule}
    \begin{equation*}
        \lim_{ x\rightarrow\infty}\frac{a_nx^n+\cdots+a_1x+a_0}{bm_x^m+\cdots+b_1x+b_0}
    \end{equation*}
    donde $a_n,b_m\neq0$, distinguiendo los casos $m=n$, $m>n$ y $m<n$. En particular, \textbf{calcule}.
    \begin{equation*}
        \lim_{ x\rightarrow\infty}\frac{x^2-3}{x-1}\quad\lim_{ x\rightarrow\infty}\frac{2-x}{x^4-1}\quad\lim_{ x\rightarrow\infty}\frac{5x+2}{x-3}
    \end{equation*}
    \item Sea $E(x)=\max\left\{n\in\mathbb{Z}\Big|n\leq x \right\}$, para todo $x\in\mathbb{R}$, la función \textbf{parte entera} de $x$. Considere las funciones siguientes:
    \begin{enumerate}
        \item $f(x)=E(x)$.
        \item $f(x)=-[x-E(x)]$.
        \item $f(x)=\sqrt{x-E(x)}$.
        \item $f(x)=E(1/x)$.
        \item $f(x)=\frac{1}{E(1/x)}$.
        \item $f(x)=E(x)+\sqrt{x-E(x)}$.
        \item $f(x)=E(x)+\sqrt{x-E(x)}$.
    \end{enumerate}
    \textbf{Determine} el dominio natrual de cada una de estas funciones y \textbf{bosqueje} su gráfica. Si existen, cálcule los siguientes límites (justificando formalmente) para cada una de las funciones anteriores:
    \begin{equation*}
        \lim_{ x\rightarrow a^-}f(x)\quad\lim_{ x\rightarrow a^+}f(x)\quad\lim_{ x\rightarrow a}f(x)
    \end{equation*}
    donde $a\in\mathbb{R}$, y
    \begin{equation*}
        \lim_{ x\rightarrow-\infty}f(x)\quad\lim_{ x\rightarrow\infty}f(x)
    \end{equation*}
    \item Sea $S\subseteq\mathbb{R}$. Sea $\cf{f}{S}{\mathbb{R}}$ una función y, $t,l\in\mathbb{R}$. \textbf{Pruebe} que:
    \begin{enumerate}
        \item $\lim_{ x\rightarrow t}f(x)=l$ si y sólo si $\lim_{ x\rightarrow t}f(x)-l=0$ si y sólo si $\lim_{ x\rightarrow t}\abs{f(x)-l}=0$.
        \item $\lim_{ x\rightarrow t}f(x)=\lim_{ h\rightarrow 0}f(t+h)$.
    \end{enumerate}
    \item \textbf{Demuestre} que si dos funciones $f$ y $g$ toman los mismos valores en todos los puntos de algún intervalo abierto que contenga a $a$, exceptuando posiblemente a $a$, entonces
    \begin{equation*}
        \lim_{ x\rightarrow a}f(x)=\lim_{ x\rightarrow a}g(x)
    \end{equation*}
    cuando alguno de los dos límites exista. Esto significa que la existencia del límite de alguna función en un punto dado es una \textbf{propiedad local}.

    \item Sea $S\subseteq\mathbb{R}$. Sean $\cf{f,g}{S}{\mathbb{R}}$ dos funciones y $a\in\mathbb{R}$. Si $f(x)\leq g(x)$, para todo $x\in S$, y si existen $\lim_{ x\rightarrow a}f(x)$ y $\lim_{ x\rightarrow a}g(x)$, pruebe que
    \begin{equation*}
        \lim_{ x\rightarrow a}f(x)\leq\lim_{ x\rightarrow a}g(x)
    \end{equation*}
    \item Sea $S\subseteq\mathbb{R}$. Sean $\cf{f,g,h}{S}{\mathbb{R}}$ tres funciones. Dije $a\in\mathbb{R}\cup\left\{-\infty,\infty \right\}$. Si $f(x)\leq g(x)\leq h(x)$, para todo $x\in S$ y $\lim_{ x\rightarrow a}f(x)=\lim_{ x\rightarrow a}h(x)$, \textbf{demuestre} que existe $\lim_{ x\rightarrow a}g(x)$ y
    \begin{equation*}
        \lim_{ x\rightarrow a}f(x)=\lim_{ x\rightarrow a}g(x)=\lim_{ x\rightarrow a}h(x)
    \end{equation*}
    \item Sea $S\subseteq\mathbb{R}$. Si $\cf{f}{S}{\mathbb{R}}$ es una función, defina la función $\cf{\abs{f}}{S}{\mathbb{R}}$ como $\abs{f}(x)=\abs{f(x)}$, para todo $x\in S$. \textbf{Pruebe} que si $\lim_{ x\rightarrow a}f(x)=l$, entonces $\lim_{ x\rightarrow a}\abs{f}(x)=\abs{l}$.
    \item \textbf{Pruebe} que si $\lim_{ x\rightarrow a}f(x)=l$ y $\lim_{ x\rightarrow a}g(x)=m$, entonces
    \begin{equation*}
        \lim_{ x\rightarrow a}\max(f,g)(x)=\max(l,m)\quad\textup{y}\quad\lim_{ x\rightarrow a}\min(f,g)(x)=\min(l,m)
    \end{equation*}
    \textit{Sugerencia.} Utilice un resultado de un problema anterior.
    \item \textbf{Determine (justificando)} el dominio nautral y el conjunto de puntos de continuidad de las siguientes funciones:
    \begin{enumerate}
        \item $P(x)=a_nx^n+a_{ n-1}x^{ n-1}+\cdots+a_1x+a_0$, donde $n\in\mathbb{N}$, $a_i\in\mathbb{R}$, para todo $i=0,1,...,n$.
        \item $R(x)=\frac{P(x)}{Q(x)}$ ,donde $P$ y $Q$ son dos polinomios.
        \item $f(x)=x^a$, donde $a\in\mathbb{Q}$.
        \item $\mathcal{N}(x)=\abs{x}$.
    \end{enumerate}
\end{enumerate}
\end{document}