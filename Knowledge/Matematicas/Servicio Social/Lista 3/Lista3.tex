\documentclass[12pt]{article}
\usepackage{lingmacros}
\usepackage{tree-dvips}
\usepackage{ragged2e}
\usepackage[spanish,es-noshorthands]{babel}
\usepackage[utf8]{inputenc}
\usepackage{amssymb}
\usepackage{tikz}
\usepackage{enumerate}
\usepackage[a4paper, margin = 1.5cm]{geometry}
\usepackage{multicol}
\usetikzlibrary{scopes}
\usepackage{amsmath,amsthm}
\usepackage{amsfonts}
\usepackage{graphicx}

\begin{document}
\title{Resolución C I. Lista 3}
\author{Alvarado Cristo Daniel}
\date{Abril de 2023}
\maketitle

Los presentes ejercicios fueron diseñados para ser resueltos conforme el lector vaya comprendiendo los conceptos y resultados dados en la teoría, si se tiene alguna duda sobre alguno(s) de ellos se recomienda sea disipada de inmediato. Se sugiere al lector redactar, según su criterio, una guía que contenga aquellos conceptos y resultados del capítulo que considere más importantes y/o útiles como referencia
rápida de consulta para la solución de los problemas.

%\renewcommand\qedsymbol{$\square$}%

\def\proof{\paragraph{Demostración:\\}}
\def\endproof{\hfill$\blacksquare$}

\def\sol{\paragraph{Solución:\\}}
\def\endsol{\hfill$\square$}

\renewcommand{\labelenumi}{\textbf{3.\theenumi.}}
\renewcommand{\labelenumii}{\textbf{\Roman{enumii}.}}
\providecommand{\abs}[1]{\left| #1 \right|}
\newcommand{\cf}[3]{\ensuremath{#1:#2\rightarrow#3}}

\begin{enumerate}
    \item Sea
    \begin{equation*}
        f(x)=\left\{
            \begin{array}{lcr}
                1 & \textup{ si } & -3\leq x<-1\\
                \abs{x} & \textup{ si } & -1\leq x<0\\
                1 & \textup{ si } & x=1/2\\
                x^2 & \textup{ si } & 1\leq x<3 
            \end{array}
        \right.
    \end{equation*}
    \begin{enumerate}
        \item ¿\textbf{Cuál} es el dominio de $f$? \textbf{Calcule}: $f(2),f(3/2),f(\sqrt{2}),f(-1/2),f(-\sqrt{2}/2),f(-2)$. \textbf{Bosqueje} la gráfica de $f$.
        \item Defina $h(x)=f(x+1)$. \textbf{Determine} el dominio de $h$. \textbf{Calcule}: $h(1),h(1/2),h(\sqrt{2}-1),h(-3/2),h(-1-\sqrt{2}/2)$ y $h(-3)$. \textbf{Bosqueje} la gráfica de $h$. ¿Existe alguna relación entre la gráfica de $f$ y la gráfica de $h$? \textbf{Explique}.
        \item Defina $k(x)=f(x)+1$. \textbf{Determine} el dominio de $k$. \textbf{Calcule} $k(2),k(3/2),k(\sqrt{2}),k(-1/2)$ ,$k(-\sqrt{2}/2)$ y $k(-2)$. \textbf{Bosqueje} la gráfica de $k$. ¿Existe alguna relación entre la gráfica de $f$ y la gráfica de $k$? \textbf{Explique}.
    \end{enumerate}

    \begin{sol}
        De (i): Los posibles valores que $f$ toma son cuando $-3\leq x <-1$, $-1\leq x<0$, $x=\frac{1}{2}$ o $1\leq x<3$, esto es, el dominio de $f$ es el conjunto:
        \begin{equation*}
            D_f=[-3,0[\cup\left\{\frac{1}{2}\right\}\cup[1,3[
        \end{equation*}
        y, se tiene que:
        \begin{itemize}
            \item $f(2)=2^2=4$.
            \item $f(3/2)=\left(3/2\right)^2=9/4$.
            \item $f(\sqrt{2})=\left(\sqrt{2} \right)^2=2$.
            \item $f(-1/2)=\abs{-1/2}=1/2$.
            \item $f(-\sqrt{2}/2)=1$.
            \item $f(-2)=1$.
        \end{itemize}
        La gráfica de $f$ está dada como se muestra en la figura 1:
        \begin{figure}
            \begin{center}
                \includegraphics[scale=1]{images/3_1.pdf}
            \end{center}
            \caption{Plot de la función $f$.}
        \end{figure}

        De (ii): El dominio de $h$ son los puntos $x\in\mathbb{R}$ para los cuales $f(x+1)$ está definido, es decir que $x+1\in D_f$, por tanto el dominio de $h$ es el conjunto
        \begin{equation*}
            D_h=[-4,-1[\cup\left\{-\frac{1}{2}\right\}\cup[0,2[
        \end{equation*}
        y, se tiene que
        \begin{itemize}
            \item $h(1)=f(1+1)=2^2=4$.
            \item $h(1/2)=f(1/2+1)=f(3/2)=\left(3/2\right)^2=9/4$.
            \item $h(\sqrt{2}-1)=f(\sqrt{2})=\left(\sqrt{2} \right)^2=2$.
            \item $h(-3/2)=f(-1/2)=\abs{-1/2}=1/2$.
            \item $h(-1-\sqrt{2}/2)=f(-\sqrt{2}/2)=1$.
            \item $h(-3)=f(-2)=1$.
        \end{itemize}
        La gráfica de $h$ está dada como se muestra en la figura 2.

        \begin{figure}
            \begin{center}
                \includegraphics[scale=1]{images/3_1_2.pdf}
            \end{center}
            \caption{Plot de la función $h$.}
        \end{figure}
        
        donde, podemos observar que lo que se hace con respecto a la gráfica de $f$, es recorrer la gráfica en el sentido horizontal una unidad hacia la izquierda.

        De (iii): El dominio de $k$ son los posibles valores para los que $f(x)+1$ está definido, es decir para cuando $f$ está definido, por lo cual
        \begin{equation*}
            D_k=D_f=[-3,0[\cup\left\{\frac{1}{2}\right\}\cup[1,3[
        \end{equation*}
        y, se tiene que
        \begin{itemize}
            \item $k(2)=f(2)+1=2^2+1=5$.
            \item $k(3/2)=f(3/2)+1=\left(3/2\right)^2+1=13/4$.
            \item $k(\sqrt{2})=f(\sqrt{2})+1=\left(\sqrt{2} \right)^2+1=3$.
            \item $k(-1/2)=f(-1/2)+1=\abs{-1/2}+1=3/2$.
            \item $k(-\sqrt{2}/2)+1=f(-\sqrt{2}/2)+1=2$.
            \item $k(-2)=f(-2)+1=2$.
        \end{itemize}
        La gráfica de $k$ está dada como se muestra en la figura 3:
        \begin{figure}
            \begin{center}
                \includegraphics[scale=1]{images/3_1_3.pdf}
            \end{center}
            \caption{Plot de la función $k$.}
        \end{figure}

        donde, podemos observar que lo que se hace con respecto a la gráfica de $f$, es recorer la gráfica verticalmente una unidad hacia arriba.
    \end{sol}

    \item \textbf{Analice} la variación de las siguientes funciones (dominio natural, raíces, intervalos de monotonía, comportamiento en los extremos de dichos intervalos, cuadro de variación y gráfica):
    \begin{enumerate}
        \item $f(x)=x^2+3x$.
        \item $g(x)=\frac{x-1}{2x+2}$.
        \item $h(x)=\abs{x}$.
    \end{enumerate}

    \begin{sol}
        %TODO
        De (i): La gráfica de la función $f$ es la de la mostrada en la figura 4:
        \begin{figure}
            \begin{center}
                \includegraphics[scale=1]{images/3_2_1.pdf}
            \end{center}
            \caption{Plot de la función $f$.}
        \end{figure}

        De (ii): La gráfica de la función $g$ es la de la mostrada en la figura 5:
        \begin{figure}
            \begin{center}
                \includegraphics[scale=1]{images/3_2_2.pdf}
            \end{center}
            \caption{Plot de la función $g$.}
        \end{figure}

        De (iii): La gráfica de la función $h$ es la de la mostrada en la figura 6:
        \begin{figure}
            \begin{center}
                \includegraphics[scale=1]{images/3_2_3.pdf}
            \end{center}
            \caption{Plot de la función $h$.}
        \end{figure}

    \end{sol}

    \item \textbf{Determine} el dominio natural de las siguientes funciones:
    \begin{enumerate}
        \item $x\mapsto \sqrt{3-x^2}$.
        \item $y\mapsto\sqrt{1-\sqrt{1-y^2}}$.
        \item $\omega\mapsto\frac{1}{\omega-1}+\frac{1}{\omega-2}$.
        \item $u\mapsto\sqrt{1-u^2}+\sqrt{u^2-1}$.
        \item $t\mapsto\sqrt{1-t}+\sqrt{t-2}$.
    \end{enumerate}

    \begin{sol}
        De (i): La función $x\mapsto\sqrt{3-x^2}$ está definida si y sólo si $3-x^2\geq0$, esto es $x^2\leq3$ lo cual ocurre si y sólo si $-\sqrt{3}\leq x\leq\sqrt{3}$.

        Así, el dominio natural de esta función es $[-\sqrt{3},\sqrt{3}]$.

        De (ii): La función $y\mapsto\sqrt{1-\sqrt{1-y^2}}$ está definida si y sólo si $1-\sqrt{1-y^2}\geq0$, es decir si y sólo si $\sqrt{1-y^2}\leq1$, esto es cuando $0\leq1-y^2\leq 1$ y,
        \begin{equation*}
            \begin{split}
                0\leq1-y^2\leq 1&\iff 0\leq y^2\leq1\\
                &\iff -1\leq y\leq1\\
            \end{split}
        \end{equation*}
        por tanto, el dominio natural de esta función es $[-1,1]$.

        De (iii): La función $\omega\mapsto\frac{1}{\omega-1}+\frac{1}{\omega-2}$ está definida cuando $\omega-1\neq 0$ y $\omega-2\neq0$, esto es:
        \begin{equation*}
            \omega\neq1,2
        \end{equation*}
        luego, el dominio natural de esta función es $\mathbb{R}\backslash\left\{1,2 \right\}$.

        De (iv): La función $u\mapsto\sqrt{1-u^2}+\sqrt{u^2-1}$ está definida si y sólo si $1-u^2,u^2-1\geq0$, esto es:
        \begin{equation*}
            1\leq u^2\quad\textup{y}\quad u^2\geq1
        \end{equation*}
        y, esto sólo ocurre cuando $u=\pm1$. Por tanto, el dominio natrual de $f$ es $\left\{-1,1 \right\}$.

        De (v): La función $t\mapsto\sqrt{1-t}+\sqrt{t-2}$ está definida cuando $1-t\geq 0$ y $t-2\geq 0$, es decir que $t\leq 1$ y $2\leq t$, pero esto no puede suceder nunca. Por tanto, el dominio nautural de esta función es $\emptyset$.
    \end{sol}

    \item \begin{enumerate}
        \item \textbf{Muestre} que si $\abs{x-1}\leq 1$, entonces $\abs{x^2+3x-4}\leq 6\abs{x-1}$.
        \item Sea $\varepsilon>0$. Use el inciso anterior para \textbf{probar} que si $\abs{x-1}<\min\left\{1,\frac{\varepsilon}{6} \right\}$, entonces $\abs{x^2+3x-4}\leq\varepsilon$. Aplique la definición de límite para \textbf{concluir} que
        \begin{equation*}
            \lim_{x\rightarrow1}x^2+3x=4
        \end{equation*}
    \end{enumerate}

    \begin{proof}
        De (i): Sea $x\in\mathbb{R}$ tal que $\abs{x-1}\leq 1$, entonces $\abs{x}-1\leq1\Rightarrow \abs{x}\leq2$. Con esto, se sigue que:
        \begin{equation*}
            \begin{split}
                \abs{x^2+3x-4}&=\abs{(x+4)(x-1)}\\
                &=\abs{x-1}\abs{x+4}\\
                &\leq\abs{x-1}\left(\abs{x}+4\right) \\
                &\leq\abs{x-1}\left(2+4\right)\\
                &\leq6 \abs{x-1}\\
                \Rightarrow \abs{x^2+3x-4}&\leq6 \abs{x-1}\\ 
            \end{split}
        \end{equation*}

        De (ii): Sea $x\in\mathbb{R}$ tal que $\abs{x-1}<\min\left\{1,\frac{\varepsilon}{6} \right\}$, es decir que
        \begin{equation*}
            \abs{x-1}<1\quad\textup{y}\quad\abs{x-1}<\frac{\varepsilon}{6}
        \end{equation*}
        por la parte (i), se sigue que $\abs{x^2-3x-4}\leq 6\abs{x-1}$ y, por la segunda desigualdad, se sigue que
        \begin{equation*}
            \abs{x^2-3x-4}\leq 6\abs{x-1}<6\cdot\frac{\varepsilon}{6}=\varepsilon
        \end{equation*}
        por tanto, $\abs{x^2-3x-4}<\varepsilon$. Ahora, como para todo $\varepsilon>0$ existe $\delta=\min\left\{1,\frac{\varepsilon}{6} \right\}>0$ tal que
        \begin{equation*}
            \forall x\in\mathbb{R},x\neq1\textup{ tal que } \abs{x-1}<\delta \Rightarrow \abs{x^2-3x-4}<\varepsilon
        \end{equation*}
        se sigue que
        \begin{equation*}
            \lim_{x\rightarrow 1}x^2-3x=4
        \end{equation*}
    \end{proof}

    \item Usando la definición de límite, \textbf{demuestre} las afirmaciones siguientes.
    \begin{enumerate}
        \item $\lim_{x\rightarrow-1}\abs{x^3}=1$.
        \item $\lim_{x\rightarrow1}f(x)=1$, donde $f(x)=\left\{\begin{array}{lcr}
            \sqrt{x} & \textup{ si } & 0\leq x<1\\
            x^2 & \textup{ si } & 1 <x\leq 2 \\
        \end{array} \right.$
        \item ¿Existe $\lim_{x\rightarrow0}f(x)$, donde $f(x)=\left\{\begin{array}{lcr}
            x^3-2x^2+x & \textup{ si } & x\neq0 \\
            1 & \textup{ si } & x=0 \\
        \end{array} \right.$? \textbf{Justifique}.
    \end{enumerate}

    \begin{proof}
        De (i): Sea $\varepsilon>0$. Observemos que si $\abs{x+1}\leq1$:
        \begin{equation*}
            \begin{split}
                \abs{\abs{x^3}-1}&\leq\abs{-x^3-1}\\
                &\leq\abs{(x+1)(x^2-x+1)}\\
                &\leq\abs{x+1}\abs{x^2-x+1}\\
            \end{split}
        \end{equation*}
        entonces, como $\abs{x}-\abs{-1}\leq\abs{x-(-1)}$, se sigue que $\abs{x}\leq2$. Luego
        \begin{equation*}
            \begin{split}
                \abs{x^2+x+1}&\leq\abs{x^2}+\abs{x}+1\\
                &\leq 2^2+2+1\\
                &=7\\
                \Rightarrow \abs{\abs{x^3}-1}&\leq7\abs{x+1}
            \end{split}
        \end{equation*}
        tomemos entonces $\delta=\min\left\{1,\frac{\varepsilon}{7}\right\}>0$. Se tiene entonces que si $\abs{x+1}<\delta$, por lo anterior, que 
        \begin{equation*}
            \abs{\abs{x^3}-1}\leq7\abs{x+1}
        \end{equation*}
        además, $\abs{x+1}<\frac{\varepsilon}{7}$, por lo cual
        \begin{equation*}
            \begin{split}
                \abs{\abs{x^3}-1}&<7\cdot\frac{\varepsilon}{7}\\
                &=\varepsilon\\
                \Rightarrow \abs{\abs{x^3}-1}&<\varepsilon \\
            \end{split}
        \end{equation*}
        es decir:
        \begin{equation*}
            \therefore \forall x\in\mathbb{R}\textup{ tal que }x\neq-1\textup{ con } \abs{x-(-1)}<\delta\Rightarrow \abs{\abs{x^3}-1}<\varepsilon
        \end{equation*}
        Luego, de la definición de límite por $\varepsilon-\delta$ se sigue que
        \begin{equation*}
            \lim_{ x\rightarrow-1}\abs{x^3}=1
        \end{equation*}

        De (ii): 
        %TODO
    \end{proof}

    \item Usando primero la definición de límite, luego algunos teoremas sobre límites y finalmente la caracterización de límites con sucesiones, \textbf{determine} los límites siguientes.
    \begin{enumerate}
        \item $\lim_{t\rightarrow-4}\frac{t^2-t-20}{t+4}$.
        \item $\lim_{y\rightarrow3}\frac{y^2-9}{y^2-2y-3}$.
        \item $\lim_{x\rightarrow1}\left(\frac{1}{1-x}-\frac{x^3-x^2-2}{x^2-1}\right)$.
        \item $\lim_{x\rightarrow0}\frac{1}{3x}\left(\frac{1}{8+x}-\frac{1}{8}\right)$.
        \item $\lim_{x\rightarrow0}\frac{1}{x^2}\left(\frac{2}{x-5}-\frac{2}{x^2+x-5} \right)$.
    \end{enumerate}

    \begin{proof}
        %TODO
        De (i): 
        De (ii): 
        De (iii): 
        De (vi): 
        De (v):  
    \end{proof}

    \item Suponga que no existen los límites $\lim_{x\rightarrow a}f(x)$ y $\lim_{x\rightarrow a}g(x)$. ¿Pueden existir $\lim_{x\rightarrow a}(f(x)+g(x))$ o $\lim_{x\rightarrow a}f(x)g(x)$? \textbf{Justifique} formalmente sus respuestas o dando contraejemplos.
    \begin{sol}
        \begin{enumerate}
            \item Analicemos primero el límite de la suma. Considere las funciones $f(x)=\frac{1}{x}$ y $g(x)=-\frac{1}{x}$, para todo $x\in\mathbb{R}\backslash\left\{0 \right\}$. Se tiene que los límites
            \begin{equation*}
                \lim_{x\rightarrow 0}f(x)\textup{ y }\lim_{x\rightarrow 0}g(x)
            \end{equation*}
            no existen, sin embargo
            \begin{equation*}
                f(x)+g(x)=\frac{1}{x}-\frac{1}{x}=0\quad\forall x\in\mathbb{R}\backslash\left\{0 \right\}
            \end{equation*}
            por tanto,
            \begin{equation*}
                \begin{split}
                    \lim_{x\rightarrow0 }(f(x)+g(x))&=\lim_{x\rightarrow0 }0\\
                    &=0\\
                \end{split}
            \end{equation*}
            es decir, el límite de la suma si existe.

            \item Analicemos ahora el límite del producto. Considere las funciones
            \begin{equation*}
                f(x)=g(x)=\left\{
                    \begin{array}{lcr}
                        1 & \textup{ si } & x\in\mathbb{Q}\\
                        -1 & \textup{ si } & x\in\mathbb{R}\backslash\mathbb{Q} \\
                    \end{array}
                \right.,\quad\forall x\in\mathbb{R}
            \end{equation*}
            se tiene que no existen los límites de $f$ y $g$ cuando $x\rightarrow 0$, sin embargo:
            \begin{equation*}
                f(x)g(x)=1,\quad\forall x\in\mathbb{R}
            \end{equation*}
            por lo cual, $\lim_{x\rightarrow 0}f(x)g(x)=1$. Es decir, el límite del producto si existe.
        \end{enumerate}
    \end{sol}

    \item \begin{enumerate}
        \item \textbf{Demuestre} que si exiten los límites $\lim_{x\rightarrow a}(f(x)+g(x))$ y $\lim_{x\rightarrow a}f(x)$, entonces también existe $\lim_{x\rightarrow a}g(x)$.
        \item \textbf{Pruebe} que si existen los límites $\lim_{x\rightarrow a}f(x)g(x)$ y, además, $\lim_{x\rightarrow a}f(x)\neq 0$, entonces también existe $\lim_{x\rightarrow a}g(x)$.
    \end{enumerate}

    \begin{proof}
        De (i): Supongamos que $\cf{f,g}{S\subseteq\mathbb{R}}{\mathbb{R}}$ y, sea $\varepsilon>0$. Como existen los límites
        \begin{equation*}
            \lim_{x\rightarrow a}(f(x)+g(x))=l_1\textup{ y }\lim_{x\rightarrow a}f(x)=l_2
        \end{equation*}
        entonces para $\frac{\varepsilon}{2}>0$ existen $\delta_1,\delta_2>0$ tales que
        \begin{equation*}
            \begin{split}
                \forall x\in S\backslash\left\{a \right\} \textup{tal que }\abs{x-a}<\delta_1&\Rightarrow \abs{f(x)+g(x)-l_1}<\frac{\varepsilon}{2}\\
                \forall x\in S\backslash\left\{a \right\} \textup{tal que }\abs{x-a}<\delta_1&\Rightarrow \abs{f(x)-l_2}<\frac{\varepsilon}{2}\\
            \end{split}
        \end{equation*}
        sea $l=l_1-l_2\in\mathbb{R}$. Tomemos $\delta=\min\left\{\delta_1,\delta_2 \right\}>0$. Si $x\in S\backslash\left\{a \right\}$ es tal que
        \begin{equation*}
            \abs{x-a}<\delta\Rightarrow \abs{x-a}<\delta_1\textup{ y }\abs{x-a}<\delta_2
        \end{equation*}
        por lo anterior se sigue que
        \begin{equation*}
            \abs{f(x)+g(x)-l_1}<\frac{\varepsilon}{2}\textup{ y }\abs{f(x)-l_2}<\frac{\varepsilon}{2}
        \end{equation*}
        luego
        \begin{equation*}
            \begin{split}
                \abs{g(x)-l}&=\abs{g(x)+f(x)-f(x)-l_1+l_2}\\
                &=\abs{f(x)+g(x)-l_1-(f(x)-l_2)}\\
                &\leq\abs{f(x)+g(x)-l_1}+\abs{f(x)-l_2}\\
                &<\frac{\varepsilon}{2}+\frac{\varepsilon}{2}\\
                &=\varepsilon\\
            \end{split}
        \end{equation*}
        por tanto
        \begin{equation*}
            \forall x\in S\backslash\left\{a \right\} \textup{tal que }\abs{x-a}<\delta\Rightarrow \abs{g(x)-l}<\varepsilon
        \end{equation*}
        de la definición de límite se sigue que $\lim_{x \rightarrow a}g(x)=l=l_1-l_2$.

        De (ii): Como existen los límites
        \begin{equation*}
            \lim_{x\rightarrow a}f(x)g(x)=l_1\quad\textup{y}\quad\lim_{x\rightarrow a}f(x)=l_2\neq 0
        \end{equation*}
        (donde $l_1,l_2\in\mathbb{R}$), en particular, existe el límite siguiente y su valor es:
        \begin{equation*}
            \lim_{x\rightarrow a}\frac{1}{f(x)}=\frac{1}{\lim_{x\rightarrow a}f(x)}=\frac{1}{l_2}
        \end{equation*}
        (por el teorema de álgebra de límites) luego, se tiene que:
        \begin{equation*}
            \lim_{x\rightarrow a }g(x)=\lim_{x\rightarrow a }\frac{1}{f(x)}g(x)f(x)=\left(\lim_{x\rightarrow a }\frac{1}{f(x)}\right)\cdot\left(\lim_{x\rightarrow a }g(x)f(x)\right)=\frac{1}{l_2}\cdot l_1=\frac{l_1}{l_2}
        \end{equation*}
        (nuevamente por el teorema de álgebra de límites y pues existe cada uno de los dos límites en el producto). Así, el límite anterior existe y su valor es el de la derecha.
    \end{proof}

    \item Suponga que exista el límite $\lim_{x\rightarrow a}f(x)g(x)$ y, además, $\lim_{x\rightarrow a}f(x)=0$. ¿Puede existir $\lim_{x\rightarrow a}g(x)$? \textbf{Justifique} formalmente sus respuestas o dando contraejemplos.
    
    \begin{proof}
        No necesariamente, tomemos por ejemplo $f(x)=0$ para todo $x\in\mathbb{R}$ y
        \begin{equation*}
            g(x)=\left\{
                \begin{array}{lcr}
                    0 & \textup{ si } & x\in\mathbb{Q}\\
                    1 & \textup{ si } & x\in\mathbb{R}\backslash\mathbb{Q}\\
                \end{array}
            \right.,\quad\forall x\in\mathbb{R}
        \end{equation*}
        se tiene que $f(x)g(x)=0$ para todo $x\in\mathbb{R}$. Luego, los límites
        \begin{equation*}
            \lim_{ x\rightarrow a}f(x)=\lim_{ x\rightarrow a}f(x)g(x)=0
        \end{equation*}
        existen y son iguales para toda $a\in\mathbb{R}$. Pero, ya se sabe que no existe $\lim_{ x\rightarrow a}g(x)$ para toda $a\in\mathbb{R}$.
    \end{proof}

    \item \begin{enumerate}
        \item \textbf{Pruebe} que si $\abs{x-2}\leq1$, entonces $\abs{x^2+3x-1}\geq1$.
        \item Sea $\delta>0$. Use el inciso anterior para \textbf{probar} que si $x=\min\left\{5/2,2+\delta/2 \right\}$, entonces $\abs{x^2+3x-1}\geq1$. Aplique la definición de límite para \textbf{concluir} que $\lim_{x\rightarrow2}x^2+3x\neq1$.
    \end{enumerate}

    \begin{proof}
        De (i): Sea $x\in\mathbb{R}$ tal que $\abs{x-2}\leq 1$, se sigue entonces que $2-\abs{x}\leq 1\Rightarrow 1\leq\abs{x}$ y que $-1\leq-\abs{x-2}$. Veamos ahora que
        \begin{equation*}
            \begin{split}
                \abs{x^2+3x-1}&=\abs{x^2-2x+5x-1}\\
                &=\abs{5x-1+x^2-2x}\\
                &\geq\abs{5x-1}-\abs{x^2-2x}\\
                &\geq5\abs{x}-1-\abs{x-2}\abs{x}\\
                &\geq5-1+(-1)(1)\\
                &\geq3\\
                &\geq 1\\
            \end{split}
        \end{equation*}

        De (ii): Sea $\delta>0$. Tomemos $x=\min\left\{5/2,2+\delta/2\right\}$, se tiene que
        \begin{equation*}
            \begin{split}
                x-2&\leq\frac{5}{2}-2\\
                &=\frac{1}{2}\\
                &\leq 1\\
            \end{split}
        \end{equation*}
        y,
        \begin{equation*}
            \begin{split}
                x-2&\geq 2-2\\
                &= 0\\
                &\geq -1\\
            \end{split}
        \end{equation*}
        pues, $\min\left\{5/2,2+\delta/2 \right\}>2$. Por tanto, $-1\leq x-2\leq 1$, así $\abs{x-2}\leq 1$, de (i) se sigue que $\abs{x^2+3x-1}\geq 1$.

        Ahora, notemos que para $\varepsilon=1>0$ y para todo $\delta>0$ existe $x_\delta=\min\left\{5/2,2+\delta/2 \right\}>2$ tal que
        \begin{equation*}
            0<\abs{x_\delta-2}=x_\delta-2\leq2+\frac{\delta}{2}-2=\frac{\delta}{2}<\delta
        \end{equation*}
        tal que
        \begin{equation*}
            \abs{x^2-3x-1}\geq\varepsilon=1
        \end{equation*}
        luego, por la definición de límite debe suceder que
        \begin{equation*}
            \lim_{ x\rightarrow 2}x^2+3x\neq1
        \end{equation*}
    \end{proof}

    \item Considere las funciones $j(x)=x$, $s(x)=x^2$ y $h(x)=\sqrt{\abs{x}}$, para todo $x\in\mathbb{R}$.
    \begin{enumerate}
        \item \textbf{Determine} el conjunto de todos los $x\in\mathbb{R}$ para los que se cumpla que $s(x)\leq j(x)$ y \textbf{haga} un dibujo en donde aparezcan simultáneamente las gráficas de $s$ y $j$.
        \item \textbf{Determine} el conjunto de todos los $x\in\mathbb{R}$ para los que se cumpla que $h(x)\leq s(x)$ y \textbf{haga} un dibujo en donde aparezcan simultáneamente las gráficas de $h$ y $s$.
        \item \textbf{Determine} el conjunto de todos los $x\in\mathbb{R}$ para los que se cumpla que $j(x)\leq h(x)$ y \textbf{haga} un dibujo en donde aparezcan simultáneamente las gráficas de $j$ y $h$.
    \end{enumerate}

    \begin{sol}
        De (i): Sea
        \begin{equation*}
            A=\left\{x\in\mathbb{R}\Big|s(x)\leq j(x) \right\}
        \end{equation*}
        se tiene que para $x\in\mathbb{R}$:
        \begin{equation*}
            \begin{split}
                s(x)\leq j(x)&\iff x^2\leq x\\
                &\iff 0\leq x-x^2 \\
                &\iff 0\leq x(1-x)\\
                &\iff (x\geq0\textup{ y }1-x\geq0)\textup{ o }(x\leq0\textup{ y }1-x\leq0)\\
                &\iff (x\geq0\textup{ y }1\geq x)\textup{ o }(x\leq0\textup{ y }1\leq x)\\
                &\iff 0\leq x\leq 1 \\
            \end{split}
        \end{equation*}
        por tanto, $A=[0,1]$. Se tiene el dibujo de las gráficas de $s$ y $j$ en la figura 7:
        \begin{figure}
            \begin{center}
                \includegraphics[scale=1]{images/3_11_1.pdf}
            \end{center}
            \caption{Gráfica de $s$ y $j$.}
        \end{figure}

        De (ii): Sea
        \begin{equation*}
            B=\left\{x\in\mathbb{R}\Big| h(x)\leq s(x) \right\}
        \end{equation*}
        se tiene que para $x\in\mathbb{R}$:
        \begin{equation*}
            \begin{split}
                h(x)\leq s(x)&\iff\sqrt{\abs{x}}\leq x^2\\
                &\iff \abs{x}\leq x^4\\
                &\iff 0\leq \abs{x}^4-\abs{x}\\
                &\iff 0\leq \abs{x}(\abs{x}^3-1)\\
                &\iff 0\leq \abs{x}^3-1\\
                &\iff 1\leq \abs{x}^3\\
                &\iff 1\leq \abs{x}\\
                &\iff x\leq-1\textup{ o }1\leq x \\
            \end{split}
        \end{equation*}
        por tanto, $B=]-\infty, -1]\cup[1,\infty[$. Se tiene el dibujo de las gráficas de $h$ y $s$ en la figura 8.
        \begin{figure}
            \begin{center}
                \includegraphics[scale=1]{images/3_11_2.pdf}
            \end{center}
            \caption{Gráfica de $h$ y $s$.}
        \end{figure}

        De (iii): Sea
        \begin{equation*}
            C=\left\{x\in\mathbb{R}\Big| h(x)\leq s(x) \right\}
        \end{equation*}
        se tiene que para $x\in\mathbb{R}$:
        \begin{equation*}
            \begin{split}
                j(x)\leq h(x)&\iff x\leq \sqrt{\abs{x}} \\
                &\iff x<0 \textup{ ó }0\leq x\leq\sqrt{\abs{x}} \\
                &\iff x<0 \textup{ ó }0\leq x\leq\sqrt{x} \\
                &\iff x<0 \textup{ ó }0\leq x^2\leq x \\
                &\iff x<0 \textup{ ó }0\leq x-x^2 \\
                &\iff x<0 \textup{ ó }0\leq x(1-x)\\
                &\iff x<0 \textup{ o }[(0\leq x \textup{ y }0\leq 1-x)\textup{ o }(x\leq 0\textup{ y }1-x\leq0)]\\
                &\iff x<0 \textup{ o }[(0\leq x \textup{ y }x\leq 1)\textup{ o }(x\leq 0\textup{ y }1\leq x)]\\
                &\iff x<0 \textup{ o } 0\leq x\leq 1)\\
            \end{split}
        \end{equation*}
        por tanto, $C=]-\infty, 1]$. Se tiene el dibujo de las gráficas de $j$ y $h$ en la figura 9.
        \begin{figure}
            \begin{center}
                \includegraphics[scale=1]{images/3_11_3.pdf}
            \end{center}
            \caption{Gráfica de $j$ y $h$.}
        \end{figure}
    \end{sol}

    \item Sea $S\subseteq\mathbb{R}$. Dadas dos funciones $\cf{f,g}{S}\mathbb{R}$ se define la \textbf{envoltura superior} de $f$ y $g$, como la función $\cf{\max\left(f,g\right)}{S}{\mathbb{R}}$ dada por
    \begin{equation*}
        \max\left(f,g\right)(x)=\max\left(f(x),g(x)\right),\quad\forall x\in S
    \end{equation*}
    y, la \textbf{envoltura inferior} de $f$y $g$ como la función $\cf{\min\left(f,g\right)}{S}{\mathbb{R}}$ dada por
    \begin{equation*}
        \min\left(f,g\right)(x)=\min\left(f(x),g(x)\right),\quad\forall x\in S
    \end{equation*}
    \begin{enumerate}
        \item Reconsidere las funciones $\cf{j,s,h}{\mathbb{R}}{\mathbb{R}}$ del problema anterior. \textbf{Bosqueje} la gráfica de las funciones $\max(j,s),\min(j,s),\max(s,h),\min(s,h),\max(j,h),\min(j,h)$.
        \item \textbf{Escriba} las funciones $\max(f,g)$ y $\min(f,g)$ en términos de $f$ y $g$ y del valor absoluto.
    \end{enumerate}

    \begin{proof}
        De (i): Se tienen las siguientes gráficas de las funciones mencionadas en las figuras 10 a 15.

        \begin{figure}
            \begin{center}
                \includegraphics[scale=1]{images/3_12_1.pdf}
            \end{center}
            \caption{Gráfica de $\max(j,s)$.}
        \end{figure}

        \begin{figure}
            \begin{center}
                \includegraphics[scale=1]{images/3_12_2.pdf}
            \end{center}
            \caption{Gráfica de $\min(j,s)$.}
        \end{figure}

        \begin{figure}
            \begin{center}
                \includegraphics[scale=1]{images/3_12_3.pdf}
            \end{center}
            \caption{Gráfica de $\max(s,h)$.}
        \end{figure}

        \begin{figure}
            \begin{center}
                \includegraphics[scale=1]{images/3_12_4.pdf}
            \end{center}
            \caption{Gráfica de $\min(s,h)$.}
        \end{figure}

        \begin{figure}
            \begin{center}
                \includegraphics[scale=1]{images/3_12_5.pdf}
            \end{center}
            \caption{Gráfica de $\max(j,h)$.}
        \end{figure}

        \begin{figure}
            \begin{center}
                \includegraphics[scale=1]{images/3_12_6.pdf}
            \end{center}
            \caption{Gráfica de $\min(j,h)$.}
        \end{figure}

        De (ii): Recordemos que si $a,b\in\mathbb{R}$ se tiene que
        \begin{equation*}
            \max\left(a,b \right)=\frac{a+b}{2}+\frac{\abs{a-b}}{2}\textup{ y }\min\left(a,b \right)=\frac{a+b}{2}-\frac{\abs{a-b}}{2}
        \end{equation*}
        usando esto, tendremos que
        \begin{equation*}
            \max\left(f,g \right)(x)=\frac{f(x)+g(x)}{2}+\frac{\abs{f(x)-g(x)}}{2}\textup{ y }\min\left(f,g \right)(x)=\frac{f(x)+g(x)}{2}-\frac{\abs{f(x)-g(x)}}{2}
        \end{equation*}
        luego,
        \begin{equation*}
            \max\left(f,g \right)(x)=\frac{(f+g)(x)}{2}+\frac{\abs{f-g}(x)}{2}\textup{ y }\min\left(f,g \right)(x)=\frac{(f+g)(x)}{2}-\frac{\abs{f-g}(x)}{2}
        \end{equation*}
    \end{proof}

    \item Aplique el teorema de comparación y/o el tereoma de álgebra de límites para \textbf{calcular} los límites siguientes.
    \begin{enumerate}
        \item $\lim_{x\rightarrow0}\frac{\tan^2x}{x}$.
        \item $\lim_{x\rightarrow a}\left[\frac{\sen x-\sen a}{x-a} \right]$, donde $a\in\mathbb{R}$.
        \item $\lim_{x\rightarrow0}\sqrt{\abs{x}}\sin\left(\frac{1}{x}\right)$.
        \item Sean $\cf{f,g}{S}{\mathbb{R}}$ dos funciones y $a\in\mathbb{R}$. Suponga que $g$ es acotada en $S$ y que $\lim_{x\rightarrow a}f(x)=0$. \textbf{Demuestre} que
        \begin{equation*}
            \lim_{x\rightarrow a}f(x)g(x)=0
        \end{equation*}
    \end{enumerate}

    \begin{sol}
        De (i): Veamos que
        \begin{equation*}
            \begin{split}
                \lim_{x\rightarrow0}\frac{\tan^2x}{x}&=\lim_{x\rightarrow0}\left(\frac{\sin x}{x}\cdot\frac{\sin x}{\cos^2 x}\right) \\
            \end{split}
        \end{equation*}
        donde, $\lim_{x\rightarrow0}\frac{\sin x}{x}=1$ y, como $\lim_{x\rightarrow0}\cos^2 x=1$ y $\lim_{x\rightarrow0}\sin x=0$, se sigue que $\lim_{x\rightarrow0}\frac{\sin x}{\cos^2 x}=0$. Por tanto,
        \begin{equation*}
            \begin{split}
                \lim_{x\rightarrow0}\frac{\tan^2x}{x}&=\left(\lim_{x\rightarrow0}\frac{\sin x}{x}\right)\cdot\left(\lim_{x\rightarrow0}\frac{\sin x}{\cos^2 x}\right)\\
                &=1\cdot 0\\
                &=0\\
            \end{split}
        \end{equation*}

        De (ii): Sea $a\in\mathbb{R}$, veamos que
        \begin{equation*}
            \begin{split}
                \lim_{x\rightarrow a}\frac{\sin x-\sin a}{x-a}&=\lim_{h\rightarrow 0}\frac{\sin (a+h)-\sin a}{a+h-a}\\
                &=\lim_{h\rightarrow 0}\frac{\sin (a+h)-\sin a}{h}\\
                &=\lim_{h\rightarrow 0}\frac{\sin a\cos h+\sin h\cos a -\sin a}{h}\\
                &=\lim_{h\rightarrow 0}\frac{[\cos h-1]\sin a +\sin h\cos a}{h}\\
                &=\lim_{h\rightarrow 0}\left(\frac{\cos h-1}{h}\cdot\sin a+\frac{\sin h}{h}\cdot\cos a\right) \\
            \end{split}
        \end{equation*}
        donde
        \begin{equation*}
            \lim_{ h\rightarrow0}\frac{\sin h}{h}\cdot\cos a=\cos a\lim_{ h\rightarrow0}\frac{\sin h}{h}=\cos a
        \end{equation*}
        y,
        \begin{equation*}
            \begin{split}
                \lim_{h\rightarrow 0}\frac{\cos h-1}{h}&=\lim_{h\rightarrow 0}\frac{\cos h-1}{h}\cdot\frac{\cos h+1}{\cos h+1}\\
                &=\lim_{h\rightarrow 0}\frac{\cos^2 h-1}{h(\cos h+1)}\\
                &=\lim_{h\rightarrow 0}\frac{-\sin^2 h}{h(\cos h+1)}\\
                &=\lim_{h\rightarrow 0}\frac{-\sin h}{\cos h+1}\cdot\frac{\sin h}{h}\\
                &=\left(\lim_{h\rightarrow 0}\frac{-\sin h}{\cos h+1}\right)\cdot\left(\lim_{h\rightarrow 0}\frac{\sin h}{h}\right)\\
                &=\frac{0}{2}\cdot 1\\
                &=0
            \end{split}
        \end{equation*}
        por ende, el límite es la suma de los límites y se sigue que 
        \begin{equation*}
            \lim_{x\rightarrow a}\frac{\sin x-\sin a}{x-a}=\cos a
        \end{equation*}

        De (iii): Veamos que
        \begin{equation*}
            -\sqrt{\abs{x}}\leq\sqrt{\abs{x}}\sin\left(\frac{1}{x}\right)\leq\sqrt{\abs{x}}
        \end{equation*}
        para todo $x\in\mathbb{R}\backslash\left\{0\right\}$. Tomando el límite de ambos lados y usando el teorema de comparación se sigue que, al tenerse que $\lim_{ x\rightarrow 0}\sqrt{\abs{x}}=0$:
        \begin{equation*}
            \lim_{x\rightarrow0}\sqrt{\abs{x}}\sin\left(\frac{1}{x} \right)=0
        \end{equation*}

        De (iv): Como $g$ es acotada en $S$, existe $M\geq0$ tal que
        \begin{equation*}
            \abs{g(x)}\leq M\quad\forall x\in S
        \end{equation*}
        Sea $\varepsilon>0$, dado que $\lim_{ x\rightarrow a}f(x)=0$, existe $\delta>0$ tal que
        \begin{equation*}
            x\in S\textup{ tal que }0<\abs{x-a}<\delta\Rightarrow\abs{f(x)-0}<\frac{\varepsilon}{M+1}
        \end{equation*}
        Como $\abs{g(x)}\leq M$ para todo $x\in S$, se tiene que
        \begin{equation*}
            x\in S\textup{ tal que }0<\abs{x-a}<\delta\Rightarrow\abs{f(x)g(x)-0}<\frac{M}{M+1}\cdot\varepsilon\leq\varepsilon
        \end{equation*}
        Por tanto,
        \begin{equation*}
            \lim_{ x\rightarrow a}f(x)g(x)=0
        \end{equation*}
    \end{sol}

    \item Use el teorema sobre la caracterización de límites de funciones por medio de sucesiones en los problemas siguientes.
    \begin{enumerate}
        \item \textbf{Calcule} $\lim_{x\rightarrow-2}\sqrt[3]{\frac{x-1}{2x+2}}$.
        \item \textbf{Calcule} $\lim_{x\rightarrow-3}\sqrt[3]{\abs{x}^3}$.
        \item \textbf{Muestre} que no existe $\lim_{x\rightarrow0}\sen\left(\frac{1}{x}\right)$.
        \item \textbf{Muestre} que no existe $\lim_{x\rightarrow2}E(x)$, donde $E$ es la función parte entera.
        \item \textbf{Muestre} que no existe $\lim_{x\rightarrow2}f(x)$, donde $f(x)=\left\{\begin{array}{lcr}
            \sqrt{x} & \textup{ si } & 0\leq x<2\\
            x^3 & \textup{ si } & 2<x\leq3 \\
        \end{array} \right.$.
        \item \textbf{Muestre} que no existe $\lim_{x\rightarrow a}f(x)$, donde $f$ es la función de Dirichlet y $a\in\mathbb{R}$ es arbitrario.
    \end{enumerate}

    \begin{sol}
        De (i): Sea $\left\{x_n\right\}_{ n=1}^\infty$ una sucesión en $\mathbb{R}\backslash\left\{-2\right\}$ que converge a $-2$. Se tiene que
        \begin{equation*}
            \begin{split}
                \lim_{ n\rightarrow\infty}\sqrt[3]{\frac{x_n-1}{2x_n+2}}&=\sqrt[3]{\lim_{ n\rightarrow\infty}\frac{x_n-1}{2x_n+2}}\\
                &=\sqrt[3]{\frac{\lim_{ n\rightarrow\infty}x_n-1}{\lim_{ n\rightarrow\infty}2x_n+2}}\\
                &=\sqrt[3]{\frac{-2-1}{-4+2}}\\
                &=\sqrt[3]{\frac{-3}{-2}}\\
                &=\sqrt[3]{\frac{3}{2}}\\
            \end{split}
        \end{equation*}
        por tanto, al ser la sucesión arbitraria se sigue que
        \begin{equation*}
            \lim_{ x\rightarrow-2}\sqrt[3]{\frac{x-1}{2x+2}}=\sqrt[3]{\frac{3}{2}}
        \end{equation*}

        De (ii): Sea $\left\{x_n\right\}_{ n=1}^\infty$ una sucesión en $\mathbb{R}\backslash\left\{-3\right\}$ que converge a $-3$. Se tiene que
        \begin{equation*}
            \lim_{ n\rightarrow\infty}
        \end{equation*}
        %TODO
    \end{sol}

    \item Usando primero la definición de límite y después el teorema sobre caracterización de límites de funciones por medio de sucesiones, \textbf{pruebe} que:
    \begin{enumerate}
        \item $\lim_{x\rightarrow1}\frac{x-1}{2x+2}\neq 3$.
        \item $\lim_{x\rightarrow 2}\abs{x}\neq -1$.
    \end{enumerate}

    \begin{proof}
        De (i): Se probará que el límite no es $3$ por medio de sucesiones. En efecto, sea $\left\{x_n \right\}_{ n=1}^\infty$ una sucesión en $\mathbb{R}\backslash\left\{-1,1 \right\}$ que converge a $1$. Se tiene que
        \begin{equation*}
            \begin{split}
                \lim_{ n\rightarrow\infty}\frac{x_n-1}{2x_n+2}&=\frac{\lim_{ n\rightarrow\infty}(x_n-1)}{\lim_{ n\rightarrow\infty}(2x_n+2)}\\
                &=\frac{1-1}{2\cdot 2+2}\\
                &=\frac{0}{6}\\
                &=0\\
            \end{split}
        \end{equation*}
        Por tanto, usando el teorema de caracterización de límite por sucesiones, se sigue que
        \begin{equation*}
            \lim_{ x\rightarrow 1}\frac{x-1}{2x+2}=1\neq 3
        \end{equation*}

        De (ii). Tomemos $\varepsilon=1$. Para todo $\delta>0$ existe $x_\delta=\min\left\{2+\frac{1}{2},2+\frac{\delta}{2} \right\}$ tal que
        \begin{equation*}
            0<\abs{x_\delta-2}=x_\delta-2\leq2+\frac{\delta}{2}-2=\frac{\delta}{2}<\delta
        \end{equation*}
        tal que
        \begin{equation*}
            \abs{f(x_\delta)-(-1)}=\abs{\abs{x_\delta}+1}\geq \abs{x_\delta}-1\geq2-1=1=\varepsilon
        \end{equation*}
        por tanto,
        \begin{equation*}
            \lim_{ x\rightarrow 2}\abs{x}\neq-1
        \end{equation*}
    \end{proof}

    \item Considere la función
    \begin{equation*}
        f(x)=\left\{
            \begin{array}{lcr}
                -x & \textup{ si } & -3\leq x\leq-1\\
                x^3 & \textup{ si } & -1< x<1\\
                \sqrt{x} & \textup{ si } & 1<x\leq3 \\
                \frac{1}{3-x}+\sqrt{3} & \textup{ si } & 3<x\\ 
            \end{array}
        \right.
    \end{equation*}
    \begin{enumerate}
        \item \textbf{Bosqueje} la gráfica de $f$.
        \item \textbf{Calcule} $\lim_{x \rightarrow-1^-}f(x)$ y $\lim_{x \rightarrow-1^+}f(x)$ ¿Existe $\lim_{x \rightarrow-1}f(x)$?
        \item \textbf{Calcule} $\lim_{x \rightarrow1^-}f(x)$ y $\lim_{x \rightarrow1^+}f(x)$ ¿Existe $\lim_{x \rightarrow1}f(x)$?
        \item ¿Existen $\lim_{x \rightarrow3^-}f(x)$, $\lim_{x \rightarrow3^+}f(x)$ y $\lim_{x \rightarrow3}f(x)$?
        \item \textbf{Calcule} $\lim_{x \rightarrow-3}f(x)$ y $\lim_{x \rightarrow\infty}f(x)$.
        \item Si $a\in[-3,\infty[\backslash\left\{-3,-1,1,3 \right\}$. \textbf{Calcule} $\lim_{ x\rightarrow a}f(x)$, $\lim_{ x\rightarrow a^+}f(x)$ y $\lim_{ x\rightarrow a^-}f(x)$.
    \end{enumerate}
    \textbf{Justifique} usando la definición del límite correspondiente y también usando la respectiva caracterización de sucesiones.

    \begin{sol}
        %TODO
    \end{sol}

    \item \textbf{Calcule}, justificando por medio de la definición del límite correspondiente y de la respectiva caracterización de sucesiones, los siguientes límites.
    \begin{enumerate}
        \item $\lim_{x\rightarrow\infty}x^2+3$, $\lim_{x\rightarrow\infty}-x^2-3$ y $\lim_{x\rightarrow\infty}[(x^2+3)-(-x^2-5)]$.
        \item $\lim_{x\rightarrow\infty}3x^2-x+5$, $\lim_{x\rightarrow\infty}x-3$ y $\lim_{x\rightarrow\infty}[(3x^2-x+5)+(x-3)]$.
        \item $\lim_{x\rightarrow3^-}\frac{5x-2}{x-3}$, $\lim_{x\rightarrow3^+}\frac{5x-2}{x-3}$ y $\lim_{x\rightarrow3}\abs{\frac{5x-2}{x-3}}$.
    \end{enumerate}

    \begin{sol}
        %TODO
    \end{sol}

    \item \textbf{Calcule}
    \begin{equation*}
        \lim_{ x\rightarrow\infty}\frac{a_nx^n+\cdots+a_1x+a_0}{b_mx^m+\cdots+b_1x+b_0}
    \end{equation*}
    donde $a_n,b_m\neq0$, distinguiendo los casos $m=n$, $m>n$ y $m<n$. En particular, \textbf{calcule}.
    \begin{equation*}
        \lim_{ x\rightarrow\infty}\frac{x^2-3}{x-1}\quad\lim_{ x\rightarrow\infty}\frac{2-x}{x^4-1}\quad\lim_{ x\rightarrow\infty}\frac{5x+2}{x-3}
    \end{equation*}

    \begin{sol}
        Se tienen 3 casos:
        \begin{enumerate}
            \item $m=n$. Veamos que
            \begin{equation*}
                \begin{split}
                    \lim_{ x\rightarrow\infty}\frac{a_nx^n+\cdots+a_1x+a_0}{b_mx^m+\cdots+b_1x+b_0}&=\lim_{ x\rightarrow\infty}\frac{a_nx^n+\cdots+a_1x+a_0}{b_nx^n+\cdots+b_1x+b_0}\cdot1 \\
                    &=\lim_{ x\rightarrow\infty}\frac{a_nx^n+\cdots+a_1x+a_0}{b_nx^n+\cdots+b_1x+b_0}\cdot\frac{\frac{1}{x^n}}{\frac{1}{x^n}}\\
                    &=\lim_{ x\rightarrow\infty}\frac{a_n+\frac{a_{ n-1}}{x}+\cdots+\frac{a_1}{x^{ n-1}}+\frac{a_0}{x^n}}{b_n+\frac{b_{ n-1}}{x}+\cdots+\frac{b_1}{x^{ n-1}}+\frac{b_0}{x^n}}\\
                \end{split}
            \end{equation*}
            y, como los límites $\lim_{ x\rightarrow\infty}a_n=a_n$, $\lim_{ x\rightarrow\infty}b_n=b_n$ y
            \begin{equation*}
                \lim_{x\rightarrow\infty}\frac{b_i}{x^{n-i}}=\lim_{x\rightarrow\infty}\frac{a_i}{x^{n-i}}=0
            \end{equation*}
            existen para todo $i\in\left\{0,1,2,...,n-1 \right\}$, por álgebra de límites se sigue que:
            \begin{equation*}
                \begin{split}
                    \lim_{ x\rightarrow\infty}\frac{a_nx^n+\cdots+a_1x+a_0}{b_mx^m+\cdots+b_1x+b_0}&=\frac{\lim_{ x\rightarrow\infty}(a_n+\frac{a_{ n-1}}{x}+\cdots+\frac{a_1}{x^{ n-1}}+\frac{a_0}{x^n})}{\lim_{ x\rightarrow\infty}(b_n+\frac{b_{ n-1}}{x}+\cdots+\frac{b_1}{x^{ n-1}}+\frac{b_0}{x^n})}\\
                    &=\frac{a_n+0+\cdots+0}{b_n+0+\cdots+0}\\
                    &=\frac{a_n}{b_n}\\
                \end{split}
            \end{equation*}

            \item $m>n$: Veamos que
            \begin{equation*}
                \begin{split}
                    \lim_{ x\rightarrow\infty}\frac{a_nx^n+\cdots+a_1x+a_0}{b_mx^m+\cdots+b_1x+b_0}&=\lim_{ x\rightarrow\infty}\frac{a_nx^n+\cdots+a_1x+a_0}{b_mx^m+\cdots+b_nx^n+\cdots+b_1x+b_0}\cdot1 \\
                    &=\lim_{ x\rightarrow\infty}\frac{a_nx^n+\cdots+a_1x+a_0}{b_mx^m+\cdots+b_nx^n+\cdots+b_1x+b_0}\cdot\frac{\frac{1}{x^m}}{\frac{1}{x^m}}\\
                    &=\lim_{ x\rightarrow\infty}\frac{a_nx^n+\cdots+a_1x+a_0}{b_mx^m+\cdots+b_nx^n+\cdots+b_1x+b_0}\cdot\frac{\frac{1}{x^m}}{\frac{1}{x^m}}\\
                    &=\lim_{ x\rightarrow\infty}\frac{\frac{a_n}{x^{ m-n}} +\cdots+\frac{a_1}{x^{ m-1}} +\frac{a_0}{x^m}}{b_m+\frac{b_ {m-1}}{x} +\cdots+\frac{b_n}{x^{ m-n}} +\cdots+\frac{b_1}{x^{ m-1}}+\frac{b_0}{x^m}}\\
                \end{split}
            \end{equation*}
            donde
            \begin{equation*}
                \begin{split}
                    \lim_{ x\rightarrow\infty}\left(\frac{a_n}{x^{ m-n}} +\cdots+\frac{a_1}{x^{ m-1}} +\frac{a_0}{x^m}\right)&=0\textup{ y,} \\
                    \lim_{ x\rightarrow\infty}\left(b_m+\frac{b_ {m-1}}{x} +\cdots+\frac{b_n}{x^{ m-n}} +\cdots+\frac{b_1}{x^{ m-1}}+\frac{b_0}{x^m}\right)&=b_m\\
                \end{split}
            \end{equation*}
            pues,
            \begin{equation*}
                \lim_{ x\rightarrow\infty}\frac{a_i}{x^{ m-i}}=0
            \end{equation*}
            pues $0<m-i$, para todo $i\in\left\{0,1,2,...,n\right\}$ y de forma similar con los $b_n$:
            \begin{equation*}
                \lim_{ x\rightarrow\infty}\frac{b_j}{x^{ m-i}}=0
            \end{equation*}
            para todo $j\in\left\{0,1,2,...,m-1\right\}$. Por tanto, por álgebra de límtes se sigue que
            \begin{equation*}
                \begin{split}
                    \lim_{ x\rightarrow\infty}\frac{a_nx^n+\cdots+a_1x+a_0}{b_mx^m+\cdots+b_1x+b_0}&=\lim_{ x\rightarrow\infty}\frac{\frac{a_n}{x^{ m-n}} +\cdots+\frac{a_1}{x^{ m-1}} +\frac{a_0}{x^m}}{b_m+\frac{b_ {m-1}}{x} +\cdots+\frac{b_n}{x^{ m-n}} +\cdots+\frac{b_1}{x^{ m-1}}+\frac{b_0}{x^m}}\\
                    &=\frac{\lim_{ x\rightarrow\infty}\frac{a_n}{x^{ m-n}} +\cdots+\frac{a_1}{x^{ m-1}} +\frac{a_0}{x^m}}{\lim_{ x\rightarrow\infty}b_m+\frac{b_ {m-1}}{x} +\cdots+\frac{b_n}{x^{ m-n}} +\cdots+\frac{b_1}{x^{ m-1}}+\frac{b_0}{x^m}}\\
                    &=\frac{0}{b_m}\\
                    &=0\\
                \end{split}
            \end{equation*}
            \item $m<n$. Afirmamos que el límite es $\pm\infty$, dependiendo del signo de $\frac{a_n}{b_m}$. En efecto, suponga que %TODO 
        \end{enumerate}
        Por lo anterior, se sigue que
        \begin{equation*}
            \lim_{ x\rightarrow\infty}\frac{x^2-3}{x-1}=\infty,\lim_{ x\rightarrow\infty}\frac{2-x}{x^4-1}=0,\lim_{ x\rightarrow\infty}\frac{5x+2}{x-3}=5
        \end{equation*}
    \end{sol}

    \item Sea $E(x)=\max\left\{n\in\mathbb{Z}\Big|n\leq x \right\}$, para todo $x\in\mathbb{R}$, la función \textbf{parte entera} de $x$. Considere las funciones siguientes:
    \begin{enumerate}
        \item $f(x)=E(x)$.
        \item $f(x)=-[x-E(x)]$.
        \item $f(x)=\sqrt{x-E(x)}$.
        \item $f(x)=E(1/x)$.
        \item $f(x)=\frac{1}{E(1/x)}$.
        \item $f(x)=E(x)+\sqrt{x-E(x)}$.
    \end{enumerate}
    \textbf{Determine} el dominio natural de cada una de estas funciones y \textbf{bosqueje} su gráfica. Si existen, cálcule los siguientes límites (justificando formalmente) para cada una de las funciones anteriores:
    \begin{equation*}
        \lim_{ x\rightarrow a^-}f(x)\quad\lim_{ x\rightarrow a^+}f(x)\quad\lim_{ x\rightarrow a}f(x)
    \end{equation*}
    donde $a\in\mathbb{R}$, y
    \begin{equation*}
        \lim_{ x\rightarrow-\infty}f(x)\quad\lim_{ x\rightarrow\infty}f(x)
    \end{equation*}

    \begin{sol}
        %TODO
    \end{sol}

    \item Sea $S\subseteq\mathbb{R}$. Sea $\cf{f}{S}{\mathbb{R}}$ una función y, $t,l\in\mathbb{R}$. \textbf{Pruebe} que:
    \begin{enumerate}
        \item $\lim_{ x\rightarrow t}f(x)=l$ si y sólo si $\lim_{ x\rightarrow t}f(x)-l=0$ si y sólo si $\lim_{ x\rightarrow t}\abs{f(x)-l}=0$.
        \item $\lim_{ x\rightarrow t}f(x)=\lim_{ h\rightarrow 0}f(t+h)$.
    \end{enumerate}

    \begin{proof}
        De (i): Si $\lim_{ x\rightarrow t}f(x)=l$, entonces para $\varepsilon>0$ existe $\delta>0$ tal que
        \begin{equation*}
            x\in S\textup{ tal que }0<\abs{x-t}<\delta\Rightarrow \abs{f(x)-l}<\varepsilon
        \end{equation*}
        lo cual es equivalente a decir que
        \begin{equation}
            x\in S\textup{ tal que }0<\abs{x-t}<\delta\Rightarrow \abs{f(x)-l-0}<\varepsilon
        \end{equation}
        que es lo mismo que $\lim_{ x\rightarrow t}f(x)-l=0$. Pero, lo anterior también es equivalente a decir que
        \begin{equation*}
            x\in S\textup{ tal que }0<\abs{x-t}<\delta\Rightarrow \abs{\abs{f(x)-l}-0}<\varepsilon
        \end{equation*}
        que es lo mismo que $\lim_{ x\rightarrow t}\abs{f(x)-l}=0$. Por ende,
        \begin{equation*}
            \lim_{ x\rightarrow t}f(x)=l\iff \lim_{ x\rightarrow t}f(x)-l=0\iff \lim_{ x\rightarrow t}\abs{f(x)-l}=0
        \end{equation*}

        De (ii): Suponga que $\lim_{ x\rightarrow t}f(x)=l$ con $l\in\mathbb{R}$. Entonces para todo $\varepsilon>0$ existe $\delta>0$ tal que
        \begin{equation*}
            x\in S\textup{ tal que }0<\abs{x-t}<\delta\Rightarrow\abs{f(x)-l}<\varepsilon
        \end{equation*}
        Si $x=t+h$ con $h\in\mathbb{R}$, esto anterior es equivalente a decir que
        \begin{equation*}
            t+h\in S\textup{ tal que }0<\abs{t+h-t}<\delta\Rightarrow\abs{f(t+h)-l}<\varepsilon
        \end{equation*}
        esto es
        \begin{equation*}
            t+h\in S\textup{ tal que }0<\abs{h}<\delta\Rightarrow\abs{f(t+h)-l}<\varepsilon
        \end{equation*}
        por la definición de límite se sigue que
        \begin{equation*}
            \lim_{ h\rightarrow 0}f(t+h)=l=\lim_{ x\rightarrow t}f(x)
        \end{equation*}
    \end{proof}

    \item \textbf{Demuestre} que si dos funciones $f$ y $g$ toman los mismos valores en todos los puntos de algún intervalo abierto que contenga a $a$, exceptuando posiblemente a $a$, entonces
    \begin{equation*}
        \lim_{ x\rightarrow a}f(x)=\lim_{ x\rightarrow a}g(x)
    \end{equation*}
    cuando alguno de los dos límites exista. Esto significa que la existencia del límite de alguna función en un punto dado es una \textbf{propiedad local}.

    \begin{proof}
        Sean $\cf{f,g}{S}{\mathbb{R}}$ dos funciones tales que existe $I=]l,m[\subseteq S$ intervalo abierto con $a\in I$ tal que
        \begin{equation*}
            f(x)=g(x),\quad\forall x\in I\backslash\left\{a\right\}
        \end{equation*}
        y suponga que existe $\lim_{ x\rightarrow a}f(x)=l$. Probaremos que $\lim_{ x\rightarrow a}g(x)$ también existe y es igual a $l$. En efecto, sea $\varepsilon>0$, como $\lim_{ x\rightarrow a}f(x)=l$ entonces existe $\delta'>0$ tal que
        \begin{equation*}
            x\in S\textup{ tal que }0<\abs{x-a}<\delta'\Rightarrow\abs{f(x)-l}<\varepsilon
        \end{equation*}
        Siendo que $a\in I$ e $I$ es un intervalo abierto, se debe tener que
        \begin{equation*}
            a-l,m-a>0
        \end{equation*}
        tomemos $\delta=\min\left\{a-l,m-a,\delta' \right\}>0$. Si $x\in S$ es tal que
        \begin{equation*}
            0<\abs{x-a}<\delta
        \end{equation*}
        por la elección de $\delta$ se tienen las siguientes desigualdades:
        \begin{equation*}
            \left\{
                \begin{split}
                    a-x&<a-l\\
                    x-a&<m-a\\
                    0<\abs{x-a}&<\delta'\\
                \end{split}
            \right.
        \end{equation*}
        las primeras dos desigualdades implican que $l<x<m$, es decir que $x\in I$, luego por hipótesis se sigue que $f(x)=g(x)$. Además, la tercera condición implica que
        \begin{equation*}
            \abs{f(x)-l}<\varepsilon
        \end{equation*} 
        es decir que
        \begin{equation*}
            \abs{g(x)-l}<\varepsilon
        \end{equation*}
        Por tanto,
        \begin{equation*}
            x\in S\textup{ tal que }0<\abs{x-a}<\delta\Rightarrow\abs{g(x)-l}<\varepsilon
        \end{equation*}
        De esta forma, $\lim_{ x\rightarrow a}g(x)$ existe y,
        \begin{equation*}
            \lim_{ x\rightarrow a}g(x)=l=\lim_{ x\rightarrow a}f(x)
        \end{equation*}
    \end{proof}

    \item Sea $S\subseteq\mathbb{R}$. Sean $\cf{f,g}{S}{\mathbb{R}}$ dos funciones y $a\in\mathbb{R}$. Si $f(x)\leq g(x)$, para todo $x\in S$, y si existen $\lim_{ x\rightarrow a}f(x)$ y $\lim_{ x\rightarrow a}g(x)$, \textbf{pruebe} que
    \begin{equation*}
        \lim_{ x\rightarrow a}f(x)\leq\lim_{ x\rightarrow a}g(x)
    \end{equation*}

    \begin{proof}
        %TODO
    \end{proof}

    \item Sea $S\subseteq\mathbb{R}$. Sean $\cf{f,g,h}{S}{\mathbb{R}}$ tres funciones. Fije $a\in\mathbb{R}\cup\left\{-\infty,\infty \right\}$. Si $f(x)\leq g(x)\leq h(x)$, para todo $x\in S$ y $\lim_{ x\rightarrow a}f(x)=\lim_{ x\rightarrow a}h(x)$, \textbf{demuestre} que existe $\lim_{ x\rightarrow a}g(x)$ y
    \begin{equation*}
        \lim_{ x\rightarrow a}f(x)=\lim_{ x\rightarrow a}g(x)=\lim_{ x\rightarrow a}h(x)
    \end{equation*}

    \begin{proof}
        Sea $\left\{x_n \right\}_{ n=1}^\infty$ una sucesión en $S\backslash\left\{a \right\}$ que converge a $a$. Tomemos
        \begin{equation*}
            l=\lim_{ x\rightarrow a}f(x)=\lim_{ x\rightarrow a}h(x)
        \end{equation*}
        Considere las tres sucesiones $\left\{f(x_n)\right\}_{ n=1}^\infty,\left\{g(x_n)\right\}_{ n=1}^\infty$ y $\left\{h(x_n)\right\}_{ n=1}^\infty$. Se tiene que
        \begin{equation*}
            f(x_n)\leq g(x_n)\leq h(x_n)
        \end{equation*}
        para todo $n\in\mathbb{N}$ (usando la desigualdad que se tiene de hipótesis). Como los límites
        \begin{equation*}
            \lim_{ x\rightarrow a}f(x)=\lim_{ x\rightarrow a}h(x)
        \end{equation*}
        existen, entonces se debe tener que
        \begin{equation*}
            l=\lim_{ n\rightarrow\infty}f(x_n)=\lim_{ n\rightarrow\infty}h(x_n)
        \end{equation*}
        Luego, por el teorema del emparedado se sigue que
        \begin{equation*}
            \lim_{ n\rightarrow\infty}g(x_n)=l
        \end{equation*}
        Como la sucesión convergente a $a$ en $S\backslash\left\{a\right\}$ fue arbitraria, se sigue que
        \begin{equation*}
            \lim_{ x\rightarrow a}g(x)=l
        \end{equation*}
        es decir que el limite anterior existe y que
        \begin{equation*}
            \lim_{ x\rightarrow a}f(x)=\lim_{ x\rightarrow a}g(x)=\lim_{ x\rightarrow a}h(x)
        \end{equation*}
    \end{proof}

    \item Sea $S\subseteq\mathbb{R}$. Si $\cf{f}{S}{\mathbb{R}}$ es una función, defina la función $\cf{\abs{f}}{S}{\mathbb{R}}$ como $\abs{f}(x)=\abs{f(x)}$, para todo $x\in S$. \textbf{Pruebe} que si $\lim_{ x\rightarrow a}f(x)=l$, entonces $\lim_{ x\rightarrow a}\abs{f}(x)=\abs{l}$.
    
    \begin{proof}
        Sea $\varepsilon>0$. Como $\lim_{ x\rightarrow a}f(x)=l$ existe $\delta>0$ tal que
        \begin{equation*}
            \forall x\in S\textup{ tal que }0<\abs{x-a}<\delta\Rightarrow\abs{f(x)-l}<\varepsilon
        \end{equation*}
        pero, recordemos que si $u,v\in\mathbb{R}$:
        \begin{equation*}
            \abs{\abs{u}-\abs{v}}\leq\abs{u-v}
        \end{equation*}
        por tanto,
        \begin{equation*}
            \forall x\in S\textup{ tal que }0<\abs{x-a}<\delta\Rightarrow\abs{\abs{f(x)}-\abs{l}}<\varepsilon
        \end{equation*}
        es decir,
        \begin{equation*}
            \forall x\in S\textup{ tal que }0<\abs{x-a}<\delta\Rightarrow\abs{\abs{f}(x)-\abs{l}}<\varepsilon
        \end{equation*}
        de la definición de límite se sigue entonces que
        \begin{equation*}
            \lim_{ x\rightarrow a}\abs{f}(x)=\abs{l}
        \end{equation*}
    \end{proof}

    \item \textbf{Pruebe} que si $\lim_{ x\rightarrow a}f(x)=l$ y $\lim_{ x\rightarrow a}g(x)=m$, entonces
    \begin{equation*}
        \lim_{ x\rightarrow a}\max(f,g)(x)=\max(l,m)\quad\textup{y}\quad\lim_{ x\rightarrow a}\min(f,g)(x)=\min(l,m)
    \end{equation*}
    \textit{Sugerencia.} Utilice un resultado de un problema anterior.

    \begin{proof}
        Solo se probará el primer límite, pues el segundo es análogo. Supongamos que $\cf{f,g}{S}{\mathbb{R}}$ donde $S\subseteq\mathbb{R}$. Sea $\left\{x_n \right\}_{ n=1}^\infty$ una sucesión en $S\backslash\left\{a \right\}$ que converge a $a$. Veamos que
        \begin{equation*}
            \begin{split}
                \lim_{ n\rightarrow\infty}\max(f,g)(x_n)&=\lim_{ n\rightarrow\infty}\left(\frac{f(x_n)+g(x_n)}{2}+\frac{\abs{f(x_n)-g(x_n)}}{2}\right)\\
                &=\lim_{ n\rightarrow\infty}\frac{f(x_n)+g(x_n)}{2}+\lim_{ n\rightarrow\infty}\frac{\abs{f(x_n)-g(x_n)}}{2}\\
                &=\frac{\lim_{ n\rightarrow\infty}[f(x_n)+g(x_n)]}{2}+\frac{\lim_{ n\rightarrow\infty}\abs{f(x_n)-g(x_n)}}{2}\\
                &=\frac{l+m}{2}+\frac{\abs{l-m}}{2}\\
                &=\max(l,m)\\
            \end{split}
        \end{equation*}
        por tanto, como la sucesión fue arbitraria, se sigue que
        \begin{equation*}
            \lim_{ x\rightarrow a}\max(f,g)(x)=\max(l,m)
        \end{equation*}
        lo que prueba el resultado.
    \end{proof}

    \item \textbf{Determine (justificando)} el dominio natural y el conjunto de puntos de continuidad de las siguientes funciones:
    \begin{enumerate}
        \item $P(x)=a_nx^n+a_{ n-1}x^{ n-1}+\cdots+a_1x+a_0$, donde $n\in\mathbb{N}$, $a_i\in\mathbb{R}$, para todo $i=0,1,...,n$.
        \item $R(x)=\frac{P(x)}{Q(x)}$ ,donde $P$ y $Q$ son dos polinomios.
        \item $f(x)=x^a$, donde $a\in\mathbb{Q}$.
        \item $\mathcal{N}(x)=\abs{x}$.
    \end{enumerate}

    \begin{sol}
        %TODO
    \end{sol}

    \item Sea $S\subseteq\mathbb{R}$. Si $\cf{f,g}{S}{\mathbb{R}}$ son dos funciones continuas en todo punto de $S$, \textbf{pruebe} que las funciones $\max(f,g)$ y $\min(f,g)$ son también continuas en todo punto de $S$.

    \begin{proof}
        Sea $x_0\in S$ y $\left\{x_n \right\}_{ n=1}^\infty$ una sucesión en $S\backslash\left\{x_0 \right\}$ que converge a $x_0$. Entonces,
        \begin{equation*}
            \begin{split}
                \lim_{ n\rightarrow\infty}\max(f,g)(x_n)&=\lim_{ n\rightarrow\infty}\left(\frac{f(x_n)+g(x_n)}{2}+\frac{\abs{f(x_n)-g(x_n)}}{2}\right)\\
                &=\lim_{ n\rightarrow\infty}\frac{f(x_n)+g(x_n)}{2}+\lim_{ n\rightarrow\infty}\frac{\abs{f(x_n)-g(x_n)}}{2}\\
                &=\frac{\lim_{ n\rightarrow\infty}[f(x_n)+g(x_n)]}{2}+\frac{\lim_{ n\rightarrow\infty}\abs{f(x_n)-g(x_n)}}{2}\\
                &=\frac{f(x_0)+g(x_0)}{2}+\frac{\abs{f(x_0)-g(x_0)}}{2}\\
                &=\max(f,g)(x_0)
            \end{split}
        \end{equation*}
        Pues $f$ y $g$ son continuas en $x_0$. Por tanto
        \begin{equation*}
            \lim_{x\rightarrow x_0}\max(f,g)(x)=\max(f,g)(x_0)
        \end{equation*}
        por ser $\left\{x_n\right\}_{n=1}^\infty$ una sucesión arbitraria. Se sigue entonces que $\max(f,g)$ es continua en $x_0$. Al ser el $x_0\in S$ arbitrario, entonces $\max(f,g)$ es continua en $S$.

        De forma análoga se prueba que $\min(f,g)$ es continua en $S$.
    \end{proof}

    \item \textbf{Determine} (justificando) el conjunto de puntos de continuidad de las siguientes funciones e \textbf{indique} el tipo de discontinuidad que ocurre en los puntos donde son discontinuas. \textbf{Bosqueje} sus gráficas.
    \begin{enumerate}
        \item $f(x)=\left\{
            \begin{array}{lcr}
                2x & \textup{ si } & x\neq 2 \\
                2 & \textup{ si } & x=2\\
            \end{array}
        \right.$
        \item $f(x)=\left\{
            \begin{array}{lcr}
                2 & \textup{ si } & -2<x<-1 \\
                x^3 & \textup{ si } & -1\leq x<1 \\
                x+1 & \textup{ si } & 1\leq x<3 \\
            \end{array}
        \right.$
        \item $f(x)=\left\{
            \begin{array}{lcr}
                x & \textup{ si } & x\in\mathbb{Q} \\
                0 & \textup{ si } & x\notin\mathbb{Q} \\
            \end{array}
        \right.$
        \item $f(x)=\left\{
            \begin{array}{lcr}
                1 & \textup{ si } & x\in\mathbb{Q} \\
                0 & \textup{ si } & x\notin\mathbb{Q} \\
            \end{array}
        \right.$
    \end{enumerate}

    \begin{sol}
        %TODO
    \end{sol}

    \item \textbf{Determine} (justificando) el conjunto de puntos de discontinuidad de las funciones dadas en el problema 3.18 e \textbf{indique} el tipo de discontuidad que ocurre en los puntos donde son discontinuas.
    
    \begin{sol}
        %TODO
    \end{sol}

    \item \textbf{Determine} (justificando) el conjunto de puntos de continuidad de la siguiente función y \textbf{bosqueje} su gráfica
    \begin{equation*}
        f(x)=\min(x-E(x),E(x+1)-x),\quad\forall x\in\mathbb{R}
    \end{equation*}

    \begin{sol}
        %TODO
    \end{sol}

    \item \begin{enumerate}
        \item \textbf{Construya} un ejemplo de una función que sea discontinua en los puntos $1,1/2,1/3,...$ pero que sea continua en los demás puntos de $\mathbb{R}$. \textbf{Justifique}.
        \item \textbf{Construya} un ejemplo de una función que sea discontinua en los puntos $1,1/2,1/3,...$ y $0$ pero que sea continua en los demás puntos de $\mathbb{R}$. \textbf{Justifique}.
    \end{enumerate}

    \begin{sol}
        De (i): Sea
        \begin{equation*}
            f(x)=\left\{
                \begin{array}{lcr}
                    \frac{1}{n} & \textup{ si } & x=\frac{1}{n}\textup{ para algún }n\in\mathbb{N}\\
                    0 & \textup{ e.o.c } & \\
                \end{array}
            \right.
        \end{equation*}

        De (ii):
        \begin{equation*}
            f(x)=\left\{
                \begin{array}{lcr}
                    1 & \textup{ si } & x=\frac{1}{n}\textup{ para algún }n\in\mathbb{N}\\
                    0 & \textup{ e.o.c } & \\
                \end{array}
            \right.
        \end{equation*}
    \end{sol}

    \item Sea
    \begin{equation*}
        f(x)=\left\{
            \begin{array}{lcr}
                -1 & \textup{ si } & x\leq 1 \\
                1 & \textup{ si } & x>1 \\
            \end{array}
        \right.
    \end{equation*}
    Defina $g=-f$. \textbf{Pruebe} que $f$ y $g$ son discontinuas en $1$. \textbf{Escriba} explícitamente las funciones $\abs{f}$, $f^2$, $f\cdot g$ y $g^2$. \textbf{Pruebe} que todas estas funciones son continuas en $1$.

    \begin{sol}
        Para probar que $f$ y $g$ son discontinuas en $1$ basta con ver que una de ellas dos es discontinua en $1$ y, de forma automática se tendrá que la otra también lo es (ya que si $f$ es continua en un punto, $-f$ también es continua en el mismo punto). Veamos que $f$ es discontinua en $1$.
        
        En efecto, tomemos $\varepsilon=1>0$, para todo $\delta>0$ existe $x_\delta=1+\frac{\delta}{2}$ tal que
        \begin{equation*}
            \begin{split}
                0<\abs{1-x_\delta}=\abs{1-1-\frac{\delta}{2}}=\frac{\delta}{2}<\delta
            \end{split}
        \end{equation*}
        tal que
        \begin{equation*}
            \begin{split}
                \abs{f(1)-f(x_\delta)}&=\abs{-1-f\left(1+\frac{\delta}{2}\right)}\\
                &=\abs{-1-1}\\
                &=2\\
                &\geq1\\
                &=\varepsilon\\
            \end{split}
        \end{equation*}
        por tanto, $f$ no es continua en $1$, luego $g$ tampoco puede serlo.

        Ahora, se tiene que
        \begin{equation*}
            \abs{f}(x)=f(x)^2=g(x)^2=1\quad\forall x\in\mathbb{R}
        \end{equation*}
        y, $f\cdot g(x)=-1$ para todo $x\in\mathbb{R}$, es decir que todas las funciones anteriores son constantes, luego son continuas en $\mathbb{R}$, en particular lo son en $1$.
    \end{sol}

    \item Considere dos funciones $\cf{g}{[a,b]}{\mathbb{R}}$ y $\cf{h}{[b,c]}{\mathbb{R}}$. Defina $\cf{f}{[a,c]}{\mathbb{R}}$ como
    \begin{equation*}
        f(x)=\left\{
            \begin{array}{lcr}
                g(x) & \textup{ si } & x\in[a,b] \\
                h(x) & \textup{ si } & x\in]b,c] \\
            \end{array}
        \right.
    \end{equation*}
    si $g$ es continua en $[a,b]$, $h$ es continua en $[b,c]$ y $g(b)=h(b)$, \textbf{demuestre} que $f$ es continua en $[a,c]$.

    \begin{proof}
        Es claro que $f$ está bien definida para todo $x\in[a,c]$. Sea $x_0\in[a,c]$, probaremos que $f$ es continua en $x_0$. Tomemos $\varepsilon>0$, se tienen 3 casos:
        \begin{enumerate}
            \item $x_0\in[a,b[$. Como $g$ es continua y $x_0\in[a,b]$, entonces existe $\delta_1>0$ tal que
            \begin{equation*}
                \forall x\in[a,b]\textup{ tal que }0<\abs{x_0-x}<\delta_1\Rightarrow \abs{g(x_0)-g(x)}<\varepsilon
            \end{equation*}
            tomemos $\delta=\min\left\{\delta_1,b-x_0 \right\}>0$. Si $x\in[a,c]$ es tal que $0<\abs{x_0-x}<\delta$, en particular se tiene que
            \begin{equation*}
                0<\abs{x_0-x}<\delta_1\textup{ y }x-x_0<b-x_0
            \end{equation*}
            por tanto, $x<b$, luego $f(x)=g(x)$. De esta forma:
            \begin{equation*}
                \abs{f(x_0)-f(x)}=\abs{g(x_0)-g(x)}<\varepsilon
            \end{equation*}
            por la continuidad de $g$ en $x_0$ y dado que $x_0<b$.
            
            \item $x_0\in]b,c]$. Como $h$ es continua y $x_0\in[b,c]$, entonces existe $\delta_2>0$ tal que
            \begin{equation*}
                \forall x\in[b,c]\textup{ tal que }0<\abs{x_0-x}<\delta_2\Rightarrow \abs{h(x_0)-h(x)}<\varepsilon
            \end{equation*}
            tomemos $\delta=\min\left\{\delta_1,x_0-b \right\}>0$. Si $x\in[a,c]$ es tal que $0<\abs{x_0-x}<\delta$, en particular se tiene que
            \begin{equation*}
                0<\abs{x_0-x}<\delta_2\textup{ y }x_0-x<x_0-b
            \end{equation*}
            por tanto, $b<x$, luego $f(x)=h(x)$. De esta forma:
            \begin{equation*}
                \abs{f(x_0)-f(x)}=\abs{h(x_0)-h(x)}<\varepsilon
            \end{equation*}
            por la continuidad de $g$ en $x_0$ y dado que $b<x_0$.

            \item $x_0=b$. Como $g$ y $h$ son continuas en $x_0$, existen $\delta_1,\delta_2>0$ tales que
            \begin{equation*}
                \left\{
                    \begin{split}
                        \forall x\in[a,b]\textup{ tal que }0<\abs{x_0-x}<\delta_1&\Rightarrow \abs{g(x_0)-g(x)}<\varepsilon\\
                        \forall x\in[b,c]\textup{ tal que }0<\abs{x_0-x}<\delta_2&\Rightarrow \abs{h(x_0)-h(x)}<\varepsilon\\
                    \end{split}
                \right.
            \end{equation*}
            pero, dado que $x_0=b$, lo anterior se reduce únicamente a pedir que
            \begin{equation*}
                \left\{
                    \begin{split}
                        \forall x\in[a,b]\textup{ tal que }0<x_0-x<\delta_1&\Rightarrow \abs{g(x_0)-g(x)}<\varepsilon\\
                        \forall x\in[b,c]\textup{ tal que }0<x-x_0<\delta_2&\Rightarrow \abs{h(x_0)-h(x)}<\varepsilon\\
                    \end{split}
                \right.
            \end{equation*}
            tomemos $\delta=\min\left(\delta_1,\delta_2\right)>0$, si $x\in[a,c]$ es tal que $0<\abs{x_0-x}<\delta$, se tienen dos casos:
            \begin{itemize}
                \item $x<x_0$, en tal caso se sigue que $0<x_0-x<\delta\leq\delta_1$, luego por la primera ecuación se tiene que
                \begin{equation*}
                    \abs{f(x_0)-f(x)}=\abs{g(x_0)-g(x)}<\varepsilon
                \end{equation*}
                pues, $f(x_0)=g(x_0)$ y $x<b$.
                \item $x_0<x$, en tal caso se sigue que $0<x-x_0<\delta\leq\delta_2$, luego por la primera ecuación se tiene que
                \begin{equation*}
                    \abs{f(x_0)-f(x)}=\abs{h(x_0)-h(x)}<\varepsilon
                \end{equation*}
                pues, $f(x_0)=h(x_0)$ y $b<x_0$.
            \end{itemize}
            en cualquier caso, se sigue que
            \begin{equation*}
                \abs{f(x_0)-f(x)}<\varepsilon
            \end{equation*}
        \end{enumerate}
        de los tres incisos anteriores, se sigue que $f$ es continua en $x_0$. Como el $x_0\in[a,c]$ fue arbitrario obtenemos entonces que $f$ es continua en $[a,c]$.
    \end{proof}

    \item \begin{enumerate}
        \item Si $f(x)=x^3$ y $g(x)=e^x$, \textbf{calcule} $f\circ g(x)$ y $g\circ f(x)$.
        \item Si $f(x)=\frac{1}{1-x}$, \textbf{calcule} $(f\circ f\circ f)(x)$.
        \item Si $f(x)=x^3$ y $g$ es la función del problema 3.16, \textbf{calcule} $g\circ f$.
        ¿En qué puntos son discontinuas las funciones anteriores? \textbf{Justifique}.
    \end{enumerate}

    \begin{sol}
        De (i): Se tiene que
        \begin{equation*}
            \begin{split}
                f\circ g(x)&=f(g(x))\\
                &=f(e^x)\\
                &=(e^x)^3\\
                &=e^{ 3x}\\
            \end{split}
        \end{equation*}
        y
        \begin{equation*}
            \begin{split}
                g\circ f(x)&=g(f(x))\\
                &=g(x^3)\\
                &=e^{ x^3}\\
            \end{split}
        \end{equation*}
        para todo $x\in\mathbb{R}$.

        De (ii): Veamos que
        \begin{equation*}
            \begin{split}
                f\circ f\circ f(x)&=f(f(f(x)))\\
                &=f\left(f\left( \frac{1}{1-x}\right)\right)\\
                &=f\left(\frac{1}{1-\frac{1}{1-x}}\right)\\
                &=f\left(\frac{1}{\frac{1-x-1}{1-x}}\right)\\
                &=f\left(\frac{x-1}{x}\right)\\
                &=\frac{1}{1-\frac{x-1}{x}}\\
                &=\frac{1}{\frac{x-x+1}{x}}\\
                &=x\\
            \end{split}
        \end{equation*}
        siempre que la composición esté bien definida.

        De (iii): Recordemos que
        \begin{equation*}
            g(x)=\left\{
                \begin{array}{lcr}
                    -x & \textup{ si } & -3\leq x\leq-1\\
                    x^3 & \textup{ si } & -1< x<1\\
                    \sqrt{x} & \textup{ si } & 1<x\leq3 \\
                    \frac{1}{3-x}+\sqrt{3} & \textup{ si } & 3<x\\ 
                \end{array}
            \right.
        \end{equation*}
        %TODO
    \end{sol}

    \item \textbf{Calcule} $g\circ f$ y \textbf{determine} el dominio natural de $g\circ f$ en los siguientes casos. ¿Es $g\circ f$ continua en su domino natural? \textbf{Justifique}.
    \begin{enumerate}
        \item $f(x)=\sqrt{x-2}$ y $g(x)=\frac{1}{x}$.
        \item $f(x)=\sqrt{x}$ y $g(x)=x^2+1$.
    \end{enumerate}

    \begin{sol}
        De (i): Veamos que
        \begin{equation*}
            \begin{split}
                g\circ f(x)&=g(f(x))\\
                &=g\left(\sqrt{x-2}\right)\\
                &=\frac{1}{\sqrt{x-2}}\\
            \end{split}
        \end{equation*}
        notemos que esta función está bien definida siempre que $x-2>0$ es decir que $x>2$. Por tanto, el dominio natural de $f$ es $D_{g\circ f}=]2,\infty[$.

        De (ii): %TODO
    \end{sol}

    \item \textbf{Represente} las siguientes funciones usando operaciones algebraicas y la composición de funciones. \textbf{Determine} el conjuto de puntos de continuidad en cada caso. \textbf{Justifique}.
    \begin{enumerate}
        \item $f(x)=\frac{3\sen^2(x+\pi)-2}{\sen(x+\pi)+1}$.
        \item $f(x)=\left\{
            \begin{array}{lcr}
                \sen^2\left(\frac{1}{x-1} \right) & \textup{ si } & x\neq 1,\\
                0 & \textup{ si } & x = 1.\\
            \end{array}
        \right.$.
        \item $f(x-1)=x^2$. ¿Quiénes son $f(x)$ y $f(x+1)$?
    \end{enumerate}
    
    \begin{sol}
        De (i): %TODO
    \end{sol}

    \item  \textbf{Determine (justificando)} el dominio natural de las siguientes funciones e indique su conjunto de puntos de continuidad.
    \begin{enumerate}
        \item $f(x)=\sin\left(\frac{1}{x-1}\right)$.
        \item $f(x)=\cos^{1/2}x$.
        \item $f(x)=\frac{1}{x}-\tan^2x$.
    \end{enumerate}

    \begin{sol}
        De (i): La función $x\mapsto\sin\left(\frac{1}{x-1} \right)$ está bien definida siempre que $\frac{1}{x-1}\in\mathbb{R}$, es decir que $x-1\neq0$. Por tanto, el dominio natural de $f$ es
        \begin{equation*}
            D_f=\mathbb{R}\backslash\left\{1 \right\}
        \end{equation*}
        Sean $\cf{f_1}{\mathbb{R}}{\mathbb{R}}$, $\cf{f_2}{\mathbb{R}\backslash\left\{0\right\}}{\mathbb{R}}$ y $\cf{f_3}{\mathbb{R}\backslash\left\{1\right\}}{\mathbb{R}\backslash\left\{0\right\}}$ tales que $f_1(x)=\sin x$, $f_2(x)=\frac{1}{x}$ y $f_3(x)=x-1$, se tiene que
        \begin{equation*}
            f=f_1\circ f_2\circ f_3
        \end{equation*}
        siendo las tres funciones $f_1,f_2$ y $f_3$ son continuas en sus respectivos dominios, se sigue entonces que $f$ es continua en $\mathbb{R}\backslash\left\{ 1\right\}$.

        De (ii): Para que $f$ esté bien definida debe suceder que $\cos x\geq 0$, esto es
        \begin{equation*}
            \cos x\geq 0\iff x\in\bigcup_{ k\in\mathbb{Z}}\left[-\frac{\pi}{2}+2k\pi,\frac{\pi}{2}2k\pi\right]
        \end{equation*}
        por tanto,
        \begin{equation*}
            D_f=\bigcup_{ k\in\mathbb{Z}}\left[-\frac{\pi}{2}+2k\pi,\frac{\pi}{2}2k\pi\right]
        \end{equation*}
        Sean ahora $\cf{f_1}{D_f}{\mathbb{R}^+\cup\left\{0\right\}}$ y $\cf{f_2}{\mathbb{R}^+\cup\left\{0\right\}}{\mathbb{R}}$ las funciones tales que $f_1(x)=\cos x$ y $f_2(x)=\sqrt{x}$. Como
        \begin{equation*}
            f=f_2\circ f_1
        \end{equation*}
        entonces al tenerse que $f_1$ y $f_2$ son continuas en sus respectivos dominios ($f_1$ es continua pues es la reestricción de la función coseno a $D_f$), se sigue que $f$ es continua en $D_f$.

        De (iii): Para que $f$ esté bien definida debe suceder que $\frac{1}{x}\in\mathbb{R}$ y $\tan x\in\mathbb{R}$, esto es que
        $x\neq 0$ y $x\neq\frac{2k+1}{2}\pi$ con $k\in\mathbb{Z}$. Por tanto,
        \begin{equation*}
            D_f=\mathbb{R}\backslash\left(\left\{0\right\}\cup\left\{\frac{2k+1}{2}\pi\Big|k\in\mathbb{Z}\right\}\right)
        \end{equation*}
        Y, como $x\mapsto\frac{1}{x}$ es continua en $\mathbb{R}\backslash\left\{0 \right\}$ y $x\mapsto\tan x$ lo es siempre que $x\neq\frac{2k+1}{2}\pi$ con $k\in\mathbb{Z}$, al reestringir estas funciones a $D_f$ se sigue que también son continuas, luego $f$ también es continua.
    \end{sol}

    \item Si $\cf{f}{\mathbb{R}}{\mathbb{R}}$ es una función continua que satisface la condición $f(x+y)=f(x)+f(y)$, $\forall x,y\in\mathbb{R}$, \textbf{pruebe} que existe un número $a\in\mathbb{R}$ tal que $f(x)=ax$ para todo $x\in\mathbb{R}$.

    \textit{Sugerencia.} Primero pruebe que $f(0)=0$. Observe que si $f(x)$ va a ser igual a $ax$, para todo $x\in\mathbb{R}$, entonces $f(1)=a$. A continuación, demuestre que la fórmula $f(x)=ax$, para todo $x\in\mathbb{N}$, luego para todo $\mathbb{Q}^+$ y, usando la continuidad de $f$ pruebe el resultado para toda $x\in\mathbb{I}^+$. Concluya que se cumple para $\mathbb{R}$.

    \begin{proof}
        Notemos que
        \begin{equation*}
            \begin{split}
                f(0+0)&=f(0)+f(0)\\
                \Rightarrow f(0)&=2f(0)\\
                \Rightarrow f(0)&=0\\
            \end{split}
        \end{equation*}
        ahora, sea $a=f(1)$. Se tiene para todo $x\in\mathbb{R}$:
        \begin{equation*}
            f(2x)=f(x+x)=2f(x)
        \end{equation*}
        Suponga que $f(kx)=kx$ para algún $k\in\mathbb{N}$, se sigue que
        \begin{equation*}
            f([k+1]x)=f(kx)+f(x)=kf(x)\cdot f(x)=(k+1)f(x)
        \end{equation*}
        aplicando inducción obtenemos que $f(nx)=nf(x)$ para todo $x\in\mathbb{N}$. Más aún, como $f(0)=0$ entonces
        \begin{equation*}
            f(nx)=nf(x),\quad\forall n\in\mathbb{N}\cup\left\{0 \right\}\textup{ y }x\in\mathbb{R}
        \end{equation*} 
        Ahora, si $m\in\mathbb{N}$ tomando $x=\frac{1}{m}$ se obtiene de lo anterior que
        \begin{equation*}
            \begin{split}
                f(1)&=mf\left(\frac{1}{m}\right)\\
                \Rightarrow f\left(\frac{1}{m} \right)&=\frac{1}{m}f(1)
            \end{split}
        \end{equation*}
        Al ser el $m\in\mathbb{N}$ arbitrario, se sigue que
        \begin{equation*}
            f\left(\frac{1}{m}\right)=\frac{1}{m}f(1),\quad\forall m\in\mathbb{N}
        \end{equation*}
        De esta forma, si $\frac{p}{q}\in\mathbb{Q}$ es tal que $p,q\geq 0$ se tiene
        \begin{equation*}
            \begin{split}
                f\left(\frac{p}{q}\right)&=f\left(p\cdot\frac{1}{q}\right)\\
                &=pf\left(\frac{1}{q}\right)\\
                &=\frac{p}{q}f(1)\\
                \Rightarrow f\left(\frac{p}{q}\right)&=a\frac{p}{q}
            \end{split}
        \end{equation*}
        (donde recordemos que $a=f(1)$). Se sigue entonces que $f(a)$. Por tanto, $f(x)=ax$ para todo $x\in\mathbb{Q}^+$. Si ahora $x\in\mathbb{R}$, se tiene que
        \begin{equation*}
            0=f(0)=f(x-x)=f(x)+f(-x)\Rightarrow -f(x)=f(-x)
        \end{equation*}
        por tanto, si $r\in\mathbb{Q}^-$,
        \begin{equation*}
            f(r)=-f(-r)=-a\cdot(-r)=ar
        \end{equation*}
        (pues $-r\in\mathbb{Q}^+$) luego,
        \begin{equation*}
            f(x)=ax,\quad\forall x\in\mathbb{Q}
        \end{equation*}
        
        Sea ahora $x_0\in\mathbb{I}$. Se sabe que existe $\left\{x_n \right\}_{ n=1}^\infty$ una sucesión en $\mathbb{Q}$ que converge a $x_0$. Como $f$ es continua en $\mathbb{R}$, en particular lo es en $x_0$, luego
        \begin{equation*}
            f(x_0)=\lim_{n\rightarrow\infty}f(x_n)=\lim_{n\rightarrow\infty}ax_n=ax_0
        \end{equation*}
        Por tanto, $f(x)=ax$ para todo $x\in\mathbb{I}$. Se concluye finalmente que
        \begin{equation*}
            f(x)=ax,\quad\forall x\in\mathbb{R}
        \end{equation*}
        donde $a=f(1)$.
    \end{proof}

    \item Considere la función $\cf{f}{]0,1[}{\mathbb{R}}$ definida como
    \begin{equation*}
        f(x)=\left\{\begin{array}{lcl}
            0 & \textup{ si } & x\in]0,1[\backslash\mathbb{Q}\\
            \frac{1}{q} & \textup{ si } & x=\frac{p}{q}, \textup{ donde }p,q\in\mathbb{N}\textup{ son primos relativos.}
        \end{array} \right.
    \end{equation*}
    \textbf{Pruebe} que $f$ es continua en todo punto de $]0,1[\backslash\mathbb{Q}$ y discontinua en todo punto de $]0,1[\cap\mathbb{Q}$.

    \begin{proof}
        Se tienen que hacer dos cosas:
        \begin{enumerate}
            \item $f$ es continua en $]0,1[\backslash\mathbb{Q}$. En efecto, sean $x_0\in]0,1[\backslash\mathbb{Q}$ y $\varepsilon>0$. Considere el conjunto
            \begin{equation*}
                A=\left\{x=\frac{p}{q}\in]0,1[\cap\mathbb{Q}\Big|p,q\in\mathbb{N}\textup{ son primos relativos y }\varepsilon\leq\frac{1}{q} \right\}
            \end{equation*}
            Se tiene que $A$ es un conjunto finito, pues por la propiedad arquimediana existe $n_0\in\mathbb{N}$ con $n_0\geq 2$ tal que
            \begin{equation*}
                \frac{1}{n_0}<\min\left\{\varepsilon,1\right\}\leq\frac{1}{n_0-1}
            \end{equation*}
            De esta forma, $A$ puede tener a lo sumo una cantidad finita de elementos. Sea
            \begin{equation*}
                \delta=\min_{ y\in A}\abs{x_0-y}>0
            \end{equation*}
            pues, $x_0\notin A$. Si $x\in]0,1[$ es tal que $0<\abs{x_0-x}<\delta$, se tienen dos casos:
            \begin{enumerate}
                \item $x\in]0,1[\backslash\mathbb{Q}$, se sigue que
                \begin{equation*}
                    \abs{f(x_0)-f(x)}=\abs{0-0}=0<\varepsilon
                \end{equation*}
                \item $x=\frac{p}{q}\in]0,1[\cap\mathbb{Q}$ siendo $p,q\in\mathbb{N}$ primos relativos, se sigue por la elección de $\delta>0$ que $\frac{1}{q}<\varepsilon$, pues $x\notin A$. Luego,
                \begin{equation*}
                    \abs{f(x_0)-f(x)}=\abs{f(x)}=\frac{1}{q}<\varepsilon
                \end{equation*}
            \end{enumerate}
            Por ambos incisos se sigue que
            \begin{equation*}
                0<\abs{x_0-x}<\delta\Rightarrow\abs{f(x_0)-f(x)}<\varepsilon
            \end{equation*}
            por tanto, $f$ es continua en $x_0$.

            \item $f$ no es continua en $]0,1[\cap\mathbb{Q}$. Sea $x_0\in]0,1[\cap\mathbb{Q}$, entonces existen $p,q\in\mathbb{N}$ primos relativos tales que $x_0=\frac{p}{q}$. Tomemos $\varepsilon=\frac{1}{q}>0$.
            
            Por la densidad de los racionales, para todo $\delta>0$ existe $x_\delta\in]0,1[\backslash\mathbb{Q}$ tal que
            \begin{equation*}
                \abs{x_0-x_\delta}<\delta
            \end{equation*}
            que cumple
            \begin{equation*}
                \abs{f(x_0)-f(x)}=\abs{\frac{1}{q}-0}=\frac{1}{q}\geq\varepsilon
            \end{equation*}
            Por tanto, $f$ no es continua en $x_0$.
        \end{enumerate}
        Por los dos incisos se sigue el resultado.
    \end{proof}

    \item \textbf{Calcule (justificando)} los siguientes límites:
    \begin{enumerate}
        \item $\lim_{x\rightarrow0}\frac{\sin^2x}{x}$.
        \item $\lim_{x\rightarrow0}\frac{1-\cos x}{x^2}$.
        \item $\lim_{x\rightarrow0}\frac{\tan^2x+2x}{x+x^2}$.
        \item $\lim_{x\rightarrow1}(x^2-1)\sin\frac{1}{x-1}$.
        \item $\lim_{x\rightarrow\infty}x\sin^2x$.
        \item $\lim_{x\rightarrow1}(x^2-1)\sin\frac{1}{(x-1)^2}$.
    \end{enumerate}

    \begin{sol}
        De (i): Veamos que
        \begin{equation*}
            \begin{split}
                \lim_{x\rightarrow 0}\frac{\sin^2 x}{x}&=\lim_{x\rightarrow 0}\left(\frac{\sin x}{x}\cdot\sin x\right)\\
            \end{split}
        \end{equation*}
        como $\lim_{x\rightarrow 0}\frac{\sin x}{x}=1$ y $\lim_{x\rightarrow 0}\sin x=0$, se puede separar el límite como el producto de límite, así
        \begin{equation*}
            \begin{split}
                \lim_{x\rightarrow 0}\frac{\sin^2 x}{x}&=\lim_{x\rightarrow 0}\left(\frac{\sin x}{x}\cdot\sin x\right)\\
                &=\lim_{x\rightarrow 0}\frac{\sin x}{x}\cdot\lim_{x\rightarrow 0}\sin x\\
                &=1\cdot 0\\
                &=0\\
            \end{split}
        \end{equation*}

        De (ii): Veamos que para $x\neq k\pi+\frac{\pi}{2}$ con $k\in\mathbb{Z}$ se tiene que
        \begin{equation*}
            \begin{split}
                \lim_{x\rightarrow 0}\frac{1-\cos x}{x^2}&=\lim_{x\rightarrow 0}\frac{1-\cos x}{x^2}\cdot\frac{1+\cos x}{1+\cos x}\\
                &=\lim_{x\rightarrow 0}\frac{1-\cos x}{x^2}\cdot\frac{1+\cos x}{1+\cos x}\\
                &=\lim_{x\rightarrow 0}\frac{1-\cos^2 x}{x^2(1+\cos x)}\\
                &=\lim_{x\rightarrow 0}\frac{\sin^2 x}{x^2(1+\cos x)}\\
                &=\lim_{x\rightarrow 0}\frac{\sin^2 x}{x^2}\cdot\frac{1}{1+\cos x} \\
            \end{split}
        \end{equation*}
        definiendo $x\mapsto \frac{\sin^ 2 x}{x^2}\cdot\frac{1}{1+\cos x}$ en $]-\frac{\pi}{2},\frac{\pi}{2}[$, el límite siguiente se reduce por álgebra de límites al producto de los mismos, debido a que
        \begin{equation*}
            \lim_{x\rightarrow 0}\frac{\sin^2 x}{x^2}=\lim_{x\rightarrow 0}\frac{\sin x}{x}\cdot\lim_{x\rightarrow 0}\frac{\sin^2 x}{x^2}=1\cdot 1 =1
        \end{equation*}
        y
        \begin{equation*}
            \lim_{x\rightarrow 0}\frac{1}{1+\cos x}=\frac{1}{\lim_{x\rightarrow 0}1+\cos x }=\frac{1}{1+\cos0 }=\frac{1}{2}
        \end{equation*}
        por tanto,
        \begin{equation*}
            \lim_{x\rightarrow 0}\frac{1-\cos x}{x^2}=\lim_{x\rightarrow 0}\frac{\sin^2 x}{x^2}\cdot\lim_{ x\rightarrow 0}\frac{1}{1+\cos x}=1\cdot\frac{1}{2}=\frac{1}{2}
        \end{equation*}

        De (iii): Veamos que
        \begin{equation*}
            \begin{split}
                \lim_{ x\rightarrow0}\frac{\tan^2x+2x}{x+x^2}&=\lim_{ x\rightarrow0}\frac{\tan^2x+2x}{x(x+1)}\\
                &=\lim_{ x\rightarrow0}\frac{\sin^2x+2x\cos^2 x}{x(x+1)\cos^2 x}\\
                &=\lim_{ x\rightarrow0}\left[\frac{\sin^2x}{x(x+1)\cos^2 x}+\frac{2x\cos^2 x}{x(x+1)\cos^2 x}\right] \\
                &=\lim_{ x\rightarrow0}\left[\frac{\sin x}{(x+1)\cos^2x}\cdot\frac{\sin x}{x}+\frac{2}{x+1}\right]\\
            \end{split}
        \end{equation*}
        por álgebra de límites, se sigue que
        \begin{equation*}
            \begin{split}
                \lim_{ x\rightarrow0}\frac{\tan^2x+2x}{x+x^2}&=\lim_{ x\rightarrow0}\left[\frac{\sin x}{(x+1)\cos^2x}\cdot\frac{\sin x}{x}\right]+\lim_{ x\rightarrow0}\left[\frac{2}{x+1}\right]\\
                &=\lim_{ x\rightarrow0}\left[\frac{\sin x}{(x+1)\cos^2x}\right]\cdot\lim_{ x\rightarrow0}\left[\frac{\sin x}{x}\right]+2\\
                &=0\cdot1+2\\
                &=2\\
            \end{split}
        \end{equation*}

        De (iv): Como $\lim_{ x\rightarrow 1}x^2-1=0$ y la función $x\mapsto\sin\frac{1}{x-1}$ es acotada, por un ejercicio anterior se sigue que
        \begin{equation*}
            \lim_{ x\rightarrow 1}(x^2-1)\sin\frac{1}{x-1}=0
        \end{equation*}

        De (v): Considere las sucesiones $\left\{n\pi \right\}_{ n=1}^\infty$ y $\left\{2n\pi+\frac{\pi}{2} \right\}_{ n=1}^\infty$. Se tiene que ambas sucesiones divergen a infinito. Además,
        \begin{equation*}
            \lim_{n\rightarrow\infty}(n\pi)^2\sin^2(n\pi)=\lim_{n\rightarrow\infty}(n\pi)^2\cdot 0=0
        \end{equation*}
        y,
        \begin{equation*}
            \lim_{n\rightarrow\infty}\left(2n\pi+\frac{\pi}{2}\right)^2\sin^2(2n\pi+\frac{\pi}{2})=\lim_{n\rightarrow\infty}\left(2n\pi+\frac{\pi}{2}\right)^2\cdot\sin^2(\frac{\pi}{2})=\lim_{n\rightarrow\infty}\left(2n\pi+\frac{\pi}{2}\right)^2\cdot1=\infty
        \end{equation*}
        por tanto, debido a la unicidad del límite, no puede existir $\lim_{ x\rightarrow\infty}x^2\sin x$.

        De (vi): Análogamente a (iv), dado que $\lim_{ x\rightarrow 1}x^2-1=0$ y $x\mapsto\frac{1}{(x-1)^2}$ es una función acotada, se sigue que
        \begin{equation*}
            \lim_{ x\rightarrow 1}(x^2-1)\sin\frac{1}{(x-1)^2}=0
        \end{equation*}
    \end{sol}

    \item \textbf{Determine (justificando)} el conjunto de puntos de continuidad de las siguientes funciones:
    \begin{enumerate}
        \item $f(x)=x\sin x$.
        \item $f(x)=\frac{x}{\tan x}$.
        \item $f(x)=\sin x\cos\theta+\sin\theta\cos x$ (donde $\theta\in\mathbb{R}$ es una constante).
        \item $f(x)=\sqrt{\frac{1+\cos x}{2}}$.
    \end{enumerate}

    \begin{sol}
        De (i): Afirmmamos que $f$ es continua en $\mathbb{R}$. En efecto, como las funciones $x\mapsto x$ y $x\mapsto \sin x$ son continuas en $\mathbb{R}$, entonces su producto también lo es en $\mathbb{R}$, luego $f$ es continua en $\mathbb{R}$.

        De (ii): Recordemos que
        \begin{equation*}
            \tan k\pi=0,\quad\forall k\in\mathbb{Z}
        \end{equation*}
        Por tanto, la función $f$ debe tener como dominio a $\mathbb{R}\backslash K$, donde
        \begin{equation*}
            K=\left\{x\in\mathbb{R}\Big|\textup{ existe }k\in\mathbb{Z}\textup{ tal que }x=k\pi \right\}
        \end{equation*}
        Afirmamos que $f$ es continua en $\mathbb{R}\backslash K$. En efecto, si $x_0\in\mathbb{R}$, como $x\mapsto x$ es continua y $x\mapsto \tan x$ son funciones continuas en $\mathbb{R}$, se sigue que
        \begin{equation*}
            \begin{split}
                \lim_{ x\rightarrow x_0}\frac{x}{\tan x}&=\frac{\lim_{ x\rightarrow x_0}x}{\lim_{ x\rightarrow x_0}\tan x}\\
                &=\frac{x_0}{\tan x_0}\\
            \end{split}
        \end{equation*}
        pues, $\tan x_0\neq 0$. Luego, $f$ es continua en $x_0$. Se sigue entonces que $f$ es continua en $\mathbb{R}\backslash K$.
        
        De (iii): Sea $\theta\in\mathbb{R}$. Afirmamos que $f$ es continua en $\mathbb{R}$. En efecto, como $x\mapsto \sin x$ y $x\mapsto \cos x$ son contiuas en $\mathbb{R}$ se sigue que $x\mapsto \sin x\cos\theta$ y $x\mapsto \sin\theta\cos x$ también lo son en $\mathbb{R}$. Luego, su suma
        \begin{equation*}
            x\mapsto f(x)=\sin x\cos\theta+\sin\theta\cos x
        \end{equation*}
        es continua en $\mathbb{R}$.

        De (iv): Sea $x\in\mathbb{R}$. Como $g(x)=\sqrt{x}$ para todo $x\in[0,\infty[$ es continua en $[0,\infty[$ y la función $h(x)=\frac{1+\cos x}{2}$ para todo $x\in\mathbb{R}$ es continua en $\mathbb{R}$ (por ser producto y suma de funciones continuas en $\mathbb{R}$), entonces su composición
        \begin{equation*}
            g\circ h(x)=\sqrt{\frac{1+\cos x}{2}}=f(x)
        \end{equation*}
        lo es también, ya que
        \begin{equation*}
            \begin{split}
                -1\leq \cos x&\Rightarrow0\leq 1+\cos x\\
                &\Rightarrow0\leq \frac{1+\cos x}{2} \\
            \end{split}
        \end{equation*}
        por tanto, $h(x)\geq 0$ para todo $x\in\mathbb{R}$. Así, $g\circ h$ está bien definida para todo $x\in\mathbb{R}$. Se sigue entonces que $f$ es continua en $\mathbb{R}$.
    \end{sol}

\end{enumerate}
\end{document}